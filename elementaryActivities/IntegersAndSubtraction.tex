\documentclass{ximera}

\usepackage{gensymb}
\usepackage{tabularx}
\usepackage{mdframed}
\usepackage{pdfpages}
%\usepackage{chngcntr}

\let\problem\relax
\let\endproblem\relax

\newcommand{\property}[2]{#1#2}




\newtheoremstyle{SlantTheorem}{\topsep}{\fill}%%% space between body and thm
 {\slshape}                      %%% Thm body font
 {}                              %%% Indent amount (empty = no indent)
 {\bfseries\sffamily}            %%% Thm head font
 {}                              %%% Punctuation after thm head
 {3ex}                           %%% Space after thm head
 {\thmname{#1}\thmnumber{ #2}\thmnote{ \bfseries(#3)}} %%% Thm head spec
\theoremstyle{SlantTheorem}
\newtheorem{problem}{Problem}[]

%\counterwithin*{problem}{section}



%%%%%%%%%%%%%%%%%%%%%%%%%%%%Jenny's code%%%%%%%%%%%%%%%%%%%%

%%% Solution environment
%\newenvironment{solution}{
%\ifhandout\setbox0\vbox\bgroup\else
%\begin{trivlist}\item[\hskip \labelsep\small\itshape\bfseries Solution\hspace{2ex}]
%\par\noindent\upshape\small
%\fi}
%{\ifhandout\egroup\else
%\end{trivlist}
%\fi}
%
%
%%% instructorIntro environment
%\ifhandout
%\newenvironment{instructorIntro}[1][false]%
%{%
%\def\givenatend{\boolean{#1}}\ifthenelse{\boolean{#1}}{\begin{trivlist}\item}{\setbox0\vbox\bgroup}{}
%}
%{%
%\ifthenelse{\givenatend}{\end{trivlist}}{\egroup}{}
%}
%\else
%\newenvironment{instructorIntro}[1][false]%
%{%
%  \ifthenelse{\boolean{#1}}{\begin{trivlist}\item[\hskip \labelsep\bfseries Instructor Notes:\hspace{2ex}]}
%{\begin{trivlist}\item[\hskip \labelsep\bfseries Instructor Notes:\hspace{2ex}]}
%{}
%}
%% %% line at the bottom} 
%{\end{trivlist}\par\addvspace{.5ex}\nobreak\noindent\hung} 
%\fi
%
%


\let\instructorNotes\relax
\let\endinstructorNotes\relax
%%% instructorNotes environment
\ifhandout
\newenvironment{instructorNotes}[1][false]%
{%
\def\givenatend{\boolean{#1}}\ifthenelse{\boolean{#1}}{\begin{trivlist}\item}{\setbox0\vbox\bgroup}{}
}
{%
\ifthenelse{\givenatend}{\end{trivlist}}{\egroup}{}
}
\else
\newenvironment{instructorNotes}[1][false]%
{%
  \ifthenelse{\boolean{#1}}{\begin{trivlist}\item[\hskip \labelsep\bfseries {\Large Instructor Notes: \\} \hspace{\textwidth} ]}
{\begin{trivlist}\item[\hskip \labelsep\bfseries {\Large Instructor Notes: \\} \hspace{\textwidth} ]}
{}
}
{\end{trivlist}}
\fi


%% Suggested Timing
\newcommand{\timing}[1]{{\bf Suggested Timing: \hspace{2ex}} #1}




\hypersetup{
    colorlinks=true,       % false: boxed links; true: colored links
    linkcolor=blue,          % color of internal links (change box color with linkbordercolor)
    citecolor=green,        % color of links to bibliography
    filecolor=magenta,      % color of file links
    urlcolor=cyan           % color of external links
}

\title{Integers and Subtraction}
\author{Vic Ferdinand, Betsy McNeal, Jenny Sheldon}

\begin{document}
\begin{abstract}
    Let's try subtracting some integers!
\end{abstract}
\maketitle



Here are some subtraction problems we'll be working with throughout the activity.

\begin{enumerate}
    \item $5 - 2$
    \item $15 - (-3)$
    \item $(-4) - 9$
    \item $(-6) - (-12)$
\end{enumerate}

\begin{problem}
We had several different interpretations of subtraction.  Using our real-life integer contexts, use at least two different interpretations of subtraction to write story problems for $5 - 2$.

\end{problem}

\begin{problem}
Use our real-life integer contexts to write story problems for the remaining three problems.  You can choose the subtraction interpretation, but you might find some interpretations easier than others!  Draw pictures explaining what a child just learning this concept might do to solve the problem.


\end{problem}

\begin{problem}
For each of the subtraction problems above, how would you use a number line to model the solution?  Draw and explain clearly, especially when subtracting a negative number.  Which interpretation of subtraction were you using for these problems?
\end{problem}

\begin{problem}
Write a story problem for each of the subtraction problems above which uses our checks-and-bills scenario.  Make sure to read your story over several times.  Does the sensible answer to your story match the answer to the subtraction problem?
\end{problem}


\newpage
\begin{instructorNotes}
This is our second activity of three for working with operations with negative numbers.  The overall goal in this second activity should be to make significant progress in the understanding of the checks and bills model.

Again, there is a reading on Integers that helps clarify some of the structures involved here.

Students generally need quite a bit of practice with writing checks and bills stories that are both stories for the operation at hand as well as asking the proper question in their story (not giving away the sign of the answer in their question, or even asking a question in the first place).  This will be even more complicated with multiplication, so getting a good start here with subtraction (and addition if you didn't have much time for that on the previous day) is a good idea.

Another item to pay attention to with these problems is which model of subtraction students are using.  As they are working with their stories and number lines, you might continually ask, ``which kind of subtraction is this?'' to help students gain experience with identifying the structure.

The activity's questions are intentionally similar to the others about integers, so you can move quickly through interpretations other than the main three we are focusing on, or just jump to problems 3 and 4.

{\bf Suggested Timing:} This activity should take an entire class period.
\end{instructorNotes}







\end{document}