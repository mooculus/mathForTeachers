\documentclass[nooutcomes]{ximera}

\graphicspath{
  {./}
  {graphics/}
  {../graphics/}
}

\usepackage{chngcntr}

\let\question\relax
\let\endquestion\relax




\newtheoremstyle{SlantTheorem}{\topsep}{\fill}%%% space between body and thm
%\newtheoremstyle{SlantTheorem}{\topsep}{\topsep}%%% space between body and thm
 {\slshape}                      %%% Thm body font
 {}                              %%% Indent amount (empty = no indent)
 {\bfseries\sffamily}            %%% Thm head font
 {}                              %%% Punctuation after thm head
 {3ex}                           %%% Space after thm head
 {\thmname{#1}\thmnumber{ #2}\thmnote{ \bfseries(#3)}}%%% Thm head spec
\theoremstyle{SlantTheorem}
\newtheorem{question}{Question}
\counterwithin*{question}{section}



\let\instructorNotes\relax
\let\endinstructorNotes\relax
%%% instructorNotes environment
\ifhandout
\newenvironment{instructorNotes}[1][false]%
{%
\def\givenatend{\boolean{#1}}\ifthenelse{\boolean{#1}}{\begin{trivlist}\item}{\setbox0\vbox\bgroup}{}
}
{%
\ifthenelse{\givenatend}{\end{trivlist}}{\egroup}{}
}
\else
\newenvironment{instructorNotes}[1][false]%
{%
  \ifthenelse{\boolean{#1}}{\begin{trivlist}\item[\hskip \labelsep\bfseries {\Large Instructor Notes: \\} \hspace{\textwidth} ]}
{\begin{trivlist}\item[\hskip \labelsep\bfseries {\Large Instructor Notes: \\} \hspace{\textwidth} ]}
{}
}
{\end{trivlist}}
\fi


%% Suggested Timing
\newcommand{\timing}[1]{{\bf Suggested Timing: \hspace{2ex}} #1}

\title{The Mighty Triangle}
\author{Vic Ferdinand, Betsy McNeal, Jenny Sheldon}


\begin{document}
\begin{abstract}
\end{abstract}
\maketitle



\begin{problem}
Erika claims, ``I think the number of triangles in a polygon is always two less than the number of sides.  Like, if you have a six-sided figure, then it has four triangles."  The picture on the left shows Erika's work.  However, her classmate Sarah is confused because she has drawn a picture of an octagon with eight triangles in it (the picture on the right). Compare Sarah's idea and Erika's idea: can they both be correct? Why or why not?

		\begin{center}
		    \begin{tikzpicture}
		    \draw (0,0)--(3,0)--(4,1)--(3,3)--(1,3)--(-1,2)--(0,0);
		    \draw[green] (0,0)--(4,1);
		    \draw[green] (0,0)--(3,3);
		    \draw[green] (0,0)--(1,3);
		    \draw (6,0)--(9,0)--(10,1)--(10,2)--(8,3)--(6,3)--(5,2)--(5,1)--(6,0);
		    \draw[green] (7.5, 1.5)--(6,0);
		    \draw[green] (7.5, 1.5)--(9,0);
		    \draw[green] (7.5, 1.5)--(10,1);
		    \draw[green] (7.5, 1.5)--(10,2);
		    \draw[green] (7.5, 1.5)--(8,3);
		    \draw[green] (7.5, 1.5)--(6,3);
		    \draw[green] (7.5, 1.5)--(5,2);
		    \draw[green] (7.5, 1.5)--(5,1);
		    \end{tikzpicture}
		\end{center}
\end{problem}

\begin{problem}
Use what we've discussed about the interior angle sum of a triangle to find a formula for the interior angle sum of:
\begin{enumerate}
    \item a non-convex quadrilateral.
    \item a regular 12-gon.
    \item a regular $n$-gon. (What about a non-regular $n$-gon?)
\end{enumerate}


\end{problem}

\newpage
\begin{problem}
What should it mean for two shapes to be ``congruent''?  Be as specific as possible.
\end{problem}


\begin{problem}
Using your definition above and the properties we've discussed, explain how you know that the two triangles formed by this parallelogram and its diagonal are congruent.

\begin{center}
    \begin{tikzpicture}
        \draw (0,0)--(4,0)--(6,1)--(2,1)--(0,0);
        \draw (0,0)--(6,1);
    \end{tikzpicture}
\end{center}
\end{problem}



\begin{problem}
Assume that drawing the diagonal in this rhombus makes two congruent triangles.  Use this information to show that the rhombus must also be a parallelogram.
\begin{center}
    \begin{tikzpicture}[scale=0.7]
        \draw (0,0)--(3,-4)--(6,0)--(3,4)--(0,0);
    \end{tikzpicture}
\end{center}


\end{problem}

\begin{problem}
How many different triangles can you draw whose side lengths are $2$cm, $3cm$, and $4cm$?  You should use your compass to help you answer this question, and describe how you know you got them all.
\end{problem}

\begin{problem}
How many different triangles can you draw whose angle measures are $20\degree$, $40\degree$, and $120\degree$?  You should use your protractor to help you answer this question, and describe how you know you got them all.
\end{problem}

\newpage

\begin{instructorNotes}
This activity was invented to illustrate how the properties of triangles affect the properties of other shapes when we decompose other shapes into triangles.

There are generally three groupings of problems here.  We only use this activity as a bringing concepts together review if we don't need a ``catch-up day'' before our first midterm.  So, we generally use this activity by picking one grouping of problems depending on where we feel our students could use the most extra practice.  Also, the full activity assumes that students have already studied parallelism, properties of polygons (including interior angle sum but not including triangle congruence), and constructions with compass, protractor, and ruler.

The first two problems are the first grouping, and they talk about how we can use the interior angle sum of a triangle to get a formula for the interior angle sum of other polygons.  

The second grouping is the next three problems, which discuss congruence of triangles in a light fashion.  We have a project at the end of the semester detailing this content, so this would be the first time our students encounter this content.  For these reasons, the activity doesn't go very much in depth with this topic.  We are using these problems as further examples of some of the properties of parallelograms we introduced earlier in the course, as well as extra practice using our definitions for various quadrilaterals.  We expect students to be able to do these problems on two levels of sophistication: by reasoning using properties, and also by cutting, folding, or matching. 

The last two problems are the third grouping, and they focus on explorations using compass, protractor, and straightedge or ruler.  We intend for students to try to do these problems with their protractors and compasses and then try to generalize their results.  Triangle congruence theorems would not be used here in our course because we would not have discussed them.




\timing{The entire activity would likely take us about 1.5 class periods.  Each problem grouping is enough for about half a class given our typical 5-10 minutes for students to work in groups, and then 15-20 minutes of whole-class discussion.  We have also found many of these problems to be excellent for homework.}

\end{instructorNotes}





\end{document}