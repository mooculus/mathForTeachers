\documentclass[nooutcomes]{ximera}
\usepackage{gensymb}
\usepackage{tabularx}
\usepackage{mdframed}
\usepackage{pdfpages}
%\usepackage{chngcntr}

\let\problem\relax
\let\endproblem\relax

\newcommand{\property}[2]{#1#2}




\newtheoremstyle{SlantTheorem}{\topsep}{\fill}%%% space between body and thm
 {\slshape}                      %%% Thm body font
 {}                              %%% Indent amount (empty = no indent)
 {\bfseries\sffamily}            %%% Thm head font
 {}                              %%% Punctuation after thm head
 {3ex}                           %%% Space after thm head
 {\thmname{#1}\thmnumber{ #2}\thmnote{ \bfseries(#3)}} %%% Thm head spec
\theoremstyle{SlantTheorem}
\newtheorem{problem}{Problem}[]

%\counterwithin*{problem}{section}



%%%%%%%%%%%%%%%%%%%%%%%%%%%%Jenny's code%%%%%%%%%%%%%%%%%%%%

%%% Solution environment
%\newenvironment{solution}{
%\ifhandout\setbox0\vbox\bgroup\else
%\begin{trivlist}\item[\hskip \labelsep\small\itshape\bfseries Solution\hspace{2ex}]
%\par\noindent\upshape\small
%\fi}
%{\ifhandout\egroup\else
%\end{trivlist}
%\fi}
%
%
%%% instructorIntro environment
%\ifhandout
%\newenvironment{instructorIntro}[1][false]%
%{%
%\def\givenatend{\boolean{#1}}\ifthenelse{\boolean{#1}}{\begin{trivlist}\item}{\setbox0\vbox\bgroup}{}
%}
%{%
%\ifthenelse{\givenatend}{\end{trivlist}}{\egroup}{}
%}
%\else
%\newenvironment{instructorIntro}[1][false]%
%{%
%  \ifthenelse{\boolean{#1}}{\begin{trivlist}\item[\hskip \labelsep\bfseries Instructor Notes:\hspace{2ex}]}
%{\begin{trivlist}\item[\hskip \labelsep\bfseries Instructor Notes:\hspace{2ex}]}
%{}
%}
%% %% line at the bottom} 
%{\end{trivlist}\par\addvspace{.5ex}\nobreak\noindent\hung} 
%\fi
%
%


\let\instructorNotes\relax
\let\endinstructorNotes\relax
%%% instructorNotes environment
\ifhandout
\newenvironment{instructorNotes}[1][false]%
{%
\def\givenatend{\boolean{#1}}\ifthenelse{\boolean{#1}}{\begin{trivlist}\item}{\setbox0\vbox\bgroup}{}
}
{%
\ifthenelse{\givenatend}{\end{trivlist}}{\egroup}{}
}
\else
\newenvironment{instructorNotes}[1][false]%
{%
  \ifthenelse{\boolean{#1}}{\begin{trivlist}\item[\hskip \labelsep\bfseries {\Large Instructor Notes: \\} \hspace{\textwidth} ]}
{\begin{trivlist}\item[\hskip \labelsep\bfseries {\Large Instructor Notes: \\} \hspace{\textwidth} ]}
{}
}
{\end{trivlist}}
\fi


%% Suggested Timing
\newcommand{\timing}[1]{{\bf Suggested Timing: \hspace{2ex}} #1}




\hypersetup{
    colorlinks=true,       % false: boxed links; true: colored links
    linkcolor=blue,          % color of internal links (change box color with linkbordercolor)
    citecolor=green,        % color of links to bibliography
    filecolor=magenta,      % color of file links
    urlcolor=cyan           % color of external links
}

\title{The Mighty Triangle}
\author{Vic Ferdinand \& Betsy McNeal \& Jenny Sheldon}


\begin{document}
\begin{abstract}
\end{abstract}
\maketitle

\begin{instructorIntro}
This activity was invented to talk more about properties of triangles in the event we have extra time before Exam 1.  So, if you need a catch-up day, feel free to just skip this.

The goals are to talk about properties of triangles and to understand how all other polygons can be made out of triangles.  So, knowing something about triangles actually tells us something about other polygons as well.

Problems 1 - 2 talk about how we can use the interior angle sum of a triangle to get a formula for the interior angle sum of other polygons.  This might have been discussed in homework, but there are usually some fine points that are nice to talk about as a group.

Problems 3 - 5 discuss congruence of triangles in a light fashion.  This content is now a project at the end of the semester, so you don't want to go very much in depth with this part.  You don't want to go in to the triangle congruence theorems!  We are using these as further examples of some of the properties of parallelograms we introduced earlier in the course.

Problems 6 - 7 are explorations using our tools.  Students should try to do these problems with their protractors and compasses and then try to generalize their results.  Again, don't worry about introducing the triangle congruence theorems - these will be discussed if someone chooses the triangle congruence project.

You should feel free to choose any of the groupings of problems that seem interesting to you.  This is way too much for one class period.  Some of these things might have come up earlier or may have been assigned as homework.

\end{instructorIntro}

\begin{problem}
Erika claims, ``I think the number of triangles in a polygon is always two less than the number of sides.  Like, if you have a six-sided figure, then it has four triangles."  The picture on the left shows Erika's work.  However, her classmate Sarah is confused because she has drawn a picture of an octagon with eight triangles in it (the picture on the right). Compare Sarah's idea and Erika's idea: can they both be correct? Why or why not?
		\[
		\includegraphics[height=1in]{graphics/hw1-3new.pdf}
		\]
		
		\begin{instructorNotes}
		One of the main ideas in this problem is that we can't just count up the number of triangles, we actually need to look at which angles are relevant for the interior angle sum of the polygon. So, we end up subtracting the $360\degree$ in Sarah's idea because we have this extra full turn.
		
		Some additional extensions:  It's nice to talk about why there are specifically two fewer triangles than the number of sides, but this argument is pretty tough for everyone to understand.  You might also ask whether these strategies are generalizable to polygons with other numbers of sides or polygons which are not convex.  In particular, there are ways to draw the ``two fewer triangles'' that work for every polygon (such as Erika's example) but there are other ways that don't generalize.  You could have students come up with their own examples and discuss this concept.
		\end{instructorNotes}
\end{problem}

\begin{problem}
Use what we've discussed about the interior angle sum of a triangle to find a formula for the interior angle sum of:
\begin{enumerate}
    \item a non-convex quadrilateral.
    \item a regular 12-gon.
    \item a regular $n$-gon. (What about a non-regular $n$-gon?)
\end{enumerate}

\begin{instructorNotes}
This should be a straightforward application of the previous problem.  You could postpone most of your discussion from problem \#1 until students have tried this problem.

\end{instructorNotes}
\end{problem}

\newpage
\begin{problem}
What should it mean for two shapes to be ``congruent''?  Be as specific as possible.
\begin{instructorNotes}
The answer here should be that they are the same - specifically they have the same side lengths and the same angles in the same configuration.
\end{instructorNotes}
\end{problem}


\begin{problem}
Using your definition above and the properties we've discussed, explain how you know that the two triangles formed by this parallelogram and its diagonal are congruent.
\[
\includegraphics[height=1in]{graphics/parallelogramDiagonal.pdf}
\]

\begin{instructorNotes}
Using the definition above, we need to check (using the properties of parallelograms that we discussed, but didn't really prove) that all of the angles are the same and all of the side lengths are the same.  It's appropriate to do this both by cutting/folding/mathcing as well as by reasoning using the properties.
\end{instructorNotes}
\end{problem}

\begin{problem}
Assume that drawing the diagonal in this rhombus makes two congruent triangles.  Use this information to show that the rhombus must also be a parallelogram.
\[
\includegraphics[height=1.5in]{graphics/rhombus.pdf}
\]

\begin{instructorNotes}
This is the logical opposite of the previous problem.  Here we assume we have congruent triangles and explain why we get a parallelogram, whereas in the previous problem, we started with a parallelogram and explained why the triangles are congruent.
\end{instructorNotes}
\end{problem}

\begin{problem}
How many different triangles can you draw whose side lengths are $2$cm, $3cm$, and $4cm$?  You should use your compass to help you answer this question, and describe how you know you got them all.
\end{problem}

\begin{problem}
How many different triangles can you draw whose angle measures are $20\degree$, $40\degree$, and $120\degree$?  You should use your protractor to help you answer this question, and describe how you know you got them all.
\end{problem}




\end{document}