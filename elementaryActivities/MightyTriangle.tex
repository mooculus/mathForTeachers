\documentclass[nooutcomes]{ximera}

\graphicspath{
  {./}
  {graphics/}
  {../graphics/}
}

\usepackage{chngcntr}

\let\question\relax
\let\endquestion\relax




\newtheoremstyle{SlantTheorem}{\topsep}{\fill}%%% space between body and thm
%\newtheoremstyle{SlantTheorem}{\topsep}{\topsep}%%% space between body and thm
 {\slshape}                      %%% Thm body font
 {}                              %%% Indent amount (empty = no indent)
 {\bfseries\sffamily}            %%% Thm head font
 {}                              %%% Punctuation after thm head
 {3ex}                           %%% Space after thm head
 {\thmname{#1}\thmnumber{ #2}\thmnote{ \bfseries(#3)}}%%% Thm head spec
\theoremstyle{SlantTheorem}
\newtheorem{question}{Question}
\counterwithin*{question}{section}



\let\instructorNotes\relax
\let\endinstructorNotes\relax
%%% instructorNotes environment
\ifhandout
\newenvironment{instructorNotes}[1][false]%
{%
\def\givenatend{\boolean{#1}}\ifthenelse{\boolean{#1}}{\begin{trivlist}\item}{\setbox0\vbox\bgroup}{}
}
{%
\ifthenelse{\givenatend}{\end{trivlist}}{\egroup}{}
}
\else
\newenvironment{instructorNotes}[1][false]%
{%
  \ifthenelse{\boolean{#1}}{\begin{trivlist}\item[\hskip \labelsep\bfseries {\Large Instructor Notes: \\} \hspace{\textwidth} ]}
{\begin{trivlist}\item[\hskip \labelsep\bfseries {\Large Instructor Notes: \\} \hspace{\textwidth} ]}
{}
}
{\end{trivlist}}
\fi


%% Suggested Timing
\newcommand{\timing}[1]{{\bf Suggested Timing: \hspace{2ex}} #1}

\title{The Mighty Triangle}
\author{Vic Ferdinand, Betsy McNeal, Jenny Sheldon}


\begin{document}
\begin{abstract}
\end{abstract}
\maketitle

\begin{instructorIntro}
This activity was invented to talk more about properties of triangles in the event we have extra time before Exam 1.  So, if you need a catch-up day, feel free to just skip this.

The goals are to talk about properties of triangles and to understand how all other polygons can be made out of triangles.  So, knowing something about triangles actually tells us something about other polygons as well.

The first two problems talk about how we can use the interior angle sum of a triangle to get a formula for the interior angle sum of other polygons.  This might have been discussed in homework, but there are usually some fine points that are nice to talk about as a group.

The next three problems discuss congruence of triangles in a light fashion.  We have a project at the end of the semester detailing this content, so the activity doesn't go very much in depth with this topic.  We are using these as further examples of some of the properties of parallelograms we introduced earlier in the course.

The last two problems are explorations using our tools.  Students should try to do these problems with their protractors and compasses and then try to generalize their results.  Don't worry about introducing the triangle congruence theorems - these are better discussed elsewhere.

You should feel free to choose any of the groupings of problems that seem interesting to you.  This is way too much for one class period.  Some of these things might have come up earlier or may have been assigned as homework.

\end{instructorIntro}

\begin{problem}
Erika claims, ``I think the number of triangles in a polygon is always two less than the number of sides.  Like, if you have a six-sided figure, then it has four triangles."  The picture on the left shows Erika's work.  However, her classmate Sarah is confused because she has drawn a picture of an octagon with eight triangles in it (the picture on the right). Compare Sarah's idea and Erika's idea: can they both be correct? Why or why not?
		\[
		\includegraphics[height=1in]{graphics/hw1-3new.pdf}
		\]
		
		\begin{instructorNotes}
		One of the main ideas in this problem is that we can't just count up the number of triangles, we actually need to look at which angles are relevant for the interior angle sum of the polygon.
		
		Some additional extensions:  It's nice to talk about why there are specifically two fewer triangles than the number of sides, but this argument is pretty tough to make clear for the entire class.  You might also ask whether these strategies are generalizable to polygons with other numbers of sides or polygons which are not convex.  In particular, there are ways to draw the ``two fewer triangles'' that work for every polygon (such as Erika's example) but there are other ways that don't generalize.  You could have students come up with their own examples and discuss this concept.
		\end{instructorNotes}
\end{problem}

\begin{problem}
Use what we've discussed about the interior angle sum of a triangle to find a formula for the interior angle sum of:
\begin{enumerate}
    \item a non-convex quadrilateral.
    \item a regular 12-gon.
    \item a regular $n$-gon. (What about a non-regular $n$-gon?)
\end{enumerate}

\begin{instructorNotes}
This should be a straightforward application of the previous problem.

\end{instructorNotes}
\end{problem}

\newpage
\begin{problem}
What should it mean for two shapes to be ``congruent''?  Be as specific as possible.
\end{problem}


\begin{problem}
Using your definition above and the properties we've discussed, explain how you know that the two triangles formed by this parallelogram and its diagonal are congruent.
\[
\includegraphics[height=1in]{graphics/parallelogramDiagonal.pdf}
\]

\begin{instructorNotes}
Using the definition above, we need to check (using the properties of parallelograms that we discussed, but didn't really prove) that all of the angles are the same and all of the side lengths are the same.  It's appropriate to do this both by cutting/folding/mathcing as well as by reasoning using the properties.
\end{instructorNotes}
\end{problem}

\begin{problem}
Assume that drawing the diagonal in this rhombus makes two congruent triangles.  Use this information to show that the rhombus must also be a parallelogram.
\[
\includegraphics[height=1.5in]{graphics/rhombus.pdf}
\]

\begin{instructorNotes}
This is the logical opposite of the previous problem.  Here we assume we have congruent triangles and explain why we get a parallelogram, whereas in the previous problem, we started with a parallelogram and explained why the triangles are congruent.
\end{instructorNotes}
\end{problem}

\begin{problem}
How many different triangles can you draw whose side lengths are $2$cm, $3cm$, and $4cm$?  You should use your compass to help you answer this question, and describe how you know you got them all.
\end{problem}

\begin{problem}
How many different triangles can you draw whose angle measures are $20\degree$, $40\degree$, and $120\degree$?  You should use your protractor to help you answer this question, and describe how you know you got them all.
\end{problem}




\end{document}