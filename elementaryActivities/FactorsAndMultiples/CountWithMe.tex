\documentclass{ximera}
\usepackage{gensymb}
\usepackage{tabularx}
\usepackage{mdframed}
\usepackage{pdfpages}
%\usepackage{chngcntr}

\let\problem\relax
\let\endproblem\relax

\newcommand{\property}[2]{#1#2}




\newtheoremstyle{SlantTheorem}{\topsep}{\fill}%%% space between body and thm
 {\slshape}                      %%% Thm body font
 {}                              %%% Indent amount (empty = no indent)
 {\bfseries\sffamily}            %%% Thm head font
 {}                              %%% Punctuation after thm head
 {3ex}                           %%% Space after thm head
 {\thmname{#1}\thmnumber{ #2}\thmnote{ \bfseries(#3)}} %%% Thm head spec
\theoremstyle{SlantTheorem}
\newtheorem{problem}{Problem}[]

%\counterwithin*{problem}{section}



%%%%%%%%%%%%%%%%%%%%%%%%%%%%Jenny's code%%%%%%%%%%%%%%%%%%%%

%%% Solution environment
%\newenvironment{solution}{
%\ifhandout\setbox0\vbox\bgroup\else
%\begin{trivlist}\item[\hskip \labelsep\small\itshape\bfseries Solution\hspace{2ex}]
%\par\noindent\upshape\small
%\fi}
%{\ifhandout\egroup\else
%\end{trivlist}
%\fi}
%
%
%%% instructorIntro environment
%\ifhandout
%\newenvironment{instructorIntro}[1][false]%
%{%
%\def\givenatend{\boolean{#1}}\ifthenelse{\boolean{#1}}{\begin{trivlist}\item}{\setbox0\vbox\bgroup}{}
%}
%{%
%\ifthenelse{\givenatend}{\end{trivlist}}{\egroup}{}
%}
%\else
%\newenvironment{instructorIntro}[1][false]%
%{%
%  \ifthenelse{\boolean{#1}}{\begin{trivlist}\item[\hskip \labelsep\bfseries Instructor Notes:\hspace{2ex}]}
%{\begin{trivlist}\item[\hskip \labelsep\bfseries Instructor Notes:\hspace{2ex}]}
%{}
%}
%% %% line at the bottom} 
%{\end{trivlist}\par\addvspace{.5ex}\nobreak\noindent\hung} 
%\fi
%
%


\let\instructorNotes\relax
\let\endinstructorNotes\relax
%%% instructorNotes environment
\ifhandout
\newenvironment{instructorNotes}[1][false]%
{%
\def\givenatend{\boolean{#1}}\ifthenelse{\boolean{#1}}{\begin{trivlist}\item}{\setbox0\vbox\bgroup}{}
}
{%
\ifthenelse{\givenatend}{\end{trivlist}}{\egroup}{}
}
\else
\newenvironment{instructorNotes}[1][false]%
{%
  \ifthenelse{\boolean{#1}}{\begin{trivlist}\item[\hskip \labelsep\bfseries {\Large Instructor Notes: \\} \hspace{\textwidth} ]}
{\begin{trivlist}\item[\hskip \labelsep\bfseries {\Large Instructor Notes: \\} \hspace{\textwidth} ]}
{}
}
{\end{trivlist}}
\fi


%% Suggested Timing
\newcommand{\timing}[1]{{\bf Suggested Timing: \hspace{2ex}} #1}




\hypersetup{
    colorlinks=true,       % false: boxed links; true: colored links
    linkcolor=blue,          % color of internal links (change box color with linkbordercolor)
    citecolor=green,        % color of links to bibliography
    filecolor=magenta,      % color of file links
    urlcolor=cyan           % color of external links
}

\title{Count with Me}

\begin{document}
\begin{abstract} We consider a real-life scenario. \end{abstract}
\maketitle

Luke is in the 2nd grade, so he has been working on all kinds of addition and subtraction problems, but hasn't practiced much multiplication yet. Jenny and Luke recently had a conversation about Luke’s math homework. The paper asked him to do an addition task on a grid, and then color in all of the ``count by 2'' numbers.

{\bf Jenny:} Count by 2 numbers?? Luke, do you know what an ``even'' number is?\\
{\bf Luke:} Of course! They’re the ones that have a buddy.\\
{\bf Jenny:} That’s a great answer!\\
{\bf Luke:} I also know my count by 3 numbers! 3…6…9…12…15…18…21…24…27…\\
{\bf Jenny:} Wow! You’re pretty good at that!\\
{\bf Luke:} And my count by 4 numbers!\\
{\bf Jenny:} Count by 4 numbers are hard! Let’s hear them.\\
{\bf Luke and Jenny together:} 4…8…12…16…20…24…28…32…\\
{\bf Jenny:} I noticed something. 24 is on BOTH your count-by-3 list and your count-by-4 list. Do you think there are other numbers on both lists?\\
{\bf Luke:} (thinks for a while) 240?\\
{\bf Jenny:} Why’d you say that one?\\
{\bf Luke:} Because it’s a lot like 24?\\
{\bf Jenny:} That’s true. (Thinks for a minute) I think you’re right about 240. Do you know any others?\\
{\bf Luke:} 2,400! \\
{\bf Jenny:} Wow!\\
{\bf Luke:} 240,000! And really, you could add any number of zeros to the end!\\
{\bf Jenny:} Yikes! You didn’t just find one more, you found infinitely many more!\\


\begin{problem} (Warm-up!) Why did Jenny ask Luke about ``even'' numbers, when the homework asked about ``count-by-2'' numbers?
\end{problem}

\begin{problem} Rephrase ``count-by-3 numbers'' into something about factors and multiples. What is this list of numbers?
\end{problem}


\begin{problem} Jenny’s question is about numbers on both lists. Rephrase this question into something about factors and multiples. What is Jenny asking Luke to do?
\end{problem}

\begin{problem} Did Luke give the answer you were expecting? Why or why not?
\end{problem}

\begin{problem} What do you think Luke means by ``240 is a lot like 24''?
\end{problem}

\begin{problem}
Is every odd number also a count-by-three number? Explain why or why not.
\end{problem}


\begin{problem} Why is Luke right that 240 is on both the count-by-three list and the count-by-four list? Use the rephrased versions of the questions about factors and multiples to answer this question.
\end{problem}

\begin{problem} Why does a number that’s a multiple of 10 appear on the count-by-3 list? Does it matter that 3 isn't a factor of 10?
\end{problem}

\begin{problem} Why is Luke right about 24 with any number of zeros on the end being on both lists? Use the rephrased versions of the questions about factors and multiples to answer this question.
\end{problem}

\begin{problem}
Luke found a lot of numbers on both lists, but he didn't find them all. Can you find them all? Explain your thinking.
\end{problem}

%\begin{question}
%I overheard someone suggest that perhaps Luke just ``squished'' together a 24 and a 10 to get another number that worked. Does this work with other numbers? For instance, can I squish together 24 and 3 (making 243)
%\end{question}



\newpage
\begin{instructorNotes}

{\bf Main Goal:} We investigate a problem about common factors.


{\bf Overall Picture:} We want to give our students a realistic look at the kinds of arguments about numbers that kids can construct. This is a true story! We want to see how this second-grade-appropriate content is related to some of the content we might remember from high school, and how Luke is building this foundation now.

\begin{itemize}
\item We want to rephrase the ``count-by'' lists as lists of multiples, but it’s good to practice with our students both the meaning of factors and multiples.
\item Once we make our connection with factors and multiples, we can rephrase the question as being about common multiples. Notice that we’re not looking for the least common multiple (but Jenny left that option for Luke in her initial observation). Sometimes students jump to least common multiple questions every time they see a common multiple. It’s worth noting that Luke didn’t do this.
\item Luke’s observation that ``240 is a lot like 24'' should be connected to the bundling or base ten system. We don’t know what Luke was actually thinking, so students should be encouraged to come up with their own interpretation. 
\item Students should use both pictures and algebraic ideas to justify their thinking. Especially with larger numbers like 240 and 2400, base ten pictures should be encouraged. However, the base-ten pictures might highlight the reason the questions about numbers divisible by ten can be multiples of 3 and 4: the bundled objects themselves aren’t naturally grouped in threes or fours. However, if we think of 240 as 24 tens (or 2400 as 24 hundreds), we can distribute these larger objects instead of individual cubes. This is a nice connection to the division algorithm!

\end{itemize}




{\bf Good Language:} Continue to help students to be careful distinguishing factors from multiples. We have found this to be a challenge!



{\bf Suggested Timing:} Give students about 10-15 minutes to read the story and answer the questions on their own. Then discuss, including having students present their drawings and algebraic arguments to the later questions. If you have extra time, choose a problem or two from Number Puzzles to have students work on.

\end{instructorNotes}


\end{document}