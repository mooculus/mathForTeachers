\documentclass[nooutcomes, noauthor]{ximera}

\graphicspath{
  {./}
  {graphics/}
  {../graphics/}
}

\usepackage{chngcntr}

\let\question\relax
\let\endquestion\relax




\newtheoremstyle{SlantTheorem}{\topsep}{\fill}%%% space between body and thm
%\newtheoremstyle{SlantTheorem}{\topsep}{\topsep}%%% space between body and thm
 {\slshape}                      %%% Thm body font
 {}                              %%% Indent amount (empty = no indent)
 {\bfseries\sffamily}            %%% Thm head font
 {}                              %%% Punctuation after thm head
 {3ex}                           %%% Space after thm head
 {\thmname{#1}\thmnumber{ #2}\thmnote{ \bfseries(#3)}}%%% Thm head spec
\theoremstyle{SlantTheorem}
\newtheorem{question}{Question}
\counterwithin*{question}{section}



\let\instructorNotes\relax
\let\endinstructorNotes\relax
%%% instructorNotes environment
\ifhandout
\newenvironment{instructorNotes}[1][false]%
{%
\def\givenatend{\boolean{#1}}\ifthenelse{\boolean{#1}}{\begin{trivlist}\item}{\setbox0\vbox\bgroup}{}
}
{%
\ifthenelse{\givenatend}{\end{trivlist}}{\egroup}{}
}
\else
\newenvironment{instructorNotes}[1][false]%
{%
  \ifthenelse{\boolean{#1}}{\begin{trivlist}\item[\hskip \labelsep\bfseries {\Large Instructor Notes: \\} \hspace{\textwidth} ]}
{\begin{trivlist}\item[\hskip \labelsep\bfseries {\Large Instructor Notes: \\} \hspace{\textwidth} ]}
{}
}
{\end{trivlist}}
\fi


%% Suggested Timing
\newcommand{\timing}[1]{{\bf Suggested Timing: \hspace{2ex}} #1}
\title{Common Time}

\begin{document}
\begin{abstract}
\end{abstract}
\maketitle

\begin{problem}
For many years, hot dogs came in packages of $10$ and hot dog buns came in packages of $8$. What is the smallest number of packages of hot dogs  and hot dog buns you can buy so that every hotdog has exactly one bun? Did you get the same answer as each of your neighbors?

\end{problem}


\begin{problem}
You have received a large piece of paper which measures 40 inches by 60 inches.  You'd like to draw a large grid (of rectangles, not necessarily squares) on this paper.  What size rectangles could you use for your grid?  Did you get the same answer as each of your neighbors?
\vfill
\end{problem}


\begin{problem}
At a local carnival, you sneak in behind one of the slot machines and rig it to come up with three cherries (one in each window to hit the jackpot) at a time only known to you.  You know the first window will come up with a cherry every $15$ times, the second window every $24$ times, and the third window every $40$ times.  When should you step in to win the jackpot?  Did you get the same answer as each of your neighbors?
\vfill
\end{problem}





\begin{problem}
You are making goodie bags for a party. You have 48 cookies that you can distribute, and you have 30 pencils that you can distribute. How many goodie bags can you make if each goodie bag has the same number of cookies, the same (possibly different) number of pencils, and you have nothing left over? Did you get the same answer as each of your neighbors?

\end{problem}


\begin{problem}
Now that you've solved them all, what are the similarities and differences amongst the problems on this page?
\end{problem}

\newpage
\begin{instructorNotes}

{\bf Main Goal:} We would like students to solve problems using the ideas of factors and multiples. These problems can also be used to introduce the idea of a greatest common factor and a least common multiple.

\begin{itemize}
	\item Students are not necessarily expecting to find either GCF or LCM values in these problems. If a student catches on to this idea, we emphasize explaining why they are looking for factors or multiples (using the definition), why they are looking for something in common (using the problem situation) and why they are looking for the greatest or least (using the problem situation).
	\item Many of these problems have multiple answers, so actually the GCF or LCM is only one possible answer in that case. It's still good for our students to see problems that have more than one correct answer at this point in the semester! Remind them to justify their thinking even if their group members have different answers.
	\item In our experience, most students come up with common multiples from listing by ``brute force''. This is a good starting method for the children they will teach, so we make a point to discuss it.
	\item Another good solution students often use is factoring the numbers into primes. This is an excellent solution to discuss in detail. Be sure to have students explain why the prime numbers were important and how they were used in the solution. Ask the students to connect their reasoning back to the problem situation.
	\item Once all of the solutions have been presented, draw the class discussion towards the similarities and differences between these problems. This should help point towards the similarities and differences between factors and multiples, and how the problem situation tells us whether we are looking for factors or for multiples.
	\item If time, we sometimes extend the discussion by asking how their methods would change if the numbers were ``not so nice'' (or very large).
	\item If time, this introduction to LCM because can help us make the point that the LCM is a ``common time'' of things that grow at a constant rate. (``slope with $y$-intercept = 0'', though most students will not be ready to think about linear relationships this way yet)
\end{itemize}

{\bf Suggested Timing:} This activity should take a whole class period. You can give students about 20 minutes in groups to work on these problems, then use the remaining time for presentations and discussion.
\end{instructorNotes}




\end{document}