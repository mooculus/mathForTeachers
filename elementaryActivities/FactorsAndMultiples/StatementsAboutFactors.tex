\documentclass[nooutcomes]{ximera}

\usepackage{gensymb}
\usepackage{tabularx}
\usepackage{mdframed}
\usepackage{pdfpages}
%\usepackage{chngcntr}

\let\problem\relax
\let\endproblem\relax

\newcommand{\property}[2]{#1#2}




\newtheoremstyle{SlantTheorem}{\topsep}{\fill}%%% space between body and thm
 {\slshape}                      %%% Thm body font
 {}                              %%% Indent amount (empty = no indent)
 {\bfseries\sffamily}            %%% Thm head font
 {}                              %%% Punctuation after thm head
 {3ex}                           %%% Space after thm head
 {\thmname{#1}\thmnumber{ #2}\thmnote{ \bfseries(#3)}} %%% Thm head spec
\theoremstyle{SlantTheorem}
\newtheorem{problem}{Problem}[]

%\counterwithin*{problem}{section}



%%%%%%%%%%%%%%%%%%%%%%%%%%%%Jenny's code%%%%%%%%%%%%%%%%%%%%

%%% Solution environment
%\newenvironment{solution}{
%\ifhandout\setbox0\vbox\bgroup\else
%\begin{trivlist}\item[\hskip \labelsep\small\itshape\bfseries Solution\hspace{2ex}]
%\par\noindent\upshape\small
%\fi}
%{\ifhandout\egroup\else
%\end{trivlist}
%\fi}
%
%
%%% instructorIntro environment
%\ifhandout
%\newenvironment{instructorIntro}[1][false]%
%{%
%\def\givenatend{\boolean{#1}}\ifthenelse{\boolean{#1}}{\begin{trivlist}\item}{\setbox0\vbox\bgroup}{}
%}
%{%
%\ifthenelse{\givenatend}{\end{trivlist}}{\egroup}{}
%}
%\else
%\newenvironment{instructorIntro}[1][false]%
%{%
%  \ifthenelse{\boolean{#1}}{\begin{trivlist}\item[\hskip \labelsep\bfseries Instructor Notes:\hspace{2ex}]}
%{\begin{trivlist}\item[\hskip \labelsep\bfseries Instructor Notes:\hspace{2ex}]}
%{}
%}
%% %% line at the bottom} 
%{\end{trivlist}\par\addvspace{.5ex}\nobreak\noindent\hung} 
%\fi
%
%


\let\instructorNotes\relax
\let\endinstructorNotes\relax
%%% instructorNotes environment
\ifhandout
\newenvironment{instructorNotes}[1][false]%
{%
\def\givenatend{\boolean{#1}}\ifthenelse{\boolean{#1}}{\begin{trivlist}\item}{\setbox0\vbox\bgroup}{}
}
{%
\ifthenelse{\givenatend}{\end{trivlist}}{\egroup}{}
}
\else
\newenvironment{instructorNotes}[1][false]%
{%
  \ifthenelse{\boolean{#1}}{\begin{trivlist}\item[\hskip \labelsep\bfseries {\Large Instructor Notes: \\} \hspace{\textwidth} ]}
{\begin{trivlist}\item[\hskip \labelsep\bfseries {\Large Instructor Notes: \\} \hspace{\textwidth} ]}
{}
}
{\end{trivlist}}
\fi


%% Suggested Timing
\newcommand{\timing}[1]{{\bf Suggested Timing: \hspace{2ex}} #1}




\hypersetup{
    colorlinks=true,       % false: boxed links; true: colored links
    linkcolor=blue,          % color of internal links (change box color with linkbordercolor)
    citecolor=green,        % color of links to bibliography
    filecolor=magenta,      % color of file links
    urlcolor=cyan           % color of external links
}


\title{Statements About Factors}
\author{Vic Ferdinand, Betsy McNeal, Jenny Sheldon}

\begin{document}
\begin{abstract} \end{abstract}
\maketitle




Remember, \emph{a is a factor of b} if there is some integer $q$ such that $b = a \cdot q$.
\begin{problem}
 Make some true statements using this terminology.
\end{problem}

\begin{problem}
Use the definition of \emph{factor} to decide which of the following are true and which are false. If it is true, find $q$ satisfying the definition of a factor. If it is false, give an explanation.

\begin{enumerate}
\item 7 is a factor of 56 
\item \label{Statementsb} 21 is a factor of 2121
\item \label{Statementsc} 3 is a factor of $9 \times 41$
\item \label{Statementsd} 2 is a factor of $(3^{16} \times 5^4 \times 7^8)$
\item \label{Statementse} 6 is a factor of $(2^4 \times 3^2 \times 7^3 \times 13^5)$?
\item \label{Statementsf} 100000 is a factor of  $(2^3 \times 3^9 \times 5^{11} \times17^8)$
\item \label{Statementsg} 6000  is a factor of  $(2^{21} \times 3^{17} \times 5^{89} \times 29^{37})$
\item \label{Statementsh} $p^3q^5r$  is a factor of $(p^5q^{13}r^7s^2t^{27}) $
\item \label{Statementsi} 7 is a factor of $(5 \times (21+14))$
\end{enumerate}
\end{problem}


\newpage
\begin{instructorNotes}
The purpose of this activity is to give students more practice with the definition of factors, and more practice with exponential notation.

\begin{itemize}
    \item  We usually include a review of exponential notation, as we have found many of our students to struggle with this concept.
    \item This activity can make some later activities using prime factorization to find GCF and LCM a little easier for students.  It could also be done after GCF and LCM are introduced and reframed for more practice on that topic.
    \item We often don't have time for all of these problems, so we pick a subset which seems to cover most of the ideas, like \ref{Statementsb}, \ref{Statementse}, \ref{Statementsf}, and \ref{Statementsh}.
    \item A related question to \ref{Statementsf} which we sometimes ask is ``How many zeros are at the end of the number $2^3\times 3^9\times 5^{11} \times 17^8$?''.  This can later be connected to the prime factorization of the denominator of a fraction in  the ``Shampoo, Rinse, Repeat'' activity.
\end{itemize}







{\bf Suggested Timing:} If we fully explore all parts of this activity, it will take a full class period.  To fit it in half, we choose some parts and leave others for homework.
\end{instructorNotes}

\end{document}