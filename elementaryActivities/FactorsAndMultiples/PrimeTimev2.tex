\documentclass[noauthor, nooutcomes]{ximera}

\usepackage{gensymb}
\usepackage{tabularx}
\usepackage{mdframed}
\usepackage{pdfpages}
%\usepackage{chngcntr}

\let\problem\relax
\let\endproblem\relax

\newcommand{\property}[2]{#1#2}




\newtheoremstyle{SlantTheorem}{\topsep}{\fill}%%% space between body and thm
 {\slshape}                      %%% Thm body font
 {}                              %%% Indent amount (empty = no indent)
 {\bfseries\sffamily}            %%% Thm head font
 {}                              %%% Punctuation after thm head
 {3ex}                           %%% Space after thm head
 {\thmname{#1}\thmnumber{ #2}\thmnote{ \bfseries(#3)}} %%% Thm head spec
\theoremstyle{SlantTheorem}
\newtheorem{problem}{Problem}[]

%\counterwithin*{problem}{section}



%%%%%%%%%%%%%%%%%%%%%%%%%%%%Jenny's code%%%%%%%%%%%%%%%%%%%%

%%% Solution environment
%\newenvironment{solution}{
%\ifhandout\setbox0\vbox\bgroup\else
%\begin{trivlist}\item[\hskip \labelsep\small\itshape\bfseries Solution\hspace{2ex}]
%\par\noindent\upshape\small
%\fi}
%{\ifhandout\egroup\else
%\end{trivlist}
%\fi}
%
%
%%% instructorIntro environment
%\ifhandout
%\newenvironment{instructorIntro}[1][false]%
%{%
%\def\givenatend{\boolean{#1}}\ifthenelse{\boolean{#1}}{\begin{trivlist}\item}{\setbox0\vbox\bgroup}{}
%}
%{%
%\ifthenelse{\givenatend}{\end{trivlist}}{\egroup}{}
%}
%\else
%\newenvironment{instructorIntro}[1][false]%
%{%
%  \ifthenelse{\boolean{#1}}{\begin{trivlist}\item[\hskip \labelsep\bfseries Instructor Notes:\hspace{2ex}]}
%{\begin{trivlist}\item[\hskip \labelsep\bfseries Instructor Notes:\hspace{2ex}]}
%{}
%}
%% %% line at the bottom} 
%{\end{trivlist}\par\addvspace{.5ex}\nobreak\noindent\hung} 
%\fi
%
%


\let\instructorNotes\relax
\let\endinstructorNotes\relax
%%% instructorNotes environment
\ifhandout
\newenvironment{instructorNotes}[1][false]%
{%
\def\givenatend{\boolean{#1}}\ifthenelse{\boolean{#1}}{\begin{trivlist}\item}{\setbox0\vbox\bgroup}{}
}
{%
\ifthenelse{\givenatend}{\end{trivlist}}{\egroup}{}
}
\else
\newenvironment{instructorNotes}[1][false]%
{%
  \ifthenelse{\boolean{#1}}{\begin{trivlist}\item[\hskip \labelsep\bfseries {\Large Instructor Notes: \\} \hspace{\textwidth} ]}
{\begin{trivlist}\item[\hskip \labelsep\bfseries {\Large Instructor Notes: \\} \hspace{\textwidth} ]}
{}
}
{\end{trivlist}}
\fi


%% Suggested Timing
\newcommand{\timing}[1]{{\bf Suggested Timing: \hspace{2ex}} #1}




\hypersetup{
    colorlinks=true,       % false: boxed links; true: colored links
    linkcolor=blue,          % color of internal links (change box color with linkbordercolor)
    citecolor=green,        % color of links to bibliography
    filecolor=magenta,      % color of file links
    urlcolor=cyan           % color of external links
}


\title{Prime Time}
\author{Vic Ferdinand, Betsy McNeal, Jenny Sheldon}

\begin{document}
\begin{abstract} \end{abstract}
\maketitle



\begin{problem}
 Write the prime factorization of 60.  That is, start by writing any factorization of 60 into two factors.  Then split up the factors continually until it is impossible to break the factors down further.  Then write 60 as the product of those ``indivisible" numbers.
\end{problem} 
\begin{problem}
 Do this again, but start with a different two number factorization of 60.  What do you notice?
\end{problem} 
\begin{problem}
 List all the factors of 60.  Make as many observations about your work as you can.
\end{problem} 
\begin{problem}\label{PrimeTime4}
 Write the prime factorization of each of the factors you found in $\#$3.  Make as many observations about your work as you can.  %Be sure to write the powers of the prime numbers involved.
\end{problem} 

\begin{problem}\label{PrimeTimev25}
Is $2 \times 3 \times 7$ a factor of $60$? (Don't use your calculator!) How does your work in problem 4 help you to answer this question?
\end{problem}

%\begin{problem}
%How are we categorizing numbers in this activity?
%\end{problem}

\begin{problem}
Extend your ideas from problem \ref{PrimeTimev25} to answer the following questions without your calculator!
\begin{enumerate}
    \item Is $30$ a factor of $2 \times 3 \times 7 \times 7 \times 11$?
    \item Is $24$ a factor of $408$?
\end{enumerate}

(If you get done early, invent some puzzles similar to these for other people to solve.)
\end{problem}
%\begin{problem}
% From your work in Problem \ref{PrimeTime4}, what must be true about the prime factorization of a number in order for it to be a ``factor of 60"? 
%\end{problem} 


\newpage
\begin{instructorNotes}

{\bf Main goal:} This activity introduces prime factorization and helps us discuss why we can think of prime numbers as the building blocks for all numbers.

{\bf Overall notes:} 

The first two problems give students a chance to remember drawing factor trees and making prime factorizations. This is currently the first activity we have to discuss factoring and prime numbers, so you may also need to review the definition of ``prime'' if it hasn't come up. Be sure to pause in discussion any time you need to in order to make sure everyone is on the same page. In these two problems, we should be noticing that we always end up with the same prime factorization.

In problem 3 and 4, we are looking for the students to observe several important ideas.
\begin{itemize}
	\item All of the factors of 60 can be found in the factor trees we drew in 1 and 2, but not all of the factors are always present in a factor tree.
	\item We can produce the factors in a systematic fashion by increasing the divisor until the quotient is larger than the divisor. At this point we can stop, because we have already checked the opposite pairs. (You can also connect this to a similar strategy for testing for primality if it makes sense to do so.)
	\item The factors of 60 all have prime factorizations which are made up of (some or all) of the prime factors of 60.
	\item The factors of 60 split the prime factorization of 60. For instance, $60 = 2^2 \times 3 \times 5$, and we have $60 = 4 \times 15$ where $4 = 2^2$ and $15 = 3 \times 5$. Together, the prime factors of 4 and 15 make up all the prime factors of 60 (counting multiplicity). 
	\item We build 60 and all of its factors out of the same collection of prime numbers.
\end{itemize}
At this point in the discussion, it's good to stop and introduce the Unique Factorization Theorem/Fundamental Theorem of Arithmetic as a sort of summary of many of our observations. State the theorem clearly with at least one of the above names. You should also point out that the theorem helps us to think of the primes as building blocks or a sort of ``molecular structure'' for numbers.

Problems 5 and 6 help us see why the uniqueness of the factorization is important. If we know the prime factorization of a number, we know all the prime factors of every factor of that number (and we can't introduce any other primes). Help students to connect this work with factorization to the uniqueness of prime factorizations in your discussion. It's also good to practice showing that $A$ is a factor of $B$ by producing the $N$ you need to have $A \times N = B$. Students often forget this part of their own explanations, and we want students to see how prime factorization actually can make writing down the $N$ easier. This practice can also help later if we do problems where we find LCM and GCF using prime factorization.


You can wrap up this activity by discussing why 1 should not be considered a prime.  If students have wondered whether or not 1 is prime, we typically save this discussion for after we discuss this theorem. If you have discussed the Sieve of Eratosthenes, it can also be good to connect this discussion back to the sieve.

{\bf Good language:} Notice that we don't require the use of exponential notation. If you decide to use exponential notation, it's good to also write out the multiplication as long as the exponent is small enough (for instance, be in the habit of writing $2^2 = 2 \times 2$ if you decide to use the exponent). In our experience, for prospective elementary teachers, the exponential notation can be an extra hurdle to understanding what's happening with the prime factors.




{\bf Suggested timing: }  Let students work on problems 1--5 on the activity sheet for about $15$ minutes, then spend about 20 minutes having groups present each part. Spend time particularly on students' observations, and bring these together to introduce the Unique Factorization Theorem and its importance to prime factorization. If students struggle with seeing how Problem 5 can be done using the ideas of factorization, work through this problem together as a class. Next, give students about 5 minutes to work on Problem 6, then discuss with any remaining time.


\end{instructorNotes}


\end{document}
