\documentclass{ximera}

\graphicspath{
  {./}
  {graphics/}
  {../graphics/}
}

\usepackage{chngcntr}

\let\question\relax
\let\endquestion\relax




\newtheoremstyle{SlantTheorem}{\topsep}{\fill}%%% space between body and thm
%\newtheoremstyle{SlantTheorem}{\topsep}{\topsep}%%% space between body and thm
 {\slshape}                      %%% Thm body font
 {}                              %%% Indent amount (empty = no indent)
 {\bfseries\sffamily}            %%% Thm head font
 {}                              %%% Punctuation after thm head
 {3ex}                           %%% Space after thm head
 {\thmname{#1}\thmnumber{ #2}\thmnote{ \bfseries(#3)}}%%% Thm head spec
\theoremstyle{SlantTheorem}
\newtheorem{question}{Question}
\counterwithin*{question}{section}



\let\instructorNotes\relax
\let\endinstructorNotes\relax
%%% instructorNotes environment
\ifhandout
\newenvironment{instructorNotes}[1][false]%
{%
\def\givenatend{\boolean{#1}}\ifthenelse{\boolean{#1}}{\begin{trivlist}\item}{\setbox0\vbox\bgroup}{}
}
{%
\ifthenelse{\givenatend}{\end{trivlist}}{\egroup}{}
}
\else
\newenvironment{instructorNotes}[1][false]%
{%
  \ifthenelse{\boolean{#1}}{\begin{trivlist}\item[\hskip \labelsep\bfseries {\Large Instructor Notes: \\} \hspace{\textwidth} ]}
{\begin{trivlist}\item[\hskip \labelsep\bfseries {\Large Instructor Notes: \\} \hspace{\textwidth} ]}
{}
}
{\end{trivlist}}
\fi


%% Suggested Timing
\newcommand{\timing}[1]{{\bf Suggested Timing: \hspace{2ex}} #1}

\title{Starting Point}

\begin{document}
\begin{abstract} We set some ground work. \end{abstract}
\maketitle



\begin{problem} What does it mean for a number to be ``even''? Give some examples and some non-examples to illustrate your ideas.
\end{problem}

\begin{problem} How might a student in kindergarten think about the idea of being ``even''? What about a middle schooler? How about someone in high school?
\end{problem}


\begin{problem} Write down at least three clear meanings for ``even''. Draw some pictures to illustrate your ideas.
\end{problem}

\begin{problem} What does it mean for a number to be ``odd''? (Answer this question in at least three ways!) Give some examples and some non-examples to illustrate your ideas. Draw pictures as appropriate.
\end{problem}


\newpage

\begin{problem}
If you add two different even numbers, do you get an even number, an odd number, or neither? Solve this problem in two different ways, and use the meanings of even and odd to explain your thinking.
\end{problem}


\begin{problem}
If you multiply two different odd numbers, do you get an even number, an odd number, or neither? Solve this problem in two different ways, and use the meanings of even and odd to explain your thinking.
\end{problem}


\newpage



\begin{problem} What does it mean for a number to be a ``factor'' of another number? Give some numerical examples and non-examples to illustrate your thinking.
\end{problem}

\begin{problem} What does it mean for a number to be a ``multiple'' of another number? Give some numerical examples and non-examples to illustrate your thinking.
\end{problem}


\begin{problem} Give some real-world examples of factors and multiples. Where do these ideas come up in school or otherwise?
\end{problem}

\begin{problem} Rewrite at least one of your definitions of ``even'' to include the word ``factor''. Explain your thinking.
\end{problem}

\begin{problem} Rewrite at least one of your definitions of ``even'' to include the word ``multiple''. Explain your thinking.
\end{problem}





\newpage
\begin{instructorNotes}

{\bf Main Goal:} We define even, odd, factor, and multiple.


{\bf Overall Picture:} We are setting the stage for further work about factors and multiples. We will talk about these ideas in other contexts (including with divisibility, prime numbers, GCF and LCM problems, and possibly more), so we want to start on common ground.

\begin{itemize}
\item You can discuss the even and odd problems as a whole. As a class, make a big list of all of the students ideas. Some ideas we are looking for are the following, representing three different ``levels'' of thinking.

\begin{itemize}
	\item A ``buddy system'' idea: a number of items is even if we can line everyone up in pairs, and everyone has a buddy. There's a one-to-one correspondence idea here that's nice to highlight.
	\item A number is even if we can divide by two with no leftovers. We like to see this in two phrasings: we can make groups of 2 with no leftovers, and we can make two groups with no leftovers.
	\item A number is even if $2$ is a factor. If we haven't defined factors and multiples yet, you will want to stop and do that here.
	\item By high school, students might be thinking about an algebraic definition of even, like ``a number is even if it can be written in the form $n = 2m$ for some integer $m$.'' This is a bit more specific than we expect our students to be!
\end{itemize}
Note that while some of these ideas will work with negative numbers, we will mostly stay in the realm of positive whole numbers for our work on factors and multiples.

As you discuss, you can also label the different levels of these ideas. You will also need to distinguish between what we want as our definition of even and odd, and what we think of more as properties of even and odd. (For instance, students will likely suggest that an even number ends in 0, 2, 4, 6, or 8. This is more like a consequence of our definition than what we want to use as a definition.) To wrap up your work on this definition, be sure to define an odd number as a whole number which is not even.
\item For factors and multiples, you can again discuss the questions as a whole. The goal is mainly to give definitions and examples (as in: 2 is a factor of 10, because $2 \times 5 = 10$. 10 is a multiple of 2, because we can find an integer (5) to multiply 2 by and get 10. In other words, $2 \times 5 = 10$.)
\item The question about real-world examples of factors and multiples is a bonus here, but if you have time to discuss it, students should be more equipped to talk about these ideas moving forward.
\item While it's not the way our students will eventually discuss even and odd with young children, we want our students to recognize that the question of even or odd is actually just a question about factors or multiples, connecting these questions to our outcomes.
\end{itemize}


When students begin working with their definitions of even and odd, we want to emphasize using the definition.
\begin{itemize}
	\item Encourage students to justify their answers in more than one way. They should be able to draw several types of pictures to represent different ideas. Some students will also be able to work with algebraic expressions. If you have several students thinking along these ideas, it's good to have them present their ideas but emphasize to the whole class that as long as they can represent their ideas in more than one way, the algebra isn't necessary.
	\item Students sometimes like to add two even numbers by saying something like $2x + 2x$ instead of $2x + 2y$. If you see this, it's good to ask students whether they mean the same thing by both of those $x$'s, or whether they mean different things (and thus could be expressed with different variables).
	\item You may have students present four or five different meanings of even or odd, so you can encourage students to use any of these meanings. You can have a range of answers presented as well, which can lead to a rich discussion. 
	\item Other bonus questions that could be asked: is $0$ even, odd, or neither? What about $-5$? What about $\frac{1}{2}$? Do we need to extend our definitions of even and odd to apply to integers rather than just whole numbers?
\end{itemize}




{\bf Good Language:} Distinguishing factors from multiples is frequently a challenge! Be patient with yourself and your students, and watch out for anyone switching the meanings of the two words. Also be on the lookout for the language ``divides evenly'', which students use to mean ``divides with no remainder''. Since we are also talking about the ideas of ``even'' and ``odd'', we'd like to avoid confusion here, so we typically suggest ``divides equally'' or ``divides with no remainder'' as replacements for ``divides evenly''. You may have to practice this language yourself!



{\bf Suggested Timing:} Give students about 5 minutes to think about the even and odd questions on the first page, and then discuss as a class for about 10 minutes. Then give students about 5 more minutes to think about the  problems on the second page where they work with even and odd numbers. Spend 10-15 minutes in presentations here. Use the remaining time to introduce factors and multiples. This can be done by giving students about 5 minutes to think about the problems and using the remaining time to discuss, or, if you're short on time you can discuss these problems as a whole class. Wrap up by emphasizing that this unit concerns factors and multiples, so we would like students to focus on these ideas when they are thinking about even and odd numbers.

\end{instructorNotes}


\end{document}