\documentclass{ximera}

\graphicspath{
  {./}
  {graphics/}
  {../graphics/}
}

\usepackage{chngcntr}

\let\question\relax
\let\endquestion\relax




\newtheoremstyle{SlantTheorem}{\topsep}{\fill}%%% space between body and thm
%\newtheoremstyle{SlantTheorem}{\topsep}{\topsep}%%% space between body and thm
 {\slshape}                      %%% Thm body font
 {}                              %%% Indent amount (empty = no indent)
 {\bfseries\sffamily}            %%% Thm head font
 {}                              %%% Punctuation after thm head
 {3ex}                           %%% Space after thm head
 {\thmname{#1}\thmnumber{ #2}\thmnote{ \bfseries(#3)}}%%% Thm head spec
\theoremstyle{SlantTheorem}
\newtheorem{question}{Question}
\counterwithin*{question}{section}



\let\instructorNotes\relax
\let\endinstructorNotes\relax
%%% instructorNotes environment
\ifhandout
\newenvironment{instructorNotes}[1][false]%
{%
\def\givenatend{\boolean{#1}}\ifthenelse{\boolean{#1}}{\begin{trivlist}\item}{\setbox0\vbox\bgroup}{}
}
{%
\ifthenelse{\givenatend}{\end{trivlist}}{\egroup}{}
}
\else
\newenvironment{instructorNotes}[1][false]%
{%
  \ifthenelse{\boolean{#1}}{\begin{trivlist}\item[\hskip \labelsep\bfseries {\Large Instructor Notes: \\} \hspace{\textwidth} ]}
{\begin{trivlist}\item[\hskip \labelsep\bfseries {\Large Instructor Notes: \\} \hspace{\textwidth} ]}
{}
}
{\end{trivlist}}
\fi


%% Suggested Timing
\newcommand{\timing}[1]{{\bf Suggested Timing: \hspace{2ex}} #1}

\title{Number Puzzles}

\begin{document}
\begin{abstract} We use our knowledge of factors and multiples to solve puzzles. \end{abstract}
\maketitle


\begin{problem}
Can a number have both $15$ and $42$ as factors? Can a number have both $15$ and $42$ as multiples? Can a number have both $8$ and $32$ as factors? Can a number have both $8$ and $32$ as multiples? Is there a general pattern here?
\vskip 0.5in
\end{problem}


\begin{problem}
For each $A$ and $B$ below, is $A$ a factor of $B$? Explain how you know. (Notice that 3373 is a prime number!)
\begin{enumerate}
\item $A = 8$, $B = 2 \times 2 \times 2 \times 2 \times 2 \times 3 \times 3 \times 11 \times 11 \times 17 \times 17 \times 41 \times 41 \times 41 \times 41  \times 3373$
\item $A = 11 \times 3373$, $B = 2 \times 5 \times 5 \times 23 \times 23 \times 23 \times 23 \times 23 \times 41 \times 3373 \times 3373$
\end{enumerate}

\vskip 0.5in
\end{problem}


\begin{problem}
If you multiplied out the following number, how many zeroes would be at the end? (Don't try this with your calculator!)
\[
2 \times 2 \times 2 \times 5 \times 5 \times 11 \times 11 \times 29 \times 29 \times 29 \times 41 \times 41 \times 3373 \times 3373 \times 3373
\]
\vskip 0.5in

Hint: maybe try this with some smaller numbers first!
\vskip 0.5in
\end{problem}



\begin{problem}
Let $N$ be the number you get if you multiply together twenty $2$'s, thirty $5$'s, and one $11$. If you converted $\frac{1}{N}$ to a decimal, would the decimal terminate or repeat? (Don't try this with your calculator!)
\vskip 0.5in

Hint \#1: maybe try this with some smaller numbers first, like $2 \times 5 \times 11$.

Hint \#2: what does this question have to do with bundling?
\vskip 0.5in
\end{problem}



\begin{problem}
What is the smallest number of steps you could take to determine whether or not $929$ is prime? How do you know?
\vskip 0.5in
\end{problem}



\begin{problem}
Say that you have some number $A$ (maybe a little like 17, but not actually 17) which has a remainder of 2 when you divide by 5. And then say that you have some number $B$ (maybe a little like 23, but not actually 23) which has a remainder of 3 when you divide by 5. What would happen if you added $A+B$, and then divided by 5? Does it matter what $A$ and $B$ are? Why or why not? What if you multiplied $A\times B$ and divided by 5? Are there any other interesting questions you could ask, here?
\vskip 0.5in
\end{problem}


\begin{problem}
What are the least common multiple and the greatest common factor of the two numbers below?
\[
A = 2 \times 2 \times 3 \times 3 \times 3 \times 5 \times 5 \times 5 \times 5
\]
\[
B = 2 \times 3 \times 5 \times 5 \times 5 \times 5 \times 5 \times 5 \times 3371
\]
\vskip 0.5in
\end{problem}



\begin{problem}
What is the greatest common multiple of $48$ and $30$?
\vskip 0.5in
\end{problem}


\newpage
\begin{instructorNotes}

{\bf Main Goal:} Students use their knowledge of factors and multiples to attack problems.


{\bf Overall Picture:} This is way too much for one class period! Many of these problems used to be entire activities on their own. They are collected here together so that you can pick and choose (or let students pick and choose during class depending on their interest). Either start by highlighting some problems for students to start their investigation, or ask students to choose something to focus on.

\begin{itemize}
\item The first two problems are essentially basic applications of the meaning of factor and multiple, and we are combining this knowledge with some knowledge about prime numbers.
\item The problem about zeroes at the end should be translated as a problem about factors of 10. If students are stuck on this one, you might suggest they first ask themselves what kind of numbers have zeroes on the end, and what factor would ensure this happens.
\item The question about $\frac{1}{N}$ is also about factors of 10. If the denominator is a power of 10, the fraction has a terminating decimal. Otherwise, it repeats. This should connect to bundling in that a terminating decimal can be drawn with some choice of ``smallest object'', and that smallest object should be the power of 10 we need.
\item The problem about determining whether a number is prime should be approached with smaller examples first. Eventually, we hope students see that they only need to test prime numbers, and then only up to the square root of the number (because a larger factor is always paired with a smaller factor).
\item The question about remainders is essentially modular arithmetic in disguise. We suggest students play with several examples, draw pictures, and think about what is happening with the remainders.
\item The LCM and GCF question is leading students towards a general method for finding LCM and GCF with prime factorizations, but we want to emphasize what it means to be a factor or a multiple. If we are looking for the GCF, for instance, since we need to be a factor of both numbers we can only use primes which are common to both. To have the greatest such number, we include all primes common to both numbers. A similar argument can be made for the LCM.
\item The last question is a trick! There is no such number, but students need to read carefully and think about multiples to come to this conclusion.
\end{itemize}




{\bf Good Language:} Help students to be careful distinguishing factors from multiples. We have found this to be a challenge!



{\bf Suggested Timing:} Students should work on this activity for a full class period. Split the time up about half-and-half, using the first half for students to investigate in small groups, and the second half for each group to present either something they learned, something they tried which didn't work, or a partial solution to a problem.

\end{instructorNotes}


\end{document}