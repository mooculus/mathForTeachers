\documentclass[nooutcomes]{ximera}

\usepackage{gensymb}
\usepackage{tabularx}
\usepackage{mdframed}
\usepackage{pdfpages}
%\usepackage{chngcntr}

\let\problem\relax
\let\endproblem\relax

\newcommand{\property}[2]{#1#2}




\newtheoremstyle{SlantTheorem}{\topsep}{\fill}%%% space between body and thm
 {\slshape}                      %%% Thm body font
 {}                              %%% Indent amount (empty = no indent)
 {\bfseries\sffamily}            %%% Thm head font
 {}                              %%% Punctuation after thm head
 {3ex}                           %%% Space after thm head
 {\thmname{#1}\thmnumber{ #2}\thmnote{ \bfseries(#3)}} %%% Thm head spec
\theoremstyle{SlantTheorem}
\newtheorem{problem}{Problem}[]

%\counterwithin*{problem}{section}



%%%%%%%%%%%%%%%%%%%%%%%%%%%%Jenny's code%%%%%%%%%%%%%%%%%%%%

%%% Solution environment
%\newenvironment{solution}{
%\ifhandout\setbox0\vbox\bgroup\else
%\begin{trivlist}\item[\hskip \labelsep\small\itshape\bfseries Solution\hspace{2ex}]
%\par\noindent\upshape\small
%\fi}
%{\ifhandout\egroup\else
%\end{trivlist}
%\fi}
%
%
%%% instructorIntro environment
%\ifhandout
%\newenvironment{instructorIntro}[1][false]%
%{%
%\def\givenatend{\boolean{#1}}\ifthenelse{\boolean{#1}}{\begin{trivlist}\item}{\setbox0\vbox\bgroup}{}
%}
%{%
%\ifthenelse{\givenatend}{\end{trivlist}}{\egroup}{}
%}
%\else
%\newenvironment{instructorIntro}[1][false]%
%{%
%  \ifthenelse{\boolean{#1}}{\begin{trivlist}\item[\hskip \labelsep\bfseries Instructor Notes:\hspace{2ex}]}
%{\begin{trivlist}\item[\hskip \labelsep\bfseries Instructor Notes:\hspace{2ex}]}
%{}
%}
%% %% line at the bottom} 
%{\end{trivlist}\par\addvspace{.5ex}\nobreak\noindent\hung} 
%\fi
%
%


\let\instructorNotes\relax
\let\endinstructorNotes\relax
%%% instructorNotes environment
\ifhandout
\newenvironment{instructorNotes}[1][false]%
{%
\def\givenatend{\boolean{#1}}\ifthenelse{\boolean{#1}}{\begin{trivlist}\item}{\setbox0\vbox\bgroup}{}
}
{%
\ifthenelse{\givenatend}{\end{trivlist}}{\egroup}{}
}
\else
\newenvironment{instructorNotes}[1][false]%
{%
  \ifthenelse{\boolean{#1}}{\begin{trivlist}\item[\hskip \labelsep\bfseries {\Large Instructor Notes: \\} \hspace{\textwidth} ]}
{\begin{trivlist}\item[\hskip \labelsep\bfseries {\Large Instructor Notes: \\} \hspace{\textwidth} ]}
{}
}
{\end{trivlist}}
\fi


%% Suggested Timing
\newcommand{\timing}[1]{{\bf Suggested Timing: \hspace{2ex}} #1}




\hypersetup{
    colorlinks=true,       % false: boxed links; true: colored links
    linkcolor=blue,          % color of internal links (change box color with linkbordercolor)
    citecolor=green,        % color of links to bibliography
    filecolor=magenta,      % color of file links
    urlcolor=cyan           % color of external links
}


\title{Division Word Problems}
\author{Vic Ferdinand, Betsy McNeal, Jenny Sheldon}

\begin{document}
\begin{abstract} \end{abstract}
\maketitle



\begin{problem}
Describe similarities and differences between the following word problems by considering how a student who has never dealt with division before might attempt to solve these problems.

\begin{enumerate}
\item We have a total of 35 hard candies.  If there are 5 boxes with an equal number of candies in each box with all the candy accounted for, how many candies are in each box? (What happens if we start with 38 pieces of hard candy?)

\item We have a total of 35 hard candies.  If there are 5 candies in each box, how many boxes are there? (What happens if we start with 38 pieces of hard candy?)

\item We have a total of 38 gallons of milk to be put in 5 containers.  If each container holds the same amount of milk and all the milk is accounted for, how much milk will each container hold?

\item We have a total of 38 gallons of milk to be put in containers holding 5 gallons each.  If all the milk is used, how many containers were used?


\item  A rectangle has length 5 inches and area 35 square inches.  What is its width?


\item  A chart has 35 cells and 5 rows.  How many columns does it have?


\item John has a total of 35 outfits (shirt and pant combinations) he can wear.  If he has 5 shirts, how many pants does he have?

\item John has 5 times as many candies as Cindy.  If John has 35 candies. how many candies does Cindy have?
\item If John has 35 candies and Cindy has 5 candies, how many times as many candies does John have than Cindy?
\end{enumerate}

\end{problem}


\begin{problem}
What would happen if we replaced the numbers $35$ and $5$ in the previous problem with $510$ and $30$?  Would your strategies change?  Would your pictures change?  Try to work out at least one of the problems with these new numbers, and compare and contrast with the previous problem.
\end{problem}

\newpage

\begin{instructorNotes}
In our course, this is the first activity we use to introduce division as an operation.  As with the activities ``Adding It All Up'' (introducing addition and subtraction) and ``Multiplication Word Problems'' (introducing multiplication), this activity has students think about how young children who have not yet learned division would solve the problems using objects or drawings.

In our course, we follow extensive work on the meaning of numbers (whole numbers, fractions, decimals) without reference to operations with activities about the operations.  This is the second activity we do to introduce an operation (the first is ``Adding It All Up'', which introduces addition and subtraction and the second is ``Multiplication Word Problems'', which introduces multiplication). Dealing with operations on numbers, rather than only with the meanings, notation, and comparison of quantities is a big shift in our course.

Again, we try to focus students on the structure of the problems, spending significant time discussing what the problems have in common with one another to develop and identify the structure of multiplication.

After discussion of the worksheet and students' solutions, we describe the two models of division: {\em how many groups?} (also known as measurement division) and {\em how many in each group? } (also known as partitive division).

\begin{itemize}
	\item We spend significant time identifying the ``groups'' and the ``objects in each group'' in each story problem - relating to how each model (``{\em how many groups}'' and ``{\em how many per group}'') was solved in the activity.
	\item We also discuss the {\em repeated subtraction} solution process for each model.  For ``How many per group'' problems, we distribute one object to each of the groups and take inventory at each stage.  In the ``How many groups'' problems, we distribute a given amount to one group, take inventory, and repeat.  The physical nature of these descriptions can help some students tell the two types apart.  %In both problem types, we ask what we need to do after each subtraction (do we have enough to do it again?). The magic time when we don't have enough and the process ends occurs when we have a unique quotient and remainder:  $a = bq + r$ with $r < b$).  
	\item These problems give us an opportunity to pay attention to different types of appropriate remainders -- sometimes we report a whole number and remainder solution, sometimes we round up to the next whole number, sometimes we report a mixed number answer.  All of these types are represented in these problems. 
    \item To shorten the discussion, we sometimes do and discuss the first problems (a and b), run through the rest, and then go on to other activities.
    \item We don't typically have time to fully discuss the final question, but we include it so that we have the chance to continue talking about the fact that changing the specific numbers in a problem does not change the structure of the operation.  We typically connect here to the same idea as we saw it with multiplication, and again return to the idea when we discuss division with fractions.
\end{itemize}



{\bf Suggested Timing:} This will take the whole class period, about 15 minutes in groups, about 15 minutes presenting at the board, and about 20 minutes in discussion.
\end{instructorNotes}


\end{document}