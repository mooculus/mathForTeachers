\documentclass[nooutcomes]{ximera}

\usepackage{gensymb}
\usepackage{tabularx}
\usepackage{mdframed}
\usepackage{pdfpages}
%\usepackage{chngcntr}

\let\problem\relax
\let\endproblem\relax

\newcommand{\property}[2]{#1#2}




\newtheoremstyle{SlantTheorem}{\topsep}{\fill}%%% space between body and thm
 {\slshape}                      %%% Thm body font
 {}                              %%% Indent amount (empty = no indent)
 {\bfseries\sffamily}            %%% Thm head font
 {}                              %%% Punctuation after thm head
 {3ex}                           %%% Space after thm head
 {\thmname{#1}\thmnumber{ #2}\thmnote{ \bfseries(#3)}} %%% Thm head spec
\theoremstyle{SlantTheorem}
\newtheorem{problem}{Problem}[]

%\counterwithin*{problem}{section}



%%%%%%%%%%%%%%%%%%%%%%%%%%%%Jenny's code%%%%%%%%%%%%%%%%%%%%

%%% Solution environment
%\newenvironment{solution}{
%\ifhandout\setbox0\vbox\bgroup\else
%\begin{trivlist}\item[\hskip \labelsep\small\itshape\bfseries Solution\hspace{2ex}]
%\par\noindent\upshape\small
%\fi}
%{\ifhandout\egroup\else
%\end{trivlist}
%\fi}
%
%
%%% instructorIntro environment
%\ifhandout
%\newenvironment{instructorIntro}[1][false]%
%{%
%\def\givenatend{\boolean{#1}}\ifthenelse{\boolean{#1}}{\begin{trivlist}\item}{\setbox0\vbox\bgroup}{}
%}
%{%
%\ifthenelse{\givenatend}{\end{trivlist}}{\egroup}{}
%}
%\else
%\newenvironment{instructorIntro}[1][false]%
%{%
%  \ifthenelse{\boolean{#1}}{\begin{trivlist}\item[\hskip \labelsep\bfseries Instructor Notes:\hspace{2ex}]}
%{\begin{trivlist}\item[\hskip \labelsep\bfseries Instructor Notes:\hspace{2ex}]}
%{}
%}
%% %% line at the bottom} 
%{\end{trivlist}\par\addvspace{.5ex}\nobreak\noindent\hung} 
%\fi
%
%


\let\instructorNotes\relax
\let\endinstructorNotes\relax
%%% instructorNotes environment
\ifhandout
\newenvironment{instructorNotes}[1][false]%
{%
\def\givenatend{\boolean{#1}}\ifthenelse{\boolean{#1}}{\begin{trivlist}\item}{\setbox0\vbox\bgroup}{}
}
{%
\ifthenelse{\givenatend}{\end{trivlist}}{\egroup}{}
}
\else
\newenvironment{instructorNotes}[1][false]%
{%
  \ifthenelse{\boolean{#1}}{\begin{trivlist}\item[\hskip \labelsep\bfseries {\Large Instructor Notes: \\} \hspace{\textwidth} ]}
{\begin{trivlist}\item[\hskip \labelsep\bfseries {\Large Instructor Notes: \\} \hspace{\textwidth} ]}
{}
}
{\end{trivlist}}
\fi


%% Suggested Timing
\newcommand{\timing}[1]{{\bf Suggested Timing: \hspace{2ex}} #1}




\hypersetup{
    colorlinks=true,       % false: boxed links; true: colored links
    linkcolor=blue,          % color of internal links (change box color with linkbordercolor)
    citecolor=green,        % color of links to bibliography
    filecolor=magenta,      % color of file links
    urlcolor=cyan           % color of external links
}


\title{GCF and LCM Problems}
\author{Vic Ferdinand, Betsy McNeal, Jenny Sheldon}

\begin{document}
\begin{abstract} \end{abstract}
\maketitle




\begin{problem} \label{GCFLCM1}

Let
$a = 2^{17}\times 3^7 \times 5^9 \times 11^{22}$ and $b =  2^{4}\times 3^{12} \times 5^{24} \times 7^{3}$.

Compute GCF$(a,b)$ and LCM$(a,b)$.

\end{problem}

\begin{problem}\label{GCFLCM2}
Make an argument for how you can use the prime factorizations of two
numbers to find the LCM and GCF of those numbers.
\end{problem}
\vfill
\newpage

\begin{problem} \label{GCFLCM3}
In the following problems, you are given three pieces of
information. In each case, use the given information to find the value
of $a$.

\begin{enumerate}
\item 
\begin{align*}
  \text{GCF}(a,b) &= 2 \times 3,\\ \text{LCM}(a,b) &= 2^2 \times
  3^3 \times 5,\\ \text{and } b &= 2^2 \times 3 \times 5.
\end{align*}

\vfill
\item  
\begin{align*}
  \text{GCF}(a,b) &= 2^2 \times 7 \times 11,\\
  \text{LCM}(a,b) &= 2^5 \times 3^2 \times 5 \times 7^3 \times 11^2,\\
  \text{and }   b &= 2^5 \times 3^2 \times 5 \times 7 \times 11.
\end{align*}
\vfill
\end{enumerate}
\end{problem}

\newpage

\begin{instructorNotes}
These problems are intended to help to solidify students' understanding of GCF and LCM through the use of prime factorization.  To encourage reasoning, the numbers are generally too large for a calculator to handle.  Students should attempt to use the definitions of these terms to solve the problems, rather than just computing.

\begin{itemize}
    \item We usually include a review of exponential notation, as we have found many of our students to struggle with this concept.
	\item For Problems \ref{GCFLCM1} - \ref{GCFLCM2}, we encourage students to think about GCF by thinking about the ``F'' first, then the ``C'', and finally the ``G''.  Similarly with LCM.
	\item We also include reminders of the definition of factors and multiples as we discuss these problems.
	\item Related to Problem \ref{GCFLCM3}, we often give a homework problem designed to help students develop rules for when the product of two numbers is the same as their LCM, and when the LCM should be smaller. %you might have them notice that, with 16 and 28, multiplying the two numbers turns out to be the same product as their LCM and GCF.  Ask them why in light of what they did in Problems \ref{GCFLCM1} - \ref{GCFLCM2} (catching all the powers from both numbers since LCM took some and GCF took the rest).  Ask them why this is helpful (If we know GCF, we automatically can calculate LCM.  And vice versa).
\end{itemize}

\timing{We typically use a full class period for this activity.  We give students 5-10 minutes to think about the first two problems, then discuss for about 20 minutes.  Then we give students 5-10 minutes to think about the third problem, and use the rest of the time to discuss.}
\end{instructorNotes}

\end{document}