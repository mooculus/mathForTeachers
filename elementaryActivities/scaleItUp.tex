\documentclass[nooutcomes]{ximera}

\graphicspath{
  {./}
  {graphics/}
  {../graphics/}
}

\usepackage{chngcntr}

\let\question\relax
\let\endquestion\relax




\newtheoremstyle{SlantTheorem}{\topsep}{\fill}%%% space between body and thm
%\newtheoremstyle{SlantTheorem}{\topsep}{\topsep}%%% space between body and thm
 {\slshape}                      %%% Thm body font
 {}                              %%% Indent amount (empty = no indent)
 {\bfseries\sffamily}            %%% Thm head font
 {}                              %%% Punctuation after thm head
 {3ex}                           %%% Space after thm head
 {\thmname{#1}\thmnumber{ #2}\thmnote{ \bfseries(#3)}}%%% Thm head spec
\theoremstyle{SlantTheorem}
\newtheorem{question}{Question}
\counterwithin*{question}{section}



\let\instructorNotes\relax
\let\endinstructorNotes\relax
%%% instructorNotes environment
\ifhandout
\newenvironment{instructorNotes}[1][false]%
{%
\def\givenatend{\boolean{#1}}\ifthenelse{\boolean{#1}}{\begin{trivlist}\item}{\setbox0\vbox\bgroup}{}
}
{%
\ifthenelse{\givenatend}{\end{trivlist}}{\egroup}{}
}
\else
\newenvironment{instructorNotes}[1][false]%
{%
  \ifthenelse{\boolean{#1}}{\begin{trivlist}\item[\hskip \labelsep\bfseries {\Large Instructor Notes: \\} \hspace{\textwidth} ]}
{\begin{trivlist}\item[\hskip \labelsep\bfseries {\Large Instructor Notes: \\} \hspace{\textwidth} ]}
{}
}
{\end{trivlist}}
\fi


%% Suggested Timing
\newcommand{\timing}[1]{{\bf Suggested Timing: \hspace{2ex}} #1}

\title{Scale It Up}
\author{Vic Ferdinand, Betsy McNeal, Jenny Sheldon} 


\begin{document}


\begin{abstract}
Recall: When an object is scaled, the new object (which is similar to the original) is the same shape as the original, but a different size.  In this activity, we explore what happens to area and volume when we scale.
\end{abstract}

\maketitle

\begin{instructorIntro}
The goal of this activity is to help students discover what happens to area and volume when we scale rectangles and rectangular prisms.  Scaling of other shapes will be covered in the next activity.  Your overall goal should be to develop the notion that lengths are scaled by $k$ (which we might call a ``linear scale factor'', adding the term ``linear'' to mean the exact same thing as we meant by scale factor in the similarity unit, while areas are scaled by $k^2$ (which we might call an ``area scale factor'') and volumes are scaled by $k^3$ (which we might call a ``volume scale factor'').  You should introduce these terms as it seems appropriate during your discussion.

Note: this should be the first time we are scaling anything other than length, so these concepts should be new for the students.  Before the activity, they would likely guess the area and volume scale factors incorrectly!
\end{instructorIntro}

\begin{problem}
The rectangle below has dimensions $L$ and $W$. If we scale it by a scale factor of 3, what will the new dimensions of the rectangle be? Draw the new rectangle in the given space.\\

\tikz{ \draw (0 cm,0) -- (2 cm,0); \draw (2 cm, 0) -- (2 cm, 1 cm); \draw (2 cm, 1 cm) -- (0 cm, 1 cm);\draw (0 cm, 1 cm) -- (0 cm, 0);} \hspace{4cm}
\vskip 2.5in

\begin{enumerate}
    \item What is the ratio of the new rectangle's length to the original  length?
    \item What is the area of the new rectangle?  How can we see this in the picture?
    \item What is the ratio of the new rectangle's area to the original area?
\end{enumerate}

\begin{instructorNotes}
Students should be familiar with the concept of a scale factor from our work on similarity, and should hopefully have no trouble with the first part of this question.

On (a), you might remind the students that our scale factor is also scaling all other lengths in this figure - so the diagonal is also scaled by a factor of three, etcetera.  This should have been covered with the original similarity unit.

Students may have trouble with the concept of the area of the new rectangle, since they are not given the actual dimensions of the original rectangle.  The idea is to see that there are 9 copies of the original rectangle, so there are 9 copies of its area.
\end{instructorNotes}
\end{problem}

\begin{problem}
If we now scale the \underline{original} rectangle by the linear scale factor of $\frac{11}{3}$, what will the dimensions of this third rectangle be?  Predict the new area.  Now, draw the third rectangle and answer the same questions as above, comparing this rectangle with the original. (Can you build on the picture you've already drawn?)

\begin{instructorNotes}
This is a more difficult version of the first problem.  Students should recognize that they are doing the same thing as the first problem - that is, making $\frac{11}{3}$ copies of the rectangle along each of the width and length, but they will likely struggle with the fractional copies - especially in the overlap.  The resulting picture should look quite a bit like some multiplication or area diagrams we discussed (particularly with multiplying decimals), except the ``units'' here are rectangles, not squares.  You might connect this with some of the irregular area units we've discussed.
\end{instructorNotes}

\end{problem}

%\pagebreak

\begin{problem}
What will happen to length and area if we scale a rectangle of dimensions $2 cm$ by $5 cm$ by a linear scale factor of $k$? What will happen to length and area if we scale a rectangle of dimensions $L$ and $W$ by a linear scale factor of $k$?

\begin{instructorNotes}
This is a general version of the first two problems, and should (hopefully) be pretty quick.  You might ask as a transition whether there was anything special about the numbers we've used so far, or whether we could use any kind of number we like.  (Decimals?  Negatives?  Zero?)
\end{instructorNotes}
\end{problem}

\begin{problem}
Justify your observations in the previous question in three different ways.  By the end of this class period, you should have an algebraic explanation (using the formula for area), a geometric explanation (using your picture in the first problem), and an explanation dealing with scaling an individual unit of area.
\begin{instructorNotes}
This should be one of the main takeaways of the activity, so don't be afraid to spend some time here.  The notion of scaling a unit of area will be the most useful for moving forward to discussing general shapes.  
\end{instructorNotes}
\end{problem}

\begin{problem}
Draw a cube whose side length is 3cm, and a rectangular prism whose dimensions are 2cm by 5cm by 3cm.  Scale each of these prisms using several different linear scale factors of your choosing, and record the ratio of the original volume to the new volume.  What do you notice?
\vskip 2.3in
\begin{enumerate}
    \item What is the ratio of the new prism's length to the original  length?
    \item What is the volume of the new prism?  How can we see this in the picture?
    \item What is the ratio of the new prism's volume to the original volume?
\end{enumerate}
\begin{instructorNotes}
Students will likely struggle with their drawings in this situation.  You could bring unit cubes to class for them to form the various solids, or you may find a video you can play or post on your LMS to show the smaller prism ``fitting'' in the larger prism.  

If you haven't introduced the new terminology for area and volume scale factors by this point, it's good to do so here or in the next problem.
\end{instructorNotes}
\end{problem}

\begin{problem}
Justify your observations in the previous question by scaling a unit of volume.
\begin{instructorNotes}
This should correlate strongly with our work on area.  You might discuss each of these methods again if you have time, but the notion of scaling a unit of volume is likely the most useful moving forward.
\end{instructorNotes}
\end{problem}

\begin{problem}
When we scaled the cube and rectangular prism above, what happened to their surface area?  Justify your claim!

\begin{instructorNotes}
This problem is tough for students to see - since the area has moved out of the plane and into 3D space.  If you don't have time for this problem, you can cover it later in another activity.
\end{instructorNotes}
\end{problem}




\end{document}