\documentclass{ximera}

\graphicspath{
  {./}
  {graphics/}
  {../graphics/}
}

\usepackage{chngcntr}

\let\question\relax
\let\endquestion\relax




\newtheoremstyle{SlantTheorem}{\topsep}{\fill}%%% space between body and thm
%\newtheoremstyle{SlantTheorem}{\topsep}{\topsep}%%% space between body and thm
 {\slshape}                      %%% Thm body font
 {}                              %%% Indent amount (empty = no indent)
 {\bfseries\sffamily}            %%% Thm head font
 {}                              %%% Punctuation after thm head
 {3ex}                           %%% Space after thm head
 {\thmname{#1}\thmnumber{ #2}\thmnote{ \bfseries(#3)}}%%% Thm head spec
\theoremstyle{SlantTheorem}
\newtheorem{question}{Question}
\counterwithin*{question}{section}



\let\instructorNotes\relax
\let\endinstructorNotes\relax
%%% instructorNotes environment
\ifhandout
\newenvironment{instructorNotes}[1][false]%
{%
\def\givenatend{\boolean{#1}}\ifthenelse{\boolean{#1}}{\begin{trivlist}\item}{\setbox0\vbox\bgroup}{}
}
{%
\ifthenelse{\givenatend}{\end{trivlist}}{\egroup}{}
}
\else
\newenvironment{instructorNotes}[1][false]%
{%
  \ifthenelse{\boolean{#1}}{\begin{trivlist}\item[\hskip \labelsep\bfseries {\Large Instructor Notes: \\} \hspace{\textwidth} ]}
{\begin{trivlist}\item[\hskip \labelsep\bfseries {\Large Instructor Notes: \\} \hspace{\textwidth} ]}
{}
}
{\end{trivlist}}
\fi


%% Suggested Timing
\newcommand{\timing}[1]{{\bf Suggested Timing: \hspace{2ex}} #1}
\title{Gertrude the Gumchewer}
\author{Vic Ferdinand, Betsy McNeal, Jenny Sheldon}

\begin{document}
\begin{abstract}
\end{abstract}
\maketitle

\begin{instructorIntro}
Goals for the activity:
\begin{enumerate}
\item students think about how to solve a new problem, practice problem-solving process
\item  students recognize a pattern and then represent the pattern in different ways appropriate to their goal
\item  introduce the idea of an arithmetic sequence
\item discuss multiple forms of representing the same relationship 
\end{enumerate}

Teaching notes:
\begin{enumerate}
\item the issue of how to represent the first day's amount will likely arise in the discussion of the different algebraic expressions:  should the initial amount be labeled as occurring on Day 1 or on Day 0?
\item it would be appropriate to distinguish between explicit and recursive forms, i.e., which representations or solution methods rely on knowing the previous day's amount, which allow us to jump to any day?
\item discussion of usefulness of each.
\end{enumerate}
\end{instructorIntro}

\begin{problem}
Gertrude the Gumchewer has an addiction to \textit{Xtra Sugarloaded
  Gum}, and it's getting worse.  Each day, she goes to her always
stocked storage vault and grabs gum to chew.  At the beginning of her
habit, she chewed three pieces and then, each day, she chews 8 more
pieces than she chewed the day before to satisfy her ever-increasing
cravings.
\begin{enumerate}
\item How many pieces will she chew on the $10$th day of her habit?
\item How many pieces will she chew on the $793$rd day of her habit? How do you know you are right?
\item How many pieces will she chew on the $k$th day of her habit?
%\item How many pieces will she chew over the course of the first $793$
%  days of her habit?
\end{enumerate}
\end{problem}


%consider switching the new problem with the generalization below.  We don't want the extra problem to be just about substitution!

\begin{problem}
Assume now that Gertrude, at the beginning of her habit, chewed $m$
pieces of gum and then, each day, she chews $n$ more pieces than she
chewed the day before to satisfy her ever-increasing cravings.  How many pieces will she chew on the $kth$
  day of her habit? Explain your formula and how you know it will work for any $m$, $n$ and $k$.  
\end{problem}

\begin{problem}
Use the method you developed in (1) and (2) to find how many days she would have taken to get to 180 pieces in a day if the following shows her pattern on the first three days:
\[
19, 26, 33, \dots , 180
\]
Give another story problem that is represented by this sum.  Make it as different as you can.
\end{problem}





\end{document}