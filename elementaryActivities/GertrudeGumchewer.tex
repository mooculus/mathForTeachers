\documentclass[nooutcomes]{ximera}
\usepackage{gensymb}
\usepackage{tabularx}
\usepackage{mdframed}
\usepackage{pdfpages}
%\usepackage{chngcntr}

\let\problem\relax
\let\endproblem\relax

\newcommand{\property}[2]{#1#2}




\newtheoremstyle{SlantTheorem}{\topsep}{\fill}%%% space between body and thm
 {\slshape}                      %%% Thm body font
 {}                              %%% Indent amount (empty = no indent)
 {\bfseries\sffamily}            %%% Thm head font
 {}                              %%% Punctuation after thm head
 {3ex}                           %%% Space after thm head
 {\thmname{#1}\thmnumber{ #2}\thmnote{ \bfseries(#3)}} %%% Thm head spec
\theoremstyle{SlantTheorem}
\newtheorem{problem}{Problem}[]

%\counterwithin*{problem}{section}



%%%%%%%%%%%%%%%%%%%%%%%%%%%%Jenny's code%%%%%%%%%%%%%%%%%%%%

%%% Solution environment
%\newenvironment{solution}{
%\ifhandout\setbox0\vbox\bgroup\else
%\begin{trivlist}\item[\hskip \labelsep\small\itshape\bfseries Solution\hspace{2ex}]
%\par\noindent\upshape\small
%\fi}
%{\ifhandout\egroup\else
%\end{trivlist}
%\fi}
%
%
%%% instructorIntro environment
%\ifhandout
%\newenvironment{instructorIntro}[1][false]%
%{%
%\def\givenatend{\boolean{#1}}\ifthenelse{\boolean{#1}}{\begin{trivlist}\item}{\setbox0\vbox\bgroup}{}
%}
%{%
%\ifthenelse{\givenatend}{\end{trivlist}}{\egroup}{}
%}
%\else
%\newenvironment{instructorIntro}[1][false]%
%{%
%  \ifthenelse{\boolean{#1}}{\begin{trivlist}\item[\hskip \labelsep\bfseries Instructor Notes:\hspace{2ex}]}
%{\begin{trivlist}\item[\hskip \labelsep\bfseries Instructor Notes:\hspace{2ex}]}
%{}
%}
%% %% line at the bottom} 
%{\end{trivlist}\par\addvspace{.5ex}\nobreak\noindent\hung} 
%\fi
%
%


\let\instructorNotes\relax
\let\endinstructorNotes\relax
%%% instructorNotes environment
\ifhandout
\newenvironment{instructorNotes}[1][false]%
{%
\def\givenatend{\boolean{#1}}\ifthenelse{\boolean{#1}}{\begin{trivlist}\item}{\setbox0\vbox\bgroup}{}
}
{%
\ifthenelse{\givenatend}{\end{trivlist}}{\egroup}{}
}
\else
\newenvironment{instructorNotes}[1][false]%
{%
  \ifthenelse{\boolean{#1}}{\begin{trivlist}\item[\hskip \labelsep\bfseries {\Large Instructor Notes: \\} \hspace{\textwidth} ]}
{\begin{trivlist}\item[\hskip \labelsep\bfseries {\Large Instructor Notes: \\} \hspace{\textwidth} ]}
{}
}
{\end{trivlist}}
\fi


%% Suggested Timing
\newcommand{\timing}[1]{{\bf Suggested Timing: \hspace{2ex}} #1}




\hypersetup{
    colorlinks=true,       % false: boxed links; true: colored links
    linkcolor=blue,          % color of internal links (change box color with linkbordercolor)
    citecolor=green,        % color of links to bibliography
    filecolor=magenta,      % color of file links
    urlcolor=cyan           % color of external links
}
\title{Gertrude the Gumchewer}
\author{Vic Ferdinand, Betsy McNeal, Jenny Sheldon}

\begin{document}
\begin{abstract}
\end{abstract}
\maketitle



\begin{problem}
Gertrude the Gumchewer has an addiction to \textit{Xtra Sugarloaded
  Gum}, and it's getting worse.  Each day, she goes to her always
stocked storage vault and grabs gum to chew.  At the beginning of her
habit, she chewed three pieces and then, each day, she chews 8 more
pieces than she chewed the day before to satisfy her ever-increasing
cravings.
\begin{enumerate}
\item How many pieces will she chew on the $10$th day of her habit?
\item How many pieces will she chew on the $793$rd day of her habit? How do you know you are right?
\item How many pieces will she chew on the $k$th day of her habit?
%\item How many pieces will she chew over the course of the first $793$
%  days of her habit?
\end{enumerate}
\end{problem}


%consider switching the new problem with the generalization below.  We don't want the extra problem to be just about substitution!

\begin{problem}
Assume now that Gertrude, at the beginning of her habit, chewed $m$
pieces of gum and then, each day, she chews $n$ more pieces than she
chewed the day before to satisfy her ever-increasing cravings.  How many pieces will she chew on the $kth$
  day of her habit? Explain your formula and how you know it will work for any $m$, $n$ and $k$.  
\end{problem}

\begin{problem}
Use the method you developed in (1) and (2) to find how many days she would have taken to get to 180 pieces in a day if the following shows her pattern on the first three days:
\[
19, 26, 33, \dots , 180
\]
Give another story problem that is represented by this sum.  Make it as different as you can.
\end{problem}

\newpage

\begin{instructorNotes}
Here are our goals for this activity.
\begin{enumerate}
\item Students think about how to solve a new problem, and practice problem-solving process.
\item  Students recognize a pattern and then represent the pattern in different ways appropriate to their goal.
\item  To introduce the idea of an arithmetic sequence.
\item To discuss multiple forms of representing the same relationship. 
\end{enumerate}

This is the first activity we use in the second semester.  Our first semester course covers numbers and operations, as well as ratios and a little bit of algebra (mainly what is an equation and how do equations represent various situations).  This activity serves us well for the beginning of the semester because we begin with problem solving in an environment which should be familiar to the students, but with an unfamiliar problem.  Algebraic sequences can also serve as a connection between the algebra we have done and the geometry we plan to do in the second semester if we consider both equations as well as graphs.  We follow this activity with ``I Walk the Line'', which is also about linear relationships, and then more about graphs and how to interpret them. 

\begin{itemize}
\item The issue of how to represent the first day's amount usually arises in the discussion of the different algebraic expressions:  should the initial amount be labeled as occurring on Day 1 or on Day 0? We usually let students decide the answer as a whole class in our discussion.
\item We use both explicit and recursive forms and discuss when each form is most appropriate or useful.  By the end of the activity, we expect students to be able to solve this problem using a table, a recursive relationship, and and explicit equation.
\item After working through these problems, we introduce the idea of an arithmetic sequence as a way of generalizing this type of problem.  This terminology is not brought up in the activity, but we use it in subsequent discussions.
\end{itemize}

\timing{Since there are many different solution methods, this activity takes us about a class period and a half.  We give students about 5-10 minutes to get started with the first problem, and then discuss.  Our discussion can take 30-45 minutes here, since we want to explore each of the different solution methods fully.  On the second day, we give students 5-10 minutes to think about whichever problems we haven't discussed on the first day, and then spend 20 minutes making sure the methods are fully understood and the students understand the general idea of an arithmetic sequence.}
\end{instructorNotes}



\end{document}