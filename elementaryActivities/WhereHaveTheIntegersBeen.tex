\documentclass{ximera}


\graphicspath{
  {./}
  {graphics/}
  {../graphics/}
}

\usepackage{chngcntr}

\let\question\relax
\let\endquestion\relax




\newtheoremstyle{SlantTheorem}{\topsep}{\fill}%%% space between body and thm
%\newtheoremstyle{SlantTheorem}{\topsep}{\topsep}%%% space between body and thm
 {\slshape}                      %%% Thm body font
 {}                              %%% Indent amount (empty = no indent)
 {\bfseries\sffamily}            %%% Thm head font
 {}                              %%% Punctuation after thm head
 {3ex}                           %%% Space after thm head
 {\thmname{#1}\thmnumber{ #2}\thmnote{ \bfseries(#3)}}%%% Thm head spec
\theoremstyle{SlantTheorem}
\newtheorem{question}{Question}
\counterwithin*{question}{section}



\let\instructorNotes\relax
\let\endinstructorNotes\relax
%%% instructorNotes environment
\ifhandout
\newenvironment{instructorNotes}[1][false]%
{%
\def\givenatend{\boolean{#1}}\ifthenelse{\boolean{#1}}{\begin{trivlist}\item}{\setbox0\vbox\bgroup}{}
}
{%
\ifthenelse{\givenatend}{\end{trivlist}}{\egroup}{}
}
\else
\newenvironment{instructorNotes}[1][false]%
{%
  \ifthenelse{\boolean{#1}}{\begin{trivlist}\item[\hskip \labelsep\bfseries {\Large Instructor Notes: \\} \hspace{\textwidth} ]}
{\begin{trivlist}\item[\hskip \labelsep\bfseries {\Large Instructor Notes: \\} \hspace{\textwidth} ]}
{}
}
{\end{trivlist}}
\fi


%% Suggested Timing
\newcommand{\timing}[1]{{\bf Suggested Timing: \hspace{2ex}} #1}


\title{Where Have the Integers Been?}
\author{Vic Ferdinand, Betsy McNeal, Jenny Sheldon}

\begin{document}
\begin{abstract}
    We begin to model integers and their operations.
\end{abstract}
\maketitle

\begin{instructorIntro}
This activity is the first of a three-day segment where we talk about operations with negative numbers.  There is an accompanying reading on integers that lays out some of the more complicated framework.

On this first day, we first make sure we know what an integer is, and give some examples of real-life contexts in which negative numbers make sense.  Having multiple contexts allows for some flexibility in discussion later, but we will focus on three contexts: checks and bills (or the shopkeeper's method or ``Mail Time''), number lines, and to a lesser extent chip models.

This first day should be mostly for experimenting.  Feel free to discuss lots of different ideas, and don't be afraid to point out places where the ideas are good, and places where they have some issues.  Every context will have issues, and so it can be nice to say something about how we've chosen our three favorite contexts because they have the fewest problems (in our opinion).  By the end of this first day, each of the three methods mentioned above should have been introduced (with the possible exception of the chips, if you are skipping this).

{\bf Suggested Timing:} This activity should take a full class period.  Don't worry if there are lingering issues at the end of the day, as there will be more time later.
\end{instructorIntro}

\begin{question}
What is an integer?  How do integers compare with whole numbers?  Where do integers show up in real-life situations?  Give as many examples as you can think of -- we'll want a big list.

%\begin{instructorNotes}
%I'm thinking about giving them about 3 minutes to think of as many contexts as they can, then writing as many on the board as they come up with.  Maybe I also throw in red/black chips and checks and bills if similar ideas don't come up on their own.
%\end{instructorNotes}
\end{question}

\begin{question}
For each of the integers below, use at least two of your real-life situations to write a sentence which uses that number in a sensible context.  Then, draw a picture which could represent your integer.
\begin{enumerate}
    \item $3$
    \item $-3$
    \item $-1.85$
    \item $-62$
\end{enumerate}

%\begin{instructorNotes}
%I'm thinking we discuss only two of these parts, and leave the other two for practice.  Probably (b) and (d) would be my choice.  I'm hoping with the pictures they come up with some sensible way to represent negative numbers which is equivalent to the red/black chips but we will see how it goes.
%\end{instructorNotes}
\end{question}

\begin{question}
Draw a number line which includes all of the numbers in the previous question.  How did you decide where to locate the negative numbers?
%\begin{instructorNotes}
%Unless I have more than half a class period, I might skip this question entirely.  This will be good for them to use for thinking, and I think we will probably have time to discuss number lines with subtraction on Day 2.  But I stuck it in here to give the quick thinkers something to do.
%\end{instructorNotes}
\end{question}

\begin{question}
For each of the addition problems below, use at least two of your real-life situations to write a story problem which would be answered by that expression.  (Be sure to include a question!)  Then, draw a picture explaining what a child learning this concept might do to solve the problem.
\begin{enumerate}
    \item $2+6$
    \item $(-3) + 5$
    \item $5 + (-8)$
    \item $(-1) + (-14)$
\end{enumerate}

%\begin{instructorNotes}
%This is where I plan to spend most of the time.  Again, I will probably only discuss 2 of these after giving them about 5 minutes to think about it.  I might even divide up the class into four groups, one for each part, and see how much discussion we can get out of that. I wanted this problem to look a lot like some of the others we've done when introducing new operations, and also to give the students more practice writing story problems, since we've been focusing on that this semester.
%\end{instructorNotes}
\end{question}

\begin{question}
Look back again at each of the addition problems you did in the previous problem.  Does the sign of the answer make sense from your story?  Could you model the addition problem using a number line?

%\begin{instructorNotes}
%I'm not planning to get to this either.  After they talk about number lines with subtraction, I'm hoping this will be easy practice.
%\end{instructorNotes}
\end{question}









\end{document}