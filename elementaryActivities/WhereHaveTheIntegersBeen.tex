\documentclass{ximera}

\usepackage{gensymb}
\usepackage{tabularx}
\usepackage{mdframed}
\usepackage{pdfpages}
%\usepackage{chngcntr}

\let\problem\relax
\let\endproblem\relax

\newcommand{\property}[2]{#1#2}




\newtheoremstyle{SlantTheorem}{\topsep}{\fill}%%% space between body and thm
 {\slshape}                      %%% Thm body font
 {}                              %%% Indent amount (empty = no indent)
 {\bfseries\sffamily}            %%% Thm head font
 {}                              %%% Punctuation after thm head
 {3ex}                           %%% Space after thm head
 {\thmname{#1}\thmnumber{ #2}\thmnote{ \bfseries(#3)}} %%% Thm head spec
\theoremstyle{SlantTheorem}
\newtheorem{problem}{Problem}[]

%\counterwithin*{problem}{section}



%%%%%%%%%%%%%%%%%%%%%%%%%%%%Jenny's code%%%%%%%%%%%%%%%%%%%%

%%% Solution environment
%\newenvironment{solution}{
%\ifhandout\setbox0\vbox\bgroup\else
%\begin{trivlist}\item[\hskip \labelsep\small\itshape\bfseries Solution\hspace{2ex}]
%\par\noindent\upshape\small
%\fi}
%{\ifhandout\egroup\else
%\end{trivlist}
%\fi}
%
%
%%% instructorIntro environment
%\ifhandout
%\newenvironment{instructorIntro}[1][false]%
%{%
%\def\givenatend{\boolean{#1}}\ifthenelse{\boolean{#1}}{\begin{trivlist}\item}{\setbox0\vbox\bgroup}{}
%}
%{%
%\ifthenelse{\givenatend}{\end{trivlist}}{\egroup}{}
%}
%\else
%\newenvironment{instructorIntro}[1][false]%
%{%
%  \ifthenelse{\boolean{#1}}{\begin{trivlist}\item[\hskip \labelsep\bfseries Instructor Notes:\hspace{2ex}]}
%{\begin{trivlist}\item[\hskip \labelsep\bfseries Instructor Notes:\hspace{2ex}]}
%{}
%}
%% %% line at the bottom} 
%{\end{trivlist}\par\addvspace{.5ex}\nobreak\noindent\hung} 
%\fi
%
%


\let\instructorNotes\relax
\let\endinstructorNotes\relax
%%% instructorNotes environment
\ifhandout
\newenvironment{instructorNotes}[1][false]%
{%
\def\givenatend{\boolean{#1}}\ifthenelse{\boolean{#1}}{\begin{trivlist}\item}{\setbox0\vbox\bgroup}{}
}
{%
\ifthenelse{\givenatend}{\end{trivlist}}{\egroup}{}
}
\else
\newenvironment{instructorNotes}[1][false]%
{%
  \ifthenelse{\boolean{#1}}{\begin{trivlist}\item[\hskip \labelsep\bfseries {\Large Instructor Notes: \\} \hspace{\textwidth} ]}
{\begin{trivlist}\item[\hskip \labelsep\bfseries {\Large Instructor Notes: \\} \hspace{\textwidth} ]}
{}
}
{\end{trivlist}}
\fi


%% Suggested Timing
\newcommand{\timing}[1]{{\bf Suggested Timing: \hspace{2ex}} #1}




\hypersetup{
    colorlinks=true,       % false: boxed links; true: colored links
    linkcolor=blue,          % color of internal links (change box color with linkbordercolor)
    citecolor=green,        % color of links to bibliography
    filecolor=magenta,      % color of file links
    urlcolor=cyan           % color of external links
}


\title{Where Have the Integers Been?}
\author{Vic Ferdinand, Betsy McNeal, Jenny Sheldon}

\begin{document}
\begin{abstract}
    We begin to model integers and their operations.
\end{abstract}
\maketitle



\begin{problem}
What is an integer?  How do integers compare with whole numbers?  Where do integers show up in real-life situations?  Give as many examples as you can think of -- we'll want a big list.
\end{problem}

\begin{problem}
For each of the integers below, use at least two of your real-life situations to write a sentence which uses that number in a sensible context.  Then, draw a picture which could represent your integer.
\begin{enumerate}
    \item $3$
    \item $-3$
    \item $-1.85$
    \item $-62$
\end{enumerate}

\end{problem}

\begin{problem}
Draw a number line which includes all of the numbers in the previous question.  How did you decide where to locate the negative numbers?
\end{problem}

\begin{problem}
For each of the addition problems below, use at least two of your real-life situations to write a story problem which would be answered by that expression.  (Be sure to include a question!)  Then, draw a picture explaining what a child learning this concept might do to solve the problem.
\begin{enumerate}
    \item $2+6$
    \item $(-3) + 5$
    \item $5 + (-8)$
    \item $(-1) + (-14)$
\end{enumerate}

\end{problem}

\begin{problem}
Look back again at each of the addition problems you did in the previous problem.  Does the sign of the answer make sense from your story?  Could you model the addition problem using a number line?

\end{problem}


\newpage
\begin{instructorNotes}
This activity is the first of three that investigate operations with negative numbers.  There is an accompanying reading on integers that lays out some of the more complicated framework.

In this first activity, we first make sure we know what an integer is, and give some examples of real-life contexts in which negative numbers make sense.  Having multiple contexts allows for some flexibility in discussion later, but we will focus on three contexts: checks and bills (or the shopkeeper's method or ``Mail Time''), number lines, and to a lesser extent chip models.

This first activity should be mostly student experimentation.  Feel free to discuss lots of different ideas, and don't be afraid to point out places where the ideas are good, and places where they have some issues.  Every context will have issues, and so it can be nice to say something about how we've chosen our three favorite contexts because they have the fewest problems (in our opinion).  By the end of this first activity, each of the three models mentioned above should have been introduced (with the possible exception of the chips, if you are skipping this).

{\bf Suggested Timing:} This activity should take a full class period.  Don't worry if there are lingering issues at the end of the day, as there will be more time later.
\end{instructorNotes}


%some comment about why we do it this way (not just distributive property to prove)
%link to the reading?
%why three contexts?
%mathematics as about reasoning, not just lands on your plate just done (connex to historical/reading); develp richness along lines of kid thinking
%every context fails somewhere





\end{document}