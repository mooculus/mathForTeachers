\documentclass{ximera}

\usepackage{gensymb}
\usepackage{tabularx}
\usepackage{mdframed}
\usepackage{pdfpages}
%\usepackage{chngcntr}

\let\problem\relax
\let\endproblem\relax

\newcommand{\property}[2]{#1#2}




\newtheoremstyle{SlantTheorem}{\topsep}{\fill}%%% space between body and thm
 {\slshape}                      %%% Thm body font
 {}                              %%% Indent amount (empty = no indent)
 {\bfseries\sffamily}            %%% Thm head font
 {}                              %%% Punctuation after thm head
 {3ex}                           %%% Space after thm head
 {\thmname{#1}\thmnumber{ #2}\thmnote{ \bfseries(#3)}} %%% Thm head spec
\theoremstyle{SlantTheorem}
\newtheorem{problem}{Problem}[]

%\counterwithin*{problem}{section}



%%%%%%%%%%%%%%%%%%%%%%%%%%%%Jenny's code%%%%%%%%%%%%%%%%%%%%

%%% Solution environment
%\newenvironment{solution}{
%\ifhandout\setbox0\vbox\bgroup\else
%\begin{trivlist}\item[\hskip \labelsep\small\itshape\bfseries Solution\hspace{2ex}]
%\par\noindent\upshape\small
%\fi}
%{\ifhandout\egroup\else
%\end{trivlist}
%\fi}
%
%
%%% instructorIntro environment
%\ifhandout
%\newenvironment{instructorIntro}[1][false]%
%{%
%\def\givenatend{\boolean{#1}}\ifthenelse{\boolean{#1}}{\begin{trivlist}\item}{\setbox0\vbox\bgroup}{}
%}
%{%
%\ifthenelse{\givenatend}{\end{trivlist}}{\egroup}{}
%}
%\else
%\newenvironment{instructorIntro}[1][false]%
%{%
%  \ifthenelse{\boolean{#1}}{\begin{trivlist}\item[\hskip \labelsep\bfseries Instructor Notes:\hspace{2ex}]}
%{\begin{trivlist}\item[\hskip \labelsep\bfseries Instructor Notes:\hspace{2ex}]}
%{}
%}
%% %% line at the bottom} 
%{\end{trivlist}\par\addvspace{.5ex}\nobreak\noindent\hung} 
%\fi
%
%


\let\instructorNotes\relax
\let\endinstructorNotes\relax
%%% instructorNotes environment
\ifhandout
\newenvironment{instructorNotes}[1][false]%
{%
\def\givenatend{\boolean{#1}}\ifthenelse{\boolean{#1}}{\begin{trivlist}\item}{\setbox0\vbox\bgroup}{}
}
{%
\ifthenelse{\givenatend}{\end{trivlist}}{\egroup}{}
}
\else
\newenvironment{instructorNotes}[1][false]%
{%
  \ifthenelse{\boolean{#1}}{\begin{trivlist}\item[\hskip \labelsep\bfseries {\Large Instructor Notes: \\} \hspace{\textwidth} ]}
{\begin{trivlist}\item[\hskip \labelsep\bfseries {\Large Instructor Notes: \\} \hspace{\textwidth} ]}
{}
}
{\end{trivlist}}
\fi


%% Suggested Timing
\newcommand{\timing}[1]{{\bf Suggested Timing: \hspace{2ex}} #1}




\hypersetup{
    colorlinks=true,       % false: boxed links; true: colored links
    linkcolor=blue,          % color of internal links (change box color with linkbordercolor)
    citecolor=green,        % color of links to bibliography
    filecolor=magenta,      % color of file links
    urlcolor=cyan           % color of external links
}


\title{Where Have the Integers Been?}
\author{Vic Ferdinand, Betsy McNeal, Jenny Sheldon}

\begin{document}
\begin{abstract}
    We begin to model integers and their operations.
\end{abstract}
\maketitle



\begin{problem}
What is an integer?  How do integers compare with whole numbers?  Where do integers show up in real-life situations?  Give as many examples as you can think of -- we'll want a big list.
\end{problem}

\begin{problem}
For each of the integers below, use at least two of your real-life situations to write a sentence which uses that number in a sensible context.  Then, draw a picture which could represent your integer.
\begin{enumerate}
    \item $3$
    \item $-3$
    \item $-1.85$
    \item $-62$
\end{enumerate}

\end{problem}

\begin{problem}
Draw a number line which includes all of the numbers in the previous question.  How did you decide where to locate the negative numbers?
\end{problem}

\begin{problem}
For each of the addition problems below, use at least two of your real-life situations to write a story problem which would be answered by that expression.  (Be sure to include a question!)  Then, draw a picture explaining what a child learning this concept might do to solve the problem.
\begin{enumerate}
    \item $2+6$
    \item $(-3) + 5$
    \item $5 + (-8)$
    \item $(-1) + (-14)$
\end{enumerate}

\end{problem}

\begin{problem}
Look back again at each of the addition problems you did in the previous problem.  Does the sign of the answer make sense from your story?  Could you model the addition problem using a number line?

\end{problem}


\newpage
\begin{instructorNotes}
This activity is the first of three that investigate operations with negative numbers.  The second is ``Integers and Subtraction'' and the third is ``Our Problems are Multiplying''.  There is an accompanying reading (\url{https://ximera.osu.edu/mftsp19/elementaryTeachersOne}) on integers that lays out some of the more complicated framework.

In this first activity, our goal is to define integers, give some examples of real-life contexts in which negative numbers make sense, and then begin using these contexts to draw pictures and solve addition problems.  Overall, we try to treat integers not just from an algebraic standpoint, but focusing more on how young children think about integers so that our students learn to reason about these types of numbers rather than simply memorize rules for operating with them.  As usual, we are trying to communicate that mathematics is about reasoning, not just rules.

The reading includes a short historical discussion about integers which can help frame our in-class discussions.  We find it helpful to acknowledge that this concept was difficult for mathematicians for centuries, so we should not be surprised if it is also difficult for children.  Furthermore, we save this unit on integers for after we have discussed all other classes of numbers as well as all operations, so this unit helps to cement many of the topics we have already treated.

We use multiple contexts to talk about integers to allow for some flexibility in discussion later.  Especially on this first day, we let students try out any context they like, but moving forward we will focus on three contexts: checks and bills (or the shopkeeper's method or ``Mail Time''), number lines, and to a lesser extent chip models.  These contexts are described in the reading, and we have chosen them because they have the fewest issues (in our opinion).

\begin{itemize}
    \item We begin our discussion with student experimentation.  First, we give students time to come up with their own contexts, and then make a big list of possibilities as a class.
    \item After making this list, we give students time to try to draw a picture of a negative number.  The list (not all of integers) was chosen to help students start to realize that some contexts are more useful than others.  We begin to discuss this idea as we discuss the student work, and also introduce the checks-and-bills method if it has not yet come up.
    \item Often students' pictures come in two forms: pictures that look vaguely like number lines (temperature scales, elevators, actual number lines) and pictures that look vaguely like a chip model (two types of objects, labeled objects, heads and tails of coins).  If we can see many student pictures at the same time, we point out these similarities and differences for the class.
    \item After this experimentation, we begin to focus students on number lines and checks and bills with whatever time is left.
\end{itemize}

{\bf Suggested Timing:} This activity takes us a full class period (or more).  We often move some of the addition on to the second day.  As indicated earlier, we use 5 minutes to let students come up with contexts, and then 10 minutes to discuss.  We then let students draw and experiment for about 10 minutes, followed by 20 minutes of discussion.  The remainder of the time is spent on the last two problems.
\end{instructorNotes}





\end{document}