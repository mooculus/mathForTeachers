\documentclass[nooutcomes]{ximera}
\usepackage{gensymb}
\usepackage{tabularx}
\usepackage{mdframed}
\usepackage{pdfpages}
%\usepackage{chngcntr}

\let\problem\relax
\let\endproblem\relax

\newcommand{\property}[2]{#1#2}




\newtheoremstyle{SlantTheorem}{\topsep}{\fill}%%% space between body and thm
 {\slshape}                      %%% Thm body font
 {}                              %%% Indent amount (empty = no indent)
 {\bfseries\sffamily}            %%% Thm head font
 {}                              %%% Punctuation after thm head
 {3ex}                           %%% Space after thm head
 {\thmname{#1}\thmnumber{ #2}\thmnote{ \bfseries(#3)}} %%% Thm head spec
\theoremstyle{SlantTheorem}
\newtheorem{problem}{Problem}[]

%\counterwithin*{problem}{section}



%%%%%%%%%%%%%%%%%%%%%%%%%%%%Jenny's code%%%%%%%%%%%%%%%%%%%%

%%% Solution environment
%\newenvironment{solution}{
%\ifhandout\setbox0\vbox\bgroup\else
%\begin{trivlist}\item[\hskip \labelsep\small\itshape\bfseries Solution\hspace{2ex}]
%\par\noindent\upshape\small
%\fi}
%{\ifhandout\egroup\else
%\end{trivlist}
%\fi}
%
%
%%% instructorIntro environment
%\ifhandout
%\newenvironment{instructorIntro}[1][false]%
%{%
%\def\givenatend{\boolean{#1}}\ifthenelse{\boolean{#1}}{\begin{trivlist}\item}{\setbox0\vbox\bgroup}{}
%}
%{%
%\ifthenelse{\givenatend}{\end{trivlist}}{\egroup}{}
%}
%\else
%\newenvironment{instructorIntro}[1][false]%
%{%
%  \ifthenelse{\boolean{#1}}{\begin{trivlist}\item[\hskip \labelsep\bfseries Instructor Notes:\hspace{2ex}]}
%{\begin{trivlist}\item[\hskip \labelsep\bfseries Instructor Notes:\hspace{2ex}]}
%{}
%}
%% %% line at the bottom} 
%{\end{trivlist}\par\addvspace{.5ex}\nobreak\noindent\hung} 
%\fi
%
%


\let\instructorNotes\relax
\let\endinstructorNotes\relax
%%% instructorNotes environment
\ifhandout
\newenvironment{instructorNotes}[1][false]%
{%
\def\givenatend{\boolean{#1}}\ifthenelse{\boolean{#1}}{\begin{trivlist}\item}{\setbox0\vbox\bgroup}{}
}
{%
\ifthenelse{\givenatend}{\end{trivlist}}{\egroup}{}
}
\else
\newenvironment{instructorNotes}[1][false]%
{%
  \ifthenelse{\boolean{#1}}{\begin{trivlist}\item[\hskip \labelsep\bfseries {\Large Instructor Notes: \\} \hspace{\textwidth} ]}
{\begin{trivlist}\item[\hskip \labelsep\bfseries {\Large Instructor Notes: \\} \hspace{\textwidth} ]}
{}
}
{\end{trivlist}}
\fi


%% Suggested Timing
\newcommand{\timing}[1]{{\bf Suggested Timing: \hspace{2ex}} #1}




\hypersetup{
    colorlinks=true,       % false: boxed links; true: colored links
    linkcolor=blue,          % color of internal links (change box color with linkbordercolor)
    citecolor=green,        % color of links to bibliography
    filecolor=magenta,      % color of file links
    urlcolor=cyan           % color of external links
}

\title{Orange You Glad We're Back to Ratios}
\author{Vic Ferdinand, Betsy McNeal, Jenny Sheldon}

\begin{document}
\begin{abstract}
For each problem in this activity, we'll be considering a shade of paint made with 3 parts yellow paint and 5 parts red paint. 
\end{abstract}
\maketitle



\begin{problem} \label{Orange1}
 An orange paint is made with 3 parts yellow paint and 5 parts red paint.  How much yellow paint is needed to mix with 7841.9 gallons of red paint to make the same color orange paint?  Justify your answer.

\end{problem}
 
 
 

\begin{problem} \label{Orange2}
Let $x$ stand for the number of gallons of red paint in our mixture, and $y$ stand for the number of gallons of yellow paint in our mixture.
\begin{enumerate}
    \item Make a table with at least three more combinations of $x$ and $y$ that would make the same color of orange paint we have been working with.
    \item On the axes below, plot at least three of the points from your table.
    \item Does it make sense to connect the dots in some fashion on your graph?
\end{enumerate}


\begin{center}
    \begin{tikzpicture}[scale=0.8]
        \draw [step=1, help lines] (0,0) grid (10,10); 
        \draw [->] (-0.2,0) -- (10.2,0);
        \draw[->] (0, -0.2)--(0,10.2); 
        \node at (0, 10.5) {$y$};
        \node at (10.5, 0){$x$};
        \foreach \x/\xtext in {1, ..., 10} 
            \draw (\x cm,1pt) -- (\x cm,-1pt) node[anchor=north] {$\xtext$};
        \foreach \y/\ytext in {1, ..., 10} 
            \node at (-0.25, \y){$\ytext$};
    \end{tikzpicture}
\end{center}
\end{problem}





\begin{problem} \label{Orange3}
\begin{enumerate}
    \item Is $(0,0)$ on your graph?  Explain why or why not using the paint situation.
    \item What can you say about all of the points $(x,y)$ which lie {\bf above} your graph?  Explain using the paint situation.
    \item Why might someone call this graph an ``isocolor graph''?  You might have to look up the prefix ``iso-''.
\end{enumerate}
\end{problem}

 

\begin{problem} \label{Orange4}
Suppose we had a vat of an unknown amount of the same shade of orange paint.  Notice that this paint is already made for us, but we don't know how it was made.
\begin{enumerate}
    \item How much yellow paint should we pour in to maintain the color if we pour in 1 gallon of red paint?
    \item How much yellow paint should we pour in to maintain the color if we pour in 2 gallons of red paint?
    \item How much yellow paint should we pour in to maintain the color if we pour in 5 gallons of red paint?
\end{enumerate}
Write your answers to each of the above in fraction and decimal form.  How do your answers relate to your graph and equation?  What does the number $\frac35  (= 0.6)$ have to do with the graph at each point on the graph?
\end{problem}


\begin{problem} \label{Orange5}
The topics in this activity, especially in the previous question, have a strong relationship to similar triangles.  What is this relationship?
\end{problem}



\begin{problem} \label{Orange6}
\begin{enumerate}
    \item What kind of graph did you draw earlier?  How can you be sure this is correct?
    \item Write an equation corresponding to your graph.
    \item Explain how each part of your equation is connected with the original paint situation.
    \item Explain how your solution method to the first question of this activity is related to this graph.  In other words, how can you see each step of your solution to that first question on the graph?
\end{enumerate}
\end{problem}


\newpage
\begin{instructorNotes}
The objective of this activity is to help students make the connection between the ``going through one'' strategy of solving pure ratio problems and the familiar (from school) slope of a line that contains $(0, 0)$.  We want the students to connect these ideas both algebraically and graphically.  

In our calendar, this activity is the first we do on more advanced ratio concepts.  We discuss the basic concepts of ratios in the first semester, and then return to ratios in the second semester partly as review and partly to demonstrate some connections between ratios and other content we have studied.  We have found ratios to be one of the more difficult concepts for students in the first semester, so we wanted to give students more practice with these ideas.

This activity assumes that students have already studied ratios as well as linear relationships both algebraically and graphically.  There is some reference to similarity near the end as well.


Here are the types of questions we would like our students to be able to answer by the end of the activity.
\begin{enumerate}
    \item Where did you run into the number $\frac35$, and why?      
    \item What does $\frac35 = 0.6$ mean in this problem?  What are its units?     
    \item Can we write the answer as $\frac35 \times 7841.9$? If so, why do we use multiplication here?  What are the groups and objects?  
    \item How would this interpretation change if we replaced ``gallons of red paint and gallons of yellow paint'' with ``hours traveled and miles traveled'' on a trip?
\end{enumerate}
\begin{itemize}
    \item Problem \ref{Orange1} is review, and we move through it quickly.  Frequently, we do this problem essentially together as a whole class.  We review the multiplicative structure of ratios, as well as the usefulness of the ``going through $1$'' strategy where we first find how much yellow paint goes with one unit of red paint (or vice versa).  We also emphasize that the fraction $\frac{3}{5}$ is a single number, not two numbers.  Our students frequently refer to the $\frac{3}{5}$ as a symbol representing ``over $5$ and up $3$'', forgetting the connection to the meaning of fractions.  We review this connection here.
    \item In Problem \ref{Orange2}, we remind students to make non-integral batches of paint, particularly so their graph will fit on the given grid.  We also ask, ``Would it make sense to include the point $(0,7)$ in your table?''.  This helps students to not just blindly make a table but to think about the meaning of the numbers.  This also helps students to realize that the points off of the line have as much meaning as the points on the line.  We bring up ``connecting the dots'' so that students realize that there are an infinite number of combinations that will make the same color orange and that each point represents one of those combinations.
    \item The purpose of Problem \ref{Orange3} is to look at the connection between the algebraic method of ``solving with a ratio'' and the geometric method of ``solving with a graph''.  We want to bring out the geometric aspects of the graph - remembering that the graph shows us both which points satisfy the relationship as well as which points don't satisfy the relationship.  As usual, we require our students' observations to be connected back to the original situation.  We have observed that when working with a graph, students tend to forget about the physical situation involved with the problem.
    \item The purpose of Problem \ref{Orange4} is for students to connect their understanding of slope as ``over and up'' to the situation with the paint.  We start with an unknown amount of paint to emphasize that the answers will be the same no matter where we are on the line, or in other words the constant rate.  We also continue to emphasize connecting the answers dealing with the graph to the physical situation with the paint.  The problem asks for answers in both fraction and decimal form to emphasize again the unit rate, and that ``3 over 5'' is a single number.  We also hope that the decimal form makes the overall pattern more clear.
    \item In Problem \ref{Orange5}, we use a geometric argument about similar triangles to see that our graph is a line.  This problem helps us connect the slope to the scale factor for these triangles, and recall that similar triangles have congruent angles.  This angle gives us the direction for the line, and we connect this direction idea to our work with parallel lines earlier in the semester.
    \item Problem \ref{Orange6} is intended to help us pull all of these ideas together.  We expect students to connect the ideas of ratios, similar triangles, and constant rate of change.  The activity ``Return to Walk the Line'' discusses some of these concepts, including writing equations for linear relationships, and so we will again remind the students of their previous work.
    \item In part (d) of Problem \ref{Orange6}, we intend to reinforce the connection between the unit rate and the slope. We bring up using a proportion to solve the original problem, and discuss how this proportion is related to the unit rate and to the slope.
\end{itemize}



\timing{We use one class period for this activity.  We spend about 5 minutes at the beginning on the first problem as a whole class.  We then give students about 15 minutes to work through the problems in their small groups, and use the rest of the time to discuss as a whole class.}
\end{instructorNotes}



\end{document}