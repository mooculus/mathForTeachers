\documentclass[nooutcomes, handout]{ximera}
\usepackage{gensymb}
\usepackage{tabularx}
\usepackage{mdframed}
\usepackage{pdfpages}
%\usepackage{chngcntr}

\let\problem\relax
\let\endproblem\relax

\newcommand{\property}[2]{#1#2}




\newtheoremstyle{SlantTheorem}{\topsep}{\fill}%%% space between body and thm
 {\slshape}                      %%% Thm body font
 {}                              %%% Indent amount (empty = no indent)
 {\bfseries\sffamily}            %%% Thm head font
 {}                              %%% Punctuation after thm head
 {3ex}                           %%% Space after thm head
 {\thmname{#1}\thmnumber{ #2}\thmnote{ \bfseries(#3)}} %%% Thm head spec
\theoremstyle{SlantTheorem}
\newtheorem{problem}{Problem}[]

%\counterwithin*{problem}{section}



%%%%%%%%%%%%%%%%%%%%%%%%%%%%Jenny's code%%%%%%%%%%%%%%%%%%%%

%%% Solution environment
%\newenvironment{solution}{
%\ifhandout\setbox0\vbox\bgroup\else
%\begin{trivlist}\item[\hskip \labelsep\small\itshape\bfseries Solution\hspace{2ex}]
%\par\noindent\upshape\small
%\fi}
%{\ifhandout\egroup\else
%\end{trivlist}
%\fi}
%
%
%%% instructorIntro environment
%\ifhandout
%\newenvironment{instructorIntro}[1][false]%
%{%
%\def\givenatend{\boolean{#1}}\ifthenelse{\boolean{#1}}{\begin{trivlist}\item}{\setbox0\vbox\bgroup}{}
%}
%{%
%\ifthenelse{\givenatend}{\end{trivlist}}{\egroup}{}
%}
%\else
%\newenvironment{instructorIntro}[1][false]%
%{%
%  \ifthenelse{\boolean{#1}}{\begin{trivlist}\item[\hskip \labelsep\bfseries Instructor Notes:\hspace{2ex}]}
%{\begin{trivlist}\item[\hskip \labelsep\bfseries Instructor Notes:\hspace{2ex}]}
%{}
%}
%% %% line at the bottom} 
%{\end{trivlist}\par\addvspace{.5ex}\nobreak\noindent\hung} 
%\fi
%
%


\let\instructorNotes\relax
\let\endinstructorNotes\relax
%%% instructorNotes environment
\ifhandout
\newenvironment{instructorNotes}[1][false]%
{%
\def\givenatend{\boolean{#1}}\ifthenelse{\boolean{#1}}{\begin{trivlist}\item}{\setbox0\vbox\bgroup}{}
}
{%
\ifthenelse{\givenatend}{\end{trivlist}}{\egroup}{}
}
\else
\newenvironment{instructorNotes}[1][false]%
{%
  \ifthenelse{\boolean{#1}}{\begin{trivlist}\item[\hskip \labelsep\bfseries {\Large Instructor Notes: \\} \hspace{\textwidth} ]}
{\begin{trivlist}\item[\hskip \labelsep\bfseries {\Large Instructor Notes: \\} \hspace{\textwidth} ]}
{}
}
{\end{trivlist}}
\fi


%% Suggested Timing
\newcommand{\timing}[1]{{\bf Suggested Timing: \hspace{2ex}} #1}




\hypersetup{
    colorlinks=true,       % false: boxed links; true: colored links
    linkcolor=blue,          % color of internal links (change box color with linkbordercolor)
    citecolor=green,        % color of links to bibliography
    filecolor=magenta,      % color of file links
    urlcolor=cyan           % color of external links
}

\title{Orange You Glad We're Back to Ratios}
\author{Vic Ferdinand, Betsy McNeal, Jenny Sheldon}

\begin{document}
\begin{abstract}
An orange paint is made with 3 parts yellow paint and 5 parts red paint. 
\end{abstract}
\maketitle

\begin{instructorIntro}
The objective of this activity is to help students make the connection between the ``going through one'' strategy of solving pure ratio problems and the familiar (from school) slope of a line that contains $(0, 0)$.  We want the students to connect these ideas both algebraically and graphically.  At some relevant point during the activity, you should pose the following questions to the students.


\begin{enumerate}
    \item Where did you run into the number $\frac35$, and why?  (This is one of the unit rates, or 3/5 of a unit of yellow paint goes with 1 unit of red paint.
    \item What does $\frac35 = 0.6$ mean in this problem?  What are its units?  (Emphasizing that this is a number, not just ``3 over 5'' or ``3 to 5''.)
    \item Can we write the answer as $\frac35 \times 7841.9$? If so, why do we use multiplication here?  What are the groups and objects? (Yes; the groups are the batches and the objects in those groups are whatever is in one batch.)
    \item How would this interpretation change if we replaced ``gallons of red paint and gallons of yellow paint'' with ``hours traveled and miles traveled'' on a trip?
\end{enumerate}
\end{instructorIntro}

\begin{question}
 An orange paint is made with 3 parts yellow paint and 5 parts red paint.  How much yellow paint is needed to mix with 7841.9 gallons of red paint to make the same color orange paint?  Justify your answer.


\begin{instructorNotes}
This part should be done as a whole class and should be done quickly.  The idea is that we want to maintain the same color and thus want to make ``copies'' of the base recipe, including non-integral copies (and thus, we multiply each quantity by the same number).  The ultimate combination toward finding how much yellow paint is needed to go with any given amount of red paint is to first find how much yellow paint will go with one unit of red paint, which is found by multiplying each amount by $1/5$ (i.e., take $1/5$ of the orange paint in the base recipe ($3/5 = .6$ units).  Now we can take advantage of the multiplicative identity to make 7841.9 copies of the ``unit'' amount.

    Ask: How many numbers is $\frac{3}{5}$ (Hint:  $\frac{3}{5}$ can also be written as 0.6)? Answer: only one (not two).  Here, $3/5 = .6$ is one number representing the unit rate of 0.6 gallons of yellow paint per one gallon of red paint (objects = amount of yellow and groups = amount of red).  This is truly a rate (i.e, ``rate-eo'') in the same sense that we would get the familiar ``miles per (one) hour'' if we replaced the quantities (red, yellow) with (time, distance).
\end{instructorNotes}
\end{question}
 

\begin{question}
Let $x$ stand for the number of gallons of red paint in our mixture, and $y$ stand for the number of gallons of yellow paint in our mixture.
\begin{enumerate}
    \item Make a table with at least three more combinations of $x$ and $y$ that would make the same color of orange paint as in Question 1.
    \item On the axes below, plot at least three of the points from your table.
    \item Does it make sense to connect the dots in some fashion on your graph?
\end{enumerate}

\includegraphics[height=4in]{graphics/axes.png}

\begin{instructorNotes}
Students should be guided to use reasonably small amounts of each color so their points will fit on the given grid (or with a need for only a small extension).  It is likely that many students will make 1, 2, 3, or 4 batches of the base recipe with perhaps some non-integral copies of it to fit the grid.

A good question to ask in the middle of this problem is, ``Would it make sense to include the point $(0,7)$ in your table?''.  This helps students to not just blindly make a table but to think about the meaning of the numbers.  This also helps students to realize that the points off of the line have as much meaning as the points on the line.

       The idea of ``connecting the dots'' is for students to realize that there are an infinite number of combinations that will make the same color orange and that each point represents one of those combinations (``isochromatic'' graph).
\end{instructorNotes}
\end{question}

\begin{question}
\begin{enumerate}
    \item Is $(0,0)$ on your graph?  Explain why or why not using the paint situation.
    \item What can you say about all of the points $(x,y)$ which lie {\bf above} your graph?  Explain using the paint situation.
    \item Why might someone call this graph an ``isocolor graph''?  You might have to look up the prefix ``iso-''.
\end{enumerate}

\begin{instructorNotes}
Our purpose in this problem is to look at the connection between the algebraic method of ``solving with a ratio'' and the geometric method of ``solving with a graph''.  We want to bring out the geometric aspects of the graph - remembering that the graph shows us both which points satisfy the relationship as well as which points don't satisfy the relationship.  All observations in this question should be connected back to the original situation.  We don't want to fall into the trap of having a graph and then forgetting about the original problem!

Part (c) is a connection to economics - where we frequently refer to an ``isocost'' line where the cost is always equal.  The prefix ``iso'' means ``equal''.
\end{instructorNotes}
\end{question}

 

\begin{question} 
Suppose we had a vat of an unknown amount of the same shade of orange paint.  Notice that this paint is already made for us, but we don't know how it was made.
\begin{enumerate}
    \item How much yellow paint should we pour in to maintain the color if we pour in 1 gallon of red paint?
    \item How much yellow paint should we pour in to maintain the color if we pour in 2 gallons of red paint?
    \item How much yellow paint should we pour in to maintain the color if we pour in 5 gallons of red paint?
\end{enumerate}
Write your answers to each of the above in fraction and decimal form.  How do your answers relate to your graph and equation?  What does the number $\frac35  (= 0.6)$ have to do with the graph at each point on the graph?


\begin{instructorNotes}
We intend students to recognize the familiar understanding of slope as ``over and up'' in developing the line as well as connecting it to the paint.  Starting with an ``unknown'' in this problem should emphasize that the answers will be the same no matter where we are on the line and that $3/5$ is a universal property of the line.  That is, although we can have different combinations of paints corresponding to different points on the line, our rate is always constant. Stating this in yet another way, our position may differ, but our velocity is always the same -- something that is learned later in calculus.  

        Be sure that students answer both in fraction and decimal language (Over 1 and up $3/5$ = over 1 and up 0.6.  Over 2 and up $6/5$ = over 2 and up 1.2 = two .6's).  All of this is with the understanding that staying on the line means the same as generating new points which means the same as keeping the same orange shade.

         In (c), the use of ``over 5 and up 3'' is what students might recall from school days with slope (almost treating $3/5$ as two numbers as change of $x$ and change of $y$).  Hopefully, answering these in decimal form (five .6’s) will reveal that what they recall is just a special case of the pattern (and that we could go over, say, 8.32 and up 8.32 .6's).
\end{instructorNotes}
\end{question}


\begin{question}
The topics in this activity, especially in Question 4, have a strong relationship to similar triangles.  What is this relationship?

\begin{instructorNotes}
Here, we see a better idea for why our graph is a line from a geometric point of view.  The arithmetic done in \#4 forms similar (right) triangles (slope = constant internal factor).  You might ask them to recall a salient feature of similar triangles: That they all have congruent angles.  Thus, the generation of new points (near or far) will be done in a way such that the direction must always be the same (never changes) and thus the points representing combos of orange paint must all lie on a line.
\end{instructorNotes}
\end{question}



\begin{question}
\begin{enumerate}
    \item What kind of graph did you draw in Question 2?  How can you be sure this is correct?
    \item Write an equation corresponding to your graph.
    \item Explain how each part of your equation is connected with the original paint situation.
    \item Explain how your solution method to Question 1 is related to this graph.  In other words, how can you see each step of your solution to Question 1 on the graph?
\end{enumerate}

\begin{instructorNotes}
In part (a), we are expecting students to connect the ideas of ratios, similar triangles, and constant rate of change from problem 5 and our work with arithmetic sequences earlier in the semester.  If you've done ``Return to Walk the Line'', you might bring up these concepts.  You should also connect with the ``over 5 and up 3'' manner brought out in \#4.  This might be developed through \#3 as we develop things algebraically and use the equation and graph in \#4 and \#5.

Once we realize this graph should be linear, it's easier to write an equation for it.  On (b), you might recall the activity ``I Walk the Line'' and discuss why the equation is $y = (3/5)x$ and not, say, $y = (3/5)x + 7$.  Most of (c) should be bringing together and reviewing the various parts of this activity.

In part (d), we return to the original question and are hoping to have multiple solution methods.  In particular, this part should reinforce the connection between the unit rate and the slope.  In particular, you should bring up or find the solution using a proportion, and discuss how the proportion is related to the unit rate and to the slope of the graph.
\end{instructorNotes}

\end{question}


\end{document}