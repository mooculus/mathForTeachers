\documentclass[nooutcomes,handout,noauthor]{ximera}
\usepackage{gensymb}
\usepackage{tabularx}
\usepackage{mdframed}
\usepackage{pdfpages}
%\usepackage{chngcntr}

\let\problem\relax
\let\endproblem\relax

\newcommand{\property}[2]{#1#2}




\newtheoremstyle{SlantTheorem}{\topsep}{\fill}%%% space between body and thm
 {\slshape}                      %%% Thm body font
 {}                              %%% Indent amount (empty = no indent)
 {\bfseries\sffamily}            %%% Thm head font
 {}                              %%% Punctuation after thm head
 {3ex}                           %%% Space after thm head
 {\thmname{#1}\thmnumber{ #2}\thmnote{ \bfseries(#3)}} %%% Thm head spec
\theoremstyle{SlantTheorem}
\newtheorem{problem}{Problem}[]

%\counterwithin*{problem}{section}



%%%%%%%%%%%%%%%%%%%%%%%%%%%%Jenny's code%%%%%%%%%%%%%%%%%%%%

%%% Solution environment
%\newenvironment{solution}{
%\ifhandout\setbox0\vbox\bgroup\else
%\begin{trivlist}\item[\hskip \labelsep\small\itshape\bfseries Solution\hspace{2ex}]
%\par\noindent\upshape\small
%\fi}
%{\ifhandout\egroup\else
%\end{trivlist}
%\fi}
%
%
%%% instructorIntro environment
%\ifhandout
%\newenvironment{instructorIntro}[1][false]%
%{%
%\def\givenatend{\boolean{#1}}\ifthenelse{\boolean{#1}}{\begin{trivlist}\item}{\setbox0\vbox\bgroup}{}
%}
%{%
%\ifthenelse{\givenatend}{\end{trivlist}}{\egroup}{}
%}
%\else
%\newenvironment{instructorIntro}[1][false]%
%{%
%  \ifthenelse{\boolean{#1}}{\begin{trivlist}\item[\hskip \labelsep\bfseries Instructor Notes:\hspace{2ex}]}
%{\begin{trivlist}\item[\hskip \labelsep\bfseries Instructor Notes:\hspace{2ex}]}
%{}
%}
%% %% line at the bottom} 
%{\end{trivlist}\par\addvspace{.5ex}\nobreak\noindent\hung} 
%\fi
%
%


\let\instructorNotes\relax
\let\endinstructorNotes\relax
%%% instructorNotes environment
\ifhandout
\newenvironment{instructorNotes}[1][false]%
{%
\def\givenatend{\boolean{#1}}\ifthenelse{\boolean{#1}}{\begin{trivlist}\item}{\setbox0\vbox\bgroup}{}
}
{%
\ifthenelse{\givenatend}{\end{trivlist}}{\egroup}{}
}
\else
\newenvironment{instructorNotes}[1][false]%
{%
  \ifthenelse{\boolean{#1}}{\begin{trivlist}\item[\hskip \labelsep\bfseries {\Large Instructor Notes: \\} \hspace{\textwidth} ]}
{\begin{trivlist}\item[\hskip \labelsep\bfseries {\Large Instructor Notes: \\} \hspace{\textwidth} ]}
{}
}
{\end{trivlist}}
\fi


%% Suggested Timing
\newcommand{\timing}[1]{{\bf Suggested Timing: \hspace{2ex}} #1}




\hypersetup{
    colorlinks=true,       % false: boxed links; true: colored links
    linkcolor=blue,          % color of internal links (change box color with linkbordercolor)
    citecolor=green,        % color of links to bibliography
    filecolor=magenta,      % color of file links
    urlcolor=cyan           % color of external links
}

\title{Formulate our thoughts}
\begin{document}
\begin{abstract}
This activity poses questions to help us pull together ideas discussed previously and help us summarize our results.
\end{abstract}
\maketitle




\begin{problem}
Suppose you have a poster that is 12 inches by 15 inches and a lot of 3 inch by 5 inch index cards (``3-by-5 cards" for short).
\begin{enumerate}
    \item What is the area of your poster in 3-by-5 cards? How do you know?  Be sure to draw a picture.  Write down your answer, including the units, and be ready to explain this to the class.  

    \item Is there another way that you could have done part (a)? If yes, explain your second method. Write down your answer, including the units, and be ready to explain this to the class. If no, explain why not.
\end{enumerate}

\end{problem}


\begin{problem} 
Suppose you have a lot of 1 inch by 1 inch squares also known as ``square inches". What is the area of your poster in square inches? How do you know your answer is correct?  If you used a formula, why does it work?  If you did not use a formula, explain your method.

\end{problem}



\begin{problem}
Now suppose you have a lot of 1 foot by 1 foot squares, also known as ``square feet". 

 \begin{enumerate}
\item Draw a picture of one square foot.
\vskip 1in
\item What is the area of your poster in square feet? How do you know your answer is correct?  If you used a formula, why does it work?  If you did not use a formula, explain your method using a picture and Math 1125 reasoning about the meaning of multiplication. (If you did not use this book when you took Math 1125 - see Chapter 4.1 in our current text.)
\end{enumerate}

\end{problem}



\newpage

\begin{problem}
If you used multiplication to solve the previous problems, you might have noticed a pattern that has something to do with the familiar ``$L \times W$ formula you learned in school. If you didn't use multiplication in the previous problems, look again with this idea in mind!
\begin{enumerate} 
\item Explain how the ``$L \times W$'' method is related to the other methods we have been using to measure area. Be as specific as you can!
\item Using your ruler, try out the ``$L \times W$" method on the purple rectangle shown below.  What is the area of that rectangle?  How do you know?  What are the units? Where are the units? How is this related to our problem with the poster?

\begin{image}
\begin{tikzpicture}
\draw[thick] (0,0) rectangle (7, 5);
\end{tikzpicture}
\end{image}


\item How did measurement of two LENGTHS generate an answer in an area unit?
\end{enumerate}
\end{problem}

\newpage

\begin{instructorNotes}
{\bf Main goal:} We explore and justify the usual ``length times width'' formula for calculating area and continue to emphasize the covering ideas for calculating area.

{\bf Overall picture:} 

Problem 1
\begin{itemize}
	\item We begin with non-standard units, connecting to our previous work covering a desk with index cards.
	\item We expect to see pictures of the students' work, and at least the two methods of directly counting the index cards as well as using multiplication to organize their counting.
	\item Many questions about area can come up here if they haven't already been settled. For instance, we want to use ``index cards'' as our unit of measure, not ``index cards squared'' or other. When we use multiplication to count the cards, they should be organized nicely in rows and columns that exactly cover the shape with no gaps and no overlaps. Because they are well-organized, we can use our shortcut to counting and multiply. If the cards are turned in other directions, we can still count or estimate the area, we just can't use multiplication to do so. The issue of the card which is ``double-counted'' may also come up: when we count the length and the width to do multiplication, one of the cards in the corner is counted twice. We recall that it is counted using different units each time (for example, we use it to indicate a whole row, and then we use it to indicate a specific card within that row) so it's not actually being double-counted. Other questions may come up, but take your time and deal with them.
\end{itemize}

Problems 2 and 3
\begin{itemize}
	\item Here, we transition back to more standard units. We want to see the parallels between what we have done with the index cards and what we are doing with the square inches and square feet. We are still covering!
	\item Students may cover the poster directly with square inches and count them (or use multiplication). They may also reason from the size of the original index card (15 square inches per index card) and then use this information to find the total from their answers in Problem 1. Both methods should be discussed!
\end{itemize}

Problem 4
\begin{itemize}
	\item Now, we justify the usual formula in terms of groups and objects multiplication. We can use either the rows or the columns as groups, and then the square inches are our objects per group. 
	\item Part (c) is important, but difficult for students to see, especially if they are still working to think of area measurement as how many of a given unit it takes to cover the space inside a 2D object.  According to our usual meaning of multiplication, it doesn't make any sense to say things like ``inches times inches gives you square inches'', and this should be brought up. The units on the two factors in a multiplication expression have different units! The second factor should be the thing we are trying to count, which is square units in this case. Once we really get a hold of this idea, the length times width idea should actually be a little bit bothersome! 
	\item Have students discuss their thoughts on part (c), but try to guide them towards the idea that we are matching one unit of length along the side to one square unit of area. Because our length units and area units are related, these should match exactly. (In fact, more specifically, we have a one-to-one correspondence between inches along the width (or height) and square units that fit against that side.) 
\end{itemize}


{\bf Good language:} After this activity, it's a good idea to ask students (perhaps in an exit ticket or participation activity) whether their thinking about area has changed over the last few classes. We are trying to break old habits in the way they think of area, so listen carefully for those old habits to reappear and correct students gently.

{\bf Suggested timing:} This activity should take the whole class period, and it's good to break up the work into three segments: Problem 1, Problems 2 and 3, and Problem 4. Each time, give students about 5-8 minutes to work, and then spend roughly 10-15 minutes having them present their work and discuss their thinking. Wrap up by letting the students know that we have now justified the formula for calculating area of rectangles, and so they may use this formula without justification unless they are specifically asked to explain it.



\end{instructorNotes}
\end{document}
