%\documentclass[nooutcomes,handout]{ximera}
\documentclass[nooutcomes,noauthor, handout]{ximera}
\usepackage{gensymb}
\usepackage{tabularx}
\usepackage{mdframed}
\usepackage{pdfpages}
%\usepackage{chngcntr}

\let\problem\relax
\let\endproblem\relax

\newcommand{\property}[2]{#1#2}




\newtheoremstyle{SlantTheorem}{\topsep}{\fill}%%% space between body and thm
 {\slshape}                      %%% Thm body font
 {}                              %%% Indent amount (empty = no indent)
 {\bfseries\sffamily}            %%% Thm head font
 {}                              %%% Punctuation after thm head
 {3ex}                           %%% Space after thm head
 {\thmname{#1}\thmnumber{ #2}\thmnote{ \bfseries(#3)}} %%% Thm head spec
\theoremstyle{SlantTheorem}
\newtheorem{problem}{Problem}[]

%\counterwithin*{problem}{section}



%%%%%%%%%%%%%%%%%%%%%%%%%%%%Jenny's code%%%%%%%%%%%%%%%%%%%%

%%% Solution environment
%\newenvironment{solution}{
%\ifhandout\setbox0\vbox\bgroup\else
%\begin{trivlist}\item[\hskip \labelsep\small\itshape\bfseries Solution\hspace{2ex}]
%\par\noindent\upshape\small
%\fi}
%{\ifhandout\egroup\else
%\end{trivlist}
%\fi}
%
%
%%% instructorIntro environment
%\ifhandout
%\newenvironment{instructorIntro}[1][false]%
%{%
%\def\givenatend{\boolean{#1}}\ifthenelse{\boolean{#1}}{\begin{trivlist}\item}{\setbox0\vbox\bgroup}{}
%}
%{%
%\ifthenelse{\givenatend}{\end{trivlist}}{\egroup}{}
%}
%\else
%\newenvironment{instructorIntro}[1][false]%
%{%
%  \ifthenelse{\boolean{#1}}{\begin{trivlist}\item[\hskip \labelsep\bfseries Instructor Notes:\hspace{2ex}]}
%{\begin{trivlist}\item[\hskip \labelsep\bfseries Instructor Notes:\hspace{2ex}]}
%{}
%}
%% %% line at the bottom} 
%{\end{trivlist}\par\addvspace{.5ex}\nobreak\noindent\hung} 
%\fi
%
%


\let\instructorNotes\relax
\let\endinstructorNotes\relax
%%% instructorNotes environment
\ifhandout
\newenvironment{instructorNotes}[1][false]%
{%
\def\givenatend{\boolean{#1}}\ifthenelse{\boolean{#1}}{\begin{trivlist}\item}{\setbox0\vbox\bgroup}{}
}
{%
\ifthenelse{\givenatend}{\end{trivlist}}{\egroup}{}
}
\else
\newenvironment{instructorNotes}[1][false]%
{%
  \ifthenelse{\boolean{#1}}{\begin{trivlist}\item[\hskip \labelsep\bfseries {\Large Instructor Notes: \\} \hspace{\textwidth} ]}
{\begin{trivlist}\item[\hskip \labelsep\bfseries {\Large Instructor Notes: \\} \hspace{\textwidth} ]}
{}
}
{\end{trivlist}}
\fi


%% Suggested Timing
\newcommand{\timing}[1]{{\bf Suggested Timing: \hspace{2ex}} #1}




\hypersetup{
    colorlinks=true,       % false: boxed links; true: colored links
    linkcolor=blue,          % color of internal links (change box color with linkbordercolor)
    citecolor=green,        % color of links to bibliography
    filecolor=magenta,      % color of file links
    urlcolor=cyan           % color of external links
}


\begin{document}
%\title{Explanations}



%\documentclass[nooutcomes]{ximera}
%\usepackage{gensymb}
\usepackage{tabularx}
\usepackage{mdframed}
\usepackage{pdfpages}
%\usepackage{chngcntr}

\let\problem\relax
\let\endproblem\relax

\newcommand{\property}[2]{#1#2}




\newtheoremstyle{SlantTheorem}{\topsep}{\fill}%%% space between body and thm
 {\slshape}                      %%% Thm body font
 {}                              %%% Indent amount (empty = no indent)
 {\bfseries\sffamily}            %%% Thm head font
 {}                              %%% Punctuation after thm head
 {3ex}                           %%% Space after thm head
 {\thmname{#1}\thmnumber{ #2}\thmnote{ \bfseries(#3)}} %%% Thm head spec
\theoremstyle{SlantTheorem}
\newtheorem{problem}{Problem}[]

%\counterwithin*{problem}{section}



%%%%%%%%%%%%%%%%%%%%%%%%%%%%Jenny's code%%%%%%%%%%%%%%%%%%%%

%%% Solution environment
%\newenvironment{solution}{
%\ifhandout\setbox0\vbox\bgroup\else
%\begin{trivlist}\item[\hskip \labelsep\small\itshape\bfseries Solution\hspace{2ex}]
%\par\noindent\upshape\small
%\fi}
%{\ifhandout\egroup\else
%\end{trivlist}
%\fi}
%
%
%%% instructorIntro environment
%\ifhandout
%\newenvironment{instructorIntro}[1][false]%
%{%
%\def\givenatend{\boolean{#1}}\ifthenelse{\boolean{#1}}{\begin{trivlist}\item}{\setbox0\vbox\bgroup}{}
%}
%{%
%\ifthenelse{\givenatend}{\end{trivlist}}{\egroup}{}
%}
%\else
%\newenvironment{instructorIntro}[1][false]%
%{%
%  \ifthenelse{\boolean{#1}}{\begin{trivlist}\item[\hskip \labelsep\bfseries Instructor Notes:\hspace{2ex}]}
%{\begin{trivlist}\item[\hskip \labelsep\bfseries Instructor Notes:\hspace{2ex}]}
%{}
%}
%% %% line at the bottom} 
%{\end{trivlist}\par\addvspace{.5ex}\nobreak\noindent\hung} 
%\fi
%
%


\let\instructorNotes\relax
\let\endinstructorNotes\relax
%%% instructorNotes environment
\ifhandout
\newenvironment{instructorNotes}[1][false]%
{%
\def\givenatend{\boolean{#1}}\ifthenelse{\boolean{#1}}{\begin{trivlist}\item}{\setbox0\vbox\bgroup}{}
}
{%
\ifthenelse{\givenatend}{\end{trivlist}}{\egroup}{}
}
\else
\newenvironment{instructorNotes}[1][false]%
{%
  \ifthenelse{\boolean{#1}}{\begin{trivlist}\item[\hskip \labelsep\bfseries {\Large Instructor Notes: \\} \hspace{\textwidth} ]}
{\begin{trivlist}\item[\hskip \labelsep\bfseries {\Large Instructor Notes: \\} \hspace{\textwidth} ]}
{}
}
{\end{trivlist}}
\fi


%% Suggested Timing
\newcommand{\timing}[1]{{\bf Suggested Timing: \hspace{2ex}} #1}




\hypersetup{
    colorlinks=true,       % false: boxed links; true: colored links
    linkcolor=blue,          % color of internal links (change box color with linkbordercolor)
    citecolor=green,        % color of links to bibliography
    filecolor=magenta,      % color of file links
    urlcolor=cyan           % color of external links
}


\title{Scale It Up}
\begin{abstract}
\end{abstract}

\maketitle


Recall: When an object is scaled, the new object (which is similar to the original) is the same shape as the original, but a different size.  In this activity, we explore what happens to area and volume when we scale.



\begin{problem}
The rectangle below is $4$ units wide and $3$ units tall. If we scale it by a scale factor of $5$, what will be the new dimensions of the rectangle? Draw the new rectangle in the given space.\\

\tikz{ \draw (0 cm,0) -- (2 cm,0); \draw (2 cm, 0) -- (2 cm, 1 cm); \draw (2 cm, 1 cm) -- (0 cm, 1 cm);\draw (0 cm, 1 cm) -- (0 cm, 0);} \hspace{4cm}
\vskip 2.5in

\begin{enumerate}
    \item How does the original length of the rectangle compare to the new length?
    \item How does the diagonal of the original rectangle compare to the new diagonal?
    \item What happened to the area when you scaled the rectangle?
\end{enumerate}


\end{problem}


\begin{problem}
What will happen to length and area if we scale a rectangle of dimensions $2.67 cm$ by $5.18 cm$ by a linear scale factor of $k$? (If this problem is tough, start with $k=3$, then try some different values of $k$ until you see a pattern.)

\end{problem}



\begin{problem}
The rectangle below is $L$ units wide and $W$ units tall. If we scale it by a scale factor of $k$, what will be the new dimensions of the rectangle? Draw the new rectangle in the given space. What happens to the area when you scaled the rectangle?\\

\tikz{ \draw (0 cm,0) -- (2 cm,0); \draw (2 cm, 0) -- (2 cm, 1 cm); \draw (2 cm, 1 cm) -- (0 cm, 1 cm);\draw (0 cm, 1 cm) -- (0 cm, 0);} \hspace{4cm}
\vskip 2.5in

\end{problem}




\begin{problem}
Justify your observations in question 3 in three different ways.  By the end of this class period, you should have an algebraic explanation (using the formula for area), a geometric explanation (using your picture in problem 1), and an explanation dealing with scaling an individual unit of area.

\end{problem}


\newpage

\begin{problem}
Draw a cube whose side length is 3cm, and a rectangular prism whose dimensions are 2cm by 5cm by 3cm.  Scale each of these prisms using several different linear scale factors of your choosing, and record the ratio of the original volume to the new volume.  What do you notice?
\vskip 2.3in
\begin{enumerate}
    \item How does the original length of the prism compare to the new length?
    \item How does the area of the base of the original prism compare to the base of the new prism?
    \item What happened to the volume when you scaled the prism?
\end{enumerate}

\end{problem}




\begin{problem}
Justify your observations in the previous question by scaling a unit of volume.

\end{problem}



\begin{problem}
When we scaled the cube and rectangular prism above, what happened to their surface area?  Justify your claim! (Hint: try imagining what happens to the 2D pattern that would create the cube.)


\end{problem}


\newpage


\begin{instructorNotes}

{\bf Main goal:} Students discover what happens to area and volume when we scale rectangles and rectangular prisms.

{\bf Overall picture:} It can help to bring manipulatives to help students visualize the scaling in this activity.


Scaling of other shapes will be covered in the next activity.  Your overall goal should be to develop the notion that lengths are scaled by $k$ (which we might call a ``linear scale factor'', adding the term ``linear'' to mean the exact same thing as we meant by scale factor in the similarity unit, while areas are scaled by $k^2$ (which we might call an ``area scale factor'') and volumes are scaled by $k^3$ (which we might call a ``volume scale factor'').  You should introduce these terms as it seems appropriate during your discussion.

Note: this should be the first time we are scaling anything other than length, so these concepts should be new for the students.  Before the activity, they would likely guess the area and volume scale factors incorrectly!

Problems 1 and 2:
\begin{itemize}
\item Students should be familiar with the concept of a scale factor from our work on similarity, and should hopefully have no trouble with the first part of this question.  
%If you'd like them, there are some copies of this rectangle in a small baggie in the supplies closet.

\item On (a), you might remind the students that our scale factor is also scaling all other lengths in this figure - so the diagonal is also scaled by a factor of five, etcetera.

\item In problem 3, students may have trouble with the concept of the area of the new rectangle, since they are not given the actual dimensions of the original rectangle.  The idea is to generalize the specific work in the first problem.

\item Problems 2 and 3 are more general versions of Problem 1. You might ask as a transition whether there was anything special about the numbers we are using, or whether we could use any kind of number we like.  (Decimals?  Negatives?  Zero?)
\end{itemize}

Problem 4:
\begin{itemize}
\item This should be one of the main takeaways of the activity, so don't be afraid to spend some time here.  The notion of scaling a unit of area will be the most useful for the next activity, where we discuss general shapes.  

\item The ``algebraic explanation'' we usually see is essentially $k L \times k W = k^2 LW$.  The ``geometric explanation'', based on their pictures, is that we now have $k^2$ copies of the original box.

\item 
Another type of example you might discuss around this activity is why a 12-inch pizza doesn't cost twice as much as a 6-inch pizza. This is related to the fact that the area scale factor isn't the same as the length scale factor!
\end{itemize}


Problems 5 and 6:
\begin{itemize}
\item Students will likely struggle with their drawings in this situation.  You could bring unit cubes to class for them to form the various solids, or we may find a video you can play or post on Carmen to show the smaller prism ``fitting'' in the larger prism.  

\item If you haven't introduced the new terminology for area and volume scale factors by this point, it's good to do so here.

\item Scaling a unit of volume is perhaps the most useful strategy for the future. However, if you have time, you can repeat each of the strategies used for area.
\end{itemize}


Problem 7:
\begin{itemize}
\item This problem is tough for students to see - since the area has moved out of the plane and into 3D space.  If you don't have time for this problem, don't worry about it - it should be covered by the end of the next activity.
\end{itemize}





{\bf Good language:} Students may start seeing some comparisons here to our conversions with area and volume. You can encourage this discussion if students find it helpful. 

At each stage, you may find that students need to be reminded about what they are scaling (area, volume, length, something else) and what dimension they are working in.

Students should notice that area is two-dimensional and our scale factor is $k^2$ while volume is three-dimensional and our scale factor is $k^3$. This isn't a coincidence, but a consequence of our rectangle area formula and box volume formula! If you have extra time, you can help students explore this connection.



{\bf Suggested timing:} Give students about 10 minutes to think about Problems 1--4, and then take about 20 minutes to discuss, including each perspective on scaling area. Then, give students about 5--8 minutes to think about problems 5--7, and use the remaining time to discuss. You may not get to talk about Problem 7, but we will see this idea elsewhere.





\end{instructorNotes}







\end{document}