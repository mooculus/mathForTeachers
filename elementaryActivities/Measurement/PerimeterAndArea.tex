\documentclass[noauthor,nooutcomes]{ximera}
%\documentclass{ximera}
\usepackage{gensymb}
\usepackage{tabularx}
\usepackage{mdframed}
\usepackage{pdfpages}
%\usepackage{chngcntr}

\let\problem\relax
\let\endproblem\relax

\newcommand{\property}[2]{#1#2}




\newtheoremstyle{SlantTheorem}{\topsep}{\fill}%%% space between body and thm
 {\slshape}                      %%% Thm body font
 {}                              %%% Indent amount (empty = no indent)
 {\bfseries\sffamily}            %%% Thm head font
 {}                              %%% Punctuation after thm head
 {3ex}                           %%% Space after thm head
 {\thmname{#1}\thmnumber{ #2}\thmnote{ \bfseries(#3)}} %%% Thm head spec
\theoremstyle{SlantTheorem}
\newtheorem{problem}{Problem}[]

%\counterwithin*{problem}{section}



%%%%%%%%%%%%%%%%%%%%%%%%%%%%Jenny's code%%%%%%%%%%%%%%%%%%%%

%%% Solution environment
%\newenvironment{solution}{
%\ifhandout\setbox0\vbox\bgroup\else
%\begin{trivlist}\item[\hskip \labelsep\small\itshape\bfseries Solution\hspace{2ex}]
%\par\noindent\upshape\small
%\fi}
%{\ifhandout\egroup\else
%\end{trivlist}
%\fi}
%
%
%%% instructorIntro environment
%\ifhandout
%\newenvironment{instructorIntro}[1][false]%
%{%
%\def\givenatend{\boolean{#1}}\ifthenelse{\boolean{#1}}{\begin{trivlist}\item}{\setbox0\vbox\bgroup}{}
%}
%{%
%\ifthenelse{\givenatend}{\end{trivlist}}{\egroup}{}
%}
%\else
%\newenvironment{instructorIntro}[1][false]%
%{%
%  \ifthenelse{\boolean{#1}}{\begin{trivlist}\item[\hskip \labelsep\bfseries Instructor Notes:\hspace{2ex}]}
%{\begin{trivlist}\item[\hskip \labelsep\bfseries Instructor Notes:\hspace{2ex}]}
%{}
%}
%% %% line at the bottom} 
%{\end{trivlist}\par\addvspace{.5ex}\nobreak\noindent\hung} 
%\fi
%
%


\let\instructorNotes\relax
\let\endinstructorNotes\relax
%%% instructorNotes environment
\ifhandout
\newenvironment{instructorNotes}[1][false]%
{%
\def\givenatend{\boolean{#1}}\ifthenelse{\boolean{#1}}{\begin{trivlist}\item}{\setbox0\vbox\bgroup}{}
}
{%
\ifthenelse{\givenatend}{\end{trivlist}}{\egroup}{}
}
\else
\newenvironment{instructorNotes}[1][false]%
{%
  \ifthenelse{\boolean{#1}}{\begin{trivlist}\item[\hskip \labelsep\bfseries {\Large Instructor Notes: \\} \hspace{\textwidth} ]}
{\begin{trivlist}\item[\hskip \labelsep\bfseries {\Large Instructor Notes: \\} \hspace{\textwidth} ]}
{}
}
{\end{trivlist}}
\fi


%% Suggested Timing
\newcommand{\timing}[1]{{\bf Suggested Timing: \hspace{2ex}} #1}




\hypersetup{
    colorlinks=true,       % false: boxed links; true: colored links
    linkcolor=blue,          % color of internal links (change box color with linkbordercolor)
    citecolor=green,        % color of links to bibliography
    filecolor=magenta,      % color of file links
    urlcolor=cyan           % color of external links
}

\title{Perimeter and Area}
\begin{document}
\begin{abstract} For this activity, you'll need a piece of string or yarn made into a loop. \end{abstract}
\maketitle



\begin{problem}
What is the perimeter of your string?  \end{problem}

\begin{problem} Use your same piece of string to make different {\bf rectangles}, and fill in the following chart.
\vskip 0.1in
\begin{tabular}{|p{2.2in}|p{2.2in}|}
\hline
\begin{center} Shape (sketch) \end{center} & \begin{center} Area \end{center}  \\ \hline
 & \\ [10ex] \hline
 & \\ [10 ex] \hline
 & \\ [10 ex] \hline
 & \\ [10 ex] \hline
 & \\ [10 ex] \hline
\end{tabular}
\begin{enumerate}

\item What is the largest possible area (if one exists) that you can make with this piece of string?
\item What is the smallest possible area (if one exists) that you can make with this same piece of string?
\end{enumerate}

%\begin{solution}
%\begin{enumerate}
%\item The largest possible area is that of a square, no matter the perimeter of your string.
%\item We can make any area larger than zero, so we could say that zero is the smallest area, even though we wouldn't have a rectangle anymore.
%\end{enumerate}
%\end{solution}
\end{problem}
\newpage



\begin{problem} Use your string to make different {\bf shapes}, and fill in the following chart.
\vskip 0.1in
\begin{tabular}{|p{2.2in}|p{2.2in}|}
\hline
\begin{center} Shape (sketch) \end{center} & \begin{center} Area \end{center}  \\ \hline
 & \\ [10ex] \hline
 & \\ [10 ex] \hline
 & \\ [10 ex] \hline
 & \\ [10 ex] \hline
 & \\ [10 ex] \hline
\end{tabular}


%\begin{tabular}{|p{1in}|p{1in}|p{1in}|p{1in}}
%\hline
%\begin{center} Angle (degrees) \end{center} & \begin{center} Angle (radians) \end{center} &
%\begin{center} $\cos \theta$  \end{center} &
%\begin{center} $\sin \theta$  \end{center}\\ \hline
% & \\ [10ex] \hline
% & \\ [10 ex] \hline
% & \\ [10 ex] \hline
% & \\ [10 ex] \hline
% & \\ [10 ex] \hline
%\end{tabular}
\begin{enumerate}

\item What is the largest possible area (if one exists) that you can make?
\item What is the smallest possible area (if one exists) that you can make?
\end{enumerate}

%\begin{solution}
%\begin{enumerate}
%\item The largest possible area is that of a circle, no matter the perimeter of your string.
%\item We can make any area larger than zero, so we could say that zero is the smallest area, even though we wouldn't have a rectangle anymore.
%\end{enumerate}
%\end{solution}

\end{problem}

\begin{problem} For problems 2 and 3, can every area between the largest and smallest be made?  Why or why not?

%\begin{solution}
% Yes, every area between the largest and the smallest can be made.  
%\end{solution}





\end{problem}


\newpage

\begin{instructorNotes}
{\bf Main goal:} We determine the largest possible area which can be made with a fixed perimeter. Students should be able to answer this question when we restrict to rectangles as well as in general.


{\bf Overall picture:}  Students conjecture a relationship between perimeter and area of first rectangles, and then shapes in general.  Students have the opportunity to use the formulae we've justified for various shapes and to think about how to calculate the area of a shape without already having the sides or heights labeled.

Problem 1 is included to make sure everyone is on the same page, and that everyone has roughly the same perimeter (so that their area calculations are relatively similar between groups). 

Problem 2: 
\begin{itemize}
\item Students are happy to work on the hands-on part of this activity.  They don't need much help with the first part about rectangles, and seem to come to a conjecture quickly.
\begin{enumerate}
\item Hand out the strings to be used as the ``perimeter'' - it's helpful if everyone's string is approximately the same length so that the students are working on the same problem.  Remind the students that they can use a ruler for this activity, and that it's much easier if you work with a partner.  One person can hold the string, and the other can measure.
	\item During the discussion, have students present enough examples that a trend is clear.  Have the students verbalize this trend.
	\item The two main points that we're trying to make are these: the largest area we can make with a fixed perimeter is a circle when we consider all shapes, or a square when we consider only rectangles.  All areas smaller than these maxima can also be made.  Particularly with rectangles, we can determine how to make a rectangle with fixed perimeter and area - this is just an algebra problem.

\end{enumerate}
\end{itemize}

Problem 3:

\begin{itemize}
\item   When allowed to make any 2D shapes, students sometimes struggle to calculate the area.  Suggesting they sketch the shape may help, or having them look back at the work they've done previously to calculate areas may also help.  

\item Students can get pretty creative with this!  Try to direct them to shapes where they can actually calculate the area, not just shapes where they have to estimate.

\item With an iPad, students might also be able to place their shape over some graph paper and count the units inside. While this is a nice strategy, you might ask them how this calculation is related to what they did with rectangles, or whether everyone using this method would get the same answer (probably not, since the grid gets scaled as they zoom in and out).

\item When it comes to seeing that the circle has the largest area, we are looking for an informal argument that as we ``push'' the sides of the shape farther and farther out, we include more area and we form more and more like a circle.  It's good to talk about taking the square and rounding out the corners, and it's also good to talk about an example that looks like a cookie with a bite taken out of it, where we have just removed some area by pushing in the perimeter a bit. We can reverse this by pushing the perimeter back out again.
\end{itemize}

Problem 4:
\begin{itemize}
	\item You might discuss with students that we need more powerful mathematics to actually prove that these give us the largest area for a fixed perimeter.  In this case, point out that we haven't checked all examples, but have only made a reasonable conjecture.
	\item We won't really ask students again about the smallest possible area, as the answer is a bit difficult to state.  Some students like to say that the minimum is zero.  You might discuss whether it's possible to actually have an area of zero in this case!
\end{itemize}

Wrap up by making sure that students understand the main idea: they should be able to state what shape gives them the largest area, and justify this with a class of examples and discussing a trend.


{\bf Good language:} It's good to remember that without calculus, the arguments here will be informal. That's okay -- but we still want students to have some reasoning behind their thinking! This is also a good time to bring up shearing again, where we discussed the difference between shearing and squashing. With squashing, we had a fixed perimeter but different areas. With shearing, we had a fixed area and different perimeters. Students may still be working to understand how the area can change when the perimeter is fixed!


\timing{About 10 minutes in groups, about 15 minutes discussion on Problems 1 and 2, and then about 10 minutes in groups for Problems 3 and 4, and the rest of the time in discussion. Be sure to have students present their work, especially with Problem 3. It can be good for both exploration problems to make the chart on the board and have students fill it in during the work time.}

\end{instructorNotes}


\end{document}