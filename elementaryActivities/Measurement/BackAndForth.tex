\documentclass[nooutcomes,noauthor]{ximera}
\usepackage{gensymb}
\usepackage{tabularx}
\usepackage{mdframed}
\usepackage{pdfpages}
%\usepackage{chngcntr}

\let\problem\relax
\let\endproblem\relax

\newcommand{\property}[2]{#1#2}




\newtheoremstyle{SlantTheorem}{\topsep}{\fill}%%% space between body and thm
 {\slshape}                      %%% Thm body font
 {}                              %%% Indent amount (empty = no indent)
 {\bfseries\sffamily}            %%% Thm head font
 {}                              %%% Punctuation after thm head
 {3ex}                           %%% Space after thm head
 {\thmname{#1}\thmnumber{ #2}\thmnote{ \bfseries(#3)}} %%% Thm head spec
\theoremstyle{SlantTheorem}
\newtheorem{problem}{Problem}[]

%\counterwithin*{problem}{section}



%%%%%%%%%%%%%%%%%%%%%%%%%%%%Jenny's code%%%%%%%%%%%%%%%%%%%%

%%% Solution environment
%\newenvironment{solution}{
%\ifhandout\setbox0\vbox\bgroup\else
%\begin{trivlist}\item[\hskip \labelsep\small\itshape\bfseries Solution\hspace{2ex}]
%\par\noindent\upshape\small
%\fi}
%{\ifhandout\egroup\else
%\end{trivlist}
%\fi}
%
%
%%% instructorIntro environment
%\ifhandout
%\newenvironment{instructorIntro}[1][false]%
%{%
%\def\givenatend{\boolean{#1}}\ifthenelse{\boolean{#1}}{\begin{trivlist}\item}{\setbox0\vbox\bgroup}{}
%}
%{%
%\ifthenelse{\givenatend}{\end{trivlist}}{\egroup}{}
%}
%\else
%\newenvironment{instructorIntro}[1][false]%
%{%
%  \ifthenelse{\boolean{#1}}{\begin{trivlist}\item[\hskip \labelsep\bfseries Instructor Notes:\hspace{2ex}]}
%{\begin{trivlist}\item[\hskip \labelsep\bfseries Instructor Notes:\hspace{2ex}]}
%{}
%}
%% %% line at the bottom} 
%{\end{trivlist}\par\addvspace{.5ex}\nobreak\noindent\hung} 
%\fi
%
%


\let\instructorNotes\relax
\let\endinstructorNotes\relax
%%% instructorNotes environment
\ifhandout
\newenvironment{instructorNotes}[1][false]%
{%
\def\givenatend{\boolean{#1}}\ifthenelse{\boolean{#1}}{\begin{trivlist}\item}{\setbox0\vbox\bgroup}{}
}
{%
\ifthenelse{\givenatend}{\end{trivlist}}{\egroup}{}
}
\else
\newenvironment{instructorNotes}[1][false]%
{%
  \ifthenelse{\boolean{#1}}{\begin{trivlist}\item[\hskip \labelsep\bfseries {\Large Instructor Notes: \\} \hspace{\textwidth} ]}
{\begin{trivlist}\item[\hskip \labelsep\bfseries {\Large Instructor Notes: \\} \hspace{\textwidth} ]}
{}
}
{\end{trivlist}}
\fi


%% Suggested Timing
\newcommand{\timing}[1]{{\bf Suggested Timing: \hspace{2ex}} #1}




\hypersetup{
    colorlinks=true,       % false: boxed links; true: colored links
    linkcolor=blue,          % color of internal links (change box color with linkbordercolor)
    citecolor=green,        % color of links to bibliography
    filecolor=magenta,      % color of file links
    urlcolor=cyan           % color of external links
}

\title{Back And Forth}

\begin{document}
\begin{abstract}\end{abstract}
\maketitle




\begin{problem}
    Jack and Jill are measuring the lengths of their desks.  The desks are identical, but Jack is using a piece of pencil lead to measure the length and Jill is using a necklace that she stretched out into a straight line.  Jack reports that his desk measures 20 units while Jill reports that her desk measures 5 units.
    \begin{enumerate}
        \item Jack claims that his desk must be bigger than Jill's, since his desk is 20 units long.  Is Jack correct?  Why or why not?
        \item Jill claims that her desk must be bigger than Jack's, since she used a bigger unit to measure the desk.  Is Jill correct?  Why or why not?
        \item What is the relationship between the two units used to measure the desks?  Justify your answer.
        \item Next, Jill measures the length of the chalkboard and finds it to be 13.5 units long.  When Jack measures the chalkboard, what will his answer be?  Justify your answer.  (Hint: some Math 1125 terminology should make its way into your explanation!)
    \end{enumerate}
    
%    \begin{solution}
%    \begin{enumerate}
%    \item Jack is not correct: the desks are identical.  Using a different unit can result in a different number for the length of the desk, even though the desks didn't change.
%    \item Jill is not correct: the desks are identical.  A larger unit just results in a smaller number of total units used, even thought the desks didn't change.
%    \item The relationship is multiplicative: there are four of Jack's unit for every one of Jill's.
%    \item Since there are four of Jack's unit for every one of Jill's, the chalkboard measures $13.5 \times 4 = 54$ of Jack's unit.
%    \end{enumerate}
%    \end{solution}

\end{problem} \vfill




\begin{problem}
    Jason knows that there are 3 feet in one yard.  He also knows that a football field is 100 yards long.  Since a yard is bigger than a foot, Jason thinks that a football field should be $100 \div 3 = 33 \frac{1}{3}$ feet long.
    \begin{enumerate}
    \item Explain Jason's reasoning.  What does he think we should do, and why did he come to that conclusion?
    \item What is the correct answer to this problem? Explain.
    \item How would you help Jason to understand his mistake?  Give at least two methods you could try.
    \end{enumerate}
    
%    \begin{solution}
%        \begin{enumerate}
%        \item Jason thinks we should divide by three since there are three feet in one yard.  Since feet are smaller than yards, and division (sometimes) makes things smaller, he reasons we should divide.
%        \item We should multiply by three in order to find the correct answer: $100 \times 3 = 300$ feet.
%        \item To help Jason understand his mistake, we could draw a picture of a football field, and divide each yard into feet.  We could then count the number of feet total to find the correct answer.  We could also return to the definition of multiplication and division, and ask Jason to identify groups and objects per group in this situation.
%        \end{enumerate}
%    \end{solution}
    
\end{problem} \vfill


\begin{problem}
These problems remind Mike of fraction comparison from 1125.  What might he be thinking?

%\begin{solution}
%This is a challenge/bonus question -- don't feel the need to discuss this one with the class unless many students are interested!  The idea is that more pieces in the denominator results in you needing more of those pieces to make the same amount.
%\end{solution}
\end{problem} \vfill

\newpage

\begin{instructorNotes}
{\bf Main goal:} Students reason to convert length measurements.

{\bf Overall picture:} The activity is designed to bring up some misconceptions about measurement conversion, as well as give students a chance to do conversions without their preconceived notions.   We do not want measurement conversion to turn in to an algebraic exercise! Students should be thinking about the measurement process as well as the meanings of the operations they are doing. We expect to identify groups and objects per group, here! We will return to measurement conversion after we discuss area and volume, and again in that regard we'd like to see students drawing pictures and reasoning rather than using dimensional analysis and ``cancelling units'' in a fence-post chart. 

Jack and Jill:
\begin{itemize}
\item     Here students confront two basic misconceptions about measurement conversion, and then convert using non-standard units of length.  The misconceptions are common amongst children - even though the desks are identical, getting different numbers for the measurement can be confusing to children.  Have students explain the misconceptions clearly for the class.  The question about the relationship between the units is intentionally vague.  Students can relate this process to several topics covered in Math 1125, including multiplication and ratios.
\end{itemize}


Jason:

\begin{itemize}
\item     Answers will probably vary for students' ideas about Jason's reasoning.  One important underlying misconception to discuss is that of ``multiplication makes things larger, and division makes things smaller''.  This will hopefully come up naturally in discussion, as it is one reason that Jason might come to this conclusion.  
    
\item    Students should try to find as many ways as they can to help Jason with his problem - including drawing accurate pictures (perhaps with a smaller number of yards).  Drawing good pictures is, as usual, a very age-appropriate task for grades K-3.
\item Students should be able to answer which type of division they are using here!
\end{itemize}



{\bf Suggested timing:} Give students about 20 minutes to think about the problems, and spend the rest of the time with student presentations and discussion.




\end{instructorNotes}


\end{document}