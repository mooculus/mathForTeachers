%\documentclass[handout]{ximera}
\documentclass[nooutcomes,noauthor]{ximera}
\usepackage{gensymb}
\usepackage{tabularx}
\usepackage{mdframed}
\usepackage{pdfpages}
%\usepackage{chngcntr}

\let\problem\relax
\let\endproblem\relax

\newcommand{\property}[2]{#1#2}




\newtheoremstyle{SlantTheorem}{\topsep}{\fill}%%% space between body and thm
 {\slshape}                      %%% Thm body font
 {}                              %%% Indent amount (empty = no indent)
 {\bfseries\sffamily}            %%% Thm head font
 {}                              %%% Punctuation after thm head
 {3ex}                           %%% Space after thm head
 {\thmname{#1}\thmnumber{ #2}\thmnote{ \bfseries(#3)}} %%% Thm head spec
\theoremstyle{SlantTheorem}
\newtheorem{problem}{Problem}[]

%\counterwithin*{problem}{section}



%%%%%%%%%%%%%%%%%%%%%%%%%%%%Jenny's code%%%%%%%%%%%%%%%%%%%%

%%% Solution environment
%\newenvironment{solution}{
%\ifhandout\setbox0\vbox\bgroup\else
%\begin{trivlist}\item[\hskip \labelsep\small\itshape\bfseries Solution\hspace{2ex}]
%\par\noindent\upshape\small
%\fi}
%{\ifhandout\egroup\else
%\end{trivlist}
%\fi}
%
%
%%% instructorIntro environment
%\ifhandout
%\newenvironment{instructorIntro}[1][false]%
%{%
%\def\givenatend{\boolean{#1}}\ifthenelse{\boolean{#1}}{\begin{trivlist}\item}{\setbox0\vbox\bgroup}{}
%}
%{%
%\ifthenelse{\givenatend}{\end{trivlist}}{\egroup}{}
%}
%\else
%\newenvironment{instructorIntro}[1][false]%
%{%
%  \ifthenelse{\boolean{#1}}{\begin{trivlist}\item[\hskip \labelsep\bfseries Instructor Notes:\hspace{2ex}]}
%{\begin{trivlist}\item[\hskip \labelsep\bfseries Instructor Notes:\hspace{2ex}]}
%{}
%}
%% %% line at the bottom} 
%{\end{trivlist}\par\addvspace{.5ex}\nobreak\noindent\hung} 
%\fi
%
%


\let\instructorNotes\relax
\let\endinstructorNotes\relax
%%% instructorNotes environment
\ifhandout
\newenvironment{instructorNotes}[1][false]%
{%
\def\givenatend{\boolean{#1}}\ifthenelse{\boolean{#1}}{\begin{trivlist}\item}{\setbox0\vbox\bgroup}{}
}
{%
\ifthenelse{\givenatend}{\end{trivlist}}{\egroup}{}
}
\else
\newenvironment{instructorNotes}[1][false]%
{%
  \ifthenelse{\boolean{#1}}{\begin{trivlist}\item[\hskip \labelsep\bfseries {\Large Instructor Notes: \\} \hspace{\textwidth} ]}
{\begin{trivlist}\item[\hskip \labelsep\bfseries {\Large Instructor Notes: \\} \hspace{\textwidth} ]}
{}
}
{\end{trivlist}}
\fi


%% Suggested Timing
\newcommand{\timing}[1]{{\bf Suggested Timing: \hspace{2ex}} #1}




\hypersetup{
    colorlinks=true,       % false: boxed links; true: colored links
    linkcolor=blue,          % color of internal links (change box color with linkbordercolor)
    citecolor=green,        % color of links to bibliography
    filecolor=magenta,      % color of file links
    urlcolor=cyan           % color of external links
}

\title{Try This On For Size}

\begin{document}
\begin{abstract}
Your instructor should hand out some paper clips to use for this activity.
\end{abstract}
\maketitle

\begin{problem}
\begin{enumerate}
\item Measure the line below using your paperclip. 
\item Explain your procedure to a partner and why your answer makes sense. What is the unit you used? Be as specific as possible. What steps did you take with your unit, and why?
\item What does your answer mean?
\end{enumerate}
\begin{center}
	\begin{tikzpicture}
		\draw[thick] (0,0)--(12,2);
	\end{tikzpicture}
\end{center}
\end{problem}

\vfill

\begin{problem}
\begin{enumerate}
\item Measure the line below using your paperclip. 
\item Explain your procedure to a partner and why your answer makes sense. What is the unit you used? Be as specific as possible. What steps did you take with your unit, and why?
\item What does your answer mean?
\end{enumerate}
\begin{center}
	\begin{tikzpicture}
		\draw[thick] plot [smooth] coordinates{(0,0) (3, -1) (8,3) (12,2)};
	\end{tikzpicture}
\end{center}
%\begin{enumerate}
%\item By yourself, draw a line that is 6 cm long.  Next, write down detailed instructions for drawing the line that could be understood by someone who has not yet learned how to measure or how to use a ruler.  \item Find a partner, and have them follow your instructions exactly.  Did they draw the same line you drew?  Is their line 6 cm long?  
%%\item Repeat the above process, except this time choose a unit of length that is not standard.  (No centimeters, inches, etc!)  Can you use the same process you used in the first part to draw a line of length 6 units?
%\item Compare this process to the process you used to measure length without a ruler. How are they alike? different?
%\item With your partner, use your experiences in the previous question to write down detailed instructions for measuring the length of an object.  You may use 1 cm as a unit, but your procedure should also work for any other appropriate unit.
%\end{enumerate}
%
%\begin{solution}
%Students' instructions should be detailed, but easy to follow.  One way for them to do this is to put down their ruler, start at the mark called 1cm, and draw a line that ends at the mark called 7cm.  (Ask them: Why is this a valid way to measure 6cm?)
%\end{solution}
\end{problem}
\vfill 
\begin{problem}
Summarize your work on the last two problems. What is the procedure for measuring something?
\end{problem}
\begin{problem}
Bonus question: is using a ruler the same as what we have just done, or different? Explain your thinking.
\end{problem}
\pagebreak
\begin{instructorNotes}

{\bf Main goal:} We develop the general process of measuring: (1) choose an aspect to measure (2) choose an appropriate unit, considering the dimension (3) iterate your unit all over the object (4) report your results.


{\bf Overall picture}

The first two questions give students the experience of measuring a length using a non-standard unit, and the third question asks students to essentially generate the four step process above. Encourage students to be specific about their process of measuring when they describe their steps to the class. For instance, when students choose their own units, you should discuss that many of the units they choose would be better {\em area} units than length units. For instance, when we use a paper clip to measure length, we are using the length of the paper clip, not the full clip.  This can get very confusing both for our students and children!

In your discussion, compare and contrast the first two questions. Could we use the same procedure on the non-straight line as we did on the straight line? You may also want to point out that since our paper clip is rigid, we can only estimate the length of the non-straight line. You could additionally ask how students could get a more exact answer for this question (perhaps using a string of the same length as the non-straight line, so it could be arranged into a straight line with the same length).


The third question is provided in order for students to organize their thoughts from the first two questions.  You should conclude your discussion by emphasizing the iterative nature of measuring length, and that we must use a 1-dimensional unit to measure 1-dimensional length.  


If you have time to discuss the ``Bonus question'', you should discuss the ways in which we can see that a ruler is already an iterative tool, or discuss the way that we might build a ruler in the first place. How could we make a ruler using paper clips? We would glue a bunch of them together, or mark a bunch of paper clip lengths on another stick. You may also want to bring up a few misconceptions that children often have when using a ruler.  See Beckmann's Activity 11C.  You could draw a ruler on the board, draw a line from 2cm to 8cm, and then report that this line is 8cm long, or 7cm long. You could also start measuring at 1cm instead of at 0cm, or at the beginning of a ruler which has a little gap before the 0cm mark.



{\bf Good language:} Continue to help students be specific with their descriptions, both in writing instructions as well as describing their units. 



{\bf Suggested timing:} Give students about 10 minutes to think about the first three problems. Have students present the first two problems, then give them another minute or two to re-think their answers to problem 3 if they want. Then discuss together, highlighting the four-step process at the end. If you have more time, give students about 5 minutes to think about the final problem, and use the remaining time to discuss.



\end{instructorNotes}


\end{document}


%\begin{problem}
%A {\it square inch} is a particular unit of area which looks like a square measuring 1 inch on each side.  By yourself, draw an object which has an area of 8 square inches, but is {\bf not} a rectangle.  After you're done, compare your answer with your partner's answer.  How many different answers are possible to this question?
%
%\begin{solution}    
%There are actually infinitely many different ways to draw such a shape!
%\end{solution}
%\begin{instructorNotes}
%Students will have to be a little creative in order to break out of the rectangle mindset, here.  Have students draw as many different shapes as possible, which will hopefully include some shapes built from pieces of the units.  For instance, if someone cut up the 8 units into two equal pieces each, and arranged these smaller pieces.
%\end{instructorNotes}
%\end{problem}
%
%\begin{problem}
%Use your experiences in the previous question to write down a procedure for measuring the area of an object.  You may use 1 square inch as a unit, but your procedure should work for any other appropriate unit.  
%
%\begin{solution}
%To measure area, you count the number of area units it takes to exactly cover your shape, with no gaps or overlaps between the units.
%\end{solution}
%\begin{instructorNotes}
%This discussion should proceed in a similar fashion to that about length.  Students should be encouraged again to see the iterative process of measuring the area, and should feel like measuring area is basically the same as measuring length, except in two dimensions instead of one.
%\end{instructorNotes}
%\end{problem}
%
%\begin{problem}
%Extend your ideas to the case of volume.  How would we measure volume?  
%\end{problem}
%
%\begin{problem}
%How are these three cases related?  How are they different?
%%\end{problem}
%
%\begin{solution}
%To measure volume, you count the number of volume units it takes to exactly fill your 3D figure, without any gaps or overlaps between the units.
%\end{solution}
%\begin{instructorNotes}
%Hopefully, students should be catching on by this point.  If you don't have time to discuss this exercise with the whole class, point out as you conclude your discussion that we are really doing the same process in each case - using the unit iteratively.  You should point out that we have only used whole units thus far, but there is nothing stopping us from using partial units.  We'll get to that in a few more activities.
%\end{instructorNotes}
%\end{problem}
%\pagebreak
%THIS second page is in the packet of files to print - and is called Measurement - Area (problems 1 and 2) - so we can keep these as two activities.
%\begin{problem}
%%\begin{enumerate}
%%\item Carefully draw a rectangle.  Then compare your rectangle to your partner's.  Which rectangle has more area?  How can you tell?
%%\vskip 2in
%
% The picture below shows two rectangles. Which rectangle has more area?  How can you tell?  Discuss with your partner.  
%%\end{enumerate}
%
%\[
%\includegraphics[height=3in]{graphics/2rectangles.png}
%\]
%\end{problem}
%
%\pagebreak
%
%\begin{problem}
%Consider the following two shapes.  
%\[
%\includegraphics[height=4in]{graphics/ComparingArea.png}
%\]
%(idea and picture taken from Battista (1982) ``Understanding Area and Area Formula" in \underline{The Mathematics Teacher}.)
%
%   
%\begin{enumerate}
%    \item Which has more area?  What are some methods for deciding?  
%    \vskip 1in
% 
%\item Choose one of your methods and carry it out.  Make a note of your process.
%%\vskip 3in
%
%
%\end{enumerate}
%\end{problem}
%
%
%
%
%
%
%
%
%\end{problem}

