\documentclass[nooutcomes,noauthor]{ximera}


\graphicspath{
  {./}
  {graphics/}
  {../graphics/}
}

\usepackage{chngcntr}

\let\question\relax
\let\endquestion\relax




\newtheoremstyle{SlantTheorem}{\topsep}{\fill}%%% space between body and thm
%\newtheoremstyle{SlantTheorem}{\topsep}{\topsep}%%% space between body and thm
 {\slshape}                      %%% Thm body font
 {}                              %%% Indent amount (empty = no indent)
 {\bfseries\sffamily}            %%% Thm head font
 {}                              %%% Punctuation after thm head
 {3ex}                           %%% Space after thm head
 {\thmname{#1}\thmnumber{ #2}\thmnote{ \bfseries(#3)}}%%% Thm head spec
\theoremstyle{SlantTheorem}
\newtheorem{question}{Question}
\counterwithin*{question}{section}



\let\instructorNotes\relax
\let\endinstructorNotes\relax
%%% instructorNotes environment
\ifhandout
\newenvironment{instructorNotes}[1][false]%
{%
\def\givenatend{\boolean{#1}}\ifthenelse{\boolean{#1}}{\begin{trivlist}\item}{\setbox0\vbox\bgroup}{}
}
{%
\ifthenelse{\givenatend}{\end{trivlist}}{\egroup}{}
}
\else
\newenvironment{instructorNotes}[1][false]%
{%
  \ifthenelse{\boolean{#1}}{\begin{trivlist}\item[\hskip \labelsep\bfseries {\Large Instructor Notes: \\} \hspace{\textwidth} ]}
{\begin{trivlist}\item[\hskip \labelsep\bfseries {\Large Instructor Notes: \\} \hspace{\textwidth} ]}
{}
}
{\end{trivlist}}
\fi


%% Suggested Timing
\newcommand{\timing}[1]{{\bf Suggested Timing: \hspace{2ex}} #1}

\title{Circle the wagons}

\begin{document}
\begin{abstract}
\end{abstract}
\maketitle



\begin{problem}
On a separate sheet of paper, draw a large circle with your compass.  Cut this circle out, then outline the circumference using a thick line.  Cut up your circle into at least 8 pieces, and then rearrange these pieces into another shape that resembles one of the shapes we have already studied.  Remember, your goal is to calculate the area of the circle.


\end{problem}

\begin{problem}
Repeat the previous question, making the same shape you did in Problem 1.  This time, cut your circle into at least 16 pieces.


\end{problem}

\begin{problem}
Repeat the previous question, making the same shape you did in Problem 1.  This time, cut your circle into at least 32 pieces.
\end{problem}

\begin{problem}
What would happen if you cut your circle into infinitely many pieces?  (Don't try this at home!)  How does this help you calculate the area of the circle?  Why does it make sense that any circle with radius $r$ units has area $\pi r^2$ square units?


\end{problem}


\newpage

\begin{instructorNotes}
{\bf Main goal:} We justify the area formula for circles.  


{\bf Overall picture:} After this activity, students should be permitted to use this formula without justification, unless asked (on a quiz or test) to give a justification.  

Problem 1--3: 
\begin{itemize}
\item This question is intentionally vague.  See the timing section to help students along with a hint.

\item Make sure the students are getting the idea that they are creating a shape that resembles a rectangle or parallelogram before you move on to the next two questions.  Students might have trouble making the area connection at this stage, since we don't yet have a rectangle.

\item Students should have no trouble with Problems 2 and 3 once they understand the first question.  These questions are to help with the limiting idea in the final question.

\end{itemize}

Problem 4:
\begin{itemize}
\item Here we discuss the limit process that helps us find the area of a circle.  Students are usually confused about this limit, the idea of cutting the circle into infinitely many pieces, and how we actually form a rectangle (not a parallelogram) in the limit.  You will need to go slowly over these concepts.


\item Students can also have some trouble with the circumference of the circle turning into the top and bottom of the rectangle, so that the width of the rectangle is half the circumference.  It can also be difficult for students to see that the radius is the height of the rectangle (especially if they are seeing a parallelogram).  Going over the algebra more than once is usually a good idea.  Another problem may occur with the assumed knowledge that the circumference of a circle is $2\pi r$  units (especially if you have mostly focused on $C = \pi \times D$).  

\item Students can also be confused about the fact that the height of the parallelogram turns into the height of the rectangle here. You may need to remind the students that this isn't the same as shearing!

\item If you have extra time and your students are very strong, you might discuss the similarity of all circles (Like similarity of a regular polygon of a given number of sides yields the same ratio of any two lengths - particularly the ratio of the perimeter to the length of a diagonal through the center).  You might discuss similarity further: the ratio of any two lengths should stay the same.  In particular, with a regular $n$-gon, the ratio of the perimeter to the length of a diagonal through the center stays the same.  This may perhaps give us another way to make estimates when we take the limit of such regular polygons. You could also return to this example after we have talked in class about similarity.
\end{itemize}

Wrap up by highlighting the overall idea: we have again used moving and additivity to change a figure whose area we didn't know how to calculate into a figure whose are we do know how to calculate. Students can use this area formula going forward unless they are asked to justify it.



\timing{This activity should take the whole class period. Give the students 5 minutes or so to draw their circles and think about the problem.  After this, you can give them a hint that they might consider ``pie-shaped wedges", and then after a few more minutes suggest they might try to form a rectangle if they still haven't figured it out. You can pause for presentations once most students have figured out Problem 1, just to make sure everyone is on the same page, and then return to group work.  Once most students have worked on the limiting ideas of Problem 4, use the rest of the time to present and discuss the results.}
\end{instructorNotes}




\end{document}

