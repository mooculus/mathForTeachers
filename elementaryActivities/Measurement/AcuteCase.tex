%\documentclass{ximera}
\documentclass[nooutcomes,noauthor]{ximera}

\graphicspath{
  {./}
  {graphics/}
  {../graphics/}
}

\usepackage{chngcntr}

\let\question\relax
\let\endquestion\relax




\newtheoremstyle{SlantTheorem}{\topsep}{\fill}%%% space between body and thm
%\newtheoremstyle{SlantTheorem}{\topsep}{\topsep}%%% space between body and thm
 {\slshape}                      %%% Thm body font
 {}                              %%% Indent amount (empty = no indent)
 {\bfseries\sffamily}            %%% Thm head font
 {}                              %%% Punctuation after thm head
 {3ex}                           %%% Space after thm head
 {\thmname{#1}\thmnumber{ #2}\thmnote{ \bfseries(#3)}}%%% Thm head spec
\theoremstyle{SlantTheorem}
\newtheorem{question}{Question}
\counterwithin*{question}{section}



\let\instructorNotes\relax
\let\endinstructorNotes\relax
%%% instructorNotes environment
\ifhandout
\newenvironment{instructorNotes}[1][false]%
{%
\def\givenatend{\boolean{#1}}\ifthenelse{\boolean{#1}}{\begin{trivlist}\item}{\setbox0\vbox\bgroup}{}
}
{%
\ifthenelse{\givenatend}{\end{trivlist}}{\egroup}{}
}
\else
\newenvironment{instructorNotes}[1][false]%
{%
  \ifthenelse{\boolean{#1}}{\begin{trivlist}\item[\hskip \labelsep\bfseries {\Large Instructor Notes: \\} \hspace{\textwidth} ]}
{\begin{trivlist}\item[\hskip \labelsep\bfseries {\Large Instructor Notes: \\} \hspace{\textwidth} ]}
{}
}
{\end{trivlist}}
\fi


%% Suggested Timing
\newcommand{\timing}[1]{{\bf Suggested Timing: \hspace{2ex}} #1}
\title{Acute case}

\begin{document}
\begin{abstract}
\end{abstract}

\maketitle

Our second type of triangle is an acute triangle.

\begin{problem}
Use a moving and additivity strategy to find a formula for the area of the triangle below in terms of $b$, the length of its base, and $h$, the length of its height. If you move and reattach pieces of the triangle or make copies of the triangle, you should prove that they fit exactly as you claim.

\begin{image}
\begin{tikzpicture}
\draw[thick] (0,0)--(4,0)--(3,2)--(0,0);
\node[below] at (2,0) {$b$};
\draw[thick, dashed] (3,0)--(3,2);
\node[left] at (3,1) {$h$};
\end{tikzpicture}
\end{image}
\end{problem} \vfill

\begin{problem}
Use the rectangle, the area formula for right triangles, and an algebraic strategy  to find a formula for the area of the triangle below in terms of $b$, the length of its base, and $h$, the length of its height. If you need to label other unknown parts of the figure you can introduce additional letters, but you should simplify your final formula to be in terms of only $b$ and $h$.

\begin{image}
\begin{tikzpicture}
\draw[thick] (0,0)--(4,0)--(3,2)--(0,0);
\node[below] at (2,0) {$b$};
\draw[thick, dashed] (3,0)--(3,2);
\node[left] at (3,1) {$h$};
\draw[thick, dotted] (0,0) rectangle (4,2);
\end{tikzpicture}
\end{image}
\end{problem} \vfill

\newpage
\begin{problem}
Use a shearing strategy and the area formula for right triangles to find a formula for the area of the triangle below in terms of $b$, the length of its base, and $h$, the length of its height. Be sure to clearly explain your shearing process and draw what happens to at least one of the strips.

\begin{image}
\begin{tikzpicture}
\draw[thick] (0,0)--(4,0)--(3,2)--(0,0);
\node[below] at (2,0) {$b$};
\draw[thick, dashed] (3,0)--(3,2);
\node[left] at (3,1) {$h$};
\end{tikzpicture}
\end{image}
\end{problem}



\newpage

\begin{instructorNotes} 

{\bf Main goal:} We begin to work towards the area formula for triangles using right triangles. We will spend three days on this formula: one for right triangles, one for acute triangles, and one for obtuse triangles.

{\bf Overall picture:} 
\begin{itemize}
	\item For the moving and additivity strategy, students should either cut the height in half or cut the base in half.  By allowing the students to come up with their own ideas, we hope that the groups begin with different ideas and can present them. Look for each of these options as you walk around the classroom and try to have both ideas presented.
	\item With the moving and additivity strategies, we will need to verify both that the rectangle we form has four $90$ degree angles as well as that it has four sides. Students may need convincing that it's not obvious that the assembled figure has four sides instead of five. Demonstrating by actually cutting out the triangle can be helpful.
	\item For the algebraic strategy, we want students to recognize that the rectangle is made out of the acute triangle and two different right triangles. They will need to identify the bases of the right triangles as $x$ and $b-x$ and then simplify the final formula to remove the $x$. This will be challenging for many of our students.
	\item For the shearing strategy, students should recognize that the height does not change and so we can shear the triangle into a right triangle and use the previous formula. 
	\item At the end of the activity, we will want to let students know that they should be able to prove the area formula for acute triangles using at least one of these strategies - all three are not required.
\end{itemize}


{\bf Good language:} Since we have recently learned about the moving and additivity principles, it's good to emphasize when and how we are using those principles here. Continue to help students recognize the difference between the base and the length of the base. Continue to remind the students that the meaning of area is the amount of 2D space an object takes up (and we measure it by covering with area units) any time it makes sense to do so!



{\bf Suggested timing:} Give students about $20$ minutes to work through the strategies here. Use the remaining time to present and discuss.



\end{instructorNotes}



\end{document}