\documentclass[nooutcomes,noauthor]{ximera}

\graphicspath{
  {./}
  {graphics/}
  {../graphics/}
}

\usepackage{chngcntr}

\let\question\relax
\let\endquestion\relax




\newtheoremstyle{SlantTheorem}{\topsep}{\fill}%%% space between body and thm
%\newtheoremstyle{SlantTheorem}{\topsep}{\topsep}%%% space between body and thm
 {\slshape}                      %%% Thm body font
 {}                              %%% Indent amount (empty = no indent)
 {\bfseries\sffamily}            %%% Thm head font
 {}                              %%% Punctuation after thm head
 {3ex}                           %%% Space after thm head
 {\thmname{#1}\thmnumber{ #2}\thmnote{ \bfseries(#3)}}%%% Thm head spec
\theoremstyle{SlantTheorem}
\newtheorem{question}{Question}
\counterwithin*{question}{section}



\let\instructorNotes\relax
\let\endinstructorNotes\relax
%%% instructorNotes environment
\ifhandout
\newenvironment{instructorNotes}[1][false]%
{%
\def\givenatend{\boolean{#1}}\ifthenelse{\boolean{#1}}{\begin{trivlist}\item}{\setbox0\vbox\bgroup}{}
}
{%
\ifthenelse{\givenatend}{\end{trivlist}}{\egroup}{}
}
\else
\newenvironment{instructorNotes}[1][false]%
{%
  \ifthenelse{\boolean{#1}}{\begin{trivlist}\item[\hskip \labelsep\bfseries {\Large Instructor Notes: \\} \hspace{\textwidth} ]}
{\begin{trivlist}\item[\hskip \labelsep\bfseries {\Large Instructor Notes: \\} \hspace{\textwidth} ]}
{}
}
{\end{trivlist}}
\fi


%% Suggested Timing
\newcommand{\timing}[1]{{\bf Suggested Timing: \hspace{2ex}} #1}

\title{New units}

\begin{document}
\begin{abstract}
\end{abstract}
\maketitle




\begin{problem}
Here is an $L$-shaped unit, shaped with two vertical blocks and one horizontal block.
\begin{center}
\begin{tikzpicture}
\draw[thick, fill=yellow] (1,1)--(2,1)--(2,2)--(3,2)--(3,3)--(1,3)--(1,1);
\draw[gray] (0,0) grid (4,4);
\end{tikzpicture}
\end{center}

Explain how to measure the three shapes below using this unit of area.
%Shape A is 8 squares, shape B is 12 squares, shape C is 7 squares
\begin{image}
\begin{tikzpicture}
\draw[thick, fill=red] (2,2)--(6,2)--(6,3)--(5,3)--(5,4)--(1,4)--(1,3)--(2,3)--(2,2);
\draw[thick, fill=cyan] (9, 2) rectangle (13, 5);
\draw[thick, fill=orange] (16, 3)--(18,3)--(18,1)--(19,1)--(19,3)--(20,3)--(20,4)--(19,4)--(19,5)--(18,5)--(18,4)--(16,4)--(16,3);
\draw[gray] (0,0) grid (21, 6);
\node at (0.5, 5.5) {$A$};
\node at (8.5, 5.5) {$B$};
\node at (15.5, 5.5) {$C$};
\end{tikzpicture}
\end{image}

%\begin{solution}
%    Shape A: $2 \frac{2}{3}$ units\\
%    Shape B: $4$ units\\
%    Shape C: $2 \frac{1}{3}$ units
%\end{solution}


\end{problem}

\begin{problem}
Using our unit of area, find an approximate measurement for the area of this circle.

\begin{image}
\begin{tikzpicture}
\draw[gray] (-5, -5) grid (5,5);
\draw[ultra thick] (0,0) circle (4cm);
\end{tikzpicture}
\end{image}

%\begin{solution}
%    The area is around 17 units.
%\end{solution}

\end{problem}

\newpage


\begin{instructorNotes}

{\bf Main goal:} We wrap up our work on the basic ideas of measuring area.

{\bf Overall picture:} Some of these problems may feel repetitive. Our goal is to finish discussing any misconceptions the students still have as well as give them a chance to attack similar problems with their new understanding.


Problem 1:
\begin{itemize}
\item    On \#2 and \#3, students might think about the question in various ways.  They can be a little unsure whether they are allowed to break up the unit into smaller pieces - you might ask whether they can do this with our usual units.  Some students count square units and divide by three to find the area, while others try to count copies of the unit inside the given shape.  You should discuss both of these solutions, and why they both give us the correct area.
    
\item    Some students will be confused as to whether they should write ``units'' or ``square units''.  This is a good time to revisit this discussion with students - the unit in this case measures area, so we can just say ``units''.  A ``square unit'' is a special kind of area unit, and we sometimes abbreviate this as ``sq. un.'' or ``un$^2$'', though the first is preferred over the second in this course.
\end{itemize}


Problem 2:
\begin{itemize}
\item     Most students do well with this problem after the discussion of problem 2.  The most common methods are to count the total number of squares and divide by three, or to outline as many units as possible, and then try to collect leftover pieces.  Occasionally, students will try other strategies (which are great to discuss) such as measuring the diameter of the circle and using a previously-known formula for area.  These ideas can lead to some pretty interesting discussions!
    
\item     Common questions from students: (1) Can we break up the unit?  You could ask students again: what does it mean to measure with this unusual unit?  What if it were squares instead? (2) Can we use the formula for the area of a circle?  You could ask students: what would that mean in this case, and how would it relate to the unit we're using?
\end{itemize}

{\bf Suggested timing:} We have a full class period for this activity. Give students 10 minutes to work through the problems, and then have them present their solutions. Afterwards, highlight the overall area measurement process to wrap up. You can use any extra time for following up on misconceptions raised over the last few class periods.


\end{instructorNotes}


\end{document}