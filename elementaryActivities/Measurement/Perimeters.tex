\documentclass{ximera}
\usepackage{gensymb}
\usepackage{tabularx}
\usepackage{mdframed}
\usepackage{pdfpages}
%\usepackage{chngcntr}

\let\problem\relax
\let\endproblem\relax

\newcommand{\property}[2]{#1#2}




\newtheoremstyle{SlantTheorem}{\topsep}{\fill}%%% space between body and thm
 {\slshape}                      %%% Thm body font
 {}                              %%% Indent amount (empty = no indent)
 {\bfseries\sffamily}            %%% Thm head font
 {}                              %%% Punctuation after thm head
 {3ex}                           %%% Space after thm head
 {\thmname{#1}\thmnumber{ #2}\thmnote{ \bfseries(#3)}} %%% Thm head spec
\theoremstyle{SlantTheorem}
\newtheorem{problem}{Problem}[]

%\counterwithin*{problem}{section}



%%%%%%%%%%%%%%%%%%%%%%%%%%%%Jenny's code%%%%%%%%%%%%%%%%%%%%

%%% Solution environment
%\newenvironment{solution}{
%\ifhandout\setbox0\vbox\bgroup\else
%\begin{trivlist}\item[\hskip \labelsep\small\itshape\bfseries Solution\hspace{2ex}]
%\par\noindent\upshape\small
%\fi}
%{\ifhandout\egroup\else
%\end{trivlist}
%\fi}
%
%
%%% instructorIntro environment
%\ifhandout
%\newenvironment{instructorIntro}[1][false]%
%{%
%\def\givenatend{\boolean{#1}}\ifthenelse{\boolean{#1}}{\begin{trivlist}\item}{\setbox0\vbox\bgroup}{}
%}
%{%
%\ifthenelse{\givenatend}{\end{trivlist}}{\egroup}{}
%}
%\else
%\newenvironment{instructorIntro}[1][false]%
%{%
%  \ifthenelse{\boolean{#1}}{\begin{trivlist}\item[\hskip \labelsep\bfseries Instructor Notes:\hspace{2ex}]}
%{\begin{trivlist}\item[\hskip \labelsep\bfseries Instructor Notes:\hspace{2ex}]}
%{}
%}
%% %% line at the bottom} 
%{\end{trivlist}\par\addvspace{.5ex}\nobreak\noindent\hung} 
%\fi
%
%


\let\instructorNotes\relax
\let\endinstructorNotes\relax
%%% instructorNotes environment
\ifhandout
\newenvironment{instructorNotes}[1][false]%
{%
\def\givenatend{\boolean{#1}}\ifthenelse{\boolean{#1}}{\begin{trivlist}\item}{\setbox0\vbox\bgroup}{}
}
{%
\ifthenelse{\givenatend}{\end{trivlist}}{\egroup}{}
}
\else
\newenvironment{instructorNotes}[1][false]%
{%
  \ifthenelse{\boolean{#1}}{\begin{trivlist}\item[\hskip \labelsep\bfseries {\Large Instructor Notes: \\} \hspace{\textwidth} ]}
{\begin{trivlist}\item[\hskip \labelsep\bfseries {\Large Instructor Notes: \\} \hspace{\textwidth} ]}
{}
}
{\end{trivlist}}
\fi


%% Suggested Timing
\newcommand{\timing}[1]{{\bf Suggested Timing: \hspace{2ex}} #1}




\hypersetup{
    colorlinks=true,       % false: boxed links; true: colored links
    linkcolor=blue,          % color of internal links (change box color with linkbordercolor)
    citecolor=green,        % color of links to bibliography
    filecolor=magenta,      % color of file links
    urlcolor=cyan           % color of external links
}

\title{Perimeters}
\begin{document}
\begin{abstract}  \end{abstract}
\maketitle



\begin{problem}
Harry thinks that the perimeter of the figure below is 16 sq. cm. Why do you think Harry thinks this? Use the meaning of perimeter and the four-step process for measurement to help Harry find the correct perimeter of this shape.
\begin{center}
\begin{tikzpicture}
\draw[gray] (0,0) grid (8,6);
\draw[ultra thick] (1,1)--(7,1)--(7,5)--(1,5)--(1,1);
\end{tikzpicture}
\end{center}
\end{problem}
\newpage

\begin{problem}
Iris thinks that the perimeter of the figure below is 18cm. Why do you think Iris thinks this? Is the actual perimeter more or less than 18cm? Use the meaning of perimeter and the four-step process for measurement to say how you know.
\begin{center}
\begin{tikzpicture}
\draw[gray] (0,0) grid (9,7);
\draw[ultra thick] (1,1)--(5,1)--(8,6)--(4,6)--(1,1);
\end{tikzpicture}
\end{center}
\end{problem}

\begin{problem}
Gray thinks that it's impossible to have a perimeter of $12.5$ cm, because ``the perimeter has to be on the lines''. What do you think Gray means by this? How could you show them a perimeter of $12.5$cm? Draw pictures to illustrate your thinking, and explain using the meaning of perimeter and the four-step process for measurement.
\end{problem}



\newpage

\begin{instructorNotes}

{\bf Main goal:} We apply the four-step process for measurement to calculate perimeters.

{\bf Overall picture:} 
\begin{itemize}
	\item Note that the figures aren't necessarily to scale; students shouldn't be measuring with their rulers here!
	\item The first child is counting the number of squares that the perimeter touches (think about shading the squares inside the figure). This is a dimension error as well as a measurement error - the squares aren't the right dimension for what we are trying to measure! Students should emphasize the unit they are using to measure here is the side of one of the squares.
	\item The second child is counting vertical and horizontal distance - perhaps they have a formula for perimeter in mind. The actual perimeter should be longer than their calculation, since it takes longer to go diagonally across the square than it does to just go from one side to another. If someone brings up the Pythagorean Theorem here (and calculates the exact perimeter) that's okay, but we want to focus mainly on the idea of matching units to the length, here. We'll discuss the Pythagorean Theorem later.
	\item The third child is likely thinking that perimeter has to be on the grid lines in order to be counted. (Perhaps the child's only experience is with a geo board?) Students should be especially clear how they are counting the half-unit while they measure. You can have several students present here if there are several different ideas throughout the room. Look out for the error of using a diagonal of a square for the half-unit, and discuss this if it comes up.
\end{itemize}
 


{\bf Good language:} Really emphasize that students are clear about the four-step process for measuring, here. Any time they do a measurement of any kind, they should be going through this process!

{\bf Suggested timing:} I'm hoping this takes about half a class so that we have a little time for catching up or homework questions or whatever else. Give students about 10 minutes to work on the questions, have a group present each problem (about 15 minutes total), then wrap up by emphasizing the meaning of perimeter and the four-step measuring process (5 min).


\end{instructorNotes}



\end{document}