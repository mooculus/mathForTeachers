%\documentclass{ximera}
\documentclass[nooutcomes,noauthor]{ximera}

\graphicspath{
  {./}
  {graphics/}
  {../graphics/}
}

\usepackage{chngcntr}

\let\question\relax
\let\endquestion\relax




\newtheoremstyle{SlantTheorem}{\topsep}{\fill}%%% space between body and thm
%\newtheoremstyle{SlantTheorem}{\topsep}{\topsep}%%% space between body and thm
 {\slshape}                      %%% Thm body font
 {}                              %%% Indent amount (empty = no indent)
 {\bfseries\sffamily}            %%% Thm head font
 {}                              %%% Punctuation after thm head
 {3ex}                           %%% Space after thm head
 {\thmname{#1}\thmnumber{ #2}\thmnote{ \bfseries(#3)}}%%% Thm head spec
\theoremstyle{SlantTheorem}
\newtheorem{question}{Question}
\counterwithin*{question}{section}



\let\instructorNotes\relax
\let\endinstructorNotes\relax
%%% instructorNotes environment
\ifhandout
\newenvironment{instructorNotes}[1][false]%
{%
\def\givenatend{\boolean{#1}}\ifthenelse{\boolean{#1}}{\begin{trivlist}\item}{\setbox0\vbox\bgroup}{}
}
{%
\ifthenelse{\givenatend}{\end{trivlist}}{\egroup}{}
}
\else
\newenvironment{instructorNotes}[1][false]%
{%
  \ifthenelse{\boolean{#1}}{\begin{trivlist}\item[\hskip \labelsep\bfseries {\Large Instructor Notes: \\} \hspace{\textwidth} ]}
{\begin{trivlist}\item[\hskip \labelsep\bfseries {\Large Instructor Notes: \\} \hspace{\textwidth} ]}
{}
}
{\end{trivlist}}
\fi


%% Suggested Timing
\newcommand{\timing}[1]{{\bf Suggested Timing: \hspace{2ex}} #1}
\title{Cases to consider}

\begin{document}
\begin{abstract}
\end{abstract}

\maketitle

For this activity, you may use any area formulas we have already proven as part of your work.

\begin{problem}
Can you use the same moving and additivity strategy on this parallelogram that you did in the previous activity? Why or why not? Then, find an area formula for this parallelogram using another strategy.

\begin{image}
\begin{tikzpicture}
\draw[thick] (0,0)--(2,0)--(5,2)--(3,2)--(0,0);
\node[below] at (1,0) {$b$};
\draw[thick, dashed] (2,0)--(5,0)--(5,2);
\node[right] at (5,1) {$h$};
\end{tikzpicture}
\end{image}
\end{problem} \vfill

\begin{problem}
Use the ideas of shearing to explain why the area of this parallelogram is {\bf not} found by multiplying the lengths of the sides.

\begin{image}
\begin{tikzpicture}
\draw[thick] (0,0)--(4,0)--(5,2)--(1,2)--(0,0);
\end{tikzpicture}
\end{image}
\end{problem} \vfill

\newpage
\begin{problem}
Use any strategy you like to find a formula for the area of this rhombus. You will need to add labeling to the figure.

\begin{image}
\begin{tikzpicture}
\draw[thick] (0,0)--(2,3)--(0,6)--(-2,3)--(0,0);
\end{tikzpicture}
\end{image}
\end{problem} \vfill


\begin{problem}
Use any strategy you like to find a formula for the area of this trapezoid. You will need to add labeling to the figure.

\begin{image}
\begin{tikzpicture}
\draw[thick] (0,0)--(4,0)--(3,2)--(2,2)--(0,0);
\end{tikzpicture}
\end{image}
\end{problem} \vfill


\newpage

\begin{instructorNotes} 

{\bf Main goal:} We practice developing area formulas.

{\bf Overall picture:} 
\begin{itemize}
	\item The first problem is designed to be a parallel to the obtuse triangle case, although students should be able to cut the parallelogram into two triangles. If this strategy was discussed with the previous activity, feel free to skip this problem.
	\item It's a common misconception that we multiply ``length times width'' to find the area of a parallelogram. The second problem gives us a chance to revisit shearing again and talk about what stays the same and what changes. Again, if these ideas have come up previously, feel free to skip them.
	\item The final two problems give us an opportunity to continue to practice all of our strategies. Since the figures are not labeled, students might come up with different formulas. For instance, there is a formula for the area of a rhombus based on its side lengths, or students might identify a base and height. Encourage students to reduce to as few variables as they can.
	\item The trapezoid has a nice ``combine two copies and take half'' strategy that turns the trapezoid into a parallelogram, or a cut into two triangles strategy similar to the first problem. We are hoping for creativity across these problems!
\end{itemize}


{\bf Good language:} Help students to be specific about their ideas and what other ideas they are drawing from as they work on these problems. Continue to remind the students that the meaning of area is the amount of 2D space an object takes up (and we measure it by covering with area units) any time it makes sense to do so!



{\bf Suggested timing:} Give students about $20$ minutes to work through the strategies here. Use the remaining time to present and discuss. You may not finish all of the problems here.



\end{instructorNotes}



\end{document}