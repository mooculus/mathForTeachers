\documentclass[nooutcomes, noauthor, handout]{ximera}

\graphicspath{
  {./}
  {graphics/}
  {../graphics/}
}

\usepackage{chngcntr}

\let\question\relax
\let\endquestion\relax




\newtheoremstyle{SlantTheorem}{\topsep}{\fill}%%% space between body and thm
%\newtheoremstyle{SlantTheorem}{\topsep}{\topsep}%%% space between body and thm
 {\slshape}                      %%% Thm body font
 {}                              %%% Indent amount (empty = no indent)
 {\bfseries\sffamily}            %%% Thm head font
 {}                              %%% Punctuation after thm head
 {3ex}                           %%% Space after thm head
 {\thmname{#1}\thmnumber{ #2}\thmnote{ \bfseries(#3)}}%%% Thm head spec
\theoremstyle{SlantTheorem}
\newtheorem{question}{Question}
\counterwithin*{question}{section}



\let\instructorNotes\relax
\let\endinstructorNotes\relax
%%% instructorNotes environment
\ifhandout
\newenvironment{instructorNotes}[1][false]%
{%
\def\givenatend{\boolean{#1}}\ifthenelse{\boolean{#1}}{\begin{trivlist}\item}{\setbox0\vbox\bgroup}{}
}
{%
\ifthenelse{\givenatend}{\end{trivlist}}{\egroup}{}
}
\else
\newenvironment{instructorNotes}[1][false]%
{%
  \ifthenelse{\boolean{#1}}{\begin{trivlist}\item[\hskip \labelsep\bfseries {\Large Instructor Notes: \\} \hspace{\textwidth} ]}
{\begin{trivlist}\item[\hskip \labelsep\bfseries {\Large Instructor Notes: \\} \hspace{\textwidth} ]}
{}
}
{\end{trivlist}}
\fi


%% Suggested Timing
\newcommand{\timing}[1]{{\bf Suggested Timing: \hspace{2ex}} #1}

\begin{document}
\title{My Feelings Are Similar...}
\begin{abstract}
\end{abstract}
\maketitle



\begin{problem}
Josh buys Josie a heart-shaped balloon for her birthday.  Josie's sister Sarah tells her friend Scott that she wants a bigger heart-shaped balloon - 2.5 times as long, 2.5 times as wide, and 2.5 times as tall as Josie's balloon.  Compared with the original balloon, how much material will it take to make Sarah's balloon?  How much helium will be needed to fill it (at the same pressure)?

%\vskip 0.1in
%Solve this problem in two ways: first by reasoning, and second by using a ratio.
\end{problem}

\begin{problem}
There is a statue of Jesse Owens on campus which is about $9$ feet tall. Let's say that the statue weighs 1200 pounds. The university wants to make a smaller version of the statue which is $3$ feet tall. How much would you expect the smaller statue to weigh?


\begin{enumerate}
    \item Here is Kerry's solution: Each foot of the big statue weighs $1200 \div 9 = 133 \frac{1}{3}$ pounds.  So we multiply that result by $3$ to get the weight of the small statue at $400$ pounds.
    \item Here is Kasha's solution: The small statue is $\frac{1}{3}$ as tall as the big statue, so it should also weigh $\frac{1}{3}$ as much. That means it should weigh $400$ pounds.
\end{enumerate}

Explain what is wrong with each of Kerry and Kasha's reasoning and their use of ratios.  Then, find the correct answer to this problem, and explain to each of Kerry and Kasha how to adjust their thinking.
\end{problem}

\begin{problem}
Solve the following problems by any method you choose.  %Then, discuss whether or not you could use a ratio to solve these problems.
\begin{enumerate}
    \item On a map, one inch represents 100 miles.  You are trying to find your location on this map.  You started from home and traveled 250 miles.  How far are you from home on the map?
    \item On a map, one inch represents 100 miles.  You know the state of Ohio has an area of approximately 45000 square miles.  How many square inches would Ohio take up on this map?
    \item A certain kind of crate will hold 450 pencils.  How many pencils will a crate that is five times larger in each dimension hold?
\end{enumerate}




\end{problem}

\newpage

\begin{instructorNotes}

{\bf Main goal:} We scale areas and volumes of shapes that are not rectangles or rectangular prisms.


{\bf Overall picture:} 

In a previous activity, you should have come to the conclusion that if lengths are scaled by $k$, then areas are scaled by $k^2$ (the ``area scale factor'') and volumes are scaled by $k^3$ (the ``volume scale factor'').  We apply this reasoning to the problems in this activity.

If students cleanly make the connection with the previous activity (though many do not), then the main goal should be discussing why this notion or rule works for any area or volume  (including things like surface area).  The discussion of scaling an individual unit from the previous activity should help with this discussion.  If the area is $A$ units, and each of those units is scaled by a factor of $k$, then each unit produces $k^2$ new units of area for a total of $Ak^2$ units, for instance.


\begin{itemize}
	\item On Problem 1, we usually say that the balloon is 2.5 times larger, but we clarify to be sure students realize this scaling is happening in each dimension. 
	\item Problem 1 is already dealing with some of the complications of dimension, as it is asking about both surface area (a common area of confusion) as well as volume.
	\item The ratio we are looking for is $1:2.5^2$ for surface area and $1:2.5^3$ for volume.
	\item In Problem 2, we have seen the errors in problem 2 made by students in the past, so they may struggle to see why these answers are incorrect. Both parts relate weight only to length, so a 3D measurement to a 1D measurement. We adjust Kerry's thinking by considering each unit of volume for the statue, and we help Kasha by thinking about the idea that since the length scale factor is $3$, the volume scale factor should be $3^3=27$.
	\item In problem 3, the main issue is again to distinguish the dimensions of what is being scaled.
\end{itemize}

{\bf Good language:} A particular challenge in this activity can be the dimension of the required measurement, as dimensionality is a struggle for many students. Remind students of our definition of dimension, as well as relevant examples we have already discussed.


{\bf Suggested timing:} Give students about 10-15 minutes to work on these problems, and then use the rest of the time to have students present their ideas and discuss.  If you can tell that everyone is stuck after about 5 minutes, you can bring the class together to discuss problem 1, and then send the students back to groups to work on problems 2 and 3.
\end{instructorNotes}



\end{document}