\documentclass[nooutcomes,noauthor, handout]{ximera}
\usepackage{gensymb}
\usepackage{tabularx}
\usepackage{mdframed}
\usepackage{pdfpages}
%\usepackage{chngcntr}

\let\problem\relax
\let\endproblem\relax

\newcommand{\property}[2]{#1#2}




\newtheoremstyle{SlantTheorem}{\topsep}{\fill}%%% space between body and thm
 {\slshape}                      %%% Thm body font
 {}                              %%% Indent amount (empty = no indent)
 {\bfseries\sffamily}            %%% Thm head font
 {}                              %%% Punctuation after thm head
 {3ex}                           %%% Space after thm head
 {\thmname{#1}\thmnumber{ #2}\thmnote{ \bfseries(#3)}} %%% Thm head spec
\theoremstyle{SlantTheorem}
\newtheorem{problem}{Problem}[]

%\counterwithin*{problem}{section}



%%%%%%%%%%%%%%%%%%%%%%%%%%%%Jenny's code%%%%%%%%%%%%%%%%%%%%

%%% Solution environment
%\newenvironment{solution}{
%\ifhandout\setbox0\vbox\bgroup\else
%\begin{trivlist}\item[\hskip \labelsep\small\itshape\bfseries Solution\hspace{2ex}]
%\par\noindent\upshape\small
%\fi}
%{\ifhandout\egroup\else
%\end{trivlist}
%\fi}
%
%
%%% instructorIntro environment
%\ifhandout
%\newenvironment{instructorIntro}[1][false]%
%{%
%\def\givenatend{\boolean{#1}}\ifthenelse{\boolean{#1}}{\begin{trivlist}\item}{\setbox0\vbox\bgroup}{}
%}
%{%
%\ifthenelse{\givenatend}{\end{trivlist}}{\egroup}{}
%}
%\else
%\newenvironment{instructorIntro}[1][false]%
%{%
%  \ifthenelse{\boolean{#1}}{\begin{trivlist}\item[\hskip \labelsep\bfseries Instructor Notes:\hspace{2ex}]}
%{\begin{trivlist}\item[\hskip \labelsep\bfseries Instructor Notes:\hspace{2ex}]}
%{}
%}
%% %% line at the bottom} 
%{\end{trivlist}\par\addvspace{.5ex}\nobreak\noindent\hung} 
%\fi
%
%


\let\instructorNotes\relax
\let\endinstructorNotes\relax
%%% instructorNotes environment
\ifhandout
\newenvironment{instructorNotes}[1][false]%
{%
\def\givenatend{\boolean{#1}}\ifthenelse{\boolean{#1}}{\begin{trivlist}\item}{\setbox0\vbox\bgroup}{}
}
{%
\ifthenelse{\givenatend}{\end{trivlist}}{\egroup}{}
}
\else
\newenvironment{instructorNotes}[1][false]%
{%
  \ifthenelse{\boolean{#1}}{\begin{trivlist}\item[\hskip \labelsep\bfseries {\Large Instructor Notes: \\} \hspace{\textwidth} ]}
{\begin{trivlist}\item[\hskip \labelsep\bfseries {\Large Instructor Notes: \\} \hspace{\textwidth} ]}
{}
}
{\end{trivlist}}
\fi


%% Suggested Timing
\newcommand{\timing}[1]{{\bf Suggested Timing: \hspace{2ex}} #1}




\hypersetup{
    colorlinks=true,       % false: boxed links; true: colored links
    linkcolor=blue,          % color of internal links (change box color with linkbordercolor)
    citecolor=green,        % color of links to bibliography
    filecolor=magenta,      % color of file links
    urlcolor=cyan           % color of external links
}

\title{Measuring area}
\begin{document}
\begin{abstract}\end{abstract}
\maketitle



\begin{problem}


 The picture below shows two rectangles. Which rectangle has more area?  How can you tell?  Discuss with your partner.  
\begin{image}
	\begin{tikzpicture}
		\draw[thick] (0,0) rectangle (5, 3);
		\draw[thick] (7, 0) rectangle (10.5, 5);
	\end{tikzpicture}
\end{image}
\end{problem}

\pagebreak

\begin{problem}
Consider the following two shapes.  Which shape has more area?

\begin{image}
\begin{tikzpicture}
\draw[thick, smooth] plot coordinates {(2, -4) (0,0) (1.2, 4) (-0.6, 8.5) (3.5, 10.3) (6.1, 6.9) (4.7, 2.9) (6.1, -2.1) (2, -4)};
\draw[thick, smooth] plot coordinates {(9,0) (8.6, 8) (10.6, 8.7) (14.8, 6.3) (14.7, 2.2) (12.9, 4) (12.7, 0.6) (12.4, -2.8) (9,0)};
\end{tikzpicture}
\end{image}
   
\begin{enumerate}
    \item Can you use the same method you used for the previous problem to solve this problem? Why or why not?
    \item Decide which shape has more area, then write a short description of your process.
    \item Can you solve the previous part in another way?

\end{enumerate}
\end{problem}

\pagebreak

\begin{problem}
Are there any similarities between your measurement processes for length and for area?
\end{problem}
\vskip 2in



\begin{problem}
Use an index card (or similar rectangular, non-square item) to find the area of your table.  Make a note of your process.  Compare your group's method to another group's method.  How do your results compare?

\end{problem}



\pagebreak


\begin{instructorNotes}
{\bf Main goal:} We explore the notion that area is about how much 2D space an object takes up.

{\bf Overall picture:}
This is the first activity helping us to transition from measuring length to measuring area. We'll continue to emphasize many of the same ideas from the previous section throughout our work with area. We have found that many students' concept of area is essentially ``area is length times width''. We need to break down this automatic reaction and replace it with more meaningful ideas about both what area means as well as how to calculate it. Non-standard units of measure are especially helpful in this process!

For the first problem, we expect many students to try to measure the length and width of each rectangle. As you discuss their methods for this problem, you might remind them we have not yet justified that formula! However, they will likely not build other tools for solving this problem until problem 2.

In problem 2, we expect students to realize that their measurement attempts will not be terribly useful, here, and then reason about area in another way.  One good way is to attempt to cover one shape with the other, somehow. They may cut out the two shapes, and cut one of them into pieces to try to ``fit'' it into the other shape. They may also try to describe characteristics of the two shapes, but this is less precise. All of their ideas should be heard here, especially as they later point to the ``covering'' notion we would like to draw out of the discussion. As you discuss this question, be sure to have several different methods presented and described by the students. Encourage students to talk about not only their results but the thought process behind what they did!

For problem 3, we are encouraging students to see that the general process of measuring is the same whether we measure length or area (or volume), but students may struggle to see this with their examples so far, especially if they have calculated a ``length times width'' and then done a direct comparison. These may not at all feel like we are doing the same thing! If students feel this way, you might as whether they think it should be different, or whether it should be the same, and encourage them to return to this question after we've done a few more activities.

The final problem is our first example of measuring a rectangle with a non-standard unit. This should begin to bring up all kinds of confusion: what does ``length times width'' mean, here? What kind of units should we put on the answer? (Index cards? Index cards squared?) Should we turn the index card to measure the length, and turn it a different way to measure the width? Do we have to organize the index cards in any particular fashion? Why does it seem like we ``double-count'' that index card that we point to both for the length and for the width? In your discussion, have students describe their methods, and then make a list of questions they have. 






{\bf Good language:} Listen carefully to students' statements about the meaning of area! We want to change the language they use to talk about area over the course of the next several activities, so in this activity it's good to get a sense of where they already are, and begin to help them talk about covering rather than measuring.

Some students can be confused as to whether they should write ``units'' or ``square units''. This notion will come up eventually if it doesn't in this activity!  The index card in this case measures area, so we can just say ``index cards'' or ``units''.  A ``square unit'' is a special kind of area unit, and we sometimes abbreviate this as ``sq. un.'' or ``un$^2$'', though the first is preferred over the second in this course. Of course ``square index cards'' doesn't make sense in this problem.

{\bf Suggested timing:} This activity should take the entire class period. Give students about 20 minutes to work through their ideas on these problems, and use the remaining time in class to discuss their ideas. You may want to provide a paper copy of problems 1 and 2 so that students can cut out and otherwise manipulate these shapes, and you will need a collection of index cards for Problem 4. Save the last five minutes of your discussion for the meaning of area.



\end{instructorNotes}





\end{document}