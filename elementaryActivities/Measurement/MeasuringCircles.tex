\documentclass[noauthor, nooutcomes, handout]{ximera}

\graphicspath{
  {./}
  {graphics/}
  {../graphics/}
}

\usepackage{chngcntr}

\let\question\relax
\let\endquestion\relax




\newtheoremstyle{SlantTheorem}{\topsep}{\fill}%%% space between body and thm
%\newtheoremstyle{SlantTheorem}{\topsep}{\topsep}%%% space between body and thm
 {\slshape}                      %%% Thm body font
 {}                              %%% Indent amount (empty = no indent)
 {\bfseries\sffamily}            %%% Thm head font
 {}                              %%% Punctuation after thm head
 {3ex}                           %%% Space after thm head
 {\thmname{#1}\thmnumber{ #2}\thmnote{ \bfseries(#3)}}%%% Thm head spec
\theoremstyle{SlantTheorem}
\newtheorem{question}{Question}
\counterwithin*{question}{section}



\let\instructorNotes\relax
\let\endinstructorNotes\relax
%%% instructorNotes environment
\ifhandout
\newenvironment{instructorNotes}[1][false]%
{%
\def\givenatend{\boolean{#1}}\ifthenelse{\boolean{#1}}{\begin{trivlist}\item}{\setbox0\vbox\bgroup}{}
}
{%
\ifthenelse{\givenatend}{\end{trivlist}}{\egroup}{}
}
\else
\newenvironment{instructorNotes}[1][false]%
{%
  \ifthenelse{\boolean{#1}}{\begin{trivlist}\item[\hskip \labelsep\bfseries {\Large Instructor Notes: \\} \hspace{\textwidth} ]}
{\begin{trivlist}\item[\hskip \labelsep\bfseries {\Large Instructor Notes: \\} \hspace{\textwidth} ]}
{}
}
{\end{trivlist}}
\fi


%% Suggested Timing
\newcommand{\timing}[1]{{\bf Suggested Timing: \hspace{2ex}} #1}

\title{Measuring Circles}

\begin{document}
\begin{abstract}
\end{abstract}
\maketitle



\begin{problem} 
Work with a partner for this problem. Grab your compass, a measuring tape, and a blank sheet of paper. Draw at least four different circles, marking their centers and drawing a diameter on each one. Then cut out each circle. Measure the circumference and diameter of each circle and record your data in the table below.
\vskip 0.1in
\begin{tabular}{|p{2.2in}|p{2.2in}|}
\hline
\begin{center} Circumference \end{center} & \begin{center} Diameter \end{center}  \\ \hline
 & \\ [5ex] \hline
 & \\ [5 ex] \hline
 & \\ [5 ex] \hline
 & \\ [5 ex] \hline
\end{tabular}
\end{problem}

\begin{problem}
Join forces with another group and share all of your data. Then make at least four observations about your data.  As you make observations, make conjectures as to why your observations might hold.

\end{problem}

\newpage

\begin{problem}
We just estimated the value of $\pi$ by trying to measure the circumference of a circle. What was our estimate? How could we make our estimate better?
\end{problem}


\begin{problem}
Some people estimate the value of $\pi$ using a dartboard shaped like a circle inside a square. What would you do to find an estimate for $\pi$ in this situation? How could you make your estimate better?
\begin{image} \begin{tikzpicture}
\draw[thick] (-2,-2) rectangle (2,2);
\draw[thick] (0,0) circle (2cm);
\end{tikzpicture}\end{image}
\end{problem}


\begin{problem}
Some people estimate the value of $\pi$ using another geometric shape whose perimeter is easier to calculate. What shape would you use, and what would be your estimate? How could you make your estimate better? Here is a circle to draw on if you need it.
\begin{image} \begin{tikzpicture}
\draw[thick] (0,0) circle (2cm);
\end{tikzpicture}\end{image}
\end{problem}

\newpage





\begin{instructorNotes}
{\bf Main goal:} This activity should bring out the value pi as the ratio between the circumference and the diameter and then estimate this value.  


{\bf Overall picture:} Students seem to like measuring circles, because it feels like a lesson they could do with their own students someday.

Bring lots of different-sized cylinders and tape measures.  Pass them out, and have the students pass them around as they need more cylinders to measure.  After students have made their own charts, try to collect as much data on the board as possible.

In their observations, students should point out that $C/d$ seems to be pretty much staying the same, and the value is approximately 3 in most cases.  There will be measurement error, which is a good thing to discuss, and even perhaps significant outliers (indicating incorrect measurements).  When you talk about why this relationship should hold, it's nice to bring up the idea that each circle is similar to every other, and then bring this up again when you discuss what similarity actually is.

It's good to connect to our work on scaling here. We can see that every circle is similar to every other circle by dilating the radius.

Wrap up your discussion by clearly stating that $C/d$ is actually constant, and we define this number to be $\pi$.

Then, move on to the second page, where our goal is to talk about how we can estimate $\pi$. 

The first estimation question asks students to connect their measuring work to estimation. This might have been done in the earlier discussion, in which case this question should be quick and doesn't need to be discussed a second time. 

The next estimation question is common in pi day activities, but also connects to the area formula for a circle. We compare the fraction of darts inside the circle to the fraction outside. To make the estimate better we would throw more darts.

The final estimation question is connected to Archimedes' estimation with polygons. Students may think of a square around the outside of the circle and may need to be encouraged to also think about shapes inside the circle. If someone suggests a hexagon you might try to calculate the estimate since the hexagon is made of equilateral triangles. To make the estimate better, more sides can be drawn, and you can connect to Archimedes' work on this topic.

Throughout the estimation conversation, you might ask students whether their estimate is an under-estimate or an over-estimate, and how they can tell.


\timing{About 5-10 minutes to measure circles and record the results in a table, then about 15 minutes of discussion. Then on page 2, about 5 minutes in groups, followed by about 10 minutes of discussion. You do not need to finish all of the estimation questions.}
\end{instructorNotes}


\end{document}
