\documentclass[nooutcomes]{ximera}
\usepackage{gensymb}
\usepackage{tabularx}
\usepackage{mdframed}
\usepackage{pdfpages}
%\usepackage{chngcntr}

\let\problem\relax
\let\endproblem\relax

\newcommand{\property}[2]{#1#2}




\newtheoremstyle{SlantTheorem}{\topsep}{\fill}%%% space between body and thm
 {\slshape}                      %%% Thm body font
 {}                              %%% Indent amount (empty = no indent)
 {\bfseries\sffamily}            %%% Thm head font
 {}                              %%% Punctuation after thm head
 {3ex}                           %%% Space after thm head
 {\thmname{#1}\thmnumber{ #2}\thmnote{ \bfseries(#3)}} %%% Thm head spec
\theoremstyle{SlantTheorem}
\newtheorem{problem}{Problem}[]

%\counterwithin*{problem}{section}



%%%%%%%%%%%%%%%%%%%%%%%%%%%%Jenny's code%%%%%%%%%%%%%%%%%%%%

%%% Solution environment
%\newenvironment{solution}{
%\ifhandout\setbox0\vbox\bgroup\else
%\begin{trivlist}\item[\hskip \labelsep\small\itshape\bfseries Solution\hspace{2ex}]
%\par\noindent\upshape\small
%\fi}
%{\ifhandout\egroup\else
%\end{trivlist}
%\fi}
%
%
%%% instructorIntro environment
%\ifhandout
%\newenvironment{instructorIntro}[1][false]%
%{%
%\def\givenatend{\boolean{#1}}\ifthenelse{\boolean{#1}}{\begin{trivlist}\item}{\setbox0\vbox\bgroup}{}
%}
%{%
%\ifthenelse{\givenatend}{\end{trivlist}}{\egroup}{}
%}
%\else
%\newenvironment{instructorIntro}[1][false]%
%{%
%  \ifthenelse{\boolean{#1}}{\begin{trivlist}\item[\hskip \labelsep\bfseries Instructor Notes:\hspace{2ex}]}
%{\begin{trivlist}\item[\hskip \labelsep\bfseries Instructor Notes:\hspace{2ex}]}
%{}
%}
%% %% line at the bottom} 
%{\end{trivlist}\par\addvspace{.5ex}\nobreak\noindent\hung} 
%\fi
%
%


\let\instructorNotes\relax
\let\endinstructorNotes\relax
%%% instructorNotes environment
\ifhandout
\newenvironment{instructorNotes}[1][false]%
{%
\def\givenatend{\boolean{#1}}\ifthenelse{\boolean{#1}}{\begin{trivlist}\item}{\setbox0\vbox\bgroup}{}
}
{%
\ifthenelse{\givenatend}{\end{trivlist}}{\egroup}{}
}
\else
\newenvironment{instructorNotes}[1][false]%
{%
  \ifthenelse{\boolean{#1}}{\begin{trivlist}\item[\hskip \labelsep\bfseries {\Large Instructor Notes: \\} \hspace{\textwidth} ]}
{\begin{trivlist}\item[\hskip \labelsep\bfseries {\Large Instructor Notes: \\} \hspace{\textwidth} ]}
{}
}
{\end{trivlist}}
\fi


%% Suggested Timing
\newcommand{\timing}[1]{{\bf Suggested Timing: \hspace{2ex}} #1}




\hypersetup{
    colorlinks=true,       % false: boxed links; true: colored links
    linkcolor=blue,          % color of internal links (change box color with linkbordercolor)
    citecolor=green,        % color of links to bibliography
    filecolor=magenta,      % color of file links
    urlcolor=cyan           % color of external links
}

\title{Scale It Up}
\author{Vic Ferdinand, Betsy McNeal, Jenny Sheldon} 


\begin{document}


\begin{abstract}
Recall: When an object is scaled, the new object (which is similar to the original) is the same shape as the original, but a different size.  In this activity, we explore what happens to area and volume when we scale.
\end{abstract}

\maketitle



\begin{problem} \label{ScaleItUp1}
The rectangle below has dimensions $L$ and $W$. If we scale it by a scale factor of 3, what will the new dimensions of the rectangle be? Draw the new rectangle in the given space.\\

\tikz{ \draw (0 cm,0) -- (2 cm,0); \draw (2 cm, 0) -- (2 cm, 1 cm); \draw (2 cm, 1 cm) -- (0 cm, 1 cm);\draw (0 cm, 1 cm) -- (0 cm, 0);} \hspace{4cm}
\vskip 2.5in

\begin{enumerate}
    \item What is the ratio of the new rectangle's length to the original  length?
    \item What is the area of the new rectangle?  How can we see this in the picture?
    \item What is the ratio of the new rectangle's area to the original area?
\end{enumerate}


\end{problem}

\begin{problem} \label{ScaleItUp2}
If we now scale the \underline{original} rectangle by the linear scale factor of $\frac{11}{3}$, what will the dimensions of this third rectangle be?  Predict the new area.  Now, draw the third rectangle and answer the same questions as above, comparing this rectangle with the original. (Can you build on the picture you've already drawn?)



\end{problem}

%\pagebreak

\begin{problem}
What will happen to length and area if we scale a rectangle of dimensions $2 cm$ by $5 cm$ by a linear scale factor of $k$? What will happen to length and area if we scale a rectangle of dimensions $L$ and $W$ by a linear scale factor of $k$?


\end{problem}

\begin{problem}
Justify your observations in the previous question in three different ways.  By the end of this class period, you should have an algebraic explanation (using the formula for area), a geometric explanation (using your picture in the first problem), and an explanation dealing with scaling an individual unit of area.

\end{problem}

\begin{problem}
Draw a cube whose side length is 3cm, and a rectangular prism whose dimensions are 2cm by 5cm by 3cm.  Scale each of these prisms using several different linear scale factors of your choosing, and record the ratio of the original volume to the new volume.  What do you notice?
\vskip 2.3in
\begin{enumerate}
    \item What is the ratio of the new prism's length to the original  length?
    \item What is the volume of the new prism?  How can we see this in the picture?
    \item What is the ratio of the new prism's volume to the original volume?
\end{enumerate}

\end{problem}

\begin{problem}
Justify your observations in the previous question by scaling a unit of volume.

\end{problem}

\begin{problem}
When we scaled the cube and rectangular prism above, what happened to their surface area?  Justify your claim!
\end{problem}


\newpage
\begin{instructorNotes}
The goal of this activity is to help students discover what happens to area and volume when we scale rectangles and rectangular prisms.  Scaling of other shapes will be covered in our next activity.  Our overall goal is to develop scale factors for one, two, and three dimensional scaling. 

For us, this activity is done while we are working on more advanced ratio material.  We discuss ratios in the first semester, and then return to them in the second semester as a way of reviewing some of our earlier concepts, and looking at how we can see the ideas of ratios across the curriculum.  The idea of scaling areas and volumes using ratios is unfamiliar to most of our students, which is another instance of ``making the familiar strange''.  Students have to grapple with these ideas in a new context, rather than relying on previously-learned material.

This activity assumes basic familiarity with scaling, as well as with the concept of ratios.  This is, however, the first time our students are scaling anything other than length.

We have used the terminology ``scale factor'' when we discussed similarity earlier in the course, so we are adding the term ``linear'' to distinguish one dimensional scaling from scaling in higher dimensions.  ``Scale factor'' and ``linear scale factor'' mean the same thing.  We use ``area scale factor'' to describe how areas are scaled, and ``volume scale factor'' to describe how volumes are scaled.  We introduce all of this terminology as we discuss the problems in this activity.

\begin{itemize}
    \item Since this activity occurs well after our initial activities on scaling, we usually spend some time reminding students what they already know: what happens when we scale, what the scale factor means, or that the scale factor scales all lengths in a figure (not just the perimeter).
    \item In Problem \ref{ScaleItUp1}, students often have trouble with the concept of the area of the new rectangle, since they are not given the actual dimensions of the original rectangle. Working with some specific examples may help them to see the general concept.
    \item Problem \ref{ScaleItUp2} is included because we have observed that students often forget to test cases which include non-integer numbers.  Working both problems helps our discussion to start to feel general, so that generalizing in the next problem is not a surprise.
    \item Our students tend to struggle with the fractional copies in Problem \ref{ScaleItUp2}.
    \item Since we work quite a bit with non-standard units of area, these problems tend to feel to our students like we are measuring the area of the new figure using the original figure.  We point this out if the observation is not made by a student.
    \item We have found the argument about scaling an individual unit of area to be the most generalizable for our students.
    \item Our students tend to struggle with their volume drawings.  We sometimes bring unit cubes to class so that students can model the volume problems.
    \item We frequently run out of time for the question about surface area, and since we usually have also not spent a lot of time on surface area, this is the part that makes the most sense for us to cut.
\end{itemize}


\timing{This activity takes us the whole class period.  We split up the discussion into length and area, and then volume and surface area (if time).  In each instance, we give students about 10 minutes to work on the problems in their groups, and then spend about 20 minutes in whole-class discussion.}
\end{instructorNotes}



\end{document}