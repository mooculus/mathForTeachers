%\documentclass[handout]{ximera}
\documentclass {ximera}

\graphicspath{
  {./}
  {graphics/}
  {../graphics/}
}

\usepackage{chngcntr}

\let\question\relax
\let\endquestion\relax




\newtheoremstyle{SlantTheorem}{\topsep}{\fill}%%% space between body and thm
%\newtheoremstyle{SlantTheorem}{\topsep}{\topsep}%%% space between body and thm
 {\slshape}                      %%% Thm body font
 {}                              %%% Indent amount (empty = no indent)
 {\bfseries\sffamily}            %%% Thm head font
 {}                              %%% Punctuation after thm head
 {3ex}                           %%% Space after thm head
 {\thmname{#1}\thmnumber{ #2}\thmnote{ \bfseries(#3)}}%%% Thm head spec
\theoremstyle{SlantTheorem}
\newtheorem{question}{Question}
\counterwithin*{question}{section}



\let\instructorNotes\relax
\let\endinstructorNotes\relax
%%% instructorNotes environment
\ifhandout
\newenvironment{instructorNotes}[1][false]%
{%
\def\givenatend{\boolean{#1}}\ifthenelse{\boolean{#1}}{\begin{trivlist}\item}{\setbox0\vbox\bgroup}{}
}
{%
\ifthenelse{\givenatend}{\end{trivlist}}{\egroup}{}
}
\else
\newenvironment{instructorNotes}[1][false]%
{%
  \ifthenelse{\boolean{#1}}{\begin{trivlist}\item[\hskip \labelsep\bfseries {\Large Instructor Notes: \\} \hspace{\textwidth} ]}
{\begin{trivlist}\item[\hskip \labelsep\bfseries {\Large Instructor Notes: \\} \hspace{\textwidth} ]}
{}
}
{\end{trivlist}}
\fi


%% Suggested Timing
\newcommand{\timing}[1]{{\bf Suggested Timing: \hspace{2ex}} #1}




\graphicspath{
  {./}
  {graphics/}
  {../graphics/}
}

\usepackage{chngcntr}

\let\question\relax
\let\endquestion\relax




\newtheoremstyle{SlantTheorem}{\topsep}{\fill}%%% space between body and thm
%\newtheoremstyle{SlantTheorem}{\topsep}{\topsep}%%% space between body and thm
 {\slshape}                      %%% Thm body font
 {}                              %%% Indent amount (empty = no indent)
 {\bfseries\sffamily}            %%% Thm head font
 {}                              %%% Punctuation after thm head
 {3ex}                           %%% Space after thm head
 {\thmname{#1}\thmnumber{ #2}\thmnote{ \bfseries(#3)}}%%% Thm head spec
\theoremstyle{SlantTheorem}
\newtheorem{question}{Question}
\counterwithin*{question}{section}



\let\instructorNotes\relax
\let\endinstructorNotes\relax
%%% instructorNotes environment
\ifhandout
\newenvironment{instructorNotes}[1][false]%
{%
\def\givenatend{\boolean{#1}}\ifthenelse{\boolean{#1}}{\begin{trivlist}\item}{\setbox0\vbox\bgroup}{}
}
{%
\ifthenelse{\givenatend}{\end{trivlist}}{\egroup}{}
}
\else
\newenvironment{instructorNotes}[1][false]%
{%
  \ifthenelse{\boolean{#1}}{\begin{trivlist}\item[\hskip \labelsep\bfseries {\Large Instructor Notes: \\} \hspace{\textwidth} ]}
{\begin{trivlist}\item[\hskip \labelsep\bfseries {\Large Instructor Notes: \\} \hspace{\textwidth} ]}
{}
}
{\end{trivlist}}
\fi


%% Suggested Timing
\newcommand{\timing}[1]{{\bf Suggested Timing: \hspace{2ex}} #1}

\title{Triangle Puzzle  }

\begin{document}
\begin{abstract} 
%This activity will help you begin to think about what it means to write an explanation for the answer to a mathematics problem. 
Welcome to Math 1126! In this course, we will continue to investigate the mathematics content typically needed to teach grades K -- 5. Throughout the course, we would like you to work together with your peers in class. Please form a group of 3-4 people to work with for today.
\end{abstract}
\maketitle



\begin{question}
We would like to be a community. Please start by introducing yourself to your group members. What do you like to be called? (Consider including your preferred pronouns!) What you are currently thinking you might like to teach? What is something interesting about you?
\end{question}


\begin{question}
Please use your stack of Post-It Notes for the next few questions. Either individually or as a team, answer the following questions on your Post-Its, then hang them on the board in the indicated area. You may use as many Post-Its as you like, and you can answer the questions in any way that seems good to you.
\begin{enumerate}
	\item What guidelines would you like for us to follow as a community?
	\item How would you like your classmates to treat you in this course?
	\item Should we have any goals as a whole class? How should we reach those goals?
\end{enumerate}
\end{question}

\begin{question}
Either individually or as a team, answer the following questions on your Post-Its, then hang them on the board in the indicated area. You may use as many Post-Its as you like, and you can answer the questions in any way that seems good to you.
\begin{enumerate}
	\item What would help you to feel safe to share your ideas in this class?
	\item What would help you to feel welcome in this class?
	\item What resources help you to access this class with ease?
\end{enumerate}

\end{question}


\begin{question}
Use your Post-Its again to write about things you would either do the same as you did in your Math 1125 or things you would do differently from your Math 1125 (or both!) You may use as many Post-Its as you like, and you can answer the questions in any way that seems good to you. Hang your Post-Its on the board in the indicated area.
\end{question}

\begin{question}
Choose one of the groups of Post-Its and work with others to rearrange the Post-Its in any way that seems right to your group. Summarize the ideas on the Post-Its. We will have some time for each group to tell the whole class about their summary, and at least two people should talk during the summary. (It's even better if everyone talks!)
\end{question}



\newpage


Now, let's do some math! Here's a puzzle: how many triangles are in the following image?

\begin{image}
\begin{tikzpicture}
\draw[thick] (0,0)--(5,0)--(2,4)--(0,0);
\draw[thick] (2,0)--(2,4);
\draw[thick] (2,4)--(4,0);
\draw[thick] (0.5, 1)--(4.25, 1);
\end{tikzpicture}
\end{image}

\begin{problem} 
Spend a few minutes thinking about this problem either by yourself or by discussing the problem with your group.  After you have a guess as to what the answer might be, discuss the following questions.
\begin{enumerate}
\item Did each member of your group come up with the same answer?  Did each member come to their answer in the same way?



\item
Pick one method and one answer from your group.  Discuss this method in detail, as if you were presenting it to the class.  What ideas are the most important?  What steps are the most complicated?

\end{enumerate}
\end{problem}



\begin{problem} 
As a group, write an explanation for one of your solutions which would {\bf not} demonstrate full understanding of this puzzle. After your explanation is ready, consider the following questions together.
\begin{enumerate}
	\item Why is full understanding {\bf not} demonstrated here? What is missing from the explanation? What might full understanding look like for this puzzle?
	\item If you were a teacher, what feedback might you give to a student who wrote this explanation? 
	\item What are the underlying ideas for this puzzle? If a teacher gave this problem to their students, what might have been that teacher's motivation? What kind of learning might a student demonstrate through this problem?
\end{enumerate}
\end{problem}


\newpage


\begin{instructorNotes}


{\bf Main goal:} We use this activity as an introduction to the course, a reminder (for students who took 1125 with us) about the importance of sharing their thoughts with a small group and the class, and to revisit (for students who took 1125 with us) what a mathematical explanation can look like. As long as the students are talking to each other and sharing their thoughts with the class, this activity is a success.

{\bf Overall picture:} 
\begin{itemize}
	\item Spend a few minutes introducing yourself to the class and dealing with overall course information. You can quickly bring up any area of the syllabus you feel would help the students understand the activity, but we actually recommend asking students to read the syllabus on their own and bring questions to class the next day.
	\item This activity is designed to be done in two stages. You can print out copies to hand out in class if you'd like, and it may help to hand out one page at a time. If the students are working from a digital copy, be sure to ask them not to move on until you ask them to.
\end{itemize}
	
Pages 1-2:


\begin{itemize}
	\item The goal for this page is to work on collectively deciding norms for our classroom discussions. These prompts will hopefully help students to think about what kinds of norms will be good for the class as well as help them work in groups from the start of class.
	\item Mark off three areas on the chalkboard for the students to hang the Post-Its. Encourage students to hang up their notes as they write them; they don't have to wait until all of their ideas are solidified in order to put them on the board. And of course if they change their minds they can always go take them down!
	\item Let students take the lead, here: the classroom community is as much theirs as it is ours! Also, if students aren't comfortable with some of the questions (or just don't know the answers), it's okay to skip any of them. We are just looking for information to make our community better at this point.
	\item Please take a photo of the groups of Post-Its when students are finished with them (before you take them down to move on to the Triangle puzzle). We will incorporate the results of this activity into some upcoming classes!
\end{itemize}


Page 3:
\begin{itemize}


\item We sometimes put the initial problem on the board for students to solve before directing them to look at the activity.  We also encourage students to work in groups, to remind them that this will be our expectation throughout the course.  This question is designed to give everyone in the group a chance to speak.

\item After students have had a chance to discuss in small groups, we try to have students who are willing present their solutions at the board.  This way, we can discuss the different answers, and see if anyone came up with other answers during the process of solving.  This also gives a good opportunity to highlight the problem-solving process which will serve students well throughout the course.

\item Asking students to think about the problem from a teacher's perspective in the second question on this page is designed to help students improve their own explanations. Ask several groups to share their explanations and their thinking while they created these explanations. Our goal is to illuminate the kind of thinking students should do while writing explanations of their own, and the kind of critical eye we want them to learn to give their own explanations.

\item There is no wrong answer to what the teacher might have been thinking, but one important idea we hope to see come up is the definition of a triangle! It's hard to show full understanding if you have never explained what a triangle actually is.
\end{itemize}


{\bf Good language:}                                                                                                                                                   
Use encouraging language as often as possible. Many students are nervous about sharing their thoughts with strangers! Be sure to thank students for their contributions. Also, keep in mind that each person is an expert in their own thoughts and experiences, so we don't want to take over the conversation as an instructor. You can add your own thoughts to the discussion, and try to point students in helpful directions, but we want more student discussion than instructor discussion.






{\bf Suggested Timing:} For page 1, give students about 5-10 minutes to work on their Post-Its, then another 5 or so minutes to rearrange them and prepare to present, then about 15 minutes on presenting. For page 2, give students about 10-15 minutes to work on the puzzle and their explanations, and then use the remaining time for discussion. You do not need to ``finish'' this activity.

%After the course introduction (which sometimes takes about half a class period), we use the rest of the first day for this activity.

\end{instructorNotes}


\end{document}