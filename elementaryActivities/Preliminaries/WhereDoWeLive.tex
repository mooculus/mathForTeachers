\documentclass[nooutcomes,noauthor, handout]{ximera}
%\documentclass[handout]{ximera}
\usepackage{gensymb}
\usepackage{tabularx}
\usepackage{mdframed}
\usepackage{pdfpages}
%\usepackage{chngcntr}

\let\problem\relax
\let\endproblem\relax

\newcommand{\property}[2]{#1#2}




\newtheoremstyle{SlantTheorem}{\topsep}{\fill}%%% space between body and thm
 {\slshape}                      %%% Thm body font
 {}                              %%% Indent amount (empty = no indent)
 {\bfseries\sffamily}            %%% Thm head font
 {}                              %%% Punctuation after thm head
 {3ex}                           %%% Space after thm head
 {\thmname{#1}\thmnumber{ #2}\thmnote{ \bfseries(#3)}} %%% Thm head spec
\theoremstyle{SlantTheorem}
\newtheorem{problem}{Problem}[]

%\counterwithin*{problem}{section}



%%%%%%%%%%%%%%%%%%%%%%%%%%%%Jenny's code%%%%%%%%%%%%%%%%%%%%

%%% Solution environment
%\newenvironment{solution}{
%\ifhandout\setbox0\vbox\bgroup\else
%\begin{trivlist}\item[\hskip \labelsep\small\itshape\bfseries Solution\hspace{2ex}]
%\par\noindent\upshape\small
%\fi}
%{\ifhandout\egroup\else
%\end{trivlist}
%\fi}
%
%
%%% instructorIntro environment
%\ifhandout
%\newenvironment{instructorIntro}[1][false]%
%{%
%\def\givenatend{\boolean{#1}}\ifthenelse{\boolean{#1}}{\begin{trivlist}\item}{\setbox0\vbox\bgroup}{}
%}
%{%
%\ifthenelse{\givenatend}{\end{trivlist}}{\egroup}{}
%}
%\else
%\newenvironment{instructorIntro}[1][false]%
%{%
%  \ifthenelse{\boolean{#1}}{\begin{trivlist}\item[\hskip \labelsep\bfseries Instructor Notes:\hspace{2ex}]}
%{\begin{trivlist}\item[\hskip \labelsep\bfseries Instructor Notes:\hspace{2ex}]}
%{}
%}
%% %% line at the bottom} 
%{\end{trivlist}\par\addvspace{.5ex}\nobreak\noindent\hung} 
%\fi
%
%


\let\instructorNotes\relax
\let\endinstructorNotes\relax
%%% instructorNotes environment
\ifhandout
\newenvironment{instructorNotes}[1][false]%
{%
\def\givenatend{\boolean{#1}}\ifthenelse{\boolean{#1}}{\begin{trivlist}\item}{\setbox0\vbox\bgroup}{}
}
{%
\ifthenelse{\givenatend}{\end{trivlist}}{\egroup}{}
}
\else
\newenvironment{instructorNotes}[1][false]%
{%
  \ifthenelse{\boolean{#1}}{\begin{trivlist}\item[\hskip \labelsep\bfseries {\Large Instructor Notes: \\} \hspace{\textwidth} ]}
{\begin{trivlist}\item[\hskip \labelsep\bfseries {\Large Instructor Notes: \\} \hspace{\textwidth} ]}
{}
}
{\end{trivlist}}
\fi


%% Suggested Timing
\newcommand{\timing}[1]{{\bf Suggested Timing: \hspace{2ex}} #1}




\hypersetup{
    colorlinks=true,       % false: boxed links; true: colored links
    linkcolor=blue,          % color of internal links (change box color with linkbordercolor)
    citecolor=green,        % color of links to bibliography
    filecolor=magenta,      % color of file links
    urlcolor=cyan           % color of external links
}

%\renewenvironment{instructorNotes}{
%\ifhandout\setbox0\vbox\bgroup\else
%\begin{trivlist}\item[\hskip \labelsep\bfseries Instructor Notes: ]\hfill \break
%\fi}
%{\ifhandout\egroup\else
%\end{trivlist}
%\fi}


\title{Where Do We Live?}

\begin{document}
\begin{abstract}
\end{abstract}


\maketitle

HI
%problem 1
\begin{problem}
    With your group, list at least four one- two- and three-dimensional aspects of a house.
    

    

\end{problem}

%problem 2
\begin{problem}
    In your own words, what is ``dimension''?  How would you describe this idea to a child in Kindergarten?  How would you help a fourth-grader who is struggling with this concept?
    
        \begin{solution}
Kindergarten: we would likely give examples of things that are one- two- and three-dimensional. Fourth grade: we would likely discuss how we can ``move around'' in that space. If you were a small bug living only in that space, how could you move? Just forward and back? Do you also have north and south? Can you fly up and down?
    \end{solution}

\end{problem}

\begin{problem}
    Sally claims that the amount of gutter on the side of your house is two-dimensional, because it goes around both the length and the width of the house.  Discuss Sally's claim.
    
    \begin{solution}
    	Sally is incorrect: the length of the gutter is one-dimensional.
    \end{solution}
\end{problem}

\begin{problem}
    Suzi argues that the amount of carpet in your house is three-dimensional, while Shawn argues that it is two-dimensional.  Could both be correct?
    
    \begin{solution}
    	The carpet itself is three-dimensional. Our friend the tiny bug could move anywhere in the plane, but could also climb up the fibers. But because the fibers are all the same length, we usually measure carpet as covering a 2D area.
    \end{solution}
\end{problem}

\begin{problem}
You have an antique grandfather clock that you would like to move into your new house.  What kind of measurements would you need to take in order to determine if the clock will fit in the house?  What dimensions are associated with these measurements?  Repeat the question for your over-sized couch.

\begin{solution}
There are many things to be considered here, so answers will vary.  Try to think not just of the usual one-dimensional measurements (like the length of the couch or the height of the clock), but also of two-dimensional (or even three-dimensional) measurements that you could take into consideration.

If students struggle with this question, you can bring up the more traditional question of one-, two-, and three-dimensional aspects of a lake.
\end{solution}

\end{problem}

\begin{problem}
    What would it mean for an object to be 0-dimensional?  Give some examples and explain your reasoning.
    
    \begin{solution}
        The best way to think about this right now is to think of a zero-dimensional object as just a dot, or as something you would count with numbers.  An example is the number of floors in the house above, or the number of windows in the house.
    \end{solution}
    

\end{problem}



\begin{problem}
    What would it mean for an object to be 4-dimensional?  What about $n$-dimensional, for $n$ greater than 4?
       \begin{solution}
        This is a bonus question!  Four dimensions is just like three, except there is an extra way to move around.  This is really hard to imagine, though, since we live in a 3-dimensional world.
    \end{solution}
 
\end{problem}

\pagebreak
\begin{instructorNotes}
    {\bf Main goal:} This activity is intended to introduce the notion of dimension.  
    
    {\bf Overall picture:} Dimension is a very difficult idea! We begin exploring it with examples, definitions, and various misconceptions.
    
    Aspects of a house:
    \begin{itemize}
            \item This question should be straightforward for students, though students occasionally have trouble correctly identifying which dimension coincides with the aspects they suggest.  In this case, suggest that students go back and test their answers to question 1 using their answer to question 2.
        
       \item  This should be the ``observe'' phase of students' investigation.  Here, they are exploring what dimension means in order to be able to define it in the next question.  The idea that students should identify four aspects in each dimension should help them to come up with creative answers beyond the obvious.
    \end{itemize}
    
    Defining dimension:
    \begin{itemize}
            \item Students will hopefully come up with the idea of dimension as being ``how many directions can we move in this space?''  This is subtle, of course, and its subtleties should be pointed out.  Two dimensions actually produces an infinite number of ways to move, though all of these ways can be built out of two.  In essence, we're describing the vector space nature of $\mathbb{R}^2$, but not using this terminology.
        
        \item Answers will vary for the differences between kindergarten and fourth grade.  Kindergarteners are taught about the idea of dimension in basic ways, in that they should be able to describe the difference between a flat (2D) object and one that takes up space (3D).  By fourth grade, students should have a deeper understanding of dimension, so this could be a more complicated discussion.
    \end{itemize}
    
    
    Sally and Suzi:
    \begin{itemize}
    	\item These problems contain misconceptions we have seen from our students in the past. They are here to draw out any misconceptions students have along these lines.
	\item Be sure to hear arguments for both sides before indicating which is correct! Students should try to test their ideas against their definitions of dimension.
	\item Don't be afraid to bring up related examples. For instance: what dimension is the paint on my wall? What dimension is a piece of yarn? Does it change dimension if I throw it on the floor?
	\item This is also a good place to point out that many of our examples of dimension are actually misleading! We think of a piece of paper as 2D, but it is actually a 3D object! It's very hard to visualize just a 2D object, with no thickness at all.
    \end{itemize}
    
    The clock and the couch:
    \begin{itemize}
\item This question should give students a chance to demonstrate their understanding of dimension.  There should be multiple aspects that need to be measured for each item: for instance, you would need to measure the (one-dimensional) height of the ceiling in order to determine if the grandfather clock will fit.  However, you should also measure the (one-dimensional) width of the door to determine if you will be able to fit the clock through the door!  Alternatively, you could consider the square footage of the side of the clock and compare with the square footage of the door, but a direct comparison here is less useful.  This is a good concept to discuss with students.
\end{itemize}

Zero-dimensional:
    \begin{itemize}
        \item Students should use this question to test their definitions of dimension from the previous problem.  They should still be able to make sense out of dimension in the 0-dimensional case, whether they consider a 0-dimensional object to be an angle (does not change under scaling) or a point (no directions in which to move).
        \item We don't usually get this far in discussion, which is okay!
    \end{itemize}




Four-dimensional:  
    \begin{itemize}
        \item This problem is optional, and should not be discussed in class.  This is a more ``mathematically oriented'' problem for advanced students to use to extend their ideas if they have a very solid grasp on the first three problems.
    \end{itemize}
    

    
    
    {\bf Good language:} Be sure to fully explore students ideas; they may surprise you at some point! Don't assume you know what they are thinking.
    
{\bf Suggested timing:} 10-15 minutes in groups, followed by discussion. The entire class can be used for this activity, or you can turn to homework questions if you have extra time.


    \end{instructorNotes}   

%problem 3 comments (grandfather clock)  


    
\end{document}