\documentclass[nooutcomes,noauthor]{ximera}
%\documentclass{ximera}
\usepackage{gensymb}
\usepackage{tabularx}
\usepackage{mdframed}
\usepackage{pdfpages}
%\usepackage{chngcntr}

\let\problem\relax
\let\endproblem\relax

\newcommand{\property}[2]{#1#2}




\newtheoremstyle{SlantTheorem}{\topsep}{\fill}%%% space between body and thm
 {\slshape}                      %%% Thm body font
 {}                              %%% Indent amount (empty = no indent)
 {\bfseries\sffamily}            %%% Thm head font
 {}                              %%% Punctuation after thm head
 {3ex}                           %%% Space after thm head
 {\thmname{#1}\thmnumber{ #2}\thmnote{ \bfseries(#3)}} %%% Thm head spec
\theoremstyle{SlantTheorem}
\newtheorem{problem}{Problem}[]

%\counterwithin*{problem}{section}



%%%%%%%%%%%%%%%%%%%%%%%%%%%%Jenny's code%%%%%%%%%%%%%%%%%%%%

%%% Solution environment
%\newenvironment{solution}{
%\ifhandout\setbox0\vbox\bgroup\else
%\begin{trivlist}\item[\hskip \labelsep\small\itshape\bfseries Solution\hspace{2ex}]
%\par\noindent\upshape\small
%\fi}
%{\ifhandout\egroup\else
%\end{trivlist}
%\fi}
%
%
%%% instructorIntro environment
%\ifhandout
%\newenvironment{instructorIntro}[1][false]%
%{%
%\def\givenatend{\boolean{#1}}\ifthenelse{\boolean{#1}}{\begin{trivlist}\item}{\setbox0\vbox\bgroup}{}
%}
%{%
%\ifthenelse{\givenatend}{\end{trivlist}}{\egroup}{}
%}
%\else
%\newenvironment{instructorIntro}[1][false]%
%{%
%  \ifthenelse{\boolean{#1}}{\begin{trivlist}\item[\hskip \labelsep\bfseries Instructor Notes:\hspace{2ex}]}
%{\begin{trivlist}\item[\hskip \labelsep\bfseries Instructor Notes:\hspace{2ex}]}
%{}
%}
%% %% line at the bottom} 
%{\end{trivlist}\par\addvspace{.5ex}\nobreak\noindent\hung} 
%\fi
%
%


\let\instructorNotes\relax
\let\endinstructorNotes\relax
%%% instructorNotes environment
\ifhandout
\newenvironment{instructorNotes}[1][false]%
{%
\def\givenatend{\boolean{#1}}\ifthenelse{\boolean{#1}}{\begin{trivlist}\item}{\setbox0\vbox\bgroup}{}
}
{%
\ifthenelse{\givenatend}{\end{trivlist}}{\egroup}{}
}
\else
\newenvironment{instructorNotes}[1][false]%
{%
  \ifthenelse{\boolean{#1}}{\begin{trivlist}\item[\hskip \labelsep\bfseries {\Large Instructor Notes: \\} \hspace{\textwidth} ]}
{\begin{trivlist}\item[\hskip \labelsep\bfseries {\Large Instructor Notes: \\} \hspace{\textwidth} ]}
{}
}
{\end{trivlist}}
\fi


%% Suggested Timing
\newcommand{\timing}[1]{{\bf Suggested Timing: \hspace{2ex}} #1}




\hypersetup{
    colorlinks=true,       % false: boxed links; true: colored links
    linkcolor=blue,          % color of internal links (change box color with linkbordercolor)
    citecolor=green,        % color of links to bibliography
    filecolor=magenta,      % color of file links
    urlcolor=cyan           % color of external links
}


\title{Unit rates}




\begin{document}

\begin{abstract}
In this activity, let's work with the following story situation: Kriti is decorating boxes with ribbon. She knows that for every 7 inches of ribbon she uses, she needs to use 2 grams of glue. 
\end{abstract}
\maketitle


\begin{problem}
To warm up: how do you know that Kriti's decorating problem involves a ratio? Be as specific as possible.

\end{problem}


\begin{problem}
How many grams of glue will Kriti need for 1 inch of ribbon? Solve this problem using a picture or reasoning. Remember to include the meaning of any operations you use!
\end{problem}


\begin{problem}
How many grams of glue will Kriti need for 13 inches of ribbon? Explain why your answer to the previous problem makes this problem easier to answer. What is a unit rate and how did we use it here?

\end{problem}



\begin{problem}
Challenge: use a unit rate to find how many inches of ribbon Kriti will use with 7.3 grams of glue.
\end{problem}

\newpage


Now, we want to find how many ounces of glue Kriti will use with 21 inches of ribbon. Maybe you already know the answer to this question, but I set up this proportion.
\[
\frac{2}{7} = \frac{x}{21}
\]

\begin{problem}
What does the fraction $\frac{2}{7}$ or the division problem $2 \div 7$ mean in relationship to Kriti's decorating problem? (Is this related to our work on the previous page?)
\end{problem}


\begin{problem}
What does the fraction $\frac{x}{21}$ or the division problem $x \div 21$ mean in relationship to Kriti's decorating problem?
\end{problem}

\begin{problem}
Why should $\frac{2}{7}$ be {\em equal} to $\frac{x}{21}$ in this story situation?
\end{problem}














\newpage
\begin{instructorNotes}

{\bf Main Goal:} Students practice with unit rates and extend to proportions.

{\bf Overall Picture:}

For the first page:
\begin{itemize}
	\item Students should remember and use the meaning of a ratio for problem 1. 
	\item Problems 2 and 3 should be a pair worked together, where students first find how much glue per inch and then scale this for 13 inches. In discussion you want to bring up the fact that this is a unit rate because it's how much goes with 1 unit of something else, and you also want to point out that it is how we connect the ratio $2:7$ to the fraction $\frac{2}{7}$. Pay particular attention to the wholes in this case.
	\item In discussion it's also good to point out that ratios are two numbers (2 and 7 in this case) where fractions are a single number. Many students at this stage are still thinking of fractions as two numbers (numerator and denominator) rather than as locations on the number line.
	\item The challenge problem requires students to find the other unit rate here; this should be discussed if you have time but it's okay to skip it. In your wrap-up, if you don't work through this problem you should mention that there are two unit rates for a ratio and we want to use them for different purposes. You might ask students how to distinguish when we know we want each particular unit rate.
\end{itemize}

For the second page:
\begin{itemize}
	\item The fractions themselves should be seen as examples of the unit rate for the ratio. The $\frac{2}{7}$ should come up quickly from the previous page, but the $\frac{x}{21}$ might be harder to see. Division might be easier in that case: we have a total of $x$ ounces of glue, and we can think of spreading that into the $21$ groups represented by the inches. This would give us how much glue per inch, which is exactly what we get with the $\frac{2}{7}$ of a gram per inch.
	\item You might encourage students to draw pictures to think about the unit rates. This idea may be new to students, so bringing back other examples (without variables) can help.
	\item The fact that the two fractions are equal should be emphasized! They are equal because they are each the unit rate for the same ratio.
	\item Finally, once we have translated our ratios into the realm of fractions, we can work with them as if they are fractions. In that case, cross-multiplication is a short-cut for finding a common denominator. This should be part of your wrap-up discussion on this page.
	\item If you have extra time, you can give students another proportion example to explain and practice.
\end{itemize}




{\bf Good language:} Seeing multiple solution methods is still really nice here, especially on page 1.





\timing{Give students about 10 minutes to work on the first page in groups, then spend about 15 minutes with groups presenting their work. You can do problem 1 together as a class for warm-up. Then give students about 5 minutes to work on the second page, and again have groups present their work. Use the last 5-10 minutes to illustrate why unit rates are important and/or to go back to the challenge problem.}

\end{instructorNotes}



\end{document}