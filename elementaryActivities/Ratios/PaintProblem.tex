\documentclass{ximera}

\graphicspath{
  {./}
  {graphics/}
  {../graphics/}
}

\usepackage{chngcntr}

\let\question\relax
\let\endquestion\relax




\newtheoremstyle{SlantTheorem}{\topsep}{\fill}%%% space between body and thm
%\newtheoremstyle{SlantTheorem}{\topsep}{\topsep}%%% space between body and thm
 {\slshape}                      %%% Thm body font
 {}                              %%% Indent amount (empty = no indent)
 {\bfseries\sffamily}            %%% Thm head font
 {}                              %%% Punctuation after thm head
 {3ex}                           %%% Space after thm head
 {\thmname{#1}\thmnumber{ #2}\thmnote{ \bfseries(#3)}}%%% Thm head spec
\theoremstyle{SlantTheorem}
\newtheorem{question}{Question}
\counterwithin*{question}{section}



\let\instructorNotes\relax
\let\endinstructorNotes\relax
%%% instructorNotes environment
\ifhandout
\newenvironment{instructorNotes}[1][false]%
{%
\def\givenatend{\boolean{#1}}\ifthenelse{\boolean{#1}}{\begin{trivlist}\item}{\setbox0\vbox\bgroup}{}
}
{%
\ifthenelse{\givenatend}{\end{trivlist}}{\egroup}{}
}
\else
\newenvironment{instructorNotes}[1][false]%
{%
  \ifthenelse{\boolean{#1}}{\begin{trivlist}\item[\hskip \labelsep\bfseries {\Large Instructor Notes: \\} \hspace{\textwidth} ]}
{\begin{trivlist}\item[\hskip \labelsep\bfseries {\Large Instructor Notes: \\} \hspace{\textwidth} ]}
{}
}
{\end{trivlist}}
\fi


%% Suggested Timing
\newcommand{\timing}[1]{{\bf Suggested Timing: \hspace{2ex}} #1}

\title{A Paint Problem}

\begin{document}
\begin{abstract}
\end{abstract}
\maketitle


A certain shade of green paint is made by mixing blue paint with yellow paint in a ratio of $2$ to $5$.  For each of the problems below, use the same shade of green paint.  Solve each problem in at least two different ways (but no proportions, please!).  Explain your solutions.

\begin{problem}
If you use $20$ cups of blue paint, how many cups of yellow paint will you need?
\end{problem}

\begin{problem}
If you use $40$ liters of yellow paint, how many liters of blue paint will you need?
\end{problem}

\begin{problem}
If you want to make $350$ gallons of green paint, how many gallons of blue paint and how many gallons of yellow paint will you need?
\end{problem}

\begin{problem}
If you use $47$ pails of yellow paint, how many (same-size) pails of blue paint will you need?
\end{problem}

\begin{problem}
If you want to make $8$ buckets of green paint, how many (same-size) buckets of blue paint and how many (same-size) buckets of yellow paint will you need?
\end{problem}

\newpage

\begin{instructorNotes}

{\bf Main goal:} Students use both the variable parts perspective as well as the variable (multiple) batches perspective to solve ratio problems, but they are not required to use this language.

{\bf Overall picture:}

These are our first examples of ratio problems. Students should be encouraged to use their creativity and try to come up with as many solutions as they can. However, in the interest of timing, it may be good to suggest that students solve all the problems, then come back and find a second way (or third! or fourth!) to solve the problems. You may also suggest that students try a different method for each problem.

Groups can tend to struggle with thinking of a ``second way'' since they haven't had a lot of experience yet. However, we will often see quite a bit of variability between the groups. It can be good to stop after everyone is ready to talk about the first problem and have groups present a variety of strategies so that they have more to choose from in the later problems. If some solution strategies are missing (for instance everyone draws a picture or everyone uses multiple batches) then the instructor should include their own solution of a different type, making it very clear that this instructor-led type is not required (only an additional option!)

At this point in the course, students can be surprised to find that they do not need to draw a picture to solve these problems. A picture isn't required, but if they don't use one, they need to clearly explain using the meaning of operations. For instance, if they multiply to solve the problem, a clear statement about what is functioning as one group and one object should be included. See the text for examples.

Double number lines may start to come up here, or you may also see the idea of a unit rate (1.5 yellow per blue, for ex) popping up in this activity.  If you see either of these around the room, be sure to have these students present.

%Goals for this activity:
%\begin{itemize}
%
%    \item Students see their first examples of ratio problems where the answers are fractions.
%\end{itemize}

When we make green paint, be on the lookout for confusion about the units. Two gallons of yellow plus three gallons of blue should make five gallons of green (not one). Some students are confused by this idea.

Note that these problems get more difficult as the activity progresses. Try to have some students present on the later problems!


{\bf Good language:} The names ``multiple batches'' and ``variable parts'' aren't required unless the students find that helpful. Similarly, we don't need to give names to things like ``strip diagrams" or ``double number lines''.  You can also ask students in the discussion to practice saying how they know this situation with the paint contains a ratio (using the definitions from the previous class). For instance, they should be able to say that the relationship of blue to yellow stays the same no matter how many batches (groups) we make, or no matter how many units we have per batch (objects per group).  Moving forward, students will benefit a lot from practicing with the meaning of multiplication here and specifically identifying their groups and objects. 

{\bf Suggested timing:} Give students about 15 minutes to work through the problems, and then have students present solutions for the rest of the class.  Note that you can interrupt the work time for a shorter presentation time if you'd like to discuss just problem 1, and then go back to more work time and more presentations. Wrap up by pointing out the various strategies we've seen.
\end{instructorNotes}



\end{document}