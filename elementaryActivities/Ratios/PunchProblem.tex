\documentclass{ximera}
\usepackage{gensymb}
\usepackage{tabularx}
\usepackage{mdframed}
\usepackage{pdfpages}
%\usepackage{chngcntr}

\let\problem\relax
\let\endproblem\relax

\newcommand{\property}[2]{#1#2}




\newtheoremstyle{SlantTheorem}{\topsep}{\fill}%%% space between body and thm
 {\slshape}                      %%% Thm body font
 {}                              %%% Indent amount (empty = no indent)
 {\bfseries\sffamily}            %%% Thm head font
 {}                              %%% Punctuation after thm head
 {3ex}                           %%% Space after thm head
 {\thmname{#1}\thmnumber{ #2}\thmnote{ \bfseries(#3)}} %%% Thm head spec
\theoremstyle{SlantTheorem}
\newtheorem{problem}{Problem}[]

%\counterwithin*{problem}{section}



%%%%%%%%%%%%%%%%%%%%%%%%%%%%Jenny's code%%%%%%%%%%%%%%%%%%%%

%%% Solution environment
%\newenvironment{solution}{
%\ifhandout\setbox0\vbox\bgroup\else
%\begin{trivlist}\item[\hskip \labelsep\small\itshape\bfseries Solution\hspace{2ex}]
%\par\noindent\upshape\small
%\fi}
%{\ifhandout\egroup\else
%\end{trivlist}
%\fi}
%
%
%%% instructorIntro environment
%\ifhandout
%\newenvironment{instructorIntro}[1][false]%
%{%
%\def\givenatend{\boolean{#1}}\ifthenelse{\boolean{#1}}{\begin{trivlist}\item}{\setbox0\vbox\bgroup}{}
%}
%{%
%\ifthenelse{\givenatend}{\end{trivlist}}{\egroup}{}
%}
%\else
%\newenvironment{instructorIntro}[1][false]%
%{%
%  \ifthenelse{\boolean{#1}}{\begin{trivlist}\item[\hskip \labelsep\bfseries Instructor Notes:\hspace{2ex}]}
%{\begin{trivlist}\item[\hskip \labelsep\bfseries Instructor Notes:\hspace{2ex}]}
%{}
%}
%% %% line at the bottom} 
%{\end{trivlist}\par\addvspace{.5ex}\nobreak\noindent\hung} 
%\fi
%
%


\let\instructorNotes\relax
\let\endinstructorNotes\relax
%%% instructorNotes environment
\ifhandout
\newenvironment{instructorNotes}[1][false]%
{%
\def\givenatend{\boolean{#1}}\ifthenelse{\boolean{#1}}{\begin{trivlist}\item}{\setbox0\vbox\bgroup}{}
}
{%
\ifthenelse{\givenatend}{\end{trivlist}}{\egroup}{}
}
\else
\newenvironment{instructorNotes}[1][false]%
{%
  \ifthenelse{\boolean{#1}}{\begin{trivlist}\item[\hskip \labelsep\bfseries {\Large Instructor Notes: \\} \hspace{\textwidth} ]}
{\begin{trivlist}\item[\hskip \labelsep\bfseries {\Large Instructor Notes: \\} \hspace{\textwidth} ]}
{}
}
{\end{trivlist}}
\fi


%% Suggested Timing
\newcommand{\timing}[1]{{\bf Suggested Timing: \hspace{2ex}} #1}




\hypersetup{
    colorlinks=true,       % false: boxed links; true: colored links
    linkcolor=blue,          % color of internal links (change box color with linkbordercolor)
    citecolor=green,        % color of links to bibliography
    filecolor=magenta,      % color of file links
    urlcolor=cyan           % color of external links
}

\title{A Punch Problem}

\begin{document}
\begin{abstract}
    Very important information: the word ``ratio'' is not allowed until after this activity is over!
\end{abstract}

\maketitle

Mr. Johnson's class made some punch for a class party using $1$ bottle of Sprite and $3$ bottles (all the same size) of fruit juice.  On the day of the party, Mr. Johnson decided to invite Ms. Perry's class to join, and so more punch is needed.  Someone suggests adding $2$ more bottles of Sprite, and $2$ more bottles of fruit juice to the mix.  The class wonders: will adding $2$ bottles of Sprite and $2$ bottles of juice make the punch taste any different?  Opinions are divided!

\begin{problem}
\begin{enumerate}
    \item Explain why some of the children in the class think that the taste will be the same.  Draw pictures as needed to support your argument.
    \item Explain why some of the children in the class think that the taste will be different.  Draw pictures as needed to support your argument.
\end{enumerate}
\end{problem}

\begin{problem}
Split your group into two factions: the ``same taste" faction and the ``different taste" faction.  Keep your personal opinions to yourself: you do not need to agree with the point of view of your faction!  Prepare arguments and stage a debate.  Be prepared to report to the class:
\begin{itemize}
    \item the most convincing argument from each faction.
    \item the debate's ``winner'' (and how you decided that they ``won'' -- based on the mathematics involved, not your opinion!)
\end{itemize}
\end{problem}

\begin{problem}
As we have worked through these issues, has your thinking changed in any way?  What is one thing you learned today, and one question you have after today's class?
\end{problem}

\newpage
\begin{instructorNotes}

{\bf Main Goal:} We play with an example of a ratio in order to understand how ratios work and build an appropriate definition.

{\bf Overall picture:}

Here we introduce the idea of ratios by bringing up one of the most common misconceptions our students struggle with on the topic of ratios.  We have found over the years that when we discuss problems like this one, students who have the ``same taste'' opinion can be too timid to express their opinions in class for everyone to hear.  But, we would like to intentionally introduce this line of thinking for all students to consider.

This activity works nicely as a ``debate'' of sorts: have students argue for one position or another (which does not necessarily have to align with their personal opinion here!) It can work nicely to split up the groups in class, with about half the groups arguing for ``same taste'' and half the groups arguing for ``different taste'', then having each group present one argument. It can also work nicely to have each individual group think about arguments for both the ``same taste'' and ``different taste'' and then have groups report one argument for each. Other options can also work!

%On our calendar, this activity usually falls right after an exam, so the ``debate'' portion of the activity can be a fun distraction from last night's exam.  Jenny has done this debate as a whole class - splitting up the entire class into two teams, and having them come up with arguments for their ``side'', but students will likely be more on-task with smaller groups than that.  It might work nicely to combine two of the usual groups so there is a normal-size group for each ``faction'', and then the groups can debate.  As the instructor, Jenny thinks it's fun to walk around and play ``devil's advocate'', especially focusing on groups who are very sure of the answer.  She thinks it's particularly fun to things like, ``Well, we did the same thing to both sides!  That always keeps things the same.''  

Here are some points to look for in the ``debate''.
\begin{itemize}
    \item Extending to bigger numbers.  If we added $100$ bottles of Sprite and $100$ bottles of fruit juice, most students can tell that now the taste will be different.
    \item Thinking about what fraction of the original mix is juice.  This is tricky, but some students sort it out nicely.
    \item Talking about recipes.  We've done enough of this with our fraction work that students are getting a sense of what it means to follow a recipe, even if they don't have a lot of experience with baking.
    \item Number lines can be a nice defense for the ``same taste'' group: we are essentially shifting on a number line, and the distance between the Sprite and juice is unchanged.  We just need to sort out whether that affects the taste.
    \item Also for the ``same taste'' camp, there's a nice argument to be made along the lines of ``Well, we did the same thing to both sides!  That always keeps things the same.''  
\end{itemize}

%Students struggle to not use the word ``ratio'' in this activity.  The students who have seen this type of problem before have to really think about what they mean when they say things like, ``of course they don't taste the same, they don't have the same ratio!''  We want to use this debate and discussion to determine what a ratio actually is, and how we could identify one in a problem.  After this discussion is over, we step back and make this definition together.

After all of the arguments have been presented in Problems 1 and 2, take any remaining time to focus on which arguments are most convincing, or any counter arguments that the class would like to make. Have the class talk about which side should ``win'' the debate. We want to resolve this discussion carefully so that the idea that the ``different taste" is the correct argument comes mostly from the students, not the instructor.

Finally, we want to wrap up the class by introducing the definition of a ratio. You have several options. One way to talk about ratios generally is to say ``a ratio is a fixed relationship between two quantities, where the relationship stays the same but the individual quantities can change''. The textbook also includes two definitions for ratios, called the ``variable batches'' perspective and the ``variable parts'' perspective. It may be a great idea to introduce all three. Students who are asked to give the definition of a ratio on a later assignment can use any of the above. If you have remaining time, explore how the ``different taste'' arguments are connected to these definitions of a ratio.

{\bf Good language:} Remember to encourage students to not use the word ``ratio'' until we've defined it at the end! This can be a challenge for some students, so encourage them to be patient with themselves.


{\bf Suggested Timing:} This activity takes a whole class period. Give students 15 minutes in their groups to develop their arguments, then use about 25 minutes to have students present their arguments and discuss. Then, use the last 15 minutes of class for introducing the definition of a ratio and potentially applying it to some of the ``different taste'' arguments. Problem 3 makes a great exit ticket participation assignment if you'd like to use the last 5 minutes of class to have students submit their answers to that question in Carmen or otherwise.



\end{instructorNotes}





\end{document}