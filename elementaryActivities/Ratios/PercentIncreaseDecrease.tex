\documentclass[noauthor,nooutcomes]{ximera}

\usepackage{gensymb}
\usepackage{tabularx}
\usepackage{mdframed}
\usepackage{pdfpages}
%\usepackage{chngcntr}

\let\problem\relax
\let\endproblem\relax

\newcommand{\property}[2]{#1#2}




\newtheoremstyle{SlantTheorem}{\topsep}{\fill}%%% space between body and thm
 {\slshape}                      %%% Thm body font
 {}                              %%% Indent amount (empty = no indent)
 {\bfseries\sffamily}            %%% Thm head font
 {}                              %%% Punctuation after thm head
 {3ex}                           %%% Space after thm head
 {\thmname{#1}\thmnumber{ #2}\thmnote{ \bfseries(#3)}} %%% Thm head spec
\theoremstyle{SlantTheorem}
\newtheorem{problem}{Problem}[]

%\counterwithin*{problem}{section}



%%%%%%%%%%%%%%%%%%%%%%%%%%%%Jenny's code%%%%%%%%%%%%%%%%%%%%

%%% Solution environment
%\newenvironment{solution}{
%\ifhandout\setbox0\vbox\bgroup\else
%\begin{trivlist}\item[\hskip \labelsep\small\itshape\bfseries Solution\hspace{2ex}]
%\par\noindent\upshape\small
%\fi}
%{\ifhandout\egroup\else
%\end{trivlist}
%\fi}
%
%
%%% instructorIntro environment
%\ifhandout
%\newenvironment{instructorIntro}[1][false]%
%{%
%\def\givenatend{\boolean{#1}}\ifthenelse{\boolean{#1}}{\begin{trivlist}\item}{\setbox0\vbox\bgroup}{}
%}
%{%
%\ifthenelse{\givenatend}{\end{trivlist}}{\egroup}{}
%}
%\else
%\newenvironment{instructorIntro}[1][false]%
%{%
%  \ifthenelse{\boolean{#1}}{\begin{trivlist}\item[\hskip \labelsep\bfseries Instructor Notes:\hspace{2ex}]}
%{\begin{trivlist}\item[\hskip \labelsep\bfseries Instructor Notes:\hspace{2ex}]}
%{}
%}
%% %% line at the bottom} 
%{\end{trivlist}\par\addvspace{.5ex}\nobreak\noindent\hung} 
%\fi
%
%


\let\instructorNotes\relax
\let\endinstructorNotes\relax
%%% instructorNotes environment
\ifhandout
\newenvironment{instructorNotes}[1][false]%
{%
\def\givenatend{\boolean{#1}}\ifthenelse{\boolean{#1}}{\begin{trivlist}\item}{\setbox0\vbox\bgroup}{}
}
{%
\ifthenelse{\givenatend}{\end{trivlist}}{\egroup}{}
}
\else
\newenvironment{instructorNotes}[1][false]%
{%
  \ifthenelse{\boolean{#1}}{\begin{trivlist}\item[\hskip \labelsep\bfseries {\Large Instructor Notes: \\} \hspace{\textwidth} ]}
{\begin{trivlist}\item[\hskip \labelsep\bfseries {\Large Instructor Notes: \\} \hspace{\textwidth} ]}
{}
}
{\end{trivlist}}
\fi


%% Suggested Timing
\newcommand{\timing}[1]{{\bf Suggested Timing: \hspace{2ex}} #1}




\hypersetup{
    colorlinks=true,       % false: boxed links; true: colored links
    linkcolor=blue,          % color of internal links (change box color with linkbordercolor)
    citecolor=green,        % color of links to bibliography
    filecolor=magenta,      % color of file links
    urlcolor=cyan           % color of external links
}


\title{Percent Increase and Decrease}

\begin{document}
\begin{abstract} In this activity, we'll discuss several types of solutions to percent problems. You don't need to be able to use all of these methods, so feel free to skip around if you are stuck! We would like you to focus on knowing how to solve these problems two different ways. \end{abstract}
\maketitle



\begin{problem}
Camdyn is making a recipe which calls for $5$ cups of flour, but she wants to make her recipe $30\%$ larger than the original. How many cups of flour should she use?

\vskip 0.1in
Camdyn starts to solve this problem by drawing the following picture. Label this picture carefully and finish her solution.
\begin{center}
	\begin{tikzpicture}
		\draw[thick] (3,0)--(10,0)--(10,3)--(3,3);
		\draw[thick, fill=gray] (0,0)--(3,0)--(3,3)--(0,3)--(0,0);
		\foreach \i in {1, 2, ..., 9} \draw[thick] (\i,0)--(\i,3);
		\node at (5, -0.5) {All the flour};
		\node at (1.5, 3.5) {$30\% = \frac{30}{100} = \frac{3}{10}$};
	\end{tikzpicture}
\end{center}
\end{problem}



\begin{problem}
A farmer is looking over a yearly weather forecast, and decides to plant $35\%$ fewer acres of soybeans this year. If the farmer planted $1600$ acres of soybeans last year, how many acres do they plan to plant this year?

\vskip 0.1in

Gabriella solves the problem above with the following equation. Answer the questions which follow the equation, and then finish Gabriella's solution.

\[
0.35 \cdot 1600 = x
\]

\begin{enumerate}
	\item Why does Gabriella use multiplication? What are the groups and objects here?
	\item Why does the decimal $0.35$ appear, and how is this related to our original definition of percents? (You should need to discuss the connection between fractions and percents here!)
	\item Use Gabriella's work to finish solving this problem.
\end{enumerate}
\end{problem}


\begin{problem}
The price of a shirt was marked up 15$\%$.  The new price is $\$$69.00.  What was the price of the shirt before it was marked up?

\vskip 0.1in

Adrian solves the problem above with the following equation. Answer the questions which follow the equation, and then finish Adrian's solution.
\[
(1+0.15)\cdot x = 69
\]
\begin{enumerate}
	\item Why does Adrian add $1+0.15$ in her solution? What does this mean?
	\item Why does Adrian use multiplication? What are the groups and objects here?
	\item Why is the $x$ on the other side of the equation from Gabriella's equation above? What does this mean for the solution?
\end{enumerate}

\end{problem}



\begin{problem}
There were 80 gallons of gas in a tank.  Now there are only 50 gallons left.  By what percent did the amount of gas in the tank decrease?  
\vskip 0.1in
Millie solves the problem above with the following equation. Answer the questions which follow the equation, and then finish Millie's solution.
\[
\frac{50}{80} = \frac{x}{100}
\]
\begin{enumerate}
	\item What does the fraction $\frac{50}{80}$ mean in this context? What is the whole for this fraction?
	\item What does the fraction $\frac{x}{100}$ mean in this context? What is the whole for this fraction?
	\item Why are these two fractions equal to one another?
	\item Use Millie's work to finish solving the problem.
\end{enumerate}


\end{problem}



\begin{problem}
Jimin works in a factory inspecting automobile parts. In a typical day, Jimin identifies $8\%$ of the automobile parts as defective. If Jimin inspects $3250$ items, how many of those items do you expect Jimin to identify as defective?
\vskip 0.1in
Here is Layla's reasoning to solve this problem. \em{``$8\%$ means $8$ out of every $100$. So $3250$ items is $32$ hundreds, and half of another hundred. So I expect Jimin to identify 32 eights and half of another eight as defective.''}
\vskip 0.1in
Do you agree with Layla's reasoning? Why or why not? Can her reasoning be used to solve the original problem? Explain your work, including the meaning of any operations you use.

\end{problem}




\begin{problem}
 Whoopiedoo makeup used to be sold in 4-ounce tubes.  Now it is sold in 5-ounce tubes for the same price.  Ashlee says the label should read ``25$\%$ more free," while Carly thinks it should read ``20$\%$ more free."  Who is right, who is wrong, and why?  Help the person with the incorrect answer understand her error and explain how to correct it.   Draw pictures to aid your explanation.

\end{problem}



\begin{problem}
The problems in this activity are percent increase or percent decrease problems. How are they different from the percent problems we solved at the beginning of the semester? Do you notice any similarities between the problems we've solved today, or any themes from our work today?
\end{problem}






\newpage

\begin{instructorNotes}
{\bf Main Goal:} We would like students to connect our work with fractions, decimals, and rates in the context of percent problems. Making these connections can be tough, so don't worry if you don't make it through very many problems!

{\bf Overall Picture:}
\begin{itemize}
	\item Students investigate several types of reasoning about percent increase or decrease problems. We begin with a familiar solution using a picture and reasoning about the meaning of percents. Students should be encouraged to use these types of solutions since they are most likely the most comfortable with this type of method. We aim to help students develop a second method for solving percent problems and be able to explain each part of their methods.
	\item As we add percent increase and decrease to our original percent work, students should identify in their solution where they are using the meaning of addition or subtraction. Does the problem ask them to combine or take away?
	\item With the first problem, we want to emphasize the usual ideas with percents as fractions. How do we see the definition of percent, and perhaps make an equivalent fraction? What is the whole for our fraction? We can now also use the meaning of multiplication or division (as appropriate) to discuss how much should go in each of the boxes. For instance, here we have 10 boxes (groups) which contain 5 cups of flour (objects) total. This is a HMIEG division situation.
	\item The algebraic solutions should be the types of things students may have used to solve percent problems before coming to the course. You want to focus on the meaning of the operations present here. For instance, in the farming problem, one whole group should be all of the land, and one object is one acre. That way, we are looking for $0.35$ of a group, where there are 1600 acres in a full group. This reasoning will be tough for students!
	\item When the percent appears as a decimal, you want to review the idea that $0.1 = \frac{1}{10}$ and as an extension $0.01 = \frac{1}{100}$. So, our percents (as fractions over 100) can easily be written as decimals.
	\item For the proportional reasoning problem, you can either think about these directly as fractions (the whole is the original amount in the tank, cut into 80 equal pieces or cut into 100 equal pieces) and consider that you're finding equivalent fractions here, or you can consider both of these fractions to be unit rates for the percent thought of as a ratio. This latter idea is likely more complicated for students, but the factory inspection problem might help students to think this way.
	\item The factory inspection problem and the Whoopiedoo makeup problem can both be skipped in the interest of time if needed.
	\item In the factory inspection problem, you'll want to help students with the idea that ``8 out of every 100'' is a ratio ($8:100$), and so the corresponding unit rate is $\frac{8}{100}$, which means that for every 1 item, $\frac{8}{100}$ of it should be ``defective''. This doesn't really make sense with whole items, but it is the meaning of the unit rate. Once we connect the ratio to the unit rate, $\frac{8}{100}$ also means $8\%$ according to our definition of percents. 
	\item The makeup problem is a common misconception about percent increase.
	\item The final problem is to help students begin to take notes and give students some ideas to contribute during  wrap-up. When you wrap up, you want to draw students' attention to the key issue here: determining the relevant whole for the increase or decrease. The crucial idea is that the whole is the ORIGINAL object in each situation.
	\item The textbook also includes ``percent tables'' as a way to organize these problems. Students can use such a mechanism, but would need to explain each operation in detail.

	
\end{itemize}


{\bf Suggested Timing:} You may want to work through these problems one at a time. Give students about 5 minutes to work on a particular problem, then discuss for about 10 minutes. Repeat this process as you have time, and use the last 5 minutes to wrap up. 

Alternately, you can work on the first problem together as a class since this should be mostly review. Then split up the remaining problems amongst the groups, giving about 10 minutes or so to solve the problem and prepare to discuss their solution with the class. Then have each group present their work. Use the last 5 minutes or so for overall observations and wrap-up.

\end{instructorNotes}

\end{document}