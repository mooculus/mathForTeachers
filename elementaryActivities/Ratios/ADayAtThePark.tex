%\documentclass[nooutcomes,noauthor,handout]{ximera}
\documentclass{ximera}
\usepackage{gensymb}
\usepackage{tabularx}
\usepackage{mdframed}
\usepackage{pdfpages}
%\usepackage{chngcntr}

\let\problem\relax
\let\endproblem\relax

\newcommand{\property}[2]{#1#2}




\newtheoremstyle{SlantTheorem}{\topsep}{\fill}%%% space between body and thm
 {\slshape}                      %%% Thm body font
 {}                              %%% Indent amount (empty = no indent)
 {\bfseries\sffamily}            %%% Thm head font
 {}                              %%% Punctuation after thm head
 {3ex}                           %%% Space after thm head
 {\thmname{#1}\thmnumber{ #2}\thmnote{ \bfseries(#3)}} %%% Thm head spec
\theoremstyle{SlantTheorem}
\newtheorem{problem}{Problem}[]

%\counterwithin*{problem}{section}



%%%%%%%%%%%%%%%%%%%%%%%%%%%%Jenny's code%%%%%%%%%%%%%%%%%%%%

%%% Solution environment
%\newenvironment{solution}{
%\ifhandout\setbox0\vbox\bgroup\else
%\begin{trivlist}\item[\hskip \labelsep\small\itshape\bfseries Solution\hspace{2ex}]
%\par\noindent\upshape\small
%\fi}
%{\ifhandout\egroup\else
%\end{trivlist}
%\fi}
%
%
%%% instructorIntro environment
%\ifhandout
%\newenvironment{instructorIntro}[1][false]%
%{%
%\def\givenatend{\boolean{#1}}\ifthenelse{\boolean{#1}}{\begin{trivlist}\item}{\setbox0\vbox\bgroup}{}
%}
%{%
%\ifthenelse{\givenatend}{\end{trivlist}}{\egroup}{}
%}
%\else
%\newenvironment{instructorIntro}[1][false]%
%{%
%  \ifthenelse{\boolean{#1}}{\begin{trivlist}\item[\hskip \labelsep\bfseries Instructor Notes:\hspace{2ex}]}
%{\begin{trivlist}\item[\hskip \labelsep\bfseries Instructor Notes:\hspace{2ex}]}
%{}
%}
%% %% line at the bottom} 
%{\end{trivlist}\par\addvspace{.5ex}\nobreak\noindent\hung} 
%\fi
%
%


\let\instructorNotes\relax
\let\endinstructorNotes\relax
%%% instructorNotes environment
\ifhandout
\newenvironment{instructorNotes}[1][false]%
{%
\def\givenatend{\boolean{#1}}\ifthenelse{\boolean{#1}}{\begin{trivlist}\item}{\setbox0\vbox\bgroup}{}
}
{%
\ifthenelse{\givenatend}{\end{trivlist}}{\egroup}{}
}
\else
\newenvironment{instructorNotes}[1][false]%
{%
  \ifthenelse{\boolean{#1}}{\begin{trivlist}\item[\hskip \labelsep\bfseries {\Large Instructor Notes: \\} \hspace{\textwidth} ]}
{\begin{trivlist}\item[\hskip \labelsep\bfseries {\Large Instructor Notes: \\} \hspace{\textwidth} ]}
{}
}
{\end{trivlist}}
\fi


%% Suggested Timing
\newcommand{\timing}[1]{{\bf Suggested Timing: \hspace{2ex}} #1}




\hypersetup{
    colorlinks=true,       % false: boxed links; true: colored links
    linkcolor=blue,          % color of internal links (change box color with linkbordercolor)
    citecolor=green,        % color of links to bibliography
    filecolor=magenta,      % color of file links
    urlcolor=cyan           % color of external links
}


\title{A Day at the Park}




\begin{document}

\begin{abstract}
\end{abstract}
\maketitle

Solve the problems below by drawing a picture, or by reasoning about quantities. No proportions, please! We'll get to those soon.  If you finish early, go back and solve each problem a second way!


\begin{problem}
There's a car wash happening at the park today!  The recipe for car wash soap says to mix $\frac{1}{4}$ of a cup of solution with $1 \frac{1}{2}$ gallons of water.  Ty wasn't paying attention, so he poured an extra gallon of water into the bucket.  In other words, right now the bucket has $\frac{1}{4}$ of a cup of solution and $2 \frac{1}{2}$ gallons of water.  How much solution should Ty add to the bucket to fix this mistake?

\end{problem}


\begin{problem}
Danielle and Judah are having a picnic at the part today!  When they arrived at the park, the ratio of cookies in Danielle's basket to cookies in Judah's basket was $2:7$.  After Judah gave Danielle $5$ of his cookies, each of them had the same amount.  How many cookies did each person have in their basket originally?
\end{problem}

\begin{problem}
A scooter is traveling along a path at the park at a steady pace of $\frac{1}{3}$ mile every $\frac{5}{2}$ minutes.  How far will the scooter go in $18$ minutes?  How long will it take the scooter to go $3$ miles?
\end{problem}

\newpage
\begin{instructorNotes}

{\bf Main Goal:} Students practice their ratio problem solving.

{\bf Overall Picture:}

These problems ramp up the difficulty of those in ``A Paint Problem''. Students should use the same problem-solving strategies (reasoning about operations, drawing a strip diagram or other picture, making a double number line, going through 1, etc) as the previous activity, but now the problems involve more difficult numbers and ask more difficult questions. Be sure that students read carefully!
Goals for this activity:
\begin{itemize}
    \item The first problem can be complicated because of the mis-matched units. Some students will ask how many cups are in a gallon. You might respond by asking if they would need that information to, say, double a recipe at home that called for 3 cups of flour and half a teaspoon of baking soda (or another similar but more familiar idea).
    \item The first problem can bring up the idea of ``going through 1'' if that idea hasn't already occurred. One natural solution path is to find out how much soap goes with one gallon. You don't have to name this strategy now, though -- we will have more time for it later.
    \item The second problem is often solved by guessing and checking. You might also suggest that the students try to draw a picture.
    \item The third problem is nicely suited for a double number line if you haven't introduced that, yet. But the problem can also be solved with a picture or reasoning about operations. Going through 1 is another nice strategy here. 
\end{itemize}

Overall, encourage students to be creative and really think about the problems at hand. We'd like to see more than one solution for each problem!

{\bf Good language:} Continue to help students identify how they know there is a ratio in each of these situations (using the definitions discussed during ``A Punch Problem''. Even though this isn't asked on the activity sheet, it's good practice for the types of questions we will eventually ask.



\timing{Give students about 20 minutes to work in groups on these problems. Have groups present their work on each problem. Take your time with these presentations, and try to see multiple solution strategies for each.}

\end{instructorNotes}



\end{document}