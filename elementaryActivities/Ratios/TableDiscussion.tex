\documentclass[nooutcomes,noauthor,handout]{ximera}

\usepackage{gensymb}
\usepackage{tabularx}
\usepackage{mdframed}
\usepackage{pdfpages}
%\usepackage{chngcntr}

\let\problem\relax
\let\endproblem\relax

\newcommand{\property}[2]{#1#2}




\newtheoremstyle{SlantTheorem}{\topsep}{\fill}%%% space between body and thm
 {\slshape}                      %%% Thm body font
 {}                              %%% Indent amount (empty = no indent)
 {\bfseries\sffamily}            %%% Thm head font
 {}                              %%% Punctuation after thm head
 {3ex}                           %%% Space after thm head
 {\thmname{#1}\thmnumber{ #2}\thmnote{ \bfseries(#3)}} %%% Thm head spec
\theoremstyle{SlantTheorem}
\newtheorem{problem}{Problem}[]

%\counterwithin*{problem}{section}



%%%%%%%%%%%%%%%%%%%%%%%%%%%%Jenny's code%%%%%%%%%%%%%%%%%%%%

%%% Solution environment
%\newenvironment{solution}{
%\ifhandout\setbox0\vbox\bgroup\else
%\begin{trivlist}\item[\hskip \labelsep\small\itshape\bfseries Solution\hspace{2ex}]
%\par\noindent\upshape\small
%\fi}
%{\ifhandout\egroup\else
%\end{trivlist}
%\fi}
%
%
%%% instructorIntro environment
%\ifhandout
%\newenvironment{instructorIntro}[1][false]%
%{%
%\def\givenatend{\boolean{#1}}\ifthenelse{\boolean{#1}}{\begin{trivlist}\item}{\setbox0\vbox\bgroup}{}
%}
%{%
%\ifthenelse{\givenatend}{\end{trivlist}}{\egroup}{}
%}
%\else
%\newenvironment{instructorIntro}[1][false]%
%{%
%  \ifthenelse{\boolean{#1}}{\begin{trivlist}\item[\hskip \labelsep\bfseries Instructor Notes:\hspace{2ex}]}
%{\begin{trivlist}\item[\hskip \labelsep\bfseries Instructor Notes:\hspace{2ex}]}
%{}
%}
%% %% line at the bottom} 
%{\end{trivlist}\par\addvspace{.5ex}\nobreak\noindent\hung} 
%\fi
%
%


\let\instructorNotes\relax
\let\endinstructorNotes\relax
%%% instructorNotes environment
\ifhandout
\newenvironment{instructorNotes}[1][false]%
{%
\def\givenatend{\boolean{#1}}\ifthenelse{\boolean{#1}}{\begin{trivlist}\item}{\setbox0\vbox\bgroup}{}
}
{%
\ifthenelse{\givenatend}{\end{trivlist}}{\egroup}{}
}
\else
\newenvironment{instructorNotes}[1][false]%
{%
  \ifthenelse{\boolean{#1}}{\begin{trivlist}\item[\hskip \labelsep\bfseries {\Large Instructor Notes: \\} \hspace{\textwidth} ]}
{\begin{trivlist}\item[\hskip \labelsep\bfseries {\Large Instructor Notes: \\} \hspace{\textwidth} ]}
{}
}
{\end{trivlist}}
\fi


%% Suggested Timing
\newcommand{\timing}[1]{{\bf Suggested Timing: \hspace{2ex}} #1}




\hypersetup{
    colorlinks=true,       % false: boxed links; true: colored links
    linkcolor=blue,          % color of internal links (change box color with linkbordercolor)
    citecolor=green,        % color of links to bibliography
    filecolor=magenta,      % color of file links
    urlcolor=cyan           % color of external links
}


\title{Table discussion}

\begin{document}
\begin{abstract} 
\end{abstract}
\maketitle

\begin{problem}
Mixing fruit juice and sparkling water in a ratio of 3 to 7 makes a delicious punch.  Draw pictures or reason about quantities to answer the following questions about this same recipe of punch.
\begin{enumerate}
    \item How many liters of juice and water will you use if you make $20$ liters of punch?
    \item How many liters of juice will you use if you use $\frac{7}{4}$ of a liter of water?  How many liters of punch will be made?
    \item How many liters of water will you use if you use $1$ liter of juice?  How many liters of punch will be made?
    \item How many liters of juice will you use if you use $1$ liter of water?  How many liters of punch will be made?
    \item How many liters of juice and water will you use if you make $1$ liter of punch?
\end{enumerate}
\end{problem}
\vfill
TURN THE PAGE \newpage
\begin{problem} 
Use your reasoning about quantities from the previous problem to fill out the following table for this same recipe of punch.  


\setlength{\tabcolsep}{23pt}
\renewcommand\arraystretch{3}
\begin{tabular}{|c|c|c|c|c|c|c|} \hline
Liters of Juice & 3  & & & 1 & & \\  \hline
Liters of water & 7  & & $\frac{7}{4}$  & & 1  & \\  \hline
Liters of punch & 10 & 20  & & & & 1  \\  \hline
\end{tabular}


What do you notice about your table?  Make as many observations as possible.
\end{problem}

\newpage
\begin{problem}
If you mix 3/4 cup of red paint with 2/3 cup of yellow paint to make an orange paint, then how many cups of red paint and how many cups of yellow paint will you need if you want to make 15 cups of the same shade of orange?  Reason about quantities and the meaning of multiplication (and maybe division), drawing pictures as necessary, to fill out the following ratio table. 


\setlength{\tabcolsep}{25pt}
\renewcommand\arraystretch{3}
\begin{tabular}{|c|c|c|c|c|} \hline
Cups of red & $\frac34$  & & &  \\  \hline
Cups of yellow & $\frac23$  &  & & \\  \hline
Cups of orange &  & 17  & 1 & 15  \\  \hline
\end{tabular}



\end{problem}


\newpage
\begin{instructorNotes}

{\bf Main goal:} We introduce the technique of using a ratio table to solve problems, and identify and define unit rates.

{\bf Overall picture:}

This activity encourages the strategy of ``going through one'' in which we have been given a ratio and the desired amount of one element (or the total amount of the mixture) and we want to know the amounts of the other elements that are needed to maintain the same ratio taste, color, demographic breakdown, etc.   Ratio tables help us to organize this information, and also help us to observe the multiplicative structure between making batches of various sizes. Multiplying by the reciprocal of the given amount of the one element then opens the door, due to the multiplicative identity, to using multiplication to find the desired amounts of the other elements.

\begin{itemize}
	\item The first problem gives us an opportunity to get more practice with solving ratio problems using reasoning. Students should draw pictures or reason about operations to solve these problems. If you are doing this activity in the large class, it may be helpful to split up the parts amongst the groups and have each group present a different part so that you can see many different strategies. Students are often still practicing basic skills at this stage.
	\item If we have discussed the first problem well, we can just enter the answers into the table in Problem 2. As you enter them, be sure to define this as a ``ratio table''. Then, give students a few minutes to make observations about the table.  Any observations are good here and should be encouraged! When you discuss, draw students' attention to the unit rates in the table: $\frac53$ of a liter of water per every liter of juice (and vice versa). Define the unit rate using the definition from the textbook, connect it to the ``going through 1 strategy'' if you have already named that, and notice that this unit rate connects the ratio (two numbers) to the fraction (a single number). 
	\item Students may struggle to connect their reasoning with strip diagrams to the entries in the table. You may want to have a student with a more algebraic solution explain their thinking, or use a student's diagram to point out the algebraic solution.
	\item Use problem 3 to showcase the power of the unit rate: when we have a unit rate, we can find any other values of the ratio using multiplication. Students should already be familiar with the use of multiplication with ratios  (because repeated or fractional copies of the given recipe gives the same taste, color, etc.), but continue to have them practice explaining with the meaning of multiplication. In general, in assessments, we will still expect students to explain the multiplication using groups and objects per group for at least one of the columns in their table.
	\item One strategy that can be useful for these problems is thinking about a ``new recipe'' for the same ratio. For instance, in problem 3 the initial recipe is $\frac34$ cups red with $\frac23$ cups blue, but in the next line of the table we can think of the new recipe as $9$ cups red with $8$ cups blue, and now use that as our basic recipe for the meaning of multiplication. We don't need to return to the first column every time, but we can use any of the other columns as our ``one batch'' to make copies of.
	\item As you move from one column of the table to the next, it tends to be easier to think about multiplying by fractional amounts instead of dividing. For instance, to get the third column of answers in the table in problem 3, we think about making $\frac{1}{17}$ of a batch of the ratio in the second column, so we take $\frac{1}{17} \times 8$ and $\frac{1}{17} \times 9$. This is easier for most students than a HMG division interpretation and recognizing that $9 \div 17 = \frac{9}{17}$.
	\item Students may also struggle to remember to explain their thinking once they get comfortable using a ratio table. As with any ``calculation recipe'', once we get calculating, we frequently stop thinking! Encourage students in all of your discussions to identify the groups and objects per group for all of their multiplications, as we've noted.
	\item It is usually a bit more natural to take the ``multiple batches'' perspective with a ratio table, where each column represents the number of items you have in some number $N$ of batches. However, it's also possible to think about a ratio table from the variable parts perspective, where the original recipe gives us the number of parts, and the $N$ we multiply by tells us how many objects per part. For instance, in the $3:5$ ratio with punch, we can think of the 3 and 5 as our numbers of groups. When we move to the column showing the $1: 5/3$ ratio, we are thinking of placing $\frac13$ of a liter per box, and this $\frac13$ is the number we multiply by to get the new column. 
     \item  If you have extra time, it can be worthwhile to discuss situations where it doesn't make sense to talk about the total amount of ``mixture'' (e.g., when the elements have different units, such as time and distance). You may also want to connect the ratio table strategy with double number lines if you have already discussed them.

\end{itemize}

{\bf Good language:} It can be helpful to call any ratio a ``recipe'', even if the two objects in question aren't typically found in the kitchen (or even mixed together). This lets the students still talk about the groups as batches of their recipe, which is helpful for understanding the multiplication.

{\bf Suggested Timing:}  Give students 5-10 minutes to work on Problem 1, and then have students present for the next 10-15 minutes. As a conclusion, fill out the table in Problem 2 together. Then give students about 3-4 minutes to make observations. Discuss their observations and introduce the unit rate idea, and finish with however much time you have left in the class for Problem 3.

\end{instructorNotes}
\end{document}