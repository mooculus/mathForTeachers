\documentclass{ximera}


\graphicspath{
  {./}
  {graphics/}
  {../graphics/}
}

\usepackage{chngcntr}

\let\question\relax
\let\endquestion\relax




\newtheoremstyle{SlantTheorem}{\topsep}{\fill}%%% space between body and thm
%\newtheoremstyle{SlantTheorem}{\topsep}{\topsep}%%% space between body and thm
 {\slshape}                      %%% Thm body font
 {}                              %%% Indent amount (empty = no indent)
 {\bfseries\sffamily}            %%% Thm head font
 {}                              %%% Punctuation after thm head
 {3ex}                           %%% Space after thm head
 {\thmname{#1}\thmnumber{ #2}\thmnote{ \bfseries(#3)}}%%% Thm head spec
\theoremstyle{SlantTheorem}
\newtheorem{question}{Question}
\counterwithin*{question}{section}



\let\instructorNotes\relax
\let\endinstructorNotes\relax
%%% instructorNotes environment
\ifhandout
\newenvironment{instructorNotes}[1][false]%
{%
\def\givenatend{\boolean{#1}}\ifthenelse{\boolean{#1}}{\begin{trivlist}\item}{\setbox0\vbox\bgroup}{}
}
{%
\ifthenelse{\givenatend}{\end{trivlist}}{\egroup}{}
}
\else
\newenvironment{instructorNotes}[1][false]%
{%
  \ifthenelse{\boolean{#1}}{\begin{trivlist}\item[\hskip \labelsep\bfseries {\Large Instructor Notes: \\} \hspace{\textwidth} ]}
{\begin{trivlist}\item[\hskip \labelsep\bfseries {\Large Instructor Notes: \\} \hspace{\textwidth} ]}
{}
}
{\end{trivlist}}
\fi


%% Suggested Timing
\newcommand{\timing}[1]{{\bf Suggested Timing: \hspace{2ex}} #1}


\title{Shampoo, Rinse, Repeat}
\author{Vic Ferdinand, Betsy McNeal, Jenny Sheldon}

\begin{document}
\begin{abstract} \end{abstract}
\maketitle

\begin{instructorIntro}
This activity connects our two historical languages of ``partial numbers'' (fractions and decimals) and then translation methods between them.  At the end, students should not only be able to translate from fractions to decimals and (terminating and repeating) decimals to fractions, but be able to logically tell why all fractions must either terminate or repeat as decimals and then make the conclusion that fractions don't represent all decimals.

There are some good challenge problems associated with this content.  For instance, ask what fractions have a terminating representation in base six or in base twelve.

{\bf Suggested Timing:} This activity is allotted 3-4 full class periods.  Usually, \#1-3 take one class period, \#4 takes the second class, \#5 the third, and \#6 the fourth, with extra time needed to put together the logical relationships involved.  In the past, we have had to omit \#7 and \#8 in the interest of saving time.  
\end{instructorIntro}

\begin{problem}

When writing the following fractions in decimal notation, which have a
``terminating'' and which have a ``non-terminating'' decimal?


\vspace*{0.5cm}

\begin{tabular}{c c c c c c c c c c}

$\displaystyle \frac{1}{2}$, & $\displaystyle \frac{1}{3}$, & $\displaystyle \frac{1}{4}$, & $\displaystyle \frac{1}{5}$, & $\displaystyle \frac{1}{9}$, & $\displaystyle \frac{1}{10}$, & $\displaystyle \frac{1}{11}$, & $\displaystyle \frac{1}{12}$, & $\displaystyle \frac{1}{13}$, & $\displaystyle \frac{1}{15}$,\\
\\
$\displaystyle \frac{1}{20}$, & $\displaystyle \frac{1}{24}$, & $\displaystyle \frac{1}{25}$, & $\displaystyle \frac{1}{28}$, & $\displaystyle \frac{1}{32}$, & $\displaystyle \frac{1}{35}$, & $\displaystyle \frac{1}{40}$, & $\displaystyle \frac{1}{42}$, & $\displaystyle \frac{1}{48}$, & $\displaystyle \frac{1}{88}$,\\
\\
$\displaystyle \frac{1}{96}$, & $\displaystyle \frac{1}{125}$, & $\displaystyle \frac{1}{160}$, & $\displaystyle \frac{1}{625}$, & $\displaystyle \frac{1}{6250}$.\\

\end{tabular}
\end{problem}

\begin{problem}
Can you find a pattern in your results from $\#1$? See if you can
conjecture a rule by which we can predict if a fraction will have a
terminating or non-terminating decimal representation.

\end{problem}


\begin{problem}
Argue why a fraction that fits the criteria from $\#2$ for a
terminating decimal must terminate. Argue why a fraction that does not
fit the criteria from $\#2$ for a terminating decimal must not
terminate.
\end{problem}


\begin{problem}
Take a fraction that does not have a terminating decimal
representation (say, $\frac{4}{7}$) and use long division to find its
decimal representation.  What type of pattern occurred?  Why did this
happen?  Was this a special case or will this happen for all fractions
with a non-terminating decimal representation?  Why?

\end{problem}

\begin{problem}
Using the facts that $\frac{1}{9}=0.\overline{1}$,
$\frac{1}{99}=0.\overline{01}$, $\frac{1}{999}=0.\overline{001}$,
etc., find a fraction representation for the following numbers:

\begin{enumerate}
\item $0.\overline{7}$
\item $0.\overline{357}$
\item $0.\overline{4598}$
\item $0.23\overline{4598}$
\item $23.\overline{459}$
\item $76.\overline{214}$
\end{enumerate}

\begin{instructorNotes}
In \#5, the last logical connection and translation is made (repeating decimals to fractions).  If time is short, Doing (b) and (d) will usually take care of the general process.  Another approach is to treat them like infinite geometric series, but this is usually more difficult (Perhaps, at the very least, showing, via geometric series, why the first two examples wind up being $\frac19$ and $\frac{1}{99}$ could be helpful and close the ``reasoning gap'' left here).
\end{instructorNotes}
\end{problem}

\begin{problem}
We've now established ($\#1$--4) that all fractions can be written in
decimal notation.  Can all decimals be written in fraction notation,
i.e. in the form \[ \frac{\text{whole number}}{\text{nonzero whole
    number}}?\] If so, argue why.  If not, list some that cannot and
explain why they are not fractions.

\end{problem}

\begin{problem}
Can$\sqrt{2}$ be written as a fraction, i.e., in the form 
\[ 
\frac{\text{whole number}}{\text{nonzero whole number}}?
\] 
Argue why or why not. 
\begin{instructorNotes}
Students can often mechanically repeat an argument for this type of proof after seeing it, but it is apparent that most do not understand it (perhaps the notion of a proof by contradiction threw them).  Only a few survived the general case.  Students also find it difficult to grasp the connection between irrational as a ``non-repeating decimal'' (\#5) and irrational as the square root of a prime (i.e., not equal to $a/b$).
\end{instructorNotes}

\end{problem}

\begin{problem}
If $p$ is a prime number, can $\sqrt{p}$ be written as a fraction,
i.e., in the form 
\[ 
\frac{\text{whole number}}{\text{nonzero whole number}}?
\] 
Argue why or why not.
\end{problem} 






\end{document}