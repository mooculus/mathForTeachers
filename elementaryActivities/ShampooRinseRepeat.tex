\documentclass{ximera}

\usepackage{gensymb}
\usepackage{tabularx}
\usepackage{mdframed}
\usepackage{pdfpages}
%\usepackage{chngcntr}

\let\problem\relax
\let\endproblem\relax

\newcommand{\property}[2]{#1#2}




\newtheoremstyle{SlantTheorem}{\topsep}{\fill}%%% space between body and thm
 {\slshape}                      %%% Thm body font
 {}                              %%% Indent amount (empty = no indent)
 {\bfseries\sffamily}            %%% Thm head font
 {}                              %%% Punctuation after thm head
 {3ex}                           %%% Space after thm head
 {\thmname{#1}\thmnumber{ #2}\thmnote{ \bfseries(#3)}} %%% Thm head spec
\theoremstyle{SlantTheorem}
\newtheorem{problem}{Problem}[]

%\counterwithin*{problem}{section}



%%%%%%%%%%%%%%%%%%%%%%%%%%%%Jenny's code%%%%%%%%%%%%%%%%%%%%

%%% Solution environment
%\newenvironment{solution}{
%\ifhandout\setbox0\vbox\bgroup\else
%\begin{trivlist}\item[\hskip \labelsep\small\itshape\bfseries Solution\hspace{2ex}]
%\par\noindent\upshape\small
%\fi}
%{\ifhandout\egroup\else
%\end{trivlist}
%\fi}
%
%
%%% instructorIntro environment
%\ifhandout
%\newenvironment{instructorIntro}[1][false]%
%{%
%\def\givenatend{\boolean{#1}}\ifthenelse{\boolean{#1}}{\begin{trivlist}\item}{\setbox0\vbox\bgroup}{}
%}
%{%
%\ifthenelse{\givenatend}{\end{trivlist}}{\egroup}{}
%}
%\else
%\newenvironment{instructorIntro}[1][false]%
%{%
%  \ifthenelse{\boolean{#1}}{\begin{trivlist}\item[\hskip \labelsep\bfseries Instructor Notes:\hspace{2ex}]}
%{\begin{trivlist}\item[\hskip \labelsep\bfseries Instructor Notes:\hspace{2ex}]}
%{}
%}
%% %% line at the bottom} 
%{\end{trivlist}\par\addvspace{.5ex}\nobreak\noindent\hung} 
%\fi
%
%


\let\instructorNotes\relax
\let\endinstructorNotes\relax
%%% instructorNotes environment
\ifhandout
\newenvironment{instructorNotes}[1][false]%
{%
\def\givenatend{\boolean{#1}}\ifthenelse{\boolean{#1}}{\begin{trivlist}\item}{\setbox0\vbox\bgroup}{}
}
{%
\ifthenelse{\givenatend}{\end{trivlist}}{\egroup}{}
}
\else
\newenvironment{instructorNotes}[1][false]%
{%
  \ifthenelse{\boolean{#1}}{\begin{trivlist}\item[\hskip \labelsep\bfseries {\Large Instructor Notes: \\} \hspace{\textwidth} ]}
{\begin{trivlist}\item[\hskip \labelsep\bfseries {\Large Instructor Notes: \\} \hspace{\textwidth} ]}
{}
}
{\end{trivlist}}
\fi


%% Suggested Timing
\newcommand{\timing}[1]{{\bf Suggested Timing: \hspace{2ex}} #1}




\hypersetup{
    colorlinks=true,       % false: boxed links; true: colored links
    linkcolor=blue,          % color of internal links (change box color with linkbordercolor)
    citecolor=green,        % color of links to bibliography
    filecolor=magenta,      % color of file links
    urlcolor=cyan           % color of external links
}


\title{Shampoo, Rinse, Repeat}
\author{Vic Ferdinand, Betsy McNeal, Jenny Sheldon}

\begin{document}
\begin{abstract} \end{abstract}
\maketitle


\begin{problem}\label{Shampoo1}

When writing the following fractions in decimal notation, which have a
``terminating'' and which have a ``non-terminating'' decimal?


\vspace*{0.5cm}

\begin{tabular}{c c c c c c c c c c}

$\displaystyle \frac{1}{2}$, & $\displaystyle \frac{1}{3}$, & $\displaystyle \frac{1}{4}$, & $\displaystyle \frac{1}{5}$, & $\displaystyle \frac{1}{9}$, & $\displaystyle \frac{1}{10}$, & $\displaystyle \frac{1}{11}$, & $\displaystyle \frac{1}{12}$, & $\displaystyle \frac{1}{13}$, & $\displaystyle \frac{1}{15}$,\\
\\
$\displaystyle \frac{1}{20}$, & $\displaystyle \frac{1}{24}$, & $\displaystyle \frac{1}{25}$, & $\displaystyle \frac{1}{28}$, & $\displaystyle \frac{1}{32}$, & $\displaystyle \frac{1}{35}$, & $\displaystyle \frac{1}{40}$, & $\displaystyle \frac{1}{42}$, & $\displaystyle \frac{1}{48}$, & $\displaystyle \frac{1}{88}$,\\
\\
$\displaystyle \frac{1}{96}$, & $\displaystyle \frac{1}{125}$, & $\displaystyle \frac{1}{160}$, & $\displaystyle \frac{1}{625}$, & $\displaystyle \frac{1}{6250}$.\\

\end{tabular}
\end{problem}

\begin{problem}\label{Shampoo2}
Can you find a pattern in your results from Problem \ref{Shampoo1}? See if you can
conjecture a rule by which we can predict if a fraction will have a
terminating or non-terminating decimal representation.

\end{problem}


\begin{problem}\label{Shampoo3}
Argue why a fraction that fits the criteria from Problem \ref{Shampoo2} for a
terminating decimal must terminate. Argue why a fraction that does not
fit the criteria from Problem \ref{Shampoo2} for a terminating decimal must not
terminate.
\end{problem}


\begin{problem} \label{Shampoo4}
Take a fraction that does not have a terminating decimal
representation (say, $\frac{4}{7}$) and use long division to find its
decimal representation.  What type of pattern occurred?  Why did this
happen?  Was this a special case or will this happen for all fractions
with a non-terminating decimal representation?  Why?

\end{problem}

\begin{problem}\label{Shampoo5}
Using the facts that $\frac{1}{9}=0.\overline{1}$,
$\frac{1}{99}=0.\overline{01}$, $\frac{1}{999}=0.\overline{001}$,
etc., find a fraction representation for the following numbers:

\begin{enumerate}
\item $0.\overline{7}$
\item $0.\overline{357}$
\item $0.\overline{4598}$
\item $0.23\overline{4598}$
\item $23.\overline{459}$
\item $76.\overline{214}$
\end{enumerate}


\end{problem}

\begin{problem} \label{Shampoo6}
We've now established (Problems \ref{Shampoo1} - \ref{Shampoo5}) that all fractions can be written in
decimal notation.  Can all decimals be written in fraction notation,
i.e. in the form \[ \frac{\text{whole number}}{\text{nonzero whole
    number}}?\] If so, argue why.  If not, list some that cannot and
explain why they are not fractions.

\end{problem}

\begin{problem}\label{Shampoo7}
Can $\sqrt{2}$ be written as a fraction, i.e., in the form 
\[ 
\frac{\text{whole number}}{\text{nonzero whole number}}?
\] 
Argue why or why not. 


\end{problem}

\begin{problem}\label{Shampoo8}
If $p$ is a prime number, can $\sqrt{p}$ be written as a fraction,
i.e., in the form 
\[ 
\frac{\text{whole number}}{\text{nonzero whole number}}?
\] 
Argue why or why not.
\end{problem} 

\newpage
\begin{instructorNotes}
This activity has students make connections between our two languages of ``partial numbers'' (fractions and decimals) and then create methods of translating between them.  At the end of this activity (which takes several class periods), students should be able to translate from fractions to decimals and 
%(terminating and repeating) 
decimals to fractions, and be able to justify that all fractions must either terminate or repeat as decimals.  Finally, students should be able to make the conclusion that not all decimals can be represented by fractions.

\begin{itemize}
\item There are some good challenge problems associated with this content.  For instance, ask what fractions have a terminating representation in base six or in base twelve.
\item In Problem \ref{Shampoo5}, the last logical connection and translation is made from repeating decimals to fractions.  If time is short, doing (b) and (d) will usually take care of the general process.  Another approach is to treat them like infinite geometric series, but this is usually more difficult. Perhaps showing, via geometric series, why the first two examples wind up being $\frac19$ and $\frac{1}{99}$ could be helpful and close the ``reasoning gap'' left here.
\item In problem \ref{Shampoo7}, students can often mechanically repeat an argument for this type of proof after seeing it, but it is apparent that most do not understand it (perhaps the notion of a proof by contradiction threw them).  Only a few survived the general case.  Students also find it difficult to grasp the connection between irrational as a ``non-repeating decimal'' (Problem \ref{Shampoo5}) and irrational as the square root of a prime (i.e., not equal to $a/b$).

\end{itemize}


{\bf Suggested Timing:} This activity is allotted 3-4 full class periods.  Usually, Problems \ref{Shampoo1} - \ref{Shampoo3} take one class period, Problem \ref{Shampoo4} takes the second class, Problem \ref{Shampoo5} the third, and Problem \ref{Shampoo6} the fourth, with extra time needed to put together the logical relationships involved.  In the past, we have had to omit Problem \ref{Shampoo7} and Problem \ref{Shampoo8} in the interest of saving time.  
\end{instructorNotes}





\end{document}