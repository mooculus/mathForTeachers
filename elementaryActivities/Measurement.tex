\documentclass{ximera}

\graphicspath{
  {./}
  {graphics/}
  {../graphics/}
}

\usepackage{chngcntr}

\let\question\relax
\let\endquestion\relax




\newtheoremstyle{SlantTheorem}{\topsep}{\fill}%%% space between body and thm
%\newtheoremstyle{SlantTheorem}{\topsep}{\topsep}%%% space between body and thm
 {\slshape}                      %%% Thm body font
 {}                              %%% Indent amount (empty = no indent)
 {\bfseries\sffamily}            %%% Thm head font
 {}                              %%% Punctuation after thm head
 {3ex}                           %%% Space after thm head
 {\thmname{#1}\thmnumber{ #2}\thmnote{ \bfseries(#3)}}%%% Thm head spec
\theoremstyle{SlantTheorem}
\newtheorem{question}{Question}
\counterwithin*{question}{section}



\let\instructorNotes\relax
\let\endinstructorNotes\relax
%%% instructorNotes environment
\ifhandout
\newenvironment{instructorNotes}[1][false]%
{%
\def\givenatend{\boolean{#1}}\ifthenelse{\boolean{#1}}{\begin{trivlist}\item}{\setbox0\vbox\bgroup}{}
}
{%
\ifthenelse{\givenatend}{\end{trivlist}}{\egroup}{}
}
\else
\newenvironment{instructorNotes}[1][false]%
{%
  \ifthenelse{\boolean{#1}}{\begin{trivlist}\item[\hskip \labelsep\bfseries {\Large Instructor Notes: \\} \hspace{\textwidth} ]}
{\begin{trivlist}\item[\hskip \labelsep\bfseries {\Large Instructor Notes: \\} \hspace{\textwidth} ]}
{}
}
{\end{trivlist}}
\fi


%% Suggested Timing
\newcommand{\timing}[1]{{\bf Suggested Timing: \hspace{2ex}} #1}

\title{Measurement}
\author{Vic Ferdinand, Betsy McNeal, Jenny Sheldon}

\begin{document}
\begin{abstract}
\end{abstract}
\maketitle



\begin{problem} \label{Measurement1}

Measure a 1-D aspect of your desk using an object you have in your backpack.  Explain your procedure and why your answer makes sense.  


\end{problem}


\begin{problem} \label{Measurement2}
Now consider this ``unit" of area.  (The unit is the shaded part of the drawing, not the entire grid.)

\begin{center}
    \begin{tikzpicture}[scale=0.5]
        \draw [step=1, help lines] (0,0) grid (4,4);
        \draw[fill=black] (1,1)--(2,1)--(2,2)--(3,2)--(3,3)--(1,3)--(1,1);
    \end{tikzpicture}
\end{center}
Explain how to measure the three shapes below using this unit of area.


\begin{center}
    \begin{tikzpicture}[scale=0.5]
        \draw [step=1, help lines] (0,0) grid (21,6);
        \draw[fill=red] (2,2)--(6,2)--(6,3)--(5,3)--(5,4)--(1,4)--(1,3)--(2,3)--(2,2);
        \node at (0.5, 5.5) {$A$};
        \draw[fill=yellow] (9,2)--(13,2)--(13,5)--(9,5)--(9,2);
        \node at (8.5, 5.5) {$B$};
        \draw[fill=blue] (18,1)--(19,1)--(19,3)--(20,3)--(20,4)--(19,4)--(19,5)--(18,5)--(18,4)--(16,4)--(16,3)--(18,3)--(18,1);
        \node at (17.5, 5.5) {$C$};
    \end{tikzpicture}
\end{center}
\end{problem}

\begin{problem} \label{Measurement3}
Using our unit of area, measure the area of this circle.

\begin{center}
    \begin{tikzpicture}[scale=0.5]
        \draw[step=1, help lines] (0,0) grid (10,10);
        \draw[ultra thick] (5,5) circle (4cm);
    \end{tikzpicture}
\end{center}





\end{problem}

\newpage

\begin{instructorNotes}
This activity is intended to introduce the idea of measuring with a non-standard unit.  We use non-standard units quite a bit in our course, so that students rely less on formulas they have memorized and more on the meaning of area and the procedure of measurement. Both length and area are considered in the activity, though this activity generally falls for us inside a unit about area. No prior experience with non-standard units is required for this activity.  By the time we work through this activity, however, we have discussed the procedure for measuring area in general, and have usually also used some non-standard units.
\begin{itemize}
    \item In Problem \ref{Measurement1}, we focus on the procedure, rather than the particular unit.  We ask students to share their results and discuss how they can know whose length is the greatest.
    \item In Problems \ref{Measurement2} and \ref{Measurement3}, we have seen students think about the question in various ways.  A common question students ask is whether the unit can be broken into smaller pieces.  We see strategies where the students fit the unit into the shapes as well as strategies using counting and division.
    \item We frequently have some students who report their answer in ``square units''.  We generally take some time in this activity to discuss (or remind students about) the meaning of a square unit. 
    \item In Problem \ref{Measurement3}, the most common methods we see are to count the total number of squares and divide by three, or to outline as many units as possible, and then try to collect leftover pieces.  Occasionally, students will try  measuring the diameter of the circle and using a previously-known formula for area.  We sometimes bring up this strategy and ask students to evaluate whether it makes sense and whether it gives us the correct answer.
\end{itemize}




\timing{We take about 10-15 minutes in groups, followed by 15-20 minutes of discussion.}

\end{instructorNotes}




\end{document}