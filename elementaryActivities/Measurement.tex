\documentclass[nooutcomes]{ximera}
\usepackage{gensymb}
\usepackage{tabularx}
\usepackage{mdframed}
\usepackage{pdfpages}
%\usepackage{chngcntr}

\let\problem\relax
\let\endproblem\relax

\newcommand{\property}[2]{#1#2}




\newtheoremstyle{SlantTheorem}{\topsep}{\fill}%%% space between body and thm
 {\slshape}                      %%% Thm body font
 {}                              %%% Indent amount (empty = no indent)
 {\bfseries\sffamily}            %%% Thm head font
 {}                              %%% Punctuation after thm head
 {3ex}                           %%% Space after thm head
 {\thmname{#1}\thmnumber{ #2}\thmnote{ \bfseries(#3)}} %%% Thm head spec
\theoremstyle{SlantTheorem}
\newtheorem{problem}{Problem}[]

%\counterwithin*{problem}{section}



%%%%%%%%%%%%%%%%%%%%%%%%%%%%Jenny's code%%%%%%%%%%%%%%%%%%%%

%%% Solution environment
%\newenvironment{solution}{
%\ifhandout\setbox0\vbox\bgroup\else
%\begin{trivlist}\item[\hskip \labelsep\small\itshape\bfseries Solution\hspace{2ex}]
%\par\noindent\upshape\small
%\fi}
%{\ifhandout\egroup\else
%\end{trivlist}
%\fi}
%
%
%%% instructorIntro environment
%\ifhandout
%\newenvironment{instructorIntro}[1][false]%
%{%
%\def\givenatend{\boolean{#1}}\ifthenelse{\boolean{#1}}{\begin{trivlist}\item}{\setbox0\vbox\bgroup}{}
%}
%{%
%\ifthenelse{\givenatend}{\end{trivlist}}{\egroup}{}
%}
%\else
%\newenvironment{instructorIntro}[1][false]%
%{%
%  \ifthenelse{\boolean{#1}}{\begin{trivlist}\item[\hskip \labelsep\bfseries Instructor Notes:\hspace{2ex}]}
%{\begin{trivlist}\item[\hskip \labelsep\bfseries Instructor Notes:\hspace{2ex}]}
%{}
%}
%% %% line at the bottom} 
%{\end{trivlist}\par\addvspace{.5ex}\nobreak\noindent\hung} 
%\fi
%
%


\let\instructorNotes\relax
\let\endinstructorNotes\relax
%%% instructorNotes environment
\ifhandout
\newenvironment{instructorNotes}[1][false]%
{%
\def\givenatend{\boolean{#1}}\ifthenelse{\boolean{#1}}{\begin{trivlist}\item}{\setbox0\vbox\bgroup}{}
}
{%
\ifthenelse{\givenatend}{\end{trivlist}}{\egroup}{}
}
\else
\newenvironment{instructorNotes}[1][false]%
{%
  \ifthenelse{\boolean{#1}}{\begin{trivlist}\item[\hskip \labelsep\bfseries {\Large Instructor Notes: \\} \hspace{\textwidth} ]}
{\begin{trivlist}\item[\hskip \labelsep\bfseries {\Large Instructor Notes: \\} \hspace{\textwidth} ]}
{}
}
{\end{trivlist}}
\fi


%% Suggested Timing
\newcommand{\timing}[1]{{\bf Suggested Timing: \hspace{2ex}} #1}




\hypersetup{
    colorlinks=true,       % false: boxed links; true: colored links
    linkcolor=blue,          % color of internal links (change box color with linkbordercolor)
    citecolor=green,        % color of links to bibliography
    filecolor=magenta,      % color of file links
    urlcolor=cyan           % color of external links
}

\title{Measurement}
\author{Vic Ferdinand, Betsy McNeal, Jenny Sheldon}

\begin{document}
\begin{abstract}
\end{abstract}
\maketitle



\begin{problem} \label{Measurement1}

Measure a 1-D aspect of your desk using an object you have in your backpack.  Explain your procedure and why your answer makes sense.  


\end{problem}


\begin{problem} \label{Measurement2}
Now consider this ``unit" of area.  (The unit is the shaded part of the drawing, not the entire grid.)

\begin{center}
    \begin{tikzpicture}[scale=0.5]
        \draw [step=1, help lines] (0,0) grid (4,4);
        \draw[fill=black] (1,1)--(2,1)--(2,2)--(3,2)--(3,3)--(1,3)--(1,1);
    \end{tikzpicture}
\end{center}
Explain how to measure the three shapes below using this unit of area.


\begin{center}
    \begin{tikzpicture}[scale=0.5]
        \draw [step=1, help lines] (0,0) grid (21,6);
        \draw[fill=red] (2,2)--(6,2)--(6,3)--(5,3)--(5,4)--(1,4)--(1,3)--(2,3)--(2,2);
        \node at (0.5, 5.5) {$A$};
        \draw[fill=yellow] (9,2)--(13,2)--(13,5)--(9,5)--(9,2);
        \node at (8.5, 5.5) {$B$};
        \draw[fill=blue] (18,1)--(19,1)--(19,3)--(20,3)--(20,4)--(19,4)--(19,5)--(18,5)--(18,4)--(16,4)--(16,3)--(18,3)--(18,1);
        \node at (17.5, 5.5) {$C$};
    \end{tikzpicture}
\end{center}
\end{problem}

\begin{problem} \label{Measurement3}
Using our unit of area, measure the area of this circle.

\begin{center}
    \begin{tikzpicture}[scale=0.5]
        \draw[step=1, help lines] (0,0) grid (10,10);
        \draw[ultra thick] (5,5) circle (4cm);
    \end{tikzpicture}
\end{center}





\end{problem}

\newpage

\begin{instructorNotes}
This activity is intended to introduce the idea of measuring with a non-standard unit.  Both length and area are considered in this activity, and this is the first activity in which our goal is to actually produce a measurement.  In our course, we have already discussed the basic ideas of measurement in ``I'm the Biggest Kid'' and dimension in ``Where Do We Live''.  We follow this activity with ``Try This On For Size'' where we measure with standard units.  Notice that we have chosen specifically to treat non-standard units before standard units.

We treat the concept of measurement with a strong emphasis on using non-standard units to measure.  This is an example of our theme of ``making the familiar strange'', or taking students out of their comfort zone so that they cannot just apply old knowledge or mathematical rules, but instead have to think critically about the matter at hand.  We have found students much more willing to focus on the cover-and-count meaning of measurement rather than just applying a formula when the students are working with units that they have never before used.  These non-standard units also help us to draw out general principles about measurement, again rather than simply formulae.

In this activity in particular, we are attempting to push back against the frequent statement made by our students that ``area is length times width''.  We will eventually see this formula, but we would like it to be a result of what is happening when we cover the object with our unit without gaps or overlaps in the copies of the unit.


\begin{itemize}
    \item In Problem \ref{Measurement1}, we focus on the procedure, rather than the particular unit.  We ask students to share their results and discuss how they can know whose length is the greatest.
    \item In Problems \ref{Measurement2} and \ref{Measurement3}, we have seen students think about the question in various ways.  A common question students ask is whether the unit can be broken into smaller pieces.  This question is common enough, and we like to emphasize this concept, so we bring up the question if it does not naturally occur.  We see strategies where the students fit the unit into the shapes as well as strategies using counting and division.
    \item We frequently have some students who report their answer in ``square units'', even though their unit is not a square.  We generally take some time in this activity to discuss (or remind students about) the meaning of a square unit. 
    \item In Problem \ref{Measurement3}, the most common methods we see are to count the total number of squares and divide by three, or to outline as many units as possible, and then try to collect leftover pieces.  Occasionally, students will try  measuring the diameter of the circle and using a previously-known formula for area.  We sometimes bring up this strategy and ask students to evaluate whether it makes sense in terms of covering with squares (or the idea that area is length times width) and whether it gives us the correct answer.
\end{itemize}




\timing{We take about 10-15 minutes in groups, followed by 15-20 minutes of discussion.}

\end{instructorNotes}




\end{document}