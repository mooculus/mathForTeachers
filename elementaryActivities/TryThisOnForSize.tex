\documentclass[handout]{ximera}
\usepackage{gensymb}
\usepackage{tabularx}
\usepackage{mdframed}
\usepackage{pdfpages}
%\usepackage{chngcntr}

\let\problem\relax
\let\endproblem\relax

\newcommand{\property}[2]{#1#2}




\newtheoremstyle{SlantTheorem}{\topsep}{\fill}%%% space between body and thm
 {\slshape}                      %%% Thm body font
 {}                              %%% Indent amount (empty = no indent)
 {\bfseries\sffamily}            %%% Thm head font
 {}                              %%% Punctuation after thm head
 {3ex}                           %%% Space after thm head
 {\thmname{#1}\thmnumber{ #2}\thmnote{ \bfseries(#3)}} %%% Thm head spec
\theoremstyle{SlantTheorem}
\newtheorem{problem}{Problem}[]

%\counterwithin*{problem}{section}



%%%%%%%%%%%%%%%%%%%%%%%%%%%%Jenny's code%%%%%%%%%%%%%%%%%%%%

%%% Solution environment
%\newenvironment{solution}{
%\ifhandout\setbox0\vbox\bgroup\else
%\begin{trivlist}\item[\hskip \labelsep\small\itshape\bfseries Solution\hspace{2ex}]
%\par\noindent\upshape\small
%\fi}
%{\ifhandout\egroup\else
%\end{trivlist}
%\fi}
%
%
%%% instructorIntro environment
%\ifhandout
%\newenvironment{instructorIntro}[1][false]%
%{%
%\def\givenatend{\boolean{#1}}\ifthenelse{\boolean{#1}}{\begin{trivlist}\item}{\setbox0\vbox\bgroup}{}
%}
%{%
%\ifthenelse{\givenatend}{\end{trivlist}}{\egroup}{}
%}
%\else
%\newenvironment{instructorIntro}[1][false]%
%{%
%  \ifthenelse{\boolean{#1}}{\begin{trivlist}\item[\hskip \labelsep\bfseries Instructor Notes:\hspace{2ex}]}
%{\begin{trivlist}\item[\hskip \labelsep\bfseries Instructor Notes:\hspace{2ex}]}
%{}
%}
%% %% line at the bottom} 
%{\end{trivlist}\par\addvspace{.5ex}\nobreak\noindent\hung} 
%\fi
%
%


\let\instructorNotes\relax
\let\endinstructorNotes\relax
%%% instructorNotes environment
\ifhandout
\newenvironment{instructorNotes}[1][false]%
{%
\def\givenatend{\boolean{#1}}\ifthenelse{\boolean{#1}}{\begin{trivlist}\item}{\setbox0\vbox\bgroup}{}
}
{%
\ifthenelse{\givenatend}{\end{trivlist}}{\egroup}{}
}
\else
\newenvironment{instructorNotes}[1][false]%
{%
  \ifthenelse{\boolean{#1}}{\begin{trivlist}\item[\hskip \labelsep\bfseries {\Large Instructor Notes: \\} \hspace{\textwidth} ]}
{\begin{trivlist}\item[\hskip \labelsep\bfseries {\Large Instructor Notes: \\} \hspace{\textwidth} ]}
{}
}
{\end{trivlist}}
\fi


%% Suggested Timing
\newcommand{\timing}[1]{{\bf Suggested Timing: \hspace{2ex}} #1}




\hypersetup{
    colorlinks=true,       % false: boxed links; true: colored links
    linkcolor=blue,          % color of internal links (change box color with linkbordercolor)
    citecolor=green,        % color of links to bibliography
    filecolor=magenta,      % color of file links
    urlcolor=cyan           % color of external links
}

\title{Try This On For Size}
\author{Vic Ferdinand \& Betsy McNeal \& Jenny Sheldon}

\begin{document}
\begin{abstract}
\end{abstract}
\maketitle

\begin{instructorIntro}
This activity is intended to help students deepen their understanding of measurement.  We begin with measuring length, then move to measuring area.  Volume can be briefly discussed at the end of the exercise.  You should help the students to see the similarities and differences between the types of measurement: in each case you choose a unit with the same dimension as the object you are measuring, then use that unit (or pieces of it) as you measure.  

It's important to try to build the idea that measurement is an iterative process through these exercises. In each case, point out the way that you are using the same unit over and over again, and filling up the space involved.  One way to outline the process of measurement: 1.  Choose an aspect to measure. (This relates to ``Where Do We Live?'')  2. Choose an appropriate unit.  (This relates to ``I'm The Biggest Kid!''   3.  Iterate your unit or compare the object with your unit.  (Dealt with in the current activity.)  4.  Record your answers.

It might be useful for this activity to have students work with a partner that they do not usually work with during class.

\timing{This activity should take about half a class period.  Have students work through questions 1 and 2 for about 10 minutes, then discuss.  Give students about 5-10 minutes to think about the remaining problems, and then discuss.}

\end{instructorIntro}


\begin{problem}
\hfill
\begin{enumerate}
\item By yourself, draw a line that is 6 cm long.  Next, write down detailed instructions for drawing the line that could be understood by someone who has not yet learned how to measure or how to use a ruler.  Find a partner, and have them follow your instructions exactly.  Did they draw the same line you drew?  Is their line 6 cm long?  
\item Repeat the above process, except this time choose a unit of length that is not standard.  (No centimeters, inches, etc!)  Can you use the same process you used in the first part to draw a line of length 6 units?
\end{enumerate}


\begin{solution}
Your instructions should be detailed, but easy to follow.  One way to do this is to put down your ruler, start at the mark called 1cm, and draw a line that ends at the mark called 7cm.  (Why is this?)
\end{solution}

\begin{instructorNotes}
As students are presenting their ideas to the class, as the instructor you should bring up a few misconceptions that children often have when using a ruler.  See Beckmann's Activity 11C.  You could draw a ruler on the board, draw a line from 2cm to 8cm, and then report that this line is 8cm long, or 7cm long.

When students choose their own units for the second part, you should discuss that many of the units they choose would be better {\em area} units than length units.  This can get very confusing both for our students and children!
\end{instructorNotes}
\end{problem}

\begin{problem}
With your partner, use your experiences in the previous question to write down detailed instructions for measuring the length of an object.  You may use 1 cm as a unit, but your procedure should also work for any other appropriate unit.

\begin{solution}
To measure an object, you should count the number of units that you can lay end-to-end without any gaps or overlaps between the units.
\end{solution}

\begin{instructorNotes}
This question is provided in order for students to organize their thoughts from the first question.  You should conclude your discussion by emphasizing the iterative nature of measuring length, and that we must use a 1-dimensional unit to measure 1-dimensional length.  You should discuss the ways in which we can see that a ruler is already an iterative tool, or discuss the way that we might build a ruler in the first place.
\end{instructorNotes}
\end{problem}

\begin{problem}
A {\it square inch} is a particular unit of area which looks like a square measuring 1 inch on each side.  By yourself, draw an object which has an area of 8 square inches, but is {\bf not} a rectangle.  After you're done, compare your answer with your partner's answer.  How many different answers are possible to this question?

\begin{solution}    
There are actually infinitely many different ways to draw such a shape!
\end{solution}
\begin{instructorNotes}
Students will have to be a little creative in order to break out of the rectangle mindset, here.  Have students draw as many different shapes as possible, which will hopefully include some shapes built from pieces of the units.  For instance, if someone cut up the 8 units into two equal pieces each, and arranged these smaller pieces.
\end{instructorNotes}
\end{problem}

\begin{problem}
Use your experiences in the previous question to write down a procedure for measuring the area of an object.  You may use 1 square inch as a unit, but your procedure should work for any other appropriate unit.  

\begin{solution}
To measure area, you count the number of area units it takes to exactly cover your shape, with no gaps or overlaps between the units.
\end{solution}
\begin{instructorNotes}
This discussion should proceed in a similar fashion to that about length.  Students should be encouraged again to see the iterative process of measuring the area, and should feel like measuring area is basically the same as measuring length, except in two dimensions instead of one.
\end{instructorNotes}
\end{problem}

\begin{problem}
Extend your ideas to the case of volume.  How would we measure volume?  How are these three cases related?  How are they different?

\begin{solution}
To measure volume, you count the number of volume units it takes to exactly fill your 3D figure, without any gaps or overlaps between the units.
\end{solution}
\begin{instructorNotes}
Hopefully, students should be catching on by this point.  If you don't have time to discuss this exercise with the whole class, point out as you conclude your discussion that we are really doing the same process in each case - using the unit iteratively.  You should point out that we have only used whole units thus far, but there is nothing stopping us from using partial units.  We'll get to that in a few more activities.
\end{instructorNotes}
\end{problem}

\end{document}