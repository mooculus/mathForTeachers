\documentclass[handout]{ximera}

\graphicspath{
  {./}
  {graphics/}
  {../graphics/}
}

\usepackage{chngcntr}

\let\question\relax
\let\endquestion\relax




\newtheoremstyle{SlantTheorem}{\topsep}{\fill}%%% space between body and thm
%\newtheoremstyle{SlantTheorem}{\topsep}{\topsep}%%% space between body and thm
 {\slshape}                      %%% Thm body font
 {}                              %%% Indent amount (empty = no indent)
 {\bfseries\sffamily}            %%% Thm head font
 {}                              %%% Punctuation after thm head
 {3ex}                           %%% Space after thm head
 {\thmname{#1}\thmnumber{ #2}\thmnote{ \bfseries(#3)}}%%% Thm head spec
\theoremstyle{SlantTheorem}
\newtheorem{question}{Question}
\counterwithin*{question}{section}



\let\instructorNotes\relax
\let\endinstructorNotes\relax
%%% instructorNotes environment
\ifhandout
\newenvironment{instructorNotes}[1][false]%
{%
\def\givenatend{\boolean{#1}}\ifthenelse{\boolean{#1}}{\begin{trivlist}\item}{\setbox0\vbox\bgroup}{}
}
{%
\ifthenelse{\givenatend}{\end{trivlist}}{\egroup}{}
}
\else
\newenvironment{instructorNotes}[1][false]%
{%
  \ifthenelse{\boolean{#1}}{\begin{trivlist}\item[\hskip \labelsep\bfseries {\Large Instructor Notes: \\} \hspace{\textwidth} ]}
{\begin{trivlist}\item[\hskip \labelsep\bfseries {\Large Instructor Notes: \\} \hspace{\textwidth} ]}
{}
}
{\end{trivlist}}
\fi


%% Suggested Timing
\newcommand{\timing}[1]{{\bf Suggested Timing: \hspace{2ex}} #1}

\title{Try This On For Size}
\author{Vic Ferdinand, Betsy McNeal, Jenny Sheldon}

\begin{document}
\begin{abstract}
\end{abstract}
\maketitle




\begin{problem} \label{Size1}
\hfill
\begin{enumerate}
\item By yourself, draw a line that is 6 cm long.  Next, write down detailed instructions for drawing the line that could be understood by someone who has not yet learned how to measure or how to use a ruler.  Find a partner, and have them follow your instructions exactly.  Did they draw the same line you drew?  Is their line 6 cm long?  
\item Repeat the above process, except this time choose a unit of length that is not standard.  (No centimeters, inches, etc!)  Can you use the same process you used in the first part to draw a line of length 6 units?
\end{enumerate}

\end{problem}



\begin{problem}
With your partner, use your experiences in the previous question to write down detailed instructions for measuring the length of an object.  You may use 1 cm as a unit, but your procedure should also work for any other appropriate unit.
\end{problem}





\begin{problem} \label{Size3}
A {\it square inch} is a particular unit of area which looks like a square measuring 1 inch on each side.  By yourself, draw an object which has an area of 8 square inches, but is {\bf not} a rectangle.  After you're done, compare your answer with your partner's answer.  How many different answers are possible to this question?
\end{problem}





\begin{problem}
Use your experiences in the previous question to write down a procedure for measuring the area of an object.  You may use 1 square inch as a unit, but your procedure should work for any other appropriate unit.  
\end{problem}





\begin{problem}
Extend your ideas to the case of volume.  How would we measure volume?  How are these three cases related?  How are they different?
\end{problem}


\newpage
\begin{instructorNotes}
This activity is intended to help students deepen their understanding of measurement, and is the first activity we do with standard units of measure.  We begin with measuring length, then move to measuring area.  If we have time, we briefly discuss volume at the end of the class.  Generally, our goal is to help students to see the similarities and differences between the types of measurement: in each case you choose a unit with the same dimension as the object you are measuring, then use that unit (or pieces of it) to cover the object in question.  In our calendar, this activity follows ``Measurement'', where we used non-standard units to measure objects, and is followed by ``Back and Forth'', where we consider unit conversion.

We treat the concept of measurement with a strong emphasis on using non-standard units to measure.  This is an example of our theme of ``making the familiar strange'', or taking students out of their comfort zone so that they cannot just apply old knowledge or mathematical rules, but instead have to think critically about the matter at hand.  We have found students much more willing to focus on the cover-and-count meaning of measurement rather than just applying a formula when the students are working with units that they have never before used.  These non-standard units also help us to draw out general principles about measurement, again rather than simply formulae.

In this activity in particular, we are trying to draw out these general principles about measurement, and see how they apply to standard units.  We are also focused on the iterative process of measurement, and relating the procedure to various activities we have completed.  We would like students to see the process of measurement as follows.
\begin{enumerate}
    \item Choose an aspect to measure. (This relates to the activity ``Where Do We Live?'')
    \item Choose an appropriate unit.  (This relates to  the activity``I'm The Biggest Kid!'')
    \item Iterate your unit or compare the object with your unit.  (Dealt with in the current activity.)
    \item Record your answers.
\end{enumerate}

Here are some more specific comments about the activity.
\begin{itemize}
    \item We sometimes suggest that students work with a partner that they do not normally work with in class so they can get honest feedback about their explanations.
    \item As we discuss students' instructions for using a ruler, we bring up children's misconceptions about ruler use.  For instance, we draw a ruler on the board, draw a line from 2cm to 8cm, report that this line is 8cm long, and then ask the students to react.
    \item In Problem \ref{Size1}, when students choose their own units they frequently choose things like their eraser or a paper clip.  We point out that these are actually more naturally area units than length units, and that they should be extremely specific about their unit (for instance, the long side along the bottom of their eraser).
    \item In Problem \ref{Size3} we introduce the definition of a square inch.  This is the first instance of our students using square inches to measure.  Some students struggle at first to come up with a shape that is not a rectangle, but eventually we see many different types of examples around the classroom and try to display as many on the board as possible.
\end{itemize}







\timing{This activity takes us about half a class period.  We have students work through the first two questions for about 10 minutes, then discuss.  Then, we give students about 5-10 minutes to think about the remaining problems, and then discuss.  We frequently don't have much time left for volume, but instead talk through the similarities and differences between the three types of measurement as a whole class.}

\end{instructorNotes}

\end{document}