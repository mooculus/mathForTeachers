\documentclass{ximera}


\graphicspath{
  {./}
  {graphics/}
  {../graphics/}
}

\usepackage{chngcntr}

\let\question\relax
\let\endquestion\relax




\newtheoremstyle{SlantTheorem}{\topsep}{\fill}%%% space between body and thm
%\newtheoremstyle{SlantTheorem}{\topsep}{\topsep}%%% space between body and thm
 {\slshape}                      %%% Thm body font
 {}                              %%% Indent amount (empty = no indent)
 {\bfseries\sffamily}            %%% Thm head font
 {}                              %%% Punctuation after thm head
 {3ex}                           %%% Space after thm head
 {\thmname{#1}\thmnumber{ #2}\thmnote{ \bfseries(#3)}}%%% Thm head spec
\theoremstyle{SlantTheorem}
\newtheorem{question}{Question}
\counterwithin*{question}{section}



\let\instructorNotes\relax
\let\endinstructorNotes\relax
%%% instructorNotes environment
\ifhandout
\newenvironment{instructorNotes}[1][false]%
{%
\def\givenatend{\boolean{#1}}\ifthenelse{\boolean{#1}}{\begin{trivlist}\item}{\setbox0\vbox\bgroup}{}
}
{%
\ifthenelse{\givenatend}{\end{trivlist}}{\egroup}{}
}
\else
\newenvironment{instructorNotes}[1][false]%
{%
  \ifthenelse{\boolean{#1}}{\begin{trivlist}\item[\hskip \labelsep\bfseries {\Large Instructor Notes: \\} \hspace{\textwidth} ]}
{\begin{trivlist}\item[\hskip \labelsep\bfseries {\Large Instructor Notes: \\} \hspace{\textwidth} ]}
{}
}
{\end{trivlist}}
\fi


%% Suggested Timing
\newcommand{\timing}[1]{{\bf Suggested Timing: \hspace{2ex}} #1}


\title{Multiplication Word Problems}
\author{Vic Ferdinand, Betsy McNeal, Jenny Sheldon}

\begin{document}
\begin{abstract} \end{abstract}
\maketitle


\begin{problem}
Describe similarities and differences between the following story problems by considering how a student who has never dealt with multiplication before might attempt to solve it.
\begin{enumerate}
\item A box holds 5 candy bars.  If there are 7 such boxes, how many total candy bars are there?

\vfill

\item A rectangle has length 7 inches and width 5 inches.  What is the area (measure of space inside) of the rectangle?


\vfill


\item John has 7 shirts and 5 pants.  How many different outfits can he wear?


\vfill

\item John has 7 times as many candies as Cindy does.  If Cindy has 5 candies, how many candies does John have?

\vfill

\end{enumerate}
\end{problem}

\newpage
\begin{instructorNotes}
As we did with Adding It All Up, we ask students to think about how young children who have not yet learned multiplication would solve the problems using objects or drawings.  Each problem addresses a different model of multiplication. 

\begin{itemize}
	\item Students should pay attention to the units of the 3 numbers in each problem as indicated below.
	\item After eliciting these interpretations and drawings, discuss what is the same in all cases (usually students come up with either ``repeated addition'' or groups).  This is a good time to introduce the definition of $A\times B$.  
	\item Going back through each multiplicative situation, we ask ``What are the groups? What are the objects?''  This will help everyone connect the more disparate models such as the Cartesian product with the array and repeated addition.  
	\item It's also important to talk about the units of quantities when dealing with multiplication: what are the units of the groups?  Of the objects?  What should be the units on the answer?  Students are often confused with this, particularly with area.  When describing the area, the units should be ``one-by-one squares''.  This will set some nice ground work for their geometry course.
\end{itemize}




{\bf Suggested Timing:} About 15-20 minutes in groups, 15 minutes in presentations, 15-20 minutes pulling all of this together.
\end{instructorNotes}

\end{document}