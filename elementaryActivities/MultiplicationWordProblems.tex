\documentclass[nooutcomes]{ximera}

\usepackage{gensymb}
\usepackage{tabularx}
\usepackage{mdframed}
\usepackage{pdfpages}
%\usepackage{chngcntr}

\let\problem\relax
\let\endproblem\relax

\newcommand{\property}[2]{#1#2}




\newtheoremstyle{SlantTheorem}{\topsep}{\fill}%%% space between body and thm
 {\slshape}                      %%% Thm body font
 {}                              %%% Indent amount (empty = no indent)
 {\bfseries\sffamily}            %%% Thm head font
 {}                              %%% Punctuation after thm head
 {3ex}                           %%% Space after thm head
 {\thmname{#1}\thmnumber{ #2}\thmnote{ \bfseries(#3)}} %%% Thm head spec
\theoremstyle{SlantTheorem}
\newtheorem{problem}{Problem}[]

%\counterwithin*{problem}{section}



%%%%%%%%%%%%%%%%%%%%%%%%%%%%Jenny's code%%%%%%%%%%%%%%%%%%%%

%%% Solution environment
%\newenvironment{solution}{
%\ifhandout\setbox0\vbox\bgroup\else
%\begin{trivlist}\item[\hskip \labelsep\small\itshape\bfseries Solution\hspace{2ex}]
%\par\noindent\upshape\small
%\fi}
%{\ifhandout\egroup\else
%\end{trivlist}
%\fi}
%
%
%%% instructorIntro environment
%\ifhandout
%\newenvironment{instructorIntro}[1][false]%
%{%
%\def\givenatend{\boolean{#1}}\ifthenelse{\boolean{#1}}{\begin{trivlist}\item}{\setbox0\vbox\bgroup}{}
%}
%{%
%\ifthenelse{\givenatend}{\end{trivlist}}{\egroup}{}
%}
%\else
%\newenvironment{instructorIntro}[1][false]%
%{%
%  \ifthenelse{\boolean{#1}}{\begin{trivlist}\item[\hskip \labelsep\bfseries Instructor Notes:\hspace{2ex}]}
%{\begin{trivlist}\item[\hskip \labelsep\bfseries Instructor Notes:\hspace{2ex}]}
%{}
%}
%% %% line at the bottom} 
%{\end{trivlist}\par\addvspace{.5ex}\nobreak\noindent\hung} 
%\fi
%
%


\let\instructorNotes\relax
\let\endinstructorNotes\relax
%%% instructorNotes environment
\ifhandout
\newenvironment{instructorNotes}[1][false]%
{%
\def\givenatend{\boolean{#1}}\ifthenelse{\boolean{#1}}{\begin{trivlist}\item}{\setbox0\vbox\bgroup}{}
}
{%
\ifthenelse{\givenatend}{\end{trivlist}}{\egroup}{}
}
\else
\newenvironment{instructorNotes}[1][false]%
{%
  \ifthenelse{\boolean{#1}}{\begin{trivlist}\item[\hskip \labelsep\bfseries {\Large Instructor Notes: \\} \hspace{\textwidth} ]}
{\begin{trivlist}\item[\hskip \labelsep\bfseries {\Large Instructor Notes: \\} \hspace{\textwidth} ]}
{}
}
{\end{trivlist}}
\fi


%% Suggested Timing
\newcommand{\timing}[1]{{\bf Suggested Timing: \hspace{2ex}} #1}




\hypersetup{
    colorlinks=true,       % false: boxed links; true: colored links
    linkcolor=blue,          % color of internal links (change box color with linkbordercolor)
    citecolor=green,        % color of links to bibliography
    filecolor=magenta,      % color of file links
    urlcolor=cyan           % color of external links
}


\title{Multiplication Word Problems}
\author{Vic Ferdinand, Betsy McNeal, Jenny Sheldon}

\begin{document}
\begin{abstract} \end{abstract}
\maketitle


\begin{problem}
Describe similarities and differences between the following story problems by considering how a student who has never dealt with multiplication before might attempt to solve it.
\begin{enumerate}
\item A box holds 5 candy bars.  If there are 7 such boxes, how many total candy bars are there?

\vfill

\item A rectangle has length 7 inches and width 5 inches.  What is the area (measure of space inside) of the rectangle?


\vfill


\item John has 7 shirts and 5 pants.  How many different outfits can he wear?


\vfill

\item John has 7 times as many candies as Cindy does.  If Cindy has 5 candies, how many candies does John have?

\vfill

\end{enumerate}
\end{problem}

\begin{problem}
What would happen if we replaced the numbers $5$ and $7$ in the previous problem with $51$ and $78$?  Would your strategies change?  Would your pictures change?  Try to work out at least one of the problems with these new numbers, and compare and contrast with the previous problem.
\end{problem}

\newpage
\begin{instructorNotes}
In our course, this is the first activity we use to introduce multiplication as an operation.  As with the activity ``Adding It All Up'' (introducing addition and subtraction), this activity has students think about how young children who have not yet learned multiplication would solve the problems using objects or drawings.  Each problem addresses a different model of multiplication.

In our course, we follow extensive work on the meaning of numbers (whole numbers, fractions, decimals) without reference to operations with activities about the operations.  This is the second activity we do to introduce an operation (the first is ``Adding It All Up'', which introduces addition and subtraction). Dealing with operations on numbers, rather than only with the meanings, notation, and comparison of quantities is a big shift in our course.

Again, we try to focus students on the structure of the problems, spending significant time discussing what the problems have in common with one another to develop and identify the structure of multiplication.

\begin{itemize}
	\item Going back through each multiplicative situation, we ask ``What are the groups? What are the objects in each group?''  This will help everyone connect the more disparate models such as the Cartesian product with the array and repeated addition.  
	\item Students should pay attention to the units of the three numbers in each problem. We ask: What are the units of the groups?  Of the objects in those groups?  What should be the units on the answer?  Students are often confused with this, particularly with area.  When describing the area, the objects in each group should be ``one-by-one squares'', and the groups are either ``rows" or ``columns", and the answer will be the total number of objects (``one-by-one squares''). It does not make sense to discuss ``rows of columns", as this would yield an answer in ``row-columns". This clarification will set some nice ground work for their geometry course.
	\item We contrast this work with units with the case of addition and subtraction, where the units of all of the objects were the same.  We have found that the more work we do with units here, the easier the division structure is for students later.
	\item We don't typically have time to fully discuss the final question, but we include it so that we have the chance to begin talk about the fact that multiplication is the same operation, no matter what numbers are used.  We typically return to this idea when we discuss fraction and decimal multiplication.
\end{itemize}




{\bf Suggested Timing:} We spend about 15-20 minutes in groups, 15 minutes in presentations, and 15-20 minutes pulling all of this together.
\end{instructorNotes}

\end{document}