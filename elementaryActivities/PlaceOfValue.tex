\documentclass[nooutcomes]{ximera}

\usepackage{gensymb}
\usepackage{tabularx}
\usepackage{mdframed}
\usepackage{pdfpages}
%\usepackage{chngcntr}

\let\problem\relax
\let\endproblem\relax

\newcommand{\property}[2]{#1#2}




\newtheoremstyle{SlantTheorem}{\topsep}{\fill}%%% space between body and thm
 {\slshape}                      %%% Thm body font
 {}                              %%% Indent amount (empty = no indent)
 {\bfseries\sffamily}            %%% Thm head font
 {}                              %%% Punctuation after thm head
 {3ex}                           %%% Space after thm head
 {\thmname{#1}\thmnumber{ #2}\thmnote{ \bfseries(#3)}} %%% Thm head spec
\theoremstyle{SlantTheorem}
\newtheorem{problem}{Problem}[]

%\counterwithin*{problem}{section}



%%%%%%%%%%%%%%%%%%%%%%%%%%%%Jenny's code%%%%%%%%%%%%%%%%%%%%

%%% Solution environment
%\newenvironment{solution}{
%\ifhandout\setbox0\vbox\bgroup\else
%\begin{trivlist}\item[\hskip \labelsep\small\itshape\bfseries Solution\hspace{2ex}]
%\par\noindent\upshape\small
%\fi}
%{\ifhandout\egroup\else
%\end{trivlist}
%\fi}
%
%
%%% instructorIntro environment
%\ifhandout
%\newenvironment{instructorIntro}[1][false]%
%{%
%\def\givenatend{\boolean{#1}}\ifthenelse{\boolean{#1}}{\begin{trivlist}\item}{\setbox0\vbox\bgroup}{}
%}
%{%
%\ifthenelse{\givenatend}{\end{trivlist}}{\egroup}{}
%}
%\else
%\newenvironment{instructorIntro}[1][false]%
%{%
%  \ifthenelse{\boolean{#1}}{\begin{trivlist}\item[\hskip \labelsep\bfseries Instructor Notes:\hspace{2ex}]}
%{\begin{trivlist}\item[\hskip \labelsep\bfseries Instructor Notes:\hspace{2ex}]}
%{}
%}
%% %% line at the bottom} 
%{\end{trivlist}\par\addvspace{.5ex}\nobreak\noindent\hung} 
%\fi
%
%


\let\instructorNotes\relax
\let\endinstructorNotes\relax
%%% instructorNotes environment
\ifhandout
\newenvironment{instructorNotes}[1][false]%
{%
\def\givenatend{\boolean{#1}}\ifthenelse{\boolean{#1}}{\begin{trivlist}\item}{\setbox0\vbox\bgroup}{}
}
{%
\ifthenelse{\givenatend}{\end{trivlist}}{\egroup}{}
}
\else
\newenvironment{instructorNotes}[1][false]%
{%
  \ifthenelse{\boolean{#1}}{\begin{trivlist}\item[\hskip \labelsep\bfseries {\Large Instructor Notes: \\} \hspace{\textwidth} ]}
{\begin{trivlist}\item[\hskip \labelsep\bfseries {\Large Instructor Notes: \\} \hspace{\textwidth} ]}
{}
}
{\end{trivlist}}
\fi


%% Suggested Timing
\newcommand{\timing}[1]{{\bf Suggested Timing: \hspace{2ex}} #1}




\hypersetup{
    colorlinks=true,       % false: boxed links; true: colored links
    linkcolor=blue,          % color of internal links (change box color with linkbordercolor)
    citecolor=green,        % color of links to bibliography
    filecolor=magenta,      % color of file links
    urlcolor=cyan           % color of external links
}


\title{A Place of Value}
\author{Vic Ferdinand, Betsy McNeal, Jenny Sheldon}

\begin{document}
\begin{abstract}\end{abstract}
\maketitle



\begin{problem}\label{PlaceOfValue1}
 Take a bunch of sticks and count them using only the following words: zero, one, two, three, four, five, six, seven, eight, nine. (Note:  There are no words such as ten, eleven, twelve, thirteen, twenty, etc. allowed during this activity!)  You may write these quantities using only the following symbols:  0, 1, 2, 3, 4, 5, 6, 7, 8, 9.
\begin{enumerate}
\item Describe the rules you use to count, as you keep putting one stick at a time in the group, including when you needed to switch to two or more of the symbols while counting.  In representing quantities in which you are forced to use two or more symbols, are each of the symbols representing the same quantities?  
\item Draw or describe how you could use bundling or blocks to count from 97 to 112.  From 697 to 712.  Use them to decide what would be the symbol for the quantity just before 400.  Use them to count backwards from 4213 to 4189.
\end{enumerate}
\end{problem}

\begin{problem} \label{PlaceOfValue2}
 Now suppose we wanted to count how many sticks we have using a similar counting system to write and speak about how many are in the set but, you are only allowed to use the symbols A, B, C, D, E, F, and zero = 0. Except for zero, the usual symbols (1, 2, 3 . . .) have been abolished.
\begin{enumerate}
\item In this system, when do you bundle and how many sticks are in a bundle?\\ How many bundles are in a superbundle?  \\
When will we bundle the ``superbundles"?  
\item Count to BAC in this system starting with A.   Develop easy-to-understand, yet complete, rules for how to count as we add one more stick to the pile.  Be thorough about when and why we bundle from one place to another (not just from the ones to the ``bundles").
\item Count from BE0 to CAB.  
\item Count backwards from ABC to EF. Explain in detail how the un-bundling process works. No cheating by going forwards from EF to ABC!  
\end{enumerate}
\end{problem}

\newpage
\begin{instructorNotes}
The intent here is to get the students to see the inner workings of our 
%socially-constructed 
place value system for quantifying sets, especially larger sets.  This activity can also give the opportunity to make the distinction between a number (the quantity attribute of a set) versus a numeral (the symbol or word we use to denote that quantity). We use different bases and different symbols to prompt our future teachers 
%get them away from their familiar crutches of bundling in powers of ten so they can 
to concentrate on the ``hidden'' rules of the counting and how they correspond with objects.

In Problem \ref{PlaceOfValue1}, we try to to help students see that ``10'' refers to 10 single sticks while ``a ten'' refers to 1 bundle.  Thus, a new unit is created when 10 singles are grouped together.  This new unit will be counted using the same counting system as that used for counting single sticks.  It is difficult for students (and children) to understand that ``10'' is simultaneously the word spoken as the last stick is laid down, the total number of sticks when one is added to nine, and the number of sticks in the bundle.  Students have occasionally thought that there were 9 individual sticks in the bundle, and that ``10'' represented something like the rubber band that encircles the bundle.

Overall, we emphasize the following points.
\begin{itemize}
    \item We usually refer to a bundle of bundles as a ``superbundle'', and a bundle of superbundles as a ``megabundle''.
	\item The rules of the bundling process reflect the actions taken with the sticks. It should be clear that all of the action begins with the ones place:  putting down one more stick. 
	%(Of course, this changes when we begin counting by bundles, or superbundles, etc.)  
	A bundle is created after the tenth stick is laid down (thus the 10th stick is pushed together with the other 9 to create a bundle).
	\item We want to highlight the different types of units in this activity as well as their relationships to one another.  Students should see that, for instance, when they are counting bundles, they are counting a different type of unit or object than when they are counting sticks, and there is a dual nature to it: two bundles is the same thing as two-zero sticks.
	\item We try to focus students mostly on actions with sticks in their explanations - the symbol here is nearly an afterthought.
	\item After the counting rules are established, we find it can help to write the quantity in ``expanded form'', e.g., $123 = 1(100) + 2(10) + 3(1)$. This once again helps us emphasize the different units involved (now we have three different units: 1 superbundle, 2 bundles, and 1 single stick).

	\item Once the students are getting comfortable with counting forwards, we want to count backwards.  We want the students to see this process as ``un-bundling'', or, to think about what is happening with the sticks on the table when we move backwards one number - particularly ``across a zero''. 
	\item These bundling and un-bundling processes play an important role in our standard arithmetic procedures of adding with ``carrying'' and subtracting with ``borrowing''. 
\end{itemize}

%Some possible extensions:

%If time, you might have a ``semi-historical'' discussion about how the Hindu-Arabic place value system is an improvement over other systems developed over time (e.g., Roman, Egyptian) to represent quantities (e.g., In the whole numbers, the longer the symbol, the greater the quantity it represents.  Also don't need to keep inventing symbols as you get to more arduous quantities to comprehend).  You might also point out that the place value system has its ``Achilles Heels'' as well, such as the loss of ``larger symbol= greater quantity'' in partial numbers, the need for scientific notation, infinitely-long decimals, etc.  

%If your students are a little more comfortable with algebra, you can bring up the ``polynomial'' structure of the place value system (successive powers of the base).  You might bring this into the activity (after the counting rules are established) by writing a quantity in ``expanded form''.  Make sure they see the bundles working to show this (singles, bundles, bundles of bundles (square), bundles of superbundles (cubed), etc. (Or give them a polynomial with single-digit whole number coefficients and ask them to (quickly) evaluate it at $x = 10$)).

%You could also discuss pros and cons of the different kinds of objects traditionally used to model numbers.  Sticks are the most concrete (can pick them apart- need to ``trade'' or rename) and blocks are next (still have the relative sizes but can't break them apart).  Then the abacus and coins have the ``bundling'' quality still there (but no relative sizes of the places).  Then the calculator has only the symbols (need to fully understand the system to understand what it means).


{\bf Suggested Timing:} We let students work on Problem \ref{PlaceOfValue1} in their small groups for a while, 5-10 minutes.  Then we have a whole class discussion for 5-10 minutes to clarify the idea of bundling into tens for our usual system. (Sometimes, students are not immediately aware that Problem \ref{PlaceOfValue1} is about our usual base ten system.) Once students understand that Problem \ref{PlaceOfValue2} involves a different system, small groups work on the remainder of the questions for 15-20 minutes.  Save about 15 minutes to go over some of the ideas (but we usually do not discuss all of the activity, as we ask students to complete Problem \ref{PlaceOfValue2} as a homework assignment).  We usually allot about 2 classes and a homework assignment to this investigation of number systems, so there will be time on the second day for further clarification.

\end{instructorNotes}



\end{document}