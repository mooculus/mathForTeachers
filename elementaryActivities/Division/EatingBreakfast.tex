\documentclass[nooutcomes,noauthor]{ximera}

\graphicspath{
  {./}
  {graphics/}
  {../graphics/}
}

\usepackage{chngcntr}

\let\question\relax
\let\endquestion\relax




\newtheoremstyle{SlantTheorem}{\topsep}{\fill}%%% space between body and thm
%\newtheoremstyle{SlantTheorem}{\topsep}{\topsep}%%% space between body and thm
 {\slshape}                      %%% Thm body font
 {}                              %%% Indent amount (empty = no indent)
 {\bfseries\sffamily}            %%% Thm head font
 {}                              %%% Punctuation after thm head
 {3ex}                           %%% Space after thm head
 {\thmname{#1}\thmnumber{ #2}\thmnote{ \bfseries(#3)}}%%% Thm head spec
\theoremstyle{SlantTheorem}
\newtheorem{question}{Question}
\counterwithin*{question}{section}



\let\instructorNotes\relax
\let\endinstructorNotes\relax
%%% instructorNotes environment
\ifhandout
\newenvironment{instructorNotes}[1][false]%
{%
\def\givenatend{\boolean{#1}}\ifthenelse{\boolean{#1}}{\begin{trivlist}\item}{\setbox0\vbox\bgroup}{}
}
{%
\ifthenelse{\givenatend}{\end{trivlist}}{\egroup}{}
}
\else
\newenvironment{instructorNotes}[1][false]%
{%
  \ifthenelse{\boolean{#1}}{\begin{trivlist}\item[\hskip \labelsep\bfseries {\Large Instructor Notes: \\} \hspace{\textwidth} ]}
{\begin{trivlist}\item[\hskip \labelsep\bfseries {\Large Instructor Notes: \\} \hspace{\textwidth} ]}
{}
}
{\end{trivlist}}
\fi


%% Suggested Timing
\newcommand{\timing}[1]{{\bf Suggested Timing: \hspace{2ex}} #1}
\title{Eating breakfast}

\begin{document}
\begin{abstract}
\end{abstract}

\maketitle

\begin{problem}
You know that if you eat $\frac{1}{4}$ of your favorite breakfast sandwich, you get $5$ grams of protein. How many grams of protein are in the whole sandwich?

\begin{enumerate}
	\item Explain why this is a story problem for $5 \div \frac{1}{4}$. What is one group? One object? What type of division is this?
	\item Solve the problem using a picture and our meaning of fractions. How does the picture help you to see the answer to this question?
\end{enumerate}


\end{problem}



\begin{problem}
You know that if you eat $\frac{1}{8}$ of your favorite breakfast casserole, you get $\frac{9}{5}$ of a gram of salt. How many grams of salt are in the entire casserole? 

\begin{enumerate}
	\item Explain why this is a story problem for $\frac{9}{5} \div \frac{1}{8}$. What is one group? One object? What type of division is this?
	\item Solve the problem using a picture and our meaning of fractions. How does the picture help you to see the answer to this question?
\end{enumerate}
\end{problem}



\begin{problem}
You know that if you eat $\frac{2}{5}$ of your favorite breakfast pastry, you get $\frac{3}{7}$ of a gram of fiber. How many grams of fiber are in the entire pastry?

\begin{enumerate}
	\item Explain why this is a story problem for $\frac{3}{7} \div \frac{2}{5}$. What is one group? One object? What type of division is this?
	\item Solve the problem using a picture and our meaning of fractions. How does the picture help you to see the answer to this question?
\end{enumerate}
\end{problem}

\newpage

\begin{problem}
Look back at your solution to the previous problem about pastry. Your goal is to be able to answer at least one of the parts below, depending on your picture.
\begin{enumerate}
	\item In your final picture, can you see that the solution is $\frac{5}{2} \times \frac{3}{7}$? What is one group? One object?
	\item In your final picture, can you see that the solution is $5 \times \left ( \frac{3}{7} \div 2 \right )$? What kind of division is this? What are the groups and objects for the multiplication?
	\item In your final picture, can you see that the solution is $\frac{5 \times 3}{2 \times 7}$ or $\frac{3 \times 5}{7 \times 2}$? How does the meaning of the numerator and denominator tell you what to count, and what are the groups and objects for each multiplication?
\end{enumerate}
\end{problem}



\begin{problem}
Write a breakfast-themed story problem for $\frac{4}{9} \div \frac{5}{2}$. Explain why your story is correct, then solve it with a picture. How can you see the ``invert and multiply'' procedure in your final picture?
\end{problem}






\newpage

\begin{instructorNotes} 



{\bf Main goal:} We solve How Many In Each Group division problems with fractions to justify the ``invert and multiply'' procedure.


{\bf Overall picture:} Problems 3 and 4 contain the main type of problem we would like students to be able to solve after this activity. 






\begin{itemize}
	\item Overall, the goal is to help the students first identify the unit fraction for a group, and then make enough copies of that unit fraction to make the entire group. For instance, when we look at the first problem ($5 \div \frac{1}{4}$), we first identify $\frac{1}{4}$ of a group. Then we make four copies of $\frac{1}{4}$ of a group, or multiply $4 \times 5$. By the third problem, we will need to divide to find the amount corresponding to our unit fraction, so we will add this step as we go along.
	\item Good labeling of their pictures is still essential for good solutions. Be sure to compliment good labeling as you walk around and as students present.
	\item Paying attention to the relationship between the two wholes in the problem is still essential. The key to solving these problems is recognizing that we have to equate the parts of these fractions, not the wholes. In the third problem, this means that the two pieces of the pastry must equal the three pieces of the gram. We don't want to equate the pastry and one gram. Watch out for this misconception as you walk around.
	\item For the three interpretations of the invert and multiply procedure, the first uses the meaning of fractions. If $\frac{2}{5}$ of the whole is the same as $\frac{3}{7}$ of a gram, then the entire whole is $\frac{5}{2}$ of this amount, or $\frac{5}{2}$ copies of $\frac{3}{7}$ of a gram.
	\item The second interpretation of the invert and multiply procedure takes the $\frac{3}{7}$ of a gram and divides it among the two equal pieces we have of the whole. One group is one piece and one object is one gram. Then we multiply this by $5$ because there are a total of $5$ pieces (or groups) that we need to fill.
	\item The third interpretation of the invert and multiply procedure is the analog of what we did with fraction multiplication. In the whole for the answer, the students may see $2$ rows with $7$ pieces per row. The shaded region (making up the original whole) can be seen as $5$ rows with $3$ pieces per row.
	\item The final question is a bonus, indicating how we want students to practice with these ideas on their own. Be sure to double-check their story problems, considering the same common errors we had with how many groups fraction division.
\end{itemize}


{\bf Good language:}  Here, we really want the ``how many in one FULL group'' interpretation of division. Highlighting this language may help students follow along.  Remind students to use the meaning of fractions in their explanations. Help students to see how the picture helps them to solve these problems, rather than drawing a picture after knowing the solution.

As you work through these problems in discussion, be sure to have students clearly state the problem they are using, as well as what they are seeing as one group and one object. They should also clarify how they know it's a how many in each group problem. (Or a how many in one full group problem!) Students who don't write a good story problem will struggle with this activity!



{\bf Suggested timing:} Give students about 10 minutes to get started on this activity. When most students have completed Problem 1, discuss together as a class. Give students another 10-15 minutes to work on as many of the rest of the problems as they can, and finish with discussion.

\end{instructorNotes}



\end{document}