\documentclass[nooutcomes,noauthor]{ximera}

\graphicspath{
  {./}
  {graphics/}
  {../graphics/}
}

\usepackage{chngcntr}

\let\question\relax
\let\endquestion\relax




\newtheoremstyle{SlantTheorem}{\topsep}{\fill}%%% space between body and thm
%\newtheoremstyle{SlantTheorem}{\topsep}{\topsep}%%% space between body and thm
 {\slshape}                      %%% Thm body font
 {}                              %%% Indent amount (empty = no indent)
 {\bfseries\sffamily}            %%% Thm head font
 {}                              %%% Punctuation after thm head
 {3ex}                           %%% Space after thm head
 {\thmname{#1}\thmnumber{ #2}\thmnote{ \bfseries(#3)}}%%% Thm head spec
\theoremstyle{SlantTheorem}
\newtheorem{question}{Question}
\counterwithin*{question}{section}



\let\instructorNotes\relax
\let\endinstructorNotes\relax
%%% instructorNotes environment
\ifhandout
\newenvironment{instructorNotes}[1][false]%
{%
\def\givenatend{\boolean{#1}}\ifthenelse{\boolean{#1}}{\begin{trivlist}\item}{\setbox0\vbox\bgroup}{}
}
{%
\ifthenelse{\givenatend}{\end{trivlist}}{\egroup}{}
}
\else
\newenvironment{instructorNotes}[1][false]%
{%
  \ifthenelse{\boolean{#1}}{\begin{trivlist}\item[\hskip \labelsep\bfseries {\Large Instructor Notes: \\} \hspace{\textwidth} ]}
{\begin{trivlist}\item[\hskip \labelsep\bfseries {\Large Instructor Notes: \\} \hspace{\textwidth} ]}
{}
}
{\end{trivlist}}
\fi


%% Suggested Timing
\newcommand{\timing}[1]{{\bf Suggested Timing: \hspace{2ex}} #1}
\title{Long division}

\begin{document}
\begin{abstract}
\end{abstract}

\maketitle

\begin{problem}
How do we perform the steps of long division? Use the following examples to explain.
\begin{enumerate}
\item $348 \div 4$ (give a whole number answer with remainder)
\item $396 \div 5$ (give a decimal answer)
\end{enumerate}
\end{problem}


\begin{problem}
Write a how many in each group division story problem for $348 \div 4$. Explain how you know your story is correct. What is one group? One object?
\end{problem}


\begin{problem}
Draw $348$ bundled objects and $4$ large groups to place these objects in. Solve your story above in a way that models the long division algorithm. What step do you take first in the algorithm, and where is that modeled in your picture? What step do you take second, and where is that modeled in your picture? Continue in this way until you have a whole number answer with remainder. Pay close attention to explaining the following steps.
\begin{itemize}
	\item In the division algorithm, we often multiply. Where do you have multiplication in your picture, and why does this make sense?
	\item In the division algorithm, we often subtract. Where do you have subtraction in your picture, and why does this make sense?
	\item In the division algorithm, we usually ``bring down'' digits from the dividend. Where do you see the ``bringing down'' in your picture, and why does this make sense?
	\item Where is the answer in your picture, and why does this make sense?
\end{itemize}

\end{problem}


\begin{problem}
There is another algorithm for division called the partial quotients method which is the opposite of the partial products method. How would we do the steps of the partial quotients method for $348 \div 4$?
\end{problem}

\newpage

Here are some additional problems to try if you have extra time.


\begin{problem}
Repeat questions 2 and 3, but for $396 \div 5$ with a decimal answer.
\end{problem}


\begin{problem}
Repeat questions 2 and 3 for $348 \div 4$, but using a how many groups interpretation of division.
\end{problem}




\newpage

\begin{instructorNotes} 



{\bf Main goal:} We justify the standard division algorithm.


{\bf Overall picture:} 


Generally, the steps that we are looking for students to justify for the long division algorithm are the following.
\begin{itemize}
	\item Students should first phrase this problem in terms of a story, for instance we are trying to distribute $348$ pencils into $4$ boxes, and our pencils are bundled according to the usual bundling system.
	\item We begin by looking just at the $3$. We have 3 superbundles, and we attempt to distribute this into our 4 groups, but there aren't enough superbundles. So we unbundle this superbundle into its 10 bundles, and combine together with the 5 other bundles. (Student should recognize this as the steps corresponding to ``4 doesn't go into 3, so we look at the 35 instead.)
	\item We distribute our 35 bundles into our 4 groups, and $8$ will fit in each group. This will use $32$ bundles, so we subtract these $32$ from the $34$ we have. We mark the $8$ bundles above the line, and they go above the $34$ bundles because they are still bundles. After we subtract the $32$ we used, we have 2 remaining bundles.
	\item We unbundle these bundle and combine  with our 8 individuals (this corresponds to ``bringing down the $8$''). We distribute again and repeat the process from above. We continue in this manner with the rest of the division.
	\item Finally, we count the number of objects in each group, but we have essentially already done this in the way we kept track of the number of each type of object that went in the groups above the division line.
	\item We want students to recognize that we are working with different objects at each stage, and that we can work with the individual digits because we can count not only the sticks involved but the larger objects themselves (bundles, superbundles, etc).
	\item We want students to realize that the reason we place the digits where we do and don't need to recount the final answer because we are working with these place value units.
	\item We want to emphasize the meaning of the ``bringing down'' step as unbundling.
	\item With the decimal answer, we would like to continue unbundling our sticks into ``mini sticks''. As long as we unbundle into $10$ equal pieces, we can continue making more place values as long as we need to.
	\item The question about the partial quotients method should hopefully be a fun puzzle for students, but it can also be a whole-class discussion to wrap up the conversation about the division algorithm. The partial quotients method is also called the scaffold method, and is a flexible algorithm like the partial products method. Encourage students to see both partial products and partial quotients as stepping stones to the more efficient standard methods.
	\item Students can feel very uncomfortable with the long division algorithm. Don't be afraid to work extra examples to make sure everyone is following along.
\end{itemize}


{\bf Good language:} Emphasize the bundling system in your discussion here. Help students to see why the place value ideas are helping the algorithm to be straightforward. 

{\bf Suggested timing:} Give students about $5$ minutes to work on the first two problems, and then discuss. Make sure everyone is feeling comfortable with the algorithm and their story problems before beginning the next problem. Next, give students about 10-15 minutes to work on problems 3 and 4, then have groups present. If you have enough chalkboard space, this is a nice activity to have all groups working on the chalkboard so that they are ready to talk about their steps with the whole class. Walk through each step of the algorithm carefully (not just the listed steps). If some groups have already done the decimal example, go ahead and present this now too. Use the last 5-10 minutes to wrap up and discuss the partial quotients method. If you have extra time beyond that, you can move to how many groups interpretations or the decimal question.




\end{instructorNotes}



\end{document}