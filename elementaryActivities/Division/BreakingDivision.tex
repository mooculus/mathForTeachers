%\documentclass{ximera}
\documentclass[nooutcomes,noauthor]{ximera}

\graphicspath{
  {./}
  {graphics/}
  {../graphics/}
}

\usepackage{chngcntr}

\let\question\relax
\let\endquestion\relax




\newtheoremstyle{SlantTheorem}{\topsep}{\fill}%%% space between body and thm
%\newtheoremstyle{SlantTheorem}{\topsep}{\topsep}%%% space between body and thm
 {\slshape}                      %%% Thm body font
 {}                              %%% Indent amount (empty = no indent)
 {\bfseries\sffamily}            %%% Thm head font
 {}                              %%% Punctuation after thm head
 {3ex}                           %%% Space after thm head
 {\thmname{#1}\thmnumber{ #2}\thmnote{ \bfseries(#3)}}%%% Thm head spec
\theoremstyle{SlantTheorem}
\newtheorem{question}{Question}
\counterwithin*{question}{section}



\let\instructorNotes\relax
\let\endinstructorNotes\relax
%%% instructorNotes environment
\ifhandout
\newenvironment{instructorNotes}[1][false]%
{%
\def\givenatend{\boolean{#1}}\ifthenelse{\boolean{#1}}{\begin{trivlist}\item}{\setbox0\vbox\bgroup}{}
}
{%
\ifthenelse{\givenatend}{\end{trivlist}}{\egroup}{}
}
\else
\newenvironment{instructorNotes}[1][false]%
{%
  \ifthenelse{\boolean{#1}}{\begin{trivlist}\item[\hskip \labelsep\bfseries {\Large Instructor Notes: \\} \hspace{\textwidth} ]}
{\begin{trivlist}\item[\hskip \labelsep\bfseries {\Large Instructor Notes: \\} \hspace{\textwidth} ]}
{}
}
{\end{trivlist}}
\fi


%% Suggested Timing
\newcommand{\timing}[1]{{\bf Suggested Timing: \hspace{2ex}} #1}
\title{Breaking division}

\begin{document}
\begin{abstract}
\end{abstract}

\maketitle

\begin{problem}
Zamir says that we can't divide by zero. He says, ``Zero means nothing, so we can't divide by it because there's nothing to divide. Like $0 \div 5$, we can't solve that.''

\begin{enumerate}
	\item Use a how many groups interpretation of division and a story problem to explore Zamir's idea. Can we solve $0 \div 5$?
	\item Use a how many in each group interpretation of division and a multiplication equation to explore Zamir's idea. Can we solve $0 \div 5$?
\end{enumerate}
\end{problem}



\begin{problem}
Yahna says that we can't divide by zero. She says, ``The zero in the bottom means we can't make groups. So $6 \div 0$ we can't answer.''
\begin{enumerate}
	\item Use a how many groups interpretation of division and a multiplication equation to explore Yahna's idea. Can we solve $6 \div 0$?
	\item Use a how many in each group interpretation of division and a story problem to explore Yahna's idea. Can we solve $6 \div 0$?
\end{enumerate}
\end{problem}



\begin{problem}
Ximena says that $6 \div 0$ should be infinity. Based on your work in Yahna's problem, do you agree or disagree with Ximena? Explain your thinking.
\end{problem}



\begin{problem}
Walter says that $0 \div 0$ should be equal to one, since anything over itself is always one. Use the techniques we have been practicing in this section to investigate Walter's idea. Do you agree with him or not?
\end{problem}






\newpage

\begin{instructorNotes} 



{\bf Main goal:} We distinguish between types of division with zero.


{\bf Overall picture:} 



\begin{itemize}
	\item When students translate division to multiplication, be sure to talk about both cases $? \times 0 = 6$ and $0 \times ? = 5$. Since each problem has a different equation, you might also write the HMG equation in problem 1 and the HMIEG equation in problem 2. You can also discuss how the story problems they have written are for the opposite division situation, or write equations that correspond to the story situations after your discuss the stories.
	\item When they write stories for $5 \div 0$ and $0 \div 6$, our goal is to determine whether the story makes sense. For instance, if I try to distribute $6$ cookies to my $0$ friends, I can't complete the action of solving. I can't count how many cookies each friend gets, because I don't have any groups to use. We want to contrast this to the situation where I distribute $0$ cookies to my $5$ friends, in which case I can count how many cookies each friend gets. When we have $0$ friends and $0$ cookies, we don't even have a problem to solve! I don't have the same problem as $6 \div 0$, where I'm left holding cookies that I don't know what to do with, but I also still can't figure out how many are in each group without any groups to look at.
	\item These solutions by story problem are difficult to distinguish. We will rely more heavily on the algebra to clearly distinguish the types. For this outcome, students should be able to use one of the two methods (either story or algebra) to discuss the main ideas.
	\item The last two problems give us a chance to talk about common misconceptions involving division with zero. For the infinity case, students should refer to the ways that they discussed in the previous problem that $6 \div 0$ does not make sense with an equation or with a story situation. For both this problem and the $0 \div 0$ problem, the related multiplication equation is likely more convincing for students and you don't need to emphasize both the story and the equation.
	\item Your discussion should really compare and contrast what's happening in each of these cases. We don't need students to use the terms ``indeterminate'' and ``undefined'', but we do want them to understand why these situations are different.
\end{itemize}


{\bf Good language:} Continue to insist that students identify what they are seeing as groups, objects, and which type of division they are using. We are still practicing with the meaning of division!

{\bf Suggested timing:} Give students about 10 minutes to work on the activity in groups. Discussion should take roughly 20 minutes. Wrap up by reviewing the different cases that can occur and checking for questions.




\end{instructorNotes}



\end{document}