\documentclass{ximera}


\graphicspath{
  {./}
  {graphics/}
  {../graphics/}
}

\usepackage{chngcntr}

\let\question\relax
\let\endquestion\relax




\newtheoremstyle{SlantTheorem}{\topsep}{\fill}%%% space between body and thm
%\newtheoremstyle{SlantTheorem}{\topsep}{\topsep}%%% space between body and thm
 {\slshape}                      %%% Thm body font
 {}                              %%% Indent amount (empty = no indent)
 {\bfseries\sffamily}            %%% Thm head font
 {}                              %%% Punctuation after thm head
 {3ex}                           %%% Space after thm head
 {\thmname{#1}\thmnumber{ #2}\thmnote{ \bfseries(#3)}}%%% Thm head spec
\theoremstyle{SlantTheorem}
\newtheorem{question}{Question}
\counterwithin*{question}{section}



\let\instructorNotes\relax
\let\endinstructorNotes\relax
%%% instructorNotes environment
\ifhandout
\newenvironment{instructorNotes}[1][false]%
{%
\def\givenatend{\boolean{#1}}\ifthenelse{\boolean{#1}}{\begin{trivlist}\item}{\setbox0\vbox\bgroup}{}
}
{%
\ifthenelse{\givenatend}{\end{trivlist}}{\egroup}{}
}
\else
\newenvironment{instructorNotes}[1][false]%
{%
  \ifthenelse{\boolean{#1}}{\begin{trivlist}\item[\hskip \labelsep\bfseries {\Large Instructor Notes: \\} \hspace{\textwidth} ]}
{\begin{trivlist}\item[\hskip \labelsep\bfseries {\Large Instructor Notes: \\} \hspace{\textwidth} ]}
{}
}
{\end{trivlist}}
\fi


%% Suggested Timing
\newcommand{\timing}[1]{{\bf Suggested Timing: \hspace{2ex}} #1}


\title{Reasoning About Division}

\begin{document}
\begin{abstract} \end{abstract}
\maketitle


\begin{problem}
We know that $25 \div 12 = 2$, remainder 1, and
$21 \div 10 = 2$, remainder 1. 
Would it be correct to say that $25 \div 12 = 21 \div 10$?  Explain.


\end{problem}


\begin{problem}
 Write and solve a simple story problem for $600 \div 20$.  Illustrate your answer in a picture.  Did you write a \underline{how many groups} story or a \underline{how many in each group} story?
\end{problem}

\begin{problem}
  Use the situation of your story problem in the previous problem to help you solve $600 \div 19$ by modifying your picture solution to $600 \div 20$.  Do not use a calculator or long division.
\end{problem}

\begin{problem}
 Use the same situation of your story problem to help you solve $600 \div 21$ by modifying your picture solution to $600 \div 20$.  Do not use a calculator or long division.

\end{problem}

\newpage

\begin{instructorNotes}
The purpose here is to have students solve division problems using only the meaning of division, and then to begin to look for relationships between problems. Both problems emphasize good number sense and the meaning of division.  In each, one should ``act out'' the story problem through repeated subtraction.

If short on time, you can just do \#2, or put this on the final exam review instead of in class.

\begin{itemize}
    \item Problem 1 is typically the most difficult part of the activity.  To help, a good story problem (in each model) will help students see the difference in the quotient's (2) meaning in each scenario (esp. in the ``\# of groups'', as the groups are of different sizes). \item In \#2-4, acting out will show how each person gets one more or less (\# groups model -- can add up the surplus to see how many more people get 19 apiece) or there is one more/less person to pass out to (again, add up the overage to see how many more pieces each can get - or do so after passing out, as Beckmann suggests).  
	\item \#2-4 can be done again using the other (how many in each group) model of division.
	\item Have students explain each problem using each type of division.  
\end{itemize}

{\bf Suggested Timing:} Allotting the entire class period to this activity allows us to have students write a story problem for \#2 (one for each type of division) as well as show a picture solution for each type of division.  About 5 minutes for \#1, then 10 minutes discussion of \#1.  Then 5 minutes writing two different division stories for \#2.  Then 10 minutes for the small groups to solve \#3 using two different types of division, 10 minutes to present solutions, and rest of time to try again with \#4. 



{\bf Suggested Timing:}  A half class to one class

\end{instructorNotes}

\end{document}