\documentclass{ximera}

\usepackage{gensymb}
\usepackage{tabularx}
\usepackage{mdframed}
\usepackage{pdfpages}
%\usepackage{chngcntr}

\let\problem\relax
\let\endproblem\relax

\newcommand{\property}[2]{#1#2}




\newtheoremstyle{SlantTheorem}{\topsep}{\fill}%%% space between body and thm
 {\slshape}                      %%% Thm body font
 {}                              %%% Indent amount (empty = no indent)
 {\bfseries\sffamily}            %%% Thm head font
 {}                              %%% Punctuation after thm head
 {3ex}                           %%% Space after thm head
 {\thmname{#1}\thmnumber{ #2}\thmnote{ \bfseries(#3)}} %%% Thm head spec
\theoremstyle{SlantTheorem}
\newtheorem{problem}{Problem}[]

%\counterwithin*{problem}{section}



%%%%%%%%%%%%%%%%%%%%%%%%%%%%Jenny's code%%%%%%%%%%%%%%%%%%%%

%%% Solution environment
%\newenvironment{solution}{
%\ifhandout\setbox0\vbox\bgroup\else
%\begin{trivlist}\item[\hskip \labelsep\small\itshape\bfseries Solution\hspace{2ex}]
%\par\noindent\upshape\small
%\fi}
%{\ifhandout\egroup\else
%\end{trivlist}
%\fi}
%
%
%%% instructorIntro environment
%\ifhandout
%\newenvironment{instructorIntro}[1][false]%
%{%
%\def\givenatend{\boolean{#1}}\ifthenelse{\boolean{#1}}{\begin{trivlist}\item}{\setbox0\vbox\bgroup}{}
%}
%{%
%\ifthenelse{\givenatend}{\end{trivlist}}{\egroup}{}
%}
%\else
%\newenvironment{instructorIntro}[1][false]%
%{%
%  \ifthenelse{\boolean{#1}}{\begin{trivlist}\item[\hskip \labelsep\bfseries Instructor Notes:\hspace{2ex}]}
%{\begin{trivlist}\item[\hskip \labelsep\bfseries Instructor Notes:\hspace{2ex}]}
%{}
%}
%% %% line at the bottom} 
%{\end{trivlist}\par\addvspace{.5ex}\nobreak\noindent\hung} 
%\fi
%
%


\let\instructorNotes\relax
\let\endinstructorNotes\relax
%%% instructorNotes environment
\ifhandout
\newenvironment{instructorNotes}[1][false]%
{%
\def\givenatend{\boolean{#1}}\ifthenelse{\boolean{#1}}{\begin{trivlist}\item}{\setbox0\vbox\bgroup}{}
}
{%
\ifthenelse{\givenatend}{\end{trivlist}}{\egroup}{}
}
\else
\newenvironment{instructorNotes}[1][false]%
{%
  \ifthenelse{\boolean{#1}}{\begin{trivlist}\item[\hskip \labelsep\bfseries {\Large Instructor Notes: \\} \hspace{\textwidth} ]}
{\begin{trivlist}\item[\hskip \labelsep\bfseries {\Large Instructor Notes: \\} \hspace{\textwidth} ]}
{}
}
{\end{trivlist}}
\fi


%% Suggested Timing
\newcommand{\timing}[1]{{\bf Suggested Timing: \hspace{2ex}} #1}




\hypersetup{
    colorlinks=true,       % false: boxed links; true: colored links
    linkcolor=blue,          % color of internal links (change box color with linkbordercolor)
    citecolor=green,        % color of links to bibliography
    filecolor=magenta,      % color of file links
    urlcolor=cyan           % color of external links
}


\title{Decimal Division}

\begin{document}
\begin{abstract} \end{abstract}
\maketitle


\begin{problem} Consider the following division question.
\begin{image}
\begin{tikzpicture}
\node at (-1, 0.5) {How many};
\draw[thick, step=0.1] (0,0) grid (0.1,1);
\draw[thick, step=0.1] (0.2, 0) grid (0.3,1);
\draw[thick, step=0.1] (0.4, 0) grid (0.5, 1);
\draw[thick, step=0.1] (0.6, 0.1) grid (0.7, 0.2);
\draw[thick, step=0.1] (0.6, 0.3) grid (0.7, 0.4);
\draw[thick, step=0.1] (0.6, 0.5) grid (0.7, 0.6);
\draw[thick, step=0.1] (0.8, 0.2) grid (0.9, 0.3);
\draw[thick, step=0.1] (0.8, 0.4) grid (0.9, 0.5);
\node at (1.75, 0.5) {are in};
\draw[thick, step=0.1] (2.5, 0) grid (3.5, 1);
\draw[thick, step=0.1] (3.6, 0) grid (4.6, 1);
\draw[thick, step=0.1] (4.7, 0) grid (4.8, 1);
\node at (4.9, 0.5) {?};
\end{tikzpicture}
\end{image}
\begin{enumerate}
\item Why is the figure above a division question? What are the groups? Objects per group? What type of division is it? 
\item Estimate the answer to the division question. Describe how you got your estimate and why it makes sense.
\item Use the figure to find the answer to the division question.
\item Explain how the figure can be interpreted as $21 \div 3.5 = ?$.
\item Explain how the figure can also be interpreted as $210 \div 35 = ?$
\item Give at least two other ways to interpret the numbers in the division question. What do all of these division questions have in common?
\end{enumerate}

\end{problem}

\begin{problem}
How would you calculate $21 \div 3.5$ using long division? Use your work in the previous problem to explain why it makes sense to move the decimal point around the way that we do.
\end{problem}




\begin{problem}
\begin{enumerate}
\item Draw a picture like the one in Problem 1 for the division problem  $2.3 \div 0.75$.  Don't solve in this step!
\item What other division problems can your picture illustrate?  Give at least three examples.
\item Use your picture to give an approximate answer to  $2.3 \div 0.75$.
\end{enumerate}
\end{problem}



\begin{problem}
Phoebe thinks that the first step when dividing decimals is to remove the decimal points, then use whole number division to calculate the answer, then put the decimal point back in.

\begin{itemize}
	\item Carry out Phoebe's procedure for $21 \div 3.5$. Does she get the right answer?
	\item Carry out Phoebe's procedure for $2.3 \div 0.75$. Does she get the right answer?
	\item How could Phoebe use her procedure along with estimation to get the correct answer for every example? Explain your thinking.
\end{itemize}
\end{problem}


\newpage
\begin{instructorNotes}

{\bf Main goal:} We understand that changing the value of a block does not change the physical interpretation of division.

{\bf Overall picture:}
These problems provide an investigation of the meaning of the rules of decimal division.

\begin{itemize}
	\item Problems 1 and 3 highlight the need to identify our ``one'' (unit) and work from there.  The division problems are equivalent because of the multiplicative relationship between the objects.  Thus, finding how many groups of 3.5 are in 12 is exactly the same problem as ``how many groups of 35 are in 120''.
	\item Problem 2 gives us a chance to talk about the procedure for decimal division. In the current calendar, we have not yet addressed long division, so students might need a quick reminder of how to do this procedure. We will focus on the procedure itself later; the point of this question is to discuss the idea that we move the decimal place on both numbers when we set up for long division, and this is the same as changing the value of the block.
	\item Problem 3 gives students another chance to practice with this idea. Students might be tripped up by the fact that the division problem does not have a whole number answer; students can feel free to approximate their answer or to give a fraction answer (even though this is a decimal division problem). You don't need to talk about how fractions and decimals are connected here, and it's really okay for the students to just say ``there are approximately 3 groups''.
	\item \#4 can be the most difficult problem for students because it does not follow the pattern of the others. Here, we want to use ``number sense'' instead of needing to ``count the decimal places''.  This is similar to a question we had about decimal multiplication; if we have covered that problem then this problem might be a little easier.  Students should recognize that estimation can be more powerful than memorization. The idea, using the example of $2.3 \div 0.75$, is to estimate that $1 < 2.3 < 3$ and $0.5 < 0.75 < 1$ so we have $1 \div 1 < 2.3 \div 0.75 < 3 \div 0.5$ so if we calculate $23 \div 75 = 0.30\overline{6}$ we must place the decimal point just after the $3$ so our answer is in the correct range. Hence $2.3 \div 0.75 = 3.0\overline{6}$.
	\item Optional: Another possible argument for the ``move the decimal point'' algorithm might be to convert the quantities to fraction form and watch how the powers of 10 ``cancel''. However, this argument does require a connection between division and fractions which we have not yet discussed, so this idea is better later in the course.
\end{itemize}

{\bf Good language:} Discuss the groups and objects per group (as well as the type of division) carefully. This is still a new idea for our students, so they should be clear and precise with communicating these ideas.

{\bf Suggested Timing:} The most important problems here are the first two; the others can be discussed if there's time. Give students about 15 minutes to work on as much of the activity as they have time for, then use about 30 minutes to have students present their work on each of the problems. Be sure to solicit lots of answers for Problem 1(d) and the similar part of problem 3. You should also have students present all the kinds of answers they get for problem 3. Use the last 5-10 minutes to wrap up the discussion and highlight the main ideas.
\end{instructorNotes}

\end{document}