\documentclass[nooutcomes,noauthor]{ximera}

\graphicspath{
  {./}
  {graphics/}
  {../graphics/}
}

\usepackage{chngcntr}

\let\question\relax
\let\endquestion\relax




\newtheoremstyle{SlantTheorem}{\topsep}{\fill}%%% space between body and thm
%\newtheoremstyle{SlantTheorem}{\topsep}{\topsep}%%% space between body and thm
 {\slshape}                      %%% Thm body font
 {}                              %%% Indent amount (empty = no indent)
 {\bfseries\sffamily}            %%% Thm head font
 {}                              %%% Punctuation after thm head
 {3ex}                           %%% Space after thm head
 {\thmname{#1}\thmnumber{ #2}\thmnote{ \bfseries(#3)}}%%% Thm head spec
\theoremstyle{SlantTheorem}
\newtheorem{question}{Question}
\counterwithin*{question}{section}



\let\instructorNotes\relax
\let\endinstructorNotes\relax
%%% instructorNotes environment
\ifhandout
\newenvironment{instructorNotes}[1][false]%
{%
\def\givenatend{\boolean{#1}}\ifthenelse{\boolean{#1}}{\begin{trivlist}\item}{\setbox0\vbox\bgroup}{}
}
{%
\ifthenelse{\givenatend}{\end{trivlist}}{\egroup}{}
}
\else
\newenvironment{instructorNotes}[1][false]%
{%
  \ifthenelse{\boolean{#1}}{\begin{trivlist}\item[\hskip \labelsep\bfseries {\Large Instructor Notes: \\} \hspace{\textwidth} ]}
{\begin{trivlist}\item[\hskip \labelsep\bfseries {\Large Instructor Notes: \\} \hspace{\textwidth} ]}
{}
}
{\end{trivlist}}
\fi


%% Suggested Timing
\newcommand{\timing}[1]{{\bf Suggested Timing: \hspace{2ex}} #1}
\title{Mini cakes}

\begin{document}
\begin{abstract}
\end{abstract}

\maketitle


\begin{problem}
Terry has a cake recipe that calls for $4$ cups of flour. If Terry makes a mini cake using this same recipe but with only $\frac{1}{3}$ of a cup of flour, what fraction of the original cake recipe is Terry making?

\begin{enumerate}
	\item Explain why this is a division problem for $\frac{1}{3} \div 4$. What is one group? One object? Which type of division is this?
	\item Solve the problem using a picture and our meaning of fractions. How does the picture help you to see the answer to this question?
\end{enumerate}
\end{problem}



\begin{problem}
Vlad has a cake recipe that calls for $\frac{8}{3}$ of a cup of flour. If Vlad makes a mini cake using this same recipe but with only $\frac{1}{4}$ of a cup of flour, what fraction of the original cake recipe is Vlad making?
\begin{enumerate}
	\item Explain why this is a division problem for $\frac{1}{4} \div \frac{8}{3}$. What is one group? One object? Which type of division is this?
	\item Solve the problem using a picture and our meaning of fractions. How does the picture help you to see the answer to this question?
\end{enumerate}
\end{problem}


\begin{problem}
Rochelle has a cake recipe that calls for $\frac{7}{9}$ of a cup of flour. If Rochelle makes a mini cake using this same recipe but with only $\frac{2}{5}$ of a cup of flour, what fraction of the original cake recipe is Rochelle making?
\begin{enumerate}
	\item Explain why this is a division problem for $\frac{2}{5} \div \frac{7}{9}$. What is one group? One object? Which type of division is this?
	\item Solve the problem using a picture and our meaning of fractions. How does the picture help you to see the answer to this question?
\end{enumerate}
\end{problem}




\begin{problem}
Write another story problem like those above, but where the cake would be larger than the original recipe. Explain how you know your story is correct and then solve it with a picture.
\end{problem}



\begin{problem}
Write a ``how many groups'' division story for $\frac{11}{2} \div \frac{3}{4}$ and then solve your story with a picture.
\end{problem}




\newpage

\begin{instructorNotes} 



{\bf Main goal:} We solve How Many Groups division problems with fractions.


{\bf Overall picture:} Problem 3 contains the main type of problem we would like students to be able to solve after this activity. The final two problems are ``extra'' and do not need to be completed by all students or discussed.

\begin{itemize}
	\item You may need to remind students that we aren't looking for the ``invert and multiply" solution, here. We'll justify that in a later activity.
	\item The problems in this activity increase in difficulty as students work through them so that they are ramping up to the general case in problem 3.
	\item An important idea here is changing the whole from the cup to the recipe. Once we draw the cups of flour, we have to change the whole to be the recipe. Some students will struggle with this, and it's worth bringing up. If you have a very brave student who makes this mistake you could ask them to present, or perhaps after the presentation for problem 2, you could ask why the pieces aren't $\frac{1}{12}$-sized pieces. 
	\item Some students might draw a single picture to solve these problems, where other students might draw two pictures (one for the full recipe, and one for the partial recipe). If you see both types of pictures, they should both be presented. In either picture, the main goal will be making ``same-size pieces''. If you have a lot of extra time at the end you could connect this to the idea of making common denominators and dividing the numerators.
	\item Students may give mixed-number answers to these questions since they are counting full groups as well as partial groups. This should be presented, and if no one in the class finds the improper fraction answer then that is a good opportunity for the instructor to add to the discussion.
	\item For fractional groups, students should be encouraged to identify what one whole group looks like, and how many pieces the whole group is cut into. Then, they should identify how many of those pieces they want (using our meaning of fractions to find the whole, then the denominator, then the numerator) and report their answer. 
	\item Overall, students should be encouraged to work with their pictures carefully and label their work carefully. Good labeling can go a long way towards helping with some of the confusion that students may face with these problems.
	\item When students make their own story problems in problems $4$ and $5$, be sure that they are correct stories. Students most common mistakes are swapping the order of the fractions (for instance writing a story for $\frac{11}{2} \div \frac{3}{4}$ instead of $\frac{3}{4} \div \frac{11}{2}$ or vice versa, and using the wrong type of division.
\end{itemize}


{\bf Good language:}  Remind students to use the meaning of fractions in their explanations. Help students to see how the picture helps them to solve these problems, rather than drawing a picture after knowing the solution. It may help to draw a separate picture of what one group looks like once you identify one group in your drawing.

Note that students who don't write a good story problem will struggle with this activity!

{\bf Suggested timing:} Give students about 5-8 minutes to think about Problems 1 and 2, and then take about 10 minutes to discuss. Have students present their drawings at the board and discuss. Give students about 5-10 minutes to think about Problem 3, and take 10 minutes to discuss. Finally, give students about 10 minutes to think about Problems 4 and 5 (or pick one of the two if you are running short on time), and then use the rest of the time to discuss.

\end{instructorNotes}



\end{document}