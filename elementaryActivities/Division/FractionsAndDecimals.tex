\documentclass[noauthor, nooutcomes]{ximera}

\usepackage{gensymb}
\usepackage{tabularx}
\usepackage{mdframed}
\usepackage{pdfpages}
%\usepackage{chngcntr}

\let\problem\relax
\let\endproblem\relax

\newcommand{\property}[2]{#1#2}




\newtheoremstyle{SlantTheorem}{\topsep}{\fill}%%% space between body and thm
 {\slshape}                      %%% Thm body font
 {}                              %%% Indent amount (empty = no indent)
 {\bfseries\sffamily}            %%% Thm head font
 {}                              %%% Punctuation after thm head
 {3ex}                           %%% Space after thm head
 {\thmname{#1}\thmnumber{ #2}\thmnote{ \bfseries(#3)}} %%% Thm head spec
\theoremstyle{SlantTheorem}
\newtheorem{problem}{Problem}[]

%\counterwithin*{problem}{section}



%%%%%%%%%%%%%%%%%%%%%%%%%%%%Jenny's code%%%%%%%%%%%%%%%%%%%%

%%% Solution environment
%\newenvironment{solution}{
%\ifhandout\setbox0\vbox\bgroup\else
%\begin{trivlist}\item[\hskip \labelsep\small\itshape\bfseries Solution\hspace{2ex}]
%\par\noindent\upshape\small
%\fi}
%{\ifhandout\egroup\else
%\end{trivlist}
%\fi}
%
%
%%% instructorIntro environment
%\ifhandout
%\newenvironment{instructorIntro}[1][false]%
%{%
%\def\givenatend{\boolean{#1}}\ifthenelse{\boolean{#1}}{\begin{trivlist}\item}{\setbox0\vbox\bgroup}{}
%}
%{%
%\ifthenelse{\givenatend}{\end{trivlist}}{\egroup}{}
%}
%\else
%\newenvironment{instructorIntro}[1][false]%
%{%
%  \ifthenelse{\boolean{#1}}{\begin{trivlist}\item[\hskip \labelsep\bfseries Instructor Notes:\hspace{2ex}]}
%{\begin{trivlist}\item[\hskip \labelsep\bfseries Instructor Notes:\hspace{2ex}]}
%{}
%}
%% %% line at the bottom} 
%{\end{trivlist}\par\addvspace{.5ex}\nobreak\noindent\hung} 
%\fi
%
%


\let\instructorNotes\relax
\let\endinstructorNotes\relax
%%% instructorNotes environment
\ifhandout
\newenvironment{instructorNotes}[1][false]%
{%
\def\givenatend{\boolean{#1}}\ifthenelse{\boolean{#1}}{\begin{trivlist}\item}{\setbox0\vbox\bgroup}{}
}
{%
\ifthenelse{\givenatend}{\end{trivlist}}{\egroup}{}
}
\else
\newenvironment{instructorNotes}[1][false]%
{%
  \ifthenelse{\boolean{#1}}{\begin{trivlist}\item[\hskip \labelsep\bfseries {\Large Instructor Notes: \\} \hspace{\textwidth} ]}
{\begin{trivlist}\item[\hskip \labelsep\bfseries {\Large Instructor Notes: \\} \hspace{\textwidth} ]}
{}
}
{\end{trivlist}}
\fi


%% Suggested Timing
\newcommand{\timing}[1]{{\bf Suggested Timing: \hspace{2ex}} #1}




\hypersetup{
    colorlinks=true,       % false: boxed links; true: colored links
    linkcolor=blue,          % color of internal links (change box color with linkbordercolor)
    citecolor=green,        % color of links to bibliography
    filecolor=magenta,      % color of file links
    urlcolor=cyan           % color of external links
}


\title{Fractions and Decimals}
\author{Vic Ferdinand, Betsy McNeal, Jenny Sheldon}

\begin{document}
\begin{abstract} 
In this activity, we try to understand how fractions and decimals are related.  Can all fractions be written as decimals?  Can all decimals be written as fractions?
\end{abstract}
\maketitle



\begin{problem}
Use our definition of fractions and our bundling system to explain the relationship between $0.1$ and $\frac{1}{10}$.
\end{problem}


\begin{problem}
Use equivalent fractions and your work in the previous problem to convert $\frac{1}{25}$ into a decimal. For a challenge, try $\frac{11}{40}$ as well! Why does your method make sense?
\end{problem}




\begin{problem}
Write a how many in one group story problem for $1 \div 6$, and explain how you know your story problem is correct. Then, use our meaning of fractions to explain why the answer to your story problem is $\frac{1}{6}$.

\end{problem}

\begin{problem}
Why does the previous problem mean that we can use long division to find the decimal representation for $\frac{1}{6}$?
\end{problem}


\begin{problem}
When it comes to fractions (more technically rational numbers), there are two types: terminating and repeating. Look up a definition for each category, and then use that definition to give several examples.

\end{problem}



\begin{problem}
%we get R=0
Use long division to find the decimal representation for $\frac{1}{25}$. What happened? Was this a special case or will this happen for all fractions with a terminating decimal representation? Why?
\end{problem}

\begin{problem}
%not a remainder of zero
Use long division to find the decimal representation for $\frac{1}{6}$. What type of pattern occurred?  Why did this happen?  Was this a special case or will this happen for all fractions with a repeating decimal representation?  Why?
\end{problem}


\begin{problem}
Shade $\frac{3}{8}$ of the grid below. What does your shading tell you that the decimal representation for $\frac{3}{8}$ should be? Use another method to confirm that you got the correct answer.
\begin{center}
    \begin{tikzpicture}[scale=0.5]
    \draw [step=1, help lines] (0,0) grid (10,10);
    \end{tikzpicture}
\end{center}
\end{problem}


\begin{problem}
Shade $\frac{2}{3}$ of the grid below. What does your shading tell you that the decimal representation for $\frac{2}{3}$ should be? Use another method to confirm that you got the correct answer.
\begin{center}
    \begin{tikzpicture}[scale=0.5]
    \draw [step=1, help lines] (0,0) grid (10,10);
    \end{tikzpicture}
\end{center}
\end{problem}



\begin{problem}
So far, we've been changing fractions into decimals. Let's go the other way! Use what you know about the base-ten system to write the following decimals as fractions.
\begin{enumerate}
    \item $0.04$
    \item $0.483$
    \item $0.0003205$
\end{enumerate}
\end{problem}


\begin{problem}
Can every decimal number be written as a fraction? Explain your thinking with examples!
\end{problem}

%\begin{problem}\label{Shampoo5}
%The decimals in the previous problem were all terminating. What about repeating decimals? Using the facts that $\frac{1}{9}=0.\overline{1}$,
%$\frac{1}{99}=0.\overline{01}$, $\frac{1}{999}=0.\overline{001}$,
%etc., find a fraction representation for the following numbers:
%
%\begin{enumerate}
%\item $0.\overline{7}$
%\item $0.\overline{357}$
%\item $0.\overline{4598}$
%\item $0.23\overline{4598}$
%\item $23.\overline{459}$
%\item $76.\overline{214}$
%\end{enumerate}


%\end{problem}



 \newpage
\begin{instructorNotes}

{\bf Main Goal:} We would like students to come away from this activity with at least two ways of converting fractions to decimals, and be able to explain why those methods are sensible.

{\bf Overall Picture:} We are drawing together many ideas in the course in this activity; don't be afraid to spend time if students need review on any topic.

 \begin{itemize}
 \item The first problem is designed to help students bring together the fractions and bundling. This could be quick if we spent significant time on it earlier in the semester, but it is still a good review. The main goal is to see that $0.1$ is made when we take our unit and divide it into 10 equal parts, selecting one of those parts. But this is exactly our definition of $\frac{1}{10}$.
 \item Once we see that we can understand fractions with powers of 10 in the denominator as the same as decimals, our goal is to convert fractions to having such a denominator. These two examples (and the reverse problem later in the activity where we convert from decimals to fractions) should emphasize this point.
 \item The problem about repeating and terminating decimals is essentially to clarify terminology. This should be a very quick discussion.
 
 \item Connecting long division and fractions should again be a review problem, but again good to discuss in detail. The idea is to see that solving the how many in one group problem involves splitting up the single item into six equal pieces, and placing one of these pieces in each group. Once we understand this connection, the answer to a long division problem is the same as the fraction, even if we represent it as a decimal. Note that if the numerator were not 1 in this case, we would be in the situation of the pies problem.
 
 \item When we use long division to find decimal representations, there are several ideas to keep in mind. The most important part of these two problems is that students can execute the long division and explain why this gives us the decimal associated with the fraction (using the previous arguments). If we have time, it's interesting to explore some more ideas here, such as how the terminating or repeating process interacts with the long division (remainder of zero versus never getting a remainder of zero). We can also explore the fact that the repeating decimal has a pattern of remainders. See the next bullet point for optional ideas to discuss if time.
 
  \item In the problem where we divide $1 \div 6$, there are three points that need to be made to complete the argument:  That the division process is necessarily infinite (no zero remainder because of the 6 in the denominator -- we have not justified this yet, but the idea should follow from our work with terminating decimals), the number of possible remainders is finite and thus we will necessarily run into a remainder we have seen before, and finally, because we are dropping a zero down each time (leaving a multiple of ten to divide into), the sequence of divisions will be the same, generating the same sequence of quotients. This argument is difficult for students to put together (most have the  pieces, some all of them, but making a full coherent argument is difficult).
  
  \item The Hundredths Grid (shading problems) is a very nice place to encourage students to be creative, and to see lots of different types of solutions presented. Here are some of the strategies students have used in the past, with the example of shading $\frac16$ of the grid (note that the actual problems are different fractions). Some students will cut the box into six equal pieces and shade one of them. Other students will shade 1 out of every 6 boxes (connecting the $\frac{1}{6}$ to the ratio $1:6$). Still other clever methods have been seen in the past! This is a nice place to remind the students of the relationship between a fraction, a rate, and a ratio.

\item Activity 8R extends the problem of converting terminating decimals to fractions. In the activity, we convert repeating decimals to fractions. If the students are ready for this extension, the activity can make a nice homework exercise.

\item The final question gives us a chance to talk briefly about irrational numbers. Depending on time, you can have the students come up with examples of irrationals. The most common first examples are famous irrationals like $\pi$, $e$, $\sqrt{2}$. You can also encourage students to come up with a ``homemade'' irrational number such as $0.01001000100001\dots$, which encourages students to think about patterns that are not repeating, and perhaps also helps students to understand that there are a lot more irrational numbers out there than they perhaps imagined! 


 \end{itemize}


{\bf Suggested Timing:} We have two class periods to work through these ideas. Start by giving students 5-10 minutes to think about the first three problems, then spend about 20 minutes in discussion with groups presenting their work. Next, give students about 10 minutes to think about problems 4 and 5, and then discuss. If you have time left in the first session, give students time to begin working through problems 6 and 7. On the second day, give 5-10 minutes more work time on problems 6 and 7, and then discuss. Here we will want to emphasize the patterns happening with terminating and repeating decimals, so this discussion will take more like 30 minutes. Finally, give students 10 minutes to think about the final three problems, and use what time you have left for discussion.
\end{instructorNotes}





\end{document}
