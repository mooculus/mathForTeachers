\documentclass[nooutcomes]{ximera}

\usepackage{gensymb}
\usepackage{tabularx}
\usepackage{mdframed}
\usepackage{pdfpages}
%\usepackage{chngcntr}

\let\problem\relax
\let\endproblem\relax

\newcommand{\property}[2]{#1#2}




\newtheoremstyle{SlantTheorem}{\topsep}{\fill}%%% space between body and thm
 {\slshape}                      %%% Thm body font
 {}                              %%% Indent amount (empty = no indent)
 {\bfseries\sffamily}            %%% Thm head font
 {}                              %%% Punctuation after thm head
 {3ex}                           %%% Space after thm head
 {\thmname{#1}\thmnumber{ #2}\thmnote{ \bfseries(#3)}} %%% Thm head spec
\theoremstyle{SlantTheorem}
\newtheorem{problem}{Problem}[]

%\counterwithin*{problem}{section}



%%%%%%%%%%%%%%%%%%%%%%%%%%%%Jenny's code%%%%%%%%%%%%%%%%%%%%

%%% Solution environment
%\newenvironment{solution}{
%\ifhandout\setbox0\vbox\bgroup\else
%\begin{trivlist}\item[\hskip \labelsep\small\itshape\bfseries Solution\hspace{2ex}]
%\par\noindent\upshape\small
%\fi}
%{\ifhandout\egroup\else
%\end{trivlist}
%\fi}
%
%
%%% instructorIntro environment
%\ifhandout
%\newenvironment{instructorIntro}[1][false]%
%{%
%\def\givenatend{\boolean{#1}}\ifthenelse{\boolean{#1}}{\begin{trivlist}\item}{\setbox0\vbox\bgroup}{}
%}
%{%
%\ifthenelse{\givenatend}{\end{trivlist}}{\egroup}{}
%}
%\else
%\newenvironment{instructorIntro}[1][false]%
%{%
%  \ifthenelse{\boolean{#1}}{\begin{trivlist}\item[\hskip \labelsep\bfseries Instructor Notes:\hspace{2ex}]}
%{\begin{trivlist}\item[\hskip \labelsep\bfseries Instructor Notes:\hspace{2ex}]}
%{}
%}
%% %% line at the bottom} 
%{\end{trivlist}\par\addvspace{.5ex}\nobreak\noindent\hung} 
%\fi
%
%


\let\instructorNotes\relax
\let\endinstructorNotes\relax
%%% instructorNotes environment
\ifhandout
\newenvironment{instructorNotes}[1][false]%
{%
\def\givenatend{\boolean{#1}}\ifthenelse{\boolean{#1}}{\begin{trivlist}\item}{\setbox0\vbox\bgroup}{}
}
{%
\ifthenelse{\givenatend}{\end{trivlist}}{\egroup}{}
}
\else
\newenvironment{instructorNotes}[1][false]%
{%
  \ifthenelse{\boolean{#1}}{\begin{trivlist}\item[\hskip \labelsep\bfseries {\Large Instructor Notes: \\} \hspace{\textwidth} ]}
{\begin{trivlist}\item[\hskip \labelsep\bfseries {\Large Instructor Notes: \\} \hspace{\textwidth} ]}
{}
}
{\end{trivlist}}
\fi


%% Suggested Timing
\newcommand{\timing}[1]{{\bf Suggested Timing: \hspace{2ex}} #1}




\hypersetup{
    colorlinks=true,       % false: boxed links; true: colored links
    linkcolor=blue,          % color of internal links (change box color with linkbordercolor)
    citecolor=green,        % color of links to bibliography
    filecolor=magenta,      % color of file links
    urlcolor=cyan           % color of external links
}


\title{More With Fractions}
\author{Vic Ferdinand, Betsy McNeal, Jenny Sheldon}

\begin{document}
\begin{abstract}
 This activity will investigate equivalent fractions and the reasoning behind the procedure for making equivalent fractions $\frac{A}{B} = \frac{A \times N}{B \times N}$.
\end{abstract}
\maketitle



\begin{problem} \label{MoreWithFractions1}
Fold a strip of paper into fourths and shade in $\frac{3}{4}$ of the paper strip. Now fold each of your fourths into 3 equal pieces.  
\begin{enumerate}
    \item   How many parts are in the strip of paper now?
    \item   Are these parts equal to each other?  Why?
    \item   What fraction of the original paper strip is each new part?
    \item   How many of the parts are shaded?  How do you know?
    \item   What new fraction of your paper strip is now shown by the shading?  
   
\end{enumerate}
\end{problem}
%\vskip 2in

\begin{problem} \label{MoreWithFractions2}
 Let's try this again, but without the paper strip.  Instead, \emph{imagine} working with a new paper strip.  
\begin{enumerate}
    \item Explain to your group how you would fold this imaginary strip to show $\frac{119}{357}$.   
    \item Explain to your group how you would make an \emph{equivalent fraction} using the same paper strip.  That is, what steps would you take?
   \item   How many parts are in the imaginary strip of paper now?
    \item   Are these parts equal to each other?  Why?
    \item   What fraction of the original imaginary paper strip is each new part?
    \item   How many of the parts are shaded? How do you know?
    \item   What new fraction of your imaginary paper strip is now shown by the shading?  
\end{enumerate}
\end{problem}
%\end{enumerate}

\newpage

\begin{problem}\label{MoreWithFractions3}
Suppose you were trying to convert the fraction $\frac{19}{20}$ into a percent.  You probably learned to do this by solving the following problem:

\[ \frac{19}{20} = \frac{?}{100} \]

Explain how you might use a strip of (imaginary!) paper  to solve this problem.
\end{problem}

%\vskip 1.5in



\begin{problem}
When you learned how to ``make equivalent fractions" in school, you learned a rule that was something like ``multiply top and bottom by the same number", as shown in this formula:

  \[ \frac{A}{B} = \frac{A \times N}{B \times N}. \]

Explain how this rule encapsulates the ideas of the process that you have done in Problems \ref{MoreWithFractions1}-\ref{MoreWithFractions3}.
\end{problem}
%\vskip 1.5in





%\end{document}
\begin{problem}
 Repeat Problem \ref{MoreWithFractions2} for the fraction $\frac{119}{357}$. 
\begin{enumerate}
    \item Explain to your group how you would fold this imaginary strip to show this fraction.    
    \item What equivalent fraction would you make if you folded each of the original 357 parts into 2 parts?
   \item   How many parts would be in the imaginary strip of paper now?
    \item  Will these parts be equal to each other?  Why?
    \item   What fraction of the original imaginary paper strip is each new part?
    \item   How many of the parts are shaded?
    \item   What new fraction of your imaginary paper strip is now shown by the shading?  
\end{enumerate}
\end{problem}


\newpage

\begin{instructorNotes}
The purpose of this activity is to connect the familiar ``multiply top and bottom'' procedure for making equivalent fractions to our definition of fractions and to physical actions with paper folding.  We try to focus students on what each step means, and how the actions with the paper communicate that the amount represented by the fraction is unchanged.




{\bf Suggested Timing:} This activity should take the whole class period to fully explore.
\end{instructorNotes}


\end{document}