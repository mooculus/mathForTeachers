\documentclass{ximera}
\usepackage{gensymb}
\usepackage{tabularx}
\usepackage{mdframed}
\usepackage{pdfpages}
%\usepackage{chngcntr}

\let\problem\relax
\let\endproblem\relax

\newcommand{\property}[2]{#1#2}




\newtheoremstyle{SlantTheorem}{\topsep}{\fill}%%% space between body and thm
 {\slshape}                      %%% Thm body font
 {}                              %%% Indent amount (empty = no indent)
 {\bfseries\sffamily}            %%% Thm head font
 {}                              %%% Punctuation after thm head
 {3ex}                           %%% Space after thm head
 {\thmname{#1}\thmnumber{ #2}\thmnote{ \bfseries(#3)}} %%% Thm head spec
\theoremstyle{SlantTheorem}
\newtheorem{problem}{Problem}[]

%\counterwithin*{problem}{section}



%%%%%%%%%%%%%%%%%%%%%%%%%%%%Jenny's code%%%%%%%%%%%%%%%%%%%%

%%% Solution environment
%\newenvironment{solution}{
%\ifhandout\setbox0\vbox\bgroup\else
%\begin{trivlist}\item[\hskip \labelsep\small\itshape\bfseries Solution\hspace{2ex}]
%\par\noindent\upshape\small
%\fi}
%{\ifhandout\egroup\else
%\end{trivlist}
%\fi}
%
%
%%% instructorIntro environment
%\ifhandout
%\newenvironment{instructorIntro}[1][false]%
%{%
%\def\givenatend{\boolean{#1}}\ifthenelse{\boolean{#1}}{\begin{trivlist}\item}{\setbox0\vbox\bgroup}{}
%}
%{%
%\ifthenelse{\givenatend}{\end{trivlist}}{\egroup}{}
%}
%\else
%\newenvironment{instructorIntro}[1][false]%
%{%
%  \ifthenelse{\boolean{#1}}{\begin{trivlist}\item[\hskip \labelsep\bfseries Instructor Notes:\hspace{2ex}]}
%{\begin{trivlist}\item[\hskip \labelsep\bfseries Instructor Notes:\hspace{2ex}]}
%{}
%}
%% %% line at the bottom} 
%{\end{trivlist}\par\addvspace{.5ex}\nobreak\noindent\hung} 
%\fi
%
%


\let\instructorNotes\relax
\let\endinstructorNotes\relax
%%% instructorNotes environment
\ifhandout
\newenvironment{instructorNotes}[1][false]%
{%
\def\givenatend{\boolean{#1}}\ifthenelse{\boolean{#1}}{\begin{trivlist}\item}{\setbox0\vbox\bgroup}{}
}
{%
\ifthenelse{\givenatend}{\end{trivlist}}{\egroup}{}
}
\else
\newenvironment{instructorNotes}[1][false]%
{%
  \ifthenelse{\boolean{#1}}{\begin{trivlist}\item[\hskip \labelsep\bfseries {\Large Instructor Notes: \\} \hspace{\textwidth} ]}
{\begin{trivlist}\item[\hskip \labelsep\bfseries {\Large Instructor Notes: \\} \hspace{\textwidth} ]}
{}
}
{\end{trivlist}}
\fi


%% Suggested Timing
\newcommand{\timing}[1]{{\bf Suggested Timing: \hspace{2ex}} #1}




\hypersetup{
    colorlinks=true,       % false: boxed links; true: colored links
    linkcolor=blue,          % color of internal links (change box color with linkbordercolor)
    citecolor=green,        % color of links to bibliography
    filecolor=magenta,      % color of file links
    urlcolor=cyan           % color of external links
}
\title{Fun With Fractions}
\author{Vic Ferdinand, Betsy McNeal, Jenny Sheldon}

\begin{document}
\begin{abstract}
\end{abstract}
\maketitle



\begin{problem}
At a carnival, you play a certain game where you roll a six-sided die.  If you roll an even number, you win a (small) prize.  If you roll an odd number, you get nothing.  What is the probability that you will win a prize?  How do you know?
\end{problem}

\begin{problem}
What does the previous problem have to do with our meaning of fractions?  You might want to draw some pictures to help you explain.
\end{problem}

\begin{problem}
At the same carnival booth, if you win a small prize, you can turn it in for a chance to roll the die again.  On your second roll, if you roll a six, you win a (medium) prize.  What is the probability that:
\begin{enumerate}
\item if you have already won the small prize, you will win a medium prize?
\item if you have just started playing the game (and have not won or lost the small prize yet) that you will win the medium prize?
\end{enumerate}

\end{problem}

\begin{problem}
What might the rules be for winning a large prize?
\end{problem}

\newpage
\begin{instructorNotes}
This activity serves as our introduction to probability.  In our course, students have already spent time working on counting problems of increasing difficulty, and now we turn to probability as a sort of different kind of information we can get from counting.  We would like students to make connections between the meaning of probability and the definition of fractions, and apply this connection to solve a multi-stage problem.

We use these activities about counting and probability to help cement the meaning of various operations for students.  Throughout these activities, we expect students to justify their operations of choice in their explanations.  There are a few main types of counting strategies we see from our students: arrays, ordered lists, tree diagrams, and algebraic expressions.  Students tend to want to use more complicated strategies before trying simpler strategies, so we are often reminding them to write down some examples, or to make sure their list is well organized. When discussing this activity as a whole class, we like to have students demonstrate multiple solution methods for each problem, even when not specifically requested by the problem.  We also continue to emphasize structure any time we can recognize a problem as being like another problem we have previously solved, as in ``Compare and Count''.

In the discussion following this activity, we introduce any vocabulary we intend to use for the probability unit, including outcome, sample space, and event space.  We also introduce the idea of outcomes being equally likely, and discuss the importance of having equally likely outcomes for how we most often calculate probability by counting the sample space and the event space.

We generally follow this activity with ``High Rollers'', whose aim is to discuss experimental and theoretical probability.

\timing{This activity is designed for one class period.}
\end{instructorNotes}


\end{document}