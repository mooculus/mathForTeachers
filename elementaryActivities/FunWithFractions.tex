\documentclass{ximera}

\graphicspath{
  {./}
  {graphics/}
  {../graphics/}
}

\usepackage{chngcntr}

\let\question\relax
\let\endquestion\relax




\newtheoremstyle{SlantTheorem}{\topsep}{\fill}%%% space between body and thm
%\newtheoremstyle{SlantTheorem}{\topsep}{\topsep}%%% space between body and thm
 {\slshape}                      %%% Thm body font
 {}                              %%% Indent amount (empty = no indent)
 {\bfseries\sffamily}            %%% Thm head font
 {}                              %%% Punctuation after thm head
 {3ex}                           %%% Space after thm head
 {\thmname{#1}\thmnumber{ #2}\thmnote{ \bfseries(#3)}}%%% Thm head spec
\theoremstyle{SlantTheorem}
\newtheorem{question}{Question}
\counterwithin*{question}{section}



\let\instructorNotes\relax
\let\endinstructorNotes\relax
%%% instructorNotes environment
\ifhandout
\newenvironment{instructorNotes}[1][false]%
{%
\def\givenatend{\boolean{#1}}\ifthenelse{\boolean{#1}}{\begin{trivlist}\item}{\setbox0\vbox\bgroup}{}
}
{%
\ifthenelse{\givenatend}{\end{trivlist}}{\egroup}{}
}
\else
\newenvironment{instructorNotes}[1][false]%
{%
  \ifthenelse{\boolean{#1}}{\begin{trivlist}\item[\hskip \labelsep\bfseries {\Large Instructor Notes: \\} \hspace{\textwidth} ]}
{\begin{trivlist}\item[\hskip \labelsep\bfseries {\Large Instructor Notes: \\} \hspace{\textwidth} ]}
{}
}
{\end{trivlist}}
\fi


%% Suggested Timing
\newcommand{\timing}[1]{{\bf Suggested Timing: \hspace{2ex}} #1}
\title{Fun With Fractions}
\author{Vic Ferdinand, Betsy McNeal, Jenny Sheldon}

\begin{document}
\begin{abstract}
\end{abstract}
\maketitle

\begin{instructorIntro}
This activity serves as an introduction to probability.  Students should make connections with the definition of fractions, and apply this connection to solve a multi-stage problem (which will likely be difficult for them).

After this activity, you should introduce any vocabulary you intend to use for the probability unit.  Suggested vocabulary is: outcome, sample space, event space.  You should also begin to discuss the importance of equally likely outcomes.
\end{instructorIntro}

\begin{problem}
At a carnival, you play a certain game where you roll a six-sided die.  If you roll an even number, you win a (small) prize.  If you roll an odd number, you get nothing.  What is the probability that you will win a prize?  How do you know?
\end{problem}

\begin{problem}
What does the previous problem have to do with our Math 1125 definition of fractions?  You might want to draw some pictures to help you explain.
\end{problem}

\begin{problem}
At the same carnival booth, if you win a small prize, you can turn it in for a chance to roll the die again.  On your second roll, if you roll a six, you win a (medium) prize.  What is the probability that:
\begin{enumerate}
\item if you have already won the small prize, you will win a medium prize?
\item if you have just started playing the game (and have not won or lost the small prize yet) that you will win the medium prize?
\end{enumerate}

\end{problem}

\begin{problem}
What might the rules be for winning a large prize?
\end{problem}


\end{document}