\documentclass{ximera}

\graphicspath{
  {./}
  {graphics/}
  {../graphics/}
}

\usepackage{chngcntr}

\let\question\relax
\let\endquestion\relax




\newtheoremstyle{SlantTheorem}{\topsep}{\fill}%%% space between body and thm
%\newtheoremstyle{SlantTheorem}{\topsep}{\topsep}%%% space between body and thm
 {\slshape}                      %%% Thm body font
 {}                              %%% Indent amount (empty = no indent)
 {\bfseries\sffamily}            %%% Thm head font
 {}                              %%% Punctuation after thm head
 {3ex}                           %%% Space after thm head
 {\thmname{#1}\thmnumber{ #2}\thmnote{ \bfseries(#3)}}%%% Thm head spec
\theoremstyle{SlantTheorem}
\newtheorem{question}{Question}
\counterwithin*{question}{section}



\let\instructorNotes\relax
\let\endinstructorNotes\relax
%%% instructorNotes environment
\ifhandout
\newenvironment{instructorNotes}[1][false]%
{%
\def\givenatend{\boolean{#1}}\ifthenelse{\boolean{#1}}{\begin{trivlist}\item}{\setbox0\vbox\bgroup}{}
}
{%
\ifthenelse{\givenatend}{\end{trivlist}}{\egroup}{}
}
\else
\newenvironment{instructorNotes}[1][false]%
{%
  \ifthenelse{\boolean{#1}}{\begin{trivlist}\item[\hskip \labelsep\bfseries {\Large Instructor Notes: \\} \hspace{\textwidth} ]}
{\begin{trivlist}\item[\hskip \labelsep\bfseries {\Large Instructor Notes: \\} \hspace{\textwidth} ]}
{}
}
{\end{trivlist}}
\fi


%% Suggested Timing
\newcommand{\timing}[1]{{\bf Suggested Timing: \hspace{2ex}} #1}

\author{Vic Ferdinand}
\title{So-So}
\outcome{Apply integrals to real-life scenarios.}
\outcome{Compute average values of functions.}

\begin{document}
\begin{abstract}
\end{abstract}
\maketitle

\begin{instructorIntro}
We wrap up the course by exploring (mostly geometric) applications of integration- extending at least two ideas that are dealt with in early algebra or geometry.  

The rest of the activities use the Riemann sum that was developed intuitively in Winter Storm Warning and consider that the sum (as the size of the intervals ($\Delta x$) approaches zero) approaches the definite integral.  The key is recognizing what the ``$f'(x)$'' now is (instead of a rate, but rather some other measurement dependent upon the function $f(x)$).  This also could be done with the area between curves, but they are better able to intuitively see that without the Riemann sum. 

Because of time, I usually derive all of the formulas in a whole-class setting and let them work on the problems- often assigning certain problems to certain groups. The hardest application is the volume by slicing, which uses the volume of a prism as the $A(x) \Delta x$.  They usually will intuitively understand the ``loaf of sliced bread'' argument (bringing a loaf to class helps!), but have difficulty in actually finding the $A(x)$ in the exercises, particularly if it involves a triangle (Often, we end up doing these as a whole class in the end).

For the average value of a function, I've found it easier for the students to compare the area under the curve of a function with the equivalent area of a rectangle of length $\Delta x$ (`` $b - a$'') and height being the average value.  Then we just solve for the average value by dividing both sides of the equation by $b – a$.  However, you can also use the given Riemann Sum argument as well.

\end{instructorIntro}

Our final (albeit non-geometric) application of Riemann sums is finding the average value of a function's ($f(x)$) values on an interval (call the interval, of all things, $[a, b]$).  For example, if $f(x)$ is the temperature at time $x$, what is the average temperature for the day?

Now, if we had a finite number of $y$-values, say $14$ of them, we'd add all $14$ function values and divide by $14$ for the average (i.e., the number that would be the constant value if all values were the same).  

(What kind of division is this and why?)

Our problem is that there are an {\em infinite} number of function values and we don't have the machinery to add an infinite number of values and divide by infinity!  Thus, how can we get that elusive average value?

Let's take the values at equal intervals along the $x$-axis.  That is, if we have $n$ intervals (i.e., a finite number of intervals), the length of each interval (i.e.,  $\Delta x$) is  $\frac{b-a}{n}$.  Now, as we just argued above with $n = 14$,  the average value of all $n$ functional values is the sum of those functional values divided by 
\[
n =  \frac{f(a) + f(2\text{nd}) + f(3\text{rd}) + \dots + f(b)}{n}.
\]
But this is the same (definition of fraction addition) as  
\[
\frac{f(a)}{n} + \frac{f(2\text{nd})}{n} + \frac{f(3\text{rd})}{n} + \dots + \frac{f(b)}{n} = \sum \frac{f(x's)}{n}.
\]

Unfortunately, this is not written as the sum of ``(somethings) times  $\Delta x$'' to get that \underline{\hspace{0.5in}} (named after a famous $19$th century mathematician named Bernhard Riemann).  

So, we take advantage that $\Delta x$ is the same as $\frac{b-a}{n}$  and multiply and divide each term by $(b-a)$ and rearrange.  Thus, we have:
\begin{align*}
\sum \frac{f(x's)}{n} &= \sum\left ( \frac{f(x's)}{n}\times \frac{b-a}{b-a}\right)\\ 
&= \sum (f(x's)) \times \frac{b-a}{n}\times \frac{1}{b-a} \\
&= \frac{1}{b-a}\sum (f(x's))\times \Delta x,
\end{align*}
with the last equality because $\frac{1}{b-a}$ is a constant factor of all the addends and thus can be factored out of the sum.  Also, this gets more and more accurate to the actual elusive average as $\Delta x$ gets smaller.  Thus, we have a \underline{\hspace{0.5in}} (named after a famous $19$th century mathematician named Bernhard Riemann)!

Now write a formula for the average value of a function $f(x)$ on an interval $[a, b]$.

After all that work, is there a ``middle-school'' way to find the same formula by just considering what kind of division we do when finding the average and applying that to what we know about the area under $f(x)$ and the area of a rectangle?

Find the average value of the following functions:

\begin{problem} 
$f(x) = x+5$ on $[2, 8]$ (Does your answer make sense?)
\end{problem}
\begin{problem} 
$f(x) = x^2+7x-3$ on $[1, 6]$
\end{problem}
\begin{problem} 
$f(x) = 2x^{\frac{3}{2}}-89$ on $[3, 8]$
\end{problem}


\end{document}