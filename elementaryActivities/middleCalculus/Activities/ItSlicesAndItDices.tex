\documentclass{ximera}
\usepackage{gensymb}
\usepackage{tabularx}
\usepackage{mdframed}
\usepackage{pdfpages}
%\usepackage{chngcntr}

\let\problem\relax
\let\endproblem\relax

\newcommand{\property}[2]{#1#2}




\newtheoremstyle{SlantTheorem}{\topsep}{\fill}%%% space between body and thm
 {\slshape}                      %%% Thm body font
 {}                              %%% Indent amount (empty = no indent)
 {\bfseries\sffamily}            %%% Thm head font
 {}                              %%% Punctuation after thm head
 {3ex}                           %%% Space after thm head
 {\thmname{#1}\thmnumber{ #2}\thmnote{ \bfseries(#3)}} %%% Thm head spec
\theoremstyle{SlantTheorem}
\newtheorem{problem}{Problem}[]

%\counterwithin*{problem}{section}



%%%%%%%%%%%%%%%%%%%%%%%%%%%%Jenny's code%%%%%%%%%%%%%%%%%%%%

%%% Solution environment
%\newenvironment{solution}{
%\ifhandout\setbox0\vbox\bgroup\else
%\begin{trivlist}\item[\hskip \labelsep\small\itshape\bfseries Solution\hspace{2ex}]
%\par\noindent\upshape\small
%\fi}
%{\ifhandout\egroup\else
%\end{trivlist}
%\fi}
%
%
%%% instructorIntro environment
%\ifhandout
%\newenvironment{instructorIntro}[1][false]%
%{%
%\def\givenatend{\boolean{#1}}\ifthenelse{\boolean{#1}}{\begin{trivlist}\item}{\setbox0\vbox\bgroup}{}
%}
%{%
%\ifthenelse{\givenatend}{\end{trivlist}}{\egroup}{}
%}
%\else
%\newenvironment{instructorIntro}[1][false]%
%{%
%  \ifthenelse{\boolean{#1}}{\begin{trivlist}\item[\hskip \labelsep\bfseries Instructor Notes:\hspace{2ex}]}
%{\begin{trivlist}\item[\hskip \labelsep\bfseries Instructor Notes:\hspace{2ex}]}
%{}
%}
%% %% line at the bottom} 
%{\end{trivlist}\par\addvspace{.5ex}\nobreak\noindent\hung} 
%\fi
%
%


\let\instructorNotes\relax
\let\endinstructorNotes\relax
%%% instructorNotes environment
\ifhandout
\newenvironment{instructorNotes}[1][false]%
{%
\def\givenatend{\boolean{#1}}\ifthenelse{\boolean{#1}}{\begin{trivlist}\item}{\setbox0\vbox\bgroup}{}
}
{%
\ifthenelse{\givenatend}{\end{trivlist}}{\egroup}{}
}
\else
\newenvironment{instructorNotes}[1][false]%
{%
  \ifthenelse{\boolean{#1}}{\begin{trivlist}\item[\hskip \labelsep\bfseries {\Large Instructor Notes: \\} \hspace{\textwidth} ]}
{\begin{trivlist}\item[\hskip \labelsep\bfseries {\Large Instructor Notes: \\} \hspace{\textwidth} ]}
{}
}
{\end{trivlist}}
\fi


%% Suggested Timing
\newcommand{\timing}[1]{{\bf Suggested Timing: \hspace{2ex}} #1}




\hypersetup{
    colorlinks=true,       % false: boxed links; true: colored links
    linkcolor=blue,          % color of internal links (change box color with linkbordercolor)
    citecolor=green,        % color of links to bibliography
    filecolor=magenta,      % color of file links
    urlcolor=cyan           % color of external links
}

\author{Vic Ferdinand}
\title{It Slices and it Dices! (Or: Round and Round She Goes! Or: The Spirit of '76)}
\outcome{Apply integrals to real-life scenarios.}
\outcome{Find volumes by slicing.}
\outcome{Find volumes of solids of revolution.}

\begin{document}
\begin{abstract}
\end{abstract}
\maketitle

\begin{instructorIntro}
We wrap up the course by exploring (mostly geometric) applications of integration- extending at least two ideas that are dealt with in early algebra or geometry.  

The rest of the activities use the Riemann sum that was developed intuitively in Winter Storm Warning and consider that the sum (as the size of the intervals ($\Delta x$) approaches zero) approaches the definite integral.  The key is recognizing what the ``$f'(x)$'' now is (instead of a rate, but rather some other measurement dependent upon the function $f(x)$).  This also could be done with the area between curves, but they are better able to intuitively see that without the Riemann sum. 

Because of time, I usually derive all of the formulas in a whole-class setting and let them work on the problems- often assigning certain problems to certain groups. The hardest application is the volume by slicing, which uses the volume of a prism as the $A(x) \Delta x$.  They usually will intuitively understand the ``loaf of sliced bread'' argument (bringing a loaf to class helps!), but have difficulty in actually finding the $A(x)$ in the exercises, particularly if it involves a triangle (Often, we end up doing these as a whole class in the end).

The volumes of solids of revolution is much easier (although they often don't see it as a special case of the slicing, perhaps because the revolution goes behind the plane).  The biggest difficulty seems to be in problems where we are revolving a region between curves about the $x$-axis in keeping the ``top and bottom'' volumes separate (i.e., they are tempted to write $(f(x)-g(x))^2$ rather than $(f(x))^2 - (g(x))^2$).  If you don't do anything else here, try to squeeze in the volume of a cone and sphere.

\end{instructorIntro}

A famous application of integration is not only to find areas, but {\em volumes} as well.  Although the volume of more general figures requires multi-variable calculus (things called double and triple integrals that come from double and triple Riemann sums- not too far afield from what we've accomplished here anyway!), we can use single-variable calculus to find volumes of special $3$-D figures:  Those whose cross-sections perpendicular to the $x$-axis are defined by a specific shape and a single variable function.

First, a review of geometry:  A {\em prism} is a figure with a base and a top with congruent (same shape and size) cross-sections throughout (A box is the most famous prism- if you cut anywhere parallel to the base, you have a rectangle that is congruent to the base rectangle. A cylinder is the second most famous:  its base and parallel cross-sections are circles).  It turns out that the volume (the amount of $3$D space inside) of a prism is its (Area of the base) times the (Vertical Height):  We basically have the base repeated infinitely in its cross-sections from top to bottom.  The actual formula was argued in the pool problem in HW \#6 (putting unit cubes on top of unit squares).  If we turn the prism on its side and lay it along the $x$-axis (So the base and cross-sections are perpendicular to the $x$-axis), its volume doesn't change at all (well, duh), but we can look at the volume of as the sum of ``(Area of a cross-section) times (length along $x$-axis)''
      
(Think of a loaf of bread laying on a table with the crust end against the $y$-axis and the loaf extending along the $x$-axis with same-shaped, although not necessarily similar, slices - with one slice being a ``cross-section'').

Due to time constraints this semester, we're going to only consider the case of the cross-sections being circles bounded by the graph of a positive function $f(x)$ and its reflection about the $x$-axis (Think of a football with the $x$-axis going through from one pointy end to the other).  An easier way to visualize this is to think of the graph of $f(x)$ (always in the above the $x$-axis for this course) as being {\em revolved about the $x$-axis} to form the solid.  Thus, we call these ``solids of revolution'' (much different than ``Daughters of the Revolution'').  

We would like to find the volume of this object   This is not a cylinder, but we could cut it into tiny slices (cuts made perpendicular to the $x$-axis- like slicing the loaf of bread) so that we have a bunch of objects that are essentially ``thin cylinders'' (Since $f(x)$  won't change much over a small interval).  Then, the area of the cross-section circular slice won't change much over the short interval, but {\em will be} dependent on at what $x$-value we sliced at.  The area of a cross-section is thus a function of $x$. (call it $A(x)$).  

Since the cross-section is a circle, whose area is  $\pi r^2$, why does it make sense that the ``radius'' at $x$ is $f(x)$ itself?  And why would the volume of the thin cylinder be  $\pi r^2 \times \Delta x = \pi (f(x))^2 \times \Delta x$, where $Delta x$ is the length of each little interval (Is the thin slice a ``prism''?)?  What would we then do to these ``little'' volumes of these ``thin cylinders'' to find the volume of the whole object?  And what would $\Delta x$ need to do in order for the estimation to become more-and-more accurate?  Do we have a (fill in the blank) \underline{\hspace{0.5in}} (named after a famous 19th century mathematician named Bernhard Riemann)?  What would be the formula of the exact volume of the entire figure (assuming it lies along the $x$-axis on the interval $[a, b]$)?


\begin{question} 
Find the volume of the solid generated by rotating the graph of $f(x) = x^3$ (on $[2,5]$) about the $x$-axis.  What will this solid look like?
\end{question}
\begin{question} 
Set up the integral that would find the volume of the solid generated by rotating the graph of $f(x) = 5x-x^2$ (on $[0,5]$) about the $x$-axis.  What will this solid look like?
\end{question}
\begin{question} 
Set up the integral that would find the volume of the solid generated by rotating the region between the graphs of $f(x) = x^2-5x+17$ and $g(x) = 3x+10$ about the $x$-axis. (What happens in the middle of this solid?)
\end{question}
\begin{question} 
Derive the formula for the volume of a sphere of radius $r$ by considering it a solid of revolution (the graph to be rotated is in the $1$st and $2$nd quadrants.  Fun Fact:  The equation of a circle of radius $r$ centered at $(0, 0)$ is  $x^2 + y^2 = r^2$).
\end{question}
\begin{question} 
Derive the formula for the volume of a cone of base radius $r$ and height $h$ by considering it to be a solid of revolution. (Hint:  What two points do you know will be on the function that is rotated?  Use them to derive the function's formula).
\end{question} 

\end{document}