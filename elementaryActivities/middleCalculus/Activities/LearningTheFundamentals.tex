\documentclass{ximera}
\usepackage{gensymb}
\usepackage{tabularx}
\usepackage{mdframed}
\usepackage{pdfpages}
%\usepackage{chngcntr}

\let\problem\relax
\let\endproblem\relax

\newcommand{\property}[2]{#1#2}




\newtheoremstyle{SlantTheorem}{\topsep}{\fill}%%% space between body and thm
 {\slshape}                      %%% Thm body font
 {}                              %%% Indent amount (empty = no indent)
 {\bfseries\sffamily}            %%% Thm head font
 {}                              %%% Punctuation after thm head
 {3ex}                           %%% Space after thm head
 {\thmname{#1}\thmnumber{ #2}\thmnote{ \bfseries(#3)}} %%% Thm head spec
\theoremstyle{SlantTheorem}
\newtheorem{problem}{Problem}[]

%\counterwithin*{problem}{section}



%%%%%%%%%%%%%%%%%%%%%%%%%%%%Jenny's code%%%%%%%%%%%%%%%%%%%%

%%% Solution environment
%\newenvironment{solution}{
%\ifhandout\setbox0\vbox\bgroup\else
%\begin{trivlist}\item[\hskip \labelsep\small\itshape\bfseries Solution\hspace{2ex}]
%\par\noindent\upshape\small
%\fi}
%{\ifhandout\egroup\else
%\end{trivlist}
%\fi}
%
%
%%% instructorIntro environment
%\ifhandout
%\newenvironment{instructorIntro}[1][false]%
%{%
%\def\givenatend{\boolean{#1}}\ifthenelse{\boolean{#1}}{\begin{trivlist}\item}{\setbox0\vbox\bgroup}{}
%}
%{%
%\ifthenelse{\givenatend}{\end{trivlist}}{\egroup}{}
%}
%\else
%\newenvironment{instructorIntro}[1][false]%
%{%
%  \ifthenelse{\boolean{#1}}{\begin{trivlist}\item[\hskip \labelsep\bfseries Instructor Notes:\hspace{2ex}]}
%{\begin{trivlist}\item[\hskip \labelsep\bfseries Instructor Notes:\hspace{2ex}]}
%{}
%}
%% %% line at the bottom} 
%{\end{trivlist}\par\addvspace{.5ex}\nobreak\noindent\hung} 
%\fi
%
%


\let\instructorNotes\relax
\let\endinstructorNotes\relax
%%% instructorNotes environment
\ifhandout
\newenvironment{instructorNotes}[1][false]%
{%
\def\givenatend{\boolean{#1}}\ifthenelse{\boolean{#1}}{\begin{trivlist}\item}{\setbox0\vbox\bgroup}{}
}
{%
\ifthenelse{\givenatend}{\end{trivlist}}{\egroup}{}
}
\else
\newenvironment{instructorNotes}[1][false]%
{%
  \ifthenelse{\boolean{#1}}{\begin{trivlist}\item[\hskip \labelsep\bfseries {\Large Instructor Notes: \\} \hspace{\textwidth} ]}
{\begin{trivlist}\item[\hskip \labelsep\bfseries {\Large Instructor Notes: \\} \hspace{\textwidth} ]}
{}
}
{\end{trivlist}}
\fi


%% Suggested Timing
\newcommand{\timing}[1]{{\bf Suggested Timing: \hspace{2ex}} #1}




\hypersetup{
    colorlinks=true,       % false: boxed links; true: colored links
    linkcolor=blue,          % color of internal links (change box color with linkbordercolor)
    citecolor=green,        % color of links to bibliography
    filecolor=magenta,      % color of file links
    urlcolor=cyan           % color of external links
}

\author{Vic Ferdinand}
\title{Learning the Fundamentals (or The Viper)}
\outcome{Argue for Fundamental Theorem of Calculus using accumulation functions.}

\begin{document}
\begin{abstract}
\end{abstract}
\maketitle

\begin{instructorIntro}
Here, we use the ``window washer function'' to give a geometric and algebraic ``proof'' of what the students have already deduced from intuition (in Winter Storm Warning and Good Ol' Days):  That the exact net change in a function is the same as the net-change of the function's derivative's antiderivative (i.e., that the antiderivative truly undoes the derivative).  Because of time constraints, I've usually led this as a whole-class discussion.  Of course, this does little good unless one can find the antiderivative to plug into (which we'll continue to endeavor to find techniques for in Pinch Hitter)!!
\end{instructorIntro}

There is one small thing that we have not dealt with yet with respect to getting the exact value of the net change of a function (i.e.,  $\int_a^b f'(t)\, dt = f(b) - f(a)$).  Although it seems obvious from the definition of ``net change'' (from Integrated Language), as well as our estimate calculations with rectangles (rate times change in time- from Winter Storm Warning), we haven't {\em officially} shown that the value of the integral (and finding the area between $f'(t)$  and the $t$-axis) is indeed exactly the same as the {\em change of the anti-derivative of} $f'(t)$ (rather than just stating that it is like we did in Integrated Language) and the area under the graph of $f'(t)$.  That is, if we can find the function whose derivative is the rate function, finding the exact net change (and the exact area) is just two ``plug-n-chugs'' away!

    We will show this here with a geometrical argument:
 
 \begin{image}
     \begin{tikzpicture}[
      declare function = {f(\x) = -sin(deg(\x)) + 3;} ]
	\begin{axis}[
            domain=-.2:7, xmin =-.2,xmax=7,ymax=5,ymin=-.2,
            width=4in,
            height=2in,
            xtick={1,6}, 
            xticklabels={$a$,$b$},
            ytick style={draw=none},
            yticklabels={},
            axis lines=center, xlabel=$t$, ylabel=$f'(t)$,
            every axis y label/.style={at=(current axis.above origin),anchor=south},
            every axis x label/.style={at=(current axis.right of origin),anchor=west},
            axis on top, clip=false
          ]
          \addplot [draw=none, fill=black!30!white,domain=1:6, smooth] {f(x)} \closedcycle;
          \addplot [ultra thick, smooth] {f(x)};
          \node at (axis cs: 3.5,-0.5) {Figure $1$};
        \end{axis}
\end{tikzpicture}
 \end{image}
 
  \begin{image}
     \begin{tikzpicture}[
      declare function = {f(\x) = -sin(deg(\x)) + 3;} ]
	\begin{axis}[
            domain=-.2:7, xmin =-.2,xmax=7,ymax=5,ymin=-.2,
            width=4in,
            height=2in,
            xtick={1,4,5,6}, 
            xticklabels={$a$,$t$, $t+h$,$b$},
            ytick style={draw=none},
            yticklabels={},
            axis lines=center, xlabel=$t$, ylabel=$f'(t)$,
            every axis y label/.style={at=(current axis.above origin),anchor=south},
            every axis x label/.style={at=(current axis.right of origin),anchor=west},
            axis on top, clip=false
          ]
          \addplot [draw=none, fill=black!30!white,domain=4:5, smooth] {f(x)} \closedcycle;
          \addplot [ultra thick, smooth] {f(x)};
          \node at (axis cs: 3.5,-1) {Figure $2$};
          \draw (axis cs: 1,0)--(axis cs:1,2.1585);
          \draw (axis cs: 6,0)--(axis cs:6,3.2794);
        \end{axis}
        
\end{tikzpicture}
 \end{image}

We have a graph of $f'(t)$ here.  We want to find the area between the graph and the $t$-axis on the interval $(a, b)$ (Figure 1).  To do so, we define the ``window-wiper'' area function  $A(t)$. $A(t)$  is the area under the curve from $0$ to $t$.  Thus, what does $A(a)$ represent?  What does $A(b)$ represent?  Write the area that we want in terms of  $A(t)$ (Hint:  A difference of values of $A(t)$ at two times).  Thus, our problem comes down to showing that $A(t)$ is the same as  $f(t)$ (or, equivalently, that $A'(t)$ is the same as $f'(t)$).

Consider two times within the interval $(a, b)$ with the times being very close to each other.  Call them $t$ and  $t+h$, where $h$ is very small.  We'd like the area underneath the curve of $f'(t)$ for this interval of times only (Figure 2). What would this area be in terms of  $A(t)$? (Hint: Another difference).  On the other hand, since $h$ is very small, this same small, skinny region looks very much like what shape?  What is the area of that shape (in terms of $h$ and  $f'(t)$)?

Thus, we have two expressions representing the very same area of the very same (skinny) region.  Thus, set them equal to each other.  Now divide both sides by $h$.  For very small $h$, what does the side involving $A(t)$ represent?  What must $A'(t)$ indeed be the same as?  Thus, what must $A(t)$ be the same as?  Thus, what is $A(b) - A(a)$  the same as (in terms of $f$)?

Thus, we now have hope of finding the exact net change in an amount function- if only we can find the antiderivative of the rate function!


\end{document}