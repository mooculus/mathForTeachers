\documentclass{ximera}
\usepackage{gensymb}
\usepackage{tabularx}
\usepackage{mdframed}
\usepackage{pdfpages}
%\usepackage{chngcntr}

\let\problem\relax
\let\endproblem\relax

\newcommand{\property}[2]{#1#2}




\newtheoremstyle{SlantTheorem}{\topsep}{\fill}%%% space between body and thm
 {\slshape}                      %%% Thm body font
 {}                              %%% Indent amount (empty = no indent)
 {\bfseries\sffamily}            %%% Thm head font
 {}                              %%% Punctuation after thm head
 {3ex}                           %%% Space after thm head
 {\thmname{#1}\thmnumber{ #2}\thmnote{ \bfseries(#3)}} %%% Thm head spec
\theoremstyle{SlantTheorem}
\newtheorem{problem}{Problem}[]

%\counterwithin*{problem}{section}



%%%%%%%%%%%%%%%%%%%%%%%%%%%%Jenny's code%%%%%%%%%%%%%%%%%%%%

%%% Solution environment
%\newenvironment{solution}{
%\ifhandout\setbox0\vbox\bgroup\else
%\begin{trivlist}\item[\hskip \labelsep\small\itshape\bfseries Solution\hspace{2ex}]
%\par\noindent\upshape\small
%\fi}
%{\ifhandout\egroup\else
%\end{trivlist}
%\fi}
%
%
%%% instructorIntro environment
%\ifhandout
%\newenvironment{instructorIntro}[1][false]%
%{%
%\def\givenatend{\boolean{#1}}\ifthenelse{\boolean{#1}}{\begin{trivlist}\item}{\setbox0\vbox\bgroup}{}
%}
%{%
%\ifthenelse{\givenatend}{\end{trivlist}}{\egroup}{}
%}
%\else
%\newenvironment{instructorIntro}[1][false]%
%{%
%  \ifthenelse{\boolean{#1}}{\begin{trivlist}\item[\hskip \labelsep\bfseries Instructor Notes:\hspace{2ex}]}
%{\begin{trivlist}\item[\hskip \labelsep\bfseries Instructor Notes:\hspace{2ex}]}
%{}
%}
%% %% line at the bottom} 
%{\end{trivlist}\par\addvspace{.5ex}\nobreak\noindent\hung} 
%\fi
%
%


\let\instructorNotes\relax
\let\endinstructorNotes\relax
%%% instructorNotes environment
\ifhandout
\newenvironment{instructorNotes}[1][false]%
{%
\def\givenatend{\boolean{#1}}\ifthenelse{\boolean{#1}}{\begin{trivlist}\item}{\setbox0\vbox\bgroup}{}
}
{%
\ifthenelse{\givenatend}{\end{trivlist}}{\egroup}{}
}
\else
\newenvironment{instructorNotes}[1][false]%
{%
  \ifthenelse{\boolean{#1}}{\begin{trivlist}\item[\hskip \labelsep\bfseries {\Large Instructor Notes: \\} \hspace{\textwidth} ]}
{\begin{trivlist}\item[\hskip \labelsep\bfseries {\Large Instructor Notes: \\} \hspace{\textwidth} ]}
{}
}
{\end{trivlist}}
\fi


%% Suggested Timing
\newcommand{\timing}[1]{{\bf Suggested Timing: \hspace{2ex}} #1}




\hypersetup{
    colorlinks=true,       % false: boxed links; true: colored links
    linkcolor=blue,          % color of internal links (change box color with linkbordercolor)
    citecolor=green,        % color of links to bibliography
    filecolor=magenta,      % color of file links
    urlcolor=cyan           % color of external links
}

\author{Vic Ferdinand}
\title{Summing It All Up (Geometric Applications of Integrals}

\begin{document}
\begin{abstract}
\end{abstract}
\maketitle

\begin{instructorIntro}
We wrap up the course by exploring (mostly geometric) applications of integration- extending at least two ideas that are dealt with in early algebra or geometry.  

The rest of the activities use the Riemann sum that was developed intuitively in Winter Storm Warning and consider that the sum (as the size of the intervals ($\Delta x$) approaches zero) approaches the definite integral.  The key is recognizing what the ``$f'(x)$'' now is (instead of a rate, but rather some other measurement dependent upon the function $f(x)$).  This also could be done with the area between curves, but they are better able to intuitively see that without the Riemann sum. 

Because of time, I usually derive all of the formulas in a whole-class setting and let them work on the problems- often assigning certain problems to certain groups. The hardest application is the volume by slicing, which uses the volume of a prism as the $A(x) \Delta x$.  They usually will intuitively understand the ``loaf of sliced bread'' argument (bringing a loaf to class helps!), but have difficulty in actually finding the $A(x)$ in the exercises, particularly if it involves a triangle (Often, we end up doing these as a whole class in the end).

\end{instructorIntro}


Back in Winter Storm Warning, we estimated that the total accumulation (net change) of snow over the $4$ hours was the sum of a bunch of terms that all looked like  $f'(t) \Delta t$, as these were the amount of snow that fell over a short time period if we assumed a constant rate (and computed it as $(\text{rate}) \times (\text{change of time})$).  Adding these up gave us an approximation of the {\em overall} change over the $4$ hours.  The notation for this summation is  $\sum_i f'(t_i)\Delta t$, where the ``$i$'s'' just mean that we take a sequence of times over the same size interval (i.e., the  $t_i$'s were the left-hand endpoints of each interval. We will be not concerning ourselves with this ``$i$'' notation anymore here- just remember that what we are plugging a sequence of $t$'s into  while the change in $t$ (length of each interval) remains the same).  

We also saw that if the length of the intervals ($\Delta t$) gets very small, our approximation gets closer and closer to the exact value (why?), which we wrote as  $\int_a^b f'(t) \, dt$, where $[a, b]$ is the interval of time we were interested in.  Thus, $\sum f'(t) \Delta t$  ``turns into'' $\int_a^b f'(t)\, dt$  as $\Delta t$ gets very small.  And they were also approximating (and eventually exactamating) the {\em area} between the graph of $f'(t)$ and the $t$-axis.  

Well, it turns out that if we replace $f'(t)$ by {\em something else} (with the goal of doing a different sort of measurement) and the measurement involves adding up the products of (``that something else'') $\times \Delta t$  and the estimation becomes more and more exact as $\Delta t$ gets closer to zero, then it turns out that the sum  $\sum (\text{something else})$ ``turns into''  $\int_a^b (\text{something else})\, dt$.  These types of sums are called ``Riemann Sums'', named after a famous $19$th century mathematician named Bernhard Riemann.  The biggest problem in most of these applications is defining the ``something''.

The following list of activities contain just a couple of a few applications of Riemann sums in which we use integrals to do more than just measure area: See the activities {\em Stuck in the Middle} , {\em It Slices and It Dices} , {\em SoSo} , and {\em It's a Long Way to Tipperary!} .

\end{document}