\documentclass{ximera}

\graphicspath{
  {./}
  {graphics/}
  {../graphics/}
}

\usepackage{chngcntr}

\let\question\relax
\let\endquestion\relax




\newtheoremstyle{SlantTheorem}{\topsep}{\fill}%%% space between body and thm
%\newtheoremstyle{SlantTheorem}{\topsep}{\topsep}%%% space between body and thm
 {\slshape}                      %%% Thm body font
 {}                              %%% Indent amount (empty = no indent)
 {\bfseries\sffamily}            %%% Thm head font
 {}                              %%% Punctuation after thm head
 {3ex}                           %%% Space after thm head
 {\thmname{#1}\thmnumber{ #2}\thmnote{ \bfseries(#3)}}%%% Thm head spec
\theoremstyle{SlantTheorem}
\newtheorem{question}{Question}
\counterwithin*{question}{section}



\let\instructorNotes\relax
\let\endinstructorNotes\relax
%%% instructorNotes environment
\ifhandout
\newenvironment{instructorNotes}[1][false]%
{%
\def\givenatend{\boolean{#1}}\ifthenelse{\boolean{#1}}{\begin{trivlist}\item}{\setbox0\vbox\bgroup}{}
}
{%
\ifthenelse{\givenatend}{\end{trivlist}}{\egroup}{}
}
\else
\newenvironment{instructorNotes}[1][false]%
{%
  \ifthenelse{\boolean{#1}}{\begin{trivlist}\item[\hskip \labelsep\bfseries {\Large Instructor Notes: \\} \hspace{\textwidth} ]}
{\begin{trivlist}\item[\hskip \labelsep\bfseries {\Large Instructor Notes: \\} \hspace{\textwidth} ]}
{}
}
{\end{trivlist}}
\fi


%% Suggested Timing
\newcommand{\timing}[1]{{\bf Suggested Timing: \hspace{2ex}} #1}

\author{Vic Ferdinand}
\title{Integrated Language}
\outcome{Notation for definite integrals.}
\outcome{Language of definite integrals as net change in amount function.}

\begin{document}
\begin{abstract}
\end{abstract}
\maketitle


\begin{instructorIntro}
We now embark on presenting integral calculus in much the same way we did with differential calculus:  Introducing the language of the definite integral (the net change in the amount of what the $f'$ is measuring ``per time unit''), approximating values of that net change, and, when it can be done algebraically, finding exact values of that net change via the Fundamental Theorem of Calculus.

In Integrated Language, introduce the notation and them let them do the problems in groups.  Although ``distance vs. change in position'' doesn't make any difference now (since Judy is traveling in the same direction all the time), you might illustrate that difference now.  Part (c) in Judy's problem makes this explicit as well.

Some students may write the integral as a difference.  For example, they might write $f(2008)-f(2005)=1,000,000$ instead of the integral.  You might ask if which notation is correct (They both are).  The point is brought up explicitly in part (d) of the airplane problem



\end{instructorIntro}

For each of the first three problems, write an equivalent statement using function and/or integral notation.


\begin{question} 
If $f$ is the population of Hicktown, 
\begin{enumerate}
\item The population of Hicktown was $5.4$ million in 2010.
\item The population of Hicktown was increasing at a rate of $110000$ per year in 2010.
\item	The population of Hicktown grew by $1,000,000$ people between 2010 and 2018.
\end{enumerate}
\end{question}


\begin{question} 
Joan decides to take a run outside.  Let $p(t)$ be the position Joan is from home (east or west) after $t$ minutes.
\begin{enumerate} 
\item 	At $t = 10$, Joan is running at $100$ meters per minute.
\item 	She comes to a hill.  At $t=12$, her velocity is dropping at a rate of $10$ meters per minute each minute.
\item 	At $12$ minutes, Joan was $185$ meters east of where she was at $10$ minutes. (Side Note:  This may or may not mean the same as Joan traveled $185$ meters over that period of time- why?).
\end{enumerate}
\end{question} 


\begin{question} 
If $T(t) =$ the temperature at OSU at time $t$ hours after noon, 
\begin{enumerate} 
\item 	The temperature at $11$AM is $67$ degrees.
\item 	At $2$PM the temperature was $75$ degrees and rising.
\item 	Between $5$PM and $6$PM, the temperature rose less than it did between $2$PM and $5$PM.
\end{enumerate}
\end{question} 



\begin{question} 
Let $H$ stand for the height of a balloon (in feet above the ground) as a function of time (in minutes). Translate the following into English: 
\begin{enumerate} 
\item $H(0) = 1000$
\item $H'(0) = 0$
\item $H'(5) = -52$
\item $H(5) - H(0) = -201$ (Is there another way we can write this now?)
\end{enumerate}                  
\end{question} 

\begin{question} 
Let $R(t)$ be Rachel's height at $t$ years of age and $J(t)$ be Joe's height at $t$ years of age.  Translate the following into English.
\begin{enumerate}
    \item $R(14) > J(14)$
    \item $R'(14) < J'(14)$
    \item $\int_{12}^{15} R'(t) \, dt > \int_{12}^{15} J'(t) \, dt$
    \item $R(3) + \int_3^{16} R'(t)\, dt$ (What does this describe and is there another way to write this?)
\end{enumerate}
\end{question}

\begin{question} 
Debbie the Dieter goes on every fad diet that she sees advertised on TV, thus wreaking havoc on her system.  She is constantly losing and gaining weight.  While her TV was being repaired one day, Deb, using her many charts of her weight through the years, figured out that her weight (in pounds) at time $t$ years (her age) is given by the function  $W(t)$.  What would $\int_{30.5}^{37} W'(t)\, dt$ mean in terms of her age and weight?

\begin{instructorNotes}
The goal here is to illustrate the idea of ``position'' again, although it's in the context of weight:  That the integral value does not reflect the journey to how it got to that number whatsoever:  It only cares about the beginning and ending amount.  
\end{instructorNotes}


\end{question}



\end{document}