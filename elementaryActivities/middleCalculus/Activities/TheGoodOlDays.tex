\documentclass{ximera}
\usepackage{gensymb}
\usepackage{tabularx}
\usepackage{mdframed}
\usepackage{pdfpages}
%\usepackage{chngcntr}

\let\problem\relax
\let\endproblem\relax

\newcommand{\property}[2]{#1#2}




\newtheoremstyle{SlantTheorem}{\topsep}{\fill}%%% space between body and thm
 {\slshape}                      %%% Thm body font
 {}                              %%% Indent amount (empty = no indent)
 {\bfseries\sffamily}            %%% Thm head font
 {}                              %%% Punctuation after thm head
 {3ex}                           %%% Space after thm head
 {\thmname{#1}\thmnumber{ #2}\thmnote{ \bfseries(#3)}} %%% Thm head spec
\theoremstyle{SlantTheorem}
\newtheorem{problem}{Problem}[]

%\counterwithin*{problem}{section}



%%%%%%%%%%%%%%%%%%%%%%%%%%%%Jenny's code%%%%%%%%%%%%%%%%%%%%

%%% Solution environment
%\newenvironment{solution}{
%\ifhandout\setbox0\vbox\bgroup\else
%\begin{trivlist}\item[\hskip \labelsep\small\itshape\bfseries Solution\hspace{2ex}]
%\par\noindent\upshape\small
%\fi}
%{\ifhandout\egroup\else
%\end{trivlist}
%\fi}
%
%
%%% instructorIntro environment
%\ifhandout
%\newenvironment{instructorIntro}[1][false]%
%{%
%\def\givenatend{\boolean{#1}}\ifthenelse{\boolean{#1}}{\begin{trivlist}\item}{\setbox0\vbox\bgroup}{}
%}
%{%
%\ifthenelse{\givenatend}{\end{trivlist}}{\egroup}{}
%}
%\else
%\newenvironment{instructorIntro}[1][false]%
%{%
%  \ifthenelse{\boolean{#1}}{\begin{trivlist}\item[\hskip \labelsep\bfseries Instructor Notes:\hspace{2ex}]}
%{\begin{trivlist}\item[\hskip \labelsep\bfseries Instructor Notes:\hspace{2ex}]}
%{}
%}
%% %% line at the bottom} 
%{\end{trivlist}\par\addvspace{.5ex}\nobreak\noindent\hung} 
%\fi
%
%


\let\instructorNotes\relax
\let\endinstructorNotes\relax
%%% instructorNotes environment
\ifhandout
\newenvironment{instructorNotes}[1][false]%
{%
\def\givenatend{\boolean{#1}}\ifthenelse{\boolean{#1}}{\begin{trivlist}\item}{\setbox0\vbox\bgroup}{}
}
{%
\ifthenelse{\givenatend}{\end{trivlist}}{\egroup}{}
}
\else
\newenvironment{instructorNotes}[1][false]%
{%
  \ifthenelse{\boolean{#1}}{\begin{trivlist}\item[\hskip \labelsep\bfseries {\Large Instructor Notes: \\} \hspace{\textwidth} ]}
{\begin{trivlist}\item[\hskip \labelsep\bfseries {\Large Instructor Notes: \\} \hspace{\textwidth} ]}
{}
}
{\end{trivlist}}
\fi


%% Suggested Timing
\newcommand{\timing}[1]{{\bf Suggested Timing: \hspace{2ex}} #1}




\hypersetup{
    colorlinks=true,       % false: boxed links; true: colored links
    linkcolor=blue,          % color of internal links (change box color with linkbordercolor)
    citecolor=green,        % color of links to bibliography
    filecolor=magenta,      % color of file links
    urlcolor=cyan           % color of external links
}

\author{Vic Ferdinand}
\title{The Good Ol' Days (Or ``History Repeats Itself'')}
\outcome{Find the amount function using the derivative and a point on the function.}
\outcome{Use point-slope and slope-intercept form of a line.}

\begin{document}
\begin{abstract}
\end{abstract}
\maketitle

\begin{instructorIntro}
Here, using two methods, we deal with finding the exact antiderivative of a given function if we have one point of the function.  We do so by connecting to the equivalent notion in early algebra (which the students may teach someday):  That one needs both the slope and only one point of a function to precisely determine that function (whether that function has a constant slope or not).  If one {\em just} has the slope (or slope function), one is left with a set of parallel curves (If we know a point, because those curves do not share any point, the precise curve is uniquely determined).  If we had {\em just} a point, there are an infinite number of lines that contain that point, all with a different slope or direction.  Thus, if we're also given the slope, we determine which of those lines we have.

Students usually have little trouble in seeing and reporting that we need a point in addition to the slope (now the derivative function) to determine the amount function and also connect it well with the Algebra I context of the ``slope-intercept'' form of the line/curve (we now have parallel curves instead of lines, but the same need exists).  They also usually don't have trouble seeing the need to (and knowing how to) find ``$C$'' (making the analogy to finding the $y$-intercept- the lunch money- in the slope-intercept form of the line, although $C$ is not always the $y$-intercept now).

They may not do so well in connecting the idea to the algebraic ``point-slope'' form (from their first content course), (That writing $\int_a^b f'(t)\, dt$ is the same as $f(b) = f(a) + \int_a^b f'(t)\, dt$ and connecting it to point-slope form of the line:  ``Start + Change'':  The initial value plus the sum of a bunch of infinitesimal ``$m$ times change of $x$'').  Making a connection to Winter Storm Warning (finding the total amount of snow on the ground) often helps.



\end{instructorIntro}

Our goal in this activity is {\em not} to find the net change in an ``amount'' or ``position'' function $f(x)$, but rather to find $f(x)$ itself so that we can find specific amounts (or positions) at any instant of time.  
     
But first, we go down memory lane….

\begin{exercise}
\begin{enumerate} 
\item Way back when, in that galaxy (or middle/high school) far far away and long long ago, you may have been given this problem in algebra class:  Find the equation of the line with slope $16$.  Then use the equation to find the point on that line when $x = 7$.  What were the answers to those questions?

\item Perhaps at another time, you were asked to find the equation of the line that contains the point $(3, 83)$.  Then use the equation to find the point on that line for when $x = 7$. What were the answers to those questions?
\item   What is wrong with both of the above problems?  Why do they not geometrically make sense?  In each case, what is the least amount of additional information that you would need to find the requested answers?  

     Let us return to our days gone by….

\item You may recall Free Lance Freddy Kroger, who stalks\dots errr\dots stocks groceries.  He brought some lunch money to work, which he started at noon today.  Let’s assume he has $\$83$ at $3$ PM and is earning money at $\$16$ per hour.  How much money will he have at $7$PM?  Solve this in two ways:  One way without ever finding his lunch money and one way by finding his lunch money first.  When you are done with that, predict how much money he'll have at $x$ hours after noon.  Call this function $f(x)$, which is the amount of money Freddy has at precisely $x$ PM!

\item OK youse guys, that's enough of this fond reminiscing!  It's time to snap out of it and return to the cold hard reality of 2018!  Here, in the future, our friend Freddy is still kicking around to annoy calculus students.  He still brings lunch money to work, which he started at noon today. Assume again he has $\$83$ at $3$PM.  However, now he earns money at a variable rate that is dependent on the time after noon, which we'll call $t$.  Freddy is earning money at  $4t^3+e^t$ dollars per hour.  We again want to know how much money he'll have at $7$PM.  Solve this in two ways: One way without ever finding his lunch money and one way by finding his lunch money first.  When you are done with that, predict how much money he'll have at $x$ hours after noon.  Call this function $f(x)$, which is the amount of money Freddy has at precisely $x$ PM!
\end{enumerate}
\end{exercise}
\begin{exercise} 
Let's do the same thing (i.e., find $f(x)$  given  $f'(x)$), only this time we'll do it graphically.  In each of the following, Given the graph of  $f'(t)$, sketch $3$ possible graphs for  $f(t)$.

\begin{enumerate}
    \item \leavevmode\vadjust{\vspace{-\baselineskip}}\newline
    \begin{center}
        \begin{tikzpicture}
        \draw[->] (0,-1.25) -- (0,3.25);
        \draw[->] (-0.25,0) -- (8.25,0);
        \foreach \x in {0,...,9} 
            \draw (\x,0) (\x cm,1pt) -- (\x cm,-1pt);
        \node at (1,-0.3) {$1$};
        \node at (8, -0.3) {$t$};
        \foreach \y in {-1,...,3}
            \draw (0,\y) (1pt,\y cm)--(-1pt,\y cm);
        \node at (-0.3, 1) {$1$};
        \node at (-0.3, 3) {$f'$};
        \draw plot [smooth] coordinates {(0,1) (1,0.7) (2,0) (3,-0.7) (4,-1) (5,-0.7) (6,0) (7,0.7) (8,1)};
        \end{tikzpicture}
    \end{center}
    \item \leavevmode\vadjust{\vspace{-\baselineskip}}\newline
    \begin{center}
        \begin{tikzpicture}
        \draw[->] (0,-1.25) -- (0,3.25);
        \draw[->] (-0.25,0) -- (8.25,0);
        \foreach \x in {0,...,9} 
            \draw (\x,0) (\x cm,1pt) -- (\x cm,-1pt);
        \node at (1,-0.3) {$1$};
        \node at (8, -0.3) {$t$};
        \foreach \y in {-1,...,3}
            \draw (0,\y) (1pt,\y cm)--(-1pt,\y cm);
        \node at (-0.3, 1) {$1$};
        \node at (-0.3, 3) {$f'$};
        \draw plot [smooth] coordinates {(0,0.5) (1,0.55) (2,0.6) (3,1) (4,1.8) (5,1) (6,0.6) (7,0.55) (8,0.5)};
        \end{tikzpicture}
    \end{center}
\end{enumerate}

\begin{instructorNotes}
In this problem, they're given the same problem graphically (given the rate function (with no value of the function), find the amount function).  This is a tough transition from the opposite problem earlier in the quarter- recognizing that the $y$-values represent slopes of $f$ and their sign represents the concavities of $f$.  I often give a hint that they could make a sign chart for the first and second derivative (but not of the function itself:  Knowing how fast one is going does not say at all where one is)).
\end{instructorNotes}

\end{exercise}

\end{document}