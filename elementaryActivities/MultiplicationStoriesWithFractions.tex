\documentclass{ximera}

\usepackage{gensymb}
\usepackage{tabularx}
\usepackage{mdframed}
\usepackage{pdfpages}
%\usepackage{chngcntr}

\let\problem\relax
\let\endproblem\relax

\newcommand{\property}[2]{#1#2}




\newtheoremstyle{SlantTheorem}{\topsep}{\fill}%%% space between body and thm
 {\slshape}                      %%% Thm body font
 {}                              %%% Indent amount (empty = no indent)
 {\bfseries\sffamily}            %%% Thm head font
 {}                              %%% Punctuation after thm head
 {3ex}                           %%% Space after thm head
 {\thmname{#1}\thmnumber{ #2}\thmnote{ \bfseries(#3)}} %%% Thm head spec
\theoremstyle{SlantTheorem}
\newtheorem{problem}{Problem}[]

%\counterwithin*{problem}{section}



%%%%%%%%%%%%%%%%%%%%%%%%%%%%Jenny's code%%%%%%%%%%%%%%%%%%%%

%%% Solution environment
%\newenvironment{solution}{
%\ifhandout\setbox0\vbox\bgroup\else
%\begin{trivlist}\item[\hskip \labelsep\small\itshape\bfseries Solution\hspace{2ex}]
%\par\noindent\upshape\small
%\fi}
%{\ifhandout\egroup\else
%\end{trivlist}
%\fi}
%
%
%%% instructorIntro environment
%\ifhandout
%\newenvironment{instructorIntro}[1][false]%
%{%
%\def\givenatend{\boolean{#1}}\ifthenelse{\boolean{#1}}{\begin{trivlist}\item}{\setbox0\vbox\bgroup}{}
%}
%{%
%\ifthenelse{\givenatend}{\end{trivlist}}{\egroup}{}
%}
%\else
%\newenvironment{instructorIntro}[1][false]%
%{%
%  \ifthenelse{\boolean{#1}}{\begin{trivlist}\item[\hskip \labelsep\bfseries Instructor Notes:\hspace{2ex}]}
%{\begin{trivlist}\item[\hskip \labelsep\bfseries Instructor Notes:\hspace{2ex}]}
%{}
%}
%% %% line at the bottom} 
%{\end{trivlist}\par\addvspace{.5ex}\nobreak\noindent\hung} 
%\fi
%
%


\let\instructorNotes\relax
\let\endinstructorNotes\relax
%%% instructorNotes environment
\ifhandout
\newenvironment{instructorNotes}[1][false]%
{%
\def\givenatend{\boolean{#1}}\ifthenelse{\boolean{#1}}{\begin{trivlist}\item}{\setbox0\vbox\bgroup}{}
}
{%
\ifthenelse{\givenatend}{\end{trivlist}}{\egroup}{}
}
\else
\newenvironment{instructorNotes}[1][false]%
{%
  \ifthenelse{\boolean{#1}}{\begin{trivlist}\item[\hskip \labelsep\bfseries {\Large Instructor Notes: \\} \hspace{\textwidth} ]}
{\begin{trivlist}\item[\hskip \labelsep\bfseries {\Large Instructor Notes: \\} \hspace{\textwidth} ]}
{}
}
{\end{trivlist}}
\fi


%% Suggested Timing
\newcommand{\timing}[1]{{\bf Suggested Timing: \hspace{2ex}} #1}




\hypersetup{
    colorlinks=true,       % false: boxed links; true: colored links
    linkcolor=blue,          % color of internal links (change box color with linkbordercolor)
    citecolor=green,        % color of links to bibliography
    filecolor=magenta,      % color of file links
    urlcolor=cyan           % color of external links
}


\title{Multiplication Stories with Fractions}
\author{Vic Ferdinand, Betsy McNeal, Jenny Sheldon}

\begin{document}
\begin{abstract} \end{abstract}
\maketitle




\begin{problem}
 Solve each of the following stories using a picture.  Then, compare your solutions.  Be sure to address:   
\begin{itemize}
    \item How are the pictures alike?  How are they different?
    \item Can these situations be viewed as multiplication?  Do they fit with our usual meaning of multiplication?
\end{itemize}

\begin{enumerate}%[label=(\alph*)]
    \item My mom's oatmeal cookie recipe calls for 3 cups of oatmeal.  
I want to make 2 batches.
How much oatmeal will I need to make this?
    \item My mom's oatmeal cookie recipe calls for 3 cups of oatmeal.  
I want to make $\frac{2}{5}$ of a batch.
How much oatmeal will I need to make this?
    \item My mom's new oatmeal cookie recipe calls for $\frac{4}{7}$ cup of oatmeal.  
I want to make 2 batches.
How much oatmeal will I need to make this?
    \item My mom's newest oatmeal cookie recipe calls for  $\frac{4}{7}$ cup of oatmeal.  
I want to make $\frac{2}{5}$ of a batch.
How much oatmeal will I need to make this?
\end{enumerate}

\end{problem}

\newpage




\begin{problem}
 Write an expression involving multiplication which represents the story below, then solve the problem using a picture or pictures.  Be ready to explain how you drew your picture and what it means.  How does the picture tell you the answer?

\emph{I have a container that holds $\frac{5}{8}$ of a gallon when it is full.  Today, my container is $\frac{2}{3}$ full of milk. How much milk is in the container?}
\end{problem}
\vfill

\begin{problem}
 For each of the stories below, first write an expression involving multiplication which represents each of these stories.  Then, solve either version of this problem using pictures.  Be ready to explain how you drew your picture and what it means.  How does the picture tell you the answer? 

\begin{enumerate}
    \item I walked for  $\frac{3}{4}$ of an hour at a speed of  $\frac{7}{5}$ miles per hour.  How far have I walked?
    \item Rewrite the story in (a) so that it is the same multiplication problem using \emph{containers} and \emph{gallons}.
\end{enumerate}
\end{problem}
\vfill

\begin{problem}
 Write an expression involving multiplication which represents the story below, then solve the problem using a picture or pictures.  Be ready to explain how you drew your picture and what it means.  How does the picture tell you the answer?


\emph{A Cub Scout troop decided to make their own chocolate-covered popcorn for the annual sale, but they ate $2 \frac{1}{3}$ batches of the popcorn before selling any! If each batch of popcorn used $\frac{6}{5}$ of a pound of chocolate, how much chocolate did the Scouts eat?}

\end{problem}
\vfill
 
\newpage
\begin{instructorNotes}
The problems in this activity are meant to first introduce students to the concept of multiplying fractions, and then give students enough practice with drawing pictures and solving such problems that we can extract the usual multiplication rule for fractions from these pictures.  

Have students begin by drawing pictures, and continually emphasize the definition or meaning of multiplication.  Have students demonstrate and talk about their solutions throughout.  

Once you have solved all of the problems, give the class time to ``step back'' from these solutions and observe any patterns they see.  This should hopefully bring up the usual procedure in the pictures, as well as other observations.

{\bf Suggested Timing:} This activity will take two class periods to fully complete.  The first day should be spent on solving problems, and the second day on clearing away misconceptions, solving more problems, and making observations about patterns.
\end{instructorNotes}



\end{document}