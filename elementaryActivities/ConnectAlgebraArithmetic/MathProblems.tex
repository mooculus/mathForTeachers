%\documentclass{ximera}
\documentclass[nooutcomes,noauthor]{ximera}
\usepackage{gensymb}
\usepackage{tabularx}
\usepackage{mdframed}
\usepackage{pdfpages}
%\usepackage{chngcntr}

\let\problem\relax
\let\endproblem\relax

\newcommand{\property}[2]{#1#2}




\newtheoremstyle{SlantTheorem}{\topsep}{\fill}%%% space between body and thm
 {\slshape}                      %%% Thm body font
 {}                              %%% Indent amount (empty = no indent)
 {\bfseries\sffamily}            %%% Thm head font
 {}                              %%% Punctuation after thm head
 {3ex}                           %%% Space after thm head
 {\thmname{#1}\thmnumber{ #2}\thmnote{ \bfseries(#3)}} %%% Thm head spec
\theoremstyle{SlantTheorem}
\newtheorem{problem}{Problem}[]

%\counterwithin*{problem}{section}



%%%%%%%%%%%%%%%%%%%%%%%%%%%%Jenny's code%%%%%%%%%%%%%%%%%%%%

%%% Solution environment
%\newenvironment{solution}{
%\ifhandout\setbox0\vbox\bgroup\else
%\begin{trivlist}\item[\hskip \labelsep\small\itshape\bfseries Solution\hspace{2ex}]
%\par\noindent\upshape\small
%\fi}
%{\ifhandout\egroup\else
%\end{trivlist}
%\fi}
%
%
%%% instructorIntro environment
%\ifhandout
%\newenvironment{instructorIntro}[1][false]%
%{%
%\def\givenatend{\boolean{#1}}\ifthenelse{\boolean{#1}}{\begin{trivlist}\item}{\setbox0\vbox\bgroup}{}
%}
%{%
%\ifthenelse{\givenatend}{\end{trivlist}}{\egroup}{}
%}
%\else
%\newenvironment{instructorIntro}[1][false]%
%{%
%  \ifthenelse{\boolean{#1}}{\begin{trivlist}\item[\hskip \labelsep\bfseries Instructor Notes:\hspace{2ex}]}
%{\begin{trivlist}\item[\hskip \labelsep\bfseries Instructor Notes:\hspace{2ex}]}
%{}
%}
%% %% line at the bottom} 
%{\end{trivlist}\par\addvspace{.5ex}\nobreak\noindent\hung} 
%\fi
%
%


\let\instructorNotes\relax
\let\endinstructorNotes\relax
%%% instructorNotes environment
\ifhandout
\newenvironment{instructorNotes}[1][false]%
{%
\def\givenatend{\boolean{#1}}\ifthenelse{\boolean{#1}}{\begin{trivlist}\item}{\setbox0\vbox\bgroup}{}
}
{%
\ifthenelse{\givenatend}{\end{trivlist}}{\egroup}{}
}
\else
\newenvironment{instructorNotes}[1][false]%
{%
  \ifthenelse{\boolean{#1}}{\begin{trivlist}\item[\hskip \labelsep\bfseries {\Large Instructor Notes: \\} \hspace{\textwidth} ]}
{\begin{trivlist}\item[\hskip \labelsep\bfseries {\Large Instructor Notes: \\} \hspace{\textwidth} ]}
{}
}
{\end{trivlist}}
\fi


%% Suggested Timing
\newcommand{\timing}[1]{{\bf Suggested Timing: \hspace{2ex}} #1}




\hypersetup{
    colorlinks=true,       % false: boxed links; true: colored links
    linkcolor=blue,          % color of internal links (change box color with linkbordercolor)
    citecolor=green,        % color of links to bibliography
    filecolor=magenta,      % color of file links
    urlcolor=cyan           % color of external links
}
\title{Math problems}

\begin{document}
\begin{abstract}
\end{abstract}

\maketitle

\begin{problem}
Emmet is working on their math homework. The first section they needs to work on has four times as many practice problems as the second section. There are $80$ practice problems in total. How many practice problems are in the second section?

\begin{enumerate}
	\item Explain how to use the following picture of a strip diagram to solve this problem.
	\begin{image} \begin{tikzpicture}
	\draw[thick] (0,0) rectangle (2,1);
	\draw[thick] (0, 1.5) rectangle (8, 2.5);
	\foreach \x in {2, 4, 6} \draw[thick, dashed] (\x, 1.5)--(\x, 2.5);
	\draw[<->] (8.5, 0)--(9,0)--(9, 2.5)--(8.5, 2.5);
	\node[right] at (9, 1.25) {$80$ problems};
	\end{tikzpicture}\end{image}
	\item Explain how to use algebra to solve this problem. If you use a variable, be sure to define it in as much detail as you can.
	\item Explain how parts (a) and (b) are connected. Point to as many specifics in your work as you can.
\end{enumerate}

\end{problem}



\begin{problem}
Fan, Georgina, and Thurston are working on their math homework. Fan and Georgina were given problem sheets with an equal number of problems on them. Fan has two of these sheets and Georgina has three. Thurston has $11$ problems to work on. If all three of them have a total of $51$ problems to work on, how many problems are on each problem sheet?

\begin{enumerate}
	\item Use a strip diagram (or several strip diagrams) like the one in Problem 1 to solve this problem. Explain how you drew your picture and how it helped you to solve the problem.
	\item Explain how to use algebra to solve this problem. If you use a variable, be sure to define it in as much detail as you can.
	\item Explain how parts (a) and (b) are connected. Point to as many specifics in your work as you can.
\end{enumerate}
\end{problem}


\newpage

\begin{problem}
Kateryna is working on her math homework. She has already completed $\frac{3}{7}$ of the practice problems in the section, and there are $36$ problems that she has not completed. How many problems does Kateryna need to complete?


\begin{enumerate}
	\item Use a strip diagram (or several strip diagrams) like the one in Problem 1 to solve this problem. Explain how you drew your picture and how it helped you to solve the problem.
	\item Explain how to use algebra to solve this problem. If you use a variable, be sure to define it in as much detail as you can.
	\item Explain how parts (a) and (b) are connected. Point to as many specifics in your work as you can.
\end{enumerate}
\end{problem}





\begin{problem}
Barish is working on his math homework. He has already completed $\frac{5}{8}$ of his assigned problems, and he plans to complete $\frac{1}{2}$ of the remaining problems tomorrow. This will leave him $9$ more problems for the day after tomorrow. If the entire assignment contains $48$ problems, how many problems has Barish already completed?


\begin{enumerate}
	\item Use a strip diagram (or several strip diagrams) like the one in Problem 1 to solve this problem. Explain how you drew your picture and how it helped you to solve the problem.
	\item Explain how to use algebra to solve this problem. If you use a variable, be sure to define it in as much detail as you can.
	\item Explain how parts (a) and (b) are connected. Point to as many specifics in your work as you can.
\end{enumerate}
\end{problem}





\begin{problem}
This semester, the ratio of practice problems that Hector has done to the number of problems that Sandy has done is $2:5$. If Hector and Sandy have done $147$ problems all together, how many problems has each student done?


\begin{enumerate}
	\item Use a strip diagram (or several strip diagrams) like the one in Problem 1 to solve this problem. Explain how you drew your picture and how it helped you to solve the problem.
	\item Explain how to use algebra to solve this problem. If you use a variable, be sure to define it in as much detail as you can.
	\item Explain how parts (a) and (b) are connected. Point to as many specifics in your work as you can.
\end{enumerate}
\end{problem}




\newpage



\begin{instructorNotes} 



{\bf Main goal:} We use strip diagrams to solve equations.


{\bf Overall picture:} In this activity, we want students to use the diagram to solve equations. It is often helpful for students to first use the diagram to solve the equation before working algebraically. It tends to be easier to recognize the algebraic solutions in their diagrams than to produce a diagram which matches their calculations. For this reason, we continue to emphasize that our pictures and expressions reflect one another; we don't want students simplifying their expressions in their heads but rather writing expressions which match what they are seeing in a diagram or picture.




\begin{itemize}
	\item Focus on connecting what is happening in the story to the diagram as well as the equation. For instance, in the first problem, we use a block to represent the second section's problems, and so the first section would need four of these blocks. We want to connect this to the meaning of multiplication: we have $4$ groups with the number of problems from the second section in each group. Encourage students to further label the given strip diagram. Similarly, if we wrote an equation, we would use $x$ for the number of problems in the second section, we get $4x$ for the number of problems in the first section, and our equation becomes $4x + x = 80$.
	\item Connect to the meaning of operations as often as you can. For instance, in the first problem we recognize the combining of the problems in the two sections as addition.
	\item Students may struggle to see that blocks can have different values. For instance, in the second problem, we have one block representing Fan's problems, one for Georgina's problems (which both need the variable involved) and then a different block for Thursston's $11$ problems (which does not need the variable). We can combine these two blocks together to represent what we need. The relative sizes of the blocks are less important than the idea that equal-sized bocks represent the same quantities.
	\item When working with fractions, students will likely be more comfortable with the diagrammatic expressions than the algebraic expressions. So for problem 3, students will use the meaning of fractions to determine that there are $9$ problems per box, and solve the problem. With an equation, we can let $x$ be the total number of problems, then $\frac37 x$ are already completed, and we write the equation $\frac37 x + 36 = x$. As we solve we end up with $\frac47 x = 36$, which we divide in four and then multiply by 7. This connection is difficult, and it's okay to take some time to talk about it!
	\item The ratio problem presents a different challenge when writing the equation. Watch out for students who write $2H = 5S$ or a similar equation since the ratio is $2:5$ for Hector's problems : Sandy's problems. Not only will this equation not help them solve the problem at hand, it also doesn't represent the correct relationship between the number of problems unless $H$ is Sandy's problems and $S$ is Hector's problems (which is probably not what the student was intending). If you see this kind of equation, have the student explain their thinking so you can see how to redirect them.
\end{itemize}


{\bf Good language:}  Try to encourage students to be as specific as possible when they define their variables. For instance, in Problem 1, students often say that the variable $x$ represents the second section however the variable more specifically represents the number of problems in the second section. 



{\bf Suggested timing:} Give students about five minutes to work on the first problem, and then have a group present their work using the diagram, and perhaps another group present their algebraic work. Then work as a whole class to connect the actions we took algebraically to the actions we took with the diagram. This will help everyone to be on the same page about using these strip diagrams to solve the problems at hand.

Then, give students about 15 minutes (or a little less than half the remaining time) to work on as many more problems as they can. It's okay if you don't get to all of these problems! Use the remaining time to have groups present their work on as many problems as possible.

\end{instructorNotes}



\end{document}