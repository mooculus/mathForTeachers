\documentclass[nooutcomes]{ximera}
\usepackage{gensymb}
\usepackage{tabularx}
\usepackage{mdframed}
\usepackage{pdfpages}
%\usepackage{chngcntr}

\let\problem\relax
\let\endproblem\relax

\newcommand{\property}[2]{#1#2}




\newtheoremstyle{SlantTheorem}{\topsep}{\fill}%%% space between body and thm
 {\slshape}                      %%% Thm body font
 {}                              %%% Indent amount (empty = no indent)
 {\bfseries\sffamily}            %%% Thm head font
 {}                              %%% Punctuation after thm head
 {3ex}                           %%% Space after thm head
 {\thmname{#1}\thmnumber{ #2}\thmnote{ \bfseries(#3)}} %%% Thm head spec
\theoremstyle{SlantTheorem}
\newtheorem{problem}{Problem}[]

%\counterwithin*{problem}{section}



%%%%%%%%%%%%%%%%%%%%%%%%%%%%Jenny's code%%%%%%%%%%%%%%%%%%%%

%%% Solution environment
%\newenvironment{solution}{
%\ifhandout\setbox0\vbox\bgroup\else
%\begin{trivlist}\item[\hskip \labelsep\small\itshape\bfseries Solution\hspace{2ex}]
%\par\noindent\upshape\small
%\fi}
%{\ifhandout\egroup\else
%\end{trivlist}
%\fi}
%
%
%%% instructorIntro environment
%\ifhandout
%\newenvironment{instructorIntro}[1][false]%
%{%
%\def\givenatend{\boolean{#1}}\ifthenelse{\boolean{#1}}{\begin{trivlist}\item}{\setbox0\vbox\bgroup}{}
%}
%{%
%\ifthenelse{\givenatend}{\end{trivlist}}{\egroup}{}
%}
%\else
%\newenvironment{instructorIntro}[1][false]%
%{%
%  \ifthenelse{\boolean{#1}}{\begin{trivlist}\item[\hskip \labelsep\bfseries Instructor Notes:\hspace{2ex}]}
%{\begin{trivlist}\item[\hskip \labelsep\bfseries Instructor Notes:\hspace{2ex}]}
%{}
%}
%% %% line at the bottom} 
%{\end{trivlist}\par\addvspace{.5ex}\nobreak\noindent\hung} 
%\fi
%
%


\let\instructorNotes\relax
\let\endinstructorNotes\relax
%%% instructorNotes environment
\ifhandout
\newenvironment{instructorNotes}[1][false]%
{%
\def\givenatend{\boolean{#1}}\ifthenelse{\boolean{#1}}{\begin{trivlist}\item}{\setbox0\vbox\bgroup}{}
}
{%
\ifthenelse{\givenatend}{\end{trivlist}}{\egroup}{}
}
\else
\newenvironment{instructorNotes}[1][false]%
{%
  \ifthenelse{\boolean{#1}}{\begin{trivlist}\item[\hskip \labelsep\bfseries {\Large Instructor Notes: \\} \hspace{\textwidth} ]}
{\begin{trivlist}\item[\hskip \labelsep\bfseries {\Large Instructor Notes: \\} \hspace{\textwidth} ]}
{}
}
{\end{trivlist}}
\fi


%% Suggested Timing
\newcommand{\timing}[1]{{\bf Suggested Timing: \hspace{2ex}} #1}




\hypersetup{
    colorlinks=true,       % false: boxed links; true: colored links
    linkcolor=blue,          % color of internal links (change box color with linkbordercolor)
    citecolor=green,        % color of links to bibliography
    filecolor=magenta,      % color of file links
    urlcolor=cyan           % color of external links
}

\title{Taking Up Space}
\author{Vic Ferdinand, Betsy McNeal, Jenny Sheldon}

\begin{document}
\begin{abstract}
\end{abstract}
\maketitle



\begin{problem}
A rectangle has one side which measures 8 inches and another side which measures 5 inches.  What is the area of this rectangle?  Explain carefully, using the meaning of any operation as well as what it means to measure area.


\end{problem}

\begin{problem}
A rectangle has one side which measures 8 inches and another side which measures 5 feet.  What is the area of this rectangle?  Give at least two answers to this question.  Explain your answers carefully, using the meaning of any operation as well as what it means to measure area.

\end{problem}


\begin{problem} \label{TakingUpSpace3}
Draw a rectangle whose area is $\frac{1}{4}$ sq. in.  Draw a shape which is not a rectangle whose area is $\frac{1}{4}$ sq. in.


\end{problem}

\begin{problem} \label{TakingUpSpace4}
Martina's teacher asked her to draw a shape which had an area of four square inches.  Martina drew a four-inch by four-inch square, which she labeled 4 in$^2$.

\begin{enumerate}
\item What is the actual area of the shape Martina drew?  How could you help her fix her drawing to actually show four square inches?
\item How could Martina's notation have contributed to her confusion?  Is there a better notation to use?
\end{enumerate}
\end{problem}

\newpage

\begin{instructorNotes}
The purpose of this activity is to help students cement their understanding of the meaning of area, as well as draw out some common misconceptions about area.  This activity follows ``Building Blocks'' as our second activity about area.  In our course, we have already done a measurement unit which brings up the meaning of area, and discusses the process for measuring area by covering with units and counting.  This activity should build on these concepts to help students clarify their notions of area.  

After this activity, students should be able to confidently answer the question: What is area, and how do we measure it?

The first two questions are included as a sort of check that students have understood the formula for calculating areas of rectangles, and when this formula is appropriate to use.  Our students are generally quick to solve these problems because of our earlier work.

The final two problems are included to bring out some misconceptions about area.  Problem \ref{TakingUpSpace3} usually produces a variety of answers among our students, including squares whose side length measures $\frac14$ inch.  We point our students back to the meaning of area, which has not changed now that we are working with fractions.  Some students also occasionally answer that this problem is impossible, because we cannot subdivide the unit.  Problem \ref{TakingUpSpace4} is a common misconception for children, but we see this issue sometimes with our own students as well.  This problem gives us an opportunity to discuss that we prefer the notation ``sq. un.'' for square units, rather than ``un$^2$'', because the first notation refers to an actual object while the second seems to be just an algebraic object to our students.

\timing{This activity takes us about half a class period.  We do not spend much time on the first two problems, as much of this has already been covered.  We spend the majority of the time, at least in discussion, on the final two problems.  Giving students 5-10 minutes to work through the problem, then spending 20 minutes in discussion seems to work well.}

\end{instructorNotes}


\end{document}