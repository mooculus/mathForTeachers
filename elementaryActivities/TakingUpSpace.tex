\documentclass{ximera}

\graphicspath{
  {./}
  {graphics/}
  {../graphics/}
}

\usepackage{chngcntr}

\let\question\relax
\let\endquestion\relax




\newtheoremstyle{SlantTheorem}{\topsep}{\fill}%%% space between body and thm
%\newtheoremstyle{SlantTheorem}{\topsep}{\topsep}%%% space between body and thm
 {\slshape}                      %%% Thm body font
 {}                              %%% Indent amount (empty = no indent)
 {\bfseries\sffamily}            %%% Thm head font
 {}                              %%% Punctuation after thm head
 {3ex}                           %%% Space after thm head
 {\thmname{#1}\thmnumber{ #2}\thmnote{ \bfseries(#3)}}%%% Thm head spec
\theoremstyle{SlantTheorem}
\newtheorem{question}{Question}
\counterwithin*{question}{section}



\let\instructorNotes\relax
\let\endinstructorNotes\relax
%%% instructorNotes environment
\ifhandout
\newenvironment{instructorNotes}[1][false]%
{%
\def\givenatend{\boolean{#1}}\ifthenelse{\boolean{#1}}{\begin{trivlist}\item}{\setbox0\vbox\bgroup}{}
}
{%
\ifthenelse{\givenatend}{\end{trivlist}}{\egroup}{}
}
\else
\newenvironment{instructorNotes}[1][false]%
{%
  \ifthenelse{\boolean{#1}}{\begin{trivlist}\item[\hskip \labelsep\bfseries {\Large Instructor Notes: \\} \hspace{\textwidth} ]}
{\begin{trivlist}\item[\hskip \labelsep\bfseries {\Large Instructor Notes: \\} \hspace{\textwidth} ]}
{}
}
{\end{trivlist}}
\fi


%% Suggested Timing
\newcommand{\timing}[1]{{\bf Suggested Timing: \hspace{2ex}} #1}

\title{Taking Up Space}
\author{Vic Ferdinand, Betsy McNeal, Jenny Sheldon}

\begin{document}
\begin{abstract}
\end{abstract}
\maketitle

\begin{instructorIntro}
This activity should build on what we've discussed in measurement to help students clarify their notions of area.  You should make sure to mention that area is the amount of 2D space an object takes up, and we measure it by the same iterative process we used when measuring length.  We choose an area unit, then count how many of this unit will fill up the desired space.  This activity also introduces fractional area, so you should probably make sure to spend sufficient time with this concept.  We do not have to use whole units, but can use parts of units as well.

The activity Building Blocks will help students to make a more clear connection between the ``length times width'' formula and the area of a rectangle.  

After this activity, students should be able to answer the question: What is area, and how do we measure it?
\end{instructorIntro}

\begin{problem}
A rectangle has one side which measures 8 inches and another side which measures 5 inches.  What is the area of this rectangle?  Explain carefully, using the meaning of any operation as well as what it means to measure area.


\end{problem}

\begin{problem}
A rectangle has one side which measures 8 inches and another side which measures 5 feet.  What is the area of this rectangle?  Give at least two answers to this question.  Explain your answers carefully, using the meaning of any operation as well as what it means to measure area.

\end{problem}


\begin{problem}
Draw a rectangle whose area is $\frac{1}{4}$ sq. in.  Draw a shape which is not a rectangle whose area is $\frac{1}{4}$ sq. in.


\end{problem}

\begin{problem}
Martina's teacher asked her to draw a shape which had an area of four square inches.  Martina drew a four-inch by four-inch square, which she labeled 4 in$^2$.

\begin{enumerate}
\item What is the actual area of the shape Martina drew?  How could you help her fix her drawing to actually show four square inches?
\item How could Martina's notation have contributed to her confusion?  Is there a better notation to use?
\end{enumerate}
\end{problem}


\end{document}