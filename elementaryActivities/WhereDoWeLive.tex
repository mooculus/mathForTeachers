\documentclass[nooutcomes]{ximera}
\usepackage{gensymb}
\usepackage{tabularx}
\usepackage{mdframed}
\usepackage{pdfpages}
%\usepackage{chngcntr}

\let\problem\relax
\let\endproblem\relax

\newcommand{\property}[2]{#1#2}




\newtheoremstyle{SlantTheorem}{\topsep}{\fill}%%% space between body and thm
 {\slshape}                      %%% Thm body font
 {}                              %%% Indent amount (empty = no indent)
 {\bfseries\sffamily}            %%% Thm head font
 {}                              %%% Punctuation after thm head
 {3ex}                           %%% Space after thm head
 {\thmname{#1}\thmnumber{ #2}\thmnote{ \bfseries(#3)}} %%% Thm head spec
\theoremstyle{SlantTheorem}
\newtheorem{problem}{Problem}[]

%\counterwithin*{problem}{section}



%%%%%%%%%%%%%%%%%%%%%%%%%%%%Jenny's code%%%%%%%%%%%%%%%%%%%%

%%% Solution environment
%\newenvironment{solution}{
%\ifhandout\setbox0\vbox\bgroup\else
%\begin{trivlist}\item[\hskip \labelsep\small\itshape\bfseries Solution\hspace{2ex}]
%\par\noindent\upshape\small
%\fi}
%{\ifhandout\egroup\else
%\end{trivlist}
%\fi}
%
%
%%% instructorIntro environment
%\ifhandout
%\newenvironment{instructorIntro}[1][false]%
%{%
%\def\givenatend{\boolean{#1}}\ifthenelse{\boolean{#1}}{\begin{trivlist}\item}{\setbox0\vbox\bgroup}{}
%}
%{%
%\ifthenelse{\givenatend}{\end{trivlist}}{\egroup}{}
%}
%\else
%\newenvironment{instructorIntro}[1][false]%
%{%
%  \ifthenelse{\boolean{#1}}{\begin{trivlist}\item[\hskip \labelsep\bfseries Instructor Notes:\hspace{2ex}]}
%{\begin{trivlist}\item[\hskip \labelsep\bfseries Instructor Notes:\hspace{2ex}]}
%{}
%}
%% %% line at the bottom} 
%{\end{trivlist}\par\addvspace{.5ex}\nobreak\noindent\hung} 
%\fi
%
%


\let\instructorNotes\relax
\let\endinstructorNotes\relax
%%% instructorNotes environment
\ifhandout
\newenvironment{instructorNotes}[1][false]%
{%
\def\givenatend{\boolean{#1}}\ifthenelse{\boolean{#1}}{\begin{trivlist}\item}{\setbox0\vbox\bgroup}{}
}
{%
\ifthenelse{\givenatend}{\end{trivlist}}{\egroup}{}
}
\else
\newenvironment{instructorNotes}[1][false]%
{%
  \ifthenelse{\boolean{#1}}{\begin{trivlist}\item[\hskip \labelsep\bfseries {\Large Instructor Notes: \\} \hspace{\textwidth} ]}
{\begin{trivlist}\item[\hskip \labelsep\bfseries {\Large Instructor Notes: \\} \hspace{\textwidth} ]}
{}
}
{\end{trivlist}}
\fi


%% Suggested Timing
\newcommand{\timing}[1]{{\bf Suggested Timing: \hspace{2ex}} #1}




\hypersetup{
    colorlinks=true,       % false: boxed links; true: colored links
    linkcolor=blue,          % color of internal links (change box color with linkbordercolor)
    citecolor=green,        % color of links to bibliography
    filecolor=magenta,      % color of file links
    urlcolor=cyan           % color of external links
}

\title{Where Do We Live?}
\author{Vic Ferdinand, Betsy McNeal, Jenny Sheldon}

\begin{document}
\begin{abstract}
\end{abstract}


\maketitle



\begin{problem} \label{WhereLive1}
    With your group, list at least four one- two- and three-dimensional aspects of a house.
\end{problem}

\begin{problem} \label{WhereLive2}
    In your own words, what is ``dimension''?  How would you describe this idea to a child in Kindergarten?  How would you help a fourth-grader who is struggling with this concept?
\end{problem}

\begin{problem} \label{WhereLive3}
    Sally claims that the amount of gutter on the side of your house is two-dimensional, because it goes around both the length and the width of the house.  Discuss Sally's claim.
\end{problem}

\begin{problem} \label{WhereLive4}
    Suzi argues that the amount of carpet in your house is three-dimensional, while Shawn argues that it is two-dimensional.  Could both be correct?
\end{problem}

\begin{problem} \label{WhereLive5}
You have an antique grandfather clock that you would like to move into your new house.  What kind of measurements would you need to take in order to determine if the clock will fit in the house?  What dimensions are associated with these measurements?  Repeat the question for your over-sized couch.
\end{problem}

\begin{problem}
    What would it mean for an object to be 0-dimensional?  Give some examples and explain your reasoning.
\end{problem}

\begin{problem}
    What would it mean for an object to be 4-dimensional?  What about $n$-dimensional, for $n$ greater than 4?
\end{problem}


\newpage
\begin{instructorNotes}
    This activity is intended to introduce the notion of dimension.  In our calendar, this activity follows ``I'm the Biggest Kid!'' where we introduce the concept of measurement and is followed by ``Measurement'' where we begin to measure area.
    
    We treat the concept of measurement with a strong emphasis on using non-standard units to measure.  This is an example of our theme of ``making the familiar strange'', or taking students out of their comfort zone so that they cannot just apply old knowledge or mathematical rules, but instead have to think critically about the matter at hand.  We have found students much more willing to focus on the cover-and-count meaning of measurement rather than just applying a formula when the students are working with units that they have never before used.  These non-standard units also help us to draw out general principles about measurement, again rather than simply formulae.
    
    
\begin{itemize}
    \item Problem \ref{WhereLive1} is generally straightforward for our students, though they occasionally have trouble correctly identifying which dimension coincides with the aspects they suggest.  In this case, we suggest that students go back and test their answers to the first question once we have answered the second.
    \item Problem \ref{WhereLive1} is the ``observe'' phase of students' investigation.  Here, they are exploring what dimension means in order to be able to define it in the next question.  The idea that students should identify four aspects in each dimension should help them to come up with creative answers beyond the most obvious.
    \item In Problem \ref{WhereLive2}, we are looking for the following answers. Dimension is essentially ``how many directions can we move in this space?'' In Kindergarten, children should be able to distinguish whether something is like a line, whether it is flat, or whether it takes up space.  As children get older, it might help to think about how many directions one could move within the space (though we need to use caution with this idea since, for instance, there are infinitely many directions we could move in 2D space).
    \item Problems \ref{WhereLive3} and \ref{WhereLive4} are examples of past answers we have seen from students on assessments about dimension.  This concept is incredibly tricky for our students, and so we hope to clear away their misconceptions with questions like these.  Some misconceptions may stem from the fact that a piece of paper is the common example of a 2D object, but the paper is of course 3D.
    \item Problem \ref{WhereLive5} is intended to give students a chance to demonstrate their understanding of dimension.  We discuss multiple aspects that need to be measured for each item: for instance, you would need to measure the (one-dimensional) height of the ceiling in order to determine if the grandfather clock will fit.  However, you should also measure the (one-dimensional) width of the door to determine if you will be able to fit the clock through the door!  Alternatively, you could consider the square footage of the side of the clock and compare with the square footage of the door, but a direct comparison here is less useful.  We try to compare and contrast these measurements.
    \item The last two problems are more of ``bonus'' problems for students to test their understanding.  We do not discuss dimensions higher than three in class, but have included the question for particularly outstanding or curious students to think about.
\end{itemize}
 
    
\timing{We give students 10-15 minutes in their groups, and follow with 10-15 minutes of discussion.}
    
\end{instructorNotes}


\end{document}