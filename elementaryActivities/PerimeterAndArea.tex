\documentclass{ximera}

\graphicspath{
  {./}
  {graphics/}
  {../graphics/}
}

\usepackage{chngcntr}

\let\question\relax
\let\endquestion\relax




\newtheoremstyle{SlantTheorem}{\topsep}{\fill}%%% space between body and thm
%\newtheoremstyle{SlantTheorem}{\topsep}{\topsep}%%% space between body and thm
 {\slshape}                      %%% Thm body font
 {}                              %%% Indent amount (empty = no indent)
 {\bfseries\sffamily}            %%% Thm head font
 {}                              %%% Punctuation after thm head
 {3ex}                           %%% Space after thm head
 {\thmname{#1}\thmnumber{ #2}\thmnote{ \bfseries(#3)}}%%% Thm head spec
\theoremstyle{SlantTheorem}
\newtheorem{question}{Question}
\counterwithin*{question}{section}



\let\instructorNotes\relax
\let\endinstructorNotes\relax
%%% instructorNotes environment
\ifhandout
\newenvironment{instructorNotes}[1][false]%
{%
\def\givenatend{\boolean{#1}}\ifthenelse{\boolean{#1}}{\begin{trivlist}\item}{\setbox0\vbox\bgroup}{}
}
{%
\ifthenelse{\givenatend}{\end{trivlist}}{\egroup}{}
}
\else
\newenvironment{instructorNotes}[1][false]%
{%
  \ifthenelse{\boolean{#1}}{\begin{trivlist}\item[\hskip \labelsep\bfseries {\Large Instructor Notes: \\} \hspace{\textwidth} ]}
{\begin{trivlist}\item[\hskip \labelsep\bfseries {\Large Instructor Notes: \\} \hspace{\textwidth} ]}
{}
}
{\end{trivlist}}
\fi


%% Suggested Timing
\newcommand{\timing}[1]{{\bf Suggested Timing: \hspace{2ex}} #1}

\title{Perimeter and Area}
\author{Vic Ferdinand, Betsy McNeal, Jenny Sheldon}
\begin{document}
\begin{abstract}\end{abstract}
\maketitle

\begin{instructorIntro}
This is a hands-on activity in which students conjecture a relationship between perimeter and area of first rectangles, and then shapes in general.  Students have the opportunity to use the formulae we've justified for various shapes and to think about how to calculate the area of a shape without already having the sides or heights labeled.

\timing{About 10 minutes in groups, about 15 minutes discussion.}

\end{instructorIntro}

\begin{problem}
What is the perimeter of your string?  \end{problem}

\begin{problem} Use your string to make different {\bf rectangles}, and fill in the following chart.
\vskip 0.1in
\begin{tabular}{|p{2.2in}|p{2.2in}|}
\hline
\begin{center} Shape (sketch) \end{center} & \begin{center} Area \end{center}  \\ \hline
 & \\ [10ex] \hline
 & \\ [10 ex] \hline
 & \\ [10 ex] \hline
 & \\ [10 ex] \hline
 & \\ [10 ex] \hline
\end{tabular}
\begin{enumerate}

\item What is the largest possible area (if one exists) that you can make?
\item What is the smallest possible area (if one exists) that you can make?
\end{enumerate}


\begin{instructorNotes}
Students are happy to work on the hands-on part of this activity.  They don't need much help with the first part about rectangles, and seem to come to a conjecture quickly.
\begin{enumerate}
\item Hand out the strings to be used as the ``perimeter'' - it's helpful if everyone's string is approximately the same length so that the students are working on the same problem.  Remind the students that they can use a ruler for this activity, and that it's much easier if you work with a partner.  One person can hold the string, and the other can measure.
	\item During the discussion, have students present enough examples that a trend is clear.  Have the students verbalize this trend.
	\item Particularly with rectangles, we can determine how to make a rectangle with fixed perimeter and area - this is just an algebra problem.  You might need to remind students that they can make rectangles with non-integer side lengths.

\end{enumerate}
\end{instructorNotes}
\end{problem}
\newpage
\begin{problem} Use your string to make different {\bf shapes}, and fill in the following chart.
\vskip 0.1in
\begin{tabular}{|p{2.2in}|p{2.2in}|}
\hline
\begin{center} Shape (sketch) \end{center} & \begin{center} Area \end{center}  \\ \hline
 & \\ [10ex] \hline
 & \\ [10 ex] \hline
 & \\ [10 ex] \hline
 & \\ [10 ex] \hline
 & \\ [10 ex] \hline
\end{tabular}

\begin{enumerate}

\item What is the largest possible area (if one exists) that you can make?
\item What is the smallest possible area (if one exists) that you can make?
\end{enumerate}

\begin{instructorNotes}
  When allowed to make any 2D shapes, students sometimes struggle to calculate the area.  Suggesting they sketch the shape may help, or having them look back at the work they've done previously to calculate areas may also help.  

Students can get pretty creative with this!  Try to direct them to shapes where they can actually calculate the area, not just shapes where they have to estimate.
\end{instructorNotes}
\end{problem}

\begin{problem} For problems 2 and 3, can every area between the largest and smallest be made?  Why or why not?


\begin{instructorNotes}
\begin{enumerate}
	\item You might discuss with students that we need more powerful mathematics to actually prove that these give us the largest area for a fixed perimeter.  In this case, point out that we haven't checked all examples, but have only made a reasonable conjecture.
	\item We won't really ask students again about the smallest possible area, as the answer is a bit difficult to state.  Some students like to say that the minimum is zero.  You might discuss whether it's possible to actually have an area of zero in this case!
\end{enumerate}

\end{instructorNotes}
\end{problem}



\end{document}