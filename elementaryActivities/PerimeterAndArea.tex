\documentclass[nooutcomes]{ximera}
\usepackage{gensymb}
\usepackage{tabularx}
\usepackage{mdframed}
\usepackage{pdfpages}
%\usepackage{chngcntr}

\let\problem\relax
\let\endproblem\relax

\newcommand{\property}[2]{#1#2}




\newtheoremstyle{SlantTheorem}{\topsep}{\fill}%%% space between body and thm
 {\slshape}                      %%% Thm body font
 {}                              %%% Indent amount (empty = no indent)
 {\bfseries\sffamily}            %%% Thm head font
 {}                              %%% Punctuation after thm head
 {3ex}                           %%% Space after thm head
 {\thmname{#1}\thmnumber{ #2}\thmnote{ \bfseries(#3)}} %%% Thm head spec
\theoremstyle{SlantTheorem}
\newtheorem{problem}{Problem}[]

%\counterwithin*{problem}{section}



%%%%%%%%%%%%%%%%%%%%%%%%%%%%Jenny's code%%%%%%%%%%%%%%%%%%%%

%%% Solution environment
%\newenvironment{solution}{
%\ifhandout\setbox0\vbox\bgroup\else
%\begin{trivlist}\item[\hskip \labelsep\small\itshape\bfseries Solution\hspace{2ex}]
%\par\noindent\upshape\small
%\fi}
%{\ifhandout\egroup\else
%\end{trivlist}
%\fi}
%
%
%%% instructorIntro environment
%\ifhandout
%\newenvironment{instructorIntro}[1][false]%
%{%
%\def\givenatend{\boolean{#1}}\ifthenelse{\boolean{#1}}{\begin{trivlist}\item}{\setbox0\vbox\bgroup}{}
%}
%{%
%\ifthenelse{\givenatend}{\end{trivlist}}{\egroup}{}
%}
%\else
%\newenvironment{instructorIntro}[1][false]%
%{%
%  \ifthenelse{\boolean{#1}}{\begin{trivlist}\item[\hskip \labelsep\bfseries Instructor Notes:\hspace{2ex}]}
%{\begin{trivlist}\item[\hskip \labelsep\bfseries Instructor Notes:\hspace{2ex}]}
%{}
%}
%% %% line at the bottom} 
%{\end{trivlist}\par\addvspace{.5ex}\nobreak\noindent\hung} 
%\fi
%
%


\let\instructorNotes\relax
\let\endinstructorNotes\relax
%%% instructorNotes environment
\ifhandout
\newenvironment{instructorNotes}[1][false]%
{%
\def\givenatend{\boolean{#1}}\ifthenelse{\boolean{#1}}{\begin{trivlist}\item}{\setbox0\vbox\bgroup}{}
}
{%
\ifthenelse{\givenatend}{\end{trivlist}}{\egroup}{}
}
\else
\newenvironment{instructorNotes}[1][false]%
{%
  \ifthenelse{\boolean{#1}}{\begin{trivlist}\item[\hskip \labelsep\bfseries {\Large Instructor Notes: \\} \hspace{\textwidth} ]}
{\begin{trivlist}\item[\hskip \labelsep\bfseries {\Large Instructor Notes: \\} \hspace{\textwidth} ]}
{}
}
{\end{trivlist}}
\fi


%% Suggested Timing
\newcommand{\timing}[1]{{\bf Suggested Timing: \hspace{2ex}} #1}




\hypersetup{
    colorlinks=true,       % false: boxed links; true: colored links
    linkcolor=blue,          % color of internal links (change box color with linkbordercolor)
    citecolor=green,        % color of links to bibliography
    filecolor=magenta,      % color of file links
    urlcolor=cyan           % color of external links
}

\title{Perimeter and Area}
\author{Vic Ferdinand, Betsy McNeal, Jenny Sheldon}
\begin{document}
\begin{abstract}\end{abstract}
\maketitle



\begin{problem}
What is the perimeter of your string?  \end{problem}

\begin{problem} Use your string to make different {\bf rectangles}, and fill in the following chart.
\vskip 0.1in
\begin{tabular}{|p{2.2in}|p{2.2in}|}
\hline
\begin{center} Shape (sketch) \end{center} & \begin{center} Area \end{center}  \\ \hline
 & \\ [10ex] \hline
 & \\ [10 ex] \hline
 & \\ [10 ex] \hline
 & \\ [10 ex] \hline
 & \\ [10 ex] \hline
\end{tabular}
\begin{enumerate}

\item What is the largest possible area (if one exists) that you can make?
\item What is the smallest possible area (if one exists) that you can make?
\end{enumerate}

\end{problem}
\newpage
\begin{problem} Use your string to make different {\bf shapes}, and fill in the following chart.
\vskip 0.1in
\begin{tabular}{|p{2.2in}|p{2.2in}|}
\hline
\begin{center} Shape (sketch) \end{center} & \begin{center} Area \end{center}  \\ \hline
 & \\ [10ex] \hline
 & \\ [10 ex] \hline
 & \\ [10 ex] \hline
 & \\ [10 ex] \hline
 & \\ [10 ex] \hline
\end{tabular}

\begin{enumerate}

\item What is the largest possible area (if one exists) that you can make?
\item What is the smallest possible area (if one exists) that you can make?
\end{enumerate}
\end{problem}

\begin{problem} For problems 2 and 3, can every area between the largest and smallest be made?  Why or why not?

\end{problem}

\newpage
\begin{instructorNotes}
This is a hands-on activity in which students conjecture a relationship between perimeter and area of first rectangles, and then shapes in general.  Students have the opportunity to use the formulae we've justified for various shapes and to think about how to calculate the area of a shape without already having the sides or heights labeled.

This activity falls for us in a collection of activities about area.  We have already discussed the meaning of area as well as formulae for calculating the area of polygons and circles.  We typically follow this activity by discussing the Pythagorean Theorem from an area standpoint.

Students are happy to work on the hands-on part of this activity.  They don't need much help with the first part about rectangles, and seem to come to a conjecture quickly.
\begin{itemize}
    \item We have pre-prepared pieces of yarn that we hand out to students which are all approximately the same length.  We made them the same length in order to have an easier time discussing examples across groups.
    \item We typically remind the students that they can use a ruler for this activity, and that it's much easier if you work with a partner.  One person can hold the string, and the other can measure.  We find we usually need to remind students that they can make rectangles with non-integer side lengths.
	\item During the discussion, we have students present enough examples that a trend is clear.  
    \item 	When allowed to make any 2D shapes, students sometimes struggle to calculate the area.  Suggesting they sketch the seems to help.  Students can be pretty creative when they are allowed to create any 2D shape.  We try to direct them to shapes where they can actually calculate the area (using a formula or otherwise), not just shapes where they have to estimate.
	\item We typically mention that we need more powerful mathematics to actually prove that our observations are correct when it comes to the largest area for a fixed perimeter.  We point out that we haven't checked all examples, but have only made a reasonable conjecture.
	\item The issue of the minimum area is a little confusing for our students.  we typically take the stance that there is not a minimum area, though we can make any area larger than zero.  While this is confusing for some students, it usually helps to discuss whether we can create a shape whose area is exactly zero.
\end{itemize}







\timing{We spend about 10 minutes in groups, then about 15 minutes in discussion.}

\end{instructorNotes}



\end{document}