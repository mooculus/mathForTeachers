\documentclass[]{ximera}
\usepackage{gensymb}
\usepackage{tabularx}
\usepackage{mdframed}
\usepackage{pdfpages}
%\usepackage{chngcntr}

\let\problem\relax
\let\endproblem\relax

\newcommand{\property}[2]{#1#2}




\newtheoremstyle{SlantTheorem}{\topsep}{\fill}%%% space between body and thm
 {\slshape}                      %%% Thm body font
 {}                              %%% Indent amount (empty = no indent)
 {\bfseries\sffamily}            %%% Thm head font
 {}                              %%% Punctuation after thm head
 {3ex}                           %%% Space after thm head
 {\thmname{#1}\thmnumber{ #2}\thmnote{ \bfseries(#3)}} %%% Thm head spec
\theoremstyle{SlantTheorem}
\newtheorem{problem}{Problem}[]

%\counterwithin*{problem}{section}



%%%%%%%%%%%%%%%%%%%%%%%%%%%%Jenny's code%%%%%%%%%%%%%%%%%%%%

%%% Solution environment
%\newenvironment{solution}{
%\ifhandout\setbox0\vbox\bgroup\else
%\begin{trivlist}\item[\hskip \labelsep\small\itshape\bfseries Solution\hspace{2ex}]
%\par\noindent\upshape\small
%\fi}
%{\ifhandout\egroup\else
%\end{trivlist}
%\fi}
%
%
%%% instructorIntro environment
%\ifhandout
%\newenvironment{instructorIntro}[1][false]%
%{%
%\def\givenatend{\boolean{#1}}\ifthenelse{\boolean{#1}}{\begin{trivlist}\item}{\setbox0\vbox\bgroup}{}
%}
%{%
%\ifthenelse{\givenatend}{\end{trivlist}}{\egroup}{}
%}
%\else
%\newenvironment{instructorIntro}[1][false]%
%{%
%  \ifthenelse{\boolean{#1}}{\begin{trivlist}\item[\hskip \labelsep\bfseries Instructor Notes:\hspace{2ex}]}
%{\begin{trivlist}\item[\hskip \labelsep\bfseries Instructor Notes:\hspace{2ex}]}
%{}
%}
%% %% line at the bottom} 
%{\end{trivlist}\par\addvspace{.5ex}\nobreak\noindent\hung} 
%\fi
%
%


\let\instructorNotes\relax
\let\endinstructorNotes\relax
%%% instructorNotes environment
\ifhandout
\newenvironment{instructorNotes}[1][false]%
{%
\def\givenatend{\boolean{#1}}\ifthenelse{\boolean{#1}}{\begin{trivlist}\item}{\setbox0\vbox\bgroup}{}
}
{%
\ifthenelse{\givenatend}{\end{trivlist}}{\egroup}{}
}
\else
\newenvironment{instructorNotes}[1][false]%
{%
  \ifthenelse{\boolean{#1}}{\begin{trivlist}\item[\hskip \labelsep\bfseries {\Large Instructor Notes: \\} \hspace{\textwidth} ]}
{\begin{trivlist}\item[\hskip \labelsep\bfseries {\Large Instructor Notes: \\} \hspace{\textwidth} ]}
{}
}
{\end{trivlist}}
\fi


%% Suggested Timing
\newcommand{\timing}[1]{{\bf Suggested Timing: \hspace{2ex}} #1}




\hypersetup{
    colorlinks=true,       % false: boxed links; true: colored links
    linkcolor=blue,          % color of internal links (change box color with linkbordercolor)
    citecolor=green,        % color of links to bibliography
    filecolor=magenta,      % color of file links
    urlcolor=cyan           % color of external links
}

\title{I'm the Biggest Kid!}
\author{Vic Ferdinand, Betsy McNeal, Jenny Sheldon}

\begin{document}
\begin{abstract}\end{abstract}
\maketitle

\begin{instructorIntro}

This activity is intended to introduce the subject of measurement.  The activity should bring up the need for measurement, the ambiguities that can occur when comparing different types of measurements, as well as give a gentle introduction to doing measurement.  We also encounter the idea of measuring different aspects of the same object, and how these different measurements can be somewhat independent.  Also included is some time to discuss children's misconceptions about measurement.

Take-aways from this activity: the concept of ``biggest" doesn't have much meaning unless we are more specific.  Draw out, but don't feel you have to answer or address many issues in measurement.  For example, what is an appropriate unit to use when measuring?  These ideas will be addressed in future activities - it's okay to not feel ``finished" with the discussion here.

\timing{This activity and its discussion should take the entire class period.  Give students about 5 minutes to discuss the situation in the first problem amongst themselves, and then discuss with the class for about 5-10 minutes.  Let students think about the rest of the problems for the next 20 minutes, and use the rest of the period to discuss their ideas.}

\end{instructorIntro}

\begin{problem}
    The students in Ms. Smith's class come in from recess having an argument.  Ms. Smith asks the children about the argument, and it soon becomes clear that the children can't decide who is the biggest in the class.  Why might the children be confused about this question?
    
    \begin{solution}
    The term ``biggest'' isn't very descriptive or very precise.  When you are measuring the attributes of the children, one child may be taller than another, but may also be skinnier.
    \end{solution}
    
    \begin{instructorNotes}
        This question brings up the concept that the ``biggest'' is an ambiguous term.  Are the children debating who is the tallest?  Who weighs the most?  Whose arms are longest?  Have your class come up with as many ways as possible that a child could be considered the ``biggest'' - including some ways that seem a little unreasonable!  Encourage their creativity.
    \end{instructorNotes}
\end{problem}


\begin{problem}
    To settle the matter, Ms. Smith divides her class up into groups, and gives each group a different item to use:
    \begin{itemize}
        \item One group gets a ball of yarn.
        \item One group gets a stack of math textbooks.
        \item One group gets an unsharpened pencil.
        \item One group gets a playground ball.
        \item One group gets a yard stick (a 3-foot ruler).
    \end{itemize}
    
    How could the children in each group use this item to help them answer the question?  Be extremely specific.
   
    
    \begin{instructorNotes}
        Here students are asked to begin to discuss measuring various aspects of the children in the class.  Most of the natural answers should be about 1-dimensional aspects of the children in the class, like their height or perhaps the distance around their middle.  Students should identify both the aspect they are measuring, as well as the unit they will use to measure (in that order).  Not all measurements need to have a unit (for instance, if the children will cut yarn to the height of each child, and then compare the lengths), but most will.  
        
        This is a good time to have a discussion about what a unit is, how we use it, and why choosing one is important!
        
        To use some of the unconventional items, students will have to be a little creative, and perhaps measure aspects of the children that are not one-dimensional.  For example, the children could try to measure the ``biggest'' child by how many math books they can hold up at once.  Another idea for the math books would be to try to figure out how many math books it would take to cover an outline of the child (the area of a child's silhouette).  The playground ball could be used to approximate the volume of a child.
        
        During discussion, when students give a way to use one of the items, ask them to try to describe what, exactly, they are measuring.  Do they think that it's a length?  An area?  A volume?  None of the above?  This is also a good place to have students practice precision in their descriptions.  Make sure the entire class understands how the item is being used in measurement.
    \end{instructorNotes}
\end{problem}

\begin{problem}
    Which item given to the groups is the most useful?  Which item is the least useful?
    
    Come up with at least one way that each item could be misused for answering this question.
    
    \begin{solution}
    Your answers to this question will depend on your preferences, and which attributes you are measuring.  Notice that a ruler is useful for some things, but not useful for others!
    \end{solution}
    
    \begin{instructorNotes}
        The first question is a matter of opinion, but should be fun to discuss.  
        
        The second question is far more interesting, in that students are asked to identify potential misconceptions that children may have about measurement.  For instance, when measuring with the pencil, the children could leave space between successive pencils, resulting in a measurement that is ``too short'' as well as inaccurate.  
        
        Other common ``ruler misconceptions'' can also be discussed, here.
    \end{instructorNotes}
\end{problem}

\begin{problem}
    In what other ways could the children in Ms. Smith's class try to determine which child is the biggest?
    
    \begin{solution}
        Again, be creative!  One obvious attribute we cannot measure with the given tools is the weight of each child (for good reason)!  So, if we had a scale, we could measure which child has the largest weight.
    \end{solution}
    
    \begin{instructorNotes}
        This is another opportunity for students to be creative.  What items would they as teachers put on Ms. Smith's list?  Some measurement could also be done simply by comparison: for instance, we could just line up the children by height to see who is the tallest.  Discuss the pros and cons of the students' methods for determining who is the biggest kid in the class.
    \end{instructorNotes}
\end{problem}



\end{document}