\documentclass[]{ximera}
\usepackage{gensymb}
\usepackage{tabularx}
\usepackage{mdframed}
\usepackage{pdfpages}
%\usepackage{chngcntr}

\let\problem\relax
\let\endproblem\relax

\newcommand{\property}[2]{#1#2}




\newtheoremstyle{SlantTheorem}{\topsep}{\fill}%%% space between body and thm
 {\slshape}                      %%% Thm body font
 {}                              %%% Indent amount (empty = no indent)
 {\bfseries\sffamily}            %%% Thm head font
 {}                              %%% Punctuation after thm head
 {3ex}                           %%% Space after thm head
 {\thmname{#1}\thmnumber{ #2}\thmnote{ \bfseries(#3)}} %%% Thm head spec
\theoremstyle{SlantTheorem}
\newtheorem{problem}{Problem}[]

%\counterwithin*{problem}{section}



%%%%%%%%%%%%%%%%%%%%%%%%%%%%Jenny's code%%%%%%%%%%%%%%%%%%%%

%%% Solution environment
%\newenvironment{solution}{
%\ifhandout\setbox0\vbox\bgroup\else
%\begin{trivlist}\item[\hskip \labelsep\small\itshape\bfseries Solution\hspace{2ex}]
%\par\noindent\upshape\small
%\fi}
%{\ifhandout\egroup\else
%\end{trivlist}
%\fi}
%
%
%%% instructorIntro environment
%\ifhandout
%\newenvironment{instructorIntro}[1][false]%
%{%
%\def\givenatend{\boolean{#1}}\ifthenelse{\boolean{#1}}{\begin{trivlist}\item}{\setbox0\vbox\bgroup}{}
%}
%{%
%\ifthenelse{\givenatend}{\end{trivlist}}{\egroup}{}
%}
%\else
%\newenvironment{instructorIntro}[1][false]%
%{%
%  \ifthenelse{\boolean{#1}}{\begin{trivlist}\item[\hskip \labelsep\bfseries Instructor Notes:\hspace{2ex}]}
%{\begin{trivlist}\item[\hskip \labelsep\bfseries Instructor Notes:\hspace{2ex}]}
%{}
%}
%% %% line at the bottom} 
%{\end{trivlist}\par\addvspace{.5ex}\nobreak\noindent\hung} 
%\fi
%
%


\let\instructorNotes\relax
\let\endinstructorNotes\relax
%%% instructorNotes environment
\ifhandout
\newenvironment{instructorNotes}[1][false]%
{%
\def\givenatend{\boolean{#1}}\ifthenelse{\boolean{#1}}{\begin{trivlist}\item}{\setbox0\vbox\bgroup}{}
}
{%
\ifthenelse{\givenatend}{\end{trivlist}}{\egroup}{}
}
\else
\newenvironment{instructorNotes}[1][false]%
{%
  \ifthenelse{\boolean{#1}}{\begin{trivlist}\item[\hskip \labelsep\bfseries {\Large Instructor Notes: \\} \hspace{\textwidth} ]}
{\begin{trivlist}\item[\hskip \labelsep\bfseries {\Large Instructor Notes: \\} \hspace{\textwidth} ]}
{}
}
{\end{trivlist}}
\fi


%% Suggested Timing
\newcommand{\timing}[1]{{\bf Suggested Timing: \hspace{2ex}} #1}




\hypersetup{
    colorlinks=true,       % false: boxed links; true: colored links
    linkcolor=blue,          % color of internal links (change box color with linkbordercolor)
    citecolor=green,        % color of links to bibliography
    filecolor=magenta,      % color of file links
    urlcolor=cyan           % color of external links
}

\title{I'm the Biggest Kid!}
\author{Vic Ferdinand, Betsy McNeal, Jenny Sheldon}

\begin{document}
\begin{abstract}\end{abstract}
\maketitle



\begin{problem}
    The students in Ms. Smith's class come in from recess having an argument.  Ms. Smith asks the children about the argument, and it soon becomes clear that the children can't decide who is the biggest in the class.  Why might the children be confused about this question?
\end{problem}


\begin{problem} \label{BiggestKid2}
    To settle the matter, Ms. Smith divides her class up into groups, and gives each group a different item to use:
    \begin{itemize}
        \item One group gets a ball of yarn.
        \item One group gets a stack of math textbooks.
        \item One group gets an unsharpened pencil.
        \item One group gets a playground ball.
        \item One group gets a yard stick (a 3-foot ruler).
    \end{itemize}
    
    How could the children in each group use this item to help them answer the question?  Be extremely specific.
\end{problem}

\begin{problem} \label{BiggestKid3}
    Which item given to the groups is the most useful?  Which item is the least useful?
    
    Come up with at least one way that each item could be misused for answering this question.
\end{problem}

\begin{problem}
    In what other ways could the children in Ms. Smith's class try to determine which child is the biggest?
\end{problem}


\newpage

\begin{instructorNotes}

This activity is intended to introduce the subject of measurement. We treat the concept of measurement with a strong emphasis on using non-standard units to measure.  This is an example of our theme of ``making the familiar strange'', or taking students out of their comfort zone so that they cannot just apply old knowledge or mathematical rules, but instead have to think critically about the matter at hand.  We have found students much more willing to focus on the cover-and-count meaning of measurement rather than just applying a formula when the students are working with units that they have never before used.  These non-standard units also help us to draw out general principles about measurement, again rather than simply formulae.

The activity in particular should bring up the need for measurement, the ambiguities that can occur when comparing different types of measurements, as well as give a gentle introduction to doing measurement.  We also encounter the idea of measuring different aspects of the same object, and how these different measurements can be somewhat independent.  Also included is some time to discuss children's misconceptions about measurement.

This activity is also meant to introduce, but not fully explore, many issues about measurement.  We feel we have accomplished our goals with this activity if we discuss how to choose an appropriate unit of measurement, give students a first experience using non-standard units, and give students space to begin to be creative in their thinking about measurement. Other issues, such as the process of measuring, dimension, unit conversion, ruler use, etc., will be treated in later activities.  If these issues come up in discussion, we typically acknowledge that we have more to learn on this topic and promise it's coming soon.

\begin{itemize}
    \item The term ``biggest'' is intentionally vague, here.  The activity is designed to feel realistic in that many of our students who have worked with children in the past have observed a similar argument.  We want this activity to feel almost like something they could do in their future classroom.
    \item The activity also intentionally includes some strange potential units.  Throughout, we encourage students to be as creative as possible, whether they are debating the meaning of ``biggest'' or finding a way to use a particular unit.  We are not necessarily concerned here with what is reasonable, but more with what is possible.  Again, our goal is to draw out general principles.
    \item As students are choosing units and beginning to measure, we insist that they are as specific as possible about how they are using their unit and what exactly they are measuring.  Being this specific can be a bit difficult at first, but students catch on quickly.
    \item In Problem \ref{BiggestKid2}, we have observed that many of our students' first choice of measurement is either a direct comparison not really requiring a measurement (cutting the yarn to the height of a child, then laying the yarn strands next to each other to find the longest) or a one-dimensional aspect (using the unsharpened pencil to measure the height of a child).  We try to encourage students to think about two- or three-dimensional aspects that could be measured.
    \item Some creative examples we have heard in the past:
        \begin{itemize}
            \item Which child can hold up the most math books at once?
            \item How many math books would it take to cover up the outline of a child?
            \item Which child can bounce the playground ball the highest?
        \end{itemize}
    \item In Problem \ref{BiggestKid3}, we like to point out that a ruler is useful for some measurements, but not others.  Before working through this activity, students might have answered that the ruler would be the most helpful, but their creative ideas tend to be more about the other objects.
    \item Problem \ref{BiggestKid3} also gives us an opportunity to begin to talk about misconceptions, as well as to start describing the process of measurement not only by what we do, but by what we don't do.  For instance, someone usually mentions that a misuse of the pencil would be to leave gaps between successive pencils.
    \item The final problem allows us to imagine ourselves in MS. Smith's shoes.  What might we have in our own classrooms that could be used as a unit of measurement?
\end{itemize}
    

\timing{This activity and its discussion takes us the entire class period.  We give students about 5 minutes to discuss the situation in the first problem among themselves, and then discuss with the class for about 5-10 minutes.  We then let students think about the rest of the problems for the next 20 minutes, and use the rest of the period to discuss their ideas.}

\end{instructorNotes}


\end{document}