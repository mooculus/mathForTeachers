\documentclass[nooutcomes]{ximera}
\usepackage{gensymb}
\usepackage{tabularx}
\usepackage{mdframed}
\usepackage{pdfpages}
%\usepackage{chngcntr}

\let\problem\relax
\let\endproblem\relax

\newcommand{\property}[2]{#1#2}




\newtheoremstyle{SlantTheorem}{\topsep}{\fill}%%% space between body and thm
 {\slshape}                      %%% Thm body font
 {}                              %%% Indent amount (empty = no indent)
 {\bfseries\sffamily}            %%% Thm head font
 {}                              %%% Punctuation after thm head
 {3ex}                           %%% Space after thm head
 {\thmname{#1}\thmnumber{ #2}\thmnote{ \bfseries(#3)}} %%% Thm head spec
\theoremstyle{SlantTheorem}
\newtheorem{problem}{Problem}[]

%\counterwithin*{problem}{section}



%%%%%%%%%%%%%%%%%%%%%%%%%%%%Jenny's code%%%%%%%%%%%%%%%%%%%%

%%% Solution environment
%\newenvironment{solution}{
%\ifhandout\setbox0\vbox\bgroup\else
%\begin{trivlist}\item[\hskip \labelsep\small\itshape\bfseries Solution\hspace{2ex}]
%\par\noindent\upshape\small
%\fi}
%{\ifhandout\egroup\else
%\end{trivlist}
%\fi}
%
%
%%% instructorIntro environment
%\ifhandout
%\newenvironment{instructorIntro}[1][false]%
%{%
%\def\givenatend{\boolean{#1}}\ifthenelse{\boolean{#1}}{\begin{trivlist}\item}{\setbox0\vbox\bgroup}{}
%}
%{%
%\ifthenelse{\givenatend}{\end{trivlist}}{\egroup}{}
%}
%\else
%\newenvironment{instructorIntro}[1][false]%
%{%
%  \ifthenelse{\boolean{#1}}{\begin{trivlist}\item[\hskip \labelsep\bfseries Instructor Notes:\hspace{2ex}]}
%{\begin{trivlist}\item[\hskip \labelsep\bfseries Instructor Notes:\hspace{2ex}]}
%{}
%}
%% %% line at the bottom} 
%{\end{trivlist}\par\addvspace{.5ex}\nobreak\noindent\hung} 
%\fi
%
%


\let\instructorNotes\relax
\let\endinstructorNotes\relax
%%% instructorNotes environment
\ifhandout
\newenvironment{instructorNotes}[1][false]%
{%
\def\givenatend{\boolean{#1}}\ifthenelse{\boolean{#1}}{\begin{trivlist}\item}{\setbox0\vbox\bgroup}{}
}
{%
\ifthenelse{\givenatend}{\end{trivlist}}{\egroup}{}
}
\else
\newenvironment{instructorNotes}[1][false]%
{%
  \ifthenelse{\boolean{#1}}{\begin{trivlist}\item[\hskip \labelsep\bfseries {\Large Instructor Notes: \\} \hspace{\textwidth} ]}
{\begin{trivlist}\item[\hskip \labelsep\bfseries {\Large Instructor Notes: \\} \hspace{\textwidth} ]}
{}
}
{\end{trivlist}}
\fi


%% Suggested Timing
\newcommand{\timing}[1]{{\bf Suggested Timing: \hspace{2ex}} #1}




\hypersetup{
    colorlinks=true,       % false: boxed links; true: colored links
    linkcolor=blue,          % color of internal links (change box color with linkbordercolor)
    citecolor=green,        % color of links to bibliography
    filecolor=magenta,      % color of file links
    urlcolor=cyan           % color of external links
}
\title{Taking Chances}
\author{Vic Ferdinand, Betsy McNeal, Jenny Sheldon}

\begin{document}
\begin{abstract}
\end{abstract}
\maketitle



\begin{problem}
David and Cassie are watching ``Survivor''.  There are six remaining contestants on the show.  The announcer boldly states: ``Each of you now has a 1 in 6 chance to win a million dollars!''.

\begin{enumerate}
\item What does this statement have to do with probability?
\item Why is David angry at the announcer?
\end{enumerate}
\end{problem}

\begin{problem}
Antwan and Jake have flipped a coin ten times.  The coin has come up ``heads'' two times and ``tails'' eight times.  Jake says, ``the next flip is more likely to be a head, since we haven't seen very many of those''.
\begin{enumerate}
\item What does this statement have to do with probability?
\item Why does Antwan disagree with Jake?
\end{enumerate}
\end{problem}

\begin{problem}
A family night at school features the following game.  There are two opaque bags, each containing red blocks and yellow blocks.  Bag 1 contains 3 red blocks and 5 yellow blocks.  Bag 2 contains 5 red blocks and 15 yellow blocks.  To play the game, you pick a bag and then pick a block out of the bag without looking.  You win a prize if you pick a red block.  Kate says, ``I'll pick from Bag 2, because it has more red blocks!''
\begin{enumerate}
\item What does this statement have to do with probability?
\item Why does Kate's mother caution her to think about the situation again?
\end{enumerate}
\end{problem}

\newpage

\begin{instructorNotes}
As a wrap-up to our discussion of probability, we consider several common misconceptions.

This activity assumes that students are now familiar with counting and probability, and able to step back and consider the concepts involved rather than just focus on solving problems.

\timing{If we have time for this activity at all, half a class is usually enough.  We then use the rest of the time to go back over other problems or trouble the students are having.}

\end{instructorNotes}

\end{document}