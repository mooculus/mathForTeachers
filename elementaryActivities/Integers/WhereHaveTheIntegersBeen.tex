\documentclass[ nooutcomes,noauthor]{ximera}


\graphicspath{
  {./}
  {graphics/}
  {../graphics/}
}

\usepackage{chngcntr}

\let\question\relax
\let\endquestion\relax




\newtheoremstyle{SlantTheorem}{\topsep}{\fill}%%% space between body and thm
%\newtheoremstyle{SlantTheorem}{\topsep}{\topsep}%%% space between body and thm
 {\slshape}                      %%% Thm body font
 {}                              %%% Indent amount (empty = no indent)
 {\bfseries\sffamily}            %%% Thm head font
 {}                              %%% Punctuation after thm head
 {3ex}                           %%% Space after thm head
 {\thmname{#1}\thmnumber{ #2}\thmnote{ \bfseries(#3)}}%%% Thm head spec
\theoremstyle{SlantTheorem}
\newtheorem{question}{Question}
\counterwithin*{question}{section}



\let\instructorNotes\relax
\let\endinstructorNotes\relax
%%% instructorNotes environment
\ifhandout
\newenvironment{instructorNotes}[1][false]%
{%
\def\givenatend{\boolean{#1}}\ifthenelse{\boolean{#1}}{\begin{trivlist}\item}{\setbox0\vbox\bgroup}{}
}
{%
\ifthenelse{\givenatend}{\end{trivlist}}{\egroup}{}
}
\else
\newenvironment{instructorNotes}[1][false]%
{%
  \ifthenelse{\boolean{#1}}{\begin{trivlist}\item[\hskip \labelsep\bfseries {\Large Instructor Notes: \\} \hspace{\textwidth} ]}
{\begin{trivlist}\item[\hskip \labelsep\bfseries {\Large Instructor Notes: \\} \hspace{\textwidth} ]}
{}
}
{\end{trivlist}}
\fi


%% Suggested Timing
\newcommand{\timing}[1]{{\bf Suggested Timing: \hspace{2ex}} #1}


\title{Where Have the Integers Been?}
\author{Vic Ferdinand, Betsy McNeal, Jenny Sheldon}

\begin{document}
\begin{abstract}
    We begin to model integers and their operations.
\end{abstract}
\maketitle



\begin{problem}
What is an integer?  How do integers compare with whole numbers?  Where do integers show up in real-life situations?  Give as many examples as you can think of -- we'll want a big list.
\end{problem}

\begin{problem}
For each of the numbers below, use at least two of your real-life situations to write a sentence which uses that number in a sensible context.  Then, draw a picture which could represent your number.
\begin{enumerate}
    \item $3$
    \item $-3$
    \item $-18$
    \item $-62$
\end{enumerate}

\end{problem}

\begin{problem}
	Are numbers like $-1.85$ integers? Could we still use any of our models for a number like this? Explain your thinking.
\end{problem}

\begin{problem}
Draw a number line which includes all of the numbers in the previous two questions.  How did you decide where to locate the negative numbers?
\end{problem}

\begin{problem}
For each of the addition problems below, use at least two of your real-life situations to write a story problem which would be answered by that expression.  (Be sure to include a question!)  Then, draw a picture explaining what a child learning this concept might do to solve the problem.
\begin{enumerate}
    \item $2+6$
    \item $(-3) + 5$
    \item $5 + (-8)$
    \item $(-1) + (-14)$
\end{enumerate}

\end{problem}

\begin{problem}
Look back again at each of the addition problems you did in the previous problem.  Does the sign of the answer make sense from your story?  Could you model the addition problem using a number line?

\end{problem}


\newpage
\begin{instructorNotes}
This activity is the first of three that investigate operations with negative numbers.  The second is ``Integers and Subtraction'' and the third is ``Our Problems are Multiplying''.  There is an accompanying reading (\url{https://ximera.osu.edu/m4t/elementaryTeachersOne}) on integers that lays out some of the more complicated framework.

In this first activity, our goal is to define integers, give some examples of real-life contexts in which negative numbers make sense, and then begin using these contexts to draw pictures and solve addition problems.  Overall, we try to treat integers not just from an algebraic standpoint, but focusing more on how young children think about integers so that our students learn to reason about these types of numbers rather than simply memorize rules for operating with them.  As usual, we are trying to communicate that mathematics is about reasoning, not just rules.

The reading includes a short historical discussion about integers which can help frame our in-class discussions.  We find it helpful to acknowledge that this concept was difficult for mathematicians for centuries, so we should not be surprised if it is also difficult for children.  Furthermore, we save this unit on integers for after we have discussed all other classes of numbers as well as all operations, so this unit helps to cement many of the topics we have already treated.

We use multiple contexts to talk about integers to allow for some flexibility in discussion later.  Especially on this first day, we let students try out any context they like, but moving forward we will focus on three contexts: checks and bills (or a Venmo interpretation; this method is also called the shopkeeper's method or ``Mail Time''), number lines, and chip models.  These contexts are described in the reading, and we have chosen them because they have the fewest issues (in our opinion).

\begin{itemize}
    \item We begin our discussion with student experimentation.  First, we give students time to come up with their own contexts, and then make a big list of possibilities as a class.
    \item After making this list, we give students time to try to draw a picture of a negative number.  Students may already start to realize that some contexts are more useful than others.  We begin to discuss this idea as we discuss the student work, and also introduce the checks-and-bills method if it has not yet come up.
    \item Often students' pictures come in two forms: pictures that look vaguely like number lines (temperature scales, elevators, actual number lines) and pictures that look vaguely like a chip model (two types of objects, labeled objects, heads and tails of coins).  If we can see many student pictures at the same time, we point out these similarities and differences for the class.
    \item  If the chips model is used, one feature that will come in handy when subtracting integers is that one can represent a given integer in an infinite number of ways. This is the most difficult aspect of the chips model for students, and will need a lot of practice.
    \item After this experimentation, we begin to focus students on number lines and checks and bills with whatever time is left.
    \item  With the checks and bills model, it is important that the question being asked reflects the true answer to the addition.  For example, $(-5) + (3)$ could be incorrectly modeled by ``I received two things in the mail: A bill for 5 dollars and a check for 3 dollars.  What is my net worth now?''  Here we don't know the initial net worth, so we cannot answer the question.  Instead, to ask this question, we would need to state that we started out with zero net worth OR we begin the problem with a net worth of negative 5 dollars and received a check for 3 dollars.
\end{itemize}

{\bf Suggested Timing:} This activity takes us a full class period (or more).  We often move some of the addition on to the second day.  As indicated earlier, we use 5 minutes to let students come up with contexts, and then 10 minutes to discuss.  We then let students draw and experiment for about 10 minutes, followed by 20 minutes of discussion.  The remainder of the time is spent on the last two problems.
\end{instructorNotes}





\end{document}
% \documentclass{ximera}

% 
\graphicspath{
  {./}
  {graphics/}
  {../graphics/}
}

\usepackage{chngcntr}

\let\question\relax
\let\endquestion\relax




\newtheoremstyle{SlantTheorem}{\topsep}{\fill}%%% space between body and thm
%\newtheoremstyle{SlantTheorem}{\topsep}{\topsep}%%% space between body and thm
 {\slshape}                      %%% Thm body font
 {}                              %%% Indent amount (empty = no indent)
 {\bfseries\sffamily}            %%% Thm head font
 {}                              %%% Punctuation after thm head
 {3ex}                           %%% Space after thm head
 {\thmname{#1}\thmnumber{ #2}\thmnote{ \bfseries(#3)}}%%% Thm head spec
\theoremstyle{SlantTheorem}
\newtheorem{question}{Question}
\counterwithin*{question}{section}



\let\instructorNotes\relax
\let\endinstructorNotes\relax
%%% instructorNotes environment
\ifhandout
\newenvironment{instructorNotes}[1][false]%
{%
\def\givenatend{\boolean{#1}}\ifthenelse{\boolean{#1}}{\begin{trivlist}\item}{\setbox0\vbox\bgroup}{}
}
{%
\ifthenelse{\givenatend}{\end{trivlist}}{\egroup}{}
}
\else
\newenvironment{instructorNotes}[1][false]%
{%
  \ifthenelse{\boolean{#1}}{\begin{trivlist}\item[\hskip \labelsep\bfseries {\Large Instructor Notes: \\} \hspace{\textwidth} ]}
{\begin{trivlist}\item[\hskip \labelsep\bfseries {\Large Instructor Notes: \\} \hspace{\textwidth} ]}
{}
}
{\end{trivlist}}
\fi


%% Suggested Timing
\newcommand{\timing}[1]{{\bf Suggested Timing: \hspace{2ex}} #1}


% \title{Where Have the Integers Been?}

% \begin{document}
% \begin{abstract}
%     We begin to model integers and their operations.
% \end{abstract}
% \maketitle

% \begin{instructorIntro}
% This activity is the first of a three-day segment where we talk about operations with negative numbers.  There is an accompanying reading on integers that lays out some of the more complicated framework.

% On this first day, we first make sure we know what an integer is, and give some examples of real-life contexts in which negative numbers make sense.  Having multiple contexts allows for some flexibility in discussion later, but we will focus on three contexts: checks and bills (or the shopkeeper's method or ``Mail Time''), number lines, and to a lesser extent, chip models.

% This first day should be mostly for experimenting.  Feel free to discuss lots of different ideas, and don't be afraid to point out places where the ideas are good, and places where they have some issues.  Every context will have issues, and so it can be nice to say something about how we've chosen our three favorite contexts because they have the fewest problems (in our opinion).  By the end of this first day, each of the three methods mentioned above should have been introduced (with the possible exception of the chips, if you are skipping this).

% {\bf Suggested Timing:} This activity should take a full class period.  Don't worry if there are lingering issues at the end of the day, as there will be more time later.
% \end{instructorIntro}

% \begin{question}
% What is an integer?  How do integers compare with whole numbers?  Where do integers show up in real-life situations?  Give as many examples as you can think of -- we'll want a big list.

% %\begin{instructorNotes}
% %I'm thinking about giving them about 3 minutes to think of as many contexts as they can, then writing as many on the board as they come up with.  Maybe I also throw in red/black chips and checks and bills if similar ideas don't come up on their own.
% %\end{instructorNotes}
% \end{question}

% \begin{question}
% For each of the integers below, use at least two of your real-life situations to write a sentence which uses that number in a sensible context.  Then, draw a picture which could represent your integer.
% \begin{enumerate}
%     \item $3$
%     \item $-3$
%     \item $-1.85$
%     \item $-62$
% \end{enumerate}

% %\begin{instructorNotes}
% %I'm thinking we discuss only two of these parts, and leave the other two for practice.  Probably (b) and (d) would be my choice.  I'm hoping with the pictures they come up with some sensible way to represent negative numbers which is equivalent to the red/black chips but we will see how it goes.
% %\end{instructorNotes}
% \end{question}

% \begin{question}
% Draw a number line which includes all of the numbers in the previous question.  How did you decide where to locate the negative numbers?
% %\begin{instructorNotes}
% %Unless I have more than half a class period, I might skip this question entirely.  This will be good for them to use for thinking, and I think we will probably have time to discuss number lines with subtraction on Day 2.  But I stuck it in here to give the quick thinkers something to do.
% %\end{instructorNotes}
% \end{question}

% \begin{question}
% For each of the addition problems below, use at least two of your real-life situations to write a story problem which would be answered by that expression.  (Be sure to include a question!)  Then, draw a picture explaining what a child learning this concept might do to solve the problem.
% \begin{enumerate}
%     \item $2+6$
%     \item $(-3) + 5$
%     \item $5 + (-8)$
%     \item $(-1) + (-14)$
% \end{enumerate}

% %\begin{instructorNotes}
% %This is where I plan to spend most of the time.  Again, I will probably only discuss 2 of these after giving them about 5 minutes to think about it.  I might even divide up the class into four groups, one for each part, and see how much discussion we can get out of that. I wanted this problem to look a lot like some of the others we've done when introducing new operations, and also to give the students more practice writing story problems, since we've been focusing on that this semester.
% %\end{instructorNotes}
% \end{question}

% \begin{question}
% Look back again at each of the addition problems you did in the previous problem.  Does the sign of the answer make sense from your story?  Could you model the addition problem using a number line?

% %\begin{instructorNotes}
% %I'm not planning to get to this either.  After they talk about number lines with subtraction, I'm hoping this will be easy practice.
% %\end{instructorNotes}
% \end{question}









% \end{document}