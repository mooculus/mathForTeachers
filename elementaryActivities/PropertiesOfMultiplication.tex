\documentclass[nooutcomes]{ximera}


\graphicspath{
  {./}
  {graphics/}
  {../graphics/}
}

\usepackage{chngcntr}

\let\question\relax
\let\endquestion\relax




\newtheoremstyle{SlantTheorem}{\topsep}{\fill}%%% space between body and thm
%\newtheoremstyle{SlantTheorem}{\topsep}{\topsep}%%% space between body and thm
 {\slshape}                      %%% Thm body font
 {}                              %%% Indent amount (empty = no indent)
 {\bfseries\sffamily}            %%% Thm head font
 {}                              %%% Punctuation after thm head
 {3ex}                           %%% Space after thm head
 {\thmname{#1}\thmnumber{ #2}\thmnote{ \bfseries(#3)}}%%% Thm head spec
\theoremstyle{SlantTheorem}
\newtheorem{question}{Question}
\counterwithin*{question}{section}



\let\instructorNotes\relax
\let\endinstructorNotes\relax
%%% instructorNotes environment
\ifhandout
\newenvironment{instructorNotes}[1][false]%
{%
\def\givenatend{\boolean{#1}}\ifthenelse{\boolean{#1}}{\begin{trivlist}\item}{\setbox0\vbox\bgroup}{}
}
{%
\ifthenelse{\givenatend}{\end{trivlist}}{\egroup}{}
}
\else
\newenvironment{instructorNotes}[1][false]%
{%
  \ifthenelse{\boolean{#1}}{\begin{trivlist}\item[\hskip \labelsep\bfseries {\Large Instructor Notes: \\} \hspace{\textwidth} ]}
{\begin{trivlist}\item[\hskip \labelsep\bfseries {\Large Instructor Notes: \\} \hspace{\textwidth} ]}
{}
}
{\end{trivlist}}
\fi


%% Suggested Timing
\newcommand{\timing}[1]{{\bf Suggested Timing: \hspace{2ex}} #1}


\title{Properties of Multiplication}
\author{Vic Ferdinand, Betsy McNeal, Jenny Sheldon}

\begin{document}
\begin{abstract} \end{abstract}
\maketitle



\begin{problem}

 The number of objects in each of the following diagrams could be counted using multiplication in many ways.  For each diagram, write as many multiplication expressions as you can see in the picture.  You might need parenthesis or even some other operation(s) in your expression!  Be sure to explain your thinking!
\begin{enumerate}
    \setlength\itemsep{1cm}
    \item \leavevmode\vadjust{\vspace{-\baselineskip}}\newline {\huge \tt
        \begin{tabular}{cccccccccccc}
            X & X & X & X & X & X & X & X & X & X & X & X\\
            X & X & X & X & X & X & X & X & X & X & X & X\\
            X & X & X & X & X & X & X & X & X & X & X & X\\
            X & X & X & X & X & X & X & X & X & X & X & X\\
        \end{tabular} }
    \item \leavevmode\vadjust{\vspace{-\baselineskip}}\newline {\large \tt 
        \begin{tabular}{p{0.5cm}p{0.5cm}p{0.5cm}p{0.5cm}p{0.5cm}p{0.5cm}p{0.5cm}p{0.5cm}p{0.5cm}p{0.5cm}p{0.5cm}}
            XX\newline XX & & XX \newline XX & & XX\newline XX & & XX \newline XX & & XX\newline XX & & XX\newline XX  \\[0.3cm]
             XX\newline XX & & XX \newline XX & & XX\newline XX & & XX \newline XX & & XX\newline XX & & XX\newline XX  \\
        \end{tabular} }
    \item \leavevmode\vadjust{\vspace{-\baselineskip}}\newline {\huge \tt
        \begin{tabular}{p{1cm}p{0.5cm}p{1cm}p{0.5cm}p{1cm}p{0.5cm}p{1cm}p{0.5cm}p{1cm}}
            OO \newline XXX & & OO \newline XXX & & OO \newline XXX & & OO \newline XXX  \\[0.3cm]
            OO \newline XXX & & OO \newline XXX & & OO \newline XXX & & OO \newline XXX  \\
        \end{tabular} }
\end{enumerate}
\end{problem}

\begin{problem}
 For at least two of the multiplication expressions you found in each part in the previous problem, write a story problem which would be answered by that expression.  Trade stories with a partner to test your writing skills!
\end{problem}

\newpage
\begin{instructorNotes}
This activity is designed to help students see why the commutative, associative, and distributive properties of multiplication are true.  Note that the activity does not ask students to identify these properties, or to use certain properties in certain places.  The activity is designed for the instructor to pull out the properties from the students' responses to the first question.

We have chosen to discuss the properties in this format because we have found that students are often able to easily identify a specified property, but often struggle to find where they have used the properties in their own work.  Also, since we are not asking for specific properties in specific places, students have more freedom to develop their creativity.

\begin{itemize}
    \item Students are usually very creative with their expressions.  We often hear answers from students who can be more reluctant to speak in front of the class, especially if we encourage them to share an idea that is different from the others.
    \item Though the parts of this problem are set up to make certain properties easier to observe, any of the parts of this problem can be used to develop any of the properties.  More creative student answers gives a greater flexibility for us to pull the properties out of student work.
    \item We usually elicit a number of student responses, then select from those responses to highlight and define the properties in question.  For instance, on part (a) if we have the expressions $4 \times 12$ and $12 \times 4$, we can bring up the commutative property.  But we also sometimes see things like $2 \times (2\times 12)$ and $(2 \times 2) \times 12$ even for part (a), and we can discuss the associative property here as well.  The distributive property can also come up in this way.
    \item We emphasize for students that there are three ideas present in each property: the two expressions, and the equals sign.  In our experience, students can be unfamiliar with justifying the presence of the equals sign, so we emphasize that have two expressions for the exact same number of items -- the picture hasn't changed.
    \item We also emphasize that we are considering specific examples here, and a full explanation for why the properties hold should attend to other arrangements of objects (for other numbers) as well.
\end{itemize}


{\bf Suggested Timing:} This activity should take about half a class period.  The other half can nicely be spent on some mental math questions.
\end{instructorNotes}


\end{document}