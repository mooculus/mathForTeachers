\documentclass{ximera}

\graphicspath{
  {./}
  {graphics/}
  {../graphics/}
}

\usepackage{chngcntr}

\let\question\relax
\let\endquestion\relax




\newtheoremstyle{SlantTheorem}{\topsep}{\fill}%%% space between body and thm
%\newtheoremstyle{SlantTheorem}{\topsep}{\topsep}%%% space between body and thm
 {\slshape}                      %%% Thm body font
 {}                              %%% Indent amount (empty = no indent)
 {\bfseries\sffamily}            %%% Thm head font
 {}                              %%% Punctuation after thm head
 {3ex}                           %%% Space after thm head
 {\thmname{#1}\thmnumber{ #2}\thmnote{ \bfseries(#3)}}%%% Thm head spec
\theoremstyle{SlantTheorem}
\newtheorem{question}{Question}
\counterwithin*{question}{section}



\let\instructorNotes\relax
\let\endinstructorNotes\relax
%%% instructorNotes environment
\ifhandout
\newenvironment{instructorNotes}[1][false]%
{%
\def\givenatend{\boolean{#1}}\ifthenelse{\boolean{#1}}{\begin{trivlist}\item}{\setbox0\vbox\bgroup}{}
}
{%
\ifthenelse{\givenatend}{\end{trivlist}}{\egroup}{}
}
\else
\newenvironment{instructorNotes}[1][false]%
{%
  \ifthenelse{\boolean{#1}}{\begin{trivlist}\item[\hskip \labelsep\bfseries {\Large Instructor Notes: \\} \hspace{\textwidth} ]}
{\begin{trivlist}\item[\hskip \labelsep\bfseries {\Large Instructor Notes: \\} \hspace{\textwidth} ]}
{}
}
{\end{trivlist}}
\fi


%% Suggested Timing
\newcommand{\timing}[1]{{\bf Suggested Timing: \hspace{2ex}} #1}

\title{Measuring Circles}
\author{Vic Ferdinand, Betsy McNeal, Jenny Sheldon}

\begin{document}
\begin{abstract}
\end{abstract}
\maketitle

\begin{instructorIntro}
This activity should bring out the value pi as the ratio between the circumference and the diameter.  Students seem to like this discussion, because it feels like a lesson they could do with their own students someday.

Bring lots of different-sized cylinders and tape measures.  Pass them out, and have the students pass them around as they need more cylinders to measure.  After students have made their own charts, try to collect as much data on the board as possible.

\timing{About 5-10 minutes to measure the cylinders and record the results in a table, then about 15 minutes of discussion.}
\end{instructorIntro}

\begin{problem} 
Measure the circumference and diameter of at least four different circles, and record your observations in the table below.
\vskip 0.1in
\begin{tabular}{|p{2.2in}|p{2.2in}|}
\hline
\begin{center} Circumference \end{center} & \begin{center} Diameter \end{center}  \\ \hline
 & \\ [5ex] \hline
 & \\ [5 ex] \hline
 & \\ [5 ex] \hline
 & \\ [5 ex] \hline
 & \\ [5 ex] \hline
\end{tabular}
\end{problem}

\begin{problem}
Make as many observations about your table as possible.  As you make observations, make conjectures as to why your observations might hold.
\begin{instructorNotes}
Students should point out here that $C/d$ seems to be pretty much staying the same, and the value is approximately 3 in most cases.  There will be measurement error, which is a good thing to discuss, and even perhaps significant outliers (indicating incorrect measurements).  When you talk about why this relationship should hold, it's nice to bring up the idea that each circle is similar to every other, and then bring this up again when you discuss what similarity actually is.
\end{instructorNotes}
\end{problem}


\end{document}
