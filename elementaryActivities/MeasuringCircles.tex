\documentclass[nooutcomes]{ximera}
\usepackage{gensymb}
\usepackage{tabularx}
\usepackage{mdframed}
\usepackage{pdfpages}
%\usepackage{chngcntr}

\let\problem\relax
\let\endproblem\relax

\newcommand{\property}[2]{#1#2}




\newtheoremstyle{SlantTheorem}{\topsep}{\fill}%%% space between body and thm
 {\slshape}                      %%% Thm body font
 {}                              %%% Indent amount (empty = no indent)
 {\bfseries\sffamily}            %%% Thm head font
 {}                              %%% Punctuation after thm head
 {3ex}                           %%% Space after thm head
 {\thmname{#1}\thmnumber{ #2}\thmnote{ \bfseries(#3)}} %%% Thm head spec
\theoremstyle{SlantTheorem}
\newtheorem{problem}{Problem}[]

%\counterwithin*{problem}{section}



%%%%%%%%%%%%%%%%%%%%%%%%%%%%Jenny's code%%%%%%%%%%%%%%%%%%%%

%%% Solution environment
%\newenvironment{solution}{
%\ifhandout\setbox0\vbox\bgroup\else
%\begin{trivlist}\item[\hskip \labelsep\small\itshape\bfseries Solution\hspace{2ex}]
%\par\noindent\upshape\small
%\fi}
%{\ifhandout\egroup\else
%\end{trivlist}
%\fi}
%
%
%%% instructorIntro environment
%\ifhandout
%\newenvironment{instructorIntro}[1][false]%
%{%
%\def\givenatend{\boolean{#1}}\ifthenelse{\boolean{#1}}{\begin{trivlist}\item}{\setbox0\vbox\bgroup}{}
%}
%{%
%\ifthenelse{\givenatend}{\end{trivlist}}{\egroup}{}
%}
%\else
%\newenvironment{instructorIntro}[1][false]%
%{%
%  \ifthenelse{\boolean{#1}}{\begin{trivlist}\item[\hskip \labelsep\bfseries Instructor Notes:\hspace{2ex}]}
%{\begin{trivlist}\item[\hskip \labelsep\bfseries Instructor Notes:\hspace{2ex}]}
%{}
%}
%% %% line at the bottom} 
%{\end{trivlist}\par\addvspace{.5ex}\nobreak\noindent\hung} 
%\fi
%
%


\let\instructorNotes\relax
\let\endinstructorNotes\relax
%%% instructorNotes environment
\ifhandout
\newenvironment{instructorNotes}[1][false]%
{%
\def\givenatend{\boolean{#1}}\ifthenelse{\boolean{#1}}{\begin{trivlist}\item}{\setbox0\vbox\bgroup}{}
}
{%
\ifthenelse{\givenatend}{\end{trivlist}}{\egroup}{}
}
\else
\newenvironment{instructorNotes}[1][false]%
{%
  \ifthenelse{\boolean{#1}}{\begin{trivlist}\item[\hskip \labelsep\bfseries {\Large Instructor Notes: \\} \hspace{\textwidth} ]}
{\begin{trivlist}\item[\hskip \labelsep\bfseries {\Large Instructor Notes: \\} \hspace{\textwidth} ]}
{}
}
{\end{trivlist}}
\fi


%% Suggested Timing
\newcommand{\timing}[1]{{\bf Suggested Timing: \hspace{2ex}} #1}




\hypersetup{
    colorlinks=true,       % false: boxed links; true: colored links
    linkcolor=blue,          % color of internal links (change box color with linkbordercolor)
    citecolor=green,        % color of links to bibliography
    filecolor=magenta,      % color of file links
    urlcolor=cyan           % color of external links
}

\title{Measuring Circles}
\author{Vic Ferdinand, Betsy McNeal, Jenny Sheldon}

\begin{document}
\begin{abstract}
\end{abstract}
\maketitle



\begin{problem} 
Measure the circumference and diameter of at least four different circles, and record your observations in the table below.
\vskip 0.1in
\begin{tabular}{|p{2.2in}|p{2.2in}|}
\hline
\begin{center} Circumference \end{center} & \begin{center} Diameter \end{center}  \\ \hline
 & \\ [5ex] \hline
 & \\ [5 ex] \hline
 & \\ [5 ex] \hline
 & \\ [5 ex] \hline
 & \\ [5 ex] \hline
\end{tabular}
\end{problem}

\begin{problem}
Make as many observations about your table as possible.  As you make observations, make conjectures as to why your observations might hold.

\end{problem}

\newpage
\begin{instructorNotes}
This activity is intended to help students see that the ratio of circumference to diameter in a circle is constant, and in discussion we name this value pi.  Students seem to like this discussion, because it feels like a lesson they could do with their own students someday.

This activity falls for us during a collection of activities about area.  We have already treated the meaning of area, and area formulas for polygons.  This activity transitions us to talking about circles, and we follow this activity by another in which we estimate the value of pi, and then an activity in which we develop the area formula for circles.

We bring lots of different-sized cylinders and tape measures, pass them out, and then have the students pass them around as they need more cylinders to measure.  Students also frequently measure circular objects they brought to class, like their water bottle.  This activity would also work if students used their compass to draw a circle, then cut it out and measured its circumference.  After students have made their own charts, we try to collect as much data on the board as possible.

Students typically point out here that $C/d$ seems to be pretty much staying the same, and the value is approximately 3 in most cases.  Students also bring up the issue of measurement error, and point out significant outliers (usually indicating incorrect measurements).  Rounding and choosing how many decimal places to report are also commonly brought up.  When we talk about why this relationship $C/d \approx 3$ should hold, we bring up the idea that each circle is similar to every other, and then later bring this up again when we discuss similarity.  We don't prove that this ratio is constant.




\timing{We spend about 5-10 minutes measuring the cylinders and recording the results in a table, then about 15 minutes in discussion.}
\end{instructorNotes}


\end{document}
