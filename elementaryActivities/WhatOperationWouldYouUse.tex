\documentclass{ximera}

\usepackage{gensymb}
\usepackage{tabularx}
\usepackage{mdframed}
\usepackage{pdfpages}
%\usepackage{chngcntr}

\let\problem\relax
\let\endproblem\relax

\newcommand{\property}[2]{#1#2}




\newtheoremstyle{SlantTheorem}{\topsep}{\fill}%%% space between body and thm
 {\slshape}                      %%% Thm body font
 {}                              %%% Indent amount (empty = no indent)
 {\bfseries\sffamily}            %%% Thm head font
 {}                              %%% Punctuation after thm head
 {3ex}                           %%% Space after thm head
 {\thmname{#1}\thmnumber{ #2}\thmnote{ \bfseries(#3)}} %%% Thm head spec
\theoremstyle{SlantTheorem}
\newtheorem{problem}{Problem}[]

%\counterwithin*{problem}{section}



%%%%%%%%%%%%%%%%%%%%%%%%%%%%Jenny's code%%%%%%%%%%%%%%%%%%%%

%%% Solution environment
%\newenvironment{solution}{
%\ifhandout\setbox0\vbox\bgroup\else
%\begin{trivlist}\item[\hskip \labelsep\small\itshape\bfseries Solution\hspace{2ex}]
%\par\noindent\upshape\small
%\fi}
%{\ifhandout\egroup\else
%\end{trivlist}
%\fi}
%
%
%%% instructorIntro environment
%\ifhandout
%\newenvironment{instructorIntro}[1][false]%
%{%
%\def\givenatend{\boolean{#1}}\ifthenelse{\boolean{#1}}{\begin{trivlist}\item}{\setbox0\vbox\bgroup}{}
%}
%{%
%\ifthenelse{\givenatend}{\end{trivlist}}{\egroup}{}
%}
%\else
%\newenvironment{instructorIntro}[1][false]%
%{%
%  \ifthenelse{\boolean{#1}}{\begin{trivlist}\item[\hskip \labelsep\bfseries Instructor Notes:\hspace{2ex}]}
%{\begin{trivlist}\item[\hskip \labelsep\bfseries Instructor Notes:\hspace{2ex}]}
%{}
%}
%% %% line at the bottom} 
%{\end{trivlist}\par\addvspace{.5ex}\nobreak\noindent\hung} 
%\fi
%
%


\let\instructorNotes\relax
\let\endinstructorNotes\relax
%%% instructorNotes environment
\ifhandout
\newenvironment{instructorNotes}[1][false]%
{%
\def\givenatend{\boolean{#1}}\ifthenelse{\boolean{#1}}{\begin{trivlist}\item}{\setbox0\vbox\bgroup}{}
}
{%
\ifthenelse{\givenatend}{\end{trivlist}}{\egroup}{}
}
\else
\newenvironment{instructorNotes}[1][false]%
{%
  \ifthenelse{\boolean{#1}}{\begin{trivlist}\item[\hskip \labelsep\bfseries {\Large Instructor Notes: \\} \hspace{\textwidth} ]}
{\begin{trivlist}\item[\hskip \labelsep\bfseries {\Large Instructor Notes: \\} \hspace{\textwidth} ]}
{}
}
{\end{trivlist}}
\fi


%% Suggested Timing
\newcommand{\timing}[1]{{\bf Suggested Timing: \hspace{2ex}} #1}




\hypersetup{
    colorlinks=true,       % false: boxed links; true: colored links
    linkcolor=blue,          % color of internal links (change box color with linkbordercolor)
    citecolor=green,        % color of links to bibliography
    filecolor=magenta,      % color of file links
    urlcolor=cyan           % color of external links
}


\title{What Operation Would You Use?}
\author{Vic Ferdinand, Betsy McNeal, Jenny Sheldon}

\begin{document}
\begin{abstract} \end{abstract}
\maketitle


\begin{problem}
For each of the story problems below, write an arithmetic expression that represents the problem, such as $13+4$.  Do NOT solve the problem.  For each problem, explain why your choice of operation is appropriate to the story situation.

\vskip 0.1in

\begin{enumerate}\itemsep0.7in

	\item Sarah has $\frac34$ of a pound of jelly worms.  Katie has six times as many jelly worms as Sarah.  How many pounds of jelly worms does Katie have?
	
	\item Two and two thirds pans of brownies are left in the kitchen.  Johnny eats half of a pan of brownies.  How many pans of brownies remain?

	\item  If $\frac23$ of a pound of nails costs $\frac34$ of a dollar,  how many pounds of nails can you buy for one dollar?

	\item  A crew is building a road.  So far, the road is $\frac34$ mile long.  This is $\frac12$ the length that the road will be when it is finished.  How many miles long will the finished road be?

	\item  Bob has $\frac{9}{16}$ of a gallon of milk.  He gives all of his milk to Joe.  If Joe had $\frac38$ of a gallon of milk before, how much milk does Joe have now?
	
	\item Jessie has $\frac25$ of a ton of candy.  Lisa has $\frac13$ of a ton of candy.  How much more candy does Jessie have than Lisa?

	\item If $\frac12$ cup of flour makes 1 batch of cookies, then how many batches of cookies can you make with $\frac34$ cup of flour?
	
	\item Last year, Dr. McNeal bought $\frac47$ of a ton of candy for all her Math 1125 students.  However, that was only enough for $\frac23$ of her students.  How many tons of candy were necessary for all her students?

	\item  If I have $\frac13$ of a pizza left over from yesterday and I eat $\frac35$ of the leftover portion, how much of one whole pizza did I eat?
		
	\item Veronica's Halloween bag is $\frac27$ full.  Kathy has an identical Halloween bag, and it is $\frac14$ full.  If they combine their Halloween candy into one of the bags, how full will it be?

\end{enumerate}
\end{problem}
\newpage
\begin{instructorNotes}
This activity assumes that all operations (addition, subtraction, multiplication, and division) have been introduced.  The goal is for students to identify and justify an appropriate operation for solving the given problem. (There is sometimes more than one reasonable choice.) Their justification should be based on the structure of the story. It is not necessary for students to solve the problem or find the answer.  We have found this activity to be incredibly difficult for most students, as they don't have a hint about which operation to use from the course context or nearby course material.

{\bf Suggested Timing:} This activity will take an entire class period.  If we don't have time, we sometimes put this on an exam review or final review.
\end{instructorNotes}

\end{document}