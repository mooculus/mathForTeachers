\documentclass[nooutcomes]{ximera}
\usepackage{gensymb}
\usepackage{tabularx}
\usepackage{mdframed}
\usepackage{pdfpages}
%\usepackage{chngcntr}

\let\problem\relax
\let\endproblem\relax

\newcommand{\property}[2]{#1#2}




\newtheoremstyle{SlantTheorem}{\topsep}{\fill}%%% space between body and thm
 {\slshape}                      %%% Thm body font
 {}                              %%% Indent amount (empty = no indent)
 {\bfseries\sffamily}            %%% Thm head font
 {}                              %%% Punctuation after thm head
 {3ex}                           %%% Space after thm head
 {\thmname{#1}\thmnumber{ #2}\thmnote{ \bfseries(#3)}} %%% Thm head spec
\theoremstyle{SlantTheorem}
\newtheorem{problem}{Problem}[]

%\counterwithin*{problem}{section}



%%%%%%%%%%%%%%%%%%%%%%%%%%%%Jenny's code%%%%%%%%%%%%%%%%%%%%

%%% Solution environment
%\newenvironment{solution}{
%\ifhandout\setbox0\vbox\bgroup\else
%\begin{trivlist}\item[\hskip \labelsep\small\itshape\bfseries Solution\hspace{2ex}]
%\par\noindent\upshape\small
%\fi}
%{\ifhandout\egroup\else
%\end{trivlist}
%\fi}
%
%
%%% instructorIntro environment
%\ifhandout
%\newenvironment{instructorIntro}[1][false]%
%{%
%\def\givenatend{\boolean{#1}}\ifthenelse{\boolean{#1}}{\begin{trivlist}\item}{\setbox0\vbox\bgroup}{}
%}
%{%
%\ifthenelse{\givenatend}{\end{trivlist}}{\egroup}{}
%}
%\else
%\newenvironment{instructorIntro}[1][false]%
%{%
%  \ifthenelse{\boolean{#1}}{\begin{trivlist}\item[\hskip \labelsep\bfseries Instructor Notes:\hspace{2ex}]}
%{\begin{trivlist}\item[\hskip \labelsep\bfseries Instructor Notes:\hspace{2ex}]}
%{}
%}
%% %% line at the bottom} 
%{\end{trivlist}\par\addvspace{.5ex}\nobreak\noindent\hung} 
%\fi
%
%


\let\instructorNotes\relax
\let\endinstructorNotes\relax
%%% instructorNotes environment
\ifhandout
\newenvironment{instructorNotes}[1][false]%
{%
\def\givenatend{\boolean{#1}}\ifthenelse{\boolean{#1}}{\begin{trivlist}\item}{\setbox0\vbox\bgroup}{}
}
{%
\ifthenelse{\givenatend}{\end{trivlist}}{\egroup}{}
}
\else
\newenvironment{instructorNotes}[1][false]%
{%
  \ifthenelse{\boolean{#1}}{\begin{trivlist}\item[\hskip \labelsep\bfseries {\Large Instructor Notes: \\} \hspace{\textwidth} ]}
{\begin{trivlist}\item[\hskip \labelsep\bfseries {\Large Instructor Notes: \\} \hspace{\textwidth} ]}
{}
}
{\end{trivlist}}
\fi


%% Suggested Timing
\newcommand{\timing}[1]{{\bf Suggested Timing: \hspace{2ex}} #1}




\hypersetup{
    colorlinks=true,       % false: boxed links; true: colored links
    linkcolor=blue,          % color of internal links (change box color with linkbordercolor)
    citecolor=green,        % color of links to bibliography
    filecolor=magenta,      % color of file links
    urlcolor=cyan           % color of external links
}

\title{Property Investigation}
\author{Vic Ferdinand, Betsy McNeal, Jenny Sheldon}

\begin{document}
\begin{abstract}\end{abstract}
\maketitle



\begin{problem} Should we expect that properties that hold for all squares should also hold for all rectangles?  Should we expect that properties that hold for all rectangles should also hold for all squares?  Why or why not? 
\end{problem}



\begin{problem} \label{PropInv2}
Some of our special quadrilaterals are special cases of others.  In each of the following, if it is possible, fill in the blanks to make a true statement of:

\begin{center} \underline{\hspace{1in}} is a special case of \underline{\hspace{1in}}. \end{center}

For example, if you are given ``food'' and ``ice cream'', ``Ice cream is a special case of food" is a true statement (but the converse is not).  Explain your reasoning on why your statement is true but the converse is not.  Try to identify where you're using the definition of the quadrilateral, and where you're using a property of the quadrilateral.
\begin{enumerate}
\item  Animal, puppy
\item  Square, rectangle
\item Rhombus, square
\item  Rhombus, parallelogram
\item  Kite, rhombus
\end{enumerate}

For parts \ref{FabFourPartf} through \ref{FabFourParth}, fill in the blanks for the following statement:

\begin{center} \underline{\hspace{0.5in}} is a special case of \underline{\hspace{0.5in}}, which is a special case of \underline{\hspace{0.5in}}. \end{center}

\begin{enumerate}
\setcounter{enumi}{5}
    \item Dessert, food, ice cream \label{FabFourPartf}
    \item  Rhombus, kite, square
    \item Rectangle, parallelogram, quadrilateral \label{FabFourParth}
\end{enumerate}
\end{problem}




\newpage
\begin{problem}  
Of squares, rhombuses, rectangles, parallelograms, kites, trapezoids, and isosceles trapezoids, 
\begin{enumerate}
    \item   Which are special cases of rectangles?
    \item Which are special cases of parallelograms? 
\end{enumerate}

Try to identify where you're using the definition of the quadrilateral, and where you're using a property of the quadrilateral.
\end{problem}




\begin{problem} If a line is a bisector of a segment, is that line also a perpendicular bisector of that segment?  If a line is perpendicular to a segment, is that line also a perpendicular bisector of that segment? 





\end{problem}


\newpage
\begin{instructorNotes}
The goal in this activity is to help students investigate the relationships between shapes and understand which shapes are special cases of which other shapes.  Generally, this activity treats triangles and quadrilaterals, and assumes that special quadrilaterals have already been defined.  The activity also comes after discussion of parallelism and the Parallel Postulate in our course.  Once these relationships are established, we follow this activity by talking about properties that automatically extend from one category to another (and properties that don't automatically extend!)  Properties of quadrilaterals can be discussed either before or after this activity.  Traditionally, this has been a difficult activity for our students because of the emphasis on logic.



\begin{itemize}
    \item Throughout this activity, we have found it helpful to suggest that students draw some more examples of specific types of quadrilaterals - particularly ``extreme cases'' like very short and very long sides on a rectangle.
    \item In our course, we don't have time to discuss the triangle congruence theorems. For this reason, some of the properties of quadrilaterals have to be essentially just observed rather than proven.  This can mean that students have a more difficult time remembering some of the properties.  For instance, a rhombus is a special case of a parallelogram, but this is not at all obvious without using triangle congruence theorems.
    \item We have seen students confuse ``rhombus'' with ``rectangle'' when working through these problems.
    \item In Problem \ref{PropInv2}, we often have students askwhether they have to keep the words in order and answer whether the statement is true or false.  
    \item When discussing Problem \ref{PropInv2}, we compare and contrast the statements ``All (blank) are (blank)'' and ``Some (blank) are (blank)''.
    \item The final problem is included because it is one of the most common incorrect statements that we see students make.
    \item This activity also provides us the opportunity to talk to students about how in general, it's easier to prove that something doesn't have the property we want than that all such shapes have the property.  
    \item If we have extra time, we sometimes introduce Venn Diagrams during this activity.
\end{itemize}

\timing{This activity takes us about a half of a class period.  We give students about 10 minutes to think about these problems in small groups, and then discuss as a whole class for around 20 minutes.}
\end{instructorNotes}




\end{document}