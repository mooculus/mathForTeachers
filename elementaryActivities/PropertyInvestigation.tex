\documentclass[handout]{ximera}

\graphicspath{
  {./}
  {graphics/}
  {../graphics/}
}

\usepackage{chngcntr}

\let\question\relax
\let\endquestion\relax




\newtheoremstyle{SlantTheorem}{\topsep}{\fill}%%% space between body and thm
%\newtheoremstyle{SlantTheorem}{\topsep}{\topsep}%%% space between body and thm
 {\slshape}                      %%% Thm body font
 {}                              %%% Indent amount (empty = no indent)
 {\bfseries\sffamily}            %%% Thm head font
 {}                              %%% Punctuation after thm head
 {3ex}                           %%% Space after thm head
 {\thmname{#1}\thmnumber{ #2}\thmnote{ \bfseries(#3)}}%%% Thm head spec
\theoremstyle{SlantTheorem}
\newtheorem{question}{Question}
\counterwithin*{question}{section}



\let\instructorNotes\relax
\let\endinstructorNotes\relax
%%% instructorNotes environment
\ifhandout
\newenvironment{instructorNotes}[1][false]%
{%
\def\givenatend{\boolean{#1}}\ifthenelse{\boolean{#1}}{\begin{trivlist}\item}{\setbox0\vbox\bgroup}{}
}
{%
\ifthenelse{\givenatend}{\end{trivlist}}{\egroup}{}
}
\else
\newenvironment{instructorNotes}[1][false]%
{%
  \ifthenelse{\boolean{#1}}{\begin{trivlist}\item[\hskip \labelsep\bfseries {\Large Instructor Notes: \\} \hspace{\textwidth} ]}
{\begin{trivlist}\item[\hskip \labelsep\bfseries {\Large Instructor Notes: \\} \hspace{\textwidth} ]}
{}
}
{\end{trivlist}}
\fi


%% Suggested Timing
\newcommand{\timing}[1]{{\bf Suggested Timing: \hspace{2ex}} #1}

\title{Property Investigation}
\author{Vic Ferdinand \& Betsy McNeal \& Jenny Sheldon}

\begin{document}
\begin{abstract}\end{abstract}
\maketitle

\begin{instructorIntro}
The goal in this activity is to help students investigate the relationships between shapes, understanding which shapes are special cases of which other shapes.  Once these relationships are settled, we can talk about properties that automatically extend from one category to another (and properties that don't automatically extend!)  This is a tough activity, since it deals with some logic which is difficult for many students.
\end{instructorIntro}

\begin{problem} Should we expect that properties that hold for all squares should also hold for all rectangles?  Should we expect that properties that hold for all rectangles should also hold for all squares?  Why or why not? 

\begin{solution}
The answer to the first question is no, and to the second question is yes.
\end{solution}

\begin{instructorNotes}
\begin{enumerate}
\item You might have students draw some more examples of specific types of quadrilaterals - particularly ``extreme cases'' like very short and very long sides on a rectangle.  This can help to determine whether every rectangle has certain properties, for instance.
\item This might also be a good time to discuss that in general, it's easier to prove that something doesn't have the property we want than that all such shapes have the property.  This seems a little backwards, but is a good reminder for students to always try some specific examples.
\end{enumerate}
\end{instructorNotes}
\end{problem}

\begin{problem}
Some of our special quadrilaterals are special cases of others.  In each of the following, if it is possible, fill in the blanks to make a true statement of:

\begin{center} \underline{\hspace{1in}} is a special case of \underline{\hspace{1in}}. \end{center}

For example, if you are given ``food'' and ``ice cream'', ``Ice cream is a special case of food" is a true statement (but the converse is not).  Explain your reasoning on why your statement is true but the converse is not.  Try to identify where you're using the definition of the quadrilateral, and where you're using a property of the quadrilateral.
\begin{enumerate}
\item  Animal, puppy
\item  Square, rectangle
\item Rhombus, square
\item  Rhombus, parallelogram
\item  Kite, rhombus
\end{enumerate}

For parts \ref{FabFourPartf} through \ref{FabFourParth}, fill in the blanks for the following statement:

\begin{center} \underline{\hspace{0.5in}} is a special case of \underline{\hspace{0.5in}}, which is a special case of \underline{\hspace{0.5in}}. \end{center}

\begin{enumerate}
\setcounter{enumi}{5}
    \item Dessert, food, ice cream \label{FabFourPartf}
    \item  Rhombus, kite, square
    \item Rectangle, parallelogram, quadrilateral \label{FabFourParth}
\end{enumerate}

\begin{solution}
\begin{enumerate}
\item Puppy is a special case of animal.
\item Square is a special case of rectangle.
\item Square is a special case of rhombus.
\item Rhombus is a special case of parallelogram.
\item Rhombus is a special case of kite.
\item Ice cream is a special case of dessert, which is a special case of food.
\item Square is a special case of rhombus, which is a special case of kite.
\item Rectangle is a special case of parallelogram, which is a special case of quadrilateral.
\end{enumerate}
\end{solution}

\begin{instructorNotes}
These notes pertain to the next problem as well.
\begin{enumerate}
\item Some students think that they have to keep the words in order and answer whether the statement is true or false.  
\item It's easy for students to get confused between a rectangle and a rhombus.  You might have them draw some more examples on their own if they're facing this confusion.
\item You should point out that a rhombus is a special case of a parallelogram, even though this is not obvious from the definitions.  Have the class discuss why this statement must be true.
\item You might discuss the statement: ``All (blank) are (blank)'' as well as ``Some (blank) are (blank)'' and compare and contrast the results.
\item You might introduce Venn Diagrams during this discussion if you have time.
\end{enumerate}
\end{instructorNotes}
\end{problem}

\newpage
\begin{problem}  
Of squares, rhombuses, rectangles, parallelograms, kites, trapezoids, and isosceles trapezoids, 
\begin{enumerate}
    \item   Which are special cases of rectangles?
    \item Which are special cases of parallelograms? 
\end{enumerate}

Try to identify where you're using the definition of the quadrilateral, and where you're using a property of the quadrilateral.

\begin{solution}
\begin{enumerate}
\item Squares (and technically rectangles are a special case of themselves, if you want!)
\item Squares, Rhombuses, rectangles (and technically parallelograms!)
\end{enumerate}
\end{solution}

\begin{instructorNotes}
    See the notes for the previous problem.
\end{instructorNotes}
\end{problem}




\begin{problem} If a line is a bisector of a segment, is that line also a perpendicular bisector of that segment?  If a line is perpendicular to a segment, is that line also a perpendicular bisector of that segment? 


\begin{solution}
The answer to both questions is no.
\end{solution}

\begin{instructorNotes}
    This problem is included because it is one of the most common incorrect statements that students make.
\end{instructorNotes}
\end{problem}



\end{document}