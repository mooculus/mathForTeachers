\documentclass[nooutcomes]{ximera}

\usepackage{gensymb}
\usepackage{tabularx}
\usepackage{mdframed}
\usepackage{pdfpages}
%\usepackage{chngcntr}

\let\problem\relax
\let\endproblem\relax

\newcommand{\property}[2]{#1#2}




\newtheoremstyle{SlantTheorem}{\topsep}{\fill}%%% space between body and thm
 {\slshape}                      %%% Thm body font
 {}                              %%% Indent amount (empty = no indent)
 {\bfseries\sffamily}            %%% Thm head font
 {}                              %%% Punctuation after thm head
 {3ex}                           %%% Space after thm head
 {\thmname{#1}\thmnumber{ #2}\thmnote{ \bfseries(#3)}} %%% Thm head spec
\theoremstyle{SlantTheorem}
\newtheorem{problem}{Problem}[]

%\counterwithin*{problem}{section}



%%%%%%%%%%%%%%%%%%%%%%%%%%%%Jenny's code%%%%%%%%%%%%%%%%%%%%

%%% Solution environment
%\newenvironment{solution}{
%\ifhandout\setbox0\vbox\bgroup\else
%\begin{trivlist}\item[\hskip \labelsep\small\itshape\bfseries Solution\hspace{2ex}]
%\par\noindent\upshape\small
%\fi}
%{\ifhandout\egroup\else
%\end{trivlist}
%\fi}
%
%
%%% instructorIntro environment
%\ifhandout
%\newenvironment{instructorIntro}[1][false]%
%{%
%\def\givenatend{\boolean{#1}}\ifthenelse{\boolean{#1}}{\begin{trivlist}\item}{\setbox0\vbox\bgroup}{}
%}
%{%
%\ifthenelse{\givenatend}{\end{trivlist}}{\egroup}{}
%}
%\else
%\newenvironment{instructorIntro}[1][false]%
%{%
%  \ifthenelse{\boolean{#1}}{\begin{trivlist}\item[\hskip \labelsep\bfseries Instructor Notes:\hspace{2ex}]}
%{\begin{trivlist}\item[\hskip \labelsep\bfseries Instructor Notes:\hspace{2ex}]}
%{}
%}
%% %% line at the bottom} 
%{\end{trivlist}\par\addvspace{.5ex}\nobreak\noindent\hung} 
%\fi
%
%


\let\instructorNotes\relax
\let\endinstructorNotes\relax
%%% instructorNotes environment
\ifhandout
\newenvironment{instructorNotes}[1][false]%
{%
\def\givenatend{\boolean{#1}}\ifthenelse{\boolean{#1}}{\begin{trivlist}\item}{\setbox0\vbox\bgroup}{}
}
{%
\ifthenelse{\givenatend}{\end{trivlist}}{\egroup}{}
}
\else
\newenvironment{instructorNotes}[1][false]%
{%
  \ifthenelse{\boolean{#1}}{\begin{trivlist}\item[\hskip \labelsep\bfseries {\Large Instructor Notes: \\} \hspace{\textwidth} ]}
{\begin{trivlist}\item[\hskip \labelsep\bfseries {\Large Instructor Notes: \\} \hspace{\textwidth} ]}
{}
}
{\end{trivlist}}
\fi


%% Suggested Timing
\newcommand{\timing}[1]{{\bf Suggested Timing: \hspace{2ex}} #1}




\hypersetup{
    colorlinks=true,       % false: boxed links; true: colored links
    linkcolor=blue,          % color of internal links (change box color with linkbordercolor)
    citecolor=green,        % color of links to bibliography
    filecolor=magenta,      % color of file links
    urlcolor=cyan           % color of external links
}


\title{Prime Numbers From 1 to 120}
\author{Vic Ferdinand, Betsy McNeal, Jenny Sheldon}

\begin{document}
\begin{abstract} 
\end{abstract}
\maketitle



\begin{problem}
Although you may know many of the prime numbers from 1 to 120, figure
out a way to find them systematically  \textit{without dividing}!

\vspace*{1cm}

\begin{tabular}{r r r r r r r r r r}

  1 &   2 &   3 &   4 &   5 &   6 &   7 &   8 &   9 &  10\\
  \\
 11 &  12 &  13 &  14 &  15 &  16 &  17 &  18 &  19 &  20\\
 \\
 21 &  22 &  23 &  24 &  25 &  26 &  27 &  28 &  29 &  30\\
 \\
 31 &  32 &  33 &  34 &  35 &  36 &  37 &  38 &  39 &  40\\
 \\
 41 &  42 &  43 &  44 &  45 &  46 &  47 &  48 &  49 &  50\\
 \\
 51 &  52 &  53 &  54 &  55 &  56 &  57 &  58 &  59 &  60\\
 \\
 61 &  62 &  63 &  64 &  65 &  66 &  67 &  68 &  69 &  70\\
 \\
 71 &  72 &  73 &  74 &  75 &  76 &  77 &  78 &  79 &  80\\
 \\
 81 &  82 &  83 &  84 &  85 &  86 &  87 &  88 &  89 &  90\\
 \\
 91 &  92 &  93 &  94 &  95 &  96 &  97 &  98 &  99 & 100\\
 \\
101 & 102 & 103 & 104 & 105 & 106 & 107 & 108 & 109 & 110\\
\\
111 & 112 & 113 & 114 & 115 & 116 & 117 & 118 & 119 & 120\\
\end{tabular}

\end{problem}
\begin{problem}
 Could your method work to find all of the primes to 320?
\end{problem}



\begin{problem}
Consider the numbers 239, 323, and 4001. Are any of these prime? How
can you be sure?
\end{problem}


\newpage
\begin{instructorNotes}
The goal of this activity is to introduce the prime numbers, and to discuss the ``Sieve of Eratosthenes'' for finding prime numbers.

Also, this activity can be used to discuss quick methods for determining whether a number is prime - checking only prime factors, and checking up to the square root of the number - if you have time for the final problem. As with any time we introduce a ``rule'' for doing something, we try to help students to be able to explain the reasoning behind the rule.  For that reason, we often spend a significant amount of time with this second question when we choose to discuss it.

{\bf Suggested Timing:} We typically give students approximately 5-10 minutes to come up with ideas about how to systematically find prime numbers, then spend about 15 minutes having them describe their methods and discussing whether they could help find all prime numbers.  We then give students another 5-10 minutes to think about the final problem, and 15 minutes for discussion.
\end{instructorNotes}


\end{document}