\documentclass[nooutcomes]{ximera}
\usepackage{gensymb}
\usepackage{tabularx}
\usepackage{mdframed}
\usepackage{pdfpages}
%\usepackage{chngcntr}

\let\problem\relax
\let\endproblem\relax

\newcommand{\property}[2]{#1#2}




\newtheoremstyle{SlantTheorem}{\topsep}{\fill}%%% space between body and thm
 {\slshape}                      %%% Thm body font
 {}                              %%% Indent amount (empty = no indent)
 {\bfseries\sffamily}            %%% Thm head font
 {}                              %%% Punctuation after thm head
 {3ex}                           %%% Space after thm head
 {\thmname{#1}\thmnumber{ #2}\thmnote{ \bfseries(#3)}} %%% Thm head spec
\theoremstyle{SlantTheorem}
\newtheorem{problem}{Problem}[]

%\counterwithin*{problem}{section}



%%%%%%%%%%%%%%%%%%%%%%%%%%%%Jenny's code%%%%%%%%%%%%%%%%%%%%

%%% Solution environment
%\newenvironment{solution}{
%\ifhandout\setbox0\vbox\bgroup\else
%\begin{trivlist}\item[\hskip \labelsep\small\itshape\bfseries Solution\hspace{2ex}]
%\par\noindent\upshape\small
%\fi}
%{\ifhandout\egroup\else
%\end{trivlist}
%\fi}
%
%
%%% instructorIntro environment
%\ifhandout
%\newenvironment{instructorIntro}[1][false]%
%{%
%\def\givenatend{\boolean{#1}}\ifthenelse{\boolean{#1}}{\begin{trivlist}\item}{\setbox0\vbox\bgroup}{}
%}
%{%
%\ifthenelse{\givenatend}{\end{trivlist}}{\egroup}{}
%}
%\else
%\newenvironment{instructorIntro}[1][false]%
%{%
%  \ifthenelse{\boolean{#1}}{\begin{trivlist}\item[\hskip \labelsep\bfseries Instructor Notes:\hspace{2ex}]}
%{\begin{trivlist}\item[\hskip \labelsep\bfseries Instructor Notes:\hspace{2ex}]}
%{}
%}
%% %% line at the bottom} 
%{\end{trivlist}\par\addvspace{.5ex}\nobreak\noindent\hung} 
%\fi
%
%


\let\instructorNotes\relax
\let\endinstructorNotes\relax
%%% instructorNotes environment
\ifhandout
\newenvironment{instructorNotes}[1][false]%
{%
\def\givenatend{\boolean{#1}}\ifthenelse{\boolean{#1}}{\begin{trivlist}\item}{\setbox0\vbox\bgroup}{}
}
{%
\ifthenelse{\givenatend}{\end{trivlist}}{\egroup}{}
}
\else
\newenvironment{instructorNotes}[1][false]%
{%
  \ifthenelse{\boolean{#1}}{\begin{trivlist}\item[\hskip \labelsep\bfseries {\Large Instructor Notes: \\} \hspace{\textwidth} ]}
{\begin{trivlist}\item[\hskip \labelsep\bfseries {\Large Instructor Notes: \\} \hspace{\textwidth} ]}
{}
}
{\end{trivlist}}
\fi


%% Suggested Timing
\newcommand{\timing}[1]{{\bf Suggested Timing: \hspace{2ex}} #1}




\hypersetup{
    colorlinks=true,       % false: boxed links; true: colored links
    linkcolor=blue,          % color of internal links (change box color with linkbordercolor)
    citecolor=green,        % color of links to bibliography
    filecolor=magenta,      % color of file links
    urlcolor=cyan           % color of external links
}
\title{Construction Zone}

\author{Vic Ferdinand, Betsy McNeal, Jenny Sheldon}

\begin{document}
\begin{abstract}
\end{abstract}
\maketitle



In the problems below, create the given shape.  You may only use the tools that are indicated in the problem.  After you've created the shape, explain how you know that your creation is exactly correct.  You should refer to the appropriate definitions!

\begin{problem}
Create an isosceles triangle.
\begin{enumerate}
\item You may fold paper and use scissors.
\item You may use your compass and ruler.
\end{enumerate}
\end{problem}

\begin{problem}
Create an equilateral triangle.
\begin{enumerate}

\item You may use your compass and/or ruler.
\item You may use your compass and/or ruler, but you may not use the markings on your ruler.
\end{enumerate}
\end{problem}

\begin{problem}
Create a rhombus.
\begin{enumerate}
\item You may fold paper and use scissors.
\item You may use your compass and ruler.
\item You may use your compass and ruler, but you may not use the markings on your ruler.
\end{enumerate}
\end{problem}

\begin{problem}
Draw any line segment.  Create a perpendicular bisector of that segment.
\begin{enumerate}
\item You may fold paper.
\item You may use your compass and ruler, but you may not use the markings on your ruler.
\end{enumerate}
\end{problem}

\begin{problem}
Create a rectangle, other than the rectangular piece of paper that you are using.
\begin{enumerate}
\item You may fold paper and use your scissors.
\item You may use your compass and ruler, but you may not use the markings on your ruler.
\end{enumerate}
\end{problem}

\newpage
\begin{instructorNotes}
In this activity, students learn several methods for constructing various triangles and quadrilaterals.  In our course, this activity comes after discussion of the definitions and properties of triangles and special quadrilaterals, as well as after discussing parallelism.  One goal we have for the activity is to help students learn to use the definitions and properties of the shapes in question as the guide to guarantee the shape is built.  This activity also gives students more practice with their tools (compass, protractor, and ruler/straightedge).  Students will need to pay attention to the attributes of the tools and which of those attributes can be taken advantage of to accomplish the goal, e.g., paper can be folded and the scissors can “double cut”.  Another important outcome of this activity is to give our students methods to reliably create these shapes, in case they would later like to make a ``classroom set'' for their future students.  Finally, this activity is another example of content where we have different levels of explanation.  We have brought together constructions of different types to illustrate these levels.  Some of the constructions are appropriate for young children, some for the middle grades, and some for high school.  Seeing various levels of explanation helps our students to see how the elementary content can set the stage for content in the later grades.



In our course, we do not have time to discuss the triangle congruence theorems, so we are not expecting students to directly use those theorems when explaining their answers.  Instead, we look for students to justify their answers based on properties of quadrilaterals and triangles that we have observed and stated (sometimes without proof).

    
    
\timing{We allot two class periods to this activity, working problem by problem and having students present their work at the board.  Each problem should take students 5-10 minutes in their small groups, and then about 15 minutes in whole-class discussion.  If we are very short on time, we split the problems up among the groups and have the groups present their constructions at the board.  This makes for shorter working time, but usually the discussion is longer this way.}    
\end{instructorNotes}




\end{document}