\documentclass{ximera}

\graphicspath{
  {./}
  {graphics/}
  {../graphics/}
}

\usepackage{chngcntr}

\let\question\relax
\let\endquestion\relax




\newtheoremstyle{SlantTheorem}{\topsep}{\fill}%%% space between body and thm
%\newtheoremstyle{SlantTheorem}{\topsep}{\topsep}%%% space between body and thm
 {\slshape}                      %%% Thm body font
 {}                              %%% Indent amount (empty = no indent)
 {\bfseries\sffamily}            %%% Thm head font
 {}                              %%% Punctuation after thm head
 {3ex}                           %%% Space after thm head
 {\thmname{#1}\thmnumber{ #2}\thmnote{ \bfseries(#3)}}%%% Thm head spec
\theoremstyle{SlantTheorem}
\newtheorem{question}{Question}
\counterwithin*{question}{section}



\let\instructorNotes\relax
\let\endinstructorNotes\relax
%%% instructorNotes environment
\ifhandout
\newenvironment{instructorNotes}[1][false]%
{%
\def\givenatend{\boolean{#1}}\ifthenelse{\boolean{#1}}{\begin{trivlist}\item}{\setbox0\vbox\bgroup}{}
}
{%
\ifthenelse{\givenatend}{\end{trivlist}}{\egroup}{}
}
\else
\newenvironment{instructorNotes}[1][false]%
{%
  \ifthenelse{\boolean{#1}}{\begin{trivlist}\item[\hskip \labelsep\bfseries {\Large Instructor Notes: \\} \hspace{\textwidth} ]}
{\begin{trivlist}\item[\hskip \labelsep\bfseries {\Large Instructor Notes: \\} \hspace{\textwidth} ]}
{}
}
{\end{trivlist}}
\fi


%% Suggested Timing
\newcommand{\timing}[1]{{\bf Suggested Timing: \hspace{2ex}} #1}
\title{Construction Zone}

\author{Vic Ferdinand \& Betsy McNeal \& Jenny Sheldon}

\begin{document}
\begin{abstract}
\end{abstract}
\maketitle

\begin{instructorIntro}
Here, we want the students to pay attention to the definitions of the shapes as the guide to guarantee the shape is built.  They also will need to pay attention to the attributes of the tools and which of those attributes can be taken advantage of to accomplish the goal, e.g., paper can be folded and the scissors can “double cut”.  A compass can generate all possible points a given distance from a fixed point:  Two of those arcs could intersect and produce a point with two attributes at the same time.   
     They could take advantage of properties to generate a given object, even though those properties have not been formally proven in this course, e.g., the diagonals of a rhombus bisect the vertex angles.  Be sure to emphasize that their methods need to be justified in one of these ways.
     
     Perhaps splitting the shapes into groups could save time and they could present their constructions at the board.
     
  There are some extension ideas at the end of the activity.
    
    This will probably take a couple days, as the analogous 10X and 10W did.
\end{instructorIntro}

In the problems below, create the given shape.  You may only use the tools that are indicated in the problem.  After you've created the shape, explain how you know that your creation is exactly correct.  You should refer to the appropriate definitions!

\begin{problem}
Create an isosceles triangle.
\begin{enumerate}
\item You may fold paper and use scissors.
\item You may use your compass and ruler.
\end{enumerate}
\end{problem}

\begin{problem}
Create an equilateral triangle.
\begin{enumerate}

\item You may use your compass and/or ruler.
\item You may use your compass and/or ruler, but you may not use the markings on your ruler.
\end{enumerate}
\end{problem}

\begin{problem}
Create a rhombus.
\begin{enumerate}
\item You may fold paper and use scissors.
\item You may use your compass and ruler.
\item You may use your compass and ruler, but you may not use the markings on your ruler.
\end{enumerate}
\end{problem}

\begin{problem}
Draw any line segment.  Create a perpendicular bisector of that segment.
\begin{enumerate}
\item You may fold paper.
\item You may use your compass and ruler, but you may not use the markings on your ruler.
\end{enumerate}
\end{problem}

\begin{problem}
Create a rectangle, other than the rectangular piece of paper that you are using.
\begin{enumerate}
\item You may fold paper and use your scissors.
\item You may use your compass and ruler, but you may not use the markings on your ruler.
\end{enumerate}
\end{problem}


\begin{instructorIntro}
Other Extensions:

+ Construct a 30-60-90 triangle using paper folding and compass (don't use the protractor!).

+ Fold to show a triangle with a perimeter that is a multiple of 3.

+ Fold to show a triangle with a side of length $\sqrt{2}$.

+ Using your ruler and your compass, construct a 3-5-7 triangle, a 3-6-2 triangle

+ Copy an angle using only compass and straightedge.
\end{instructorIntro}



\end{document}