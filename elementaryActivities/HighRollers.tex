\documentclass[nooutcomes]{ximera}
\usepackage{gensymb}
\usepackage{tabularx}
\usepackage{mdframed}
\usepackage{pdfpages}
%\usepackage{chngcntr}

\let\problem\relax
\let\endproblem\relax

\newcommand{\property}[2]{#1#2}




\newtheoremstyle{SlantTheorem}{\topsep}{\fill}%%% space between body and thm
 {\slshape}                      %%% Thm body font
 {}                              %%% Indent amount (empty = no indent)
 {\bfseries\sffamily}            %%% Thm head font
 {}                              %%% Punctuation after thm head
 {3ex}                           %%% Space after thm head
 {\thmname{#1}\thmnumber{ #2}\thmnote{ \bfseries(#3)}} %%% Thm head spec
\theoremstyle{SlantTheorem}
\newtheorem{problem}{Problem}[]

%\counterwithin*{problem}{section}



%%%%%%%%%%%%%%%%%%%%%%%%%%%%Jenny's code%%%%%%%%%%%%%%%%%%%%

%%% Solution environment
%\newenvironment{solution}{
%\ifhandout\setbox0\vbox\bgroup\else
%\begin{trivlist}\item[\hskip \labelsep\small\itshape\bfseries Solution\hspace{2ex}]
%\par\noindent\upshape\small
%\fi}
%{\ifhandout\egroup\else
%\end{trivlist}
%\fi}
%
%
%%% instructorIntro environment
%\ifhandout
%\newenvironment{instructorIntro}[1][false]%
%{%
%\def\givenatend{\boolean{#1}}\ifthenelse{\boolean{#1}}{\begin{trivlist}\item}{\setbox0\vbox\bgroup}{}
%}
%{%
%\ifthenelse{\givenatend}{\end{trivlist}}{\egroup}{}
%}
%\else
%\newenvironment{instructorIntro}[1][false]%
%{%
%  \ifthenelse{\boolean{#1}}{\begin{trivlist}\item[\hskip \labelsep\bfseries Instructor Notes:\hspace{2ex}]}
%{\begin{trivlist}\item[\hskip \labelsep\bfseries Instructor Notes:\hspace{2ex}]}
%{}
%}
%% %% line at the bottom} 
%{\end{trivlist}\par\addvspace{.5ex}\nobreak\noindent\hung} 
%\fi
%
%


\let\instructorNotes\relax
\let\endinstructorNotes\relax
%%% instructorNotes environment
\ifhandout
\newenvironment{instructorNotes}[1][false]%
{%
\def\givenatend{\boolean{#1}}\ifthenelse{\boolean{#1}}{\begin{trivlist}\item}{\setbox0\vbox\bgroup}{}
}
{%
\ifthenelse{\givenatend}{\end{trivlist}}{\egroup}{}
}
\else
\newenvironment{instructorNotes}[1][false]%
{%
  \ifthenelse{\boolean{#1}}{\begin{trivlist}\item[\hskip \labelsep\bfseries {\Large Instructor Notes: \\} \hspace{\textwidth} ]}
{\begin{trivlist}\item[\hskip \labelsep\bfseries {\Large Instructor Notes: \\} \hspace{\textwidth} ]}
{}
}
{\end{trivlist}}
\fi


%% Suggested Timing
\newcommand{\timing}[1]{{\bf Suggested Timing: \hspace{2ex}} #1}




\hypersetup{
    colorlinks=true,       % false: boxed links; true: colored links
    linkcolor=blue,          % color of internal links (change box color with linkbordercolor)
    citecolor=green,        % color of links to bibliography
    filecolor=magenta,      % color of file links
    urlcolor=cyan           % color of external links
}
\title{High Rollers}
\author{Vic Ferdinand, Betsy McNeal, Jenny Sheldon}

\begin{document}
\begin{abstract}
\end{abstract}
\maketitle



Suppose that you roll two six-sided dice: one is red, and the other is pink.  You are interested in the sum of the two dice.

\begin{problem}
How many different outcomes are possible when rolling these two dice?
\end{problem}

\begin{problem} 
Are you more likely to roll a sum of two or a sum of eight?  Or, are these two things equally likely?
\end{problem}

With a partner, roll a pair of dice at least 100 times, and record the sum of the two dice in a table.  You can physically roll the dice, or use an online dice roller like \url{http://www.roll-dice-online.com/} (which will roll them all at once) or \url{http://www.dicesimulator.com/} (where you can tally them as you roll).


\begin{problem}
Based on the table of results you just made, what is the probability of rolling
\begin{enumerate}
\item a sum of two?
\item a sum of seven?
\item a sum of eight?
\end{enumerate}
Were your earlier predictions correct?
\end{problem}

\begin{problem}
Using reasoning (not our data), what is the probability of rolling
\begin{enumerate}
\item a sum of two?
\item a sum of seven?
\item a sum of eight?
\end{enumerate}
How close are these values to the ones predicted by the chart?  How could we make them closer?
\end{problem}

\newpage
\begin{instructorNotes}
This activity is intended to introduce students to the notion of experimental probability, as well as to help students start making connections between the counting methods we discussed previously and their application to probability problems.

We use these activities about counting and probability to help cement the meaning of various operations for students.  Throughout these activities, we expect students to justify their operations of choice in their explanations.  There are a few main types of counting strategies we see from our students: arrays, ordered lists, tree diagrams, and algebraic expressions.  Students tend to want to use more complicated strategies before trying simpler strategies, so we are often reminding them to write down some examples, or to make sure their list is well organized. When discussing this activity as a whole class, we like to have students demonstrate multiple solution methods for each problem, even when not specifically requested by the problem.  We also continue to emphasize structure any time we can recognize a problem as being like another problem we have previously solved, as in ``Compare and Count''.

In our course, this activity follows our introduction to probability called ``Fun with Fractions'', so it assumes that students at least understand the basic meaning of probability and how to calculate probability in simple situations.  The problems in this activity are more difficult than those in the first activity.  We follow this activity with ``You Can Count On Probability'', an activity in which students get more practice solving probability problems.

In this activity, we have a discussion in the first two problems about the outcomes possible in this experiment as well as whether those outcomes are equally likely.  There are three important theories which come up (or we bring up if necessary).
\begin{itemize}
	\item There are 11 total possibilities (the sums themselves), each one equally likely.
	\item There are 21 total possibilities (smaller number, larger number), each one equally likely (sometimes we see this as 42 possible sums).
	\item There are 36 total possibilities (red die, pink die), each one equally likely.
\end{itemize}
The data produced in the final problems is intended to help students choose between these theories, especially once everyone's data is collected together.  We have found that it's not too difficult to look at several hundred or a thousand data points here, especially if we have the students do the dice rolling at home before they come to class, and then just collect their data on the board.

\timing{This activity is intended to take one class period.  We generally give students about 5-10 minutes to think about the situation and the first two problems, and then discuss theories as a whole class.  We then give students time to roll the dice and collect data.  If we are running out of time, we work through the last two problems as a whole class.}
\end{instructorNotes}



\end{document}