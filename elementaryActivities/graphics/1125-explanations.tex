\documentclass[nooutcomes, handout]{ximera}

\usepackage{tabularx}
\usepackage{mdframed}

\title{Explanations}

\begin{document}
\begin{abstract}
    We discuss explanations and how to write them.
\end{abstract}
\maketitle

{\bf Problem.} How many squares are in the following image?

\[
\includegraphics[height=0.75in]{graphics/squares.pdf}
\]

%Spend a few minutes thinking about this problem, and conjecture your own answer.  Then, imagine that you gave this problem to your class, and got the following four papers amongst the explanations given by your students.

\begin{question}
Did each member of your group come up with the same answer?  Did each member come to their answer in the same way?
\end{question}



\begin{question}
Pick one method and answer from your group.  Discuss this method in detail, as if you were presenting it to the class.  What ideas are the most important?  What steps are the most complicated?
\end{question}


\begin{question}
Now imagine that you are a teacher, and you have given this problem to your students.  On the back of this paper are four example explanations that you could get from your students.  With your group mates, compare and contrast these four explanations.  Discuss things you like about them, and things you don't like.
\end{question}

\pagebreak
\parbox{2.5in}{
\begin{mdframed}

{\bf Explanation 1:}
\[
\includegraphics[height=0.5in]{graphics/squaresSol1.pdf}
\]

I counted 9 squares in the picture, and labeled them on the figure.

\end{mdframed}}
\parbox{2in}{
\begin{mdframed}
{\bf Explanation 2:}

$$ 14 = 3\times 3 + 2\times 2 + 1 $$


\end{mdframed}}

%\parbox{5in}{
\begin{mdframed}
{\bf Explanation 3:}

First, I remember that to be a ``square'', you have to have the same length on both sides.  The easiest squares to see are the tiny ones, and I can see three rows with three tiny squares in each of those rows for a total of nine tiny squares.  The next easiest square to see is the biggest one: the entire picture is a square!  Lastly, there are some medium-sized squares.  Look at my picture to see an example.  I found four of these medium-sized squares: two with their tops on the top row, and two with their bottoms on the bottom row.  So, there are 14 total squares.

\[
\includegraphics[height=0.5in]{graphics/squaresSol3.pdf}
\]

I'm convinced these are all of the squares in the figure, because the tiny ones are $1\times 1$, the medium ones are $2 \times 2$, and the large one is $3\times 3$, which is the entire original figure.
\end{mdframed}
%}
%\parbox{2.25in}{

\begin{mdframed}
{\bf Explanation 4:}

\begin{tabularx}{\textwidth}{X|X}
    Step & Why  \\ \hline \hline
    We are looking for squares. & The problem tells us so. \\ \hline
    Squares have the same length on both sides. & That's what it means to be a square. \\ \hline
    There are 9 tiny squares. & We count the $1\times 1$ squares. \\ \hline
    There are 4 medium square. & We count the $2 \times 2$ squares. \\ \hline
    There is 1 big square. & We count the entire square. \\ \hline
    There are 14 squares total. & We add up the squares we found. \\
\end{tabularx}
\end{mdframed}
%}




\end{document}