\documentclass[11pt]{article}
\usepackage[margin=1in]{geometry}                % See geometry.pdf to learn the layout options. There are lots.
\geometry{letterpaper}                   % ... or a4paper or a5paper or ... 
%\geometry{landscape}                % Activate for for rotated page geometry
\usepackage[parfill]{parskip}    % Activate to begin paragraphs with an empty line rather than an indent
\usepackage{graphicx}
\usepackage{amssymb}
\usepackage{epstopdf}
\DeclareGraphicsRule{.tif}{png}{.png}{`convert #1 `dirname #1`/`basename #1 .tif`.png}

\title{Math 1125: Divisibility and Remainders}
%\author{The Author}
\date{}                                           % Activate to display a given date or no date

\begin{document}
\begin{center} \bf Math 1125: Divisibility and Remainders \rm \end{center}

1. Let's use $A$ to represent a number that has a remainder of 1 when you divide by $5$, and let's use $B$ to represent a number that has a remainder of 4 when you divide by $5$.

(a) Write down some possibilities for the numbers $A$ and $B$.

\vskip 0.3in

(b) What is the remainder when you calculate $(A + B) \div 5$?  Start by working with your examples, then make an argument about what should be true for any $A$ and $B$.

\vskip 2in

(c) What is the remainder when you calculate $(AB) \div 5$?  Start by working with your examples, then make an argument about what should be true for any $A$ and $B$.

\vskip 2in

2.  Now let's use $C$ to represent a number that has a remainder of 2 when you divide by $5$.

(a) Write down some possibilities for the numbers $C$.

\vskip 0.3in

(b)  What is the remainder when you calculate $(C + B) \div 5$?  

\vskip 2in

(c) What is the remainder when you calculate $(CB) \div 5$?

\vskip 2in

3.  Is there a general rule, here?  If so, what is it?  Describe it as accurately as you can, and explain why your rule should always hold.



\end{document}  