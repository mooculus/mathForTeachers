\documentclass{ximera}

\usepackage{gensymb}
\usepackage{tabularx}
\usepackage{mdframed}
\usepackage{pdfpages}
%\usepackage{chngcntr}

\let\problem\relax
\let\endproblem\relax

\newcommand{\property}[2]{#1#2}




\newtheoremstyle{SlantTheorem}{\topsep}{\fill}%%% space between body and thm
 {\slshape}                      %%% Thm body font
 {}                              %%% Indent amount (empty = no indent)
 {\bfseries\sffamily}            %%% Thm head font
 {}                              %%% Punctuation after thm head
 {3ex}                           %%% Space after thm head
 {\thmname{#1}\thmnumber{ #2}\thmnote{ \bfseries(#3)}} %%% Thm head spec
\theoremstyle{SlantTheorem}
\newtheorem{problem}{Problem}[]

%\counterwithin*{problem}{section}



%%%%%%%%%%%%%%%%%%%%%%%%%%%%Jenny's code%%%%%%%%%%%%%%%%%%%%

%%% Solution environment
%\newenvironment{solution}{
%\ifhandout\setbox0\vbox\bgroup\else
%\begin{trivlist}\item[\hskip \labelsep\small\itshape\bfseries Solution\hspace{2ex}]
%\par\noindent\upshape\small
%\fi}
%{\ifhandout\egroup\else
%\end{trivlist}
%\fi}
%
%
%%% instructorIntro environment
%\ifhandout
%\newenvironment{instructorIntro}[1][false]%
%{%
%\def\givenatend{\boolean{#1}}\ifthenelse{\boolean{#1}}{\begin{trivlist}\item}{\setbox0\vbox\bgroup}{}
%}
%{%
%\ifthenelse{\givenatend}{\end{trivlist}}{\egroup}{}
%}
%\else
%\newenvironment{instructorIntro}[1][false]%
%{%
%  \ifthenelse{\boolean{#1}}{\begin{trivlist}\item[\hskip \labelsep\bfseries Instructor Notes:\hspace{2ex}]}
%{\begin{trivlist}\item[\hskip \labelsep\bfseries Instructor Notes:\hspace{2ex}]}
%{}
%}
%% %% line at the bottom} 
%{\end{trivlist}\par\addvspace{.5ex}\nobreak\noindent\hung} 
%\fi
%
%


\let\instructorNotes\relax
\let\endinstructorNotes\relax
%%% instructorNotes environment
\ifhandout
\newenvironment{instructorNotes}[1][false]%
{%
\def\givenatend{\boolean{#1}}\ifthenelse{\boolean{#1}}{\begin{trivlist}\item}{\setbox0\vbox\bgroup}{}
}
{%
\ifthenelse{\givenatend}{\end{trivlist}}{\egroup}{}
}
\else
\newenvironment{instructorNotes}[1][false]%
{%
  \ifthenelse{\boolean{#1}}{\begin{trivlist}\item[\hskip \labelsep\bfseries {\Large Instructor Notes: \\} \hspace{\textwidth} ]}
{\begin{trivlist}\item[\hskip \labelsep\bfseries {\Large Instructor Notes: \\} \hspace{\textwidth} ]}
{}
}
{\end{trivlist}}
\fi


%% Suggested Timing
\newcommand{\timing}[1]{{\bf Suggested Timing: \hspace{2ex}} #1}




\hypersetup{
    colorlinks=true,       % false: boxed links; true: colored links
    linkcolor=blue,          % color of internal links (change box color with linkbordercolor)
    citecolor=green,        % color of links to bibliography
    filecolor=magenta,      % color of file links
    urlcolor=cyan           % color of external links
}


\title{The Shepherd's Necklace}
\author{Vic Ferdinand, Betsy McNeal, Robin Pemantle, Jenny Sheldon}

\begin{document}
\begin{abstract}\end{abstract}
\maketitle




\begin{problem} 
A small shepherd boy was responsible for taking 23 sheep to pasture each day.  These sheep would graze freely all day, and in the evening the boy was to round them all up and drive them home.  Unfortunately, the boy could not count, so each day he had difficulty deciding whether he had found all of the sheep before returning home. 

\begin{enumerate}

\item	Can you devise a method (other than teaching the boy to count) that would make it possible for him to be sure he had them all?  If so, describe it.







\vfill


TURN THE PAGE AFTER YOU HAVE COME UP WITH AN IDEA

\newpage
 
\item 	The sheep's owner came up with a simple scheme.  It involved making, for each sheep, a string necklace that would fit around the sheep's neck.  In the morning, the boy would take the necklaces and place them all around his own neck.  What should he do with them in the evening?
\vskip 1in

\item \label{ShepherdsNecklaceC} 	The ``necklace idea" spread throughout the region and soon all the young people tending sheep went to the meadows with string necklaces around their necks. One day, an argument arose between a shepherd and a shepherdess as to which was tending the largest number of sheep. Resolving the argument was made difficult since neither could count. Can you describe a method by which they could resolve their argument while leaving their sheep grazing undisturbed?
\vskip 1.5in

\item 	What is possible to determine about two sets if one does not know how to count?

\vskip 1.5in
\end{enumerate}
\end{problem}
\begin{problem} 
What does it mean to ``know how to count''?  How does it relate to the sheep and necklaces?  How could we define the number ``seven" if we knew no other quantities?

\end{problem}



\newpage
\begin{instructorNotes}

This activity brings up a situation in which a one-to-one correspondence is useful for comparing two sets.  We use this opportunity to define one-to-one correspondence, and show the importance and power of one-to-one correspondence in comparing two sets without the use of counting.  This is also the conceptual foundation for quantifying a set because quantifying or counting is just a one-to-one correspondence between the (ordered) number words and the objects.  The final question in the activity asks the students to try to define what it means to know how to count. 


Elicit a variety of ideas from the students about how to be sure all the sheep are accounted for.  You can point out that all of their ideas are possible answers to the problem posed and, in fact, they all have something in common - the idea of one-to-one correspondence between the sheep and something that is easier to track.  Order is not significant for this, that is, all sheep have returned if all colors, letters, or necklaces have been accounted for.


The situation in part \ref{ShepherdsNecklaceC} requires comparison of the necklaces in order to see who has more without disturbing the sheep.  So counting is not needed in order to determine greater, less than, or equal. It's good to define the term ``one-to-one correspondence'' sometime either here or after the next part, depending on your discussion.



The last question asks us to define ``seven''. To do this, you need a ``master set'' with seven objects in it to compare any other set (obviously, we're just working discretely here, not continuously - no Dedekind Cuts, please!).  ``Seven'' is the attribute that all the sets that match with the master set share (shaking out all other attributes in which they are different, such as size, type, color, etc. of the actual set members).  If there is time (or just to give as a ``thought question''), you might ask them to think about how we could define ``seven'' if we do not know anything about ``one'', ``two'', etc.  In essence, we are asking: ``What does it mean to count?''

This final part also provides a platform for discussion of the meaning of counting.  Relating the meaning of counting to children's development of the meaning of numbers is not part of this activity, so it must be a mini-lecture by the instructor.  
%When children first are learning to count, they mimic the number words (a social, not mathematical, construct) that others use.  Then children try to point and count objects, but sometimes miss or repeat some objects in their point-and-say scheme.  This is once again an imitation of a social tradition and children often first believe that there are ``fewer'' objects in a clumped-up pile than in a spread out set.  Then finally the child ``realizes'' the importance of a one-to-one correspondence between the words in the socially accepted order and the objects.  Children then conclude that the number of objects is the same as the last counting word spoken, but they may have the order incorrect and even, sometimes, restate the last number word as the total even if one of the objects counted earlier is removed.


{\bf Suggested Timing:} About 15 minutes in small groups to discuss the activity (encourage them to finish answering part (a) before turning the page).  Touring the room and checking out the variety of answers will help you see a variety of ideas to have students share with the whole group.  Discussing their ideas as a whole class usually takes about 20 minutes before summarizing the key points in counting.  



\end{instructorNotes}

\end{document}