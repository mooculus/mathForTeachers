\documentclass{ximera}

\graphicspath{
  {./}
  {graphics/}
  {../graphics/}
}

\usepackage{chngcntr}

\let\question\relax
\let\endquestion\relax




\newtheoremstyle{SlantTheorem}{\topsep}{\fill}%%% space between body and thm
%\newtheoremstyle{SlantTheorem}{\topsep}{\topsep}%%% space between body and thm
 {\slshape}                      %%% Thm body font
 {}                              %%% Indent amount (empty = no indent)
 {\bfseries\sffamily}            %%% Thm head font
 {}                              %%% Punctuation after thm head
 {3ex}                           %%% Space after thm head
 {\thmname{#1}\thmnumber{ #2}\thmnote{ \bfseries(#3)}}%%% Thm head spec
\theoremstyle{SlantTheorem}
\newtheorem{question}{Question}
\counterwithin*{question}{section}



\let\instructorNotes\relax
\let\endinstructorNotes\relax
%%% instructorNotes environment
\ifhandout
\newenvironment{instructorNotes}[1][false]%
{%
\def\givenatend{\boolean{#1}}\ifthenelse{\boolean{#1}}{\begin{trivlist}\item}{\setbox0\vbox\bgroup}{}
}
{%
\ifthenelse{\givenatend}{\end{trivlist}}{\egroup}{}
}
\else
\newenvironment{instructorNotes}[1][false]%
{%
  \ifthenelse{\boolean{#1}}{\begin{trivlist}\item[\hskip \labelsep\bfseries {\Large Instructor Notes: \\} \hspace{\textwidth} ]}
{\begin{trivlist}\item[\hskip \labelsep\bfseries {\Large Instructor Notes: \\} \hspace{\textwidth} ]}
{}
}
{\end{trivlist}}
\fi


%% Suggested Timing
\newcommand{\timing}[1]{{\bf Suggested Timing: \hspace{2ex}} #1}
\title{Compare And Count}
\author{Vic Ferdinand, Betsy McNeal, Robin Pemantle, Jenny Sheldon}

\begin{document}
\begin{abstract}
This activity includes several sets of problems.  First, solve each problem.  Then, compare and contrast the problems in each set.  Which problems are the same?  Which problems are different?
\end{abstract}
\maketitle

\begin{instructorIntro}
This activity is intended to give students a chance to compute basic permutations and combinations, and discuss the similarities and differences amongst these problems.
\end{instructorIntro}

\begin{problem}
There are 10 students in the ``Surf Lake Erie'' club.
\begin{enumerate}
\item In an election, how many ways are there to choose a President and a Vice President for the club?
\item At a bake sale, how many ways are there to choose two students to taste-test the pies? (Each of the testers tastes all of the pie.)
\item In an election, how many ways are there to choose a President, Vice President, Secretary, and Treasurer for the club?  (No student may hold two or more offices.)
\item How many ways are there to choose a squad of three students to wash the principal's car?

\end{enumerate}
\end{problem}

\begin{problem}
\begin{enumerate}
\item How many ways are there to choose three cookies off of a plate containing five (different) cookies?
\item How many ways are there to hand out first, second, and third place at the spelling bee if five students compete?
\item How many ways are there to choose a password using three different letters from a set of five?
\item How many ways are there to give a child three stickers out of a bag of five different stickers?
\end{enumerate}
\end{problem}


\begin{problem}
\begin{enumerate}
\item Write a story problem that has the same answer as 2(a).
\item Write a story problem that has the same answer as 2(b).
\end{enumerate}
\end{problem}


\end{document}