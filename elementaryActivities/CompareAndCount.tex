\documentclass{ximera}
\usepackage{gensymb}
\usepackage{tabularx}
\usepackage{mdframed}
\usepackage{pdfpages}
%\usepackage{chngcntr}

\let\problem\relax
\let\endproblem\relax

\newcommand{\property}[2]{#1#2}




\newtheoremstyle{SlantTheorem}{\topsep}{\fill}%%% space between body and thm
 {\slshape}                      %%% Thm body font
 {}                              %%% Indent amount (empty = no indent)
 {\bfseries\sffamily}            %%% Thm head font
 {}                              %%% Punctuation after thm head
 {3ex}                           %%% Space after thm head
 {\thmname{#1}\thmnumber{ #2}\thmnote{ \bfseries(#3)}} %%% Thm head spec
\theoremstyle{SlantTheorem}
\newtheorem{problem}{Problem}[]

%\counterwithin*{problem}{section}



%%%%%%%%%%%%%%%%%%%%%%%%%%%%Jenny's code%%%%%%%%%%%%%%%%%%%%

%%% Solution environment
%\newenvironment{solution}{
%\ifhandout\setbox0\vbox\bgroup\else
%\begin{trivlist}\item[\hskip \labelsep\small\itshape\bfseries Solution\hspace{2ex}]
%\par\noindent\upshape\small
%\fi}
%{\ifhandout\egroup\else
%\end{trivlist}
%\fi}
%
%
%%% instructorIntro environment
%\ifhandout
%\newenvironment{instructorIntro}[1][false]%
%{%
%\def\givenatend{\boolean{#1}}\ifthenelse{\boolean{#1}}{\begin{trivlist}\item}{\setbox0\vbox\bgroup}{}
%}
%{%
%\ifthenelse{\givenatend}{\end{trivlist}}{\egroup}{}
%}
%\else
%\newenvironment{instructorIntro}[1][false]%
%{%
%  \ifthenelse{\boolean{#1}}{\begin{trivlist}\item[\hskip \labelsep\bfseries Instructor Notes:\hspace{2ex}]}
%{\begin{trivlist}\item[\hskip \labelsep\bfseries Instructor Notes:\hspace{2ex}]}
%{}
%}
%% %% line at the bottom} 
%{\end{trivlist}\par\addvspace{.5ex}\nobreak\noindent\hung} 
%\fi
%
%


\let\instructorNotes\relax
\let\endinstructorNotes\relax
%%% instructorNotes environment
\ifhandout
\newenvironment{instructorNotes}[1][false]%
{%
\def\givenatend{\boolean{#1}}\ifthenelse{\boolean{#1}}{\begin{trivlist}\item}{\setbox0\vbox\bgroup}{}
}
{%
\ifthenelse{\givenatend}{\end{trivlist}}{\egroup}{}
}
\else
\newenvironment{instructorNotes}[1][false]%
{%
  \ifthenelse{\boolean{#1}}{\begin{trivlist}\item[\hskip \labelsep\bfseries {\Large Instructor Notes: \\} \hspace{\textwidth} ]}
{\begin{trivlist}\item[\hskip \labelsep\bfseries {\Large Instructor Notes: \\} \hspace{\textwidth} ]}
{}
}
{\end{trivlist}}
\fi


%% Suggested Timing
\newcommand{\timing}[1]{{\bf Suggested Timing: \hspace{2ex}} #1}




\hypersetup{
    colorlinks=true,       % false: boxed links; true: colored links
    linkcolor=blue,          % color of internal links (change box color with linkbordercolor)
    citecolor=green,        % color of links to bibliography
    filecolor=magenta,      % color of file links
    urlcolor=cyan           % color of external links
}
\title{Compare And Count}
\author{Vic Ferdinand, Betsy McNeal, Robin Pemantle, Jenny Sheldon}

\begin{document}
\begin{abstract}
This activity includes several sets of problems.  First, solve each problem.  Then, compare and contrast the problems in each set.  Which problems are the same?  Which problems are different?
\end{abstract}
\maketitle



\begin{problem}
There are 10 students in the ``Surf Lake Erie'' club.
\begin{enumerate}
\item In an election, how many ways are there to choose a President and a Vice President for the club?
\item At a bake sale, how many ways are there to choose two students to taste-test the pies? (Each of the testers tastes all of the pie.)
\item In an election, how many ways are there to choose a President, Vice President, Secretary, and Treasurer for the club?  (No student may hold two or more offices.)
\item How many ways are there to choose a squad of three students to wash the principal's car?

\end{enumerate}
\end{problem}

\begin{problem}
\begin{enumerate}
\item How many ways are there to choose three cookies off of a plate containing five (different) cookies?
\item How many ways are there to hand out first, second, and third place at the spelling bee if five students compete?
\item How many ways are there to choose a password using three different letters from a set of five?
\item How many ways are there to give a child three stickers out of a bag of five different stickers?
\end{enumerate}
\end{problem}


\begin{problem}
\begin{enumerate}
\item Write a story problem that has the same answer as 2(a).
\item Write a story problem that has the same answer as 2(b).
\end{enumerate}
\end{problem}


\newpage
\begin{instructorNotes}
This activity is intended to give students a chance to compute basic permutations and combinations, and discuss the similarities and differences among these problems.  In our calendar, we have already completed one activity about basic counting problems (``You Can Count On It''), and this is the second counting activity.  We follow this activity with ``You Can Count On It Getting More Difficult'', where the problems begin to increase in difficulty.  We use these activities about counting and probability to help cement the meaning of various operations for students.  Throughout these activities, we expect students to justify their operations of choice in their explanations.  There are a few main types of counting strategies we see from our students: arrays, ordered lists, tree diagrams, and algebraic expressions.  Students tend to want to use more complicated strategies before trying simpler strategies, so we are often reminding them to write down some examples, or to make sure their list is well organized. When discussing this activity as a whole class, we like to have students demonstrate multiple solution methods for each problem, even when not specifically requested by the problem.  

In this activity in particular, we are trying to give students an additional strategy of recognizing the structure of a problem.  If students can begin to recognize problems as the same structure as ones they have previously solved (but perhaps with different numbers), then they can apply the same solution method to a new problem.  This is the same idea of structure that we have tried to emphasize with operations, and so we like to also see the idea of structure in another place.

Because we are developing this idea of structure, for the rest of the semester, we often refer to these problems by name.  We ask questions like, ``is this like the cookies problem?'' or ``is this like the election problem?'' to distinguish the two situations.  We have chosen to not  introduce the formal terminology of ``permutations'' and ``combinations'', as we have found this to be more confusing for students than helpful.  

\timing{This activity is designed for one class period, but we do not typically finish all of the problems in that amount of time.  Instead, we give students 5 or 10 minutes to work on the first two problems, splitting the parts among the small groups.  Then, the groups present their work at the board, and we spend 15-20 minutes in whole-class discussion drawing out the structure idea.  If time remains, we go on to the final problem.}
\end{instructorNotes}


\end{document}