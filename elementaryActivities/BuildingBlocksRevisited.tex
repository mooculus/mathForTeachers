\documentclass[nooutcomes]{ximera}
\usepackage{gensymb}
\usepackage{tabularx}
\usepackage{mdframed}
\usepackage{pdfpages}
%\usepackage{chngcntr}

\let\problem\relax
\let\endproblem\relax

\newcommand{\property}[2]{#1#2}




\newtheoremstyle{SlantTheorem}{\topsep}{\fill}%%% space between body and thm
 {\slshape}                      %%% Thm body font
 {}                              %%% Indent amount (empty = no indent)
 {\bfseries\sffamily}            %%% Thm head font
 {}                              %%% Punctuation after thm head
 {3ex}                           %%% Space after thm head
 {\thmname{#1}\thmnumber{ #2}\thmnote{ \bfseries(#3)}} %%% Thm head spec
\theoremstyle{SlantTheorem}
\newtheorem{problem}{Problem}[]

%\counterwithin*{problem}{section}



%%%%%%%%%%%%%%%%%%%%%%%%%%%%Jenny's code%%%%%%%%%%%%%%%%%%%%

%%% Solution environment
%\newenvironment{solution}{
%\ifhandout\setbox0\vbox\bgroup\else
%\begin{trivlist}\item[\hskip \labelsep\small\itshape\bfseries Solution\hspace{2ex}]
%\par\noindent\upshape\small
%\fi}
%{\ifhandout\egroup\else
%\end{trivlist}
%\fi}
%
%
%%% instructorIntro environment
%\ifhandout
%\newenvironment{instructorIntro}[1][false]%
%{%
%\def\givenatend{\boolean{#1}}\ifthenelse{\boolean{#1}}{\begin{trivlist}\item}{\setbox0\vbox\bgroup}{}
%}
%{%
%\ifthenelse{\givenatend}{\end{trivlist}}{\egroup}{}
%}
%\else
%\newenvironment{instructorIntro}[1][false]%
%{%
%  \ifthenelse{\boolean{#1}}{\begin{trivlist}\item[\hskip \labelsep\bfseries Instructor Notes:\hspace{2ex}]}
%{\begin{trivlist}\item[\hskip \labelsep\bfseries Instructor Notes:\hspace{2ex}]}
%{}
%}
%% %% line at the bottom} 
%{\end{trivlist}\par\addvspace{.5ex}\nobreak\noindent\hung} 
%\fi
%
%


\let\instructorNotes\relax
\let\endinstructorNotes\relax
%%% instructorNotes environment
\ifhandout
\newenvironment{instructorNotes}[1][false]%
{%
\def\givenatend{\boolean{#1}}\ifthenelse{\boolean{#1}}{\begin{trivlist}\item}{\setbox0\vbox\bgroup}{}
}
{%
\ifthenelse{\givenatend}{\end{trivlist}}{\egroup}{}
}
\else
\newenvironment{instructorNotes}[1][false]%
{%
  \ifthenelse{\boolean{#1}}{\begin{trivlist}\item[\hskip \labelsep\bfseries {\Large Instructor Notes: \\} \hspace{\textwidth} ]}
{\begin{trivlist}\item[\hskip \labelsep\bfseries {\Large Instructor Notes: \\} \hspace{\textwidth} ]}
{}
}
{\end{trivlist}}
\fi


%% Suggested Timing
\newcommand{\timing}[1]{{\bf Suggested Timing: \hspace{2ex}} #1}




\hypersetup{
    colorlinks=true,       % false: boxed links; true: colored links
    linkcolor=blue,          % color of internal links (change box color with linkbordercolor)
    citecolor=green,        % color of links to bibliography
    filecolor=magenta,      % color of file links
    urlcolor=cyan           % color of external links
}

\title{Building Blocks, Revisited}
\author{Vic Ferdinand, Betsy McNeal, Jenny Sheldon, Mike Steward}

\begin{document}
\begin{abstract}
\end{abstract}
\maketitle




\begin{problem}
First measure the area of the rectangle below.  Describe how you would build a rectangular prism with volume $8$ cubic inches with its base on the rectangle.

%\vspace{1 in}
\tikz{ \draw (-2 in,-1 in) -- (2 in,-1 in); \draw (-2 in, -1 in) -- (-2 in, 1 in); \draw (-2 in, 1 in) -- (2 in, 1 in);\draw (2 in, 1 in) -- (2 in, -1 in);\draw (-2 in, -1 in) -- (2 in, -1 in); \draw(-2 in, 0)--(2 in, 0);\draw(-1 in, -1 in)--(-1 in, 1 in);\draw(0 in, -1 in)--(0 in, 1 in);\draw(1 in, -1 in)--(1 in, 1 in);}

\end{problem}

\begin{problem}
Using the same base, build a rectangular prism with volume $24$ cubic inches.

\end{problem}


\begin{problem}
We are familiar with the formula for the volume of a rectangular prism $V = \ell \times w \times h = A \times h$.  Using the meaning of multiplication (groups and objects per group), explain why this formula is accurate.


\end{problem}

\begin{problem} 
Think about a right circular cylinder.  How can you justify the volume formula $V = \pi r^2 h$?


\end{problem}

\newpage

\begin{instructorNotes}
This activity helps students consider why and when the volume formula for right rectangular prisms $A = L \times H$ is appropriate.  Just as students placed squares on top of lines to justify the area formula for rectangles in ``Building Blocks'', they will place unit cubes on top of squares in this activity, and hopefully make connections between the concepts of area formulae and volume formulae.  The key idea of this activity is subtle, however:  the volume of one layer of cubes is the same number of volume units as the area of the base.  Area (which is 2-dimensional) gives us a number of (3-dimensional) volume units.  We often point this out as a one-to-one correspondence between the area of the base and the volume of the first layer.  The idea of one-to-one correspondence helps students to remember the physical action of placing blocks on top of the squares.  


This activity is our first activity about volume, since we have considered the meaning of volume earlier in the course when we discussed measurement.  After this activity, we move to calculating volumes of increasingly complicated solids.

We have found it helpful to have $1$ cubic inch blocks available for students to use during this activity.  If printed at full size, the cubes will fit exactly on the squares.



\timing{This activity takes us about half of a class period.}

\end{instructorNotes}

\end{document}