\documentclass[nooutcomes,noauthor]{ximera}

\graphicspath{
  {./}
  {graphics/}
  {../graphics/}
}

\usepackage{chngcntr}

\let\question\relax
\let\endquestion\relax




\newtheoremstyle{SlantTheorem}{\topsep}{\fill}%%% space between body and thm
%\newtheoremstyle{SlantTheorem}{\topsep}{\topsep}%%% space between body and thm
 {\slshape}                      %%% Thm body font
 {}                              %%% Indent amount (empty = no indent)
 {\bfseries\sffamily}            %%% Thm head font
 {}                              %%% Punctuation after thm head
 {3ex}                           %%% Space after thm head
 {\thmname{#1}\thmnumber{ #2}\thmnote{ \bfseries(#3)}}%%% Thm head spec
\theoremstyle{SlantTheorem}
\newtheorem{question}{Question}
\counterwithin*{question}{section}



\let\instructorNotes\relax
\let\endinstructorNotes\relax
%%% instructorNotes environment
\ifhandout
\newenvironment{instructorNotes}[1][false]%
{%
\def\givenatend{\boolean{#1}}\ifthenelse{\boolean{#1}}{\begin{trivlist}\item}{\setbox0\vbox\bgroup}{}
}
{%
\ifthenelse{\givenatend}{\end{trivlist}}{\egroup}{}
}
\else
\newenvironment{instructorNotes}[1][false]%
{%
  \ifthenelse{\boolean{#1}}{\begin{trivlist}\item[\hskip \labelsep\bfseries {\Large Instructor Notes: \\} \hspace{\textwidth} ]}
{\begin{trivlist}\item[\hskip \labelsep\bfseries {\Large Instructor Notes: \\} \hspace{\textwidth} ]}
{}
}
{\end{trivlist}}
\fi


%% Suggested Timing
\newcommand{\timing}[1]{{\bf Suggested Timing: \hspace{2ex}} #1}
\title{You Can Count On It}

\begin{document}
\begin{abstract}
\end{abstract}
\maketitle



\begin{problem}
Nikki has three different stores to visit this afternoon: Target, Walmart, and Meijer.  How many ways are there for Nikki to complete her shopping trip?  Explain.
\end{problem}

%\begin{problem}
%Calvin is trying to complete the ``Columbus Coffee Experience", which includes buying coffee at five different locations.  

%\begin{enumerate}
%\item Assuming Calvin chooses five locations in advance, how many ways are there for him to reach his goal?
%\item Assuming Calvin does not choose the locations in advance, and there are 10 participating coffee shops, how many ways are there for him to reach his goal?
%\end{enumerate}
%\end{problem}



\begin{problem}
A certain passcode must be made in the following way: choose one of O, S ,or U, then choose a whole number between 6 and 9 (inclusive).  How many passcodes can be made?  Answer this question in the following ways:
\begin{enumerate}
\item By writing down all of the possibilities.
\item By organizing the possibilities in a table.
\item By drawing a tree diagram containing all of the possibilities.
\end{enumerate}
\end{problem}

\begin{problem}
You are a contestant on ``The Amazing Race''.  You must travel from your home to New York City, then from New York City to London, and finally from London to Paris.  You will travel from your home to New York City by car, train, bus, or plane.  You will travel from New York City to London by ship or by plane.  Finally, you will travel from London to Paris by ship, plane, or helicopter.
\begin{enumerate}
\item How many trips are possible from home to New York City to London?
\item How many trips are possible from home to New York City to London to Paris?
\end{enumerate}
\end{problem}

\begin{problem}
From 2001 - 2004, an Ohio license plate consisted of two letters followed by two digits followed by two letters.
\begin{enumerate}
\item How many different Ohio license plates could be made?
\item How many different Ohio license plates could be made if there are no repeats of numbers or letters allowed?
\end{enumerate}
\end{problem}

\newpage

\begin{instructorNotes}

{\bf Main goal:} We introduce counting. 

{\bf Overall picture:} Since this is our first activity about counting, we'd like to go slowly and carefully over these problems. Students should work on developing solution strategies that they can use over multiple counting activities.

\begin{itemize}
	\item You should have students present as many solution strategies as you see in the class (and perhaps add your own if there aren't too many amongst the students). Making an ordered list should be the first strategy that students use, though sometimes they think this isn't ``math'' enough! Other strategies should include drawing an array, making a tree diagram, or other visual representations of their options.
	\item The activity doesn't suggest that students multiply to solve the problem, so this will need to be something you add in your whole-class discussion using the students' ideas.
	\item Your discussion should help students to see that they have naturally organized their items into some equal groups. An array or a tree diagram makes this most obvious. Students will struggle with describing their groups here. Language like `` the groups are organized by which store Nikki goes to first'' should be encouraged, or students can also draw a diagram and put an arrow to their groups. The objects in these groups are also difficult for students, especially if they draw a tree diagram. Often, they want to think of the objects as what's listed at the end of the tree, but the object should actually be all branches together. For instance, in the Amazing Race, the objects are trips from home to NY to London to Paris (not just how we got to Paris).
	\item Practice the multiplication with groups and objects with as many of these examples as you have time for. You can't do too much here!
	\item We often don't get to discuss Problem 4!
\end{itemize}

{\bf Good language:} The application to multiplication is one of our main goals for this section, so that should be emphasized. But we are also developing a collection of strategies for solving these problems, so encourage students to take note of what strategies they like and understand and what strategies they need to work on a bit more. We would like students to be able to solve these problems in multiple ways!


{\bf Suggested timing:} Give students about 5-10 minutes to get started on the activity, and then discuss the first problem in depth. Point out all of the different strategies students have used and the ways we could count Nikki's trips with multiplication. Give students about 10 minutes to apply their knowledge on the rest of the problems, and then discuss as much as you have time for.

\end{instructorNotes}



\end{document}