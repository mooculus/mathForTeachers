\documentclass[nooutcomes,noauthor]{ximera}

\graphicspath{
  {./}
  {graphics/}
  {../graphics/}
}

\usepackage{chngcntr}

\let\question\relax
\let\endquestion\relax




\newtheoremstyle{SlantTheorem}{\topsep}{\fill}%%% space between body and thm
%\newtheoremstyle{SlantTheorem}{\topsep}{\topsep}%%% space between body and thm
 {\slshape}                      %%% Thm body font
 {}                              %%% Indent amount (empty = no indent)
 {\bfseries\sffamily}            %%% Thm head font
 {}                              %%% Punctuation after thm head
 {3ex}                           %%% Space after thm head
 {\thmname{#1}\thmnumber{ #2}\thmnote{ \bfseries(#3)}}%%% Thm head spec
\theoremstyle{SlantTheorem}
\newtheorem{question}{Question}
\counterwithin*{question}{section}



\let\instructorNotes\relax
\let\endinstructorNotes\relax
%%% instructorNotes environment
\ifhandout
\newenvironment{instructorNotes}[1][false]%
{%
\def\givenatend{\boolean{#1}}\ifthenelse{\boolean{#1}}{\begin{trivlist}\item}{\setbox0\vbox\bgroup}{}
}
{%
\ifthenelse{\givenatend}{\end{trivlist}}{\egroup}{}
}
\else
\newenvironment{instructorNotes}[1][false]%
{%
  \ifthenelse{\boolean{#1}}{\begin{trivlist}\item[\hskip \labelsep\bfseries {\Large Instructor Notes: \\} \hspace{\textwidth} ]}
{\begin{trivlist}\item[\hskip \labelsep\bfseries {\Large Instructor Notes: \\} \hspace{\textwidth} ]}
{}
}
{\end{trivlist}}
\fi


%% Suggested Timing
\newcommand{\timing}[1]{{\bf Suggested Timing: \hspace{2ex}} #1}
\title{Counting structure}

\begin{document}
\begin{abstract}
\end{abstract}
\maketitle



\begin{problem}

Write a counting story problem whose answer is
\[
\frac{7 \times 6 \times 5}{6}.
\]
Explain how you know your story problem has this solution.


\end{problem}


\begin{problem}

Write a story problem that has the same structure as the one in the previous problem, but has a different answer. Explain how you know the structure is the same, and then find the answer.

\end{problem}



\begin{problem}

Eve is playing a game where you roll a die and then spin a spinner. The die has 6 sides and the spinner has 8 equal sections labeled with letters A through H. How many combinations of one dice roll and one spin are there?

\begin{enumerate}
	\item Does this problem have the same structure as the cookie problem, the election problem, both, or neither? Explain how you know.
	\item What is the answer to the original problem?
\end{enumerate}

\end{problem}

\begin{problem}

Garrett is designing a game where you spin a spinner. The game also has a board with a ``start'' square. The spinner has 12 equal sections labeled with letters A through L. Garrett decides to choose three of the spinner sections to have the instruction ``go back to the start square''. How many ways are there for Garrett to choose these three sections on the spinner?

\begin{enumerate}
	\item Does this problem have the same structure as the cookie problem, the election problem, both, or neither? Explain how you know.
	\item What is the answer to the original problem?
	\item Challenge: how would your answer change if Garrett decided to choose four of the spinner sections to have the instruction ``move forward 2 spaces''?
\end{enumerate}

\end{problem}

\newpage

\begin{instructorNotes}

{\bf Main goal:} We practice identifying counting structure.


{\bf Overall picture:} These problems follow up our work on ``Compare and Count'' to help students practice talking about when problems have the same structure and when they do not.

\begin{itemize}
	\item Problem 1 could be a challenge if students aren't yet comfortable with the ``dividing by 6'' from Compare and Count. You can take time to go over this again if needed. If students are familiar with these ideas, the simplest solution would be to choose three different cookies from a plate containing seven cookies. This solution should be discussed, along with other creative ideas.
	\item Problem 2 could simply be back to the cookie problem again, but we are hoping students also feel free to be creative. Have students share examples that are ``choose three'' type as well as ``choose two'' type - and perhaps if some people try a ``choose four'' situation that would be great to discuss!
	\item Problem 3 reminds us again that not all problems are permutations and combinations. Problem 4 gives us a chance to extend to choosing four items if we have time to talk about that.
	
\end{itemize}

{\bf Good language:} Students might be looking for a definition of structure; the closest thing is probably to say that we take the same procedure for solving, or the setup is the same, but we use different numbers in our calculations. So the story can be exactly the same or using different objects, but we should be able to match up pieces of the two stories and see that they are really the same underlying idea.


{\bf Suggested timing:} Give students about 20 minutes to work on these problems, and then discuss as much as you can.

\end{instructorNotes}



\end{document}