\documentclass[handout]{ximera}
\usepackage{gensymb}
\usepackage{tabularx}
\usepackage{mdframed}
\usepackage{pdfpages}
%\usepackage{chngcntr}

\let\problem\relax
\let\endproblem\relax

\newcommand{\property}[2]{#1#2}




\newtheoremstyle{SlantTheorem}{\topsep}{\fill}%%% space between body and thm
 {\slshape}                      %%% Thm body font
 {}                              %%% Indent amount (empty = no indent)
 {\bfseries\sffamily}            %%% Thm head font
 {}                              %%% Punctuation after thm head
 {3ex}                           %%% Space after thm head
 {\thmname{#1}\thmnumber{ #2}\thmnote{ \bfseries(#3)}} %%% Thm head spec
\theoremstyle{SlantTheorem}
\newtheorem{problem}{Problem}[]

%\counterwithin*{problem}{section}



%%%%%%%%%%%%%%%%%%%%%%%%%%%%Jenny's code%%%%%%%%%%%%%%%%%%%%

%%% Solution environment
%\newenvironment{solution}{
%\ifhandout\setbox0\vbox\bgroup\else
%\begin{trivlist}\item[\hskip \labelsep\small\itshape\bfseries Solution\hspace{2ex}]
%\par\noindent\upshape\small
%\fi}
%{\ifhandout\egroup\else
%\end{trivlist}
%\fi}
%
%
%%% instructorIntro environment
%\ifhandout
%\newenvironment{instructorIntro}[1][false]%
%{%
%\def\givenatend{\boolean{#1}}\ifthenelse{\boolean{#1}}{\begin{trivlist}\item}{\setbox0\vbox\bgroup}{}
%}
%{%
%\ifthenelse{\givenatend}{\end{trivlist}}{\egroup}{}
%}
%\else
%\newenvironment{instructorIntro}[1][false]%
%{%
%  \ifthenelse{\boolean{#1}}{\begin{trivlist}\item[\hskip \labelsep\bfseries Instructor Notes:\hspace{2ex}]}
%{\begin{trivlist}\item[\hskip \labelsep\bfseries Instructor Notes:\hspace{2ex}]}
%{}
%}
%% %% line at the bottom} 
%{\end{trivlist}\par\addvspace{.5ex}\nobreak\noindent\hung} 
%\fi
%
%


\let\instructorNotes\relax
\let\endinstructorNotes\relax
%%% instructorNotes environment
\ifhandout
\newenvironment{instructorNotes}[1][false]%
{%
\def\givenatend{\boolean{#1}}\ifthenelse{\boolean{#1}}{\begin{trivlist}\item}{\setbox0\vbox\bgroup}{}
}
{%
\ifthenelse{\givenatend}{\end{trivlist}}{\egroup}{}
}
\else
\newenvironment{instructorNotes}[1][false]%
{%
  \ifthenelse{\boolean{#1}}{\begin{trivlist}\item[\hskip \labelsep\bfseries {\Large Instructor Notes: \\} \hspace{\textwidth} ]}
{\begin{trivlist}\item[\hskip \labelsep\bfseries {\Large Instructor Notes: \\} \hspace{\textwidth} ]}
{}
}
{\end{trivlist}}
\fi


%% Suggested Timing
\newcommand{\timing}[1]{{\bf Suggested Timing: \hspace{2ex}} #1}




\hypersetup{
    colorlinks=true,       % false: boxed links; true: colored links
    linkcolor=blue,          % color of internal links (change box color with linkbordercolor)
    citecolor=green,        % color of links to bibliography
    filecolor=magenta,      % color of file links
    urlcolor=cyan           % color of external links
}
\title{High Rollers}

\begin{document}
\begin{abstract}
\end{abstract}
\maketitle


Suppose that you roll two six-sided dice: one is scarlet, and the other is grey.  You are interested in the sum of the two dice.

\begin{problem}
How many different outcomes are possible when rolling these two dice?
\end{problem}

\begin{problem} 
Are you more likely to roll a sum of two or a sum of eight?  Or, are these two things equally likely?
\end{problem}

With a partner, roll a pair of dice at least 100 times, and record the sum of the two dice in a table.  You can physically roll the dice, or use an online dice roller like \url{http://www.roll-dice-online.com/} (which will roll them all at once) or \url{http://www.dicesimulator.com/} (where you can tally them as you roll).


\begin{problem}
Based on the table of results you just made, what is the probability of rolling
\begin{enumerate}
\item a sum of two?
\item a sum of seven?
\item a sum of eight?
\end{enumerate}
Were your earlier predictions correct?
\end{problem}

\begin{problem}
Using reasoning (not our data), what is the probability of rolling
\begin{enumerate}
\item a sum of two?
\item a sum of seven?
\item a sum of eight?
\end{enumerate}
How close are these values to the ones predicted by the chart?  How could we make them closer?
\end{problem}

\newpage

\begin{instructorNotes}
{\bf Main goal:} We introduce probability and distinguish between experimental probability and theoretical probability.

{\bf Overall picture:} 

We want to help students connect their counting skills to the idea of probability. You should finish up this activity by introducing our main formula for probability in terms of the probability of an event being a fraction whose denominator is the total number of ways an event can occur and the numerator is the total number of ways we are trying to count.  See the main text.

This activity also introduces the idea of outcomes being equally likely in that students sometimes think that there are $11$ different outcomes (sums of two through $12$) and so the probability of each item should be the same. Part of the point of making the histogram is to help with this misconception. You should have a discussion in the first two problems about the outcomes possible in this experiment as well as whether those outcomes are equally likely.  The three most common numbers of total outcomes are $11$, $36$, and $42$ For the $42$, students are double-counting the matched dice like two threes. The point of having the dice be two different colors is to combat this misconception.

\link[Here]{https://www.geogebra.org/m/UsoH4eNl} is a nice website that makes a histogram (for potential use in class).

{\bf Suggested timing:} Give students about 15 minutes to think about the first two problems and collect their data. Spend about 10 minutes putting the data together and discussing the students' ideas. Give students about $10$ minutes to work through the probability problems, then use the rest of the time on discussion and introducing the main ideas of probability.


\end{instructorNotes}



\end{document}