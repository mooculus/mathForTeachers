\documentclass[noauthor,nooutcomes]{ximera}
\usepackage{gensymb}
\usepackage{tabularx}
\usepackage{mdframed}
\usepackage{pdfpages}
%\usepackage{chngcntr}

\let\problem\relax
\let\endproblem\relax

\newcommand{\property}[2]{#1#2}




\newtheoremstyle{SlantTheorem}{\topsep}{\fill}%%% space between body and thm
 {\slshape}                      %%% Thm body font
 {}                              %%% Indent amount (empty = no indent)
 {\bfseries\sffamily}            %%% Thm head font
 {}                              %%% Punctuation after thm head
 {3ex}                           %%% Space after thm head
 {\thmname{#1}\thmnumber{ #2}\thmnote{ \bfseries(#3)}} %%% Thm head spec
\theoremstyle{SlantTheorem}
\newtheorem{problem}{Problem}[]

%\counterwithin*{problem}{section}



%%%%%%%%%%%%%%%%%%%%%%%%%%%%Jenny's code%%%%%%%%%%%%%%%%%%%%

%%% Solution environment
%\newenvironment{solution}{
%\ifhandout\setbox0\vbox\bgroup\else
%\begin{trivlist}\item[\hskip \labelsep\small\itshape\bfseries Solution\hspace{2ex}]
%\par\noindent\upshape\small
%\fi}
%{\ifhandout\egroup\else
%\end{trivlist}
%\fi}
%
%
%%% instructorIntro environment
%\ifhandout
%\newenvironment{instructorIntro}[1][false]%
%{%
%\def\givenatend{\boolean{#1}}\ifthenelse{\boolean{#1}}{\begin{trivlist}\item}{\setbox0\vbox\bgroup}{}
%}
%{%
%\ifthenelse{\givenatend}{\end{trivlist}}{\egroup}{}
%}
%\else
%\newenvironment{instructorIntro}[1][false]%
%{%
%  \ifthenelse{\boolean{#1}}{\begin{trivlist}\item[\hskip \labelsep\bfseries Instructor Notes:\hspace{2ex}]}
%{\begin{trivlist}\item[\hskip \labelsep\bfseries Instructor Notes:\hspace{2ex}]}
%{}
%}
%% %% line at the bottom} 
%{\end{trivlist}\par\addvspace{.5ex}\nobreak\noindent\hung} 
%\fi
%
%


\let\instructorNotes\relax
\let\endinstructorNotes\relax
%%% instructorNotes environment
\ifhandout
\newenvironment{instructorNotes}[1][false]%
{%
\def\givenatend{\boolean{#1}}\ifthenelse{\boolean{#1}}{\begin{trivlist}\item}{\setbox0\vbox\bgroup}{}
}
{%
\ifthenelse{\givenatend}{\end{trivlist}}{\egroup}{}
}
\else
\newenvironment{instructorNotes}[1][false]%
{%
  \ifthenelse{\boolean{#1}}{\begin{trivlist}\item[\hskip \labelsep\bfseries {\Large Instructor Notes: \\} \hspace{\textwidth} ]}
{\begin{trivlist}\item[\hskip \labelsep\bfseries {\Large Instructor Notes: \\} \hspace{\textwidth} ]}
{}
}
{\end{trivlist}}
\fi


%% Suggested Timing
\newcommand{\timing}[1]{{\bf Suggested Timing: \hspace{2ex}} #1}




\hypersetup{
    colorlinks=true,       % false: boxed links; true: colored links
    linkcolor=blue,          % color of internal links (change box color with linkbordercolor)
    citecolor=green,        % color of links to bibliography
    filecolor=magenta,      % color of file links
    urlcolor=cyan           % color of external links
}
\title{How Likely}

\begin{document}
\begin{abstract}
These problems are connected to the ones on ``You Can Count On It''. You may want to review your work there before you get started here!
\end{abstract}
\maketitle



\begin{problem}
Remember Nikki, who went to Target, WalMart, and Meijer? Let's calculate some probabilities related to her afternoon. Assume that Nikki choses her shopping sequence randomly. In each case, explain your thinking to your group.

\begin{enumerate}
	\item What is the probability that Nikki chooses to go to WalMart, then Target, then Meijer?
	\item What is the probability that Nikki chooses to go to Target last?
\end{enumerate}
\vfill
\end{problem}

\begin{problem}
Remember the passcodes made with one letter from O, S, and U, and one number between 6 and 9 (inclusive)?
Assume that you choose a passcode randomly. Let's calculate some probabilities related to these passcodes. In each case, explain your thinking to your group.
\begin{enumerate}
	\item What is the probability that you choose the passcode O9?
	\item What is the probability that you choose a passcode that begins with S?
	\item What is the probability that you choose a passcode with an even number in it?
\end{enumerate} \vfill
\end{problem}

\newpage

\begin{problem}
Remember the Amazing Race, where you travel from home to New York by car, train, bus, or plane, then travel to London by ship or plane, and finally to Paris by ship, plane or helicopter? Assume that you choose each leg of your journey randomly. Let's calculate some probabilities related to these journies. In each case, explain your thinking to your group.
\begin{enumerate}
	\item What is the probability that you choose to travel by train, then ship, then plane?
	\item What is the probability that you choose to travel by helicopter?
	\item What is the probability that you choose to travel by ship?
\end{enumerate} \vfill
\end{problem}


\begin{problem}
Remember the license plates consisting of two letters followed by two digits followed by two letters? Assume that you choose a license plate randomly. Let's calculate some probabilities related to these license plates. In each case, explain your thinking to your group.
\begin{enumerate}
	\item What is the probability that you choose the license plate $AB07YN$?
	\item What is the probability that you choose the first letter of your license plate to be $Q$?
	\item What is the probability that you choose a license plate with two odd numbers?
\end{enumerate}\vfill
\end{problem}



\newpage

\begin{instructorNotes}
{\bf Main goal:} We calculate basic probabilities using problems we have already practiced counting.

{\bf Overall picture:} 
Here we would like to emphasize the idea that we can calculate probabilities of equally likely events by calculating the number of options in the sample space, calculating the number of options in the event space, and writing the corresponding fraction. The denominator of each fraction should already be known if you completed ``You Can Count on It", so that students only have to count the numerator. The point is to practice more with counting, and connect the ideas of counting and probability.

As you discuss, you will want to point out the similarities and differences between the problems. For instance, when we travel by helicopter, there is only one part of the journey where we might use a helicopter. However, when we travel by ship, this could be from New York to London, from London to Paris, or both. All of those equally likely events should be counted!

When you wrap up the discussion, you will want to remind students that there are probabilities we can calculate where the options are not equally likely, and we would use a different strategy in that case. For instance, you might highlight that it's not very realistic for Nikki to choose her shopping trip randomly, as she likely has different reasons for going to each store. Similarly, on the Amazing Race, you would not choose your trips randomly but based on your competition.




{\bf Good language:} We don't need to name the event space or the sample space unless it's helping students describe what they are counting. We do want to connect to our meaning of fractions, where we think of the denominator talking about how many equal pieces our whole is cut into (in this case the number of equally likely outcomes) and our numerator as counting how many such pieces we want (in this case the number of outcomes that we are looking for). 

{\bf Suggested timing:}  This is likely more than you will be able to cover in a single class. Give students 5-8 minutes to work on problem 1, then discuss. Have students present their work, and encourage them to explain their thinking. Then repeat this process with the other problems as you have time. The later problems may need more than 5-8 minutes of work time!


\end{instructorNotes}


\end{document}





