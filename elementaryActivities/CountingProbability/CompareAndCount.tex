%\documentclass[handout]{ximera}
\documentclass[nooutcomes, noauthor]{ximera}

\graphicspath{
  {./}
  {graphics/}
  {../graphics/}
}

\usepackage{chngcntr}

\let\question\relax
\let\endquestion\relax




\newtheoremstyle{SlantTheorem}{\topsep}{\fill}%%% space between body and thm
%\newtheoremstyle{SlantTheorem}{\topsep}{\topsep}%%% space between body and thm
 {\slshape}                      %%% Thm body font
 {}                              %%% Indent amount (empty = no indent)
 {\bfseries\sffamily}            %%% Thm head font
 {}                              %%% Punctuation after thm head
 {3ex}                           %%% Space after thm head
 {\thmname{#1}\thmnumber{ #2}\thmnote{ \bfseries(#3)}}%%% Thm head spec
\theoremstyle{SlantTheorem}
\newtheorem{question}{Question}
\counterwithin*{question}{section}



\let\instructorNotes\relax
\let\endinstructorNotes\relax
%%% instructorNotes environment
\ifhandout
\newenvironment{instructorNotes}[1][false]%
{%
\def\givenatend{\boolean{#1}}\ifthenelse{\boolean{#1}}{\begin{trivlist}\item}{\setbox0\vbox\bgroup}{}
}
{%
\ifthenelse{\givenatend}{\end{trivlist}}{\egroup}{}
}
\else
\newenvironment{instructorNotes}[1][false]%
{%
  \ifthenelse{\boolean{#1}}{\begin{trivlist}\item[\hskip \labelsep\bfseries {\Large Instructor Notes: \\} \hspace{\textwidth} ]}
{\begin{trivlist}\item[\hskip \labelsep\bfseries {\Large Instructor Notes: \\} \hspace{\textwidth} ]}
{}
}
{\end{trivlist}}
\fi


%% Suggested Timing
\newcommand{\timing}[1]{{\bf Suggested Timing: \hspace{2ex}} #1}
\title{Compare And Count}

\begin{document}
\begin{abstract}
This activity includes several sets of problems.  First, solve each problem.  Then, compare and contrast the problems in each set.  Which problems are the same?  Which problems are different?
\end{abstract}
\maketitle



\begin{problem}
There are 10 students in the ``Surf Lake Erie'' club.
\begin{enumerate}
\item In an election, how many ways are there to choose a President and a Vice President for the club?
\item At a bake sale, how many ways are there to choose two students to taste-test the pies? (Each of the testers tastes all of the pie.)
\item In an election, how many ways are there to choose a President, Vice President, Secretary, and Treasurer for the club?  (No student may hold two or more offices.)
\item How many ways are there to choose a squad of three students to wash the principal's car?

\end{enumerate}
\end{problem}

\begin{problem}
\begin{enumerate}
\item How many ways are there to choose three cookies off of a plate containing five (different) cookies?
\item How many ways are there to hand out first, second, and third place at the spelling bee if five students compete?
\item How many ways are there to choose a password using three different letters from a set of five?
\item How many ways are there to give a child three stickers out of a bag of five different stickers?
\end{enumerate}
\end{problem}


\begin{problem}
\begin{enumerate}
\item Write a story problem that has the same answer as 2(a).
\item Write a story problem that has the same answer as 2(b).
\end{enumerate}
\end{problem}

\newpage

In case you find it helpful, here are the letters $A$, $B$, $C$, $D$, $E$ written three at a time. How are these related to the stories above?
\huge \begin{center}
\begin{tabular} {|c|c|c|c|c|} \hline
ABC & BAC & CAB & DAB & EAB \\ \hline
ABD & BAD & CAD & DAC & EAC \\ \hline
ABE & BAE & CAE & DAE & EAD \\ \hline
ACB & BCA & CBA & DBA & EBA \\ \hline
ACD & BCD & CBD & DBC & EBC \\ \hline
ACE & BCE & CBE & DBE & EBD \\ \hline
ADB & BDA & CDA & DCA & ECA \\ \hline
ADC & BDC & CDB & DCB & ECB \\ \hline
ADE & BDE & CDE & DCE & ECD \\ \hline
AEB & BEA & CEA & DEA & EDA \\ \hline
AEC & BEC & CEB & DEB & EDB \\ \hline
AED & BED & CED & DEC & EDC \\ \hline

 \end{tabular}
\end{center} \rm

\newpage

\begin{instructorNotes}

{\bf Main goal:} We identify counting problems that have the same structure.

{\bf Overall picture:}  This activity is intended to give students a chance to compute basic permutations and combinations, and discuss the similarities and differences amongst these problems. We won't name permutations and combinations unless the students themselves insist on these names. We have found them in general to be more confusing than they are worth. Instead, we will aim to have students identify problem structure based on other problems they have done. For instance, they could identify a permutation as ``like the election problem'' and a combination as ``like the cookies problem'' from this activity. Your overarching goal in this activity is to help students move towards these comparisons as well as practice their counting.

\begin{itemize}
	\item Students should be applying the strategies (both visual and multiplicative) that they learned in the previous activity. The first thing to discuss here is how to apply each of our strategies to these particular problems. Students should be clear about their groups and objects when they are using multiplication!
	\item With the Surf Lake Erie club, if students are a bit overwhelmed by the size of the numbers, suggest they start with smaller numbers and see if they can find a pattern. This is one of our strategies!
	\item By part (b), students encounter a combination situation and need to eliminate some of their choices. You will likely have some disagreement amongst the groups as to whether the answer should be 90 or 45. Have students discuss why they made each choice, and then relate back to the problem. If students $A$ and $B$ are chosen to taste-test, is this the same situation as when students $B$ and $A$ are chosen? (It may be tempting to ask ``does order matter?'' in this situation, but try to stay away from this rather loaded phrasing.) You could ask whether it matters which of the taste-testers was chosen first, or whether the distinction of who was chosen first lasts beyond the choosing. It can be good to contrast this with part (a). You will likely need to phrase this difference in a number of ways.
	\item Problem 1(d), 2(a), and 2(d) (the combination problems with three in the group) are very difficult here. Have students present their ideas, and focus on the first day on strategies that don't try to use multiplication and division. We will try on the second (and third) day to connect the solution to our meaning of operations. Students will often talk about getting rid of the repetitions, and may try to do this manually. The cookie plate and sticker situations will be a bit easier for this manual removal than the squad, so if students are getting stuck on 1(d) you can suggest they just move on to Problem 2.
	\item When you discuss the structure of the problems, have students try to match each piece of one problem with each piece of another. In other words, we would like to focus on the question, ``how can you tell?''  For instance, with the squad of students and the cookie plate: we choose the first student/cookie, and then we have one fewer for our second choice. When we choose the second student/cookie, we again have one fewer for our third choice. In both situations, the squad/cookie plate can be rearranged in any fashion and we still get the same squad/plate, so we need to get rid of the repetitions.
	\item When you discuss the students' problems, repeat this process all over again. Be sure to ask them to describe their thinking!
	\item We typically would not give the students the chart on page 2 until they have tried to solve these problems on their own. The page may not be needed if students have their own methods of counting when order doesn't matter.
	\item Concerning the chart on page 2, it can be helpful for students to use one color to circle all the groups that are the same team as A, B, and C.  Then use another color to circle all of the groups that are the same team as A, B, D.  This is supposed to help students see that dividing the total count of $5 \times 4 \times 3$ by $3 \times 2 \times 1$ makes sense.
	\item You may need to remind students why we multiplied $5 \times 4 \times 3$ to get the total number of items on the page! This day is for wrapping up, so don't be afraid to confront any questions still lingering.
	\item It's okay if the students initially simply count the six repetitions of each squad instead of multiplying $3 \times 2 \times 1$. The chart is designed to emphasize that there really are six of them, not just three (the most common incorrect denominator)! However, we would like the students to move towards this more general formulation. You should also connect this multiplication to Nikki's trips to Walmart, Target, and Meijer from the first activity. Once we know what three letters we need, we should have $3 \times 2 \times 1$ ways of writing these down.
\end{itemize}

Wrap up the second day on structure by reminding students that there are many other types of counting problems out there! These two types are common, but not the only types.


{\bf Good language:} Notice that the students' idea about getting rid of repetitive solutions sounds much more like subtraction than division. It's good to point this out -- they are trying to take away all of the solutions that are repeats of some other solution. But these are hard to count! We can suggest instead making groupings of solutions that are all the same, which should lead us towards division.

Be patient and willing to follow the students' ideas! These problems are difficult and sometimes frustrating, but we have the time to spend on them to fully understand what's going on. If students are making good progress on these ideas, there's no need to rush to the next thing.


{\bf Suggested timing:} We have three days to talk about this activity. On the first day, students should work on counting each of the situations in these problems. Give about 15-20 minutes to work on problems 1 and 2, and then use as much time as you need to discuss. Include as many strategies as you can. It's likely that students will not understand the division ideas during this first session (and probably will not naturally use a correct division on their first solution of the problem). It's not necessary for the students to have the correct answers to the combination problems (the squad, the taste-testers, the cookie plate, and the stickers) during this first class. 

On the second day, focus more on the structures of the problems. Give students about 5-10 minutes to compare and contrast the problems, and work on Problem 3, then use as much time to discuss as you need. For the rest of the time, return to understanding the solution to the combination problems . 

On the third day, start by pointing students to the chart on the second page. Introduce the activity by comparing the five letters to the five cookies from the cookie plate problem, or to a club with five members who are trying to choose three students for a squad. Give students about 10 minutes to work through this activity, and then discuss. Follow up the discussion by returning to any unsolved problems from the Compare and Count activity. If you have extra time, either move to the next activity or give students time to make up their own problems that fit the patterns of permutations and combinations, and work on solving them. If the students do particularly well with this activity, you could ask about a squad of four students!

\end{instructorNotes}


\end{document}