\documentclass{ximera}

\usepackage{gensymb}
\usepackage{tabularx}
\usepackage{mdframed}
\usepackage{pdfpages}
%\usepackage{chngcntr}

\let\problem\relax
\let\endproblem\relax

\newcommand{\property}[2]{#1#2}




\newtheoremstyle{SlantTheorem}{\topsep}{\fill}%%% space between body and thm
 {\slshape}                      %%% Thm body font
 {}                              %%% Indent amount (empty = no indent)
 {\bfseries\sffamily}            %%% Thm head font
 {}                              %%% Punctuation after thm head
 {3ex}                           %%% Space after thm head
 {\thmname{#1}\thmnumber{ #2}\thmnote{ \bfseries(#3)}} %%% Thm head spec
\theoremstyle{SlantTheorem}
\newtheorem{problem}{Problem}[]

%\counterwithin*{problem}{section}



%%%%%%%%%%%%%%%%%%%%%%%%%%%%Jenny's code%%%%%%%%%%%%%%%%%%%%

%%% Solution environment
%\newenvironment{solution}{
%\ifhandout\setbox0\vbox\bgroup\else
%\begin{trivlist}\item[\hskip \labelsep\small\itshape\bfseries Solution\hspace{2ex}]
%\par\noindent\upshape\small
%\fi}
%{\ifhandout\egroup\else
%\end{trivlist}
%\fi}
%
%
%%% instructorIntro environment
%\ifhandout
%\newenvironment{instructorIntro}[1][false]%
%{%
%\def\givenatend{\boolean{#1}}\ifthenelse{\boolean{#1}}{\begin{trivlist}\item}{\setbox0\vbox\bgroup}{}
%}
%{%
%\ifthenelse{\givenatend}{\end{trivlist}}{\egroup}{}
%}
%\else
%\newenvironment{instructorIntro}[1][false]%
%{%
%  \ifthenelse{\boolean{#1}}{\begin{trivlist}\item[\hskip \labelsep\bfseries Instructor Notes:\hspace{2ex}]}
%{\begin{trivlist}\item[\hskip \labelsep\bfseries Instructor Notes:\hspace{2ex}]}
%{}
%}
%% %% line at the bottom} 
%{\end{trivlist}\par\addvspace{.5ex}\nobreak\noindent\hung} 
%\fi
%
%


\let\instructorNotes\relax
\let\endinstructorNotes\relax
%%% instructorNotes environment
\ifhandout
\newenvironment{instructorNotes}[1][false]%
{%
\def\givenatend{\boolean{#1}}\ifthenelse{\boolean{#1}}{\begin{trivlist}\item}{\setbox0\vbox\bgroup}{}
}
{%
\ifthenelse{\givenatend}{\end{trivlist}}{\egroup}{}
}
\else
\newenvironment{instructorNotes}[1][false]%
{%
  \ifthenelse{\boolean{#1}}{\begin{trivlist}\item[\hskip \labelsep\bfseries {\Large Instructor Notes: \\} \hspace{\textwidth} ]}
{\begin{trivlist}\item[\hskip \labelsep\bfseries {\Large Instructor Notes: \\} \hspace{\textwidth} ]}
{}
}
{\end{trivlist}}
\fi


%% Suggested Timing
\newcommand{\timing}[1]{{\bf Suggested Timing: \hspace{2ex}} #1}




\hypersetup{
    colorlinks=true,       % false: boxed links; true: colored links
    linkcolor=blue,          % color of internal links (change box color with linkbordercolor)
    citecolor=green,        % color of links to bibliography
    filecolor=magenta,      % color of file links
    urlcolor=cyan           % color of external links
}


\title{Our Problems are Multiplying}
\author{Vic Ferdinand, Betsy McNeal, Jenny Sheldon}

\begin{document}
\begin{abstract}
    Let's try multiplying some integers!
\end{abstract}
\maketitle

\begin{instructorIntro}
This is the final day of the three days on operations with negative numbers.  The main purpose of this day is to talk about issues with multiplication, as well as clear up as much confusion and as many questions as we can.

Please see the Integers reading for more specific structural information about each of the models discussed in depth here.

Again, the activity is designed so you can spend time if you like on the students' choices of interpretations, or just skip to checks and bills and number lines.  Students should be fairly comfortable with each by the end of the activity.

With multiplication, we should take extreme care with the checks and bills method.  Again, attend carefully to the question being asked, to ensure that it is answered by the appropriate expression.  For instance, ``what is my balance?'' is not often a good question, unless we explicitly start with a zero balance.  In the same vein, ``how much has my balance gone down?'' should have a positive answer, even though the question suggests it should be negative.  The more practice you can get evaluating stories for correctness, the better.

Having a negative number in the first factor can still feel contrived to students.  Remind them again that all of these contexts have issues, and that none is perfect!  To help with this issue, you may want to think about starting with a zero balance and then adding or taking from this zero with multiplication stories.  

{\bf Suggested Timing:} This activity should take one class period.  If there are enough lingering questions, you might consider spending a small amount of the next class clearing those up.
\end{instructorIntro}

Here are some multiplication problems we'll be working with all day.

\begin{enumerate}
    \item $5 \times 2$
    \item $15 \times (-3)$
    \item $(-4) \times 9$
    \item $(-6) \times (-12)$
\end{enumerate}


\begin{question}
Use our real-life integer contexts to write story problems for the multiplication problems above.  Draw pictures explaining what a child just learning this concept might do to solve the problem.

%\begin{instructorNotes}
%Asking all the same questions again should give us the chance to catch up with whatever we didn't finish the previous day... I hope!
%\end{instructorNotes}
\end{question}

\begin{question}
For each of the multiplication problems above, how would you use a number line to model the solution?  Draw and explain clearly, especially when your number of groups is a negative number.
\end{question}

\begin{question}
Write a story problem for each of the multiplication problems above which uses our checks-and-bills scenario.  Make sure to read your story over several times.  Does the sensible answer to your story match the answer to the multiplication problem?
\end{question}

\begin{question}
How would you explain the ``rules'' for division with integers?  Give several examples.

%If we haven't yet, this would be a good time to mention rules with subtracting negatives, multiplying negatives, etc.  We're starting to wrap up with this question.
\end{question}


\begin{question}
Write down some overall thoughts about operations with integers.  What did each operation have in common?  What were the most difficult things to interpret or justify?  What interpretations or contexts made the most sense to you?

%I'm hoping this gives us a good way to wrap up the three days of discussion, but of course we can skip it if there's no time!
\end{question}







\end{document}