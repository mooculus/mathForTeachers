\documentclass[nooutcomes]{ximera}

\usepackage{gensymb}
\usepackage{tabularx}
\usepackage{mdframed}
\usepackage{pdfpages}
%\usepackage{chngcntr}

\let\problem\relax
\let\endproblem\relax

\newcommand{\property}[2]{#1#2}




\newtheoremstyle{SlantTheorem}{\topsep}{\fill}%%% space between body and thm
 {\slshape}                      %%% Thm body font
 {}                              %%% Indent amount (empty = no indent)
 {\bfseries\sffamily}            %%% Thm head font
 {}                              %%% Punctuation after thm head
 {3ex}                           %%% Space after thm head
 {\thmname{#1}\thmnumber{ #2}\thmnote{ \bfseries(#3)}} %%% Thm head spec
\theoremstyle{SlantTheorem}
\newtheorem{problem}{Problem}[]

%\counterwithin*{problem}{section}



%%%%%%%%%%%%%%%%%%%%%%%%%%%%Jenny's code%%%%%%%%%%%%%%%%%%%%

%%% Solution environment
%\newenvironment{solution}{
%\ifhandout\setbox0\vbox\bgroup\else
%\begin{trivlist}\item[\hskip \labelsep\small\itshape\bfseries Solution\hspace{2ex}]
%\par\noindent\upshape\small
%\fi}
%{\ifhandout\egroup\else
%\end{trivlist}
%\fi}
%
%
%%% instructorIntro environment
%\ifhandout
%\newenvironment{instructorIntro}[1][false]%
%{%
%\def\givenatend{\boolean{#1}}\ifthenelse{\boolean{#1}}{\begin{trivlist}\item}{\setbox0\vbox\bgroup}{}
%}
%{%
%\ifthenelse{\givenatend}{\end{trivlist}}{\egroup}{}
%}
%\else
%\newenvironment{instructorIntro}[1][false]%
%{%
%  \ifthenelse{\boolean{#1}}{\begin{trivlist}\item[\hskip \labelsep\bfseries Instructor Notes:\hspace{2ex}]}
%{\begin{trivlist}\item[\hskip \labelsep\bfseries Instructor Notes:\hspace{2ex}]}
%{}
%}
%% %% line at the bottom} 
%{\end{trivlist}\par\addvspace{.5ex}\nobreak\noindent\hung} 
%\fi
%
%


\let\instructorNotes\relax
\let\endinstructorNotes\relax
%%% instructorNotes environment
\ifhandout
\newenvironment{instructorNotes}[1][false]%
{%
\def\givenatend{\boolean{#1}}\ifthenelse{\boolean{#1}}{\begin{trivlist}\item}{\setbox0\vbox\bgroup}{}
}
{%
\ifthenelse{\givenatend}{\end{trivlist}}{\egroup}{}
}
\else
\newenvironment{instructorNotes}[1][false]%
{%
  \ifthenelse{\boolean{#1}}{\begin{trivlist}\item[\hskip \labelsep\bfseries {\Large Instructor Notes: \\} \hspace{\textwidth} ]}
{\begin{trivlist}\item[\hskip \labelsep\bfseries {\Large Instructor Notes: \\} \hspace{\textwidth} ]}
{}
}
{\end{trivlist}}
\fi


%% Suggested Timing
\newcommand{\timing}[1]{{\bf Suggested Timing: \hspace{2ex}} #1}




\hypersetup{
    colorlinks=true,       % false: boxed links; true: colored links
    linkcolor=blue,          % color of internal links (change box color with linkbordercolor)
    citecolor=green,        % color of links to bibliography
    filecolor=magenta,      % color of file links
    urlcolor=cyan           % color of external links
}


\title{Our Problems are Multiplying}
\author{Vic Ferdinand, Betsy McNeal, Jenny Sheldon}

\begin{document}
\begin{abstract}
    Let's try multiplying some integers!
\end{abstract}
\maketitle



Here are some multiplication problems we'll be working with throughout the activity.

\begin{enumerate}
    \item $5 \times 2$
    \item $15 \times (-3)$
    \item $(-4) \times 9$
    \item $(-6) \times (-12)$
\end{enumerate}


\begin{problem}
Use our real-life integer contexts to write story problems for the multiplication problems above.  Draw pictures explaining what a child just learning this concept might do to solve the problem.

\end{problem}

\begin{problem}
For each of the multiplication problems above, how would you use a number line to model the solution?  Draw and explain clearly, especially when your number of groups is a negative number.
\end{problem}

\begin{problem}
Write a story problem for each of the multiplication problems above which uses our checks-and-bills scenario.  Make sure to read your story over several times.  Does the sensible answer to your story match the answer to the multiplication problem?
\end{problem}

\begin{problem}
How would you explain the ``rules'' for division with integers?  Give several examples.


\end{problem}


\begin{problem}
Write down some overall thoughts about operations with integers.  What did each operation have in common?  What were the most difficult things to interpret or justify?  What interpretations or contexts made the most sense to you?


\end{problem}

\newpage
\begin{instructorNotes}
This activity is the third of three that investigate operations with negative numbers.  The first is ``Where Have the Integers Been?'' and the second is ``Integers and Subtraction''.  There is an accompanying reading (\url{https://ximera.osu.edu/mftsp19/elementaryTeachersOne}) on integers that lays out some of the more complicated framework, and the instructor notes for ``Where Have the Integers Been?'' contain background information about this unit.

The main purpose of this final activity is to talk about issues with multiplication, as well as clear up as much confusion and as many questions as we can.  Our goal is to have students be familiar with number lines and checks and bills by the end of the activity.

This activity's questions are intentionally similar to those on the other handouts about integers, so that we can quickly remind our students of the previous activity, and then select the problems where they need the most practice.  We do not often discuss all of the problems on this page.

With multiplication, we take extreme care with the checks and bills method.  Again, we attend carefully to the question being asked, to ensure that it is answered by the appropriate expression.  For instance, in the story, ``I receive 3 checks, each for 5 dollars.  What is my balance now?''.  Here, ``what is my balance?'' is not a good question for this story problem, but we could adjust it by explicitly starting with a zero balance.  In the same vein, the question ``how much has my balance gone down?'' should have a positive answer, even though the question suggests it should be negative.  We find that students need a lot of practice evaluating stories for correctness.

Stories that correspond with multiplication problems with a negative number in the first factor can still feel contrived to students.  If this comes up, we remind them again that all of these contexts have issues, and that none is perfect!  To help with this issue, you may want to think about starting with a zero balance and then adding or taking from this zero with multiplication stories.  

As we discussed with subtraction, we also would like to use this activity to help students see the reasoning behind the usual rule ``a negative times a negative gives a positive''.  Both the number line context as well as the checks and bills context can be used to help make this idea clear.

{\bf Suggested Timing:} This activity takes us an entire class period.  We often do a 5-minute reminder of the previous work, then give students 5 minutes on number lines, and then 15 minutes to discuss the number line context.  We then give 5-10 minutes for checks and bills, and the remainder of the time to discussion.  We often swap the order of these contexts as well, or just talk about checks and bills if the students are doing well with number lines.  If there are enough lingering questions, we sometimes spend a small amount of the next class clearing those up.
\end{instructorNotes}





\end{document}