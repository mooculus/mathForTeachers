\documentclass{ximera}

\title{Instructor Overview: Math for Elementary Teachers Activities}
\author{Vic Ferdinand, Betsy McNeal, Jenny Sheldon}

\begin{document}

\begin{abstract}\end{abstract}\maketitle

Many of these activities come from Ohio State University, where we have a two-semester sequence 
(Math 1125 and 1126) of mathematics content courses for preservice elementary school teachers.  This sequence is
also taken as general math courses for students preparing 
to teach Special Education in grades K-12 and for those preparing to teach subjects other than 
mathematics in middle school.  At Ohio State, the courses are each five credit 
hours, and run five days per week.  Many of the activities are designed to take the whole class 
period, or 55 minutes, while others use approximately half the class.  
The instructor notes attached to the activities often give a suggested amount of time for the 
activity, but reference the 55-minute ``class period''.

We have two main goals for the course.  First, we aim to have students refresh and deepen their 
understanding of elementary school mathematical procedures and concepts.  Although most of our students have low interest in mathematics and lack confidence in this subject area, they often THINK they already know elementary school math and tend to respond to a request 
for further explanation with either ``that's what I learned in school'' or ``I don't know''.  Throughout the 
course, we try to challenge their starting assumptions and provoke deeper reasoning by selecting 
text problems and supplements that cause our students to approach these basic ideas from fresh 
points of view.  We constantly ask students why a procedure works as it does and what it 
means.  We do not accept a formula as an explanation.  We require explanations that are
based on the meaning of a number (e.g., a fraction) and the meaning of an operation (e.g., 
multiplication as a number of groups with the same number of objects in each group).  

Our second goal is to encourage a positive attitude towards mathematics and an enriched view 
of what mathematics is all about.  We hope to move students towards an understanding of mathematics as a process of 
reasoning and away from a view of math as a body of 
rules and facts to be memorized. Such a view of mathematics is important for teachers to make a productive and 
creative classroom. A teacher who understands the intent of mathematics as a way of 
thinking will encourage her students to engage more deeply in that way of thinking.  

Both of these goals are addressed in our emphasis on explanation.  The more 
advanced students are challenged to view the focus on explanations and grade appropriate justifications as building toward the more rigorous proofs they experienced in high school.  
Students' explanations must follow directly and logically from the context in which we are working.  
All class meetings (lecture and recitation) at OSU are designed to get students thinking about, talking 
about, writing, and representing mathematical ideas.  This encourages deeper and more connected 
understanding of the mathematics on their part as well as relating the learning to their preparation as future teachers.  

Our class meetings are frequently composed of deep discussion about a handful of problems.  Since we also require 
a textbook (currently ``Mathematics for Elementary Teachers, 5ed'' by Sybilla Beckmann), we rely on 
students reading the text after class to help with background information or other details.  The 
classroom activities were designed with such background reading in mind.

As a sort of ``capstone'' experience, we designed a unit of projects for the end of the second 
course.  These projects help us to expose students to some more sophisticated content without 
having to require mastery of this content by all students.  They also help build problem-solving skills and 
encourage independent learning.  In the ``Instructor's Notes'' with some activities in the second 
course, you'll see references to places some of the content can be left for the projects at the end. 
These are indications that the content is more advanced and could equally well be skipped.

Some instructors might read our activities and initially think they are easy or superficial.  The purpose of these notes is to illustrate how these activities can be used to probe the depths of each topic in the traditional content for elementary mathematics teachers.  Our descriptions are drawn from our actual experiences of teaching many future teachers over many years.

Finally, we have more detailed solutions and Instructor's Notes for many of these activities than those 
which are publicly posted here.  We are happy to share those privately as well as answer any 
questions you may have about the activities, calendar, or other aspects of the course.  Please feel 
free to contact Vic Ferdinand, Betsy McNeal, or Jenny Sheldon at Ohio State for more information.


\end{document}
