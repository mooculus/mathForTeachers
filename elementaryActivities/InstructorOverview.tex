\documentclass{ximera}

\title{Instructor Overview: Math for Elementary Teachers Activities}

\begin{document}

\begin{abstract}\end{abstract}\maketitle

Many of these activities come from Ohio State, where we have a two-semester course sequence 
(Math 1125 and 1126) of mathematics content for preservice elementary school teachers 
(grades pre-K-3).  These courses are also taken as a general math course for students preparing 
to teach Special Education in grades K-12 and for those preparing to teach subjects other than 
mathematics in middle schools (grades 4-9).  At Ohio State, the courses are each five credit 
hours, and run five days per week.  Many of the activities are designed to take the whole class 
period, or 55 minutes, while others use approximately half the class, or about 25-30 minutes.  
The instructor notes attached to the activities often give a suggested amount of time for the 
activity, but reference the 55-minute ``class period''.

We have two main goals for the course.  First, we aim to have students refresh and deepen their 
understanding of elementary school mathematical procedures and concepts.  Most of our students 
are not specifically interested in mathematics and lack confidence in this subject area.  At the 
same time, they THINK they already know elementary school math and tend to respond to a request 
for further explanation with either ``it just is that way'' or ``I don't know''.  Throughout the 
course, we hope to challenge their starting assumptions and provoke deeper reasoning by selecting 
text problems and supplements that cause our students to approach these basic ideas from fresh 
points of view.  We will constantly ask students why a procedure works as it does and what it 
means.  We will not accept a formula as an explanation.  The explanations that we require will be 
based on the meaning of a number (e.g., a fraction) and the meaning of an operation (e.g., 
multiplication as a number of groups with the same number of objects in each group).  

Our second goal is to encourage a more positive attitude towards mathematics and an enriched view 
of what mathematics is all about.  We hope to move students away from a view of math as a body of 
rules and facts to be memorized and towards an understanding of mathematics as a process of 
reasoning. Such a view of mathematics is important for teachers not just to make a productive and 
creative classroom, but because a teacher who understands the intent of mathematics as a way of 
thinking will encourage her students to engage more deeply in that way of thinking.  

Both of these goals are addressed in our emphasis on explanation.  By the more mathematically 
advanced, the explanations we aim for can be thought of as building blocks toward rigorous proofs.  
Students' explanations must follow directly and logically from the context in which we are working.  
All class meetings (lecture and recitation) will be designed to get students thinking about, talking 
about, writing, and representing mathematical ideas.  This encourages deeper and more connected 
learning on their part as well as relating the learning to their own preparation as future teachers.  


\end{document}
