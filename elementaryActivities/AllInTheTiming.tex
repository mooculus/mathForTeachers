\documentclass{ximera}


\graphicspath{
  {./}
  {graphics/}
  {../graphics/}
}

\usepackage{chngcntr}

\let\question\relax
\let\endquestion\relax




\newtheoremstyle{SlantTheorem}{\topsep}{\fill}%%% space between body and thm
%\newtheoremstyle{SlantTheorem}{\topsep}{\topsep}%%% space between body and thm
 {\slshape}                      %%% Thm body font
 {}                              %%% Indent amount (empty = no indent)
 {\bfseries\sffamily}            %%% Thm head font
 {}                              %%% Punctuation after thm head
 {3ex}                           %%% Space after thm head
 {\thmname{#1}\thmnumber{ #2}\thmnote{ \bfseries(#3)}}%%% Thm head spec
\theoremstyle{SlantTheorem}
\newtheorem{question}{Question}
\counterwithin*{question}{section}



\let\instructorNotes\relax
\let\endinstructorNotes\relax
%%% instructorNotes environment
\ifhandout
\newenvironment{instructorNotes}[1][false]%
{%
\def\givenatend{\boolean{#1}}\ifthenelse{\boolean{#1}}{\begin{trivlist}\item}{\setbox0\vbox\bgroup}{}
}
{%
\ifthenelse{\givenatend}{\end{trivlist}}{\egroup}{}
}
\else
\newenvironment{instructorNotes}[1][false]%
{%
  \ifthenelse{\boolean{#1}}{\begin{trivlist}\item[\hskip \labelsep\bfseries {\Large Instructor Notes: \\} \hspace{\textwidth} ]}
{\begin{trivlist}\item[\hskip \labelsep\bfseries {\Large Instructor Notes: \\} \hspace{\textwidth} ]}
{}
}
{\end{trivlist}}
\fi


%% Suggested Timing
\newcommand{\timing}[1]{{\bf Suggested Timing: \hspace{2ex}} #1}


\title{All In The Timing}
\author{Vic Ferdinand, Betsy McNeal, Robin Pemantle, Jenny Sheldon}

\begin{document}
\begin{abstract} \end{abstract}
\maketitle



\begin{problem}
 You take your fifth grade class to the local carnival, where there are three games to rip off even the best of players.  These games cost 21 cents, 8 cents, and 18 cents, respectively, to play.  Your most pesky student, Gullible Gil, forgot to bring money from home and hounds you to lend some to him.  You lend him some and he gets addicted to the first game.  He plays until he's out of money, but then wants to play the second game.  To shut him up, you give him the same amount as before.  Gil comes back broke again - begging and whining to play the third game.  Once again, you give the same amount to Gil.  Of course, he blows all of this money at the third game with no prize for you.  How much money will you demand from Gil's mother, No Pay Pauline?
\end{problem}

\begin{problem}
 While trying to avoid Gil, you sneak in behind one of the slot machines and rig it to come up with three cherries (one in each window to hit the jackpot) at a time known only to you. You know the first window will come up with a cherry every 15 times, the second window every 24 times, and the third window every 40 times.  How many times should the machine be played before you step in to win your ill-gotten gain of $\$$1.98?

\end{problem}

\newpage
\begin{instructorNotes}
This is usually the first activity we do towards introducing the Least Common Multiple and Greatest Common Factor.  For this reason, students are not necessarily expecting to find either of these values when they first approach the problems.  Note that each problem has several answers, and we can highlight that sometimes we want the {\em least} common multiple, and sometimes we want just any common multiple.  The story situation should tell us the difference.

\begin{itemize}
	\item Most students came up with the idea of the LCM from listing by ``brute force''. This is a good starting method for the children they will teach, so is worthy of discussion.
	\item You might ask them to come up with a method if the numbers were ``not so nice''.  Give a hint about prime factorization of the numbers and what the prime factorization of a multiple (and then common multiple) would have to have in it.  If you have already begun talking about prime factorization, some students will have the idea to factor these numbers.
	\item Make sure they depart the activity with the idea that the LCM is a ``common time'' of things that grow at a constant (amount) rate (``slope with $y$-intercept = 0'')
\end{itemize}
\end{instructorNotes}




\end{document}