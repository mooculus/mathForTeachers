\documentclass[nooutcomes,noauthor]{ximera}
\usepackage{gensymb}
\usepackage{tabularx}
\usepackage{mdframed}
\usepackage{pdfpages}
%\usepackage{chngcntr}

\let\problem\relax
\let\endproblem\relax

\newcommand{\property}[2]{#1#2}




\newtheoremstyle{SlantTheorem}{\topsep}{\fill}%%% space between body and thm
 {\slshape}                      %%% Thm body font
 {}                              %%% Indent amount (empty = no indent)
 {\bfseries\sffamily}            %%% Thm head font
 {}                              %%% Punctuation after thm head
 {3ex}                           %%% Space after thm head
 {\thmname{#1}\thmnumber{ #2}\thmnote{ \bfseries(#3)}} %%% Thm head spec
\theoremstyle{SlantTheorem}
\newtheorem{problem}{Problem}[]

%\counterwithin*{problem}{section}



%%%%%%%%%%%%%%%%%%%%%%%%%%%%Jenny's code%%%%%%%%%%%%%%%%%%%%

%%% Solution environment
%\newenvironment{solution}{
%\ifhandout\setbox0\vbox\bgroup\else
%\begin{trivlist}\item[\hskip \labelsep\small\itshape\bfseries Solution\hspace{2ex}]
%\par\noindent\upshape\small
%\fi}
%{\ifhandout\egroup\else
%\end{trivlist}
%\fi}
%
%
%%% instructorIntro environment
%\ifhandout
%\newenvironment{instructorIntro}[1][false]%
%{%
%\def\givenatend{\boolean{#1}}\ifthenelse{\boolean{#1}}{\begin{trivlist}\item}{\setbox0\vbox\bgroup}{}
%}
%{%
%\ifthenelse{\givenatend}{\end{trivlist}}{\egroup}{}
%}
%\else
%\newenvironment{instructorIntro}[1][false]%
%{%
%  \ifthenelse{\boolean{#1}}{\begin{trivlist}\item[\hskip \labelsep\bfseries Instructor Notes:\hspace{2ex}]}
%{\begin{trivlist}\item[\hskip \labelsep\bfseries Instructor Notes:\hspace{2ex}]}
%{}
%}
%% %% line at the bottom} 
%{\end{trivlist}\par\addvspace{.5ex}\nobreak\noindent\hung} 
%\fi
%
%


\let\instructorNotes\relax
\let\endinstructorNotes\relax
%%% instructorNotes environment
\ifhandout
\newenvironment{instructorNotes}[1][false]%
{%
\def\givenatend{\boolean{#1}}\ifthenelse{\boolean{#1}}{\begin{trivlist}\item}{\setbox0\vbox\bgroup}{}
}
{%
\ifthenelse{\givenatend}{\end{trivlist}}{\egroup}{}
}
\else
\newenvironment{instructorNotes}[1][false]%
{%
  \ifthenelse{\boolean{#1}}{\begin{trivlist}\item[\hskip \labelsep\bfseries {\Large Instructor Notes: \\} \hspace{\textwidth} ]}
{\begin{trivlist}\item[\hskip \labelsep\bfseries {\Large Instructor Notes: \\} \hspace{\textwidth} ]}
{}
}
{\end{trivlist}}
\fi


%% Suggested Timing
\newcommand{\timing}[1]{{\bf Suggested Timing: \hspace{2ex}} #1}




\hypersetup{
    colorlinks=true,       % false: boxed links; true: colored links
    linkcolor=blue,          % color of internal links (change box color with linkbordercolor)
    citecolor=green,        % color of links to bibliography
    filecolor=magenta,      % color of file links
    urlcolor=cyan           % color of external links
}
\title{Comparing fractions}

\begin{document}
\begin{abstract}

\end{abstract}

\maketitle


We used a common denominator to compare fractions in a previous activity, but there are other strategies we can use. To solve these problems, we would like you to use methods other than making a common denominator. You may make equivalent fractions to help you think about the problems, but your strategy must include additional reasoning.

\begin{problem}
Which fraction is larger, $\frac{1}{12}$ or $\frac{1}{13}$? Use our meaning of fractions to explain how you know.
\end{problem}


\begin{problem}
Which fraction is larger, $\frac{1}{93}$ or $\frac{1}{127}$? Use our meaning of fractions to explain how you know.
\end{problem}


\begin{problem}
Which fraction is larger, $\frac{35}{64}$ or $\frac{37}{52}$? Use our meaning of fractions to explain how you know.
\end{problem}



\begin{problem}
Which fraction is larger, $\frac{41}{80}$ or $\frac{43}{90}$? Use our meaning of fractions to explain how you know.
\end{problem}



\begin{problem}

Imran says that $\frac{23}{24}$ is larger than $\frac{19}{20}$ because $\frac{23}{24}$ has more pieces. Do you agree with Imran? Explain your thinking using our meaning of fractions.

\end{problem}


\begin{problem}
Joshua says that $\frac{14}{14}$ is smaller than $\frac{16}{16}$ because $14 < 16$. Do you agree with Joshua? Explain your thinking using our meaning of fractions.
\end{problem}


\begin{problem}
We are likely to ask you similar questions to those on this page. For each of the problems above, write a different fraction comparison problem which is solved using the strategy of that problem. If you have time, trade problems with a neighboring team and solve them!
\end{problem}





\newpage

\begin{instructorNotes} 



{\bf Main goal:} Use our fraction definition to compare fractions.

{\bf Overall picture:} For the duration of these activities, we'd like to focus on using the meaning of fractions. So, we'll outlaw  the method of common denominators for now. On homework, problems will specify what methods students are allowed or not allowed to use.




\begin{itemize}
	\item In problems 1 and 2, we work essentially with the denominators only as we have the same numerators for these fractions. We can think only about the relative sizes of the pieces involved, not how many there are. It helps to create a story problem to accompany these problems, particularly about cake or brownies or pizza, etc. If you hear a group discussing these ideas, be sure to have them discuss it with the class. If you don't hear anyone using ideas like these, you should bring them up yourself.
	\item We need to emphasize that it doesn't make sense to compare fractions unless they have the same wholes. For instance, $\frac13$ of a pan of brownies shouldn't be compared with $\frac14$ of the area in the backyard. Similarly, we can't compare $\frac13$ of a small pan of brownies with $\frac15$ of a large pan of brownies. Once the two pans are the same size, we can say things like, ``would you rather have a brownie from the pan that got cut into four equal-sized pieces, or from the pan that got cut into five equal-sized pieces?'' Since the pans are the same size, we know which pieces are larger.
	\item In problems 3 -- 5, we have several different types of strategies we would like to highlight. The first is the idea of more larger pieces. We know that $\frac{35}{64} < \frac{37}{52}$ because each $\frac{1}{52}$ is larger than each $\frac{1}{64}$, and we have more of these larger pieces.
	\item The second strategy is using what we can call a benchmark. We see that $\frac{41}{80} > \frac{43}{90}$ because $\frac{41}{80}$ is more than half, and $\frac{43}{90}$ is less than half. Here students may benefit from noticing that $\frac{1}{2}$ is equivalent to $\frac{40}{80}$, for instance. This is allowed because we are not making common denominators and are using other reasoning than just making equivalent fractions.
	\item The third strategy is thinking about how much of the whole is missing rather than how much is present. For instance, $\frac{23}{24}>\frac{19}{20}$ because we are missing one piece from each whole. The $\frac{1}{20}$ piece is a larger missing piece, so we have less of the pan of brownies overall. This problem also combats the student misconception that a larger numerator means a larger fraction.
	\item The final problem should be simple for our students but is a common misconception from children that should be discussed.
	\item Some students might want to draw pictures to see which shaded portion is larger. It's worth discussing this method, which works well for some examples and not as well for other examples. We would rather not rely on the accuracy of our drawings if we don't need to! (After all, we tend to not produce very accurate drawings, especially when the denominators are large.)
\end{itemize}




{\bf Good language:}  Continue to remind students about our definition of fractions, and help them to see how the definition is connected to the conclusions we are making about which fraction is larger.

Help students to express out loud what we are looking for when we say one fraction is larger than the other: we have a greater portion of our whole (in the case that we have less than a whole), or we have a greater total quantity.



{\bf Suggested timing:} Give students about 5 minutes to work on problems 1-2, and then take about 10 minutes to discuss. Then give students about 10 minutes to work on the rest of the problems, and use the rest of the class time for discussion. The last question is a bonus question for students who work quickly and does not need to be discussed unless most groups get to this problem.




\end{instructorNotes}



\end{document}