\documentclass{ximera}


\graphicspath{
  {./}
  {graphics/}
  {../graphics/}
}

\usepackage{chngcntr}

\let\question\relax
\let\endquestion\relax




\newtheoremstyle{SlantTheorem}{\topsep}{\fill}%%% space between body and thm
%\newtheoremstyle{SlantTheorem}{\topsep}{\topsep}%%% space between body and thm
 {\slshape}                      %%% Thm body font
 {}                              %%% Indent amount (empty = no indent)
 {\bfseries\sffamily}            %%% Thm head font
 {}                              %%% Punctuation after thm head
 {3ex}                           %%% Space after thm head
 {\thmname{#1}\thmnumber{ #2}\thmnote{ \bfseries(#3)}}%%% Thm head spec
\theoremstyle{SlantTheorem}
\newtheorem{question}{Question}
\counterwithin*{question}{section}



\let\instructorNotes\relax
\let\endinstructorNotes\relax
%%% instructorNotes environment
\ifhandout
\newenvironment{instructorNotes}[1][false]%
{%
\def\givenatend{\boolean{#1}}\ifthenelse{\boolean{#1}}{\begin{trivlist}\item}{\setbox0\vbox\bgroup}{}
}
{%
\ifthenelse{\givenatend}{\end{trivlist}}{\egroup}{}
}
\else
\newenvironment{instructorNotes}[1][false]%
{%
  \ifthenelse{\boolean{#1}}{\begin{trivlist}\item[\hskip \labelsep\bfseries {\Large Instructor Notes: \\} \hspace{\textwidth} ]}
{\begin{trivlist}\item[\hskip \labelsep\bfseries {\Large Instructor Notes: \\} \hspace{\textwidth} ]}
{}
}
{\end{trivlist}}
\fi


%% Suggested Timing
\newcommand{\timing}[1]{{\bf Suggested Timing: \hspace{2ex}} #1}


\title{Sharing cookies}

\begin{document}
\begin{abstract}\end{abstract}
\maketitle

\begin{problem}
You went to your favorite cookie shop and purchased a cinnamon crunch cookie and a cookies and cream cookie. You are taking these cookies to a dinner party that will have four people attending. How should you cut the cookies for the dinner party, and why?
\end{problem}


\begin{problem}
What fraction of the cookie purchase will each person receive, and why? Use our meaning of fractions to explain.
\end{problem}


\begin{problem}
Would your answer to the previous question change if you bought three cookies instead of two? Explain your thinking using pictures and our meaning of fractions.
\end{problem}

\vfill
Wait for your instructor before continuing on the next two problems.
\newpage

\begin{problem}
On a different day, you bought the two cookies from two different cookie shops, and these cookies are not even close to the same size. You are still taking the cookies to a dinner party with four people attending. How would you cut the cookies, and what fraction of the cookie purchase will each person receive?
\end{problem}


\begin{problem}
  Your student Kelsey cut each cookie into $4$ equal pieces and then gave each person one piece from each cookie to make $\frac{1}{4}$.  She justified her answer by saying, ``This represents $\frac{1}{4}$ of the cookie purchase because I added one piece from each cookie and   $\frac{1}{8}+\frac{1}{8}= \frac{2}{8} =\frac{1}{4}$."  Do you agree with her statement?  Why or why not?  How will you respond to her idea?
  
  
  
  
\end{problem}
\newpage

\begin{instructorNotes}
The key point here again is the meaning of ``denominator'' and numerator'' and paying attention to what the fraction(s) is ``of''.  


 \begin{itemize}
	\item This activity is pretty hard for the students.  Students have difficulty comprehending that we are looking for $\frac14$ of the {\em combined} amount.  The key idea is to see that fair sharing of two different cookies would mean that each person will get an equal sized piece from each of the cookies. 
	\item At some point, it should come out in the discussion that $\frac14$ of the combined amount of cookies (the cookie purchase) would look like $\frac14$ of each cookie on one person's plate.  That is, we want to make sure it is understood that the sensible real-life answer is deemed acceptable in math class as well!
	\item Some students will have the idea of splitting each cookie into $2$ parts (for a total of $4$ equal parts when the cookies are the same size) and taking one of these.  Other students, like the person in problem 5,  may split each cookie into 4 parts, take one piece from each cookie, and then incorrectly argue that they are correct because $\frac{1}{8} + \frac{1}{8} = \frac{2}{8} = \frac14$.  This is a great opportunity to ask again what the $\frac14$ or the $\frac{1}{8}$ means and remind them that we need 4 or 8 equal parts of the whole (in this case, the {\em combined} amount of cookies). Make sure to emphasize the wholes for these fractions -- the whole for the $\frac{1}{8}$ is different from the whole for $\frac{1}{4}$!
	\item The question about cookies that are different sizes might bring out some of these issues. Some students may argue that the cookies are the same size in the first problem, which is okay. You might ask them if they think it's fair for the people at the dinner party to not be able to try both kinds of cookies, but if the cookies are equally sized they might think this is fair. However, when the cookies are not the same size in problem 4, they will need to adjust their strategy.
	\item To extend this discussion, you can change the number of people at the dinner party, the numerator of the fraction, or the number of cookies purchased. The issue of the number of cookies purchased is dealt with in problem 3, but this can still be extended. For instance, perhaps we are going to a dinner party with $8$ people and we bought $6$ cookies. We want to know what fraction of the cookies will go to the three people we invited to the party (to get $\frac{3}{8}$ of the cookie purchase). 
	\item A further extension would be to combine objects that are different shapes, like a cake and a pie.
	\item These problems are an early look at the distributive property (of multiplication over addition), i.e., $\frac56(\texttt{big}) + \frac56(\texttt{little}) = \frac56(\texttt{big} + \texttt{little})$.  We can return to this example later in the semester when we discuss these properties since we aren't ready to talk about multiplying fractions just yet.
	
 
\end{itemize}

{\bf Suggested Timing:} 
About 10 minutes of work in small groups on problems 1 -- 3, followed by 15-20 minutes in whole class discussion. Repeat with problems 4 and 5.
\end{instructorNotes}
\end{document}