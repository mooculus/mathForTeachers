\documentclass[noauthor,nooutcomes]{ximera}

\usepackage{gensymb}
\usepackage{tabularx}
\usepackage{mdframed}
\usepackage{pdfpages}
%\usepackage{chngcntr}

\let\problem\relax
\let\endproblem\relax

\newcommand{\property}[2]{#1#2}




\newtheoremstyle{SlantTheorem}{\topsep}{\fill}%%% space between body and thm
 {\slshape}                      %%% Thm body font
 {}                              %%% Indent amount (empty = no indent)
 {\bfseries\sffamily}            %%% Thm head font
 {}                              %%% Punctuation after thm head
 {3ex}                           %%% Space after thm head
 {\thmname{#1}\thmnumber{ #2}\thmnote{ \bfseries(#3)}} %%% Thm head spec
\theoremstyle{SlantTheorem}
\newtheorem{problem}{Problem}[]

%\counterwithin*{problem}{section}



%%%%%%%%%%%%%%%%%%%%%%%%%%%%Jenny's code%%%%%%%%%%%%%%%%%%%%

%%% Solution environment
%\newenvironment{solution}{
%\ifhandout\setbox0\vbox\bgroup\else
%\begin{trivlist}\item[\hskip \labelsep\small\itshape\bfseries Solution\hspace{2ex}]
%\par\noindent\upshape\small
%\fi}
%{\ifhandout\egroup\else
%\end{trivlist}
%\fi}
%
%
%%% instructorIntro environment
%\ifhandout
%\newenvironment{instructorIntro}[1][false]%
%{%
%\def\givenatend{\boolean{#1}}\ifthenelse{\boolean{#1}}{\begin{trivlist}\item}{\setbox0\vbox\bgroup}{}
%}
%{%
%\ifthenelse{\givenatend}{\end{trivlist}}{\egroup}{}
%}
%\else
%\newenvironment{instructorIntro}[1][false]%
%{%
%  \ifthenelse{\boolean{#1}}{\begin{trivlist}\item[\hskip \labelsep\bfseries Instructor Notes:\hspace{2ex}]}
%{\begin{trivlist}\item[\hskip \labelsep\bfseries Instructor Notes:\hspace{2ex}]}
%{}
%}
%% %% line at the bottom} 
%{\end{trivlist}\par\addvspace{.5ex}\nobreak\noindent\hung} 
%\fi
%
%


\let\instructorNotes\relax
\let\endinstructorNotes\relax
%%% instructorNotes environment
\ifhandout
\newenvironment{instructorNotes}[1][false]%
{%
\def\givenatend{\boolean{#1}}\ifthenelse{\boolean{#1}}{\begin{trivlist}\item}{\setbox0\vbox\bgroup}{}
}
{%
\ifthenelse{\givenatend}{\end{trivlist}}{\egroup}{}
}
\else
\newenvironment{instructorNotes}[1][false]%
{%
  \ifthenelse{\boolean{#1}}{\begin{trivlist}\item[\hskip \labelsep\bfseries {\Large Instructor Notes: \\} \hspace{\textwidth} ]}
{\begin{trivlist}\item[\hskip \labelsep\bfseries {\Large Instructor Notes: \\} \hspace{\textwidth} ]}
{}
}
{\end{trivlist}}
\fi


%% Suggested Timing
\newcommand{\timing}[1]{{\bf Suggested Timing: \hspace{2ex}} #1}




\hypersetup{
    colorlinks=true,       % false: boxed links; true: colored links
    linkcolor=blue,          % color of internal links (change box color with linkbordercolor)
    citecolor=green,        % color of links to bibliography
    filecolor=magenta,      % color of file links
    urlcolor=cyan           % color of external links
}


\title{Explanations}
\author{Vic Ferdinand, Betsy McNeal, Jenny Sheldon}

\begin{document}
\begin{abstract} 
%This activity will help you begin to think about what it means to write an explanation for the answer to a mathematics problem. 
Welcome to Math 1125! In this course, we will investigate the mathematics content typically needed to teach grades K -- 5. Throughout the course, we would like you to work together with your peers in class. Please form a group of 3-4 people to work with for today.
\end{abstract}
\maketitle



\begin{question}
We would like to be a community. Please start by introducing yourself to your group members. What do you like to be called? (Consider including your preferred pronouns!) What you are currently thinking you might like to teach? What is something interesting about you?
\end{question}


\begin{question}
It's always helpful when you begin group work to make sure you'll be ready to discuss your work with the class when we are finished with work time. For today, who in your group will take down notes? Will those notes be shared or sent to the other group members? Who in your group will speak for the group? (These roles should change with each class, even if you work with the same group!)
\end{question}


\begin{question}
While we will not discuss teaching methods in this course, we won't ignore the fact that as teachers, we are always learning from those around us. What teachers in your past have inspired you, and why were they inspirational?
\end{question}


\begin{question}
Everyone comes to this course with different experiences with mathematics. What has been your own experience? How do you imagine your own experiences might influence your future classroom?
\end{question}


\begin{question}
Ultimately, we are here to learn together. What does it actually mean to learn something? Discuss a time when you learned something. Your example could be something you learned in school, learning to play an instrument or a sport, learning to do a new job, or anything else! What was the process of learning like? How did you know that you had learned?
\end{question}




\newpage


Now, let's do some math! How many squares are in the following image?
\begin{center}
\begin{tikzpicture}
    \draw (0,0)--(3,0)--(3,3)--(0,3)--(0,0);
    \draw (0,1)--(3,1);
    \draw (0,2)--(3,2);
    \draw (1,0)--(1,3);
    \draw (2,0)--(2,3);
\end{tikzpicture}
\end{center}


\begin{problem} 
Spend a few minutes thinking about this problem either by yourself or by discussing the problem with your group.  After you have a guess as to what the answer might be, discuss the following questions.
\begin{enumerate}
\item Did each member of your group come up with the same answer?  Did each member come to their answer in the same way?



\item
Pick one method and one answer from your group.  Discuss this method in detail, as if you were presenting it to the class.  What ideas are the most important?  What steps are the most complicated?

\end{enumerate}
\end{problem}

\begin{problem} 
Now imagine that you are a teacher, and you have given this problem to your students.  On the next page are four example explanations that you could get from your students.  With your group, discuss the following.
\begin{itemize}
\item Compare and contrast these four explanations. What do they have in common? What is different about them?
\item What do you like about the explanations? What do you dislike? Why?
\item From this work, what can you determine that each student understands about this problem?
\item What feedback would you give to each student to improve their explanation?
\end{itemize}
\end{problem}
\newpage


\parbox{2.5in}{
\begin{mdframed}

{\bf Explanation 1:}
\begin{center}
\begin{tikzpicture}[scale=0.6]
    \draw (0,0)--(3,0)--(3,3)--(0,3)--(0,0);
    \draw (0,1)--(3,1);
    \draw (0,2)--(3,2);
    \draw (1,0)--(1,3);
    \draw (2,0)--(2,3);
    \node at (0.5, 2.5) {$1$};
    \node at (1.5, 2.5) {$2$};
    \node at (2.5, 2.5) {$3$};
    \node at (0.5, 1.5) {$4$};
    \node at (1.5, 1.5) {$5$};
    \node at (2.5, 1.5) {$6$};
    \node at (0.5, 0.5) {$7$};
    \node at (1.5, 0.5) {$8$};
    \node at (2.5, 0.5) {$9$};
\end{tikzpicture}
\end{center}

I counted 9 squares in the picture, and labeled them on the figure.

\end{mdframed}}
\parbox{2in}{
\begin{mdframed}
{\bf Explanation 2:}

$$ 14 = 3\times 3 + 2\times 2 + 1 $$


\end{mdframed}}

%\parbox{5in}{
\begin{mdframed}
{\bf Explanation 3:}

First, I remember that to be a ``square'', you have to have the same length on both sides.  The easiest squares to see are the tiny ones, and I can see three rows with three tiny squares in each of those rows for a total of nine tiny squares.  The next easiest square to see is the biggest one: the entire picture is a square!  Lastly, there are some medium-sized squares.  Look at my picture to see an example.  I found four of these medium-sized squares: two with their tops on the top row, and two with their bottoms on the bottom row.  So, there are 14 total squares.


\begin{center}
\begin{tikzpicture}[scale=0.6]
    \draw (0,0)--(3,0)--(3,3)--(0,3)--(0,0);
    \draw (0,1)--(3,1);
    \draw (0,2)--(3,2);
    \draw (1,0)--(1,3);
    \draw (2,0)--(2,3);
    \draw[very thick] (0,1)--(2,1)--(2,3)--(0,3)--(0,1);
\end{tikzpicture}
\end{center}

I'm convinced these are all of the squares in the figure, because the tiny ones are $1\times 1$, the medium ones are $2 \times 2$, and the large one is $3\times 3$, which is the entire original figure.
\end{mdframed}
%}
%\parbox{2.25in}{

\begin{mdframed}
{\bf Explanation 4:}

\begin{tabularx}{\textwidth}{X|X}
    Step & Why  \\ \hline \hline
    We are looking for squares. & The problem tells us so. \\ \hline
    Squares have the same length on both sides. & That's what it means to be a square. \\ \hline
    There are 9 tiny squares. & We count the $1\times 1$ squares. \\ \hline
    There are 4 medium square. & We count the $2 \times 2$ squares. \\ \hline
    There is 1 big square. & We count the entire square. \\ \hline
    There are 14 squares total. & We add up the squares we found. \\
\end{tabularx}
\end{mdframed}




\newpage


\begin{instructorNotes}


{\bf Main goal:} We use this activity as an introduction to the course, a first experience at sharing their thoughts with a small group and the class, and a first experience of what a mathematical explanation can look like. As long as the students are talking to each other and sharing their thoughts with the class, this activity is a success.

{\bf Overall picture:} 
\begin{itemize}
	\item Spend a few minutes introducing yourself to the class, passing out the syllabus, and dealing with overall course information. You can quickly bring up any area of the syllabus you feel would help the students understand the activity, but we actually recommend asking students to read the syllabus on their own and bring questions to class the next day.
	\item This activity is designed to be done page-by-page. You can print out copies to hand out in class if you'd like, and it may help to hand out one page at a time. If the students are working from a digital copy, be sure to ask them not to move on until you ask them to.
\end{itemize}
	
Page 1:
\begin{itemize}	
	\item You can do the first question (introductions) together as a class if you prefer. This gives the entire class the opportunity to hear everyone's name and interests, and emphasizes the fact that we are trying to build a community here!
	\item The purpose of the first page is to help introduce the overarching goals of the course. We want to hear answers from some (if not all) of the groups here, to practice hearing from others and to practice valuing what the students have to contribute. You don't need to add anything to what they say, or guide the discussion in any particular direction, but it's good to keep in mind the overall goals of the course. The question about roles in group work sets students up for success in their groups in the future. The question about teachers that have inspired them should hopefully bring up things they would like to learn to do (whether they are content-related or otherwise). It would be especially nice if here the students begin to discuss themes around the idea that knowing math for teaching is different from knowing math as a student. If the students take the discussion in this direction, please encourage it! However, we're not expecting this topic to necessarily come up. The question about past experiences with mathematics can help you get a sense of who your students are and how your discussions might go in the future. The question about learning hopefully brings up the idea that learning nearly always requires making mistakes, among other good learning practices.
	\item See the end of the document for timing suggestions on this! You don't need to feel pressured to discuss every aspect of each of these questions, and students don't need to feel like they have ``finished'' these problems before you move on.
	\item Your whole-class discussion should give most (not necessarily all) of the groups a chance to share. You can accomplish this in any way that seems comfortable to you. One way to structure the discussion is to begin with question 3 and ask a particular group to summarize their discussion. After they have finished, ask another particular group to summarize their discussion, or ask more generally if anyone would like to also share their thoughts. Then move on to question 4 and repeat, and question 5 and repeat again. If you have 6 groups in class, you could ask two groups to share per problem so that each group has a chance.
\end{itemize}

Page 2:
\begin{itemize}
\item   We have found that giving students such concrete examples helps them begin to form their own explanations.


\item We sometimes put the initial problem on the board for students to solve before directing them to look at the activity.  We also encourage students to work in groups, as this course may be the first time they have done so in a math class.  This question is designed to give everyone in the group a chance to speak.

\item After students have had a chance to discuss in small groups, we try to have students who are willing present their solutions at the board.  This way, we can discuss the different answers, and see if anyone came up with other answers during the process of solving.  This also gives a good opportunity to highlight the problem-solving process which will serve students well throughout the course.
\end{itemize}

Page 3:
\begin{itemize}
\item 
Looking at the provided explanations is often eye-opening for students.  Explanation 1 helps students to see how we might deal with an incorrect solution, with a partial explanation.  Explanation 2 is what they might have written in other math classes; we want to emphasize that this will not be enough in our course!  Explanations 3 and 4 are different ways of writing a nearly complete explanation, with many details, but the ideas are organized in fundamentally different ways.  This question can also bring up the grading system in that we are looking for students to demonstrate full understanding with their solutions. Discussing what kinds of understanding are being demonstrated (in contrast to what the person knows or doesn't know) should help students begin to see what we are looking for in their explanations!
\end{itemize}


{\bf Good language:}                                                                                                                                                   
Use encouraging language as often as possible. Many students are nervous about sharing their thoughts with strangers! Be sure to thank students for their contributions. Also, keep in mind that each person is an expert in their own thoughts and experiences, so we don't want to take over the conversation as an instructor. You can add your own thoughts to the discussion, and try to point students in helpful directions, but we want more student discussion than instructor discussion.






{\bf Suggested Timing:} Give students about 5-10 minutes to discuss page 1, then take 10 minutes to discuss as a class. Repeat this process for the other two pages, covering as much ground as you have time for! You do not need to ``finish'' this activity.

%After the course introduction (which sometimes takes about half a class period), we use the rest of the first day for this activity.

\end{instructorNotes}

\end{document}
