\documentclass{ximera}


\graphicspath{
  {./}
  {graphics/}
  {../graphics/}
}

\usepackage{chngcntr}

\let\question\relax
\let\endquestion\relax




\newtheoremstyle{SlantTheorem}{\topsep}{\fill}%%% space between body and thm
%\newtheoremstyle{SlantTheorem}{\topsep}{\topsep}%%% space between body and thm
 {\slshape}                      %%% Thm body font
 {}                              %%% Indent amount (empty = no indent)
 {\bfseries\sffamily}            %%% Thm head font
 {}                              %%% Punctuation after thm head
 {3ex}                           %%% Space after thm head
 {\thmname{#1}\thmnumber{ #2}\thmnote{ \bfseries(#3)}}%%% Thm head spec
\theoremstyle{SlantTheorem}
\newtheorem{question}{Question}
\counterwithin*{question}{section}



\let\instructorNotes\relax
\let\endinstructorNotes\relax
%%% instructorNotes environment
\ifhandout
\newenvironment{instructorNotes}[1][false]%
{%
\def\givenatend{\boolean{#1}}\ifthenelse{\boolean{#1}}{\begin{trivlist}\item}{\setbox0\vbox\bgroup}{}
}
{%
\ifthenelse{\givenatend}{\end{trivlist}}{\egroup}{}
}
\else
\newenvironment{instructorNotes}[1][false]%
{%
  \ifthenelse{\boolean{#1}}{\begin{trivlist}\item[\hskip \labelsep\bfseries {\Large Instructor Notes: \\} \hspace{\textwidth} ]}
{\begin{trivlist}\item[\hskip \labelsep\bfseries {\Large Instructor Notes: \\} \hspace{\textwidth} ]}
{}
}
{\end{trivlist}}
\fi


%% Suggested Timing
\newcommand{\timing}[1]{{\bf Suggested Timing: \hspace{2ex}} #1}


\title{Improper Fractions}

\begin{document}
\begin{abstract}\end{abstract}
\maketitle



\begin{problem}
 Show a plot of land that is $\frac{5}{3}$ the area shown below.  Is there more than one possible correct answer? Justify your answer.

\begin{image}
\begin{tikzpicture}
\draw[thick] (0,0) rectangle (5, 2);
\end{tikzpicture}
\end{image}
\end{problem}

\begin{problem}
 Use the picture below to answer the following questions. In the picture, you may assume that pieces which look equal are actually equal. 
\begin{enumerate}
\item What fraction is shape $A$ of the entire drawing?
\item  What fraction is the whole square of shape $B$?
\end{enumerate}

\begin{image}
\begin{tikzpicture}
\draw[thick, fill=orange] (0,2)--(1,2)--(1,3)--(2,3)--(2,4)--(0,4)--(0,2);
\draw[thick, fill=blue] (3,0) rectangle (4,1);
\draw[thick] (0,0) rectangle (4,4);
\draw[thick] (0,2)--(4,2);
\draw[thick] (2,0)--(2,4);
\draw[thick] (2,3)--(4,3);
\draw[thick] (3,2)--(3,4);
\node at (1, 3.5) {$B$};
\node[text=white] at (3.5, 0.5) {$A$};

\end{tikzpicture}
\end{image}
\end{problem}
\newpage

\begin{problem} In the figure below, the top row shows a shape we will call the original.  Shapes $A$, $B$, and $C$ are other shapes which we will compare to the original. 
\begin{enumerate}
\item What fraction is shape $A$ of the original?
\item  What fraction is shape $B$ of the original?
\item What fraction is shape $C$ of the original?
\end{enumerate}

\begin{image}
\begin{tikzpicture}
\draw[thick] (0, 6) rectangle (1, 8);
\node[left] at (0, 7) {original};
\draw[thick] (4, 6) rectangle (5, 7);
\node[left] at (4, 6.5) {shape $A$};
\draw[thick] (8, 6) grid (11, 8);
\node[left] at (8, 7) {shape $B$};
\draw[thick] (4, 2) rectangle (5, 3);
\draw[thick] (5, 2) grid (8, 4);
\node[left] at (4, 3) {shape $C$};
\end{tikzpicture}
\end{image}


\end{problem}

\begin{problem}

Denise, Paige, and Charles each cut a length of string. The length of Denise's string is shown by the line below. 
\begin{image} \begin{tikzpicture}
\draw[thick] (0,0)--(5,0);
\end{tikzpicture}\end{image}
\vskip 1in
\begin{enumerate}
	\item Paige's string is $\frac{5}{4}$ of the length of Denise's string. Use Denise's line to draw the length of Paige's string.
	\item What is Denise's string length as a fraction of Paige's string length? Explain your thinking.
	\item Charles' string is $\frac{2}{5}$ of the length of Paige's string. Draw a line that has the same length as Charles' string.
	\item What is Charles' string length as a fraction of Denise's string length? Explain your thinking.
\end{enumerate}
\end{problem}




\newpage

\begin{instructorNotes}

{\bf Main goal:} We would like to see that an ``improper'' fraction is a fraction and we can use the same definition of fractions that we have been using for ``proper'' fractions.

{\bf Overall notes:} We want students to emphasize here that the numerator tells us how many copies to make of one of the equal pieces of the whole. For example, think of $\frac{5}{4}$ as 5 pieces, each of size $\frac{1}{4}$ of whatever the whole is. This language tends to be easier to interpret than the ``out of'' language that students can be more familiar with before our course. For each of the problems in this activity, be sure to have students discuss how and where they are seeing the whole, denominator, and numerator of the fraction.
\begin{itemize}
	\item The first problem is typically not too difficult for students, but is a good warm-up to practice with the definition of fractions. You may see students drawing different-looking pictures of the resulting land; if so make sure to present the different pictures and have students discuss whether all of the above are valid answers. Here, it's also good to start emphasizing how the figure is labeled, since with improper fractions it can be hard to tell at a glance which part of the diagram refers to the whole.
	\item The second problem can be the most difficult for students because of the wording. However, it's essential that students learn to determine the whole for a fraction from the wording of a problem. In part (a), the whole is the entire drawing, but in part (b) the whole is just shape $B$. It can be very helpful to walk around and check answers for this particular problem, and also in discussion to have students talk in discussion about how they decided from the wording what they were taking as the whole in each part. If you have some groups who got an incorrect answer in part (b) and then changed their mind, it can be especially helpful to have these groups present and to talk about what they got at first and why they changed their minds.
	\item The third problem tends to be an easier problem if the students feel comfortable with the first two problems, so the discussion here can be quick.
	\item The fourth problem returns to the issue of thinking about the whole, then the denominator, then the numerator. As we move from one part to the next, we have to keep track of each person's whole. The final question about what whole to use when comparing  Charles  to Denise  may need to be clarified with students as you walk around.
\end{itemize}

Wrapping up this activity, you might want to have a short discussion on when it makes sense to have improper fractions and when it doesn't. For instance, Paige's string can be $\frac{5}{4}$ of Denise's length, but we cannot purchase $\frac{5}{4}$ of all the almonds available at the grocery store (unless we go to another grocery store that has a larger amount of almonds).

{\bf Good language:} As we've mentioned repeatedly above, this activity helps us to continue emphasizing that the whole for a fraction should always be considered.  As we change the whole, even if we don't change the amount being considered, the fraction itself should change.  Or, put another way, we can't know what fraction is represented in a picture unless we know what is being considered as the whole.


{\bf Suggested Timing:} This activity can be flexible depending on how much time we have for it. Problems 2 and 4 are the most important to discuss, so a shortened version of this activity (about half a class) could be giving about 10 minutes for students to think about problems 2 and 4, then 20 minutes in presentation and discussion.

Alternatively, if you have the entire class for this activity, give students about 15 minutes to work through the four problems, then spend about 30 minutes in presentations and discussion, allowing each group to present and possibly having multiple presentations for each problem (depending on the language students are using and the pictures they are drawing). Wrap up with 10 minutes of discussion clarifying any issues, emphasizing the meaning of fractions, and perhaps discussing the use of improper fractions in real-life situations.


\end{instructorNotes}


\end{document}