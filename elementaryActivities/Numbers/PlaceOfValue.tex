\documentclass{ximera}

\usepackage{gensymb}
\usepackage{tabularx}
\usepackage{mdframed}
\usepackage{pdfpages}
%\usepackage{chngcntr}

\let\problem\relax
\let\endproblem\relax

\newcommand{\property}[2]{#1#2}




\newtheoremstyle{SlantTheorem}{\topsep}{\fill}%%% space between body and thm
 {\slshape}                      %%% Thm body font
 {}                              %%% Indent amount (empty = no indent)
 {\bfseries\sffamily}            %%% Thm head font
 {}                              %%% Punctuation after thm head
 {3ex}                           %%% Space after thm head
 {\thmname{#1}\thmnumber{ #2}\thmnote{ \bfseries(#3)}} %%% Thm head spec
\theoremstyle{SlantTheorem}
\newtheorem{problem}{Problem}[]

%\counterwithin*{problem}{section}



%%%%%%%%%%%%%%%%%%%%%%%%%%%%Jenny's code%%%%%%%%%%%%%%%%%%%%

%%% Solution environment
%\newenvironment{solution}{
%\ifhandout\setbox0\vbox\bgroup\else
%\begin{trivlist}\item[\hskip \labelsep\small\itshape\bfseries Solution\hspace{2ex}]
%\par\noindent\upshape\small
%\fi}
%{\ifhandout\egroup\else
%\end{trivlist}
%\fi}
%
%
%%% instructorIntro environment
%\ifhandout
%\newenvironment{instructorIntro}[1][false]%
%{%
%\def\givenatend{\boolean{#1}}\ifthenelse{\boolean{#1}}{\begin{trivlist}\item}{\setbox0\vbox\bgroup}{}
%}
%{%
%\ifthenelse{\givenatend}{\end{trivlist}}{\egroup}{}
%}
%\else
%\newenvironment{instructorIntro}[1][false]%
%{%
%  \ifthenelse{\boolean{#1}}{\begin{trivlist}\item[\hskip \labelsep\bfseries Instructor Notes:\hspace{2ex}]}
%{\begin{trivlist}\item[\hskip \labelsep\bfseries Instructor Notes:\hspace{2ex}]}
%{}
%}
%% %% line at the bottom} 
%{\end{trivlist}\par\addvspace{.5ex}\nobreak\noindent\hung} 
%\fi
%
%


\let\instructorNotes\relax
\let\endinstructorNotes\relax
%%% instructorNotes environment
\ifhandout
\newenvironment{instructorNotes}[1][false]%
{%
\def\givenatend{\boolean{#1}}\ifthenelse{\boolean{#1}}{\begin{trivlist}\item}{\setbox0\vbox\bgroup}{}
}
{%
\ifthenelse{\givenatend}{\end{trivlist}}{\egroup}{}
}
\else
\newenvironment{instructorNotes}[1][false]%
{%
  \ifthenelse{\boolean{#1}}{\begin{trivlist}\item[\hskip \labelsep\bfseries {\Large Instructor Notes: \\} \hspace{\textwidth} ]}
{\begin{trivlist}\item[\hskip \labelsep\bfseries {\Large Instructor Notes: \\} \hspace{\textwidth} ]}
{}
}
{\end{trivlist}}
\fi


%% Suggested Timing
\newcommand{\timing}[1]{{\bf Suggested Timing: \hspace{2ex}} #1}




\hypersetup{
    colorlinks=true,       % false: boxed links; true: colored links
    linkcolor=blue,          % color of internal links (change box color with linkbordercolor)
    citecolor=green,        % color of links to bibliography
    filecolor=magenta,      % color of file links
    urlcolor=cyan           % color of external links
}


\title{A Place of Value}
\author{Vic Ferdinand, Betsy McNeal, Jenny Sheldon}

\begin{document}
\begin{abstract} Our goal in this activity is to develop rules for counting. \end{abstract}
\maketitle



\begin{problem}
In this problem you are allowed to use the following symbols (and their corresponding words): 0, 1, 2, 3, 4, 5, 6, 7, 8, 9. (Note:  There are no words such as ten, eleven, twelve, thirteen, twenty, etc. allowed during this activity!) 

Play with the sticks for a little while with the goal of counting them. Draw pictures to represent your ideas and be prepared to show your pictures and discuss your ideas with the class. Once you have some ideas, consider the following questions.

\begin{enumerate}
	\item In your system, how would you represent the following quantity of sticks with the given symbols, and why?
	\begin{image}
		\begin{tikzpicture}
		\foreach \x in {0, 0.3, 0.6, ..., 3} \draw[thick] (\x, 0) -- (\x, 0.5);
		\end{tikzpicture}
	\end{image}
	\item In your system, how would you represent the following quantity of sticks with the given symbols, and why?
	\begin{image}
		\begin{tikzpicture}
		\foreach \x in {0, 0.3, 0.6, ..., 3.3} \draw[thick] (\x, 0) -- (\x, 0.5);
		\end{tikzpicture}
	\end{image}
\end{enumerate}
\end{problem}


\begin{problem}
In this problem we are still only using the following symbols: $0$, $1$, $2$, $3$, $4$, $5$, $6$, $7$, $8$, $9$. In discussing problem 1 we defined what we mean by a ``bundle''. 
\begin{enumerate}
	\item How would you draw and represent using the given symbols a quantity of sticks that has $3$ bundles and $8$ sticks? How do you know your answer is correct?
	\item How would you draw and represent using the given symbols the next quantity after $3$ bundles and $8$ sticks. How do you know your answer is correct?
	\item How would you draw and represent using the given symbols the largest quantity we can make using only bundles and individual sticks? How do you know your answer is correct?
	\item How would you draw and represent using the given symbols the quantity after the one in part (c), and why?
\end{enumerate}
\end{problem}


\newpage

For all of the remaining problems, we will count using the symbols $0$ (zero), $A, B, C, D, E$, and $F$. (Note that now we don't have any symbols or words like $1$, $2$, $3$, etc.)

\begin{problem}

Begin to count in this system. For each quantity you count, be sure to draw a picture of the quantity using sticks and label your picture using the symbols of this system. Count as high as you can -- don't stop until your instructor gives you the go ahead to move to the next problem. 

\end{problem}


\begin{problem}
\begin{enumerate}
	\item How would you draw and represent using the given symbols a quantity of sticks that has $B$ bundles and $E$ sticks?
	\item How would you draw and represent using the given symbols the next quantity after $B$ bundles and $E$ sticks?How do you know your answer is correct?
	\item How would you draw and represent using the given symbols the largest quantity you could draw using only bundles and individual sticks? How do you know your answer is correct?
	\item How would you draw and represent using the given symbols the next quantity after the one in part (c)? How do you know your answer is correct?
	\item Using the given symbols in this system, how many sticks are in a bundle? How many sticks are in a superbundle? Describe your reasoning. 
\end{enumerate}
\end{problem}

\newpage

Remember, we are only using the symbols $0$, $A$, $B$, $C$, $D$, $E$, and $F$ for our counting.

\begin{problem}
\begin{enumerate}
	\item Using sticks, draw a picture of the quantity $BCF$ in this system. Explain how you know your answer is correct.
	\item What quantity comes right before $BCF$ in this system? Be sure to use a picture (or series of pictures) of sticks to find your answer, not a pattern involving letters only. Explain your thinking.
\end{enumerate}
\end{problem}



\begin{problem}
\begin{enumerate}
	\item Using sticks, draw a picture of the quantity $C00$ in this system. Explain how you know your answer is correct.
	\item What quantity comes right before $C00$ in this system? Be sure to use a picture (or series of pictures) of sticks to find your answer, not a pattern involving letters only. Explain your thinking. What does it mean to count backwards?
\end{enumerate}
\end{problem}


\begin{problem}
Is there a highest number you could count in this system? Explain how you know. 
\end{problem}


%\begin{problem}
% Take a bunch of sticks and count them using only the following words: zero, one, two, three, four, five, six, seven, eight, nine.  You may write these quantities using only the following symbols:  .
%\begin{enumerate}
%\item Describe the rules you use to count, as you keep putting one stick at a time in the group, including when you needed to switch to two or more of the symbols while counting.  In representing quantities in which you are forced to use two or more symbols, are each of the symbols representing the same quantities?  
%\item Draw or describe how you could use bundling or blocks to count from 97 to 112.  From 697 to 712.  Use them to decide what would be the symbol for the quantity just before 400.  Use them to count backwards from 4213 to 4189.
%\end{enumerate}
%\end{problem}

%\begin{problem} 
% Now suppose we wanted to count how many sticks we have using a similar counting system to write and speak about how many are in the set but, you are only allowed to use the symbols A, B, C, D, E, F, and zero = 0. Except for zero, the usual symbols (1, 2, 3 . . .) have been abolished.
%\begin{enumerate}
%\item In this system, when do you bundle and how many sticks are in a bundle?\\ How many bundles are in a superbundle?  \\
%When will we bundle the ``superbundles"?  
%\item Count to BAC in this system starting with A.   Develop easy-to-understand, yet complete, rules for how to count as we add one more stick to the pile.  Be thorough about when and why we bundle from one place to another (not just from the ones to the ``bundles").
%\item Count from BE0 to CAB.  
%\item Count backwards from ABC to EF. Explain in detail how the un-bundling process works. No cheating by going forwards from EF to ABC!  
%\end{enumerate}
%\end{problem}

\newpage
\begin{instructorNotes}
The intent here is to get the students to see the inner workings of our socially-constructed place value system for quantifying sets, especially larger sets. We want to communicate that a place value system is a way of writing quantities where the position of a digit in the written quantity determines its value. We want to emphasize the fact that we are making new units using the same pattern each time, enabling us to re-use a finite number of digits to write quantities as large as we like. In essence, a place value system is solving the problem of ``running out of symbols to use'' to describe larger and larger numbers.  

This activity extends the idea that counting is a one-to-one correspondence between a number (the quantity attribute of a set), its numeral (the symbol we use to denote that quantity), and the pronunciation or English word used to say that numeral out loud. We will have discussed this with Shepherd's Necklace and we want to help students connect these two activities together. We use different bases and different symbols to prompt our future teachers to concentrate on the ``hidden'' rules of the counting and how they correspond with objects.

In the first problem, we try to help students see that ``10'' (one-zero) refers to 10 single sticks while ``a ten'' refers to 1 bundle.  Thus, a new unit (bundle) is created when 10 singles are grouped together.  This new unit will be counted using the same counting system as that used for counting single sticks.  It is difficult for students (and children) to understand that ``10'' is simultaneously the word spoken as the last stick is laid down, the total number of sticks when one is added to nine, and the number of sticks in the one bundle.  Students have occasionally thought that there were 9 individual sticks in the bundle, and that ``10'' represented something like the rubber band that encircles the bundle.

Overall, we emphasize the following points.
\begin{itemize}
    \item We usually refer to a bundle of bundles as a ``superbundle'', and a bundle of superbundles as a ``megabundle''.
	\item The rules of the bundling process reflect the actions taken with the sticks. It should be clear that all of the action begins with the ones place:  putting down one more stick. (Of course, this changes later when we begin counting by bundles, or superbundles, etc.)  Watch for students who are just stating ``rules'' rather than motivating creating new places.
	\item A bundle is created after the tenth stick is laid down (thus the 10th stick is pushed together with the other 9 to create a bundle, leaving zero individual sticks: Thus, one-zero sticks). Watch for students who place 9 sticks in the bundle, or perhaps 11 sticks in the bundle. Watch also for students who want to count the last stick as ``zero'', when we want to reserve ``zero'' to describe nothing.
	\item We want to highlight the different types of units in this activity as well as their relationships to one another.  Students should see that, for instance, when they are counting bundles, they are counting a different type of unit or object than when they are counting sticks, and there is a dual nature to it: two bundles is the same thing as two-zero sticks.
	\item We try to focus students mostly on actions with sticks in their explanations - the symbol here is nearly an afterthought. This will be a theme throughout the course.
	\item Students often have difficulty when a second bundling must be made.  For example, in the usual base ten, starting at 299, we place down one new stick.  Since we don't have an allowable symbol now for the number of individual sticks, we bundle them, leaving zero individual sticks.  But now we don't have an allowable symbol for the number of bundles, so we bundle the bundles, making a new superbundle, leaving zero individual bundles.  We now have an allowable symbol for the number of superbundles (3), so our result is 3 superbundles, zero bundles, and zero sticks (300) or ``three-zero-zero''.
	\item After the counting rules are established, we find it can help to write the quantity in ``expanded form'', e.g., $123 = 1(100) + 2(10) + 3(1)$. This once again helps us emphasize the different units involved (now we have three different units: 1 superbundle, 2 bundles, and 1 single stick).
	

	\item Once the students are getting comfortable with counting forwards, we want to count backwards.  We want the students to see this process as ``un-bundling'', or, to think about what is happening with the sticks on the table when we move backwards one number - particularly ``across a zero''. 
	\item These bundling and un-bundling processes play an important role in our standard arithmetic procedures of adding with ``carrying'' and subtracting with ``borrowing''. 
\end{itemize}

Some possible extensions:

\begin{itemize}

    \item If time, you might have a ``semi-historical'' discussion about how the Hindu-Arabic place value system is an improvement over other systems developed over time (e.g., Roman, Egyptian) to represent quantities (e.g., In the whole numbers, the longer the symbol, the greater the quantity it represents.  Also don't need to keep inventing symbols as you get to more arduous quantities to comprehend).  You might also point out that the place value system has its ``Achilles Heels'' as well, such as the loss of ``larger symbol= greater quantity'' in partial numbers, the need for scientific notation, infinitely-long decimals, etc.  

     \item If and only if your students are a little more comfortable with algebra AND you have the time, you can bring up the ``polynomial'' structure of the place value system (successive powers of the base).  You might bring this into the activity (after the counting rules are established) by writing a quantity in ``expanded form''.  Make sure they see the bundles working to show this (singles, bundles, bundles of bundles (square), bundles of superbundles (cubed), etc. (Or give them a polynomial with single-digit whole number coefficients and ask them to (quickly) evaluate it at $x = 10$)).

     \item You could also discuss pros and cons of the different kinds of objects traditionally used to model numbers.  Sticks are the most concrete (can pick them apart- need to ``trade'' or rename) and blocks are next (still have the relative sizes but can't break them apart).  Then the abacus and coins have the ``bundling'' quality still there (but no relative sizes of the places).  Then the calculator has only the symbols (need to fully understand the system to understand what it means).
\end{itemize}

{\bf Suggested Timing:} The current calendar has three class periods for this activity, one for each page. On the first day, you should aim to get through the first page (though most of the way through it is fine). One suggestion is to give students about 15 minutes to try some things for problem 1, and then discuss. By the end of the first discussion, try to settle on the idea of bundling as good for moving forward. Then give students 10 or so minutes to work on the rest of the page, and then discuss. This problem should give you the opportunity to introduce superbundles. 

On the second day, let students count independently until most groups have at least made a bundle in the new system, and then discuss. We aren't expecting groups to have made superbundles before we discuss. We expect the independent work to take 8-10 minutes and then discussion for another 10 minutes. Then give students about 15 minutes to work on problem 4 and discuss with any remaining time.

On the third day, our main goal is to make sure students are feeling more confident about the place value system, though we expect they will still need to practice on their own to truly understand. We want students to feel equipped to try problems on their own after this work, not necessarily to have no questions left. Use these problems (and others you may invent as needed) to achieve this goal. Give students 10-15 minutes to work, and walk around to determine what issues they are still having. Then, discuss their solutions, giving preference to things they need most. 


\end{instructorNotes}



\end{document}
