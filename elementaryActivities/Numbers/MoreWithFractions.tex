\documentclass{ximera}


\graphicspath{
  {./}
  {graphics/}
  {../graphics/}
}

\usepackage{chngcntr}

\let\question\relax
\let\endquestion\relax




\newtheoremstyle{SlantTheorem}{\topsep}{\fill}%%% space between body and thm
%\newtheoremstyle{SlantTheorem}{\topsep}{\topsep}%%% space between body and thm
 {\slshape}                      %%% Thm body font
 {}                              %%% Indent amount (empty = no indent)
 {\bfseries\sffamily}            %%% Thm head font
 {}                              %%% Punctuation after thm head
 {3ex}                           %%% Space after thm head
 {\thmname{#1}\thmnumber{ #2}\thmnote{ \bfseries(#3)}}%%% Thm head spec
\theoremstyle{SlantTheorem}
\newtheorem{question}{Question}
\counterwithin*{question}{section}



\let\instructorNotes\relax
\let\endinstructorNotes\relax
%%% instructorNotes environment
\ifhandout
\newenvironment{instructorNotes}[1][false]%
{%
\def\givenatend{\boolean{#1}}\ifthenelse{\boolean{#1}}{\begin{trivlist}\item}{\setbox0\vbox\bgroup}{}
}
{%
\ifthenelse{\givenatend}{\end{trivlist}}{\egroup}{}
}
\else
\newenvironment{instructorNotes}[1][false]%
{%
  \ifthenelse{\boolean{#1}}{\begin{trivlist}\item[\hskip \labelsep\bfseries {\Large Instructor Notes: \\} \hspace{\textwidth} ]}
{\begin{trivlist}\item[\hskip \labelsep\bfseries {\Large Instructor Notes: \\} \hspace{\textwidth} ]}
{}
}
{\end{trivlist}}
\fi


%% Suggested Timing
\newcommand{\timing}[1]{{\bf Suggested Timing: \hspace{2ex}} #1}


\title{More With Fractions}
\author{Vic Ferdinand, Betsy McNeal, Jenny Sheldon}

\begin{document}
\begin{abstract}
 This activity will investigate equivalent fractions and the reasoning behind the procedure for making equivalent fractions. %$\frac{A}{B} = \frac{A \times N}{B \times N}$.
\end{abstract}
\maketitle



\begin{problem} \label{MoreWithFractions1}
Fold a strip of paper into fourths and shade in $\frac{3}{4}$ of the paper strip. Now fold each of your fourths into 3 equal pieces.  
\begin{enumerate}
    \item   How many parts are in the strip of paper now?
    \item   Are these parts equal to each other?  Why?
    \item   What fraction of the original paper strip is each individual new part after you finish all of your folding? How do you know?
    \item   How many of the parts are shaded?  How do you know?
    \item   What new fraction of your paper strip is now shown by the shading?  
    \item Draw ONE careful picture showing both the original strip and the new folds in the strip.  Your drawing should illustrate the method and reasoning behind your creation of a fraction equivalent to $\frac{3}{4}$ of your original strip of paper.
   
\end{enumerate}
\end{problem}
%\vskip 2in

\pagebreak

\begin{problem} \label{MoreWithFractions2}
 Let's try this again, but without the paper strip.  Instead, \underline{imagine} working with a new paper strip. Make sketches to illustrate your thoughts, but don't draw all of the little pieces.  
\begin{enumerate}
    \item Explain to your group how you would fold your imaginary strip to show the fraction $\frac{11}{63}$.   
    \item Explain to your group how you would make a fraction  \underline{equivalent} to $\frac{11}{63}$ using the same paper strip.  That is, what steps would you take?
   
    \item   What new fraction of your imaginary paper strip is now shown by the shading?  
    %\end{enumerate}
    
    \item How many different fractions could you make that are equivalent to the original fraction $\frac{11}{63}$?
   
\end{enumerate}
\end{problem}
\vskip 1in

\begin{problem}
When you learned how to ``make equivalent fractions" in school, you learned a rule that was something like ``multiply top and bottom by the same number", as shown in this formula:

  \[ \frac{A}{B} = \frac{A \times N}{B \times N}. \]

Explain how this rule encapsulates the ideas of the process that you have done in the previous problems. We haven't talked about multiplication, yet, so how can you use your picture to explain why this multiplication makes sense?  %\ref{MoreWithFractions1}-\ref{MoreWithFractions3}.
\end{problem}

\newpage

\begin{problem}
 Use folding to compare the fractions $\frac{3}{7}$ and $\frac{2}{5}$. 
\begin{enumerate}
    \item Explain to your group how you would fold two different imaginary strips to show these fractions. (What must we keep in mind about these imaginary strips?)   
    \item Once you have a sketch (because we aren't going to make that many folds), how could you re-fold the two strips to make it easy to compare the two fractions? 
   \item How many parts would be in each imaginary strip of paper now? Why?
    \item   How many of the parts are shaded in each strip?
    \item   Which fraction is larger?  How can you be sure? 
\end{enumerate}
\end{problem}

\begin{problem}
%\label{MoreWithFractions3}
Suppose you were trying to convert the fraction $\frac{19}{20}$ into a percent.  You probably learned to do this by solving the following problem:

\[ \frac{19}{20} = \frac{?}{100} \]

Explain how you might use a strip of (imaginary!) paper  to solve this problem.
\end{problem}

%\vskip 1.5in


%\end{document}


\newpage

\begin{instructorNotes}

{\bf Main goal:} This is our main activity to discuss why we make equivalent fractions the way that we do. Students should be able to talk about cutting the pieces into more equal-sized pieces, and then see that the whole and the shaded region don't change, so the resulting fraction is equal to the original.

{\bf Overall picture:}
The first three problems introduce the main ideas that we want students to be thinking about. We draw or imagine paper strips for our wholes, then we use the meaning of fractions to represent the original fraction. Then, we cut each of the equal pieces into more equal pieces. In problem 1, each equal piece is cut into three equal pieces. In problem 2, students can choose any number of equal pieces to cut the original pieces into, but we recommend something small like 2, 3, or even 4. In problem 3, we would like students to connect the abstract $N$ to the number of pieces each of the original $B$ pieces is being cut into. 

Then, we want to observe that cutting all of the pieces also cuts the shaded pieces in the same way, because the shaded pieces are just part of all of the pieces. (The shaded pieces are a subset of all of the pieces, but you might rephrase that in several ways since most students will not be comfortable with the language of sets.) So, we can count the number of pieces in the whole after cutting, and the number of shaded pieces after cutting, and get a new fraction.

The key to this argument is that the new fraction is equal to the original fraction because the whole didn't change and the shaded region didn't change. 

In your discussion of the first three problems, have students present their work and talk about what they observed through the questions. Focus on the physical actions that are happening with the strips, and then connect those physical actions to the algebraic ideas in problem 3. Then pull out the observations about the process overall as well as how we know the two fractions are equal. You might contrast this argument with one where we draw two different strips and try to compare the fractions by looking at them. In the visual method, we could have a hard time distinguishing fractions that are very close to one another versus exactly equal to one another. Exact equality should require more than just ``they look kinda the same''!

Problem 4 is an application of equivalent fractions, so students should practice the same explanations we have gone through in problems 1--3 with each fraction here. Once we make equivalent fractions with the same denominator, all of our pieces are now the same size, and so we can just look at how many pieces we have. The fraction with more equal pieces should be the larger fraction (of course assuming that the original wholes are equal! It's good to remember this from our initial discussion about comparing fractions). In your discussion, have students show their diagrams and talk about how they decided what denominator to use as well as how they got their equivalent fractions. Also focus on having students describe exactly how they know which fraction is larger. 

Students sometimes get concerned about how to choose a denominator that's the same for both fractions without multiplying, but multiplication is okay, here. In their explanations, students should focus on what is happening to their diagrams as they make equivalent fractions more than what denominator they chose (other than to say they chose denominators that were the same for both fractions so that we would have equal-sized pieces). So, in this case it's okay for students to say that $35$ is a common denominator because $5\times 7 = 35$ and $7 \times 5 = 35$, or it's okay to say that we can make $35$ equal pieces by splitting each of the $5$ pieces into $7$ and each of the $7$ pieces into $5$ and counting the result.

Problem 5 helps us to connect fractions and percents in that we are going to think about making equivalent fractions frequently when working with percents in the next topic. This problem can be discussed now as an application of equivalent fractions or later in the context of percents.


{\bf Good language:} Note that we prefer the ``splitting each piece" language over ``splitting the whole''. For instance in problem 1, we are splitting each piece into 3 pieces rather than splitting the whole into $3$ more pieces. The first phrasing is a bit more clear in terms of what's happening with the pieces.

The multiplication issue in this problem can be difficult for students. Since we haven't talked about multiplication yet, and we stress not using mathematics we haven't explained together yet, some students will struggle to see the multiplication here. Try to focus the students on what is happening with the pieces: each piece is being split into so many more equal pieces. Then we can count the number of total pieces and count the number of shaded pieces. For instance, if we are trying to show that $\frac{1}{2} = \frac{4}{8}$, we have each of the original two pieces split into four equal pieces. We count the number of total pieces and get eight, and we count the number of shaded pieces and get $4$. We don't need to discuss the fraction $\frac{1\times 4}{2 \times 4}$ at all.

Once we discuss multiplication, however, we may return to this idea and see that we do in fact have a groups and objects per group interpretation for what is happening.

It's also important to emphasize that the shading doesn't change when we cut the pieces into more pieces. The fact that the whole doesn't change and the shaded region doesn't change is the reason we are confident that the new fraction is equal to the original.



{\bf Suggested Timing:} This activity should take the whole class period to fully explore. Give students about 10 minutes to think about problems 1, 2, and 3, then spend about 20 minutes in presentation and discussions. Make sure to wrap up this part of the discussion by talking about how we expect students to write explanations for equivalent fractions. This wrap-up is more important than the remaining problems! With the remaining time, give students 5-8 minutes to work on problems 4 and 5, and then use the remaining time for discussion. Problem 4 might end up on the next day's activities, and problem 5 can be skipped for now or placed on homework.
\end{instructorNotes}



\end{document}
