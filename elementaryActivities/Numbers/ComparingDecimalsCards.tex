\documentclass{ximera}

\usepackage{gensymb}
\usepackage{tabularx}
\usepackage{mdframed}
\usepackage{pdfpages}
%\usepackage{chngcntr}

\let\problem\relax
\let\endproblem\relax

\newcommand{\property}[2]{#1#2}




\newtheoremstyle{SlantTheorem}{\topsep}{\fill}%%% space between body and thm
 {\slshape}                      %%% Thm body font
 {}                              %%% Indent amount (empty = no indent)
 {\bfseries\sffamily}            %%% Thm head font
 {}                              %%% Punctuation after thm head
 {3ex}                           %%% Space after thm head
 {\thmname{#1}\thmnumber{ #2}\thmnote{ \bfseries(#3)}} %%% Thm head spec
\theoremstyle{SlantTheorem}
\newtheorem{problem}{Problem}[]

%\counterwithin*{problem}{section}



%%%%%%%%%%%%%%%%%%%%%%%%%%%%Jenny's code%%%%%%%%%%%%%%%%%%%%

%%% Solution environment
%\newenvironment{solution}{
%\ifhandout\setbox0\vbox\bgroup\else
%\begin{trivlist}\item[\hskip \labelsep\small\itshape\bfseries Solution\hspace{2ex}]
%\par\noindent\upshape\small
%\fi}
%{\ifhandout\egroup\else
%\end{trivlist}
%\fi}
%
%
%%% instructorIntro environment
%\ifhandout
%\newenvironment{instructorIntro}[1][false]%
%{%
%\def\givenatend{\boolean{#1}}\ifthenelse{\boolean{#1}}{\begin{trivlist}\item}{\setbox0\vbox\bgroup}{}
%}
%{%
%\ifthenelse{\givenatend}{\end{trivlist}}{\egroup}{}
%}
%\else
%\newenvironment{instructorIntro}[1][false]%
%{%
%  \ifthenelse{\boolean{#1}}{\begin{trivlist}\item[\hskip \labelsep\bfseries Instructor Notes:\hspace{2ex}]}
%{\begin{trivlist}\item[\hskip \labelsep\bfseries Instructor Notes:\hspace{2ex}]}
%{}
%}
%% %% line at the bottom} 
%{\end{trivlist}\par\addvspace{.5ex}\nobreak\noindent\hung} 
%\fi
%
%


\let\instructorNotes\relax
\let\endinstructorNotes\relax
%%% instructorNotes environment
\ifhandout
\newenvironment{instructorNotes}[1][false]%
{%
\def\givenatend{\boolean{#1}}\ifthenelse{\boolean{#1}}{\begin{trivlist}\item}{\setbox0\vbox\bgroup}{}
}
{%
\ifthenelse{\givenatend}{\end{trivlist}}{\egroup}{}
}
\else
\newenvironment{instructorNotes}[1][false]%
{%
  \ifthenelse{\boolean{#1}}{\begin{trivlist}\item[\hskip \labelsep\bfseries {\Large Instructor Notes: \\} \hspace{\textwidth} ]}
{\begin{trivlist}\item[\hskip \labelsep\bfseries {\Large Instructor Notes: \\} \hspace{\textwidth} ]}
{}
}
{\end{trivlist}}
\fi


%% Suggested Timing
\newcommand{\timing}[1]{{\bf Suggested Timing: \hspace{2ex}} #1}




\hypersetup{
    colorlinks=true,       % false: boxed links; true: colored links
    linkcolor=blue,          % color of internal links (change box color with linkbordercolor)
    citecolor=green,        % color of links to bibliography
    filecolor=magenta,      % color of file links
    urlcolor=cyan           % color of external links
}


\title{Comparing Decimals}
\author{Jenny Sheldon}

\begin{document}
\begin{abstract} \end{abstract}
\maketitle



\begin{problem}
For each of the following pairs of decimal numbers, use bundling to say which number is larger. Be sure to explain how you made your representation as well as how you know which decimal is the bigger one from the picture.
\begin{enumerate}
\item $1.2$ vs $1.21$
\item $10.2$ vs $9.9$
\item $1.04$ vs $1.4$
\item $0.034$ vs $0.0304$
\item $0.05$ vs $0.0333$
\end{enumerate}

\end{problem}

\begin{problem}
For each of the following pairs of decimal numbers, use paper strips to say which number is larger. Be sure to explain how you made your representation as well as how you know which decimal is the bigger one from the picture.
\begin{enumerate}
\item $0.56$ vs $0.6$
\item $2.19$ vs $2.23$
\item $0.23$ vs $0.234$
\item $0.004$ vs $0.039$
\end{enumerate}

\end{problem}

\begin{problem}
For each of the following pairs of decimal numbers, use a number line to say which number is larger. Be sure to explain how you made your representation as well as how you know which decimal is the bigger one from the picture.
\begin{enumerate}
\item $5.7$ vs $5.68$
\item $34.52$ vs $3.452$
\item $0.00216$ vs $0.0003015$
\end{enumerate}

\end{problem}


\begin{problem}
Many of these pairs of numbers would be confusing for children. Pick three different pairs of numbers and explain how a child might get confused about which number is larger and why. Then, discuss how your representation could help a kid with their confusion.
\end{problem}

\newpage
\begin{instructorNotes}

{\bf Main goal:} Students use our three representations for decimal numbers to compare decimals.

{\bf Overall picture:}

The main goal of this activity really is to continue practicing with the three representations for decimals: bundling, paper strips, and number lines. As students are working through this activity, be sure to keep asking for explanations related to how the representation was produced. So, for bundling, students should talk about the value of the individual stick, and they should talk about how the value of the stick determines the value of a bundle (either 10 times the value of a stick, or that putting 10 things together forces us to bundle and move to the next place value to the left). For paper strips, students should talk about starting with a unit value and then ``unbundling'' that unit to make tenths, etc. For number lines, students should talk about marking a zero and a unit length (even if those ticks marks aren't on the actual number line) and then how they ``unbundle" the space between the tick marks to see the next place value to the right.

We also want to add to this discussion a precise discussion about how to tell which decimal number is larger. For bundling, this should boil down to which number has more sticks. (We don't want to rely on ``rules'' of place value comparisons here, but rather see why those rules are sensible from our bundling representation.) For paper strips, the larger number should be the longer strip, and for number lines the larger number should be farther to the right if the line is constructed in the typical fashion. For number lines, it can also be good to talk about ``farther from zero in the positive direction'' to add some sense-making to the idea that numbers farther to the right are larger.

To play the game with the index cards, you would pass out the cards randomly to groups in class. Each card has one of the parts of the activity written on it, so the activity is not needed if you are using the cards in class. Give students time to work, and then have each group present (since each group should have a different problem). You can play a few rounds of this game, depending on time, or you can give groups multiple cards at the beginning of class. The cards are color-coded for which representation they are asking for, so that you can ensure if you want to that each group has at least one of each representation. You can also pass out new cards as groups finish with the card they are working on.

To use the activity without index cards, you will want to ensure that you have time to talk about all three representations. Ask students to work all of the ``part (a)'' questions, then all of the part (b) questions, etc.

The last question brings us back to children's mathematics and the misconceptions that students frequently have about comparing decimal numbers. Be sure to have students bring up the following ideas.
\begin{itemize}
	\item Longer numbers are bigger.
	\item We only need to compare the first non-zero digits that don't match (ie ignoring place value).
	\item Zeros mean nothing, so we can ignore them.
	\item Numbers with more zeros are larger (or even numbers with more zeros are smaller).
	\item Numbers with more digits are bigger.
\end{itemize}
Most of these ``rules'' are appropriate in some circumstances but not others. In discussing why these rules are problematic, students should be able to give counterexamples. Students should also point to the fact that using their representations show more sticks/longer strips/farther right will always work because these representations are based on the meaning of the numbers. This ``rule'' works, and we can say why it works!

{\bf Good language:} Encourage students to keep building up the vocabulary they use to talk about bundling. We are hoping students are finding this easier a second time around. If students are hesitant to use words like ``ten'' they can of course still use ``one-zero'', but ten is not banned in this activity.

{\bf Suggested Timing:} The timing will depend on how you are using the cards or activity. Each individual part (or each individual card) should take students about 5 minutes to work on and you should expect about $5$ minutes per presentation. So, for instance, if you have three groups in class and are doing rounds of presentations, you will give students about $5$ minutes to work on their card, then have about $15$ minutes of presentations, then repeat with all the time you have. If you continue giving students more cards, save enough time at the end for all groups to present their work (probably at least 20 minutes).

If you are using the activity and giving students time to work on the part (a) problems, then part (b), etc, give students about $20$ minutes to get as far as they can, and then spend about $20$ minutes in presentations. This should leave about $5$ minutes to discuss the presentations as a whole, pointing out things that you would like students to talk about in their explanations or features of their diagrams that are particularly nice. Then you'll have about $10$ minutes left to wrap up with the last question.


\end{instructorNotes}


\end{document}
