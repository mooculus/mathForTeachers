\documentclass[nooutcomes,noauthor]{ximera}
\usepackage{gensymb}
\usepackage{tabularx}
\usepackage{mdframed}
\usepackage{pdfpages}
%\usepackage{chngcntr}

\let\problem\relax
\let\endproblem\relax

\newcommand{\property}[2]{#1#2}




\newtheoremstyle{SlantTheorem}{\topsep}{\fill}%%% space between body and thm
 {\slshape}                      %%% Thm body font
 {}                              %%% Indent amount (empty = no indent)
 {\bfseries\sffamily}            %%% Thm head font
 {}                              %%% Punctuation after thm head
 {3ex}                           %%% Space after thm head
 {\thmname{#1}\thmnumber{ #2}\thmnote{ \bfseries(#3)}} %%% Thm head spec
\theoremstyle{SlantTheorem}
\newtheorem{problem}{Problem}[]

%\counterwithin*{problem}{section}



%%%%%%%%%%%%%%%%%%%%%%%%%%%%Jenny's code%%%%%%%%%%%%%%%%%%%%

%%% Solution environment
%\newenvironment{solution}{
%\ifhandout\setbox0\vbox\bgroup\else
%\begin{trivlist}\item[\hskip \labelsep\small\itshape\bfseries Solution\hspace{2ex}]
%\par\noindent\upshape\small
%\fi}
%{\ifhandout\egroup\else
%\end{trivlist}
%\fi}
%
%
%%% instructorIntro environment
%\ifhandout
%\newenvironment{instructorIntro}[1][false]%
%{%
%\def\givenatend{\boolean{#1}}\ifthenelse{\boolean{#1}}{\begin{trivlist}\item}{\setbox0\vbox\bgroup}{}
%}
%{%
%\ifthenelse{\givenatend}{\end{trivlist}}{\egroup}{}
%}
%\else
%\newenvironment{instructorIntro}[1][false]%
%{%
%  \ifthenelse{\boolean{#1}}{\begin{trivlist}\item[\hskip \labelsep\bfseries Instructor Notes:\hspace{2ex}]}
%{\begin{trivlist}\item[\hskip \labelsep\bfseries Instructor Notes:\hspace{2ex}]}
%{}
%}
%% %% line at the bottom} 
%{\end{trivlist}\par\addvspace{.5ex}\nobreak\noindent\hung} 
%\fi
%
%


\let\instructorNotes\relax
\let\endinstructorNotes\relax
%%% instructorNotes environment
\ifhandout
\newenvironment{instructorNotes}[1][false]%
{%
\def\givenatend{\boolean{#1}}\ifthenelse{\boolean{#1}}{\begin{trivlist}\item}{\setbox0\vbox\bgroup}{}
}
{%
\ifthenelse{\givenatend}{\end{trivlist}}{\egroup}{}
}
\else
\newenvironment{instructorNotes}[1][false]%
{%
  \ifthenelse{\boolean{#1}}{\begin{trivlist}\item[\hskip \labelsep\bfseries {\Large Instructor Notes: \\} \hspace{\textwidth} ]}
{\begin{trivlist}\item[\hskip \labelsep\bfseries {\Large Instructor Notes: \\} \hspace{\textwidth} ]}
{}
}
{\end{trivlist}}
\fi


%% Suggested Timing
\newcommand{\timing}[1]{{\bf Suggested Timing: \hspace{2ex}} #1}




\hypersetup{
    colorlinks=true,       % false: boxed links; true: colored links
    linkcolor=blue,          % color of internal links (change box color with linkbordercolor)
    citecolor=green,        % color of links to bibliography
    filecolor=magenta,      % color of file links
    urlcolor=cyan           % color of external links
}
\title{A marketing meeting}

\begin{document}
\begin{abstract}
\end{abstract}

\maketitle


\begin{problem}
The marketing department is designing a flyer for your new favorite product. They would like the flyer to include an image which takes up $\frac{2}{3}$ of the space on the paper. Then, they would like to place text over $\frac{1}{5}$ of the image. What fraction of the entire flyer will be taken up by the text on the image?

Solve this problem using a picture and our meaning of fractions. Carefully explain what you drew and why you drew it, paying close attention to what you are using for the whole for any fraction you have.
\end{problem}

\begin{problem}
The marketing department is designing another flyer for your new favorite product. The text on the flyer is currently $\frac{3}{8}$ of an inch tall, but they would like for the text to be $\frac{2}{5}$ of an inch tall. What fraction is the current text size of the desired text size? 

Solve this problem using a picture and our meaning of fractions. Carefully explain what you drew and why you drew it, paying close attention to what you are using for the whole for any fraction you have.
\end{problem}


\begin{problem}
The marketing department found a fragment of a flyer from an older product in the back of someone's drawer. Here is a sketch of the fragment of the flyer, which is $\frac{3}{5}$ of the entire old flyer. 
\begin{image}
\begin{tikzpicture}
\draw[thick] (0,0) rectangle (2, 3); 
\end{tikzpicture}
\end{image}
The marketing department wants to make a flyer for a new product which is $\frac{7}{4}$ of the size of the entire old flyer. Using the sketch of the flyer's fragment above, draw the size of the flyer for the new product.

Carefully explain what you drew and why you drew it, paying close attention to what you are using for the whole for any fraction you have.
\end{problem}





\newpage

\begin{instructorNotes} 
{\bf Main goal:} We use our meaning of fractions to solve problems, which involve making same-sized pieces.


{\bf Overall picture:} We want to emphasize that common denominators occur when, in the course of solving a problem, we cut two wholes into same-sized pieces. Usually, these wholes would start out with different-sized pieces, so we need to make further cuts in order to make these same-sized pieces. Students can frequently make common denominators just because we are working with fractions, so we are trying to add meaning to this common procedure. After all, same-sized pieces aren't always necessary! 

We emphasize using the information given in the question to draw pictures, and then using our pictures and the meaning of fractions to find the solution we want. We don't want to solve and then draw!

\begin{itemize}
	\item This is nearing the end of our work with the basic meaning of fractions, so we hope students have gained some confidence by this point. Have students work through their solutions in whole-class discussion, continuing to emphasize each part of our meaning of fractions. What is the whole? What tells us how to cut the whole? How many pieces do we need?
	\item You can encourage students to draw a sequence of pictures to solve these problems, rather than a single picture!
	\item Keep emphasizing the role of the various wholes. Students can still have trouble identifying what the question is asking us to use as the whole in various parts of the problem. It can be nice to ask students to talk about how they decided which whole to use at which point: essentially having students give each other advice about this.
\end{itemize}


{\bf Good language:}  It is still a good point in the semester to talk about what should be included in a good explanation for these problems. Emphasize any point at which students are getting stuck! Emphasize how the meaning of fractions is useful in finding the answer (not just setting up the problem).


{\bf Suggested timing:} Give students about 10-15 minutes to work through these problems, and then have students present their work and discuss. You can have multiple groups present their solutions, especially if there are different-looking solutions in the room. You can focus more on problems 1 and 2 than problem 3, especially if you are finishing up some leftover material in the beginning of the class.






\end{instructorNotes}



\end{document}