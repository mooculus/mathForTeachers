\documentclass{ximera}


\graphicspath{
  {./}
  {graphics/}
  {../graphics/}
}

\usepackage{chngcntr}

\let\question\relax
\let\endquestion\relax




\newtheoremstyle{SlantTheorem}{\topsep}{\fill}%%% space between body and thm
%\newtheoremstyle{SlantTheorem}{\topsep}{\topsep}%%% space between body and thm
 {\slshape}                      %%% Thm body font
 {}                              %%% Indent amount (empty = no indent)
 {\bfseries\sffamily}            %%% Thm head font
 {}                              %%% Punctuation after thm head
 {3ex}                           %%% Space after thm head
 {\thmname{#1}\thmnumber{ #2}\thmnote{ \bfseries(#3)}}%%% Thm head spec
\theoremstyle{SlantTheorem}
\newtheorem{question}{Question}
\counterwithin*{question}{section}



\let\instructorNotes\relax
\let\endinstructorNotes\relax
%%% instructorNotes environment
\ifhandout
\newenvironment{instructorNotes}[1][false]%
{%
\def\givenatend{\boolean{#1}}\ifthenelse{\boolean{#1}}{\begin{trivlist}\item}{\setbox0\vbox\bgroup}{}
}
{%
\ifthenelse{\givenatend}{\end{trivlist}}{\egroup}{}
}
\else
\newenvironment{instructorNotes}[1][false]%
{%
  \ifthenelse{\boolean{#1}}{\begin{trivlist}\item[\hskip \labelsep\bfseries {\Large Instructor Notes: \\} \hspace{\textwidth} ]}
{\begin{trivlist}\item[\hskip \labelsep\bfseries {\Large Instructor Notes: \\} \hspace{\textwidth} ]}
{}
}
{\end{trivlist}}
\fi


%% Suggested Timing
\newcommand{\timing}[1]{{\bf Suggested Timing: \hspace{2ex}} #1}


\title{Calculations With Percents}

\begin{document}
\begin{abstract}\end{abstract}
\maketitle




\begin{problem}
 Sixty percent of a city's 100 mailboxes are located within 5 miles of the library.  How many mailboxes are located within 5 miles of the library?  Solve the problem by drawing a picture.  Explain how your picture helps you solve the problem. 
 
 

\vfill
\end{problem}
\begin{problem}

Caroline planted $6$ flowers in her garden, which is $30\%$ of the flowers she plans to plant. How many flowers does Caroline plan to plant in her garden? Solve the problem by drawing a picture.  Explain how your picture helps you solve the problem.  

\vfill
\end{problem}
\newpage





\begin{problem}
 If 95$\%$ of the 80 football players on the team voted for the team captain, then how many football players on the team voted for the team captain?  Solve the problem by drawing a picture.  Explain how your picture helps you solve the problem.  
 
 
 
 
 \vfill
\end{problem}




\begin{problem}
An art teacher is framing drawings for an upcoming art show. He has already framed $39$ drawings, and needs to frame a total of $60$ drawings. What percent of the drawings are already framed? Solve the problem by working with common fractions and by drawing a picture.  Explain how your picture helps you solve the problem.  

 
\vfill
\end{problem}





\begin{problem}

At a large warehouse, $75\%$ of the shelves are currently full, and there are $12$ million items on the shelves. How many items will fit in the entire warehouse? Solve the problem by drawing a picture.  Explain how your picture helps you solve the problem.  


\vfill
\end{problem}


\newpage

\begin{instructorNotes}
We would like our students to be able to use a variety of methods for solving problems involving percent: pictures, percent tables, ``going through 1'', and equivalent fractions (proportion method). Our intent with this first activity is to build on the prior work with fractions so that students have a strong conceptual basis for working with percents.  A proportional or numerical method will not be permitted at this time.  Later in the course, if students can clearly explain what each fraction (or ratio) represents in the problem and what makes it reasonable to set them equal, then this method will be acceptable.  We focus entirely on pictorial methods in this activity and develop other strategies later.   We are looking for students to develop their own reasoning in order to solve these problems rather than giving them a pre-made solution strategy. 

\begin{itemize}
	\item Percents are fractions with denominators of 100 and so the reasoning is the same as we used in earlier fraction activities.
	\item Problems involving percents include an amount that is the ``whole'' or 100\%, an amount that is compared to that whole, and the percent itself.  Any of these three values can be the unknown in a problem.  When problem-solving, a rough picture can always be used to sort through the knowns of the problem and to clarify the question, as well as to find the actual solution.
	\item  As usual, students need to be careful to label their pictures with words from the context.  In particular, labeling something as 100\% may end up being confusing (100\% of what?).
	\item It is not always best to start with a picture that is cut into 10 equal parts (or 100 equal parts!).   Sometimes it is useful to represent fifths (worth 20\%) or quarters (worth 25\%) etc.
	\item Students often struggle a bit with \#1 - not recognizing that they have already solved the problem in fraction form earlier in the semester and can re-interpret it in percent form.  (They may use their knowledge of equivalent fractions to figure that $\frac35$ is the same as $\frac{60}{100}$, but should also be able to explain how that could be seen in a picture.)  
	\item Sometimes students' representations are troublesome in that they might not actually show the meaning of percents (as fractions) at all!  For example, in \#2, students often draw 6 discrete flowers representing 30\% of the garden, figure out that each represents 5\% of the garden, and correctly conclude they need a total of 20 flowers.  This reasoning is terrific, but students need to make it clear what the whole is in the problem and what the relationship is between the initial 6 flowers and the final answer. 
	\item \#4 is tricky for students, but usually a couple of groups can figure it out and give a good presentation of the answer.
\end{itemize}



{\bf Suggested Timing:} After a 5-minute introduction reviewing the meaning of ``percent'', small groups should work for about 20 minutes on the problems.  Follow this with 25 minutes of whole class presentations.  
\end{instructorNotes}


\end{document}