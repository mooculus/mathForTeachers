%\documentclass{ximera}
\documentclass[nooutcomes,noauthor]{ximera}
\usepackage{gensymb}
\usepackage{tabularx}
\usepackage{mdframed}
\usepackage{pdfpages}
%\usepackage{chngcntr}

\let\problem\relax
\let\endproblem\relax

\newcommand{\property}[2]{#1#2}




\newtheoremstyle{SlantTheorem}{\topsep}{\fill}%%% space between body and thm
 {\slshape}                      %%% Thm body font
 {}                              %%% Indent amount (empty = no indent)
 {\bfseries\sffamily}            %%% Thm head font
 {}                              %%% Punctuation after thm head
 {3ex}                           %%% Space after thm head
 {\thmname{#1}\thmnumber{ #2}\thmnote{ \bfseries(#3)}} %%% Thm head spec
\theoremstyle{SlantTheorem}
\newtheorem{problem}{Problem}[]

%\counterwithin*{problem}{section}



%%%%%%%%%%%%%%%%%%%%%%%%%%%%Jenny's code%%%%%%%%%%%%%%%%%%%%

%%% Solution environment
%\newenvironment{solution}{
%\ifhandout\setbox0\vbox\bgroup\else
%\begin{trivlist}\item[\hskip \labelsep\small\itshape\bfseries Solution\hspace{2ex}]
%\par\noindent\upshape\small
%\fi}
%{\ifhandout\egroup\else
%\end{trivlist}
%\fi}
%
%
%%% instructorIntro environment
%\ifhandout
%\newenvironment{instructorIntro}[1][false]%
%{%
%\def\givenatend{\boolean{#1}}\ifthenelse{\boolean{#1}}{\begin{trivlist}\item}{\setbox0\vbox\bgroup}{}
%}
%{%
%\ifthenelse{\givenatend}{\end{trivlist}}{\egroup}{}
%}
%\else
%\newenvironment{instructorIntro}[1][false]%
%{%
%  \ifthenelse{\boolean{#1}}{\begin{trivlist}\item[\hskip \labelsep\bfseries Instructor Notes:\hspace{2ex}]}
%{\begin{trivlist}\item[\hskip \labelsep\bfseries Instructor Notes:\hspace{2ex}]}
%{}
%}
%% %% line at the bottom} 
%{\end{trivlist}\par\addvspace{.5ex}\nobreak\noindent\hung} 
%\fi
%
%


\let\instructorNotes\relax
\let\endinstructorNotes\relax
%%% instructorNotes environment
\ifhandout
\newenvironment{instructorNotes}[1][false]%
{%
\def\givenatend{\boolean{#1}}\ifthenelse{\boolean{#1}}{\begin{trivlist}\item}{\setbox0\vbox\bgroup}{}
}
{%
\ifthenelse{\givenatend}{\end{trivlist}}{\egroup}{}
}
\else
\newenvironment{instructorNotes}[1][false]%
{%
  \ifthenelse{\boolean{#1}}{\begin{trivlist}\item[\hskip \labelsep\bfseries {\Large Instructor Notes: \\} \hspace{\textwidth} ]}
{\begin{trivlist}\item[\hskip \labelsep\bfseries {\Large Instructor Notes: \\} \hspace{\textwidth} ]}
{}
}
{\end{trivlist}}
\fi


%% Suggested Timing
\newcommand{\timing}[1]{{\bf Suggested Timing: \hspace{2ex}} #1}




\hypersetup{
    colorlinks=true,       % false: boxed links; true: colored links
    linkcolor=blue,          % color of internal links (change box color with linkbordercolor)
    citecolor=green,        % color of links to bibliography
    filecolor=magenta,      % color of file links
    urlcolor=cyan           % color of external links
}
\title{The Bake Shop}



\begin{document}
\begin{abstract}
\end{abstract}

\maketitle

\begin{problem}
Rhiannon works at a bake shop. In the last hour, Rhiannon used $\frac{4}{7}$ of a pound of chocolate to decorate cookies. If Rhiannon is working at a steady pace, what fraction of an hour did it take her to use $\frac{3}{7}$ of a pound of chocolate?

Draw a picture and use our meaning of fractions to answer this question. Be sure to carefully explain what you are seeing as the whole for any fraction you use.
\end{problem}

\begin{problem}
Sora works at a bake shop with a big case full of donuts. Today, $\frac{2}{5}$ of the donuts have chocolate frosting. Sora's boss has asked for sprinkles to be placed on $\frac{1}{2}$ of the donuts with chocolate frosting. Terry wonders: what fraction of the full case of donuts will have both chocolate frosting and sprinkles?

Draw a picture and use our meaning of fractions to answer this question. Be sure to carefully explain what you are seeing as the whole for any fraction you use.
\end{problem}


\begin{problem}
Terry works at a bake shop. Terry knows that $\frac{5}{8}$ of a cup of chocolate chips are needed  in order to make $\frac{1}{3}$ of a batch of cookies. How many cups of chocolate chips are needed to make a full batch of the same cookies?

Draw a picture and use our meaning of fractions to answer this question. Be sure to carefully explain what you are seeing as the whole for any fraction you use.
\end{problem}





\newpage

\begin{instructorNotes} 



{\bf Main goal:} Use our fraction definition to solve problems.

{\bf Overall picture:} 

Throughout this activity, students should pay attention to what they are seeing as the whole for a fraction at any moment. In problems 1 and 2, the whole must be changed in order to solve the problem. Problem 3 is a bonus problem in case you finish with the others. 

\begin{itemize}
	\item Carefully labeled pictures are essential for this work. Encourage students to label their pictures with what they are seeing as the whole at any moment.
	\item Encourage any student who is drawing a sequence of pictures to present their work. Many students try to fit everything in a single picture when sometimes more than one picture will make things more clear.
	\item It's good to have more than one student present, especially if you see pictures that look a little different with different groups. As the instructor, be sure to point out how the pictures are similar, how they are different, and what features help us to understand what's happening in the problem. It can be even more powerful if you let students point out things they like about the pictures and things they find confusing.
	\item Throughout this work, look for points where the same physical region is described with two different fractions. This is the main idea we want students to focus on: the whole is necessary for understanding a fraction.
	\item Encourage students to draw rectangles for their wholes rather than circles. Circles aren't incorrect, but they tend to be more difficult to divide into pieces that look equally-sized.
	\item While problem 2 is technically a fraction multiplication problem and problem 3 is a fraction division problem, we don't want algebraic solutions here. It's okay if students check their work this way, but you might ask them questions like ``how do you know this is a multiplication problem? What does multiplication mean?" Be sure to encourage students that we will talk about these ideas later, but we aren't really ready for them yet.
\end{itemize}


{\bf Good language:}  Emphasize each part of our definition, as well as the idea that the pieces we are cutting should be equal! Pay attention to how students word their explanations. Encourage phrasings that make ideas clear, and help students clarify at any point where they are struggling. 

You will also want to talk about what kinds of things we would like to see in students' written explanations as we work through these problems together. While we don't want to give a sample explanation that students could end up copying, we do want to help students to see the important points to highlight in their work. Here, the role of the whole is an important concept to highlight, as well as the definition of fractions (in general) and its specific application to this problem.




{\bf Suggested timing:} Give students about 10-15 minutes to work in groups on these problems. The next 15-20 minutes should be spent with student presentations, and the last 10-15 minutes should be spent on instructor-led discussion about similarities and differences between these solutions as well as how to write fraction explanations.




\end{instructorNotes}



\end{document}