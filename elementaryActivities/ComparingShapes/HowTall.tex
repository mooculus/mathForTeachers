%\documentclass{ximera}
\documentclass[nooutcomes,noauthor]{ximera}
\usepackage{gensymb}
\usepackage{tabularx}
\usepackage{mdframed}
\usepackage{pdfpages}
%\usepackage{chngcntr}

\let\problem\relax
\let\endproblem\relax

\newcommand{\property}[2]{#1#2}




\newtheoremstyle{SlantTheorem}{\topsep}{\fill}%%% space between body and thm
 {\slshape}                      %%% Thm body font
 {}                              %%% Indent amount (empty = no indent)
 {\bfseries\sffamily}            %%% Thm head font
 {}                              %%% Punctuation after thm head
 {3ex}                           %%% Space after thm head
 {\thmname{#1}\thmnumber{ #2}\thmnote{ \bfseries(#3)}} %%% Thm head spec
\theoremstyle{SlantTheorem}
\newtheorem{problem}{Problem}[]

%\counterwithin*{problem}{section}



%%%%%%%%%%%%%%%%%%%%%%%%%%%%Jenny's code%%%%%%%%%%%%%%%%%%%%

%%% Solution environment
%\newenvironment{solution}{
%\ifhandout\setbox0\vbox\bgroup\else
%\begin{trivlist}\item[\hskip \labelsep\small\itshape\bfseries Solution\hspace{2ex}]
%\par\noindent\upshape\small
%\fi}
%{\ifhandout\egroup\else
%\end{trivlist}
%\fi}
%
%
%%% instructorIntro environment
%\ifhandout
%\newenvironment{instructorIntro}[1][false]%
%{%
%\def\givenatend{\boolean{#1}}\ifthenelse{\boolean{#1}}{\begin{trivlist}\item}{\setbox0\vbox\bgroup}{}
%}
%{%
%\ifthenelse{\givenatend}{\end{trivlist}}{\egroup}{}
%}
%\else
%\newenvironment{instructorIntro}[1][false]%
%{%
%  \ifthenelse{\boolean{#1}}{\begin{trivlist}\item[\hskip \labelsep\bfseries Instructor Notes:\hspace{2ex}]}
%{\begin{trivlist}\item[\hskip \labelsep\bfseries Instructor Notes:\hspace{2ex}]}
%{}
%}
%% %% line at the bottom} 
%{\end{trivlist}\par\addvspace{.5ex}\nobreak\noindent\hung} 
%\fi
%
%


\let\instructorNotes\relax
\let\endinstructorNotes\relax
%%% instructorNotes environment
\ifhandout
\newenvironment{instructorNotes}[1][false]%
{%
\def\givenatend{\boolean{#1}}\ifthenelse{\boolean{#1}}{\begin{trivlist}\item}{\setbox0\vbox\bgroup}{}
}
{%
\ifthenelse{\givenatend}{\end{trivlist}}{\egroup}{}
}
\else
\newenvironment{instructorNotes}[1][false]%
{%
  \ifthenelse{\boolean{#1}}{\begin{trivlist}\item[\hskip \labelsep\bfseries {\Large Instructor Notes: \\} \hspace{\textwidth} ]}
{\begin{trivlist}\item[\hskip \labelsep\bfseries {\Large Instructor Notes: \\} \hspace{\textwidth} ]}
{}
}
{\end{trivlist}}
\fi


%% Suggested Timing
\newcommand{\timing}[1]{{\bf Suggested Timing: \hspace{2ex}} #1}




\hypersetup{
    colorlinks=true,       % false: boxed links; true: colored links
    linkcolor=blue,          % color of internal links (change box color with linkbordercolor)
    citecolor=green,        % color of links to bibliography
    filecolor=magenta,      % color of file links
    urlcolor=cyan           % color of external links
}
\title{How tall?}

\begin{document}
\begin{abstract}
\end{abstract}

\maketitle

\begin{problem}
Howard thinks that he heard somewhere that two triangles are similar if all of their angles have the same measure. Does this make sense with our meaning of similarity? Why or why not? Does this same thinking work for quadrilaterals?
\end{problem}

\begin{problem}
Triangle $ABC$ has $\angle ABC$ measuring $36^{\circ}$ and $\angle BCA$ measuring $87^{\circ}$. Triangle $DEF$ has $\angle EFD$ measuring $36^{\circ}$ and $\angle FDE$ measuring $87^{\circ}$. Are these triangles similar? Explain your thinking.
\end{problem}


\begin{problem}
Salma would like to know how tall the Rhodes Tower is in Columbus. She is downtown on a sunny day and so she decides to try a measuring strategy she heard about in school. First, she measures her own shadow, which is about $1$ and a half feet long. Next, she measures the shadow of the tower, which is about $180$ feet long. Salma knows that she is $5$ and a half feet tall. She wants to set up some similar triangles to approximate the height of the tower, but needs a little help.
\begin{enumerate}
	\item Draw triangles to represent the building and its shadow and Salma and her shadow.
	\item Explain why these triangles are similar.
	\item Label the triangles with the information in the problem.
	\item Approximately how tall is the tower? Explain how you know.
\end{enumerate}
\end{problem}




\begin{problem}
The standard interior door height is $80$ inches. Can you use a similar strategy to the shadow problem to find the distance between you and the door to the classroom? To do this, you will need to hold out a ruler and measure the apparent size of the door from where you are sitting.
\begin{enumerate}
	\item Start by drawing similar triangles including the door and your distance to the door, and your outstretched arm and the measuring ruler.
	\item Explain why these triangles are similar.
	\item Measure any sides of the triangle that are easy to measure.
	\item How far are you from the door?
\end{enumerate}
\end{problem}



\newpage

\begin{instructorNotes} 


{\bf Main goal:} We apply our knowledge about similar triangles to solve real-world problems.

{\bf Overall picture:} You will need measuring tapes (from the supply closet) to measure students' arm lengths if you are measuring distances from the door. Students should have their own ruler available to use.


\begin{itemize}
	\item The activity starts with an introduction of the AA criterion for triangle similarity. For the first problem, we are looking for students to say that if the angles are the same, the shapes should be the same (but to contrast with rectangles and squares). The second problem reduces the question to having two of the same angles since the interior angles must sum to $180^{\circ}$.
	\item The tower measuring question might need the additional information that we assume that the sun's rays are parallel if we measure at the same time of day. Students may or may not notice this when looking for the equal angles in the triangles. You may also have to hint at the right angles in the triangles if students don't notice these themselves.
	\item Be sure to emphasize how we are still using the meaning of similarity, even as we have transitioned to using the AA criterion because we are working with triangles.
	\item Have students explain their calculations carefully, pointing out whether they are using a scale factor or an internal factor. As usual the terminology is less important than recognizing both strategies and how they are different.
	\item For the measuring distance from the door problem, you may have to give students the additional instruction that they should measure the distance from their arm to the held ruler. You may also have to encourage students to hold their arms straight out. Students are also sometimes confused if they get a different measurement for the approximate size of the door from their neighbor, but you can ask if they are exactly the same distance away.
	\item Drawing the similar triangles is the challenging step here. There are two versions of the picture, and both should be discussed. One version has the arm and distance to the yardstick at the base of a right triangle. The other has the arm and distance to the yardstick in the center of an isosceles triangle.
	\item There is also an issue here that our arms aren't connected to our eyes. Some students may struggle to draw the picture for this reason, or draw the sighting eye unconnected to the outstretched arm. In your discussion, emphasize that this situation will necessarily mean we are estimating our distance, not calculating it exactly.
	\item For both cases, you should discuss why the triangles are similar using the AA criterion. With overlapping triangles, this can be a bit tricky!
	\item Once we know the triangles are similar, we should be able to apply a scaling argument. You should go over both the scale factor approach as well as the internal factor approach so that students get to practice both techniques.
	\item You will want to discuss why two people who are sitting next to each other might have gotten different answers for their distance to the door. This could include measurement error, calculation error, estimation error, or just the fact that even when we are close together, we are technically different distances away from the board!
	\item You may want to include a short discussion on how indirect measurement is common in life - we often use geometry to find measurements of things that we are either unable or too lazy to make directly.
\end{itemize}

Wrap up by highlighting the use of similar triangles here. This is an application of our similarity work! You can also tell students that this is a simplified version of an actual surveying technique!



{\bf Good language:} Be sure that students are paying attention to and discussing their units of measure here. Are they all the same? Different? Could we measure using both feet and inches?



{\bf Suggested timing:} Give students about 5 minutes to think about the first two problems, then discuss for about 10 minutes. Next, give students about 15 minutes to work on the two applications, and discuss as much as you have time for.




\end{instructorNotes}



\end{document}