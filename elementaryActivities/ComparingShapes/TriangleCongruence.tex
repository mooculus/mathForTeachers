\documentclass{ximera}
\usepackage{gensymb}
\usepackage{tabularx}
\usepackage{mdframed}
\usepackage{pdfpages}
%\usepackage{chngcntr}

\let\problem\relax
\let\endproblem\relax

\newcommand{\property}[2]{#1#2}




\newtheoremstyle{SlantTheorem}{\topsep}{\fill}%%% space between body and thm
 {\slshape}                      %%% Thm body font
 {}                              %%% Indent amount (empty = no indent)
 {\bfseries\sffamily}            %%% Thm head font
 {}                              %%% Punctuation after thm head
 {3ex}                           %%% Space after thm head
 {\thmname{#1}\thmnumber{ #2}\thmnote{ \bfseries(#3)}} %%% Thm head spec
\theoremstyle{SlantTheorem}
\newtheorem{problem}{Problem}[]

%\counterwithin*{problem}{section}



%%%%%%%%%%%%%%%%%%%%%%%%%%%%Jenny's code%%%%%%%%%%%%%%%%%%%%

%%% Solution environment
%\newenvironment{solution}{
%\ifhandout\setbox0\vbox\bgroup\else
%\begin{trivlist}\item[\hskip \labelsep\small\itshape\bfseries Solution\hspace{2ex}]
%\par\noindent\upshape\small
%\fi}
%{\ifhandout\egroup\else
%\end{trivlist}
%\fi}
%
%
%%% instructorIntro environment
%\ifhandout
%\newenvironment{instructorIntro}[1][false]%
%{%
%\def\givenatend{\boolean{#1}}\ifthenelse{\boolean{#1}}{\begin{trivlist}\item}{\setbox0\vbox\bgroup}{}
%}
%{%
%\ifthenelse{\givenatend}{\end{trivlist}}{\egroup}{}
%}
%\else
%\newenvironment{instructorIntro}[1][false]%
%{%
%  \ifthenelse{\boolean{#1}}{\begin{trivlist}\item[\hskip \labelsep\bfseries Instructor Notes:\hspace{2ex}]}
%{\begin{trivlist}\item[\hskip \labelsep\bfseries Instructor Notes:\hspace{2ex}]}
%{}
%}
%% %% line at the bottom} 
%{\end{trivlist}\par\addvspace{.5ex}\nobreak\noindent\hung} 
%\fi
%
%


\let\instructorNotes\relax
\let\endinstructorNotes\relax
%%% instructorNotes environment
\ifhandout
\newenvironment{instructorNotes}[1][false]%
{%
\def\givenatend{\boolean{#1}}\ifthenelse{\boolean{#1}}{\begin{trivlist}\item}{\setbox0\vbox\bgroup}{}
}
{%
\ifthenelse{\givenatend}{\end{trivlist}}{\egroup}{}
}
\else
\newenvironment{instructorNotes}[1][false]%
{%
  \ifthenelse{\boolean{#1}}{\begin{trivlist}\item[\hskip \labelsep\bfseries {\Large Instructor Notes: \\} \hspace{\textwidth} ]}
{\begin{trivlist}\item[\hskip \labelsep\bfseries {\Large Instructor Notes: \\} \hspace{\textwidth} ]}
{}
}
{\end{trivlist}}
\fi


%% Suggested Timing
\newcommand{\timing}[1]{{\bf Suggested Timing: \hspace{2ex}} #1}




\hypersetup{
    colorlinks=true,       % false: boxed links; true: colored links
    linkcolor=blue,          % color of internal links (change box color with linkbordercolor)
    citecolor=green,        % color of links to bibliography
    filecolor=magenta,      % color of file links
    urlcolor=cyan           % color of external links
}

\title{Triangle Congruence}
\begin{document}
\begin{abstract} We say that two triangles are ``congruent'' if we can find a sequence of transformations which are also rigid motions taking the first shape to the second shape. \end{abstract}
\maketitle

\begin{problem}
Let's unpack this definition of congruence. Can you draw two triangles which look congruent? How would you decide whether or not the triangles are actually congruent?
\end{problem}

\begin{problem}
Use your ruler, compass, and protractor to draw another triangle which is congruent to the triangle below. Use at least a translation and a rotation to make the congruent shape. Indicate what transformations you used and how you know you used them correctly. Then explain how you know your triangle is congruent to the original.
\begin{center}
\begin{tikzpicture}
	\draw[thick] (0,0)--(3, -1)--(2,2)--(0,0);
\end{tikzpicture}
\end{center}
\end{problem}


\begin{problem}
Use your ruler, compass, and protractor to draw another triangle which is congruent to the triangle below. Use at least a translation and a reflection to make the congruent shape. Indicate what transformations you used and how you know you used them correctly. Then explain how you know your triangle is congruent to the original.
\begin{center}
\begin{tikzpicture}
	\draw[thick] (0,0)--(4, 1)--(2,2)--(0,0);
\end{tikzpicture}
\end{center}
\end{problem}

\newpage

\begin{problem}
A triangle has length of $AB = 5$, length of $BC = 4$, and $m(\angle ABC) = 60\degree$. Use your ruler, compass, and protractor to draw such a triangle. Do you have more than one option for this triangle, or are all triangles with these measurements congruent? Explain your thinking.
\end{problem}



\begin{problem}
A triangle has $m(\angle ABC) = 30\degree$, length of $BC = 10$, and length of $CA = 7$. Use your ruler, compass, and protractor to draw such a triangle. Do you have more than one option for this triangle, or are all triangles with these measurements congruent? Explain your thinking.
\end{problem}


\newpage

\begin{instructorNotes}

{\bf Main goal:} We introduce congruence and congruence criteria for triangles.

{\bf Overall picture:} Our main goal here is not the congruence criteria themselves, but to continue to investigate properties of quadrilaterals. The next activity will have us applying the congruence criteria for triangles in the case of some special quadrilaterals, so we want to treat this as introducing a tool we can use for investigating other shapes and properties.
 
\begin{itemize}
	\item The first two questions give us another opportunity to practice with transformations. Go ahead and spend plenty of time here. The questions are non-specific enough that different groups should have different answers; have at least two groups present their work for each problem.
	\item The last two questions will give us an opportunity to introduce the triangle congruence theorems. The first problem should have all such triangles congruent; have one group present and then discuss as a class whether we can draw more such triangles. Explain where the construction has to make exactly one kind of triangle. Return to the congruence definition and have students think about what transformations would produce the congruence. The last question should produce two distinct triangles. Have at least two groups present two different results, and then in your discussion compare and contrast with the previous problem.
	\item After this activity, be sure to give the list of congruence theorems and show how to use at least one of them. We'll need them in the next activity!
\end{itemize}

{\bf Good language:} Memorizing the specific names of the congruence theorems isn't important; we want students more focused on transformations and the fact that there even are such theorems. While working on transformations, be sure to have students return to the definitions and explain how they know they have constructed the correct figure from the definition of a translation, reflection, or rotation.

{\bf Suggested timing:} Give students about 10 minutes to work on Problems 1 and 2, and then take 15 minutes for presentation and 5 minutes to discuss any overall themes that come up in the student work (or to go over again what to explain when we work on transformations). Then give students about 5-10 minutes to work on Problems 3 and 4, take 10 minutes for discussion, and use the last 5 minutes to wrap up with presenting the congruence theorems. Make sure to present these theorems even if you have to cut the work on Problems 3 and 4 a little short (or skip the second half entirely if students still need a lot of help on transformations).


\end{instructorNotes}



\end{document}