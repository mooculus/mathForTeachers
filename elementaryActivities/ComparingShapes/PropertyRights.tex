\documentclass[noauthor, nooutcomes]{ximera}

\graphicspath{
  {./}
  {graphics/}
  {../graphics/}
}

\usepackage{chngcntr}

\let\question\relax
\let\endquestion\relax




\newtheoremstyle{SlantTheorem}{\topsep}{\fill}%%% space between body and thm
%\newtheoremstyle{SlantTheorem}{\topsep}{\topsep}%%% space between body and thm
 {\slshape}                      %%% Thm body font
 {}                              %%% Indent amount (empty = no indent)
 {\bfseries\sffamily}            %%% Thm head font
 {}                              %%% Punctuation after thm head
 {3ex}                           %%% Space after thm head
 {\thmname{#1}\thmnumber{ #2}\thmnote{ \bfseries(#3)}}%%% Thm head spec
\theoremstyle{SlantTheorem}
\newtheorem{question}{Question}
\counterwithin*{question}{section}



\let\instructorNotes\relax
\let\endinstructorNotes\relax
%%% instructorNotes environment
\ifhandout
\newenvironment{instructorNotes}[1][false]%
{%
\def\givenatend{\boolean{#1}}\ifthenelse{\boolean{#1}}{\begin{trivlist}\item}{\setbox0\vbox\bgroup}{}
}
{%
\ifthenelse{\givenatend}{\end{trivlist}}{\egroup}{}
}
\else
\newenvironment{instructorNotes}[1][false]%
{%
  \ifthenelse{\boolean{#1}}{\begin{trivlist}\item[\hskip \labelsep\bfseries {\Large Instructor Notes: \\} \hspace{\textwidth} ]}
{\begin{trivlist}\item[\hskip \labelsep\bfseries {\Large Instructor Notes: \\} \hspace{\textwidth} ]}
{}
}
{\end{trivlist}}
\fi


%% Suggested Timing
\newcommand{\timing}[1]{{\bf Suggested Timing: \hspace{2ex}} #1}


\title{Property Rights}
\begin{document}
\begin{abstract}\end{abstract}
\maketitle

Teachers: \link[Amplify classroom version]{https://classroom.amplify.com/activity/68b745042afd50cf431c925e?utm_campaign=share&utm_content=activity}

Students: for the Amplify Classroom version, your teacher should give you a link.


\begin{problem}
Draw a diagonal of the shape below. 
\begin{image}
\begin{tikzpicture}
\draw[thick] (0,0)--(3, -2)--(5, -1)--(2, 2)--(0,0);
\end{tikzpicture}
\end{image}
Is there more than one correct answer to this question? Explain your thinking.
\end{problem}


\begin{problem}
Bisect the angle below. 
\begin{image} \begin{tikzpicture}
\draw[thick] (4, -1)--(0,0)--(5, 2);
\end{tikzpicture} \end{image}
Is there more than one correct answer to this question? Explain your thinking.
\end{problem}

\newpage
\begin{problem}
Bisect the segment below. 
\begin{center} \begin{tikzpicture}
\draw[thick] (0, 4)--(3, -1);
\end{tikzpicture} \end{center}
Is there more than one correct answer to this question? Explain your thinking.
\end{problem}

Here is a list of properties that we will use for the next several questions.
\begin{enumerate}[label=P{\arabic*}.]
    \item Opposite sides have equal length.
    \item Opposite angles are equal.
    \item Opposite sides are parallel. %new instead of symmetry
    \item The diagonals have equal length.
    \item The diagonals bisect each other.
   % \item The diagonals meet at right angles.
    \item The diagonals bisect the interior angles.
    %\item A diagonal is a line of symmetry.
\end{enumerate} 

\pagebreak

\begin{problem}
Here is a rhombus.
\begin{image} \begin{tikzpicture}
\draw[thick] (0, 3)--(1, 1) --(0, -1)--(-1,1)--(0,3);
\end{tikzpicture} \end{image}
\begin{enumerate}
	\item Explain how you would use the definition of a rhombus to show that this figure is actually a rhombus.
	\item Which of the properties above does this rhombus have? Explain how you checked.
\end{enumerate}
\end{problem}

\pagebreak

\begin{problem}
Here is a kite.
\begin{image} \begin{tikzpicture}
\draw[thick] (0, 3)--(2, 2)--(0, -2)--(-2, 2)--(0, 3);
\end{tikzpicture} \end{image}
\begin{enumerate}
	\item Explain how you would use the definition of a kite to show that this figure is actually a kite.
	\item Which of the properties above does this kite have? Explain how you checked.
\end{enumerate}
\end{problem}

\pagebreak
\begin{problem}
Here is a parallelogram.
\begin{image} \begin{tikzpicture}
\draw[thick] (0, 4)--(2,6)--(5, 0)--(3, -2)--(0,4);
\end{tikzpicture} \end{image}
\begin{enumerate}
	\item Explain how you would use the definition of a parallelogram to show that this figure is actually a parallelogram.
	\item Which of the properties above does this parallelogram have? Explain how you checked.
\end{enumerate}
\end{problem}




\begin{problem}
Look back over your work checking properties in the previous problems.
\begin{enumerate}
	\item Which properties were easy to check, and why?
	\item Which properties were hard to check, and why?
	\item Would you guess that the same properties hold and don't hold for other examples of rhombuses, kites, and parallelograms? Draw a few other examples to explain your thinking. 
\end{enumerate}
\end{problem}

\newpage


\begin{problem}
For each of the properties P1 through P6, draw another example of a shape that would fit in that category and a shape that would not fit in that category. 
\end{problem}


\begin{problem}
For each item below, sketch examples of shapes that fit in each category but do not fit in the other, and then examples of shapes that fit in both categories. Then organize your sketches into a Venn diagram.

\begin{enumerate}
	\item P4 and P6
	\item P1 and P4
	\item P3 and P4 and P5
\end{enumerate}
\end{problem}




\newpage

\begin{instructorNotes}

{\bf Main goal:} We use properties of special quadrilaterals to classify quadrilaterals.


{\bf Overall picture:} 

This activity is intended to give students more experience with various special kinds of quadrilaterals, and to help them discover properties of these quadrilaterals.  The discovery is done in a way that can seem more relevant to the grade levels they will be teaching. After this discussion, students can assume (unless asked to justify) that the special quadrilaterals have the properties we discuss here.

\begin{enumerate}
\item The Amplify classroom version is for the problems through checking the properties; the follow up questions are not included there.
\item The preliminary questions are included to be sure that students are on the same page with the necessary definitions for checking properties. We have frequently found that students are unsure of the definition of a diagonal or confuse ``bisector'' with ``perpendicular bisector'' for segments.
\item The first focus is on definitions, to emphasize again that we check the definition to ensure we are working with a particular shape or concept. This will likely be tough with parallelograms, and students should check whether the lines are parallel by using the converse of the Parallel Postulate.
\item We want to distinguish definitions from properties with this activity. Students can really struggle with including extra properties in their definitions and we want to be looking for this throughout the activity.
\item An extension here would be to ask if, for instance, the rhombus fulfills any other definitions. We want to start talking about overlaps in categories of shapes (though this may come up more naturally with the Venn diagrams towards the end).
\item Checking the properties should be done both by measuring with ruler and protractor as well as by folding and matching different parts of the shape (if you are using cutout shapes). Both strategies should be discussed! We can also bring tracing paper if it's helpful for students to explore the shapes.
\item In discussion you might talk about how imprecise measurements can affect our results, especially if the students are struggling with this idea.
\item You might ask students how they could use their compass to check if we have congruent segments.  This is a nice connection to our work with circles.
\item Problems 7--9 (starting with Look back) are follow-up problems indicated to help us talk about categorizing shapes. These properties are another way to think about building categories, but we want to keep properties distinct from definitions. 
\item When talking about whether these shapes hold more generally, we can start talking about relationships between special quadrilaterals. Generally we want to see that these properties should hold for all examples, but that we need triangle congruence to be able to prove these observations. We'll return to some of this later.
\item You may need to introduce Venn diagrams to be able to do the final problem; feel free to do part (a) together so that the students know what to expect. Students should be encouraged to think of new examples that aren't amongst the shapes we've been investigating as they populate their Venn diagrams. Students should also be encouraged to draw correct relationships between the circles in the Venn diagrams, adjusting if they need to as they think about examples. For instance, P2 and P3 should be the same property, so this question is a bit tricky!
\item At the end of this activity, we should give a short wrap-up indicating which properties hold for which shapes, and which of the properties we should assume.
\end{enumerate}

%{\bf Good language:} 

{\bf Suggested timing:} We have two class periods to explore this activity. On the first day, give students about 5-10 minutes to get situated with the first three problems and cut out their shapes (if using paper copies), and then spend about 15 minutes on this discussion. Students will need significant time to work through the properties, so the rest of the time on the first day can be spent on this and discussion. On the second day, give students about 15 minutes to think about Problems 7--9. Use most of rest of the time to have the groups present their findings. Wrap up with an overall discussion clarifying which properties are true for which shapes, what students could be expected to prove (using the Parallel Postulate), and what we can now assume (properties which require triangle congruence to prove).
\end{instructorNotes}


\end{document}

