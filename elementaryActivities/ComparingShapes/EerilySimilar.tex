\documentclass[noauthor,nooutcomes]{ximera}
\usepackage{gensymb}
\usepackage{tabularx}
\usepackage{mdframed}
\usepackage{pdfpages}
%\usepackage{chngcntr}

\let\problem\relax
\let\endproblem\relax

\newcommand{\property}[2]{#1#2}




\newtheoremstyle{SlantTheorem}{\topsep}{\fill}%%% space between body and thm
 {\slshape}                      %%% Thm body font
 {}                              %%% Indent amount (empty = no indent)
 {\bfseries\sffamily}            %%% Thm head font
 {}                              %%% Punctuation after thm head
 {3ex}                           %%% Space after thm head
 {\thmname{#1}\thmnumber{ #2}\thmnote{ \bfseries(#3)}} %%% Thm head spec
\theoremstyle{SlantTheorem}
\newtheorem{problem}{Problem}[]

%\counterwithin*{problem}{section}



%%%%%%%%%%%%%%%%%%%%%%%%%%%%Jenny's code%%%%%%%%%%%%%%%%%%%%

%%% Solution environment
%\newenvironment{solution}{
%\ifhandout\setbox0\vbox\bgroup\else
%\begin{trivlist}\item[\hskip \labelsep\small\itshape\bfseries Solution\hspace{2ex}]
%\par\noindent\upshape\small
%\fi}
%{\ifhandout\egroup\else
%\end{trivlist}
%\fi}
%
%
%%% instructorIntro environment
%\ifhandout
%\newenvironment{instructorIntro}[1][false]%
%{%
%\def\givenatend{\boolean{#1}}\ifthenelse{\boolean{#1}}{\begin{trivlist}\item}{\setbox0\vbox\bgroup}{}
%}
%{%
%\ifthenelse{\givenatend}{\end{trivlist}}{\egroup}{}
%}
%\else
%\newenvironment{instructorIntro}[1][false]%
%{%
%  \ifthenelse{\boolean{#1}}{\begin{trivlist}\item[\hskip \labelsep\bfseries Instructor Notes:\hspace{2ex}]}
%{\begin{trivlist}\item[\hskip \labelsep\bfseries Instructor Notes:\hspace{2ex}]}
%{}
%}
%% %% line at the bottom} 
%{\end{trivlist}\par\addvspace{.5ex}\nobreak\noindent\hung} 
%\fi
%
%


\let\instructorNotes\relax
\let\endinstructorNotes\relax
%%% instructorNotes environment
\ifhandout
\newenvironment{instructorNotes}[1][false]%
{%
\def\givenatend{\boolean{#1}}\ifthenelse{\boolean{#1}}{\begin{trivlist}\item}{\setbox0\vbox\bgroup}{}
}
{%
\ifthenelse{\givenatend}{\end{trivlist}}{\egroup}{}
}
\else
\newenvironment{instructorNotes}[1][false]%
{%
  \ifthenelse{\boolean{#1}}{\begin{trivlist}\item[\hskip \labelsep\bfseries {\Large Instructor Notes: \\} \hspace{\textwidth} ]}
{\begin{trivlist}\item[\hskip \labelsep\bfseries {\Large Instructor Notes: \\} \hspace{\textwidth} ]}
{}
}
{\end{trivlist}}
\fi


%% Suggested Timing
\newcommand{\timing}[1]{{\bf Suggested Timing: \hspace{2ex}} #1}




\hypersetup{
    colorlinks=true,       % false: boxed links; true: colored links
    linkcolor=blue,          % color of internal links (change box color with linkbordercolor)
    citecolor=green,        % color of links to bibliography
    filecolor=magenta,      % color of file links
    urlcolor=cyan           % color of external links
}
\title{Eerily Similar}

\begin{document}
\begin{abstract}
\end{abstract}
\maketitle

\begin{problem}
What does it mean for two objects to be ``similar'' in a mathematical sense? Draw two examples of objects that are similar and two examples of objects that are not similar. 
\end{problem}


\begin{problem}
Ivory has a rectangle which is $3$cm long and $8$ cm wide. Logan has a rectangle which is $10$cm long and $5$cm wide. Are these two rectangles similar? Explain your answer in two different ways.
\end{problem}


\begin{problem}
Willow has a circle with a radius of $4.5$ inches. River has a circle with a radius of $15.75$ inches. Are these two circles similar? Explain your answer in two different ways.
\end{problem}



\begin{problem}

A gift shop at a popular national park wants to scale up the photograph on one of their postcards to sell as a piece of artwork. The original post card is $4$ inches tall and $6$ inches wide. If the artwork will be $10$ inches tall, how wide should it be? Solve this problem in two different ways. Explain your work in each case. How did the meaning of similarity apply in your solution?

\end{problem}





\begin{problem}
An action figure company wants to make a keychain size figure of their most popular full size figure. The full size figure is $12$ inches tall and its legs are $3$ inches long. If the height of the keychain size figure will be $3$ inches, how long will the keychain size figure's legs be? Solve this problem in two different ways. Explain your work in each case. How did the meaning of similarity apply in your solution?



\end{problem}




\newpage

\begin{instructorNotes}
{\bf Main goal:} We define mathematical similarity and solve problems using similarity.

{\bf Overall picture:} We would like students to move towards thinking about similarity as ``one object is a scaled version of the other'', or more specifically that we can find a scale factor $k$ for lengths between the two objects: every length in object 1 is scaled by a factor of $k$ to produce object 2. This is much more specific than a colloquial meaning for similarity!

We expect the two main methods to be using a scale factor (between the object and the similar version) and an internal factor (a ratio of lengths on the original which is preserved in the scaled object). Other methods might include drawing a picture to show how the original version fits into the scaled version, or using a proportion. 

\begin{itemize}
 \item Gather students' thoughts about similarity after they have worked through the first problem. These ideas should center around one item being a scaled version of another, but they may also say that they are the same shape but different sizes, or talk about a zoomed-in or zoomed-out version of the object. Our goal is to ``mathematize'' these ideas. Usually someone will come up with an idea similar to a scale factor, but if not you can ask students to think for a minute about what we might calculate to determine whether one object was a scaled version of another.
 \item As you discuss the first problem, be sure to finish up by drawing students ideas together to introduce the meaning of similarity and the idea of a scale factor. You can also emphasize any ratio ideas that also came up. 
 \item Watch out for students who claim rectangles and squares are similar because they are ``kind of the same''. This isn't as precise as we want to be for mathematical similarity. 
 \item The next two problems are a first look at solving similarity problems. In the rectangles problem, we have added two to both length and width. Students should notice that these are no longer the same shape (perhaps using a ratio of length to width) and that we do not have a consistent scale factor. With the circle problem, since the circle gives us all points at distance $r$ from the center, these are all being scaled and the circles are similar. This will help us talk about the meaning of $\pi$ later in the course.
 \item The postcard should then be a practical application of scaling, and the action figure an example where we don't have side lengths to measure (the figure is not a rectangle). A further extension would be to a problem where the lengths on one object are given in different units than the lengths on the other (ie $4$in and $8$in versus $9$m and $18$m).
 \item Using a proportion is allowed, but students should explain their work. As a reminder, the two fractions are likely unit rates for the same ratio, so they are equal, and then the method of cross-multiplication is a short-cut for making equivalent fractions. 
 \item Have students explain carefully any multiplication or division they do in terms of the meaning of the operation. The goal here is to understand why students are taking the steps they are taking.
 \item Emphasize the definition of similarly as often as you can. You can ask students to practice using the definition of similarity to explain how they know the two objects are similar (which is one aspect of our similarity outcome) as well as have students identify when they are using the definition in their calculations. You might also keep bringing up the more informal version of the definition, which is that objects are similar when they are the same shape but different sizes. You can also continue to point out that any length is being scaled by the same scale factor, as well as the fact that angles don't change under scaling.
 \item Later we will refer to the scale factor as a ``length scale factor'' to distinguish it from how area and volume are being scaled. That language isn't necessary now, but you should try to emphasize that we are using the scale factor only on lengths.
\end{itemize}








{\bf Good language:} Students should be able to identify the scale factor when they are using one, and perhaps an internal factor as well, though we don't require students to use the second term. Otherwise, continue to encourage students to talk about how their work is connected to the meaning of similarity.

{\bf Suggested timing:}  Give students about 5 minutes to think about their scaling examples, then discuss for about 10 minutes, making sure to show many examples of similar and not similar objects. Next, give students about 10 minutes to work through as many of the rest of the problems as they can, and spend 20 minutes in presentations and discussion.

\end{instructorNotes}


\end{document}





