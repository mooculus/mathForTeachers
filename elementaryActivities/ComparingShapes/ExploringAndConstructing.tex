\documentclass[noauthor, nooutcomes,handout]{ximera}
%\documentclass[handout]{ximera}

\graphicspath{
  {./}
  {graphics/}
  {../graphics/}
}

\usepackage{chngcntr}

\let\question\relax
\let\endquestion\relax




\newtheoremstyle{SlantTheorem}{\topsep}{\fill}%%% space between body and thm
%\newtheoremstyle{SlantTheorem}{\topsep}{\topsep}%%% space between body and thm
 {\slshape}                      %%% Thm body font
 {}                              %%% Indent amount (empty = no indent)
 {\bfseries\sffamily}            %%% Thm head font
 {}                              %%% Punctuation after thm head
 {3ex}                           %%% Space after thm head
 {\thmname{#1}\thmnumber{ #2}\thmnote{ \bfseries(#3)}}%%% Thm head spec
\theoremstyle{SlantTheorem}
\newtheorem{question}{Question}
\counterwithin*{question}{section}



\let\instructorNotes\relax
\let\endinstructorNotes\relax
%%% instructorNotes environment
\ifhandout
\newenvironment{instructorNotes}[1][false]%
{%
\def\givenatend{\boolean{#1}}\ifthenelse{\boolean{#1}}{\begin{trivlist}\item}{\setbox0\vbox\bgroup}{}
}
{%
\ifthenelse{\givenatend}{\end{trivlist}}{\egroup}{}
}
\else
\newenvironment{instructorNotes}[1][false]%
{%
  \ifthenelse{\boolean{#1}}{\begin{trivlist}\item[\hskip \labelsep\bfseries {\Large Instructor Notes: \\} \hspace{\textwidth} ]}
{\begin{trivlist}\item[\hskip \labelsep\bfseries {\Large Instructor Notes: \\} \hspace{\textwidth} ]}
{}
}
{\end{trivlist}}
\fi


%% Suggested Timing
\newcommand{\timing}[1]{{\bf Suggested Timing: \hspace{2ex}} #1}

\title{Exploring and Constructing Shapes}
\begin{document}
\begin{abstract}

\end{abstract}
\maketitle
In this activity, we will use paper folding to create some shapes.  Then we will use these shapes to explore their properties. Finally, we will use elementary school reasoning to justify that those properties will hold every time we make such shapes. You will need scrap paper to do this. 

\begin{problem}

 
In each case, explain how you know that you have created the required picture.  Write your justifications in the space below. You may use your ruler, but only to make straight lines. No measuring!
\begin{enumerate}
    \item Draw a line segment on a scrap paper.  Fold your paper to make a line that will bisect this segment. No measuring - just reasoning! How do you know that you have created two equal parts in the segment?
    \item Draw a line segment on a scrap paper.  Fold your paper to make another line that is perpendicular to this segment. No protractors - just reasoning! How do you know that the fold is perpendicular to the segment?
    \item Draw an angle on a scrap paper. (You may use a ruler as a straightedge, but still no measuring!) Fold your paper to bisect this angle.  How do you know that you have bisected the angle?
\end{enumerate}

\end{problem}

\vfill



\newpage


\begin{problem}
Start with a new piece of scrap paper and follow the instructions below to make a rhombus using ONLY folding.  \

Instructions:
\begin{enumerate}
\item Fold your piece of paper vertically.
\item Now fold your already-folded paper again, but now fold it horizontally.
\item  Make one cut to create a rhombus.
\item How do you know that you have a rhombus?
\end{enumerate}
\end{problem}

\vfill

\begin{problem}
Now use your paper rhombus to explore the following conjectures. (Note: A ``conjecture" is a statement of something that you observed and think is true.  We state the idea, then try to apply reasoning to check it out!) \\
\vskip 0.1in
Conjecture 1: The diagonals of a rhombus form right angles.\\

Conjecture 2: The diagonals of a rhombus bisect the interior angles of the rhombus.\\

Are either of these conjectures true for your rhombus? Will these hold true for some rhombuses? no rhombuses? all rhombuses? Explain using the thinking of an elementary school student. Assume they know our definition of ``angle".
\end{problem}


\vfill


\pagebreak

\begin{problem}
Using yet another piece of scrap paper, draw an isosceles triangle using ONLY your compass and a straight edge.  No measuring!

\end{problem}


\pagebreak

\begin{problem}
Draw a line segment near the bottom of this page that is 2 inches long. Now, use only your compass and straightedge to construct a rhombus.  Be ready to explain your steps. (Hint: What is the definition of a rhombus?)
\end{problem} 

\pagebreak

\begin{instructorNotes}

{\bf Main goal:} Students learn to create accurate versions of several geometric shapes by folding and by using their construction tools.

{\bf Overall picture:} 
The overall purpose of this activity is to engage students in some  geometric ``play" and also to recognize that their upper elementary students will be able to reason about these shapes in some very basic ways if they know the definitions of angle, line, triangle, rhombus, circle, etc.  We also hope that they will experience a kind of natural cycle of mathematics: explore (play), observe, conjecture, and justify (at an appropriate level). We feel that this is a powerful learning cycle that reflects the work of mathematicians and generates more questions and a love of learning!\\

In this activity, you may need to:
\begin{itemize}
    \item assist students in their use of the construction tools
    \item remind students to use the definitions of the shapes
    \item remind our future teachers that children can do this
    \item and focus on reasoning from the physical shapes!
\end{itemize}

Problem 1:
\begin{itemize}
	\item We begin by thinking about how folding can help us make relatively accurate constructions. The main idea is to match pieces of the object together, and describe how the matching guarantees that two objects are the same. We are laying one part of the object exactly on top of the other side, creating a sort of one-to-one correspondence between the two parts.
	\item For part (b), we are again connecting lines to straight angles. Matching the two sides of the straight angle guarantees that we have divided the angle into two equal pieces, creating the 90 degree angles we want.
\end{itemize}

Problems 2 and 3:
\begin{itemize}
	\item Remember to emphasize using the definition of a rhombus to ensure that what we've created is actually a rhombus. The fact that we cut all pieces of the paper at the same time to create all four sides at the same time guarantees that all four sides are the same length.
	\item Using our paper rhombus helps us to justify at the level of young children some of the properties we previously stated about rhombuses. You should connect this idea to the ``tearing off corners'' proof we did for the interior angles of a triangle: we haven't checked all rhombuses, but we have developed some intuition about why we expect these ideas to be true.
	\item Continue to emphasize how the folding produces a ``matching'' between the two pieces. 
\end{itemize}

Problems 4 and 5:
\begin{itemize}
	\item Remind students to consider the definition of an isosceles triangle, and use that definition to motivate their use of the tools. As you discuss, try to connect each step to a definition.
	\item Be sure to highlight why a circle is helpful here, connecting to our previous activity. We should be drawing circles because we are trying to identify points at a specified distance from a specified point.
	\item Students will have several strategies for this problem, and they are each worth discussing: some students begin with a line segment and draw two circles for the two equal sides. Other students draw a single circle and use two radii for the two equal sides. Some students may accidentally construct their rhombus here instead of in the next problem!
	\item The rhombus should hopefully feel like an extension of the isosceles triangle, but students may have trouble connecting the two isosceles triangles together. Again, encourage students to rely on the definition of the rhombus and the definition of the circle, and emphasize the connection here. Circles are particularly helpful when we would like to locate a point at a certain distance from two other points, as with our first problem in Buried Treasure.
\end{itemize}


{\bf Suggested timing:} We have two days to complete this activity. The problems in the activity should be discussed one at a time: give students 5-10 minutes to think about the problem, and then discuss as a class. Be sure to have students present their work or talk about their thinking before drawing out the main idea in each problem. 

We may not finish this activity even over two class periods, but the experience is the thing!
\end{instructorNotes}



\end{document}


%some additional problems

% \begin{problem}
% Now draw an angle on a scrap of paper.  Bisect this angle using your compass and straightedge only.  (Hint:  Imagine this angle is part of a rhombus)
% \end{problem}
 
%\begin{problem}
%Can you draw a rhombus that is not a square?  Can you draw a square that is not a rhombus?
%\end{problem}

%\begin{problem}
%What about a square that is not a rectangle?  A rectangle that is not a square?
%\end{problem}


%\begin{problem}
%Use your compass to construct a tall, ``skinny" isosceles triangle on another piece of paper.  Cut it out.  Using this paper triangle and possibly some folding, what properties of an isosceles triangle can you argue must be true?  List them here and then write down how you will argue their truth.

%\end{problem}




%\begin{problem}
%Use your compass to construct an equilateral triangle on another piece of paper.  Cut it out.  Using this paper triangle and possibly some folding, what properties of an equilateral triangle can you argue must be true?  List them here and then write down how you will argue their truth.

%\end{problem}
