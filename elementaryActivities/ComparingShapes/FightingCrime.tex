\documentclass[noauthor,nooutcomes]{ximera}
\usepackage{gensymb}
\usepackage{tabularx}
\usepackage{mdframed}
\usepackage{pdfpages}
%\usepackage{chngcntr}

\let\problem\relax
\let\endproblem\relax

\newcommand{\property}[2]{#1#2}




\newtheoremstyle{SlantTheorem}{\topsep}{\fill}%%% space between body and thm
 {\slshape}                      %%% Thm body font
 {}                              %%% Indent amount (empty = no indent)
 {\bfseries\sffamily}            %%% Thm head font
 {}                              %%% Punctuation after thm head
 {3ex}                           %%% Space after thm head
 {\thmname{#1}\thmnumber{ #2}\thmnote{ \bfseries(#3)}} %%% Thm head spec
\theoremstyle{SlantTheorem}
\newtheorem{problem}{Problem}[]

%\counterwithin*{problem}{section}



%%%%%%%%%%%%%%%%%%%%%%%%%%%%Jenny's code%%%%%%%%%%%%%%%%%%%%

%%% Solution environment
%\newenvironment{solution}{
%\ifhandout\setbox0\vbox\bgroup\else
%\begin{trivlist}\item[\hskip \labelsep\small\itshape\bfseries Solution\hspace{2ex}]
%\par\noindent\upshape\small
%\fi}
%{\ifhandout\egroup\else
%\end{trivlist}
%\fi}
%
%
%%% instructorIntro environment
%\ifhandout
%\newenvironment{instructorIntro}[1][false]%
%{%
%\def\givenatend{\boolean{#1}}\ifthenelse{\boolean{#1}}{\begin{trivlist}\item}{\setbox0\vbox\bgroup}{}
%}
%{%
%\ifthenelse{\givenatend}{\end{trivlist}}{\egroup}{}
%}
%\else
%\newenvironment{instructorIntro}[1][false]%
%{%
%  \ifthenelse{\boolean{#1}}{\begin{trivlist}\item[\hskip \labelsep\bfseries Instructor Notes:\hspace{2ex}]}
%{\begin{trivlist}\item[\hskip \labelsep\bfseries Instructor Notes:\hspace{2ex}]}
%{}
%}
%% %% line at the bottom} 
%{\end{trivlist}\par\addvspace{.5ex}\nobreak\noindent\hung} 
%\fi
%
%


\let\instructorNotes\relax
\let\endinstructorNotes\relax
%%% instructorNotes environment
\ifhandout
\newenvironment{instructorNotes}[1][false]%
{%
\def\givenatend{\boolean{#1}}\ifthenelse{\boolean{#1}}{\begin{trivlist}\item}{\setbox0\vbox\bgroup}{}
}
{%
\ifthenelse{\givenatend}{\end{trivlist}}{\egroup}{}
}
\else
\newenvironment{instructorNotes}[1][false]%
{%
  \ifthenelse{\boolean{#1}}{\begin{trivlist}\item[\hskip \labelsep\bfseries {\Large Instructor Notes: \\} \hspace{\textwidth} ]}
{\begin{trivlist}\item[\hskip \labelsep\bfseries {\Large Instructor Notes: \\} \hspace{\textwidth} ]}
{}
}
{\end{trivlist}}
\fi


%% Suggested Timing
\newcommand{\timing}[1]{{\bf Suggested Timing: \hspace{2ex}} #1}




\hypersetup{
    colorlinks=true,       % false: boxed links; true: colored links
    linkcolor=blue,          % color of internal links (change box color with linkbordercolor)
    citecolor=green,        % color of links to bibliography
    filecolor=magenta,      % color of file links
    urlcolor=cyan           % color of external links
}
\title{Fighting crime}

\begin{document}
\begin{abstract}
\end{abstract}
\maketitle

\begin{problem}

An evil villain is making plans to tie up Spiderman in an abandoned warehouse. The warehouse has two long chains attached to the floor at a distance of $60$ feet apart. The chain on the left is $30$ feet long and the chain on the right is $50$ feet long. The villain would like to attach a chair to the floor where the two chains meet when they are fully extended so that he can use the chair and the two chains to tie up Spiderman. What are all of the locations in the warehouse should this villain put the chair? You can draw your own picture of this situation or use the sketch below showing where the two chains $R$ and $L$ are attached to the floor.  Finally, explain your work. How do you know you've found {\bf all} possible locations where the villain could place the chair?

\vfill
\begin{image}
\begin{tikzpicture}
\draw[thick] (0,0)--(6,0);
\node[above] at (0,0) {$L$};
\node[above] at (6,0) {$R$};
\draw[fill=black] (0,0) circle (2pt);
\draw[fill=black] (6,0) circle (2pt);
\end{tikzpicture}
\end{image}
\vfill


\end{problem}

\newpage

\begin{problem}
A detective has information about where two crimes have occurred that she thinks are connected. Let's call the two crime scenes $A$ and $B$. She is trying to predict the location of the next crime. The two crime scenes are $18$ miles apart, and based on past crime data, the detective believes that the next crime will be less than $14$ miles from crime scene $A$ and less than $10$ miles from crime scene $B$.  Using a scale of $\frac{1}{4}$ inch represents 2 miles, draw all of the locations where the detective thinks the next crime will occur (and thus where she should recommend the police patrol).  Explain your picture and how you know that you have shown all possible locations.


\end{problem}

\pagebreak

\begin{problem}
Batman and Robin are trying to save Gotham City from a laughing gas bomb hidden by the Joker.  They have split up to find the bomb faster.  Each of them has taken a bomb-detecting device into a skyscraper near where they suspect the bomb is hidden.  At a certain moment, Batman's device indicates that the bomb is 30 feet from his location.  At the same moment, Robin's device indicates that the bomb is 50 feet from his location.  What can you say about the location of the bomb? (Note that this question is a bit different because it does not specify where Batman and Robin are - of course, they are furiously hunting through the buildings to find the bomb in time to disarm it!)
\end{problem}

\pagebreak

\begin{instructorNotes}

{\bf Main goal:} We introduce the definition of a circle (and potentially a sphere) through problem-solving.

{\bf Overall picture:} 
In this activity, students are to develop the definition of a circle via solving the first problem.  They hopefully will recognize that the possible locations of the chair are an equal distance away from the specified points. By combining with other circles, they can then narrow down the possible locations.  After this problem, ask the students to write down a formal definition of a circle.

In your discussion, ask them how a compass can be helpful in drawing a circle in terms of the mechanics of the compass and how the mechanics (open to a fixed distance) connect to the newly-formed definition.  Thinking about producing a given distance using the circle-drawing tool will be helpful in the next activity when they will be asked to construct points a specified distance from a fixed point.

There are two technical points which are important here: first, that the center of the circle is necessary for drawing the circle, but is not part of the circle. Second: there is a difference between the circle (just the outside) and the disk (the circle plus the space inside). With young children (especially in kindergarten) we tend to conflate the two ideas (asking children to select the ``yellow circle'', when we mean the yellow disk). While this is appropriate in kindergarten, we want our students to differentiate these ideas so that we can appropriately use circles later.

After defining the circle, students should extend their work with the other problems in this activity. The detective problem is helpful for future examples where we locate a region rather than an individual point. In your discussion, help students to translate the wording of the problem to their solution so that they can distinguish the different regions we could ask for in the future (inside one circle but not the other, inside both circles, etc).

The Batman and Robin problem helps to bring up the sphere as a 3D version of a circle. If you have time for this problem, have students give a definition of the sphere as you discuss. Be sure to have students share their drawings as well as discuss how they are dealing with the extension to 3D space instead of 2D. You should also discuss in this case what the intersection of two spheres could possibly look like (you might find an animation on the internet to help).

Finally, if you have more time, you can wrap up by connecting circles to polygons. You can talk with them about an alternative definition for a circle:  the limit of regular polygons as the number of sides increase without bound.  This can be a helpful guide when developing an alternative argument for the formula for the area of a circle. 

{\bf Good language:} It's sometimes helpful to refer to the compass as a ``circle-drawing tool". Students can often get tangled between the name of the protractor and the name of the compass. 

{\bf Suggested timing:}  Give students about 5-10 minutes to work on Problem 1, then take 15 minutes to discuss their solutions as well as the definition of a circle. Then give students another 10 minutes to think about the rest of the problems, and spend what time you have remaining in presentations and discussion.
\end{instructorNotes}

\end{document}