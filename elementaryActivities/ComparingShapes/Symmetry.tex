\documentclass{ximera}
\usepackage{gensymb}
\usepackage{tabularx}
\usepackage{mdframed}
\usepackage{pdfpages}
%\usepackage{chngcntr}

\let\problem\relax
\let\endproblem\relax

\newcommand{\property}[2]{#1#2}




\newtheoremstyle{SlantTheorem}{\topsep}{\fill}%%% space between body and thm
 {\slshape}                      %%% Thm body font
 {}                              %%% Indent amount (empty = no indent)
 {\bfseries\sffamily}            %%% Thm head font
 {}                              %%% Punctuation after thm head
 {3ex}                           %%% Space after thm head
 {\thmname{#1}\thmnumber{ #2}\thmnote{ \bfseries(#3)}} %%% Thm head spec
\theoremstyle{SlantTheorem}
\newtheorem{problem}{Problem}[]

%\counterwithin*{problem}{section}



%%%%%%%%%%%%%%%%%%%%%%%%%%%%Jenny's code%%%%%%%%%%%%%%%%%%%%

%%% Solution environment
%\newenvironment{solution}{
%\ifhandout\setbox0\vbox\bgroup\else
%\begin{trivlist}\item[\hskip \labelsep\small\itshape\bfseries Solution\hspace{2ex}]
%\par\noindent\upshape\small
%\fi}
%{\ifhandout\egroup\else
%\end{trivlist}
%\fi}
%
%
%%% instructorIntro environment
%\ifhandout
%\newenvironment{instructorIntro}[1][false]%
%{%
%\def\givenatend{\boolean{#1}}\ifthenelse{\boolean{#1}}{\begin{trivlist}\item}{\setbox0\vbox\bgroup}{}
%}
%{%
%\ifthenelse{\givenatend}{\end{trivlist}}{\egroup}{}
%}
%\else
%\newenvironment{instructorIntro}[1][false]%
%{%
%  \ifthenelse{\boolean{#1}}{\begin{trivlist}\item[\hskip \labelsep\bfseries Instructor Notes:\hspace{2ex}]}
%{\begin{trivlist}\item[\hskip \labelsep\bfseries Instructor Notes:\hspace{2ex}]}
%{}
%}
%% %% line at the bottom} 
%{\end{trivlist}\par\addvspace{.5ex}\nobreak\noindent\hung} 
%\fi
%
%


\let\instructorNotes\relax
\let\endinstructorNotes\relax
%%% instructorNotes environment
\ifhandout
\newenvironment{instructorNotes}[1][false]%
{%
\def\givenatend{\boolean{#1}}\ifthenelse{\boolean{#1}}{\begin{trivlist}\item}{\setbox0\vbox\bgroup}{}
}
{%
\ifthenelse{\givenatend}{\end{trivlist}}{\egroup}{}
}
\else
\newenvironment{instructorNotes}[1][false]%
{%
  \ifthenelse{\boolean{#1}}{\begin{trivlist}\item[\hskip \labelsep\bfseries {\Large Instructor Notes: \\} \hspace{\textwidth} ]}
{\begin{trivlist}\item[\hskip \labelsep\bfseries {\Large Instructor Notes: \\} \hspace{\textwidth} ]}
{}
}
{\end{trivlist}}
\fi


%% Suggested Timing
\newcommand{\timing}[1]{{\bf Suggested Timing: \hspace{2ex}} #1}




\hypersetup{
    colorlinks=true,       % false: boxed links; true: colored links
    linkcolor=blue,          % color of internal links (change box color with linkbordercolor)
    citecolor=green,        % color of links to bibliography
    filecolor=magenta,      % color of file links
    urlcolor=cyan           % color of external links
}
\title{Symmetry}

\begin{document}
\begin{abstract}

\end{abstract}
\maketitle

We say an object has symmetry if we apply some transformation to that object, and the result looks exactly the same as before we applied that transformation.  Let's think about what this definition really means.



\begin{problem}
This problem is brought to you by the letter H.
\begin{image}
\begin{tikzpicture}
\draw[ultra thick] (0,0)--(0,4);
\draw[ultra thick] (3,0)--(3,4);
\draw[ultra thick] (0,2)--(3,2);
\end{tikzpicture}
\end{image}
Find all of the symmetries of the letter H.  In other words: apply a transformation to the letter H.  Does it still look like H?  Is it still in the same place on the page?  Are both of these questions important?

\end{problem}

\begin{problem}
What are all of the symmetries of first letter of your first name?  (If your name begins with H, choose another random letter.)

\end{problem}



\begin{problem}
Draw a design that has a rotation symmetry but no reflection symmetry. Explain how you know your answer is correct.
\end{problem}


\begin{problem}
Draw a design that has both rotation symmetry and reflection symmetry. Compare and contrast your design to the one in the previous problem.
\end{problem}


\begin{problem}
What would it mean for a design to have translation symmetry? Draw examples to illustrate your thinking.
\end{problem}






\newpage
\begin{instructorNotes}

{\bf Main goal:} We introduce the notion of symmetry and produce examples.

{\bf Overall picture:} Tracing paper can be a huge help for this activity - encourage students to make two copies of their letters or designs to experiment with.

\begin{itemize}
	\item The main thing we want to accomplish here is practicing with our definitions of rotations, reflections, and translations. In discussion, be sure to have students specify their transformations exactly and review their definitions as needed.
	\item you should discuss that when specifying a symmetry, one should make sure to specify the transformation exactly.  For instance, with a reflection symmetry, students should give the line of reflection.  With a rotational symmetry, students should give the angle of rotation.
	\item With rotation symmetry, you want to bring up the idea of the order of the rotation, as in the text.
	\item In discussion, you want to work through at least four different letters including the H.
	\item Be sure to present several examples for the student-created designs.
	\item The idea of translation symmetry may have come up with some of the earlier problems, as students tested letters of the alphabet, but we definitely want to discuss it with the final problem. Hopefully a student will come up with the idea of the object extending infinitely, or perhaps living on a sphere so that it has no ends. Students can struggle with translation symmetry if they don't understand that the design must end up in the same location on the page. 
\end{itemize}


\timing{Give students about 5 minutes to think about the letter H and then discuss to make sure everyone is on the same page about the definitions of symmetry. Then give students about 10 minutes to work on the rest of the page and spend the remaining time in discussion using as many examples as possible.}
\end{instructorNotes}

\end{document}





