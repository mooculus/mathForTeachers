\documentclass[nooutcomes,noauthor, handout]{ximera}
%\documentclass[handout]{ximera}

\graphicspath{
  {./}
  {graphics/}
  {../graphics/}
}

\usepackage{chngcntr}

\let\question\relax
\let\endquestion\relax




\newtheoremstyle{SlantTheorem}{\topsep}{\fill}%%% space between body and thm
%\newtheoremstyle{SlantTheorem}{\topsep}{\topsep}%%% space between body and thm
 {\slshape}                      %%% Thm body font
 {}                              %%% Indent amount (empty = no indent)
 {\bfseries\sffamily}            %%% Thm head font
 {}                              %%% Punctuation after thm head
 {3ex}                           %%% Space after thm head
 {\thmname{#1}\thmnumber{ #2}\thmnote{ \bfseries(#3)}}%%% Thm head spec
\theoremstyle{SlantTheorem}
\newtheorem{question}{Question}
\counterwithin*{question}{section}



\let\instructorNotes\relax
\let\endinstructorNotes\relax
%%% instructorNotes environment
\ifhandout
\newenvironment{instructorNotes}[1][false]%
{%
\def\givenatend{\boolean{#1}}\ifthenelse{\boolean{#1}}{\begin{trivlist}\item}{\setbox0\vbox\bgroup}{}
}
{%
\ifthenelse{\givenatend}{\end{trivlist}}{\egroup}{}
}
\else
\newenvironment{instructorNotes}[1][false]%
{%
  \ifthenelse{\boolean{#1}}{\begin{trivlist}\item[\hskip \labelsep\bfseries {\Large Instructor Notes: \\} \hspace{\textwidth} ]}
{\begin{trivlist}\item[\hskip \labelsep\bfseries {\Large Instructor Notes: \\} \hspace{\textwidth} ]}
{}
}
{\end{trivlist}}
\fi


%% Suggested Timing
\newcommand{\timing}[1]{{\bf Suggested Timing: \hspace{2ex}} #1}

\title{Same or Different?}
\begin{document}
\begin{abstract}\end{abstract}
\maketitle



\begin{problem}
Some of our special quadrilaterals are special cases of others.  In each of the following, if it is possible, fill in the blanks to make a true statement of:

\begin{center} \underline{\hspace{1in}} is a special case of \underline{\hspace{1in}}. \end{center}

For example, if you are given ``food'' and ``ice cream'', ``Ice cream is a special case of food" is a true statement (but the reverse is not).  Explain your reasoning on why your statement is true but the reverse is not.  Try to identify where you're using the definition of the quadrilateral, and where you're using a property of the quadrilateral.
\begin{enumerate}
\item  Animal, puppy
\item  Square, rectangle
\item Rhombus, square
\item  Rhombus, parallelogram
\item  Kite, rhombus
\end{enumerate}

For parts \ref{FabFourPartf} through \ref{FabFourParth}, fill in the blanks for the following statement:

\begin{center} \underline{\hspace{0.5in}} is a special case of \underline{\hspace{0.5in}}, which is a special case of \underline{\hspace{0.5in}}. \end{center}

\begin{enumerate}
\setcounter{enumi}{5}
    \item Dessert, food, ice cream \label{FabFourPartf}
    \item  Rhombus, kite, square
    \item Rectangle, parallelogram, quadrilateral \label{FabFourParth}
\end{enumerate}

\begin{solution}
\begin{enumerate}
\item Puppy is a special case of animal.
\item Square is a special case of rectangle.
\item Square is a special case of rhombus.
\item Rhombus is a special case of parallelogram.
\item Rhombus is a special case of kite.
\item Ice cream is a special case of dessert, which is a special case of food.
\item Square is a special case of rhombus, which is a special case of kite.
\item Rectangle is a special case of parallelogram, which is a special case of quadrilateral.
\end{enumerate}
\end{solution}
\end{problem}

For the problems below, you might need the triangle congruence theorems we've discussed. Remember that you don't need to memorize them; feel free to look them up!

\begin{problem}

\begin{enumerate}
	\item Prove your statement about squares and rectangles.
	\item Prove your statement about rhombuses and parallelograms.
	\item If you get done early, pick another statement and prove it!
\end{enumerate}
\end{problem}

\newpage

In ``Property Rights'', we discussed some properties that quadrilaterals can have.
\begin{enumerate}[label= P{\arabic*}.]
    \item Opposite sides are equal.
    \item Opposite angles are equal.
    \item Opposite sides are parallel. %new instead of symmetry
    \item The diagonals are equal.
    \item The diagonals bisect each other.
   % \item The diagonals meet at right angles.
    \item The diagonals bisect the angles at each end.
    %\item A diagonal is a line of symmetry.
    
\end{enumerate} 

\begin{problem}
\begin{enumerate}
	\item Prove that P1 is true for every parallelogram.
	\item Prove that P6 is true for every rhombus.
	\item If you get done early, choose some other shapes and properties and continue proving!
\end{enumerate}
\end{problem}


\begin{problem}
How could you use a Venn diagram to represent the things we've been working on in this activity?
\end{problem}





%\begin{problem}  
%Of squares, rhombuses, rectangles, parallelograms, kites, trapezoids, and isosceles trapezoids, 
%\begin{enumerate}
%    \item   Which are special cases of rectangles?
%    \item Which are special cases of parallelograms? 
%\end{enumerate}
%
%Try to identify where you're using the definition of the quadrilateral, and where you're using a property of the quadrilateral.
%
%\begin{solution}
%\begin{enumerate}
%\item Squares (and technically rectangles are a special case of themselves, if you want!)
%\item Squares, Rhombuses, rectangles (and technically parallelograms!)
%\end{enumerate}
%\end{solution}
%
%\end{problem}




%\begin{problem} If a line is a bisector of a segment, is that line also a perpendicular bisector of that segment?  If a line is perpendicular to a segment, is that line also a perpendicular bisector of that segment? 
%
%
%\begin{solution}
%The answer to both questions is no.
%\end{solution}
%
%\end{problem}

\pagebreak

\begin{instructorNotes}
{\bf Main goal: } The goal in this activity is to showcase how the congruence theorems can help us prove some of the properties we've observed.


{\bf Overall picture:} Once the relationships between categories are settled, we can talk about properties that automatically extend from one category to another (and properties that don't automatically extend!)  This is a tough activity, since it deals with some logic which is difficult for many students.

For the ``special case'' problem:
\begin{enumerate}
\item Some students think that they have to keep the words in order and answer whether the statement is true or false.  You can watch out for this as you walk around to the groups.
\item We don't want to force the ``special case'' language for these: this can be very confusing for some students. Encourage similar language like ``all -- are --'' or ``some -- are --''. Another way to think about it is ``every -- has to be a --''. Whatever language the students can make sense of (and is correct) is good!
\item Students sometimes don't naturally think about cutting a quadrilateral into two triangles with a diagonal. Watch for this as you walk around to talk with the groups.
\item It's easy for students to get confused between a rectangle and a rhombus.  You might have them draw some more examples on their own if they're facing this confusion.
\item In your discussion, have the students justify their work as best as they can. Drawing pictures and discussing definitions and properties should be the main reasoning of the day. Continue to distinguish between definitions and properties.
\item You might discuss the statement: ``All (blank) are (blank)'' as well as ``Some (blank) are (blank)'' and compare and contrast the results.
\item The very last question give you a chance to return to Venn Diagrams during this discussion if you have time.
\end{enumerate}

When discussing whether various properties hold for all such shapes, be sure to go slowly and carefully. Students should feel free to use the Parallel Postulate or its converse, or any of the triangle congruence theorems. Continue to emphasize that the point here is not memorizing the criteria but showing how this idea commonly discussed in high school helps us to continue to explore conjectures that we have made about relationships between shapes or the way we have categorized shapes.

%The final problem is included because it is one of the most common incorrect statements that students make. Again, have students draw pictures to justify their results.


{\bf Good language:} Continue to have students identify when they are using a definition and when they are using a property.

{\bf Suggested timing:} Give students 10-15 minutes to discuss all of the problems in their groups, and then use the remaining time for presentations and discussion.


\end{instructorNotes}

\end{document}