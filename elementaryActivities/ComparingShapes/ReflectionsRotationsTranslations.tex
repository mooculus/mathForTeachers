%\documentclass[handout]{ximera}
\documentclass[nooutcomes,noauthor, handout]{ximera}
\usepackage{gensymb}
\usepackage{tabularx}
\usepackage{mdframed}
\usepackage{pdfpages}
%\usepackage{chngcntr}

\let\problem\relax
\let\endproblem\relax

\newcommand{\property}[2]{#1#2}




\newtheoremstyle{SlantTheorem}{\topsep}{\fill}%%% space between body and thm
 {\slshape}                      %%% Thm body font
 {}                              %%% Indent amount (empty = no indent)
 {\bfseries\sffamily}            %%% Thm head font
 {}                              %%% Punctuation after thm head
 {3ex}                           %%% Space after thm head
 {\thmname{#1}\thmnumber{ #2}\thmnote{ \bfseries(#3)}} %%% Thm head spec
\theoremstyle{SlantTheorem}
\newtheorem{problem}{Problem}[]

%\counterwithin*{problem}{section}



%%%%%%%%%%%%%%%%%%%%%%%%%%%%Jenny's code%%%%%%%%%%%%%%%%%%%%

%%% Solution environment
%\newenvironment{solution}{
%\ifhandout\setbox0\vbox\bgroup\else
%\begin{trivlist}\item[\hskip \labelsep\small\itshape\bfseries Solution\hspace{2ex}]
%\par\noindent\upshape\small
%\fi}
%{\ifhandout\egroup\else
%\end{trivlist}
%\fi}
%
%
%%% instructorIntro environment
%\ifhandout
%\newenvironment{instructorIntro}[1][false]%
%{%
%\def\givenatend{\boolean{#1}}\ifthenelse{\boolean{#1}}{\begin{trivlist}\item}{\setbox0\vbox\bgroup}{}
%}
%{%
%\ifthenelse{\givenatend}{\end{trivlist}}{\egroup}{}
%}
%\else
%\newenvironment{instructorIntro}[1][false]%
%{%
%  \ifthenelse{\boolean{#1}}{\begin{trivlist}\item[\hskip \labelsep\bfseries Instructor Notes:\hspace{2ex}]}
%{\begin{trivlist}\item[\hskip \labelsep\bfseries Instructor Notes:\hspace{2ex}]}
%{}
%}
%% %% line at the bottom} 
%{\end{trivlist}\par\addvspace{.5ex}\nobreak\noindent\hung} 
%\fi
%
%


\let\instructorNotes\relax
\let\endinstructorNotes\relax
%%% instructorNotes environment
\ifhandout
\newenvironment{instructorNotes}[1][false]%
{%
\def\givenatend{\boolean{#1}}\ifthenelse{\boolean{#1}}{\begin{trivlist}\item}{\setbox0\vbox\bgroup}{}
}
{%
\ifthenelse{\givenatend}{\end{trivlist}}{\egroup}{}
}
\else
\newenvironment{instructorNotes}[1][false]%
{%
  \ifthenelse{\boolean{#1}}{\begin{trivlist}\item[\hskip \labelsep\bfseries {\Large Instructor Notes: \\} \hspace{\textwidth} ]}
{\begin{trivlist}\item[\hskip \labelsep\bfseries {\Large Instructor Notes: \\} \hspace{\textwidth} ]}
{}
}
{\end{trivlist}}
\fi


%% Suggested Timing
\newcommand{\timing}[1]{{\bf Suggested Timing: \hspace{2ex}} #1}




\hypersetup{
    colorlinks=true,       % false: boxed links; true: colored links
    linkcolor=blue,          % color of internal links (change box color with linkbordercolor)
    citecolor=green,        % color of links to bibliography
    filecolor=magenta,      % color of file links
    urlcolor=cyan           % color of external links
}

\title{Reflections, Rotations, Translations}
\begin{document}
\begin{abstract}
\end{abstract}
\maketitle


\begin{problem}
Point $P$ is the reflection of point $Q$ across some line.  Using your tools as needed, draw this line of reflection.  Explain how you knew where to draw the line and why you are sure that reflecting $Q$ across this line will put $Q$ on $P$.\\
\vskip 2in
\begin{image}
\begin{tikzpicture}
\draw[fill=black] (0,0) circle (2pt) node[below] {$P$};
\draw[fill=black] (6, -4) circle (2pt) node[below] {$Q$};
\end{tikzpicture}
\end{image}


\end{problem}
\vfill
\begin{problem}
Summarize the characteristics of a reflection.  What information must be specified in order to fully describe this kind of motion?  What relationship does the figure have to the line of reflection?
\end{problem}
\newpage



\begin{problem}
Using your tools as needed, rotate the point $P$ $110$ degrees clockwise about point $Q$.  Explain how you drew the new position of point $P$ and why you are sure that this is the correct location of the rotated point $P$.\\
\vskip 2in
\begin{image}
\begin{tikzpicture}
\draw[fill=black] (0,0) circle (2pt) node[below] {$P$};
\draw[fill=black] (6, -4) circle (2pt) node[below] {$Q$};
\end{tikzpicture}
\end{image}


\end{problem}
\vfill
\begin{problem}
Summarize the characteristics of a rotation.  What information must be specified in order to fully describe this kind of motion?  What relationship does the figure have to the point of rotation?
\end{problem}
\newpage


\begin{problem}
Use your tools as needed to translate the given point $P$ the distance and direction of the vector (arrow) shown.  Explain how you decided where to put the new point.\\
\vskip 1.5in

\begin{image}
\begin{tikzpicture}
\draw[fill=black] (0,-2) circle (2pt) node[below] {$P$};
\draw[thick, ->] (4, 2)-- (7, 3);
\end{tikzpicture}
\end{image}

\end{problem}
\vfill
\begin{problem}
Summarize the characteristics of a translation.  What information must be specified in order to fully describe this kind of motion?  What relationship does the figure have to its original location in the plane? To the vector describing the distance and direction of the slide?
\end{problem}
\newpage 



%second round of reflection practice
\begin{problem}
Use your ruler and possibly your protractor to reflect the given triangle across the given line. Explain how you drew the reflection of your triangle and why you are sure that it is correctly positioned relative to the line of reflection.\

\vskip 1in

\begin{image}
\begin{tikzpicture}

\draw[thick] (0, 4)--(10, -1);
\draw[thick] (2,6)--(8,1)--(2.5, -4)--(2,6);
\node[above left] at (2,6) {$A$};
\node[right] at (8,1) {$B$};
\node[below] at (2.5, -4) {$C$};
\end{tikzpicture}
\end{image}



\vfill
For a challenge, try these other cases.
\begin{enumerate}
    \item The triangle lies entirely on one side of the line.
    \item One side of the triangle lies on the line.
    %\item The triangle lies across the line.
\end{enumerate}

\end{problem}

\newpage
%second round of rotation practice

\begin{problem}
Using your tools as needed, rotate the given triangle 50 degrees clockwise about point $Q$.  Explain how you drew the new position of the triangle and why you are sure that this is the correct location of the rotated triangle.\\

\begin{image}
\begin{tikzpicture}
\draw[thick] (0,0)--(0,4)--(5, 1)--(0,0);
\draw[fill=black] (-1, -2) circle (2pt) node[below] {$Q$};
\node[left] at (0,0) {$A$};
\node[above] at (0,4) {$B$};
\node[right] at (5,1) {$C$};
\end{tikzpicture}
\end{image}
\vfill
\end{problem}


\newpage

\begin{problem}
Use your ruler to translate the given triangle the distance and direction given by the vector shown.  Explain how you decided where to put the new triangle.\\
\vskip 1.5in
\begin{image}
\begin{tikzpicture}
\draw[thick] (0,0)--(0,4)--(5, 1)--(0,0);
\draw[thick, ->] (7, 5)--(4, 3);
\node[left] at (0,0) {$A$};
\node[above] at (0,4) {$B$};
\node[right] at (5,1) {$C$};
\end{tikzpicture}
\end{image}
\vfill
\end{problem}

%\newpage

%\begin{problem}
%\begin{enumerate}
%    \item Summarize the characteristics of a reflection.  What information must be specified in order to fully describe this kind of motion?  What relationship does the figure have to the line of reflection?
 %   \item Summarize the characteristics of a rotation.  What information must be specified in order to fully describe this kind of motion?  What relationship does the figure have to the point of rotation?
%    \item Summarize the characteristics of a translation.  What information must be specified in order to fully describe this kind of motion?  What relationship does the figure have to its original location in the plane? To the vector describing the distance and direction of the slide?
%\end{enumerate}


%\end{problem}

\newpage


\begin{instructorNotes}
{\bf Main goal:} We define and construct transformations.

{\bf Overall picture:} 

This activity begins our study of reflections, rotations, and translations.  We will accomplish these transformations using careful drawing, compass, ruler, protractor, and reasoning.  These are not the traditional ``mathematical constructions" which are done without any measurements, unless the pertinent constructions have been completed earlier in the course.


This activity generally requires two class periods.  The purpose of the activity is to  introduce the ideas of rigid motions (reflection, translation, rotation) and have students work with these ideas using compass and protractor to develop their ability to visualize these transformations.  We can follow this activity by doing the same transformations on grid paper on a homework assignment.  Our intent here is to focus students on the meaning of each type of transformation without the distraction that grid paper often provides, since many students have learned to do transformations merely by transforming coordinates according to a rule.

In the past, students have found these transformation activities quite interesting and challenging at the right level.

You should begin with a discussion of what kind of moves in the plane (and then in space) would maintain the shape and size of an object. The students will need to at least be shown pictorial definitions of each transformation before they begin this exercise.  Along the way, students are asked to summarize the characteristics of reflections, rotations, and translations, so it's not necessary to give a full verbal statement of the definitions before you begin. However, you should wrap up the two days of activities with careful statements of the definitions.  

\begin{enumerate}
\item For reflections, the students will likely be confused that they must make corresponding points the same distance from the line in a perpendicular manner. They are permitted to use their protractors to measure angles - we have not usually had time to work through the construction of a right angle.  Sometimes they do not realize that the distance from a point to a line changes according to the direction.\\ 

As the students work through this activity, be sure that they are actually constructing their solutions, not just guessing.  They should need protractor and ruler!

Check that students can justify that their technique will always work to get the desired result.   Student explanations should include that, under a reflection, all points maintain their perpendicular distance from the line of reflection.

\item Students often find rotations difficult to visualize - particularly once we work in the coordinate plane.

Check that students can justify that their technique will always work to get the desired result.   Student explanations should include that a rotation follows a circular path - all points will stay the same distance from the given center.

\item Sometimes students have trouble reflecting or rotating a whole object.  You might encourage them to identify and move key points (e.g., vertices), connecting them up later.  Usually, someone in each group will recognize this and help those who are confused without intervention by the instructor.  It's nice to then ask the students to justify why this method works!

\item {\bf Possible extensions:} If time (now or in a later activity), might expand these motions to what they would be like in 3-D (e.g., a plane of reflection, what rotating by an angle would be defined as, etc.)

You also might give the result of a motion and ask for the line of reflection or center of rotation, etc.


\end{enumerate}

{\bf Good language:} Any opportunity we have to re-emphasize definitions (for example, of a circle) is a good opportunity to take!

\timing{Your introduction may take about 10 minutes.  Afterwards, students will work on these problems for as long as they are given.   You could give the students about 10 minutes in small groups to work on the first two problems, and then about 20 minutes in discussion. Repeat this process as necessary!  If you need to shorten the discussion time, you can discuss only problems \#1-6 which will give one example of each, and leave \#7-9 for additional practice or review.}

%Repeat this timing with problems 3 and 4, and then finally problems 5 - 7 with a longer discussion at the end including problem 7.  

%If you need to shorten the discussion time, you can discuss only problems \#2, 4, 5, and 7, and leave the other problems for review.

\end{instructorNotes}



\end{document}