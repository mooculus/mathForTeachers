%\documentclass{ximera}
\documentclass[nooutcomes,noauthor]{ximera}
\usepackage{gensymb}
\usepackage{tabularx}
\usepackage{mdframed}
\usepackage{pdfpages}
%\usepackage{chngcntr}

\let\problem\relax
\let\endproblem\relax

\newcommand{\property}[2]{#1#2}




\newtheoremstyle{SlantTheorem}{\topsep}{\fill}%%% space between body and thm
 {\slshape}                      %%% Thm body font
 {}                              %%% Indent amount (empty = no indent)
 {\bfseries\sffamily}            %%% Thm head font
 {}                              %%% Punctuation after thm head
 {3ex}                           %%% Space after thm head
 {\thmname{#1}\thmnumber{ #2}\thmnote{ \bfseries(#3)}} %%% Thm head spec
\theoremstyle{SlantTheorem}
\newtheorem{problem}{Problem}[]

%\counterwithin*{problem}{section}



%%%%%%%%%%%%%%%%%%%%%%%%%%%%Jenny's code%%%%%%%%%%%%%%%%%%%%

%%% Solution environment
%\newenvironment{solution}{
%\ifhandout\setbox0\vbox\bgroup\else
%\begin{trivlist}\item[\hskip \labelsep\small\itshape\bfseries Solution\hspace{2ex}]
%\par\noindent\upshape\small
%\fi}
%{\ifhandout\egroup\else
%\end{trivlist}
%\fi}
%
%
%%% instructorIntro environment
%\ifhandout
%\newenvironment{instructorIntro}[1][false]%
%{%
%\def\givenatend{\boolean{#1}}\ifthenelse{\boolean{#1}}{\begin{trivlist}\item}{\setbox0\vbox\bgroup}{}
%}
%{%
%\ifthenelse{\givenatend}{\end{trivlist}}{\egroup}{}
%}
%\else
%\newenvironment{instructorIntro}[1][false]%
%{%
%  \ifthenelse{\boolean{#1}}{\begin{trivlist}\item[\hskip \labelsep\bfseries Instructor Notes:\hspace{2ex}]}
%{\begin{trivlist}\item[\hskip \labelsep\bfseries Instructor Notes:\hspace{2ex}]}
%{}
%}
%% %% line at the bottom} 
%{\end{trivlist}\par\addvspace{.5ex}\nobreak\noindent\hung} 
%\fi
%
%


\let\instructorNotes\relax
\let\endinstructorNotes\relax
%%% instructorNotes environment
\ifhandout
\newenvironment{instructorNotes}[1][false]%
{%
\def\givenatend{\boolean{#1}}\ifthenelse{\boolean{#1}}{\begin{trivlist}\item}{\setbox0\vbox\bgroup}{}
}
{%
\ifthenelse{\givenatend}{\end{trivlist}}{\egroup}{}
}
\else
\newenvironment{instructorNotes}[1][false]%
{%
  \ifthenelse{\boolean{#1}}{\begin{trivlist}\item[\hskip \labelsep\bfseries {\Large Instructor Notes: \\} \hspace{\textwidth} ]}
{\begin{trivlist}\item[\hskip \labelsep\bfseries {\Large Instructor Notes: \\} \hspace{\textwidth} ]}
{}
}
{\end{trivlist}}
\fi


%% Suggested Timing
\newcommand{\timing}[1]{{\bf Suggested Timing: \hspace{2ex}} #1}




\hypersetup{
    colorlinks=true,       % false: boxed links; true: colored links
    linkcolor=blue,          % color of internal links (change box color with linkbordercolor)
    citecolor=green,        % color of links to bibliography
    filecolor=magenta,      % color of file links
    urlcolor=cyan           % color of external links
}
\title{Solid shapes}

\begin{document}
\begin{abstract}
\end{abstract}

\maketitle

\begin{problem}
One mathematical task that very young children learn is to identify shapes around them. What kind of 3D shapes do we have in this classroom, and what are their names? Come up with at least $3$ examples.
\end{problem}

\begin{problem}
What kind of 3D shapes do you have at home that we did not find in the classroom? What are their names? Come up with at least $3$ examples.
\end{problem}

\begin{problem}
What 2D shapes would you need to build a right hexagonal prism? Draw these shapes on paper. If you have scissors and tape available, explain how to use your shapes to build the prism.
\end{problem}

\begin{problem}
A \dfn{net} for a solid shape is a connected pattern that we could cut out and fold in order to build that shape. Is the pattern you made in the previous problem a net for a hexagonal prism? If not, how can you adjust your drawing to make a net for a hexagonal prism?
\end{problem}

\begin{problem}
What 3D shape would be built by cutting out and folding a half circle so that the two halves of the diameter are taped together?

Before you cut the shape out and fold it, write your prediction for the solid here.

\begin{image}
\begin{tikzpicture}
	\draw[thick] (-2,0) arc [start angle=180, end angle=360, radius=2];
	\draw[fill=black] (0,0) circle (2pt);
	\draw[thick] (-2,0)--(2,0);
\end{tikzpicture}
\end{image}

Was your prediction correct?
\end{problem}
\newpage

\begin{problem}
What 3D shape would be built by cutting out and folding a square with triangles attached to each side?

Before you cut the shape out and fold it, write your prediction for the solid here.

\begin{image}
\begin{tikzpicture}
	\draw[thick] (0,0) rectangle (3,3);
	\draw[thick] (0, 3)--(1.5, 7)--(3,3)--(7, 1.5)--(3,0)--(1.5, -4)--(0,0)--(-4, 1.5)--(0,3);
\end{tikzpicture}
\end{image}


Was your prediction correct?
\end{problem}


\begin{problem}
Draw two different nets that would each build a cube. If you glue these two cubes together, do you still have a cube? Explain your thinking.
\end{problem}


\begin{problem}
Draw a net that would build a cylinder. (Make two copies of it if you are cutting it out to experiment!) If you glue these two cylinders together, do you still have a cylinder? Explain your thinking.
\end{problem}


\begin{problem}
Explain how you could glue together two 3D shapes that have names to build a 3D shape that does not have a name. Explain your thinking.
\end{problem}





\newpage

\begin{instructorNotes} 


{\bf Main goal:} We visualize 3D shapes and introduce vocabulary for working with 3D shapes.

{\bf Overall picture:} 


\begin{itemize}
	\item Visualization of 3D objects is difficult for many of our students, so this activity attacks visualization from several angles.
	\item The first two questions are intended to draw out the vocabulary and definitions we use with 3D solids. Be sure to introduce prisms (right and oblique), pyramids (right and oblique), cylinders, cones, and spheres. You should also include some discussion on the vertices, edges, and faces of each. You don't need to prove any results about these; our main goal in this activity is to define the terminology we want to use moving forward.
	\item The rest of the problems deal with building patterns and nets for 3D shapes. This is useful for calculating surface area later in the course, but also for visualizing the shapes being created. You can always add extra examples here if you have extra time.
	\item When students draw a pattern for a hexagonal prism, the difference between a net and a pattern is whether the shapes are all connected. You might encourage them to cut out the shapes needed and then rearrange them into a net.
	\item When students start with a given net, be sure they predict what shape will be made before they cut. You can ask students to discuss how they visualized what shape would be made or how their ideas changed as they cut and formed the shapes.
	\item When discussing whether two cubes together make a cube, we want to practice using our definition here. When we join two cylinders we have a more interesting case!
	\item The final problem reminds students that as with 2D shapes, we can draw or build 3D shapes that don't have a specific name. This doesn't make them less interesting!
	\item For future exercises, we would like students to both be able to recognize what shape is made by a net as well as be able to draw their own nets for some 3D shapes.

\end{itemize}







{\bf Good language:} Encourage students to be precise with their language. In English we often use ``square'' when we mean ``rectangle'' or ``rectangle'' when we mean ``rectangular prism''. In math we want to be more careful!



{\bf Suggested timing:} Give students about 5 minutes to work on the first two problems and then spend 15 minutes in discussion and introducing definitions. Next, give students about 10 minutes to work through as much of the rest of the activity as they can, and then use the remaining time to discuss. Wrap up by reminding students to take care with their language and that we would like them to be able to both use and draw nets and patterns for 3D objects.




\end{instructorNotes}



\end{document}