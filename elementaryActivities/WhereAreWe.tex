\documentclass{ximera}
\usepackage{gensymb}
\usepackage{tabularx}
\usepackage{mdframed}
\usepackage{pdfpages}
%\usepackage{chngcntr}

\let\problem\relax
\let\endproblem\relax

\newcommand{\property}[2]{#1#2}




\newtheoremstyle{SlantTheorem}{\topsep}{\fill}%%% space between body and thm
 {\slshape}                      %%% Thm body font
 {}                              %%% Indent amount (empty = no indent)
 {\bfseries\sffamily}            %%% Thm head font
 {}                              %%% Punctuation after thm head
 {3ex}                           %%% Space after thm head
 {\thmname{#1}\thmnumber{ #2}\thmnote{ \bfseries(#3)}} %%% Thm head spec
\theoremstyle{SlantTheorem}
\newtheorem{problem}{Problem}[]

%\counterwithin*{problem}{section}



%%%%%%%%%%%%%%%%%%%%%%%%%%%%Jenny's code%%%%%%%%%%%%%%%%%%%%

%%% Solution environment
%\newenvironment{solution}{
%\ifhandout\setbox0\vbox\bgroup\else
%\begin{trivlist}\item[\hskip \labelsep\small\itshape\bfseries Solution\hspace{2ex}]
%\par\noindent\upshape\small
%\fi}
%{\ifhandout\egroup\else
%\end{trivlist}
%\fi}
%
%
%%% instructorIntro environment
%\ifhandout
%\newenvironment{instructorIntro}[1][false]%
%{%
%\def\givenatend{\boolean{#1}}\ifthenelse{\boolean{#1}}{\begin{trivlist}\item}{\setbox0\vbox\bgroup}{}
%}
%{%
%\ifthenelse{\givenatend}{\end{trivlist}}{\egroup}{}
%}
%\else
%\newenvironment{instructorIntro}[1][false]%
%{%
%  \ifthenelse{\boolean{#1}}{\begin{trivlist}\item[\hskip \labelsep\bfseries Instructor Notes:\hspace{2ex}]}
%{\begin{trivlist}\item[\hskip \labelsep\bfseries Instructor Notes:\hspace{2ex}]}
%{}
%}
%% %% line at the bottom} 
%{\end{trivlist}\par\addvspace{.5ex}\nobreak\noindent\hung} 
%\fi
%
%


\let\instructorNotes\relax
\let\endinstructorNotes\relax
%%% instructorNotes environment
\ifhandout
\newenvironment{instructorNotes}[1][false]%
{%
\def\givenatend{\boolean{#1}}\ifthenelse{\boolean{#1}}{\begin{trivlist}\item}{\setbox0\vbox\bgroup}{}
}
{%
\ifthenelse{\givenatend}{\end{trivlist}}{\egroup}{}
}
\else
\newenvironment{instructorNotes}[1][false]%
{%
  \ifthenelse{\boolean{#1}}{\begin{trivlist}\item[\hskip \labelsep\bfseries {\Large Instructor Notes: \\} \hspace{\textwidth} ]}
{\begin{trivlist}\item[\hskip \labelsep\bfseries {\Large Instructor Notes: \\} \hspace{\textwidth} ]}
{}
}
{\end{trivlist}}
\fi


%% Suggested Timing
\newcommand{\timing}[1]{{\bf Suggested Timing: \hspace{2ex}} #1}




\hypersetup{
    colorlinks=true,       % false: boxed links; true: colored links
    linkcolor=blue,          % color of internal links (change box color with linkbordercolor)
    citecolor=green,        % color of links to bibliography
    filecolor=magenta,      % color of file links
    urlcolor=cyan           % color of external links
}

\title{Where Are We?}
\author{Vic Ferdinand, Betsy McNeal, Jenny Sheldon}
\begin{document}
\begin{abstract}\end{abstract}
\maketitle
\begin{instructorIntro}
Goals for this activity:
\begin{enumerate}
\item Students think about our usual coordinate system from a new perspective, i.e., beginning from the need to locate something in the plane or in space.
\item  Our intent in this activity is to get students to think about the essential features of any location system: reference point, direction, unit of distance.
\item We want students to describe the locations in this activity's fourth problem using explanations of the meaning of coordinates, rather than through memorized formulae.
\end{enumerate}

Teaching Notes:
\begin{enumerate}
\item  You probably will only need to have students do four or five of the parts in the ``Describe and draw all locations'' problem.  The remaining questions could be for homework.
\item  The last question is intended to connect this activity's ideas to the work done in I Walk the Line.  Now we do want students to think about the independent and dependent variables, plot these to provide a second representation of the charts done earlier, and finally find an equation for each situation as a third representation of the same relationship.
\end{enumerate}
\end{instructorIntro}


\begin{problem}
Assume you can only stand at some place along one wall of a room.  How could you describe where you are using numbers?  What is the least information that would communicate your location?

\end{problem}

\begin{problem}
Now assume you can stand anywhere on the floor in the room.  How could you describe where you are using numbers?  What is the least information that would communicate your location?


\end{problem}



\begin{problem}
Describe the location of at least 5 points on the floor using your system from number 2.
\end{problem}


\begin{problem}
Assume you have a grid laid out in the Cartesian system.  Describe and draw ALL locations where:
\begin{enumerate}
\item The y-coordinate is 5.
\item The x-coordinate is -3.
\item The sum of the coordinates is 10.
\item The product of the coordinates is 24.
\item The y-coordinate is the same as the x-coordinate.
\item The y-coordinate is double the x-coordinate.
\item The y-coordinate is 3 less than double the x-coordinate.
\item The x-coordinate is the square of the y-coordinate.
\end{enumerate}

\end{problem}

\begin{problem}
Now try to describe any point in the air in the room in as many ways as possible.
\end{problem}

\begin{problem}
Let's return to the earlier activity, I Walk the Line.
    \begin{enumerate}
        \item Graph your solutions to at least two of the problems.  Does it make sense to ``connect the dots'' on each graph?
        \item Write an equation that describes each story in terms of the coordinates that you used.
        \item How are your equations and graphs alike? How are they different?
    \end{enumerate}    

\end{problem}






\end{document}