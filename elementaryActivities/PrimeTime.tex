\documentclass{ximera}


\graphicspath{
  {./}
  {graphics/}
  {../graphics/}
}

\usepackage{chngcntr}

\let\question\relax
\let\endquestion\relax




\newtheoremstyle{SlantTheorem}{\topsep}{\fill}%%% space between body and thm
%\newtheoremstyle{SlantTheorem}{\topsep}{\topsep}%%% space between body and thm
 {\slshape}                      %%% Thm body font
 {}                              %%% Indent amount (empty = no indent)
 {\bfseries\sffamily}            %%% Thm head font
 {}                              %%% Punctuation after thm head
 {3ex}                           %%% Space after thm head
 {\thmname{#1}\thmnumber{ #2}\thmnote{ \bfseries(#3)}}%%% Thm head spec
\theoremstyle{SlantTheorem}
\newtheorem{question}{Question}
\counterwithin*{question}{section}



\let\instructorNotes\relax
\let\endinstructorNotes\relax
%%% instructorNotes environment
\ifhandout
\newenvironment{instructorNotes}[1][false]%
{%
\def\givenatend{\boolean{#1}}\ifthenelse{\boolean{#1}}{\begin{trivlist}\item}{\setbox0\vbox\bgroup}{}
}
{%
\ifthenelse{\givenatend}{\end{trivlist}}{\egroup}{}
}
\else
\newenvironment{instructorNotes}[1][false]%
{%
  \ifthenelse{\boolean{#1}}{\begin{trivlist}\item[\hskip \labelsep\bfseries {\Large Instructor Notes: \\} \hspace{\textwidth} ]}
{\begin{trivlist}\item[\hskip \labelsep\bfseries {\Large Instructor Notes: \\} \hspace{\textwidth} ]}
{}
}
{\end{trivlist}}
\fi


%% Suggested Timing
\newcommand{\timing}[1]{{\bf Suggested Timing: \hspace{2ex}} #1}


\title{Prime Time}
\author{Vic Ferdinand, Betsy McNeal, Jenny Sheldon}

\begin{document}
\begin{abstract} \end{abstract}
\maketitle



\begin{problem}
 Write the prime factorization of 60.  That is, start by writing any factorization of 60 into two factors.  Then split up the factors continually until it is impossible to break the factors down further.  Then write 60 as the product of those ``indivisible" numbers.
\end{problem} 
\begin{problem}
 Do this again, but start with a different two number factorization of 60.  What do you notice?
\end{problem} 
\begin{problem}
 List all the factors of 60.
\end{problem} 
\begin{problem}\label{PrimeTime4}
 Write the prime factorization of each of the factors you found in $\#$3.  Be sure to write the powers of the prime numbers involved.
\end{problem} 
\begin{problem}
 From your work in Problem \ref{PrimeTime4}, what must be true about the prime factorization of a number in order for it to be a ``factor of 60"? 
\end{problem} 


\newpage
\begin{instructorNotes}
This activity introduces the content of the Unique Factorization Theorem (whether you choose to use the terminology or not). 

\begin{itemize}
	\item We usually talk about the prime factorization as the ``molecular structure'' that is unique to that number only.
	\item You might discuss (at the end) why 1 should not be considered a prime.  We sometimes keep this question open for a period of time; this is a convenient opportunity to bring up the concept.
	\item This activity also leads nicely into a discussion about using prime factorizations to determine whether one number is a factor of another. The activity ``Statements About Factors'' can be used for this purpose.
\end{itemize}

\timing{This activity usually takes us about half a class period.  The students can work through most of the problems in about 10 minutes, and then we take 20 minutes for presentation and discussion.}
\end{instructorNotes}


\end{document}