\documentclass{ximera}

\usepackage{gensymb}
\usepackage{tabularx}
\usepackage{mdframed}
\usepackage{pdfpages}
%\usepackage{chngcntr}

\let\problem\relax
\let\endproblem\relax

\newcommand{\property}[2]{#1#2}




\newtheoremstyle{SlantTheorem}{\topsep}{\fill}%%% space between body and thm
 {\slshape}                      %%% Thm body font
 {}                              %%% Indent amount (empty = no indent)
 {\bfseries\sffamily}            %%% Thm head font
 {}                              %%% Punctuation after thm head
 {3ex}                           %%% Space after thm head
 {\thmname{#1}\thmnumber{ #2}\thmnote{ \bfseries(#3)}} %%% Thm head spec
\theoremstyle{SlantTheorem}
\newtheorem{problem}{Problem}[]

%\counterwithin*{problem}{section}



%%%%%%%%%%%%%%%%%%%%%%%%%%%%Jenny's code%%%%%%%%%%%%%%%%%%%%

%%% Solution environment
%\newenvironment{solution}{
%\ifhandout\setbox0\vbox\bgroup\else
%\begin{trivlist}\item[\hskip \labelsep\small\itshape\bfseries Solution\hspace{2ex}]
%\par\noindent\upshape\small
%\fi}
%{\ifhandout\egroup\else
%\end{trivlist}
%\fi}
%
%
%%% instructorIntro environment
%\ifhandout
%\newenvironment{instructorIntro}[1][false]%
%{%
%\def\givenatend{\boolean{#1}}\ifthenelse{\boolean{#1}}{\begin{trivlist}\item}{\setbox0\vbox\bgroup}{}
%}
%{%
%\ifthenelse{\givenatend}{\end{trivlist}}{\egroup}{}
%}
%\else
%\newenvironment{instructorIntro}[1][false]%
%{%
%  \ifthenelse{\boolean{#1}}{\begin{trivlist}\item[\hskip \labelsep\bfseries Instructor Notes:\hspace{2ex}]}
%{\begin{trivlist}\item[\hskip \labelsep\bfseries Instructor Notes:\hspace{2ex}]}
%{}
%}
%% %% line at the bottom} 
%{\end{trivlist}\par\addvspace{.5ex}\nobreak\noindent\hung} 
%\fi
%
%


\let\instructorNotes\relax
\let\endinstructorNotes\relax
%%% instructorNotes environment
\ifhandout
\newenvironment{instructorNotes}[1][false]%
{%
\def\givenatend{\boolean{#1}}\ifthenelse{\boolean{#1}}{\begin{trivlist}\item}{\setbox0\vbox\bgroup}{}
}
{%
\ifthenelse{\givenatend}{\end{trivlist}}{\egroup}{}
}
\else
\newenvironment{instructorNotes}[1][false]%
{%
  \ifthenelse{\boolean{#1}}{\begin{trivlist}\item[\hskip \labelsep\bfseries {\Large Instructor Notes: \\} \hspace{\textwidth} ]}
{\begin{trivlist}\item[\hskip \labelsep\bfseries {\Large Instructor Notes: \\} \hspace{\textwidth} ]}
{}
}
{\end{trivlist}}
\fi


%% Suggested Timing
\newcommand{\timing}[1]{{\bf Suggested Timing: \hspace{2ex}} #1}




\hypersetup{
    colorlinks=true,       % false: boxed links; true: colored links
    linkcolor=blue,          % color of internal links (change box color with linkbordercolor)
    citecolor=green,        % color of links to bibliography
    filecolor=magenta,      % color of file links
    urlcolor=cyan           % color of external links
}


\title{Divisibility With Remainders}
\author{Vic Ferdinand, Betsy McNeal, Jenny Sheldon}

\begin{document}
\begin{abstract} \end{abstract}
\maketitle



\begin{problem} Let's use $A$ to represent a number that has a remainder of 1 when you divide by $5$, and let's use $B$ to represent a number that has a remainder of 4 when you divide by $5$.
\begin{enumerate}
\item Write down some possibilities for the numbers $A$ and $B$.
\item  What is the remainder when you calculate $(A + B) \div 5$?  Start by working with your examples, then make an argument about what should be true for any $A$ and $B$.
\item  What is the remainder when you calculate $(AB) \div 5$?  Start by working with your examples, then make an argument about what should be true for any $A$ and $B$.
\end{enumerate}
\end{problem}

\begin{problem}
Now let's use $C$ to represent a number that has a remainder of 2 when you divide by $5$.
\begin{enumerate}
\item  Write down some possibilities for the numbers $C$.
\item   What is the remainder when you calculate $(C + B) \div 5$?  
\item  What is the remainder when you calculate $(CB) \div 5$?
\end{enumerate}
\end{problem}

\begin{problem}
  Is there a general rule, here?  If so, what is it?  Describe it as accurately as you can, and explain why your rule should always hold.
\end{problem}


\newpage
\begin{instructorNotes}
The idea of this activity is to engage students in discussion conclusions that can be made about divisibility relationships.  We typically use this activity after discussing divisibility tests for the numbers $2$ and $3$, and so this activity can feel like a more general case of divisibility testing.  This activity is also nice for exploring conjectures about some numbers or a class of numbers versus conjectures about all numbers.  We guide students through the ``observe, conjecture, justify'' process.

\begin{itemize}
    \item For each of the problems, students should be encouraged to justify their answers in more than one way.  It is good to have an algebraic method as well as a pictorial method demonstrated for the class.
    \item Students should begin by working with concrete examples, but then extend to a general argument.
\end{itemize}

{\bf Suggested Timing:} This activity takes us between a half-class and a full class, depending on the time we have.  For a half-class, we just focus on the first problem, giving students 5-10 minutes to think about the problem, and then 15-20 minutes to discuss.  In that case, we do the generalization as a whole class discussion.  For a full class, we repeat this again with the second case, and then give students time to try to generalize on their own before we discuss.

\end{instructorNotes}

 



\end{document}