%\documentclass{ximera}
\documentclass[nooutcomes,noauthor]{ximera}
\usepackage{gensymb}
\usepackage{tabularx}
\usepackage{mdframed}
\usepackage{pdfpages}
%\usepackage{chngcntr}

\let\problem\relax
\let\endproblem\relax

\newcommand{\property}[2]{#1#2}




\newtheoremstyle{SlantTheorem}{\topsep}{\fill}%%% space between body and thm
 {\slshape}                      %%% Thm body font
 {}                              %%% Indent amount (empty = no indent)
 {\bfseries\sffamily}            %%% Thm head font
 {}                              %%% Punctuation after thm head
 {3ex}                           %%% Space after thm head
 {\thmname{#1}\thmnumber{ #2}\thmnote{ \bfseries(#3)}} %%% Thm head spec
\theoremstyle{SlantTheorem}
\newtheorem{problem}{Problem}[]

%\counterwithin*{problem}{section}



%%%%%%%%%%%%%%%%%%%%%%%%%%%%Jenny's code%%%%%%%%%%%%%%%%%%%%

%%% Solution environment
%\newenvironment{solution}{
%\ifhandout\setbox0\vbox\bgroup\else
%\begin{trivlist}\item[\hskip \labelsep\small\itshape\bfseries Solution\hspace{2ex}]
%\par\noindent\upshape\small
%\fi}
%{\ifhandout\egroup\else
%\end{trivlist}
%\fi}
%
%
%%% instructorIntro environment
%\ifhandout
%\newenvironment{instructorIntro}[1][false]%
%{%
%\def\givenatend{\boolean{#1}}\ifthenelse{\boolean{#1}}{\begin{trivlist}\item}{\setbox0\vbox\bgroup}{}
%}
%{%
%\ifthenelse{\givenatend}{\end{trivlist}}{\egroup}{}
%}
%\else
%\newenvironment{instructorIntro}[1][false]%
%{%
%  \ifthenelse{\boolean{#1}}{\begin{trivlist}\item[\hskip \labelsep\bfseries Instructor Notes:\hspace{2ex}]}
%{\begin{trivlist}\item[\hskip \labelsep\bfseries Instructor Notes:\hspace{2ex}]}
%{}
%}
%% %% line at the bottom} 
%{\end{trivlist}\par\addvspace{.5ex}\nobreak\noindent\hung} 
%\fi
%
%


\let\instructorNotes\relax
\let\endinstructorNotes\relax
%%% instructorNotes environment
\ifhandout
\newenvironment{instructorNotes}[1][false]%
{%
\def\givenatend{\boolean{#1}}\ifthenelse{\boolean{#1}}{\begin{trivlist}\item}{\setbox0\vbox\bgroup}{}
}
{%
\ifthenelse{\givenatend}{\end{trivlist}}{\egroup}{}
}
\else
\newenvironment{instructorNotes}[1][false]%
{%
  \ifthenelse{\boolean{#1}}{\begin{trivlist}\item[\hskip \labelsep\bfseries {\Large Instructor Notes: \\} \hspace{\textwidth} ]}
{\begin{trivlist}\item[\hskip \labelsep\bfseries {\Large Instructor Notes: \\} \hspace{\textwidth} ]}
{}
}
{\end{trivlist}}
\fi


%% Suggested Timing
\newcommand{\timing}[1]{{\bf Suggested Timing: \hspace{2ex}} #1}




\hypersetup{
    colorlinks=true,       % false: boxed links; true: colored links
    linkcolor=blue,          % color of internal links (change box color with linkbordercolor)
    citecolor=green,        % color of links to bibliography
    filecolor=magenta,      % color of file links
    urlcolor=cyan           % color of external links
}
\title{Partial Products}

\begin{document}
\begin{abstract}
\end{abstract}

\maketitle

Before beginning this activity, you should know how to do the steps of both the partial products algorithm and the standard algorithm for multiplication.

\begin{problem}
Below we have a $14$-by-$36$ array. Explain why this array can be used to model $14 \times 36$, and then explain why the array could also be used to model $36 \times 14$. What are you seeing as one group? One object?

\begin{center} \begin{tikzpicture}
\draw[step=0.1] (0,0) grid (3.6, 1.4);
\end{tikzpicture}\end{center}
\end{problem}


\begin{problem}
Use the partial products method to calculate $14 \times 36$. Clearly label your steps.
\end{problem}


\begin{problem}
Subdivide the $14$-by-$36$ array to show the same multiplication steps that you took in the partial products multiplication. For instance, you calculated $30 \times 4$ or $4 \times 30$ as part of the algorithm. Where can we see this step in the array? What are you seeing as one group? One object?

\begin{center} \begin{tikzpicture}
\draw[step=0.1] (0,0) grid (3.6, 1.4);
\end{tikzpicture}\end{center}
\end{problem}


\begin{problem}
Why does adding together the subdivisions in the array give you the correct answer to the multiplication problem $14 \times 36$? How was the meaning of addition important, and how did we use the meaning of multiplication?
\end{problem}



\begin{problem}
Why have we now justified the partial products algorithm for the example of $14 \times 36$?
\end{problem}


\begin{problem}
How are the steps in the partial products algorithm related to the properties of multiplication? Explain your thinking with equations and/or pictures.
\end{problem}



\begin{problem}
Use the standard algorithm to calculate $14 \times 36$. How are these steps related to the steps in the partial products algorithm?
\end{problem}



\begin{problem}
If you have extra time, repeat the steps above to justify both the partial products algorithm and the standard algorithm using the example of $143 \times 82$.
\end{problem}



\newpage

\begin{instructorNotes} 



{\bf Main goal:} We justify the partial products algorithm for multiplication.


{\bf Overall picture:} In this activity, we will discuss both the partial products algorithm (see the textbook if you have never used this method) as well as the standard multiplication algorithm. Our goal in terms of assessment is to understand the partial products algorithm via the distributive property, but we would also like our students to see how the two algorithms are related.


These problems follow an overall pattern of (1) using a diagram to decompose the multiplication problem, (2) using the distributive property to decompose the multiplication, and (3) connecting the algorithm to the decomposition. 
\begin{itemize}
	\item With the diagram, remember to connect counting the squares in the array to the meaning of multiplication. Be sure to have students identify what they are thinking of as the groups and the objects in this situation!
	\item Connect to the meaning of addition here, emphasizing the idea that we are combining all of the results of our small multiplications (thus we should add them together).
	\item Connect the multiplications for the subdivisions also to the meaning of multiplication by having students identify the groups and objects per group they are seeing for these multiplication problems as well. Then, emphasize that we have counted all of the dots or squares in the diagram -- we have not left anything out by combining these smaller pieces!
	\item When discussing the distributive property, you can review the distributive property itself (especially since we are using the extended distributive property or FOIL).
	\item Connect each step in the distributive property to the diagram, and emphasize again why these multiplications should produce the same answer as the original multiplication (we are doing the same action -- counting the total number of dots -- and using all of the same diagram to do so).
	\item Connect each step in the distributive property to each step in the partial products method. Again, emphasize why we are getting the same answer as previously!
	\item Discuss why addition is the appropriate operation to use with the partial products. When we are multiplying overall, why does it make sense to add things up? Connect this idea to the distributive property and the diagram again.
	\item When you connect to the standard algorithm, the idea is to put together some of the ``partial products'' to see the usual steps. This is outlined in the text.
	\item Work through the standard algorithm slowly, explaining each step as you go. Why do we multiply the ones place? What is happening when we ``carry''? Why does the carried number get added to the result of the multiplication, instead of multiplied again? How is place value incorporated into our calculations? Why do we put a zero at the beginning of the second line (or just move over one place)? Why do we add to get the result?
\end{itemize}


{\bf Good language:}  Help students connect the meaning of multiplication (counting the total number of dots) with the steps we use to calculate the answer. Students are often not thinking about the two ideas as significantly different. We would actually like to make these connections at each step, not just with the overall multiplication and the overall answer. Each step in the partial products method should be justified as helping us move towards counting the entire number of dots in the picture!


{\bf Suggested timing:} Begin by having a student demonstrate the standard multiplication algorithm on the board. The student doesn't need to do any explaining at this stage. Ask if anyone knows a different multiplication algorithm. At this point you may get the partial products method or the lattice method. If no one demonstrates the partial products method, please do this yourself at this point.

Give students about 15-20 minutes to work through the problems about $14 \times 36$, then have students present their results and discuss. Wrap up by highlighting how each step is important for justifying the algorithm. If you have extra time, move on to the final problem. If you don't get a chance to connect to the standard algorithm, that's also okay -- the partial products method is our main focus here.




\end{instructorNotes}



\end{document}