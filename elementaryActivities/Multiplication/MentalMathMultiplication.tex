\documentclass{ximera}
\usepackage{gensymb}
\usepackage{tabularx}
\usepackage{mdframed}
\usepackage{pdfpages}
%\usepackage{chngcntr}

\let\problem\relax
\let\endproblem\relax

\newcommand{\property}[2]{#1#2}




\newtheoremstyle{SlantTheorem}{\topsep}{\fill}%%% space between body and thm
 {\slshape}                      %%% Thm body font
 {}                              %%% Indent amount (empty = no indent)
 {\bfseries\sffamily}            %%% Thm head font
 {}                              %%% Punctuation after thm head
 {3ex}                           %%% Space after thm head
 {\thmname{#1}\thmnumber{ #2}\thmnote{ \bfseries(#3)}} %%% Thm head spec
\theoremstyle{SlantTheorem}
\newtheorem{problem}{Problem}[]

%\counterwithin*{problem}{section}



%%%%%%%%%%%%%%%%%%%%%%%%%%%%Jenny's code%%%%%%%%%%%%%%%%%%%%

%%% Solution environment
%\newenvironment{solution}{
%\ifhandout\setbox0\vbox\bgroup\else
%\begin{trivlist}\item[\hskip \labelsep\small\itshape\bfseries Solution\hspace{2ex}]
%\par\noindent\upshape\small
%\fi}
%{\ifhandout\egroup\else
%\end{trivlist}
%\fi}
%
%
%%% instructorIntro environment
%\ifhandout
%\newenvironment{instructorIntro}[1][false]%
%{%
%\def\givenatend{\boolean{#1}}\ifthenelse{\boolean{#1}}{\begin{trivlist}\item}{\setbox0\vbox\bgroup}{}
%}
%{%
%\ifthenelse{\givenatend}{\end{trivlist}}{\egroup}{}
%}
%\else
%\newenvironment{instructorIntro}[1][false]%
%{%
%  \ifthenelse{\boolean{#1}}{\begin{trivlist}\item[\hskip \labelsep\bfseries Instructor Notes:\hspace{2ex}]}
%{\begin{trivlist}\item[\hskip \labelsep\bfseries Instructor Notes:\hspace{2ex}]}
%{}
%}
%% %% line at the bottom} 
%{\end{trivlist}\par\addvspace{.5ex}\nobreak\noindent\hung} 
%\fi
%
%


\let\instructorNotes\relax
\let\endinstructorNotes\relax
%%% instructorNotes environment
\ifhandout
\newenvironment{instructorNotes}[1][false]%
{%
\def\givenatend{\boolean{#1}}\ifthenelse{\boolean{#1}}{\begin{trivlist}\item}{\setbox0\vbox\bgroup}{}
}
{%
\ifthenelse{\givenatend}{\end{trivlist}}{\egroup}{}
}
\else
\newenvironment{instructorNotes}[1][false]%
{%
  \ifthenelse{\boolean{#1}}{\begin{trivlist}\item[\hskip \labelsep\bfseries {\Large Instructor Notes: \\} \hspace{\textwidth} ]}
{\begin{trivlist}\item[\hskip \labelsep\bfseries {\Large Instructor Notes: \\} \hspace{\textwidth} ]}
{}
}
{\end{trivlist}}
\fi


%% Suggested Timing
\newcommand{\timing}[1]{{\bf Suggested Timing: \hspace{2ex}} #1}




\hypersetup{
    colorlinks=true,       % false: boxed links; true: colored links
    linkcolor=blue,          % color of internal links (change box color with linkbordercolor)
    citecolor=green,        % color of links to bibliography
    filecolor=magenta,      % color of file links
    urlcolor=cyan           % color of external links
}

\title{Mental Math with Multiplication}

\begin{document}
\begin{abstract} We practice multiplying without pencil and paper. \end{abstract}
\maketitle


\begin{problem}
For each of the following, perform the multiplication without writing anything down, and without using a calculator or other tool. You can write down your answer so that you don't forget it!

\begin{enumerate}
\item $4 \times 99$
\item $51 \times 4$
\item $12 \times 125$

\end{enumerate}

\end{problem}


\begin{problem}
As a group, illustrate your solutions to each problem using a picture. Your pictures can include objects, number lines, or whatever else makes sense to you! Try using different types of pictures for each problem.
\end{problem}


\begin{problem}
As a group, illustrate your solutions to each problem using equations. Be sure to place the equals sign only between things which are actually equal! Did you use any properties of arithmetic in your equations? If so, what were they, and where did you use them? Be as specific as you can.

\end{problem}


\begin{problem}
Cade solves $12 \times 125$ in the following way. 

\emph{I'm going to split $12$ into $3 \times 4$ and $125$ into $120 + 5$. Now that I see the $120$, I'm going to split that into $3 \times 40$. So we take $3 \times 4$ and multiply that by $3 \times 40 + 5$. Now $4$ times $5$ is $20$, and multiplying that by $3$ gives me $60$. If I do $3 \times 3 \times 40$ I can start with $9 \times 40$ which is $360$. Now multiplying that by $4$ I get $4$ times $300$ which is $1200$ and $4 \times 60$ which is $240$ so all together that's $1440$. Adding the $60$ gives me $1500$ and that's my answer.}

Why does Cade's strategy give us the correct answer? Draw pictures or write equations to help you understand this strategy. Did Cade use any properties of arithmetic in this strategy?  Explain your thinking.
\end{problem}



\newpage
\begin{instructorNotes}

{\bf Main Goal:} Our main goal here is to continue to develop flexibility with mental calculations. We have several secondary goals, including writing equations to describe a mental method and recognizing the properties of arithmetic.


{\bf Overall Picture:}

\begin{itemize}
\item You may want to work through the first two problems as a class, one at a time. Write one of the problems on the board, give students a few minutes to think, and then ask for 3-5 volunteers to talk through their solutions. Consider asking students to give a thumbs-up (visible only to the instructor) when they are ready, much like with Number Talks. See the previous mental math activity for a link to an example.
\item Have students present as many types of drawings and equations as you see while you walk around the room. The equations need not be a string of equations, but do need to represent the student's thinking. If the student does write a string of equations, watch for the type where the equals sign means ``operate'' rather than ``equal'', for instance $2 + 7 = 9 + 1 = 10$ when solving $2 + 7 + 1$.
\item Since this is the second time we have looked at equations corresponding to mental methods, it's appropriate to give a bit more emphasis to the strings of equations rather than collections of equations. This also gives us the opportunity to emphasize that the property is what's happening ``at the equals sign'' rather than elsewhere in the equations.
\item If students struggle with making a drawing, encourage them to move to equations and come back if they have time. Drawings are much more difficult here than with addition and subtraction.
\item Use this opportunity to continue discussing the properties of arithmetic, including the distributive property. Point out the properties in both pictures (if applicable) and equations.

\end{itemize}



%{\bf Good Language:}



{\bf Suggested Timing:} This activity should take most of the class period. If you are working together for the first part, spend about 20 minutes here, then give students about 10 minutes to draw and write equations, and have groups present for the final 20 minutes. If the students are working in groups for the first part, encourage silence for about five minutes, then give groups about 15 minutes to discuss and progress through the activity. Then, use the last 25 minutes in whole-class discussion. If you have extra time, you can have students come up with their own mental math problems, or go back and discuss some more examples of each of the properties.

\end{instructorNotes}


\end{document}