\documentclass{ximera}

\usepackage{gensymb}
\usepackage{tabularx}
\usepackage{mdframed}
\usepackage{pdfpages}
%\usepackage{chngcntr}

\let\problem\relax
\let\endproblem\relax

\newcommand{\property}[2]{#1#2}




\newtheoremstyle{SlantTheorem}{\topsep}{\fill}%%% space between body and thm
 {\slshape}                      %%% Thm body font
 {}                              %%% Indent amount (empty = no indent)
 {\bfseries\sffamily}            %%% Thm head font
 {}                              %%% Punctuation after thm head
 {3ex}                           %%% Space after thm head
 {\thmname{#1}\thmnumber{ #2}\thmnote{ \bfseries(#3)}} %%% Thm head spec
\theoremstyle{SlantTheorem}
\newtheorem{problem}{Problem}[]

%\counterwithin*{problem}{section}



%%%%%%%%%%%%%%%%%%%%%%%%%%%%Jenny's code%%%%%%%%%%%%%%%%%%%%

%%% Solution environment
%\newenvironment{solution}{
%\ifhandout\setbox0\vbox\bgroup\else
%\begin{trivlist}\item[\hskip \labelsep\small\itshape\bfseries Solution\hspace{2ex}]
%\par\noindent\upshape\small
%\fi}
%{\ifhandout\egroup\else
%\end{trivlist}
%\fi}
%
%
%%% instructorIntro environment
%\ifhandout
%\newenvironment{instructorIntro}[1][false]%
%{%
%\def\givenatend{\boolean{#1}}\ifthenelse{\boolean{#1}}{\begin{trivlist}\item}{\setbox0\vbox\bgroup}{}
%}
%{%
%\ifthenelse{\givenatend}{\end{trivlist}}{\egroup}{}
%}
%\else
%\newenvironment{instructorIntro}[1][false]%
%{%
%  \ifthenelse{\boolean{#1}}{\begin{trivlist}\item[\hskip \labelsep\bfseries Instructor Notes:\hspace{2ex}]}
%{\begin{trivlist}\item[\hskip \labelsep\bfseries Instructor Notes:\hspace{2ex}]}
%{}
%}
%% %% line at the bottom} 
%{\end{trivlist}\par\addvspace{.5ex}\nobreak\noindent\hung} 
%\fi
%
%


\let\instructorNotes\relax
\let\endinstructorNotes\relax
%%% instructorNotes environment
\ifhandout
\newenvironment{instructorNotes}[1][false]%
{%
\def\givenatend{\boolean{#1}}\ifthenelse{\boolean{#1}}{\begin{trivlist}\item}{\setbox0\vbox\bgroup}{}
}
{%
\ifthenelse{\givenatend}{\end{trivlist}}{\egroup}{}
}
\else
\newenvironment{instructorNotes}[1][false]%
{%
  \ifthenelse{\boolean{#1}}{\begin{trivlist}\item[\hskip \labelsep\bfseries {\Large Instructor Notes: \\} \hspace{\textwidth} ]}
{\begin{trivlist}\item[\hskip \labelsep\bfseries {\Large Instructor Notes: \\} \hspace{\textwidth} ]}
{}
}
{\end{trivlist}}
\fi


%% Suggested Timing
\newcommand{\timing}[1]{{\bf Suggested Timing: \hspace{2ex}} #1}




\hypersetup{
    colorlinks=true,       % false: boxed links; true: colored links
    linkcolor=blue,          % color of internal links (change box color with linkbordercolor)
    citecolor=green,        % color of links to bibliography
    filecolor=magenta,      % color of file links
    urlcolor=cyan           % color of external links
}

\title{Decimal Multiplication}
\author{Vic Ferdinand, Betsy McNeal, Jenny Sheldon}

\begin{document}

\begin{abstract}  \end{abstract}
\maketitle


\begin{problem}
We would like to begin to work with multiplication in the case where the number of groups and the number of objects per group are not whole numbers.

\begin{enumerate}
\item Give three examples of situations where it makes sense to have a decimal number of groups. For example, think about stories for $12.6 \times 4$.
\item Give three examples of situations where it makes sense to have a decimal number of objects in each group. For example, think about stories for $16 \times 9.35$.
\item Give one example of a situation where it does not make sense to have a decimal number for either the number of groups or the number of objects per group. Why is this situation different than the ones you had above?
\end{enumerate}
\end{problem}


\begin{problem}
Write a word problem for $3.4 \times 2.1$ which does not involve money. Use our definition of multiplication to explain why it is a problem for that expression.
\end{problem}


\begin{problem}
Draw a picture to help you solve your story in the previous question. How does the picture show what one group looks like? How does the picture show one object? How does the picture show the total number of objects, and what is this total? Explain your thinking in each case.
\end{problem}

\newpage

\begin{problem}
Using the largest square in the image, find the area of the 3.4-unit by 2.1-unit rectangle {\bf without} multiplying. Explain your work.

\begin{image}
	\begin{tikzpicture}
		\draw[thick] (0,0) grid (3.4, -2.1);
		\foreach \x in {3.1, 3.2, 3.3, 3.4} \draw[thick] (\x, 0)--(\x, -2.1);
		\foreach \y in {-2.1} \draw[thick] (0, \y)--(3.4, \y);
	\end{tikzpicture}
\end{image}
\end{problem}

\begin{problem}
Why is finding the area of the rectangle above related to solving the multiplication problem $3.4 \times 2.1$? Use our groups-and-objects meaning of multiplication to explain. Where are the groups? What are the objects?
\end{problem}

\begin{problem}
Using the smallest square in the image, find the area of the 3.4-unit by 2.1-unit rectangle {\bf without} multiplying. Explain your work.

\begin{image}
	\begin{tikzpicture}
		\draw[thick] (0,0) grid (3.4, -2.1);
		\foreach \x in {3.1, 3.2, 3.3, 3.4} \draw[thick] (\x, 0)--(\x, -2.1);
		\foreach \y in {-2.1} \draw[thick] (0, \y)--(3.4, \y);
	\end{tikzpicture}
\end{image}
\end{problem}

\begin{problem}
What does the previous question tell us about the relationship between $3.4 \times 2.1$ and $34 \times 21$? Be sure to use the groups-and-objects meaning of multiplication as well as bundling to explain your thinking.
\end{problem}


\newpage

\begin{instructorNotes}

{\bf Main Goal:} Students encounter partial groups and partial objects in multiplication.

{\bf Overall Picture:}

This activity helps students to expand their ideas about multiplication to decimal numbers, and works as a first introduction to multiplying with partial numbers. We really want to focus here on what it looks like to make part of a group or to use part of an object. If we get to the second part of the activity, we also begin to think about how the bundling system affects multiplication.


\begin{itemize}
	\item The first problem should encourage students to think about when it makes sense to use partial groups and when it does not. This could be a complex discussion. For instance: it makes sense to think about $8.3$ baskets in the sense that we have 8 full baskets, and the 9th basket is only $0.3$ full. But since we physically have $9$ baskets in this case, we want to speak about this quantity carefully. Another good contrast is to think about people versus servings. While we can distribute enough for $6.2$ servings, we probably don't want $6.2$ people. However, we can sometimes say ``$6.2$ people'' when we mean $6.2$ servings. Encourage students to be creative with their answers here, and share as many contexts with the class as you can.
	\item For the second problem, we would like to see students extend the same meaning of multiplication now to decimal numbers. The prohibition on contexts involving money is just to help students be creative with their ideas. Typically, situations involving money (I buy 3.4 pounds of rice for \$2.10 per pound) are easy to come up with, and students need practice broadening their perspective. Be sure to point out in discussion that we still have groups and objects here!
	\item The third problem encourages students to model their multiplication in some way. Again, encourage creativity. If some groups are completely stuck, you can encourage them to move on to the next problem with the area model and see if they can make sense of how this picture is modeling $3.4 \times 2.1$.
	\item The area model problem can be difficult for some students. You may need to point out that their job is to count the number of big squares total, not the number of tiny squares. Students will have to recognize that each long rectangle is a tenth of the full unit, and each tiny square is a hundredth of the full unit. Then, they can count the rectangles and tiny squares to find out what fraction of a full unit they have. Counting should be the emphasis, here, much like our work with bundling!
	\item When the students relate their counting to the figure, they should see three full groups with 2.1 squares per group, and then four tenths of another group. Encourage students to circle their full group! It may also be helpful to draw a full group and then cut it into ten equal pieces, so that we can see what a tenth of a group looks like. (If it's easier for students to see two full groups and a tenth of another group with each full group containing $3.4$ blocks, this is also fine.)
	\item When we consider $3.4 \times 2.1$ and $34 \times 21$, we are building understanding about the usual algorithm for multiplying decimals -- first multiply the whole numbers, then add in the correct number of decimal points. But we would like to see this from a bundling perspective. To change our picture into one for $34 \times 21$, we will need to produce 23 groups instead of $3.4$, so each group should be cut into 10 pieces. Similarly, we need to have 21 objects per group, so we'll need to also cut into 10 pieces here. Overall now, our entire unit has been subdivided into 100 pieces, so we will have 100 times as many units as we did before. Because our decimal system lets us cut things into 10 without changing the actual digits involved, we'll have the same digits as before. We just have to work out how the sizes are related. We aren't expecting the students to understand the full depth of this argument, but we would like them to understanding the cutting-into-ten relationship between the two pictures. The activity question asks the students to count the number of small squares, so we are looking for them to see this bundling idea using the small blocks.
	%\item The last question about estimation is a bonus if you have time! The goal is to see that our answer should be close to 4 rather than 40 or 400. Combining this problem with the previous one where we think about cutting into ten should help build this intuition about how our bundling system is helping with this problem.
\end{itemize}

{\bf Good Language:} Be on the lookout for how students are describing their groups and objects with this multiplication! We would like to see one row as one group, and one big square (square unit) as one object. The rows are organizing our square units for us. Many students like to say that the rows are the groups and the columns are the objects, but we aren't counting columns here!

{\bf Suggested Timing:} This activity should take a whole class period. Give students about $5$ minutes to come up with story situations, and then have groups present. Next, give students 8-12 minutes to think about problems 2 and 3, then have students present their work at the board and discuss. Then give students another 5-10 minutes to think about the remaining problems, and use the rest of the time for discussion. The first three problems contain the main idea here, so if you don't get to the last three problems we can still move on to other activities.


\end{instructorNotes}




\end{document}