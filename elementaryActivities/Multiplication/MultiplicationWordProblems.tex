\documentclass{ximera}


\graphicspath{
  {./}
  {graphics/}
  {../graphics/}
}

\usepackage{chngcntr}

\let\question\relax
\let\endquestion\relax




\newtheoremstyle{SlantTheorem}{\topsep}{\fill}%%% space between body and thm
%\newtheoremstyle{SlantTheorem}{\topsep}{\topsep}%%% space between body and thm
 {\slshape}                      %%% Thm body font
 {}                              %%% Indent amount (empty = no indent)
 {\bfseries\sffamily}            %%% Thm head font
 {}                              %%% Punctuation after thm head
 {3ex}                           %%% Space after thm head
 {\thmname{#1}\thmnumber{ #2}\thmnote{ \bfseries(#3)}}%%% Thm head spec
\theoremstyle{SlantTheorem}
\newtheorem{question}{Question}
\counterwithin*{question}{section}



\let\instructorNotes\relax
\let\endinstructorNotes\relax
%%% instructorNotes environment
\ifhandout
\newenvironment{instructorNotes}[1][false]%
{%
\def\givenatend{\boolean{#1}}\ifthenelse{\boolean{#1}}{\begin{trivlist}\item}{\setbox0\vbox\bgroup}{}
}
{%
\ifthenelse{\givenatend}{\end{trivlist}}{\egroup}{}
}
\else
\newenvironment{instructorNotes}[1][false]%
{%
  \ifthenelse{\boolean{#1}}{\begin{trivlist}\item[\hskip \labelsep\bfseries {\Large Instructor Notes: \\} \hspace{\textwidth} ]}
{\begin{trivlist}\item[\hskip \labelsep\bfseries {\Large Instructor Notes: \\} \hspace{\textwidth} ]}
{}
}
{\end{trivlist}}
\fi


%% Suggested Timing
\newcommand{\timing}[1]{{\bf Suggested Timing: \hspace{2ex}} #1}


\title{Multiplication Word Problems}
\author{Vic Ferdinand, Betsy McNeal, Jenny Sheldon}

\begin{document}
\begin{abstract} \end{abstract}
\maketitle


\begin{problem}
Describe similarities and differences between the following story problems by considering how a student who has never dealt with multiplication before might attempt to solve it.
\begin{enumerate}
\item A box holds 5 candy bars.  If there are 7 such boxes, how many total candy bars are there?

\vfill

\item A shelf has cans arranged in a rectangular shape which is 5 cans wide and 7 cans deep. How many cans are on the shelf?
%A rectangle has length 7 inches and width 5 inches.  What is the area (measure of space inside) of the rectangle?


\vfill


\item John has 7 shirts and 5 pants.  How many different outfits can he wear?


\vfill

\item John has 7 times as many candies as Cindy does.  If Cindy has 5 candies, how many candies does John have?

\vfill

\end{enumerate}
\end{problem}

\newpage
\begin{instructorNotes}
In our course, this is the first activity we use to introduce multiplication as an operation.  As with the activity ``Adding It All Up'' (introducing addition and subtraction), this activity has students think about how young children who have not yet learned multiplication would solve the problems using objects or drawings.  Each problem addresses a different model of multiplication.

In our course, we follow extensive work on the meaning of numbers (whole numbers, fractions, decimals) without reference to operations with activities about the operations.  This is the second activity we do to introduce an operation (the first is ``Adding It All Up'', which introduces addition and subtraction). Dealing with operations on numbers, rather than only with the meanings, notation, and comparison of quantities is a big shift in our course.

Again, we try to focus students on the structure of the problems, spending significant time discussing what the problems have in common with one another to develop and identify the structure of multiplication.

\begin{itemize}
	\item After eliciting these interpretations and drawings, discuss what is the same in all cases (usually students come up with either ``repeated addition'' or groups).  This is a good time to introduce the definition of $A\times B$ , where B is a rate:  Number of objects in each group.  This ``per'' quantity (a unit that contains two units at once.  Also a concept that will pervade the curriculum from here on out - e.g., with ratios, slopes, and, later, with derivatives) helps explain why multiplication can be a difficult leap for children
	\item Next, we go back through each multiplicative situation, we ask ``What are the groups? What are the objects in each group?''  This will help everyone connect the more disparate models such as the Cartesian product with the array and repeated addition. 
	\item Students should pay attention to the units of the three numbers in each problem. We ask: What are the units of the groups?  Of the objects in those groups?  What should be the units on the answer?  Students are often confused with this, particularly with area. Here are some notes for when such a problem arises.  When describing area, the objects in each group should usually be ``one-by-one squares'' (usually square inches), and the groups are either ``rows" or ``columns", and the answer will be the total number of objects (``one-by-one squares''). It does not make sense to discuss ``rows of columns", as this would yield an answer in ``row-columns". This clarification will set some nice ground work for their geometry course.
	\item We contrast this work with units with the case of addition and subtraction, where the units of all of the objects were the same.  We have found that the more work we do with units here, the easier the division structure is for students later.
\end{itemize}




{\bf Suggested Timing:} We spend about 15-20 minutes in groups, 15 minutes in presentations, and 15-20 minutes pulling all of this together.
\end{instructorNotes}

\end{document}
