%\documentclass{ximera}
\documentclass[nooutcomes,noauthor]{ximera}
\usepackage{gensymb}
\usepackage{tabularx}
\usepackage{mdframed}
\usepackage{pdfpages}
%\usepackage{chngcntr}

\let\problem\relax
\let\endproblem\relax

\newcommand{\property}[2]{#1#2}




\newtheoremstyle{SlantTheorem}{\topsep}{\fill}%%% space between body and thm
 {\slshape}                      %%% Thm body font
 {}                              %%% Indent amount (empty = no indent)
 {\bfseries\sffamily}            %%% Thm head font
 {}                              %%% Punctuation after thm head
 {3ex}                           %%% Space after thm head
 {\thmname{#1}\thmnumber{ #2}\thmnote{ \bfseries(#3)}} %%% Thm head spec
\theoremstyle{SlantTheorem}
\newtheorem{problem}{Problem}[]

%\counterwithin*{problem}{section}



%%%%%%%%%%%%%%%%%%%%%%%%%%%%Jenny's code%%%%%%%%%%%%%%%%%%%%

%%% Solution environment
%\newenvironment{solution}{
%\ifhandout\setbox0\vbox\bgroup\else
%\begin{trivlist}\item[\hskip \labelsep\small\itshape\bfseries Solution\hspace{2ex}]
%\par\noindent\upshape\small
%\fi}
%{\ifhandout\egroup\else
%\end{trivlist}
%\fi}
%
%
%%% instructorIntro environment
%\ifhandout
%\newenvironment{instructorIntro}[1][false]%
%{%
%\def\givenatend{\boolean{#1}}\ifthenelse{\boolean{#1}}{\begin{trivlist}\item}{\setbox0\vbox\bgroup}{}
%}
%{%
%\ifthenelse{\givenatend}{\end{trivlist}}{\egroup}{}
%}
%\else
%\newenvironment{instructorIntro}[1][false]%
%{%
%  \ifthenelse{\boolean{#1}}{\begin{trivlist}\item[\hskip \labelsep\bfseries Instructor Notes:\hspace{2ex}]}
%{\begin{trivlist}\item[\hskip \labelsep\bfseries Instructor Notes:\hspace{2ex}]}
%{}
%}
%% %% line at the bottom} 
%{\end{trivlist}\par\addvspace{.5ex}\nobreak\noindent\hung} 
%\fi
%
%


\let\instructorNotes\relax
\let\endinstructorNotes\relax
%%% instructorNotes environment
\ifhandout
\newenvironment{instructorNotes}[1][false]%
{%
\def\givenatend{\boolean{#1}}\ifthenelse{\boolean{#1}}{\begin{trivlist}\item}{\setbox0\vbox\bgroup}{}
}
{%
\ifthenelse{\givenatend}{\end{trivlist}}{\egroup}{}
}
\else
\newenvironment{instructorNotes}[1][false]%
{%
  \ifthenelse{\boolean{#1}}{\begin{trivlist}\item[\hskip \labelsep\bfseries {\Large Instructor Notes: \\} \hspace{\textwidth} ]}
{\begin{trivlist}\item[\hskip \labelsep\bfseries {\Large Instructor Notes: \\} \hspace{\textwidth} ]}
{}
}
{\end{trivlist}}
\fi


%% Suggested Timing
\newcommand{\timing}[1]{{\bf Suggested Timing: \hspace{2ex}} #1}




\hypersetup{
    colorlinks=true,       % false: boxed links; true: colored links
    linkcolor=blue,          % color of internal links (change box color with linkbordercolor)
    citecolor=green,        % color of links to bibliography
    filecolor=magenta,      % color of file links
    urlcolor=cyan           % color of external links
}
\title{The Inchworm}

\begin{document}
\begin{abstract}
\end{abstract}

\maketitle


\begin{problem}
An inchworm is crawling along at a steady speed of $\frac{3}{8}$ of a foot per minute. How far will the inchworm crawl in $\frac{4}{5}$ of a minute?

\begin{enumerate}
	\item What multiplication problem is this, and why? Use groups and objects per group to explain.
	\item Solve the original problem using a picture. Use our meaning of fractions to explain what you drew and why you drew it.
	\item How can we see that the denominator is the product of the original denominators? Use the meaning of the denominator and the meaning of multiplication to explain. What is one group? One object?
	\item How can we see that the numerator is the product of the original numerators? Use the meaning of the numerator and the meaning of multiplication to explain. What is one group? One object?
\end{enumerate}
\end{problem}



\begin{problem}
Write your own multiplication story for $\frac{8}{5} \times \frac{2}{3}$.
\begin{enumerate}
	\item Explain how you know your story is correct. What is one group? One object?
	\item Solve the original problem using a picture. Use our meaning of fractions to explain what you drew and why you drew it.
	\item How can we see that the denominator is the product of the original denominators? Use the meaning of the denominator and the meaning of multiplication to explain. What is one group? One object?
	\item How can we see that the numerator is the product of the original numerators? Use the meaning of the numerator and the meaning of multiplication to explain. What is one group? One object?
\end{enumerate}
\end{problem}




\begin{problem}
An inchworm crawled for $\frac{2}{7}$ of a minute before stopping to take a break. The inchworm is $\frac{5}{8}$ of an inch long. How many minutes will it take the inchworm to crawl twice its body length?

Explain why this is not a story problem for $\frac{2}{7} \times \frac{5}{8}$. 
\end{problem}




\newpage

\begin{instructorNotes} 



{\bf Main goal:} We practice identifying the meaning of multiplication with fraction problems.


{\bf Overall picture:} This activity can be in addition to our work with ``Multiplication Stories with Fractions'', or we may skip this activity if we are short on time or have covered what we need to elsewhere.

\begin{itemize}
	\item Generally, these are more direct questions that can be used to help students practice with fraction multiplication and also consider why we are multiplying the numerators and denominators. See the ``Multiplication Stories with Fractions'' instructor's guide.
	\item Students typically struggle to write their own problems for the requested multiplication. Often they have the groups and objects swapped without meaning to do so. Encourage them to write down what they are seeing as one group and one object and match this information to our meaning of multiplication. (This may also help you decide whether the problem is correct or not!)
	\item The final problem reminds us of our work with determining whether we have a fraction addition problem or not; this problem should be challenging for students. It's also a bonus problem that you don't need to make time for. However, it again gives students a good opportunity to match groups and objects per group to our meaning of multiplication, though they will be unable to do so in this problem.
\end{itemize}


{\bf Good language:}  It continues to be helpful to ask students what is acting as one group, what is one object, and to clarify how many objects make up one full group. Since we are working with fractional groups, it can easily be forgotten that the number of objects per group should be the number in a full group!


{\bf Suggested timing:} Give students about 10 minutes to work through the first problem, then discuss. Repeat with the second problem, and finish with the third problem if you have time.




\end{instructorNotes}



\end{document}