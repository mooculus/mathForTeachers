\documentclass{ximera}

\usepackage{gensymb}
\usepackage{tabularx}
\usepackage{mdframed}
\usepackage{pdfpages}
%\usepackage{chngcntr}

\let\problem\relax
\let\endproblem\relax

\newcommand{\property}[2]{#1#2}




\newtheoremstyle{SlantTheorem}{\topsep}{\fill}%%% space between body and thm
 {\slshape}                      %%% Thm body font
 {}                              %%% Indent amount (empty = no indent)
 {\bfseries\sffamily}            %%% Thm head font
 {}                              %%% Punctuation after thm head
 {3ex}                           %%% Space after thm head
 {\thmname{#1}\thmnumber{ #2}\thmnote{ \bfseries(#3)}} %%% Thm head spec
\theoremstyle{SlantTheorem}
\newtheorem{problem}{Problem}[]

%\counterwithin*{problem}{section}



%%%%%%%%%%%%%%%%%%%%%%%%%%%%Jenny's code%%%%%%%%%%%%%%%%%%%%

%%% Solution environment
%\newenvironment{solution}{
%\ifhandout\setbox0\vbox\bgroup\else
%\begin{trivlist}\item[\hskip \labelsep\small\itshape\bfseries Solution\hspace{2ex}]
%\par\noindent\upshape\small
%\fi}
%{\ifhandout\egroup\else
%\end{trivlist}
%\fi}
%
%
%%% instructorIntro environment
%\ifhandout
%\newenvironment{instructorIntro}[1][false]%
%{%
%\def\givenatend{\boolean{#1}}\ifthenelse{\boolean{#1}}{\begin{trivlist}\item}{\setbox0\vbox\bgroup}{}
%}
%{%
%\ifthenelse{\givenatend}{\end{trivlist}}{\egroup}{}
%}
%\else
%\newenvironment{instructorIntro}[1][false]%
%{%
%  \ifthenelse{\boolean{#1}}{\begin{trivlist}\item[\hskip \labelsep\bfseries Instructor Notes:\hspace{2ex}]}
%{\begin{trivlist}\item[\hskip \labelsep\bfseries Instructor Notes:\hspace{2ex}]}
%{}
%}
%% %% line at the bottom} 
%{\end{trivlist}\par\addvspace{.5ex}\nobreak\noindent\hung} 
%\fi
%
%


\let\instructorNotes\relax
\let\endinstructorNotes\relax
%%% instructorNotes environment
\ifhandout
\newenvironment{instructorNotes}[1][false]%
{%
\def\givenatend{\boolean{#1}}\ifthenelse{\boolean{#1}}{\begin{trivlist}\item}{\setbox0\vbox\bgroup}{}
}
{%
\ifthenelse{\givenatend}{\end{trivlist}}{\egroup}{}
}
\else
\newenvironment{instructorNotes}[1][false]%
{%
  \ifthenelse{\boolean{#1}}{\begin{trivlist}\item[\hskip \labelsep\bfseries {\Large Instructor Notes: \\} \hspace{\textwidth} ]}
{\begin{trivlist}\item[\hskip \labelsep\bfseries {\Large Instructor Notes: \\} \hspace{\textwidth} ]}
{}
}
{\end{trivlist}}
\fi


%% Suggested Timing
\newcommand{\timing}[1]{{\bf Suggested Timing: \hspace{2ex}} #1}




\hypersetup{
    colorlinks=true,       % false: boxed links; true: colored links
    linkcolor=blue,          % color of internal links (change box color with linkbordercolor)
    citecolor=green,        % color of links to bibliography
    filecolor=magenta,      % color of file links
    urlcolor=cyan           % color of external links
}


\title{Multiplication Stories with Fractions}
\author{Vic Ferdinand, Betsy McNeal, Jenny Sheldon}

\begin{document}
\begin{abstract} \end{abstract}
\maketitle




\begin{problem}
 Solve each of the following stories using a picture.  Then, compare your solutions.  Be sure to address:   
\begin{itemize}
    \item How are the pictures alike?  How are they different?
    \item Can these situations be viewed as multiplication?  Do they fit with our usual meaning of multiplication?
\end{itemize}

\begin{enumerate}%[label=(\alph*)]
    \item My mom's oatmeal cookie recipe calls for 3 cups of oatmeal.  
I want to make 2 batches.
How much oatmeal will I need to make this?
    \item My mom's oatmeal cookie recipe calls for 3 cups of oatmeal.  
I want to make $\frac{2}{5}$ of a batch.
How much oatmeal will I need to make this?
    \item My mom's new oatmeal cookie recipe calls for $\frac{4}{7}$ cup of oatmeal.  
I want to make 2 batches.
How much oatmeal will I need to make this?
    \item My mom's newest oatmeal cookie recipe calls for  $\frac{4}{7}$ cup of oatmeal.  
I want to make $\frac{2}{5}$ of a batch.
How much oatmeal will I need to make this?
\end{enumerate}

\end{problem}

\newpage




\begin{problem}
 Write an expression involving multiplication which represents the story below, then solve the problem using a picture or pictures.  Be ready to explain how you drew your picture and what it means.  How does the picture tell you the answer?

\emph{I have a container that holds $\frac{5}{8}$ of a gallon when it is full.  Today, my container is $\frac{2}{3}$ full of milk. How much milk is in the container?}
\end{problem}
\vfill

\begin{problem}
 For each of the stories below, first write an expression involving multiplication which represents each of these stories.  Then, solve either version of this problem using pictures.  Be ready to explain how you drew your picture and what it means.  How does the picture tell you the answer? 

\begin{enumerate}
    \item I walked for  $\frac{3}{4}$ of an hour at a speed of  $\frac{7}{5}$ miles per hour.  How far have I walked?
    \item Rewrite the story in (a) so that it is the same multiplication problem using \emph{containers} and \emph{gallons}.
\end{enumerate}
\end{problem}
\vfill

\begin{problem}
 Write an expression involving multiplication which represents the story below, then solve the problem using a picture or pictures.  Be ready to explain how you drew your picture and what it means.  How does the picture tell you the answer?


\emph{A Cub Scout pack decided to make their own chocolate-covered popcorn for the annual sale, but they ate $2 \frac{1}{3}$ batches of the popcorn before selling any! If each batch of popcorn used $\frac{6}{5}$ of a pound of chocolate, how much chocolate did the Scouts eat?}

\end{problem}
\vfill
 
\newpage
\begin{instructorNotes}
This activity is ambitious in attempting to use these problems to first introduce students to the concept of multiplying fractions by building on the definition of multiplication for whole numbers, then help review the pictures they've drawn to solve each problems for patterns in their work.  These patterns should eventually yield the usual multiplication rule for fractions.  

In our course, this activity follows discussion of multiplication for whole numbers, so we can build off of that structure.  One goal we have in grouping operations together for the different classes of numbers is to help students see that the operation is unchanged - for instance, multiplication works the same way whether we work with whole numbers, decimals, fractions, or even negative numbers (with this last class of numbers coming later in our course).  The first problem in the activity helps us discuss this concept.

In small groups, students draw pictures to solve the problem, share their solutions, and discuss how their drawings illustrate the definition of multiplication.  The rest of the problems after the first one are for more practice with solving these problems by drawing a picture rather than a rule or a calculator.  Drawing and interpreting their pictures is a struggle for many students with these problems, which is why we have so many for practice.

We use whole class discussion to share the different pictures and analyses more broadly and also provide time for the class to ``step back''  and observe any patterns they see.  This should hopefully bring up the usual procedure in the pictures, as well as other observations.  We usually don't try to start this ``stepping back'' part of the activity until students are feeling more comfortable with solving the problems and seeing the answers in their pictures.

\begin{itemize}
     \item It may be a good idea to ask if there are multiplication problems where improper fractions do not make sense.  An example would be one where a ratio fraction is used:  $\frac{2}{5}$ of the class are girls.  $\frac{4}{7}$ of the girls have blond hair.  What fraction of the class are blond-haired girls?
     \item  The story problem above could initially be given to the class in its first line only:  $\frac{2}{5}$ of the class are girls.  Then have the class write the rest of the story so that it models $\frac{2}{5}$ times $\frac{4}{7}$ (Check to see if the story works by drawing a picture and see if it is equivalent to that of the previous problems).
\end{itemize}

{\bf Suggested Timing:} We spend two or three class periods to fully discuss this activity.  
\begin{itemize}
	\item On the first day, give students about 10-15 minutes to work on all of the parts of problem 1, then spend the next 30 minutes in discussion. Have several groups present their work on each of the parts. Be sure to discuss what the students are seeing as one group and one object for each of the parts. For the final 10-15 minutes, use whole-class discussion to compare and contrast the parts of problem 1 to make the point that we are still using multiplication in the same way for fractions as whole numbers. Clear up any confusion and perhaps have students start thinking about problem 2.
	\item On the second day, give students about 15-20 minutes to work on problems 2, 3, and 4, then spend the next 30 minutes in discussion. Again, have several groups present their work on each of the problems. Be on a careful watch for the incorrect ``criss-cross'' picture which has incorrect wholes shaded and discuss this as much as needed. The goal is for students to be able to solve these fraction problems using a picture by the end of this class period. Spend the last 10-15 minutes again clearing up confusion and highlighting the general process we use to solve these problems using a picture. You can also spend extra time here working on how we identify the groups and objects per group for fraction multiplication problems.
	\item On the third day, the goal is to explain why we multiply the numerators and multiply the denominators. You can either use the four problems from the first two days in parallel, or you can start by working through another practice problem as a whole class and clarifying any lingering issues. Pose the question (on the board): "How does the picture show us that we should multiply the denominators to get the denominator of the answer? What are the groups and objects per group for this multiplication, and how is the meaning of the denominator involved? How does the picture show us that we should multiply the numerators to get the numerator of the answer? What are the groups and objects per group for this multiplication, and how is the meaning of the numerator involved?" Give students about 10 minutes to think about this problem, and then discuss for about 20 minutes (or longer if needed). If you have extra time, ask students to repeat this process by first writing their own fraction multiplication problem, solving their problem, then explaining why we multiply numerators and denominators again.
\end{itemize}

\end{instructorNotes}



\end{document}
