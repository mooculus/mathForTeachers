\documentclass[nooutcomes]{ximera}
\usepackage{gensymb}
\usepackage{tabularx}
\usepackage{mdframed}
\usepackage{pdfpages}
%\usepackage{chngcntr}

\let\problem\relax
\let\endproblem\relax

\newcommand{\property}[2]{#1#2}




\newtheoremstyle{SlantTheorem}{\topsep}{\fill}%%% space between body and thm
 {\slshape}                      %%% Thm body font
 {}                              %%% Indent amount (empty = no indent)
 {\bfseries\sffamily}            %%% Thm head font
 {}                              %%% Punctuation after thm head
 {3ex}                           %%% Space after thm head
 {\thmname{#1}\thmnumber{ #2}\thmnote{ \bfseries(#3)}} %%% Thm head spec
\theoremstyle{SlantTheorem}
\newtheorem{problem}{Problem}[]

%\counterwithin*{problem}{section}



%%%%%%%%%%%%%%%%%%%%%%%%%%%%Jenny's code%%%%%%%%%%%%%%%%%%%%

%%% Solution environment
%\newenvironment{solution}{
%\ifhandout\setbox0\vbox\bgroup\else
%\begin{trivlist}\item[\hskip \labelsep\small\itshape\bfseries Solution\hspace{2ex}]
%\par\noindent\upshape\small
%\fi}
%{\ifhandout\egroup\else
%\end{trivlist}
%\fi}
%
%
%%% instructorIntro environment
%\ifhandout
%\newenvironment{instructorIntro}[1][false]%
%{%
%\def\givenatend{\boolean{#1}}\ifthenelse{\boolean{#1}}{\begin{trivlist}\item}{\setbox0\vbox\bgroup}{}
%}
%{%
%\ifthenelse{\givenatend}{\end{trivlist}}{\egroup}{}
%}
%\else
%\newenvironment{instructorIntro}[1][false]%
%{%
%  \ifthenelse{\boolean{#1}}{\begin{trivlist}\item[\hskip \labelsep\bfseries Instructor Notes:\hspace{2ex}]}
%{\begin{trivlist}\item[\hskip \labelsep\bfseries Instructor Notes:\hspace{2ex}]}
%{}
%}
%% %% line at the bottom} 
%{\end{trivlist}\par\addvspace{.5ex}\nobreak\noindent\hung} 
%\fi
%
%


\let\instructorNotes\relax
\let\endinstructorNotes\relax
%%% instructorNotes environment
\ifhandout
\newenvironment{instructorNotes}[1][false]%
{%
\def\givenatend{\boolean{#1}}\ifthenelse{\boolean{#1}}{\begin{trivlist}\item}{\setbox0\vbox\bgroup}{}
}
{%
\ifthenelse{\givenatend}{\end{trivlist}}{\egroup}{}
}
\else
\newenvironment{instructorNotes}[1][false]%
{%
  \ifthenelse{\boolean{#1}}{\begin{trivlist}\item[\hskip \labelsep\bfseries {\Large Instructor Notes: \\} \hspace{\textwidth} ]}
{\begin{trivlist}\item[\hskip \labelsep\bfseries {\Large Instructor Notes: \\} \hspace{\textwidth} ]}
{}
}
{\end{trivlist}}
\fi


%% Suggested Timing
\newcommand{\timing}[1]{{\bf Suggested Timing: \hspace{2ex}} #1}




\hypersetup{
    colorlinks=true,       % false: boxed links; true: colored links
    linkcolor=blue,          % color of internal links (change box color with linkbordercolor)
    citecolor=green,        % color of links to bibliography
    filecolor=magenta,      % color of file links
    urlcolor=cyan           % color of external links
}
\title{Symmetry}
\author{Vic Ferdinand, Betsy McNeal, Jenny Sheldon}

\begin{document}
\begin{abstract}
We say an object has symmetry if we apply some transformation to that object, and the result looks exactly the same as before we applied that transformation.  Let's think about what this definition really means.
\end{abstract}
\maketitle



\begin{problem}
This problem is brought to you by the letter H.

\begin{center}
    \begin{tikzpicture}
        \draw[fill=black] (0,0)--(0.5,0)--(0.5,1.6)--(2.2,1.6)--(2.2,0)--(2.7,0)--(2.7,3.7)--(2.2,3.7)--(2.2,2.1)--(0.5,2.1)--(0.5,3.7)--(0,3.7)--(0,0);
    \end{tikzpicture}
\end{center}
Find all of the symmetries of the letter H.  In other words: apply a transformation to the letter H.  Does it still look like H?  Is it still in the same place on the page?  Are both of these questions important?


\end{problem}

\begin{problem}
What are all of the symmetries of first letter of your first name?  (If your name begins with H, choose another random letter.)


\end{problem}

\begin{problem} \label{Symmetry3}
Are there any letters of the alphabet that have no symmetry?  Why or why not?

\end{problem}

\begin{problem}
Are there any letters of the alphabet that have translation symmetry?  Why or why not?

\end{problem}



\newpage
\begin{instructorNotes}
This activity is designed to introduce the notion of symmetry.  The definition of symmetry that we use is included in the activity, so students also get practice with applying a new definition to these problems.  In discussion, we emphasize the types of exact information that students need to include in their answers.  For instance, with a reflection symmetry, students should give the line of reflection.  In our calendar, this activity comes after we have defined the basic transformations (reflections, rotations, and transformations).

Our textbook uses the terminology of an ``$n$-fold rotational symmetry'', but in our experience students find this deceptive because it refers to the number of possible rotations, not to folding (which is suggested as a way to test reflection symmetry).  

\begin{itemize}
    \item We usually suggest that students work in pairs to solve this problem, using two copies of the activity.  Placing one copy of the activity on top of another makes it easier to check for symmetries and visualize the result of the transformations.  We also sometimes have tracing paper available for students to use with this activity.
    \item If students are confused by testing letters, or if they test all the letters too quickly, we usually suggest that they try drawing their own shapes and challenging their group members to find all of the symmetries.
    \item Some students have had some difficulty understanding and visualizing rotational symmetry, especially if they aren't trying to actually draw the rotation.  It can be unclear that the object needs to be rotated onto itself to yield the same picture.  Students also frequently forget that the center of the rotation needs to be clearly stated.
    \item Problem \ref{Symmetry3} helps us to be sure we don't gloss over the $360\degree$ rotational symmetry.
    \item Students often find translation symmetry a bit confusing since they sometimes do not understand that the design itself must be made up of translated copies of a shape in order for the design to have translation symmetry.  We usually ask students if they can create any design that has translation symmetry, and usually someone comes up with the idea of a design extending infinitely, or being wrapped around a sphere or other circle.


\end{itemize}



\timing{This activity takes us about half a class period.  We give the students about 5 minutes to think about the meaning of symmetry and try the first problem, then discuss.  Then, we give the students about 5-10 minutes to do the rest of the problems, and discuss the activity as a whole.}
\end{instructorNotes}


\end{document}





