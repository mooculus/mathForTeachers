\documentclass{ximera}
\usepackage{gensymb}
\usepackage{tabularx}
\usepackage{mdframed}
\usepackage{pdfpages}
%\usepackage{chngcntr}

\let\problem\relax
\let\endproblem\relax

\newcommand{\property}[2]{#1#2}




\newtheoremstyle{SlantTheorem}{\topsep}{\fill}%%% space between body and thm
 {\slshape}                      %%% Thm body font
 {}                              %%% Indent amount (empty = no indent)
 {\bfseries\sffamily}            %%% Thm head font
 {}                              %%% Punctuation after thm head
 {3ex}                           %%% Space after thm head
 {\thmname{#1}\thmnumber{ #2}\thmnote{ \bfseries(#3)}} %%% Thm head spec
\theoremstyle{SlantTheorem}
\newtheorem{problem}{Problem}[]

%\counterwithin*{problem}{section}



%%%%%%%%%%%%%%%%%%%%%%%%%%%%Jenny's code%%%%%%%%%%%%%%%%%%%%

%%% Solution environment
%\newenvironment{solution}{
%\ifhandout\setbox0\vbox\bgroup\else
%\begin{trivlist}\item[\hskip \labelsep\small\itshape\bfseries Solution\hspace{2ex}]
%\par\noindent\upshape\small
%\fi}
%{\ifhandout\egroup\else
%\end{trivlist}
%\fi}
%
%
%%% instructorIntro environment
%\ifhandout
%\newenvironment{instructorIntro}[1][false]%
%{%
%\def\givenatend{\boolean{#1}}\ifthenelse{\boolean{#1}}{\begin{trivlist}\item}{\setbox0\vbox\bgroup}{}
%}
%{%
%\ifthenelse{\givenatend}{\end{trivlist}}{\egroup}{}
%}
%\else
%\newenvironment{instructorIntro}[1][false]%
%{%
%  \ifthenelse{\boolean{#1}}{\begin{trivlist}\item[\hskip \labelsep\bfseries Instructor Notes:\hspace{2ex}]}
%{\begin{trivlist}\item[\hskip \labelsep\bfseries Instructor Notes:\hspace{2ex}]}
%{}
%}
%% %% line at the bottom} 
%{\end{trivlist}\par\addvspace{.5ex}\nobreak\noindent\hung} 
%\fi
%
%


\let\instructorNotes\relax
\let\endinstructorNotes\relax
%%% instructorNotes environment
\ifhandout
\newenvironment{instructorNotes}[1][false]%
{%
\def\givenatend{\boolean{#1}}\ifthenelse{\boolean{#1}}{\begin{trivlist}\item}{\setbox0\vbox\bgroup}{}
}
{%
\ifthenelse{\givenatend}{\end{trivlist}}{\egroup}{}
}
\else
\newenvironment{instructorNotes}[1][false]%
{%
  \ifthenelse{\boolean{#1}}{\begin{trivlist}\item[\hskip \labelsep\bfseries {\Large Instructor Notes: \\} \hspace{\textwidth} ]}
{\begin{trivlist}\item[\hskip \labelsep\bfseries {\Large Instructor Notes: \\} \hspace{\textwidth} ]}
{}
}
{\end{trivlist}}
\fi


%% Suggested Timing
\newcommand{\timing}[1]{{\bf Suggested Timing: \hspace{2ex}} #1}




\hypersetup{
    colorlinks=true,       % false: boxed links; true: colored links
    linkcolor=blue,          % color of internal links (change box color with linkbordercolor)
    citecolor=green,        % color of links to bibliography
    filecolor=magenta,      % color of file links
    urlcolor=cyan           % color of external links
}
\title{Symmetry}
\author{Vic Ferdinand, Betsy McNeal, Jenny Sheldon}

\begin{document}
\begin{abstract}
We say an object has symmetry if we apply some transformation to that object, and the result looks exactly the same as before we applied that transformation.  Let's think about what this definition really means.
\end{abstract}
\maketitle

\begin{instructorIntro}
This activity is designed to introduce the notion of symmetry.  The alphabet letters have both reflection and rotational symmetry, and you should discuss that when specifying a symmetry, one should make sure to specify the transformation exactly.  For instance, with a reflection symmetry, students should give the line of reflection.  With a rotational symmetry, students should give the angle of rotation.  You can bring up the idea of $n$-fold rotational symmetry if you would like to use this terminology, but it will likely be easiest for students to grasp this idea as collecting together all of the rotational symmetries they have already found.

Students find the term \emph{n-fold symmetry} deceptive because it refers to the number of possible rotations, not to folding (which is suggested as a way to test reflection symmetry).  

Students find translation symmetry a bit confusing because any shape can be translated - they do not seem to understand that the design itself must be made up of translated copies of a shape in order for the design to have translation symmetry.  Translation symmetry also has the added complication that it technically can only occur when the design is infinitely long. Sometimes we will still refer to a finite object as having translation symmetry (e.g., not paying attention to the ``blank'' left side if translate to the right), but this can muddy the water further.

Some students have had some difficulty understanding rotational symmetry (esp. without doing the rotations -- especially that the entire figure needed to be rotated onto itself to yield the same picture). 

\timing{This activity will probably take about half a class period.  Give the students about 5 minutes to think about the meaning of symmetry and try the first problem, then discuss.  Give the students about 5-10 minutes to do the rest of the problems, and then discuss the activity as a whole.}
\end{instructorIntro}

\begin{problem}
This problem is brought to you by the letter H.
\[
\includegraphics[width=1in]{graphics/H.pdf}
\]
Find all of the symmetries of the letter H.  In other words: apply a transformation to the letter H.  Does it still look like H?  Is it still in the same place on the page?  Are both of these questions important?

\begin{instructorNotes}
You might suggest that students work in groups to solve this problem, and use two copies of the activity.  If students place one copy on top of the other, they can more easily check for symmetries and visualize the result of the transformations.

Discuss how the two rotational symmetries can be thought of separately as a rotation of $180\degree$ and $360\degree$, or together as $2$-fold rotational symmetry. 
\end{instructorNotes}
\end{problem}

\begin{problem}
What are all of the symmetries of first letter of your first name?  (If your name begins with H, choose another random letter.)

\begin{instructorNotes}
You should suggest that students try as many letters as they need to in order to understand the concepts.  In the next problem, they may have to test most of the letters in order to solve the problem anyway, so this work will not be wasted.  You can also suggest that students draw their own shapes and test for symmetry, or give the students another design to think about that is not a letter of the alphabet.
\end{instructorNotes}
\end{problem}

\begin{problem}
Are there any letters of the alphabet that have no symmetry?  Why or why not?
\begin{instructorNotes}
The official answer to this question should be ``no'', as every letter has $360\degree$ rotational symmetry.  However, if students catch on to this quickly, you can suggest they search for letters with only $1$-fold rotational symmetry.  You may clarify for students that if we ask a question like this on a test, we will generally exclude the case of $1$-fold rotational symmetry.
\end{instructorNotes}
\end{problem}

\begin{problem}
Are there any letters of the alphabet that have translation symmetry?  Why or why not?
\begin{instructorNotes}
Here is the first opportunity to discuss translation symmetry in depth.  The idea may have come up earlier, as students tested letters of the alphabet, but it's good to summarize the discussion here.  Ask students if they could create any design that had translation symmetry.  Hopefully a student will come up with the idea of the object extending infinitely, or perhaps living on a sphere so that it has no ends.
\end{instructorNotes}
\end{problem}


\end{document}





