\documentclass[nooutcomes]{ximera}

\usepackage{gensymb}
\usepackage{tabularx}
\usepackage{mdframed}
\usepackage{pdfpages}
%\usepackage{chngcntr}

\let\problem\relax
\let\endproblem\relax

\newcommand{\property}[2]{#1#2}




\newtheoremstyle{SlantTheorem}{\topsep}{\fill}%%% space between body and thm
 {\slshape}                      %%% Thm body font
 {}                              %%% Indent amount (empty = no indent)
 {\bfseries\sffamily}            %%% Thm head font
 {}                              %%% Punctuation after thm head
 {3ex}                           %%% Space after thm head
 {\thmname{#1}\thmnumber{ #2}\thmnote{ \bfseries(#3)}} %%% Thm head spec
\theoremstyle{SlantTheorem}
\newtheorem{problem}{Problem}[]

%\counterwithin*{problem}{section}



%%%%%%%%%%%%%%%%%%%%%%%%%%%%Jenny's code%%%%%%%%%%%%%%%%%%%%

%%% Solution environment
%\newenvironment{solution}{
%\ifhandout\setbox0\vbox\bgroup\else
%\begin{trivlist}\item[\hskip \labelsep\small\itshape\bfseries Solution\hspace{2ex}]
%\par\noindent\upshape\small
%\fi}
%{\ifhandout\egroup\else
%\end{trivlist}
%\fi}
%
%
%%% instructorIntro environment
%\ifhandout
%\newenvironment{instructorIntro}[1][false]%
%{%
%\def\givenatend{\boolean{#1}}\ifthenelse{\boolean{#1}}{\begin{trivlist}\item}{\setbox0\vbox\bgroup}{}
%}
%{%
%\ifthenelse{\givenatend}{\end{trivlist}}{\egroup}{}
%}
%\else
%\newenvironment{instructorIntro}[1][false]%
%{%
%  \ifthenelse{\boolean{#1}}{\begin{trivlist}\item[\hskip \labelsep\bfseries Instructor Notes:\hspace{2ex}]}
%{\begin{trivlist}\item[\hskip \labelsep\bfseries Instructor Notes:\hspace{2ex}]}
%{}
%}
%% %% line at the bottom} 
%{\end{trivlist}\par\addvspace{.5ex}\nobreak\noindent\hung} 
%\fi
%
%


\let\instructorNotes\relax
\let\endinstructorNotes\relax
%%% instructorNotes environment
\ifhandout
\newenvironment{instructorNotes}[1][false]%
{%
\def\givenatend{\boolean{#1}}\ifthenelse{\boolean{#1}}{\begin{trivlist}\item}{\setbox0\vbox\bgroup}{}
}
{%
\ifthenelse{\givenatend}{\end{trivlist}}{\egroup}{}
}
\else
\newenvironment{instructorNotes}[1][false]%
{%
  \ifthenelse{\boolean{#1}}{\begin{trivlist}\item[\hskip \labelsep\bfseries {\Large Instructor Notes: \\} \hspace{\textwidth} ]}
{\begin{trivlist}\item[\hskip \labelsep\bfseries {\Large Instructor Notes: \\} \hspace{\textwidth} ]}
{}
}
{\end{trivlist}}
\fi


%% Suggested Timing
\newcommand{\timing}[1]{{\bf Suggested Timing: \hspace{2ex}} #1}




\hypersetup{
    colorlinks=true,       % false: boxed links; true: colored links
    linkcolor=blue,          % color of internal links (change box color with linkbordercolor)
    citecolor=green,        % color of links to bibliography
    filecolor=magenta,      % color of file links
    urlcolor=cyan           % color of external links
}


\title{Adding It All Up}
\author{Vic Ferdinand, Betsy McNeal, Jenny Sheldon}

\begin{document}
\begin{abstract} In this activity, we will explore the meanings of addition and subtraction. \end{abstract}
\maketitle



\begin{problem}
Imagine that you are a young student in Kindergarten or first grade.  If you had blocks or other objects, how would you solve the following problems?  What are the differences and similarities between them?
\begin{enumerate}
\item Johnny has 7 apples and Susie has 5 apples.  How many apples are there altogether?
\item Johnny has 7 apples.  Susie gives him 5 apples.  How many apples does Johnny have now?
\item Johnny has some apples.  He gives 7 of the apples away to Susie and now has 5 apples.  How many apples did Johnny start with?
\item Johnny has 7 more apples than Susie.  If Susie has 5 apples, how many apples does Johnny have?
\item How would these problems change if the numbers 7 and 5 were replaced with 7264 and 567?
\end{enumerate}
\end{problem}

\begin{problem}
 Thinking about how you would use the blocks to solve the following problems, what are the differences and similarities between them?
\begin{enumerate}
\item Altogether Johnny and Susie have 12 apples.  If Johnny has 7 apples, how many apples does Susie have?
\item Johnny has 7 apples.  Susie gives him some apples and Johnny now has 12 apples.  How many apples did Susie give Johnny?
\item Johnny has 12 apples.  He gives 7 of the apples away to Susie.  How many apples does Johnny have now?
\item Johnny has 12 apples and Susie has 7 apples.  How many more apples does Johnny have than Susie?
\end{enumerate}
\end{problem}

\begin{problem}
There are 60 people in the band and 40 people in the choir.  If everyone from each group (and no one else) shows up at the annual band-choir party, how many people will be there?
\end{problem}

\begin{problem} \label{AddingUpSummary}
 Come up with a definition of addition.  That is, if $a$ and $b$ are numbers, what does $a + b$ mean?  Likewise, come up with a definition of ``take-away" subtraction.  Likewise, come up with a definition of ``missing-addend" subtraction.  
\end{problem}


\newpage
\begin{instructorNotes}
This set of example problems was developed to draw out and illustrate the different types of addition and subtraction.  We ask students to be able to identify a comparison or missing addend situation as addition, and take away, compare or missing addend as subtraction.  We try to focus very heavily on the structure of the problem, keeping in mind that our students have usually not encountered this idea very much prior to our course.

Students are asked to read each problem, consider how a child might solve it using blocks or objects, and discuss how those actions help us decide whether a story problem shows the addition operation, the subtraction operation, or both.  In whole-class discussion, we point out the similar structures between problems. Students enjoy trying to imagine how children would use objects to act out the problems and are very willing to take on that role themselves.


\begin{itemize}
	\item In our course, this activity follows extensive work on the meaning of numbers (whole numbers, fractions, decimals) without reference to operations.  This activity thus marks a change of direction -- we are now focusing on binary operations. Before, especially with fractions, we dealt only with the meanings, notation, and comparison of quantities.  Now we are operating on them. This is a BIG difference.
	\item We found students to be unfamiliar with the idea of thinking about ``mathematical structure'', so this activity was designed to	help our future teachers uncover the differences among the story problems.  Because they already easily recognize addition and subtraction as adults, we have our students think about how the actions very young children would take to solve these problems with blocks mimic the structure of the problems. This seems to help our students make these differences explicit in terms of problem structure.  We use the word ``model'' to refer to these structures.
	\item Our main model of addition is the {\em joining} of two sets, which children will first do by pushing blocks together and then counting the result.  {\em Missing-addend} situations are also a process of ``joining'', it is just a matter of how they are joined.
	\item We help students see that there are three (not two) numbers involved in each story problem and each one plays a certain role within the model (and any one of the three could serve as the ``unknown'').  Also, we point out either here or later that the unit of each quantity is the same in addition and subtraction, in contrast to multiplication and division.
	%\item You might note that where we have, for example, $7 + ? = 12$, it later becomes $7 + x = 12$.
	\item Problem \ref{AddingUpSummary} is a summary, which we sometimes do in whole-class discussion.
\end{itemize}

{\bf Suggested Timing:} We use the whole class period for this activity; about 20 minutes for groups to read and study the handout, the rest of the time for discussion. 
\end{instructorNotes}

\end{document}