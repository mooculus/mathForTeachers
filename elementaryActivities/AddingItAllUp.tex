\documentclass{ximera}


\graphicspath{
  {./}
  {graphics/}
  {../graphics/}
}

\usepackage{chngcntr}

\let\question\relax
\let\endquestion\relax




\newtheoremstyle{SlantTheorem}{\topsep}{\fill}%%% space between body and thm
%\newtheoremstyle{SlantTheorem}{\topsep}{\topsep}%%% space between body and thm
 {\slshape}                      %%% Thm body font
 {}                              %%% Indent amount (empty = no indent)
 {\bfseries\sffamily}            %%% Thm head font
 {}                              %%% Punctuation after thm head
 {3ex}                           %%% Space after thm head
 {\thmname{#1}\thmnumber{ #2}\thmnote{ \bfseries(#3)}}%%% Thm head spec
\theoremstyle{SlantTheorem}
\newtheorem{question}{Question}
\counterwithin*{question}{section}



\let\instructorNotes\relax
\let\endinstructorNotes\relax
%%% instructorNotes environment
\ifhandout
\newenvironment{instructorNotes}[1][false]%
{%
\def\givenatend{\boolean{#1}}\ifthenelse{\boolean{#1}}{\begin{trivlist}\item}{\setbox0\vbox\bgroup}{}
}
{%
\ifthenelse{\givenatend}{\end{trivlist}}{\egroup}{}
}
\else
\newenvironment{instructorNotes}[1][false]%
{%
  \ifthenelse{\boolean{#1}}{\begin{trivlist}\item[\hskip \labelsep\bfseries {\Large Instructor Notes: \\} \hspace{\textwidth} ]}
{\begin{trivlist}\item[\hskip \labelsep\bfseries {\Large Instructor Notes: \\} \hspace{\textwidth} ]}
{}
}
{\end{trivlist}}
\fi


%% Suggested Timing
\newcommand{\timing}[1]{{\bf Suggested Timing: \hspace{2ex}} #1}


\title{Adding It All Up}
\author{Vic Ferdinand, Betsy McNeal, Jenny Sheldon}

\begin{document}
\begin{abstract} In this activity, we will explore the meanings of addition and subtraction. \end{abstract}
\maketitle

\begin{instructorIntro}
This set of problem examples provides a way to draw out and illustrate the different types of addition and subtraction problems:  addition can be interpreted as the joining of disjoint sets or as a missing addend situation, while subtraction can be interpreted in three basic forms: take away, compare, or missing addend.  Students are asked to read and discuss the problems, considering which are addition, which subtraction, and what the similarities and differences are among the problems in each collection.

We found students unfamiliar with the idea of thinking about ``mathematical structure''.  Students enjoy trying to imagine how children would use objects to act out the problems and are very willing to take on that role themselves.


\begin{itemize}
	\item It is important to note that we're now exclusively dealing with operations (binary -- using two numbers in a way to get a third, possibly equivalent, number). Before, especially with fractions, we just dealt with the meanings, notation, and comparing of quantities.  Now we are operating on them. This is a BIG difference that often gets overlooked.
	\item We want to emphasize that the actions very young children would take to solve these problems with blocks mimic the structure of the problems and highlight the differences among them.  We are using the word ``model'' to refer to these structures.
	\item Our main model of addition is the {\em joining} of two sets, which children will first do by pushing blocks together and then counting the result.  {\em Missing-addend} situations are also a process of ``joining'', it is just a matter of how they go about being joined.
	\item A major point to make is that there are three (not two) numbers involved in each story problem and each one plays a certain role within the model (and any one of the three could serve as the ``unknown'').  Also, the unit of each quantity is the same (an important point to recall when introducing multiplication).
	\item You might note that where we have, for example, $7 + ? = 12$, it later becomes $7 + x = 12$.
	\item \#4 is a summary, which can be done in whole-class discussion if necessary.
\end{itemize}

{\bf Suggested Timing:} The whole class period will be needed for this activity; about 20 minutes for groups to read and study the handout, the rest of the time for discussion. 
\end{instructorIntro}

\begin{question}
Imagine that you are a young student in Kindergarten or first grade.  If you had blocks or other objects, how would you solve the following problems?  What are the differences and similarities between them?
\begin{enumerate}
\item Johnny has 7 apples and Susie has 5 apples.  How many apples are there altogether?
\item Johnny has 7 apples.  Susie gives him 5 apples.  How many apples does Johnny have now?
\item Johnny has some apples.  He gives 7 of the apples away to Susie and now has 5 apples.  How many apples did Johnny start with?
\item Johnny has 7 more apples than Susie.  If Susie has 5 apples, how many apples does Johnny have?
\item How would these problems change if the numbers 7 and 5 were replaced with 7264 and 567?
\end{enumerate}
\end{question}

%\item What are the differences and similarities between the following problems?
%\begin{enumerate}
%\item Johnny has 7264 apples and Susie has 567 apples.  How many apples are there altogether?
%\item Johnny has 7264 apples.  Susie gives him 567 apples.  How many apples does Johnny have now?
%\item Johnny has some apples.  He gives 7264 of the apples away to Susie and now has 567 apples.  How many apples did Johnny start with?
%\item Johnny has 7264 more apples than Susie.  If Susie has 567 apples, how many apples does Johnny have?
%\end{enumerate}

\begin{question}
 Thinking about how you would use the blocks to solve the following problems, what are the differences and similarities between them?
\begin{enumerate}
\item Altogether Johnny and Susie have 12 apples.  If Johnny has 7 apples, how many apples does Susie have?
\item Johnny has 7 apples.  Susie gives him some apples and Johnny now has 12 apples.  How many apples did Susie give Johnny?
\item Johnny has 12 apples.  He gives 7 of the apples away to Susie.  How many apples does Johnny have now?
\item Johnny has 12 apples and Susie has 7 apples.  How many more apples does Johnny have than Susie?
\end{enumerate}
\end{question}

\begin{question}
There are 60 people in the band and 40 people in the choir.  If everyone from each group (and no one else) shows up at the annual band-choir party, how many people will be there?
\end{question}

\begin{question}
 Come up with a definition of addition.  That is, if \emph{a} and \emph{b} are numbers, what does $a + b$ mean?  Likewise, come up with a definition of ``take-away" subtraction.  Likewise, come up with a definition of ``missing-addend" subtraction.  
\end{question}


\end{document}