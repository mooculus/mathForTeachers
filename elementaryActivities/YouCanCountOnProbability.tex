\documentclass{ximera}
\usepackage{gensymb}
\usepackage{tabularx}
\usepackage{mdframed}
\usepackage{pdfpages}
%\usepackage{chngcntr}

\let\problem\relax
\let\endproblem\relax

\newcommand{\property}[2]{#1#2}




\newtheoremstyle{SlantTheorem}{\topsep}{\fill}%%% space between body and thm
 {\slshape}                      %%% Thm body font
 {}                              %%% Indent amount (empty = no indent)
 {\bfseries\sffamily}            %%% Thm head font
 {}                              %%% Punctuation after thm head
 {3ex}                           %%% Space after thm head
 {\thmname{#1}\thmnumber{ #2}\thmnote{ \bfseries(#3)}} %%% Thm head spec
\theoremstyle{SlantTheorem}
\newtheorem{problem}{Problem}[]

%\counterwithin*{problem}{section}



%%%%%%%%%%%%%%%%%%%%%%%%%%%%Jenny's code%%%%%%%%%%%%%%%%%%%%

%%% Solution environment
%\newenvironment{solution}{
%\ifhandout\setbox0\vbox\bgroup\else
%\begin{trivlist}\item[\hskip \labelsep\small\itshape\bfseries Solution\hspace{2ex}]
%\par\noindent\upshape\small
%\fi}
%{\ifhandout\egroup\else
%\end{trivlist}
%\fi}
%
%
%%% instructorIntro environment
%\ifhandout
%\newenvironment{instructorIntro}[1][false]%
%{%
%\def\givenatend{\boolean{#1}}\ifthenelse{\boolean{#1}}{\begin{trivlist}\item}{\setbox0\vbox\bgroup}{}
%}
%{%
%\ifthenelse{\givenatend}{\end{trivlist}}{\egroup}{}
%}
%\else
%\newenvironment{instructorIntro}[1][false]%
%{%
%  \ifthenelse{\boolean{#1}}{\begin{trivlist}\item[\hskip \labelsep\bfseries Instructor Notes:\hspace{2ex}]}
%{\begin{trivlist}\item[\hskip \labelsep\bfseries Instructor Notes:\hspace{2ex}]}
%{}
%}
%% %% line at the bottom} 
%{\end{trivlist}\par\addvspace{.5ex}\nobreak\noindent\hung} 
%\fi
%
%


\let\instructorNotes\relax
\let\endinstructorNotes\relax
%%% instructorNotes environment
\ifhandout
\newenvironment{instructorNotes}[1][false]%
{%
\def\givenatend{\boolean{#1}}\ifthenelse{\boolean{#1}}{\begin{trivlist}\item}{\setbox0\vbox\bgroup}{}
}
{%
\ifthenelse{\givenatend}{\end{trivlist}}{\egroup}{}
}
\else
\newenvironment{instructorNotes}[1][false]%
{%
  \ifthenelse{\boolean{#1}}{\begin{trivlist}\item[\hskip \labelsep\bfseries {\Large Instructor Notes: \\} \hspace{\textwidth} ]}
{\begin{trivlist}\item[\hskip \labelsep\bfseries {\Large Instructor Notes: \\} \hspace{\textwidth} ]}
{}
}
{\end{trivlist}}
\fi


%% Suggested Timing
\newcommand{\timing}[1]{{\bf Suggested Timing: \hspace{2ex}} #1}




\hypersetup{
    colorlinks=true,       % false: boxed links; true: colored links
    linkcolor=blue,          % color of internal links (change box color with linkbordercolor)
    citecolor=green,        % color of links to bibliography
    filecolor=magenta,      % color of file links
    urlcolor=cyan           % color of external links
}
\title{You Can Count On Probability}
\author{Vic Ferdinand, Betsy McNeal, Jenny Sheldon}

\begin{document}
\begin{abstract}
\end{abstract}
\maketitle




\begin{problem}
You choose a card from a deck consisting of 10 distinct cards.  Though the cards are all distinct, four of them are blue, two are red, and the rest are yellow.  Then you roll a six-sided die.  What is the probability that:
\begin{enumerate}
\item you draw a blue card?
\item you draw a blue card and roll a 1?
\item you draw a blue or yellow card and roll an odd number?
\end{enumerate}
\vfill
\end{problem}

\begin{problem}
A class has 10 children, three of which are named Jay.  
\begin{enumerate}
\item What is the probability that if you choose a group of three children, exactly one of them will be named Jay?
\item What is the probability that if you choose a group of three children, at least one of them will be named Jay?
\end{enumerate}
\vfill
\end{problem}

\newpage
\begin{instructorNotes}
In this activity, we take the problems from ``You Can Count On It Getting More Difficult'' and rephrase them as probability problems.  We are trying to focus students on whether the outcomes are equally likely, and to continue to make use of the structure of problems by using work they have previously done.

In our course, we have already introduced the meaning of probability as well as discussed experimental versus theoretical probability in the activities ``Fun with Fractions'' and ``High Rollers''.  This knowledge is assumed in the current activity.  We generally follow this activity with using tree diagrams to calculate probability.

In general, we use these activities about counting and probability to help cement the meaning of various operations for students.  Throughout these activities, we expect students to justify their operations of choice in their explanations.  There are a few main types of counting strategies we see from our students: arrays, ordered lists, tree diagrams, and algebraic expressions.  Students tend to want to use more complicated strategies before trying simpler strategies, so we are often reminding them to write down some examples, or to make sure their list is well organized. When discussing this activity as a whole class, we like to have students demonstrate multiple solution methods for each problem, even when not specifically requested by the problem.  

We have found the final problem to be difficult, especially if the original counting problem was not thoroughly discussed in a previous class.  We often spend significant time on this problem.

\timing{We use one class period for this activity.  If we are behind schedule from previous activities, we tend to choose those parts which we feel will most benefit our students.}
\end{instructorNotes}


\end{document}