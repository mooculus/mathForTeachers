\documentclass{ximera}


\graphicspath{
  {./}
  {graphics/}
  {../graphics/}
}

\usepackage{chngcntr}

\let\question\relax
\let\endquestion\relax




\newtheoremstyle{SlantTheorem}{\topsep}{\fill}%%% space between body and thm
%\newtheoremstyle{SlantTheorem}{\topsep}{\topsep}%%% space between body and thm
 {\slshape}                      %%% Thm body font
 {}                              %%% Indent amount (empty = no indent)
 {\bfseries\sffamily}            %%% Thm head font
 {}                              %%% Punctuation after thm head
 {3ex}                           %%% Space after thm head
 {\thmname{#1}\thmnumber{ #2}\thmnote{ \bfseries(#3)}}%%% Thm head spec
\theoremstyle{SlantTheorem}
\newtheorem{question}{Question}
\counterwithin*{question}{section}



\let\instructorNotes\relax
\let\endinstructorNotes\relax
%%% instructorNotes environment
\ifhandout
\newenvironment{instructorNotes}[1][false]%
{%
\def\givenatend{\boolean{#1}}\ifthenelse{\boolean{#1}}{\begin{trivlist}\item}{\setbox0\vbox\bgroup}{}
}
{%
\ifthenelse{\givenatend}{\end{trivlist}}{\egroup}{}
}
\else
\newenvironment{instructorNotes}[1][false]%
{%
  \ifthenelse{\boolean{#1}}{\begin{trivlist}\item[\hskip \labelsep\bfseries {\Large Instructor Notes: \\} \hspace{\textwidth} ]}
{\begin{trivlist}\item[\hskip \labelsep\bfseries {\Large Instructor Notes: \\} \hspace{\textwidth} ]}
{}
}
{\end{trivlist}}
\fi


%% Suggested Timing
\newcommand{\timing}[1]{{\bf Suggested Timing: \hspace{2ex}} #1}


\title{Comparing Decimals}
\author{Vic Ferdinand, Betsy McNeal, Jenny Sheldon}

\begin{document}
\begin{abstract}\end{abstract}
\maketitle



\begin{problem}
 Reorganize the following list of decimal numbers in order from least to greatest:\\

\begin{center}
    1.002, 0.2345, 1.2,  0.234, 1.21
\end{center}

\end{problem}


\begin{problem}
For each of the following statements, what is the minimal amount of correct understanding that the child must have to make the given statement?  Is their statement accurate for all decimal numbers?
\begin{enumerate}
    

\item Shaun says that 3432 is greater than 578 because ``the longest number is always the biggest".

\vfill




\item Janie says that 3432 is a smaller amount than 3451 because ``the first two numbers are the same and then 3432 has the smaller next number".
\vfill

\item Mary Jane says that 34.52  is greater than 3.451 because ``the first two numbers are the same and then 3.451 has the smaller next number".
\vfill

\item Caroline says that 0.0635  is a smaller number than  0.0098347 because ``0.0635 has fewer zeroes than 0.00983467".
\vfill

\item Sally says that 0.5  represents a smaller amount than 0.$\overline{3}$  (i.e., ``0.3 repeating forever") because ``0.$\overline{3}$ goes on forever", but Sam says that 0.5 represents a larger amount than 0.$\overline{3}$ because ``0.$\overline{3}$ has all threes (no digit bigger than a 5)".  Do you agree with either of these children?
\vfill

\end{enumerate}

\end{problem}

\newpage
\begin{instructorNotes}
A repeated theme of our course is that, without an emphasis on reasoning, school math can devolve into rules that are not based on the meanings of numbers and operations.  This handout, based on research on children's common errors, was developed to illustrate this point.  We have students examine each line from a child's point of view - what would be the most basic rule/pattern they could use to correctly decide which number is greater? When would that rule fail? Our goal is to consider why children make errors and how we can help them develop rules that reflect the true meaning of our decimal system and will work for any number.  We have found that physical representations like blocks or strip diagrams can help here.


{\bf Suggested Timing:} We give students 10 minutes in groups, and then spend 10 minutes in whole class discussion.  Working the first and second examples as a class generally makes this activity quite clear for the students.
\end{instructorNotes}


\end{document}