\documentclass{ximera}


\graphicspath{
  {./}
  {graphics/}
  {../graphics/}
}

\usepackage{chngcntr}

\let\question\relax
\let\endquestion\relax




\newtheoremstyle{SlantTheorem}{\topsep}{\fill}%%% space between body and thm
%\newtheoremstyle{SlantTheorem}{\topsep}{\topsep}%%% space between body and thm
 {\slshape}                      %%% Thm body font
 {}                              %%% Indent amount (empty = no indent)
 {\bfseries\sffamily}            %%% Thm head font
 {}                              %%% Punctuation after thm head
 {3ex}                           %%% Space after thm head
 {\thmname{#1}\thmnumber{ #2}\thmnote{ \bfseries(#3)}}%%% Thm head spec
\theoremstyle{SlantTheorem}
\newtheorem{question}{Question}
\counterwithin*{question}{section}



\let\instructorNotes\relax
\let\endinstructorNotes\relax
%%% instructorNotes environment
\ifhandout
\newenvironment{instructorNotes}[1][false]%
{%
\def\givenatend{\boolean{#1}}\ifthenelse{\boolean{#1}}{\begin{trivlist}\item}{\setbox0\vbox\bgroup}{}
}
{%
\ifthenelse{\givenatend}{\end{trivlist}}{\egroup}{}
}
\else
\newenvironment{instructorNotes}[1][false]%
{%
  \ifthenelse{\boolean{#1}}{\begin{trivlist}\item[\hskip \labelsep\bfseries {\Large Instructor Notes: \\} \hspace{\textwidth} ]}
{\begin{trivlist}\item[\hskip \labelsep\bfseries {\Large Instructor Notes: \\} \hspace{\textwidth} ]}
{}
}
{\end{trivlist}}
\fi


%% Suggested Timing
\newcommand{\timing}[1]{{\bf Suggested Timing: \hspace{2ex}} #1}


\title{Comparing Decimals}
\author{Vic Ferdinand, Betsy McNeal, Jenny Sheldon}

\begin{document}
\begin{abstract}\end{abstract}
\maketitle

\begin{instructorIntro}
A repeated theme of this course is that school math can devolve into rules that often are not based on the meaning of numbers and operations.  This handout was developed to illustrate that point (and is based on research on children's errors).  Each line should be examined from a child's point of view - what would be the simplest rule they could use to correctly decide which number is greater? Then, when would that rule fail? Our goal is to develop rules that will work for any numbers.  Don't forget that physical representations like blocks or strip diagrams can help here.


{\bf Suggested Timing:} 10 minutes in groups, 10 minutes in whole class discussion.  Working the first and second examples as a class generally makes this activity quite clear for the students.
\end{instructorIntro}

\begin{question}
 Reorganize the following list of decimal numbers in order from least to greatest:\\

\begin{center}
    1.002, 0.2345, 1.2,  0.234, 1.21
\end{center}

\end{question}


\begin{question}
For each of the following statements, what is the minimal amount of correct understanding that the child must have to make the given statement?  Is their statement accurate for all decimal numbers?
\begin{enumerate}
    

\item Shaun says that 3432 is greater than 578 because ``the longest number is always the biggest".

\vfill
%Some students were asked to decide which number in each of the following pairsone represents the larger quantity, assuming the same unit for all examples (e.g., gallons).  Describe a minimal "rule" that you could use to make your determination.  The first one is done for you.  For each subsequent pair, describe why some of the ``minimal rules" you have used before are now inadequate or even deceiving for making the decision about the new pair.  Also, think about how you might use base ten blocks or toothpicks to convince students of these ``rules".  (Hint:  Think of one-to-one correspondence.)\\



\item Janie says that 3432 is a smaller amount than 3451 because ``the first two numbers are the same and then 3432 has the smaller next number".
\vfill

\item Mary Jane says that 34.52  is greater than 3.451 because ``the first two numbers are the same and then 3.451 has the smaller next number".
\vfill

\item Caroline says that 0.0635  is a smaller number than  0.0098347 because ``0.0635 has fewer zeroes than 0.00983467".
\vfill

\item Sally says that 0.5  represents a smaller amount than 0.$\overline{3}$  (i.e., ``0.3 repeating forever") because ``0.$\overline{3}$ goes on forever", but Sam says that 0.5 represents a larger amount than 0.$\overline{3}$ because ``0.$\overline{3}$ has all threes (no digit bigger than a 5)".  Do you agree with either of these children?
\vfill

\end{enumerate}

\end{question}



\end{document}