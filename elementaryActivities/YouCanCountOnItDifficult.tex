\documentclass[nooutcomes]{ximera}
\usepackage{gensymb}
\usepackage{tabularx}
\usepackage{mdframed}
\usepackage{pdfpages}
%\usepackage{chngcntr}

\let\problem\relax
\let\endproblem\relax

\newcommand{\property}[2]{#1#2}




\newtheoremstyle{SlantTheorem}{\topsep}{\fill}%%% space between body and thm
 {\slshape}                      %%% Thm body font
 {}                              %%% Indent amount (empty = no indent)
 {\bfseries\sffamily}            %%% Thm head font
 {}                              %%% Punctuation after thm head
 {3ex}                           %%% Space after thm head
 {\thmname{#1}\thmnumber{ #2}\thmnote{ \bfseries(#3)}} %%% Thm head spec
\theoremstyle{SlantTheorem}
\newtheorem{problem}{Problem}[]

%\counterwithin*{problem}{section}



%%%%%%%%%%%%%%%%%%%%%%%%%%%%Jenny's code%%%%%%%%%%%%%%%%%%%%

%%% Solution environment
%\newenvironment{solution}{
%\ifhandout\setbox0\vbox\bgroup\else
%\begin{trivlist}\item[\hskip \labelsep\small\itshape\bfseries Solution\hspace{2ex}]
%\par\noindent\upshape\small
%\fi}
%{\ifhandout\egroup\else
%\end{trivlist}
%\fi}
%
%
%%% instructorIntro environment
%\ifhandout
%\newenvironment{instructorIntro}[1][false]%
%{%
%\def\givenatend{\boolean{#1}}\ifthenelse{\boolean{#1}}{\begin{trivlist}\item}{\setbox0\vbox\bgroup}{}
%}
%{%
%\ifthenelse{\givenatend}{\end{trivlist}}{\egroup}{}
%}
%\else
%\newenvironment{instructorIntro}[1][false]%
%{%
%  \ifthenelse{\boolean{#1}}{\begin{trivlist}\item[\hskip \labelsep\bfseries Instructor Notes:\hspace{2ex}]}
%{\begin{trivlist}\item[\hskip \labelsep\bfseries Instructor Notes:\hspace{2ex}]}
%{}
%}
%% %% line at the bottom} 
%{\end{trivlist}\par\addvspace{.5ex}\nobreak\noindent\hung} 
%\fi
%
%


\let\instructorNotes\relax
\let\endinstructorNotes\relax
%%% instructorNotes environment
\ifhandout
\newenvironment{instructorNotes}[1][false]%
{%
\def\givenatend{\boolean{#1}}\ifthenelse{\boolean{#1}}{\begin{trivlist}\item}{\setbox0\vbox\bgroup}{}
}
{%
\ifthenelse{\givenatend}{\end{trivlist}}{\egroup}{}
}
\else
\newenvironment{instructorNotes}[1][false]%
{%
  \ifthenelse{\boolean{#1}}{\begin{trivlist}\item[\hskip \labelsep\bfseries {\Large Instructor Notes: \\} \hspace{\textwidth} ]}
{\begin{trivlist}\item[\hskip \labelsep\bfseries {\Large Instructor Notes: \\} \hspace{\textwidth} ]}
{}
}
{\end{trivlist}}
\fi


%% Suggested Timing
\newcommand{\timing}[1]{{\bf Suggested Timing: \hspace{2ex}} #1}




\hypersetup{
    colorlinks=true,       % false: boxed links; true: colored links
    linkcolor=blue,          % color of internal links (change box color with linkbordercolor)
    citecolor=green,        % color of links to bibliography
    filecolor=magenta,      % color of file links
    urlcolor=cyan           % color of external links
}
\title{You Can Count On It Getting More Difficult}
\author{Vic Ferdinand, Betsy McNeal, Jenny Sheldon}

\begin{document}
\begin{abstract}
\end{abstract}
\maketitle





\begin{problem}
You choose a card from a deck consisting of 10 distinct cards, and then you roll a six-sided die.  How many different outcomes are possible?
\end{problem}

\begin{problem}
You choose two cards at the same time from a deck consisting of 10 distinct cards, and then you draw three marbles at the same time from a bag consisting of five different marbles.  How many different outcomes are possible?
\begin{enumerate}
\item Before you solve this problem, list at least three potential outcomes.
\item Have you seen any parts of this problem before?
\end{enumerate}
\vfill
\end{problem}

\begin{problem}
A class has 10 children, three of which are named Jay.  How many ways are there to choose a group of three children containing at least one person named Jay?
\end{problem}

\newpage

\begin{instructorNotes}
This activity consists of more complicated counting problems.  We use these activities about counting and probability to help cement the meaning of various operations for students.  Throughout these activities, we expect students to justify their operations of choice in their explanations.  There are a few main types of counting strategies we see from our students: arrays, ordered lists, tree diagrams, and algebraic expressions.  Students tend to want to use more complicated strategies before trying simpler strategies, so we are often reminding them to write down some examples, or to make sure their list is well organized.  We also continue to emphasize structure in this activity, as students are encouraged to look for similarities between problems.


Another of our goals in these counting activities is to slowly build up the difficulty of the problems. We would like students to build confidence in their algebraic solutions with simpler problems which they can check by listing all possibilities, and then leverage their knowledge as the problems become more complicated.  This activity is the first where students encounter problems which cannot reasonably be solved by listing and counting (though some students in the past have been very determined to try).  

We expect our students to be comfortable at this point with comparing and contrasting problems they are trying to solve with problems they have already solved.  This activity continues to emphasize this practice of identifying the structure of a problem.

This activity also introduces counting two separate events using multiplication.  We take significant time to review the groups and objects per group for the multiplication.

\timing{We allot one class period for this activity, but don't usually finish all of the problems in that amount of time.}

\end{instructorNotes}



\end{document}