\documentclass{ximera}
\usepackage{gensymb}
\usepackage{tabularx}
\usepackage{mdframed}
\usepackage{pdfpages}
%\usepackage{chngcntr}

\let\problem\relax
\let\endproblem\relax

\newcommand{\property}[2]{#1#2}




\newtheoremstyle{SlantTheorem}{\topsep}{\fill}%%% space between body and thm
 {\slshape}                      %%% Thm body font
 {}                              %%% Indent amount (empty = no indent)
 {\bfseries\sffamily}            %%% Thm head font
 {}                              %%% Punctuation after thm head
 {3ex}                           %%% Space after thm head
 {\thmname{#1}\thmnumber{ #2}\thmnote{ \bfseries(#3)}} %%% Thm head spec
\theoremstyle{SlantTheorem}
\newtheorem{problem}{Problem}[]

%\counterwithin*{problem}{section}



%%%%%%%%%%%%%%%%%%%%%%%%%%%%Jenny's code%%%%%%%%%%%%%%%%%%%%

%%% Solution environment
%\newenvironment{solution}{
%\ifhandout\setbox0\vbox\bgroup\else
%\begin{trivlist}\item[\hskip \labelsep\small\itshape\bfseries Solution\hspace{2ex}]
%\par\noindent\upshape\small
%\fi}
%{\ifhandout\egroup\else
%\end{trivlist}
%\fi}
%
%
%%% instructorIntro environment
%\ifhandout
%\newenvironment{instructorIntro}[1][false]%
%{%
%\def\givenatend{\boolean{#1}}\ifthenelse{\boolean{#1}}{\begin{trivlist}\item}{\setbox0\vbox\bgroup}{}
%}
%{%
%\ifthenelse{\givenatend}{\end{trivlist}}{\egroup}{}
%}
%\else
%\newenvironment{instructorIntro}[1][false]%
%{%
%  \ifthenelse{\boolean{#1}}{\begin{trivlist}\item[\hskip \labelsep\bfseries Instructor Notes:\hspace{2ex}]}
%{\begin{trivlist}\item[\hskip \labelsep\bfseries Instructor Notes:\hspace{2ex}]}
%{}
%}
%% %% line at the bottom} 
%{\end{trivlist}\par\addvspace{.5ex}\nobreak\noindent\hung} 
%\fi
%
%


\let\instructorNotes\relax
\let\endinstructorNotes\relax
%%% instructorNotes environment
\ifhandout
\newenvironment{instructorNotes}[1][false]%
{%
\def\givenatend{\boolean{#1}}\ifthenelse{\boolean{#1}}{\begin{trivlist}\item}{\setbox0\vbox\bgroup}{}
}
{%
\ifthenelse{\givenatend}{\end{trivlist}}{\egroup}{}
}
\else
\newenvironment{instructorNotes}[1][false]%
{%
  \ifthenelse{\boolean{#1}}{\begin{trivlist}\item[\hskip \labelsep\bfseries {\Large Instructor Notes: \\} \hspace{\textwidth} ]}
{\begin{trivlist}\item[\hskip \labelsep\bfseries {\Large Instructor Notes: \\} \hspace{\textwidth} ]}
{}
}
{\end{trivlist}}
\fi


%% Suggested Timing
\newcommand{\timing}[1]{{\bf Suggested Timing: \hspace{2ex}} #1}




\hypersetup{
    colorlinks=true,       % false: boxed links; true: colored links
    linkcolor=blue,          % color of internal links (change box color with linkbordercolor)
    citecolor=green,        % color of links to bibliography
    filecolor=magenta,      % color of file links
    urlcolor=cyan           % color of external links
}

\title{Mental Math with Addition}

\begin{document}
\begin{abstract} We practice adding and subtracting without pencil and paper. \end{abstract}
\maketitle


\begin{problem}
For each of the following, find the total without writing anything down, and without using a calculator or other tool. You can write down your answer so that you don't forget it!

\begin{enumerate}
\item $499 + 58 + 1$
\item $397 + 23$
\item $158 + 239$
\item $205 - 18$
\item $302 - 57$
\end{enumerate}

\end{problem}


%\begin{problem}
%Choose at least three of your solutions above and write down a description of your solution method in words. Discuss your answers with your group. Did anyone use the same methods?
%\end{problem}


\begin{problem}
As a group, illustrate your solutions to each problem using a picture. Your pictures can include objects, number lines, or whatever else makes sense to you! Try using different types of pictures for each problem.
\end{problem}


\begin{problem}
As a group, illustrate your solutions to each problem using equations. Be sure to place the equals sign only between things which are actually equal! Did you use any properties of arithmetic in your equations? If so, what were they, and where did you use them? Be as specific as you can.

\end{problem}


\begin{problem}
Mel solves $302 - 57$ in the following way.

\emph{I'm going to start by shifting this problem from $302 - 57$ to be $300 - 55$. This will have the same answer as my original problem because I took away $2$ from both numbers. Now I take $300 - 50$ and get $250$. Then I take away $5$ more and get $245$, which is my answer.}

Why does Mel's strategy give us the correct answer? Draw pictures or write equations to help you understand this strategy. Did Mel use any properties of arithmetic in this strategy?  Explain your thinking.
\end{problem}



\newpage
\begin{instructorNotes}

{\bf Main Goal:} Our main goal here is to begin to develop flexibility with mental calculations. We have several secondary goals, including writing equations to describe a mental method and recognizing the properties of arithmetic.


{\bf Overall Picture:}

\begin{itemize}
\item You may want to work through the first two problems as a class, one at a time. Write one of the problems on the board, give students a few minutes to think, and then ask for 3-5 volunteers to talk through their solutions. Consider asking students to give a thumbs-up (visible only to the instructor) when they are ready, much like with \link[Number Talks]{https://www.insidemathematics.org/classroom-videos/number-talks/3rd-grade-math-one-digit-by-two-digit-multiplication} (see the ``Part 1'' video linked for an example).
\item If you are short on time or want to move through this first part more quickly, you can eliminate parts (b) and (d).
\item Have students present as many types of drawings and equations as you see while you walk around the room. The equations need not be a string of equations, but do need to represent the student's thinking. If the student does write a string of equations, watch for the type where the equals sign means ``operate'' rather than ``equal'', for instance $2 + 7 = 9 + 1 = 10$ when solving $2 + 7 + 1$.
\item If we have not yet discussed the commutative and associative properties of addition, you have a nice opportunity here to ask students why (using the meaning of addition) these ideas make sense.

\end{itemize}



%{\bf Good Language:}



{\bf Suggested Timing:} This activity should take most of the class period. If you are working together for the first part, spend about 20 minutes here, then give students about 10 minutes to draw and write equations, and have groups present for the final 20 minutes. If the students are working in groups for the first part, encourage silence for about five minutes, then give groups about 15 minutes to discuss and progress through the activity. Then, use the last 25 minutes in whole-class discussion.

\end{instructorNotes}


\end{document}