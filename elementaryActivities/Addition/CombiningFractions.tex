%\documentclass{ximera}
\documentclass[nooutcomes,noauthor, handout]{ximera}
\usepackage{gensymb}
\usepackage{tabularx}
\usepackage{mdframed}
\usepackage{pdfpages}
%\usepackage{chngcntr}

\let\problem\relax
\let\endproblem\relax

\newcommand{\property}[2]{#1#2}




\newtheoremstyle{SlantTheorem}{\topsep}{\fill}%%% space between body and thm
 {\slshape}                      %%% Thm body font
 {}                              %%% Indent amount (empty = no indent)
 {\bfseries\sffamily}            %%% Thm head font
 {}                              %%% Punctuation after thm head
 {3ex}                           %%% Space after thm head
 {\thmname{#1}\thmnumber{ #2}\thmnote{ \bfseries(#3)}} %%% Thm head spec
\theoremstyle{SlantTheorem}
\newtheorem{problem}{Problem}[]

%\counterwithin*{problem}{section}



%%%%%%%%%%%%%%%%%%%%%%%%%%%%Jenny's code%%%%%%%%%%%%%%%%%%%%

%%% Solution environment
%\newenvironment{solution}{
%\ifhandout\setbox0\vbox\bgroup\else
%\begin{trivlist}\item[\hskip \labelsep\small\itshape\bfseries Solution\hspace{2ex}]
%\par\noindent\upshape\small
%\fi}
%{\ifhandout\egroup\else
%\end{trivlist}
%\fi}
%
%
%%% instructorIntro environment
%\ifhandout
%\newenvironment{instructorIntro}[1][false]%
%{%
%\def\givenatend{\boolean{#1}}\ifthenelse{\boolean{#1}}{\begin{trivlist}\item}{\setbox0\vbox\bgroup}{}
%}
%{%
%\ifthenelse{\givenatend}{\end{trivlist}}{\egroup}{}
%}
%\else
%\newenvironment{instructorIntro}[1][false]%
%{%
%  \ifthenelse{\boolean{#1}}{\begin{trivlist}\item[\hskip \labelsep\bfseries Instructor Notes:\hspace{2ex}]}
%{\begin{trivlist}\item[\hskip \labelsep\bfseries Instructor Notes:\hspace{2ex}]}
%{}
%}
%% %% line at the bottom} 
%{\end{trivlist}\par\addvspace{.5ex}\nobreak\noindent\hung} 
%\fi
%
%


\let\instructorNotes\relax
\let\endinstructorNotes\relax
%%% instructorNotes environment
\ifhandout
\newenvironment{instructorNotes}[1][false]%
{%
\def\givenatend{\boolean{#1}}\ifthenelse{\boolean{#1}}{\begin{trivlist}\item}{\setbox0\vbox\bgroup}{}
}
{%
\ifthenelse{\givenatend}{\end{trivlist}}{\egroup}{}
}
\else
\newenvironment{instructorNotes}[1][false]%
{%
  \ifthenelse{\boolean{#1}}{\begin{trivlist}\item[\hskip \labelsep\bfseries {\Large Instructor Notes: \\} \hspace{\textwidth} ]}
{\begin{trivlist}\item[\hskip \labelsep\bfseries {\Large Instructor Notes: \\} \hspace{\textwidth} ]}
{}
}
{\end{trivlist}}
\fi


%% Suggested Timing
\newcommand{\timing}[1]{{\bf Suggested Timing: \hspace{2ex}} #1}




\hypersetup{
    colorlinks=true,       % false: boxed links; true: colored links
    linkcolor=blue,          % color of internal links (change box color with linkbordercolor)
    citecolor=green,        % color of links to bibliography
    filecolor=magenta,      % color of file links
    urlcolor=cyan           % color of external links
}
\title{Combining fractions}

\begin{document}
\begin{abstract}
\end{abstract}

\maketitle



\begin{problem}
At a coffee shop, Kinfe makes a drink by combining $\frac{2}{5}$ of a cup of coffee with $\frac{3}{4}$ of a cup of milk. How many cups of coffee and milk are in this drink all together?

\begin{enumerate}
	\item Is this an addition problem for $\frac{2}{5} + \frac{3}{4}$? Explain how you know.  If this is not a story problem for $\frac{2}{5} + \frac{3}{4}$, explain how you would adjust the existing story problem so that it is a problem for $\frac{2}{5} + \frac{3}{4}$.
	\item Solve the problem for $\frac{2}{5} + \frac{3}{4}$ using a picture. Be sure to explain why your solution makes sense using the meaning of addition and the meaning of fractions.
\end{enumerate}
\end{problem}



\begin{problem}
At a smoothie shop, Ileana makes a drink by combining $\frac{3}{5}$ of a cup of strawberries with $\frac{4}{7}$ of an ounce of bananas. How many cups of strawberries and bananas are in this drink all together?

\begin{enumerate}
	\item Is this an addition problem for $\frac{3}{5} + \frac{4}{7}$? Explain how you know.  If this is not a story problem for $\frac{3}{5} + \frac{4}{7}$, explain how you would adjust the existing story problem so that it is a problem for $\frac{3}{5} + \frac{4}{7}$.
	\item Solve the problem for $\frac{3}{5} + \frac{4}{7}$ using a picture. Be sure to explain why your solution makes sense using the meaning of addition and the meaning of fractions.
\end{enumerate}
\end{problem}




\begin{problem}
At a tea shop, Joey must remove $\frac{1}{4}$ of the liquid from a bowl that contains $\frac{5}{8}$ of a gallon of water. How much liquid will Joey remove from the bowl?

\begin{enumerate}
	\item Is this a subtraction problem for $\frac{5}{8} - \frac{1}{4}$? Explain how you know.  If this is not a story problem for  $\frac{5}{8} - \frac{1}{4}$, explain how you would adjust the existing story problem so that it is a problem for  $\frac{5}{8} - \frac{1}{4}$.
	\item Solve the problem for  $\frac{5}{8} - \frac{1}{4}$ using a picture. Be sure to explain why your solution makes sense using the meaning of subtraction and the meaning of fractions.
\end{enumerate}
\end{problem}




\begin{problem}
At a milkshake shop, Lara has two gallons of ice cream in front of her. She has sold $\frac{5}{7}$ of the first gallon and $\frac{2}{9}$ of the second gallon. How much of the total two gallons of ice cream has Lara sold?

\begin{enumerate}
	\item Is this a subtraction problem for $\frac{5}{7} - \frac{2}{9}$? Explain how you know.  If this is not a story problem for  $\frac{5}{7} - \frac{2}{9}$, explain how you would adjust the existing story problem so that it is a problem for  $\frac{5}{7} - \frac{2}{9}$.
	\item Solve the problem for  $\frac{5}{7} - \frac{2}{9}$ using a picture. Be sure to explain why your solution makes sense using the meaning of subtraction and the meaning of fractions.
\end{enumerate}
\end{problem}


\begin{problem}
Based on our work in this activity, why does it make sense to make common denominators when adding or subtracting fractions?
\end{problem}



\newpage

\begin{instructorNotes} 



{\bf Main goal:} Understand the role that wholes play in addition and subtraction stories with fractions and explain why we make common denominators when adding and subtracting fractions.

{\bf Overall picture:}

Each problem gives students a chance to think about both of our goals for this activity. The wording in the second part might be confusing: we want students to practice with whatever problem they find for addition or subtraction. This may or may not be the original problem in the story!

\begin{itemize}
	\item To pay attention to the role of the wholes, the first problem is a problem for the suggested addition expression. The second problem does not have the same wholes for the addends, the third problem does not have the same wholes for the minuend and the subtrahend, and the final problem does not have the same whole for the answer as the minuend and subtrahend. The fourth problem should be the most challenging. Overall, we are looking for the observation that all three wholes must be the same when we consider story problems for addition and subtraction. Just a small change in wording can make a huge difference in the word problem! This is a particularly useful observation for them as future teachers.
	\item Students will have different answers for how they adjust the problems to fit the given expression. It would be good to have groups with different story types present so that we can keep practicing with identifying the type of the story as well as remind ourselves that different types can have different-looking solution strategies.
	\item Have groups present their solutions for the addition and subtraction problems. If you can keep all of the solutions on the board at once, it would be good to compare and contrast when we consider the final problem about making common denominators. Our goal is to notice that it's easier to tell what fraction we have as our answer if the pieces are all the same size. 
	\item The issue that two fractions may have overlapping wholes does not appear in these problems, but is a good challenge problem or extension for groups who are finished quickly (or a homework problem).
\end{itemize}




{\bf Good language:}  We extend our meanings of addition and subtraction to include fractions. It's good to refer back to that meaning throughout, even though the more complicated part of this activity is what's happening with the various wholes. Before jumping in to the discussion of the problem, ask students to identify whether things are being combined, taken away, compared, etc, so that we practice with these ideas!

Students who aren't used to always stating the whole with their fraction might struggle with this activity! Be sure to emphasize the wholes throughout and encourage students to draw pictures to help them think about what is happening.




{\bf Suggested timing:} Give students about 15 minutes to work on the first four problems, and then have groups present. Point out the overall ideas about the complexities with the whole as a wrap up. Then give students about 5 minutes to think about all of their solutions overall (problem 5) and use any remaining time to discuss.


\end{instructorNotes}



\end{document}