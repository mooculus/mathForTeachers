%\documentclass{ximera}
\documentclass[nooutcomes,noauthor]{ximera}

\graphicspath{
  {./}
  {graphics/}
  {../graphics/}
}

\usepackage{chngcntr}

\let\question\relax
\let\endquestion\relax




\newtheoremstyle{SlantTheorem}{\topsep}{\fill}%%% space between body and thm
%\newtheoremstyle{SlantTheorem}{\topsep}{\topsep}%%% space between body and thm
 {\slshape}                      %%% Thm body font
 {}                              %%% Indent amount (empty = no indent)
 {\bfseries\sffamily}            %%% Thm head font
 {}                              %%% Punctuation after thm head
 {3ex}                           %%% Space after thm head
 {\thmname{#1}\thmnumber{ #2}\thmnote{ \bfseries(#3)}}%%% Thm head spec
\theoremstyle{SlantTheorem}
\newtheorem{question}{Question}
\counterwithin*{question}{section}



\let\instructorNotes\relax
\let\endinstructorNotes\relax
%%% instructorNotes environment
\ifhandout
\newenvironment{instructorNotes}[1][false]%
{%
\def\givenatend{\boolean{#1}}\ifthenelse{\boolean{#1}}{\begin{trivlist}\item}{\setbox0\vbox\bgroup}{}
}
{%
\ifthenelse{\givenatend}{\end{trivlist}}{\egroup}{}
}
\else
\newenvironment{instructorNotes}[1][false]%
{%
  \ifthenelse{\boolean{#1}}{\begin{trivlist}\item[\hskip \labelsep\bfseries {\Large Instructor Notes: \\} \hspace{\textwidth} ]}
{\begin{trivlist}\item[\hskip \labelsep\bfseries {\Large Instructor Notes: \\} \hspace{\textwidth} ]}
{}
}
{\end{trivlist}}
\fi


%% Suggested Timing
\newcommand{\timing}[1]{{\bf Suggested Timing: \hspace{2ex}} #1}
\title{Standard Subtraction}

\begin{document}
\begin{abstract}
\end{abstract}

\maketitle

\begin{problem}
Show how to use base ten blocks or sticks and bundles to solve the subtraction problem $231 - 48$. Feel free to use your creativity here -- any correct picture or sequence of pictures is suitable for this problem.
\end{problem}


\begin{problem}
Did you use more of a take-away strategy or more of a comparison strategy in your picture above? How can you tell?
\end{problem}


\begin{problem}
Use the standard subtraction algorithm to solve $231 - 48$. Do the steps in your picture correspond to the steps in the standard algorithm? How can you tell?
\end{problem}


\begin{problem}
Show how to use base ten blocks or sticks and bundles to solve the subtraction problem $104 - 26$. This time, draw a sequence of pictures that correspond to the steps in the standard algorithm. Explain how each step in the algorithm is reflected in one of your pictures. How do we see the final answer in your picture, and how do we know it is correct?
\end{problem}



\newpage

\begin{instructorNotes} 



{\bf Main goal:} We understand why the standard subtraction algorithm makes sense with our ideas of place value and our meaning of subtraction.


{\bf Overall picture:} There are several important ideas to highlight in your discussion.

\begin{itemize}
	\item Students should pay attention to which type of story problem they are using in their drawings. In general, if there is no story problem associated with a subtraction, we would like students to draw a ``take-away'' picture. If you see ``comparison'' pictures (usually identified by the presence of both the minuend and the subtrahend in the drawing), you might point out that this is appropriate with a comparison story, but not otherwise. The second problem should give you an opportunity to discuss this directly.
	\item Drawings that represent the algorithm should probably look a bit different than the first round. It's natural for many people to work with the largest objects first, considering the subtraction left to right, and then making adjustments as needed as they solve. But the standard algorithm begins from the smallest objects so that this ``going back'' stage is unnecessary.
	\item Each step in the algorithm should be explained in terms of the meaning of place value and/or the meaning of subtraction, and each step should be represented in the drawing. Students may find it helpful to make more than one drawing to represent these ideas. For instance, when we look at $6-4$ and decide ``it can't be done'' (actually, the answer is $-2$!), what are we seeing in our picture? When we cross off the $0$ and write a $10$ or a $9$, why are we doing this? How are we representing it in our picture? After a student discusses their diagram, it's good to go back  and comb through each step individually. Check out the textbook for more information.
	%\item The expanded forms may help some students to see what's happening, but might be more confusing for others. It's good in your discussion to connect the expanded forms to the bundling drawings, so that students can connect these two ideas. There are also several examples using expanded forms in the text. We won't emphasize this content (we will focus on the bundling pictures).
	\item Be sure to suggest that students also practice with the addition algorithm, which is very similar to the subtraction algorithm (but we bundle instead of unbundling).
	\item Wrap up by indicating the importance of both understanding the meaning of the operation (helping you to know when the operation is appropriate, as well as helping you to calculate flexibly) as well as being able to use the standard algorithm (giving you speed with your calculations). 
\end{itemize}

Note: we want to follow up this activity by asking students why we add and subtract decimal numbers the way that we do.


{\bf Good language:}  Examine your own way of describing the algorithm before working through this activity. We want to stop after essentially each statement we make and determine why that statement works.


{\bf Suggested timing:} This activity should take the whole class period. Give students about 5-8 minutes to work through their initial drawings and their assessment of their strategy (problems 1 - 3) and then discuss. Have students present a variety of drawings so that you can really compare and contrast a take away drawing from a comparison drawing, and a drawing that shows the steps of the algorithm from one that does not. If you don't have any drawings that show the steps of the algorithm, just discuss why this is the case. What is missing, and how might we adjust our drawings in order to include that information? Be sure to have a student demonstrate the standard algorithm as part of this discussion in case students have forgotten this method.

Next, give students about 5-8 minutes to work on the final problem. Have one group present. Wrap up by reiterating the kind of diagrams we are looking for and how we want to show the steps of the algorithm in our pictures.


If you have extra time, add a question about the decimal addition or subtraction algorithm (perhaps model $2.31 - 0.48$ or $23.1 - 0.48$), and give students about 5 minutes to think about this and then use the remaining time to present.


\end{instructorNotes}



\end{document}