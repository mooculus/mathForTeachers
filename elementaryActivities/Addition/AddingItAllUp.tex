\documentclass{ximera}

\usepackage{gensymb}
\usepackage{tabularx}
\usepackage{mdframed}
\usepackage{pdfpages}
%\usepackage{chngcntr}

\let\problem\relax
\let\endproblem\relax

\newcommand{\property}[2]{#1#2}




\newtheoremstyle{SlantTheorem}{\topsep}{\fill}%%% space between body and thm
 {\slshape}                      %%% Thm body font
 {}                              %%% Indent amount (empty = no indent)
 {\bfseries\sffamily}            %%% Thm head font
 {}                              %%% Punctuation after thm head
 {3ex}                           %%% Space after thm head
 {\thmname{#1}\thmnumber{ #2}\thmnote{ \bfseries(#3)}} %%% Thm head spec
\theoremstyle{SlantTheorem}
\newtheorem{problem}{Problem}[]

%\counterwithin*{problem}{section}



%%%%%%%%%%%%%%%%%%%%%%%%%%%%Jenny's code%%%%%%%%%%%%%%%%%%%%

%%% Solution environment
%\newenvironment{solution}{
%\ifhandout\setbox0\vbox\bgroup\else
%\begin{trivlist}\item[\hskip \labelsep\small\itshape\bfseries Solution\hspace{2ex}]
%\par\noindent\upshape\small
%\fi}
%{\ifhandout\egroup\else
%\end{trivlist}
%\fi}
%
%
%%% instructorIntro environment
%\ifhandout
%\newenvironment{instructorIntro}[1][false]%
%{%
%\def\givenatend{\boolean{#1}}\ifthenelse{\boolean{#1}}{\begin{trivlist}\item}{\setbox0\vbox\bgroup}{}
%}
%{%
%\ifthenelse{\givenatend}{\end{trivlist}}{\egroup}{}
%}
%\else
%\newenvironment{instructorIntro}[1][false]%
%{%
%  \ifthenelse{\boolean{#1}}{\begin{trivlist}\item[\hskip \labelsep\bfseries Instructor Notes:\hspace{2ex}]}
%{\begin{trivlist}\item[\hskip \labelsep\bfseries Instructor Notes:\hspace{2ex}]}
%{}
%}
%% %% line at the bottom} 
%{\end{trivlist}\par\addvspace{.5ex}\nobreak\noindent\hung} 
%\fi
%
%


\let\instructorNotes\relax
\let\endinstructorNotes\relax
%%% instructorNotes environment
\ifhandout
\newenvironment{instructorNotes}[1][false]%
{%
\def\givenatend{\boolean{#1}}\ifthenelse{\boolean{#1}}{\begin{trivlist}\item}{\setbox0\vbox\bgroup}{}
}
{%
\ifthenelse{\givenatend}{\end{trivlist}}{\egroup}{}
}
\else
\newenvironment{instructorNotes}[1][false]%
{%
  \ifthenelse{\boolean{#1}}{\begin{trivlist}\item[\hskip \labelsep\bfseries {\Large Instructor Notes: \\} \hspace{\textwidth} ]}
{\begin{trivlist}\item[\hskip \labelsep\bfseries {\Large Instructor Notes: \\} \hspace{\textwidth} ]}
{}
}
{\end{trivlist}}
\fi


%% Suggested Timing
\newcommand{\timing}[1]{{\bf Suggested Timing: \hspace{2ex}} #1}




\hypersetup{
    colorlinks=true,       % false: boxed links; true: colored links
    linkcolor=blue,          % color of internal links (change box color with linkbordercolor)
    citecolor=green,        % color of links to bibliography
    filecolor=magenta,      % color of file links
    urlcolor=cyan           % color of external links
}


\title{Adding It All Up}
\author{Vic Ferdinand, Betsy McNeal, Jenny Sheldon}

\begin{document}
\begin{abstract} In this activity, we will explore the meanings of addition and subtraction. \end{abstract}
\maketitle



\begin{problem}
Imagine that you are a young student in Kindergarten or first grade.  If you had blocks or other objects, how would you solve the following problems?  What are the differences and similarities between them?
\begin{enumerate}
\item Johnny has 7 apples and Susie has 5 apples.  How many apples are there altogether?
\item Johnny has 7 apples.  Susie gives him 5 apples.  How many apples does Johnny have now?
\item Johnny has some apples.  He gives 7 of the apples away to Susie and now has 5 apples.  How many apples did Johnny start with?
\item Johnny has 7 more apples than Susie.  If Susie has 5 apples, how many apples does Johnny have?
\item How would these problems change if the numbers 7 and 5 were replaced with 7264 and 567?
\end{enumerate}
\end{problem}

\begin{problem}
What does it mean to ``add'' two numbers? What action or actions are we taking when we add? How do we recognize addition?
\end{problem}

\begin{problem}
 Thinking about how you would use the blocks to solve the following problems, what are the differences and similarities between them?
\begin{enumerate}
\item Altogether Johnny and Susie have 12 apples.  If Johnny has 7 apples, how many apples does Susie have?
\item Johnny has 7 apples.  Susie gives him some apples and Johnny now has 12 apples.  How many apples did Susie give Johnny?
\item Johnny has 12 apples.  He gives 7 of the apples away to Susie.  How many apples does Johnny have now?
\item Johnny has 12 apples and Susie has 7 apples.  How many more apples does Johnny have than Susie?
\end{enumerate}
\end{problem}

\begin{problem}
What does it mean to ``subtract'' two numbers? What action or actions are we taking when we subtract? How do we recognize subtraction?
\end{problem}

\newpage

%\begin{problem}
%There are 60 people in the band and 40 people in the choir.  If everyone from each group (and no one else) shows up at the annual band-choir party, how many people will be there?
%\end{problem}


\begin{problem}
There are three different types of addition and subtraction problems. The three types are:
\begin{itemize}
\item Put together/take apart problems (two sets are joined or separated)
\item Change over time problems (the actions happen over time)
\item Comparison problems (the action of the problem is primarily comparing two sets)
\end{itemize}
\begin{enumerate}
	\item Classify each of the problems above. Which type are they, and why?
	\item Write your own example of each of the three types. How do you know your answers are correct?
\end{enumerate}
\end{problem}


%
%\begin{problem} \label{AddingUpSummary}
% Come up with a definition of addition.  That is, if $a$ and $b$ are numbers, what does $a + b$ mean?  Likewise, come up with a definition of ``take-away" subtraction.  Likewise, come up with a definition of ``missing-addend" subtraction.  
%\end{problem}


\newpage
\begin{instructorNotes}
This set of example problems was developed to draw out and illustrate the meaning and structure for addition and subtraction. In terms of the most basic meaning of addition, we would like students to recognize ``combining" and the basic meaning of subtraction as ``taking from''. 

When we have a context or story situation, we will follow the textbook's categories for structure:
\begin{itemize}
\item Put together/take apart problems
\item Add to/take from problems (change problems which usually include some sense of time)
\item Compare problems
\end{itemize}
We will not ask students to distinguish between addition and subtraction with these structures (again following the text), as any of them could be used for either operation.  However, we will ask students to match a structure to a story situation, and to match their solving actions to the structure they have in mind. When there is no story or context, students' models should be a ``put together/take apart'' type, as this is our most basic meaning. Keep in mind that our students have usually not encountered the idea of structure very much prior to our course, so this can be a challenge!

Students are asked to read each problem, consider how a child might solve it using blocks or objects, and discuss how those actions help us decide whether a story problem shows the addition operation, the subtraction operation, or both.  In whole-class discussion, we define the three structures and discuss how students could recognize these structures in both stories as well as pictures. Students enjoy trying to imagine how children would use objects to act out the problems and are very willing to take on that role themselves.


\begin{itemize}
	\item In our course, this activity follows extensive work on the meaning of numbers (whole numbers, fractions, decimals) without reference to operations.  This activity thus marks a change of direction -- we are now focusing on binary operations. Before, especially with fractions, we dealt only with the meanings, notation, and comparison of quantities.  Now we are operating on them. This is a BIG difference.
	\item We found students to be unfamiliar with the idea of thinking about ``mathematical structure'', so this activity was designed to help our future teachers uncover the differences among the story problems.  Because they already easily recognize addition and subtraction as adults, we have our students think about how the actions very young children would take to solve these problems with blocks mimic the structure of the problems. This seems to help our students make these differences explicit in terms of problem structure.  We use the word ``model'' to refer to these structures.
	%\item Our main model of addition is the {\em joining} of two sets, which children will first do by pushing blocks together and then counting the result.  {\em Missing-addend} situations are also a process of ``joining'', it is just a matter of how they are joined.  Here, we start with one set and another set gets joined to it.
	%\item The 3 main models of subtraction are:  {\em take-away} (traditional:  start with one set, then remove a subset), {\em missing addend} (start with one set and a goal set, figure out what must be joined to first set to get to the goal set), and {\em comparison} (start with 2 sets, attempt to make a 1-1 correspondence between the sets, then count what is left over).
	\item Each set of problems has the same 4 models in order (Put together/take apart, Add-To or Missing Addend, Take From, Comparison), with the subtraction examples having the same unknown to point out that the mathematics is the same no matter what numbers happen to be used (so that students don't think $2+3$ is different from $2098+347$, except in terms of calculation efficiency).
	\item We help students see that there are three (not two) numbers involved in each story problem and each one plays a certain role within the model (and any one of the three could serve as the ``unknown'').  Students should think about how to show each of the three numbers, as well as the action(s) taken, in a picture.  If the student is drawing a picture to solve a specific story, the student's picture should also match the type of the story. For instance, we don't want comparison pictures with put-together problems! This is very difficult but very important to point out to students.
	\item Also, we want to point out either here or later (possibly when we study multiplication) that the unit of each quantity is the same in addition and subtraction, in contrast to multiplication and division.
	\item Optional: You might note that where we have, for example, $7 + ? = 12$, it later becomes $7 + x = 12$.
\end{itemize}

{\bf Suggested Timing:} Give students about 5-10 minutes to work through their drawings on the first problem and perhaps think about the meaning of addition, and use about 10 minutes to have groups present. Follow-up on this with a discussion of the meaning of addition, driven by student observations. Repeat this sequence with the subtraction problems and the meaning of subtraction. Finally, give students about 5 minutes to think about the problem types and use the last 10 minutes to discuss this idea.
\end{instructorNotes}

\end{document}
