\documentclass{ximera}

\usepackage{gensymb}
\usepackage{tabularx}
\usepackage{mdframed}
\usepackage{pdfpages}
%\usepackage{chngcntr}

\let\problem\relax
\let\endproblem\relax

\newcommand{\property}[2]{#1#2}




\newtheoremstyle{SlantTheorem}{\topsep}{\fill}%%% space between body and thm
 {\slshape}                      %%% Thm body font
 {}                              %%% Indent amount (empty = no indent)
 {\bfseries\sffamily}            %%% Thm head font
 {}                              %%% Punctuation after thm head
 {3ex}                           %%% Space after thm head
 {\thmname{#1}\thmnumber{ #2}\thmnote{ \bfseries(#3)}} %%% Thm head spec
\theoremstyle{SlantTheorem}
\newtheorem{problem}{Problem}[]

%\counterwithin*{problem}{section}



%%%%%%%%%%%%%%%%%%%%%%%%%%%%Jenny's code%%%%%%%%%%%%%%%%%%%%

%%% Solution environment
%\newenvironment{solution}{
%\ifhandout\setbox0\vbox\bgroup\else
%\begin{trivlist}\item[\hskip \labelsep\small\itshape\bfseries Solution\hspace{2ex}]
%\par\noindent\upshape\small
%\fi}
%{\ifhandout\egroup\else
%\end{trivlist}
%\fi}
%
%
%%% instructorIntro environment
%\ifhandout
%\newenvironment{instructorIntro}[1][false]%
%{%
%\def\givenatend{\boolean{#1}}\ifthenelse{\boolean{#1}}{\begin{trivlist}\item}{\setbox0\vbox\bgroup}{}
%}
%{%
%\ifthenelse{\givenatend}{\end{trivlist}}{\egroup}{}
%}
%\else
%\newenvironment{instructorIntro}[1][false]%
%{%
%  \ifthenelse{\boolean{#1}}{\begin{trivlist}\item[\hskip \labelsep\bfseries Instructor Notes:\hspace{2ex}]}
%{\begin{trivlist}\item[\hskip \labelsep\bfseries Instructor Notes:\hspace{2ex}]}
%{}
%}
%% %% line at the bottom} 
%{\end{trivlist}\par\addvspace{.5ex}\nobreak\noindent\hung} 
%\fi
%
%


\let\instructorNotes\relax
\let\endinstructorNotes\relax
%%% instructorNotes environment
\ifhandout
\newenvironment{instructorNotes}[1][false]%
{%
\def\givenatend{\boolean{#1}}\ifthenelse{\boolean{#1}}{\begin{trivlist}\item}{\setbox0\vbox\bgroup}{}
}
{%
\ifthenelse{\givenatend}{\end{trivlist}}{\egroup}{}
}
\else
\newenvironment{instructorNotes}[1][false]%
{%
  \ifthenelse{\boolean{#1}}{\begin{trivlist}\item[\hskip \labelsep\bfseries {\Large Instructor Notes: \\} \hspace{\textwidth} ]}
{\begin{trivlist}\item[\hskip \labelsep\bfseries {\Large Instructor Notes: \\} \hspace{\textwidth} ]}
{}
}
{\end{trivlist}}
\fi


%% Suggested Timing
\newcommand{\timing}[1]{{\bf Suggested Timing: \hspace{2ex}} #1}




\hypersetup{
    colorlinks=true,       % false: boxed links; true: colored links
    linkcolor=blue,          % color of internal links (change box color with linkbordercolor)
    citecolor=green,        % color of links to bibliography
    filecolor=magenta,      % color of file links
    urlcolor=cyan           % color of external links
}


\title{Tell me a story!}
\author{Jenny Sheldon}

\begin{document}
\begin{abstract} In this activity, we will solve addition and subtraction problems using different strategies. \end{abstract}
\maketitle



\begin{problem}
Here is an example of a ``change over time'' story problem. 

\emph{Geraldine volunteers at the local cat shelter. On Monday, there are $18$ cats at the shelter. On Tuesday, $5$ more cats arrive at the shelter. How many cats are there now?}

Solve this story problem with a picture. Did you use addition or subtraction to solve this problem?
\end{problem}

\begin{problem}
Write another ``change over time'' story problem and then solve it with a picture. Challenge yourself to use the opposite operation as the first problem, and to not use time-indicating words like ``Monday" or ``tomorrow''.
\end{problem}


\begin{problem}
Here is an example of a ``part-part-whole'' story problem. 

\emph{Pamela has $21$ stickers on her notebook. $8$ of those stickers are from her local metro parks. How many of Pamela's stickers are not from her local metro parks?}

Solve this story problem with a picture. Did you use addition or subtraction to solve this problem?
\end{problem}

\begin{problem}
Write another ``part-part-whole'' story problem and then solve it with a picture. Challenge yourself to use the opposite operation as the first problem, and to not use time-indicating words like ``Monday" or ``tomorrow''.
\end{problem}

\begin{problem}
Here is an example of a ``comparison'' story problem. 

\emph{Santiago baked $12$ dozen cookies on Monday, and then baked $4$ dozen cookies on Tuesday. How many more dozen cookies did Santiago bake on Monday than he did on Tuesday?}

Solve this story problem with a picture. Did you use addition or subtraction to solve this problem?
\end{problem}

\begin{problem}
Write another ``comparison'' story problem and then solve it with a picture. Challenge yourself to use the opposite operation as the first problem, and to not use time-indicating words like ``Monday" or ``tomorrow''.
\end{problem}

\begin{problem}
Compare and contrast your solutions to each problem type. How did addition and/or subtraction show up? How were the pictures the same? How were they different?
\end{problem}




\newpage
\begin{instructorNotes}

{\bf Main goal:} Students practice with the meanings of addition and subtraction using story problems of different types.

{\bf Overall picture:} 

\begin{itemize}
	\item The main goal of this activity is the final problem: to compare and contrast different solution methods for addition and subtraction. We would like students to be able to use these methods flexibly, and we would like to emphasize that {\bf without} a story problem, we do not want to use a comparison-type picture. This will be essential when we look at the addition and subtraction algorithms, which should not be done by comparison since this is not the most basic meaning of the operation. 
	\item Each pair of questions gives students the opportunity to practice identifying addition (combining) and subtraction (taking away) actions within their solutions. We want students to say things like ``I combined these two quantities during my solution, so I know I used addition to solve this problem''. This is the heart of our addition/subtraction outcome.
	\item Each pair of questions also gives students the opportunity to come up with their own creative story problems. Students should be asked to discuss how they know their story is the correct type. What about the wording of the story is indicating the type (more than just key words in the problem)?
	\item Each pair of questions is designed to help students see that each type of problem can be solved with either operation. If students solve the problem with a different expression or equation than they might use to write their solution (ie. $21 = 8 + ?$ vs $21-8 = ?$, we want to point this out. We also want to emphasize that children have a lot of creative ways of solving problems, and we want to recognize each of them as addition or subtraction. But the distinction between addition and subtraction in a story problem can be unclear, since two people might solve the same story using two different operations. The operation comes from the solution, not the story itself, though we can sometimes use an expression or equation to model what we are seeing in the story.
	\item Be sure to point out the differences in the picture solutions for each type of story. We don't need a particular picture to go with a particular type of story problem, but we do want to try to model what the story is actually telling us to do. In other words, the pictures need to match what the students are claiming the story is asking them to do.
\end{itemize}

{\bf Suggested Timing:} This activity should take the entire class period. You can also split up the pairs of problems between groups, having different groups start with different types so that you have plenty of examples to discuss. Every group should be asked to think about the final problem by the end of class. Give students about 10-15 minutes to work through the first six problems, then use about 20 minutes to present. Next, give students about 5 minutes in groups to compare and contrast, and then use the rest of the time to discuss. It will be good to still have many examples of stories and pictures available around the room (if possible) to facilitate the final discussion.

\end{instructorNotes}

\end{document}
