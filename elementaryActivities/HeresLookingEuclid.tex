\documentclass{ximera}
\usepackage{gensymb}
\usepackage{tabularx}
\usepackage{mdframed}
\usepackage{pdfpages}
%\usepackage{chngcntr}

\let\problem\relax
\let\endproblem\relax

\newcommand{\property}[2]{#1#2}




\newtheoremstyle{SlantTheorem}{\topsep}{\fill}%%% space between body and thm
 {\slshape}                      %%% Thm body font
 {}                              %%% Indent amount (empty = no indent)
 {\bfseries\sffamily}            %%% Thm head font
 {}                              %%% Punctuation after thm head
 {3ex}                           %%% Space after thm head
 {\thmname{#1}\thmnumber{ #2}\thmnote{ \bfseries(#3)}} %%% Thm head spec
\theoremstyle{SlantTheorem}
\newtheorem{problem}{Problem}[]

%\counterwithin*{problem}{section}



%%%%%%%%%%%%%%%%%%%%%%%%%%%%Jenny's code%%%%%%%%%%%%%%%%%%%%

%%% Solution environment
%\newenvironment{solution}{
%\ifhandout\setbox0\vbox\bgroup\else
%\begin{trivlist}\item[\hskip \labelsep\small\itshape\bfseries Solution\hspace{2ex}]
%\par\noindent\upshape\small
%\fi}
%{\ifhandout\egroup\else
%\end{trivlist}
%\fi}
%
%
%%% instructorIntro environment
%\ifhandout
%\newenvironment{instructorIntro}[1][false]%
%{%
%\def\givenatend{\boolean{#1}}\ifthenelse{\boolean{#1}}{\begin{trivlist}\item}{\setbox0\vbox\bgroup}{}
%}
%{%
%\ifthenelse{\givenatend}{\end{trivlist}}{\egroup}{}
%}
%\else
%\newenvironment{instructorIntro}[1][false]%
%{%
%  \ifthenelse{\boolean{#1}}{\begin{trivlist}\item[\hskip \labelsep\bfseries Instructor Notes:\hspace{2ex}]}
%{\begin{trivlist}\item[\hskip \labelsep\bfseries Instructor Notes:\hspace{2ex}]}
%{}
%}
%% %% line at the bottom} 
%{\end{trivlist}\par\addvspace{.5ex}\nobreak\noindent\hung} 
%\fi
%
%


\let\instructorNotes\relax
\let\endinstructorNotes\relax
%%% instructorNotes environment
\ifhandout
\newenvironment{instructorNotes}[1][false]%
{%
\def\givenatend{\boolean{#1}}\ifthenelse{\boolean{#1}}{\begin{trivlist}\item}{\setbox0\vbox\bgroup}{}
}
{%
\ifthenelse{\givenatend}{\end{trivlist}}{\egroup}{}
}
\else
\newenvironment{instructorNotes}[1][false]%
{%
  \ifthenelse{\boolean{#1}}{\begin{trivlist}\item[\hskip \labelsep\bfseries {\Large Instructor Notes: \\} \hspace{\textwidth} ]}
{\begin{trivlist}\item[\hskip \labelsep\bfseries {\Large Instructor Notes: \\} \hspace{\textwidth} ]}
{}
}
{\end{trivlist}}
\fi


%% Suggested Timing
\newcommand{\timing}[1]{{\bf Suggested Timing: \hspace{2ex}} #1}




\hypersetup{
    colorlinks=true,       % false: boxed links; true: colored links
    linkcolor=blue,          % color of internal links (change box color with linkbordercolor)
    citecolor=green,        % color of links to bibliography
    filecolor=magenta,      % color of file links
    urlcolor=cyan           % color of external links
}
\title{Here's Looking At Euclid}
\author{Vic Ferdinand, Betsy McNeal, Jenny Sheldon}

\begin{document}
\begin{abstract}
\end{abstract}

\maketitle





\begin{question}
How should we define ``parallel lines'' if the lines are on a flat piece of paper?  Will your definition always make sense if the two lines are in three dimensions (perhaps use pencils to model your 3-D lines)?
\end{question}









\begin{problem} \label{Euclid1}
The converse of the Parallel Postulate says that we can check that two lines are parallel by checking for congruent {\bf corresponding angles} (a and b in the picture below).  Why is this a reasonable assumption? That is, what would happen if angles a and b were not equal?  

\begin{center}
    \begin{tikzpicture}
        \draw[domain=0:8] plot (\x, {0.1*\x});
        \draw[domain=0:8] plot (\x, {0.1*\x + 3});
        \draw[domain=2:6] plot (\x, {2*\x-6});
        \node at (4.5, 3.7) {$a$};
        \node at (2.9, 0.6) {$b$};
        \node at (5, 3.3) {$c$};
    \end{tikzpicture}
\end{center}



\end{problem}







\begin{problem}
Assuming $a = b$, argue why angles b and c should also be of equal measure (we call these {\bf alternate interior angles}).
\end{problem}

\begin{problem}
In the figure above, label all of the angles with letters.  Then, make a list (or more than one list) of all angles which are equal to each other.
\end{problem}


\newpage

\begin{instructorNotes}
The idea in this activity is to help students to see the generalizability of the ``same direction'' definition of parallel lines as opposed to the still-useful (in 2-D) lines that never intersect.  For us, this activity comes after introducing the Parallel Postulate and before discussing definitions and properties of shapes.

We have found that it is much easier for students to make sense of the angles and parallelism in the Parallel Postulate using the concept of direction rather than the concept of non-intersection.  Also, we have seen some student difficulties with vertical angles to reappear in this context.

\timing{This activity takes us about half a class period.  We give students about 5-10 minutes to think about the problems in their small groups, then discuss for 10-15 minutes as a whole class.}
\end{instructorNotes}


\end{document}