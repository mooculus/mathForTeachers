\documentclass{ximera}

\graphicspath{
  {./}
  {graphics/}
  {../graphics/}
}

\usepackage{chngcntr}

\let\question\relax
\let\endquestion\relax




\newtheoremstyle{SlantTheorem}{\topsep}{\fill}%%% space between body and thm
%\newtheoremstyle{SlantTheorem}{\topsep}{\topsep}%%% space between body and thm
 {\slshape}                      %%% Thm body font
 {}                              %%% Indent amount (empty = no indent)
 {\bfseries\sffamily}            %%% Thm head font
 {}                              %%% Punctuation after thm head
 {3ex}                           %%% Space after thm head
 {\thmname{#1}\thmnumber{ #2}\thmnote{ \bfseries(#3)}}%%% Thm head spec
\theoremstyle{SlantTheorem}
\newtheorem{question}{Question}
\counterwithin*{question}{section}



\let\instructorNotes\relax
\let\endinstructorNotes\relax
%%% instructorNotes environment
\ifhandout
\newenvironment{instructorNotes}[1][false]%
{%
\def\givenatend{\boolean{#1}}\ifthenelse{\boolean{#1}}{\begin{trivlist}\item}{\setbox0\vbox\bgroup}{}
}
{%
\ifthenelse{\givenatend}{\end{trivlist}}{\egroup}{}
}
\else
\newenvironment{instructorNotes}[1][false]%
{%
  \ifthenelse{\boolean{#1}}{\begin{trivlist}\item[\hskip \labelsep\bfseries {\Large Instructor Notes: \\} \hspace{\textwidth} ]}
{\begin{trivlist}\item[\hskip \labelsep\bfseries {\Large Instructor Notes: \\} \hspace{\textwidth} ]}
{}
}
{\end{trivlist}}
\fi


%% Suggested Timing
\newcommand{\timing}[1]{{\bf Suggested Timing: \hspace{2ex}} #1}
\title{Here's Looking At Euclid}
\author{Vic Ferdinand \& Betsy McNeal \& Jenny Sheldon}

\begin{document}
\begin{abstract}
\end{abstract}

\maketitle


\begin{instructorIntro}
The idea in this activity is to get the students to see the generalizability of the ``same direction'' definition of parallel lines as opposed to the still-useful (in 2-D) lines that never intersect.  It is much easier to argue with the concept of direction (for both ``angle'' and ``parallel'') in making sense of the Parallel Postulate.  As with vertical angles, transitory arguments may be difficult for some to grasp.
\end{instructorIntro}


\begin{question}
How should we define ``parallel lines'' if the lines are on a flat piece of paper?  Will your definition always make sense if the two lines are in three dimensions (perhaps use pencils to model your 3-D lines)?
\end{question}









\begin{problem} \label{Euclid1}
The converse of the Parallel Postulate says that we can check that two lines are parallel by checking for congruent {\bf corresponding angles} (a and b in the picture below).  Why is this a reasonable assumption? That is, what would happen if angles a and b were not equal?  
\[
\includegraphics[width=3in]{graphics/parallelEuclid.pdf}
\]
\end{problem}







\begin{problem}
Assuming $a = b$, argue why angles b and c should also be of equal measure (we call these {\bf alternate interior angles}).
\end{problem}

\begin{problem}
In the figure above, label all of the angles with letters.  Then, make a list (or more than one list) of all angles which are equal to each other.
\end{problem}






\end{document}