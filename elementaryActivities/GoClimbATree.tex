\documentclass[nooutcomes]{ximera}
\usepackage{gensymb}
\usepackage{tabularx}
\usepackage{mdframed}
\usepackage{pdfpages}
%\usepackage{chngcntr}

\let\problem\relax
\let\endproblem\relax

\newcommand{\property}[2]{#1#2}




\newtheoremstyle{SlantTheorem}{\topsep}{\fill}%%% space between body and thm
 {\slshape}                      %%% Thm body font
 {}                              %%% Indent amount (empty = no indent)
 {\bfseries\sffamily}            %%% Thm head font
 {}                              %%% Punctuation after thm head
 {3ex}                           %%% Space after thm head
 {\thmname{#1}\thmnumber{ #2}\thmnote{ \bfseries(#3)}} %%% Thm head spec
\theoremstyle{SlantTheorem}
\newtheorem{problem}{Problem}[]

%\counterwithin*{problem}{section}



%%%%%%%%%%%%%%%%%%%%%%%%%%%%Jenny's code%%%%%%%%%%%%%%%%%%%%

%%% Solution environment
%\newenvironment{solution}{
%\ifhandout\setbox0\vbox\bgroup\else
%\begin{trivlist}\item[\hskip \labelsep\small\itshape\bfseries Solution\hspace{2ex}]
%\par\noindent\upshape\small
%\fi}
%{\ifhandout\egroup\else
%\end{trivlist}
%\fi}
%
%
%%% instructorIntro environment
%\ifhandout
%\newenvironment{instructorIntro}[1][false]%
%{%
%\def\givenatend{\boolean{#1}}\ifthenelse{\boolean{#1}}{\begin{trivlist}\item}{\setbox0\vbox\bgroup}{}
%}
%{%
%\ifthenelse{\givenatend}{\end{trivlist}}{\egroup}{}
%}
%\else
%\newenvironment{instructorIntro}[1][false]%
%{%
%  \ifthenelse{\boolean{#1}}{\begin{trivlist}\item[\hskip \labelsep\bfseries Instructor Notes:\hspace{2ex}]}
%{\begin{trivlist}\item[\hskip \labelsep\bfseries Instructor Notes:\hspace{2ex}]}
%{}
%}
%% %% line at the bottom} 
%{\end{trivlist}\par\addvspace{.5ex}\nobreak\noindent\hung} 
%\fi
%
%


\let\instructorNotes\relax
\let\endinstructorNotes\relax
%%% instructorNotes environment
\ifhandout
\newenvironment{instructorNotes}[1][false]%
{%
\def\givenatend{\boolean{#1}}\ifthenelse{\boolean{#1}}{\begin{trivlist}\item}{\setbox0\vbox\bgroup}{}
}
{%
\ifthenelse{\givenatend}{\end{trivlist}}{\egroup}{}
}
\else
\newenvironment{instructorNotes}[1][false]%
{%
  \ifthenelse{\boolean{#1}}{\begin{trivlist}\item[\hskip \labelsep\bfseries {\Large Instructor Notes: \\} \hspace{\textwidth} ]}
{\begin{trivlist}\item[\hskip \labelsep\bfseries {\Large Instructor Notes: \\} \hspace{\textwidth} ]}
{}
}
{\end{trivlist}}
\fi


%% Suggested Timing
\newcommand{\timing}[1]{{\bf Suggested Timing: \hspace{2ex}} #1}




\hypersetup{
    colorlinks=true,       % false: boxed links; true: colored links
    linkcolor=blue,          % color of internal links (change box color with linkbordercolor)
    citecolor=green,        % color of links to bibliography
    filecolor=magenta,      % color of file links
    urlcolor=cyan           % color of external links
}
\title{Go Climb A Tree!}
\author{Vic Ferdinand, Betsy McNeal, Jenny Sheldon}

\begin{document}
\begin{abstract}
\end{abstract}
\maketitle



\begin{problem}
\begin{enumerate}
\item Starting with the statement ``$\frac25$ of the class are girls'', write a story problem that is modeled by the expression $\frac37 \times \frac25$.  Once you have a story, solve it using pictures and remind yourself why multiplying fractions is the same as multiplying the numerators and the denominators.
\item Restate your story as a probability problem and use a tree diagram to illustrate the solution.
\end{enumerate}
\end{problem}
\vfill
\begin{problem}\label{NBAStar}
Suppose an NBA superstar has been a 70\% free throw shooter throughout his long and storied career.  (That is, he has made 70\% of all free throw shots he has taken.)  Suppose he is now in a ``one-and-one'' situation: he shoots a free throw.  If he makes it, he gets one point and shoots one additional free throw, worth one point.  If he misses, he gets zero points and does not shoot another free throw.  What is the probability that:
\begin{enumerate}
\item he makes zero points?
\item he makes one point?
\item he makes two points?
\end{enumerate}
\end{problem}
\vfill
\begin{problem}
Why does it make sense that the probabilities in Problem 2 add up to 1?
\end{problem}
\vfill
\newpage

\begin{problem}
The Indians and the Yankees face each other in a playoff series.  The probability of the Indians winning any given game is 0.3, or 30\%.  Say that the two teams will play a best-of-three series, which means that the first team to win two games will win the whole series.

\begin{enumerate}
\item What is the probability that the Indians win games 1 and 3 to win the series?
\item What is the probability that the Indians win the series in exactly three games?
\end{enumerate}
\end{problem}
\vfill
\begin{problem}
The Indians and the Yankees face each other in a playoff series.  The probability of the Indians winning any given game is 0.3, or 30\%.  Say that the two teams will play a best-of-seven series, which means that the first team to win four games will win the whole series.

\begin{enumerate}
\item What is the probability that the Indians win games 1, 3, 4, and 7 to win the series?
\item What is the probability that the Indians win the series in exactly six games?
\end{enumerate}
\end{problem}
\vfill

\newpage

\begin{instructorNotes}
This activity introduces students to the notion of using a tree diagram to help calculate probabilities, and emphasizes the relationship between fraction multiplication as we saw when we originally discussed fractions, and fraction multiplication in the situation of probability.  

In our calendar, this activity follows ``You Can Count On Probability'', where students practice solving probability problems.  This activity assumes that students are comfortable with probability as a concept and can compute basic probabilities.  This activity is sometimes the last one we do about counting and probability, and is sometimes followed by ``Taking Chances'', depending on our time.

In general, we use these activities about counting and probability to help cement the meaning of various operations for students.  Throughout these activities, we expect students to justify their operations of choice in their explanations.  We also try to recognize structure when we can, comparing problems we have solved in the past to ones we are solving currently.  In this activity, the final part of the final problem may be too difficult for students unless they realize that they have counted this situation several times before.

The first problem is intended to help students make the right connections before getting started.  We sometimes go through the second problem as a whole class, in order to help students see what they should do for later problems.


\timing{We usually choose enough of these problems to get us through one class period, and leave the rest for homework or extra enrichment.  The first three problems are likely enough for a whole period.}
\end{instructorNotes}


\end{document}