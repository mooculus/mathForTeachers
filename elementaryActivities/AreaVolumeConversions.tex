\documentclass{ximera}
\usepackage{gensymb}
\usepackage{tabularx}
\usepackage{mdframed}
\usepackage{pdfpages}
%\usepackage{chngcntr}

\let\problem\relax
\let\endproblem\relax

\newcommand{\property}[2]{#1#2}




\newtheoremstyle{SlantTheorem}{\topsep}{\fill}%%% space between body and thm
 {\slshape}                      %%% Thm body font
 {}                              %%% Indent amount (empty = no indent)
 {\bfseries\sffamily}            %%% Thm head font
 {}                              %%% Punctuation after thm head
 {3ex}                           %%% Space after thm head
 {\thmname{#1}\thmnumber{ #2}\thmnote{ \bfseries(#3)}} %%% Thm head spec
\theoremstyle{SlantTheorem}
\newtheorem{problem}{Problem}[]

%\counterwithin*{problem}{section}



%%%%%%%%%%%%%%%%%%%%%%%%%%%%Jenny's code%%%%%%%%%%%%%%%%%%%%

%%% Solution environment
%\newenvironment{solution}{
%\ifhandout\setbox0\vbox\bgroup\else
%\begin{trivlist}\item[\hskip \labelsep\small\itshape\bfseries Solution\hspace{2ex}]
%\par\noindent\upshape\small
%\fi}
%{\ifhandout\egroup\else
%\end{trivlist}
%\fi}
%
%
%%% instructorIntro environment
%\ifhandout
%\newenvironment{instructorIntro}[1][false]%
%{%
%\def\givenatend{\boolean{#1}}\ifthenelse{\boolean{#1}}{\begin{trivlist}\item}{\setbox0\vbox\bgroup}{}
%}
%{%
%\ifthenelse{\givenatend}{\end{trivlist}}{\egroup}{}
%}
%\else
%\newenvironment{instructorIntro}[1][false]%
%{%
%  \ifthenelse{\boolean{#1}}{\begin{trivlist}\item[\hskip \labelsep\bfseries Instructor Notes:\hspace{2ex}]}
%{\begin{trivlist}\item[\hskip \labelsep\bfseries Instructor Notes:\hspace{2ex}]}
%{}
%}
%% %% line at the bottom} 
%{\end{trivlist}\par\addvspace{.5ex}\nobreak\noindent\hung} 
%\fi
%
%


\let\instructorNotes\relax
\let\endinstructorNotes\relax
%%% instructorNotes environment
\ifhandout
\newenvironment{instructorNotes}[1][false]%
{%
\def\givenatend{\boolean{#1}}\ifthenelse{\boolean{#1}}{\begin{trivlist}\item}{\setbox0\vbox\bgroup}{}
}
{%
\ifthenelse{\givenatend}{\end{trivlist}}{\egroup}{}
}
\else
\newenvironment{instructorNotes}[1][false]%
{%
  \ifthenelse{\boolean{#1}}{\begin{trivlist}\item[\hskip \labelsep\bfseries {\Large Instructor Notes: \\} \hspace{\textwidth} ]}
{\begin{trivlist}\item[\hskip \labelsep\bfseries {\Large Instructor Notes: \\} \hspace{\textwidth} ]}
{}
}
{\end{trivlist}}
\fi


%% Suggested Timing
\newcommand{\timing}[1]{{\bf Suggested Timing: \hspace{2ex}} #1}




\hypersetup{
    colorlinks=true,       % false: boxed links; true: colored links
    linkcolor=blue,          % color of internal links (change box color with linkbordercolor)
    citecolor=green,        % color of links to bibliography
    filecolor=magenta,      % color of file links
    urlcolor=cyan           % color of external links
}

\title{Area and Volume Conversions}
\author{Vic Ferdinand, Betsy McNeal, Jenny Sheldon}
\begin{document}
\begin{abstract}\end{abstract}
\maketitle

\begin{instructorIntro}

This activity is designed to help students think through the calculations for converting area and volume between English and metric units.  We encourage the students to avoid ``dimensional analysis'' or a ``fence-post'' method, which we have found many students have seen in the past, particularly in science courses.  Dimensional analysis tends to be another ``rule'' that students have memorized, and obscures the meaning of multiplication and division in conversion problems.  Encourage students to explain why their calculations are the correct ones, not just why they put numbers in various positions!


\timing{Give the students about 10 minutes to start working on the problem.  Afterwards, spend 10-15 minutes discussing the first problem, to ensure that all students have an idea from which to start.  Then, give the students about 5 more minutes to complete the worksheet, and finish with 10 minutes of discussion for the second problem.}

\end{instructorIntro}

For all of the problems in this activity, you may use the fact that 2.54 cm is equal to 1 inch.  You may not use the usual ``dimensional analysis" as one of your methods - stretch your thinking!



\begin{problem}
Convert $25m^2$ into square feet.  Do this problem in at least two ways!
\begin{instructorNotes}
Students occasionally struggle to find a method to solve these problems without resorting to dimensional analysis.  If students are stuck, suggest they draw a picture to help them solve the problem.  Students also don't naturally try to give their answers in terms of the meaning of multiplication and division, and will need to be coaxed in this direction.
\begin{enumerate}
	\item Encourage the students to try to think about this problem in multiple ways.  Feel free to highlight any misconceptions you have seen in the past, or that come up in the students' work.  In particular, pay attention to the dimension of the measurement!
	\item Encourage students to draw a picture if they're stuck.  Using two different methods could simply involve drawing two different pictures to solve the problem!
	\item If the students draw a picture, have them explain how they know they can draw that particular picture.  Essentially, this is a consequence of moving and additivity.  We can rearrange the area in any fashion we like because of the principles of moving and additivity - no area is created or lost when we rearrange. 
	\item Emphasize the meaning of multiplication and division in these problems.  Students should identify the groups and objects to describe why they decide to multiply, or why they decide to divide.  With division, students should be able to identify which type of division they are using - though in the case of conversion it is usually ``how many groups?''.
\end{enumerate}

\end{instructorNotes}
\end{problem}

\begin{problem}
Convert $64$ cubic feet into cubic centimeters.  Do this problem in at least two ways!

\begin{instructorNotes}
See the notes for the previous problem.
\end{instructorNotes}
\end{problem}



\end{document}