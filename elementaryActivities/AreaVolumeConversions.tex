\documentclass{ximera}

\graphicspath{
  {./}
  {graphics/}
  {../graphics/}
}

\usepackage{chngcntr}

\let\question\relax
\let\endquestion\relax




\newtheoremstyle{SlantTheorem}{\topsep}{\fill}%%% space between body and thm
%\newtheoremstyle{SlantTheorem}{\topsep}{\topsep}%%% space between body and thm
 {\slshape}                      %%% Thm body font
 {}                              %%% Indent amount (empty = no indent)
 {\bfseries\sffamily}            %%% Thm head font
 {}                              %%% Punctuation after thm head
 {3ex}                           %%% Space after thm head
 {\thmname{#1}\thmnumber{ #2}\thmnote{ \bfseries(#3)}}%%% Thm head spec
\theoremstyle{SlantTheorem}
\newtheorem{question}{Question}
\counterwithin*{question}{section}



\let\instructorNotes\relax
\let\endinstructorNotes\relax
%%% instructorNotes environment
\ifhandout
\newenvironment{instructorNotes}[1][false]%
{%
\def\givenatend{\boolean{#1}}\ifthenelse{\boolean{#1}}{\begin{trivlist}\item}{\setbox0\vbox\bgroup}{}
}
{%
\ifthenelse{\givenatend}{\end{trivlist}}{\egroup}{}
}
\else
\newenvironment{instructorNotes}[1][false]%
{%
  \ifthenelse{\boolean{#1}}{\begin{trivlist}\item[\hskip \labelsep\bfseries {\Large Instructor Notes: \\} \hspace{\textwidth} ]}
{\begin{trivlist}\item[\hskip \labelsep\bfseries {\Large Instructor Notes: \\} \hspace{\textwidth} ]}
{}
}
{\end{trivlist}}
\fi


%% Suggested Timing
\newcommand{\timing}[1]{{\bf Suggested Timing: \hspace{2ex}} #1}

\title{Area and Volume Conversions}
\author{Vic Ferdinand, Betsy McNeal, Jenny Sheldon}
\begin{document}
\begin{abstract}\end{abstract}
\maketitle


For all of the problems in this activity, you may use the fact that 2.54 cm is equal to 1 inch.  You may not use the usual ``dimensional analysis" as one of your methods - stretch your thinking!



\begin{problem}
Convert $25m^2$ into square feet.  Do this problem in at least two ways!

\end{problem}

\begin{problem}
Convert $64$ cubic feet into cubic centimeters.  Do this problem in at least two ways!

\end{problem}


\newpage
\begin{instructorNotes}

This activity is designed to help students think through the calculations for converting area and volume between English and metric units.  We encourage the students to avoid ``dimensional analysis'' or a ``fence-post'' method, which we have found many students have seen in the past, particularly in science courses.  Dimensional analysis tends to be another ``rule'' that students have memorized, and obscures the meaning of multiplication and division in conversion problems.  We would like students to explain why their calculations are the correct ones, not just why they put numbers in various positions!

This activity assumes that students know how to calculate area and volume for various shapes, and that they understand the moving and additivity principle whereby we can rearrange a shape without changing its area.  We also require students to be familiar with the definitions of the four basic operations. The activity does not assume that students have previously done any measurement conversion.

Students occasionally struggle at first to find a method to solve these problems without resorting to dimensional analysis.  If students are stuck, we suggest they draw a picture to help them solve the problem.  Students also don't naturally try to give their answers in terms of the meaning of multiplication and division, but we coax them in this direction.
\begin{itemize}
	\item We encourage the students to try to think about this problem in multiple ways.  During discussion, we highlight any misconceptions that come up in the students' work, and sometimes also add misconceptions we have seen in the past.  The most common error we see is students not paying attention to the dimension of the measurement, for instance converting from square inches to square feet by multiplying by $12$.
	\item We do allow starting with drawing two different pictures as two different methods to solve the problem.  This can seem ``not different enough'' to some students, but can be quite surprising to other students.
	\item If the students draw a picture, we have them explain how they know they can draw that particular picture.  We expect them to answer that we can rearrange the area in any fashion we like because of the principles of moving and additivity - no area is created or lost when we rearrange. 
	\item We heavily emphasize the meaning of multiplication and division in these problems.  Students should identify the groups and objects to describe why they decide to multiply, or why they decide to divide.  With division, students should be able to identify which type of division they are using - though in the case of conversion it is usually ``how many groups?''.
\end{itemize}




\timing{We give the students about 10 minutes to start working on the problem.  Afterwards, we spend 10-15 minutes discussing the first problem, to ensure that all students have an idea from which to start.  Then, we give the students about 5 more minutes to complete the worksheet, and finish with 10 minutes of discussion for the second problem.}

\end{instructorNotes}




\end{document}