%\documentclass[handout]{ximera}
\documentclass{ximera}
\usepackage{gensymb}
\usepackage{tabularx}
\usepackage{mdframed}
\usepackage{pdfpages}
%\usepackage{chngcntr}

\let\problem\relax
\let\endproblem\relax

\newcommand{\property}[2]{#1#2}




\newtheoremstyle{SlantTheorem}{\topsep}{\fill}%%% space between body and thm
 {\slshape}                      %%% Thm body font
 {}                              %%% Indent amount (empty = no indent)
 {\bfseries\sffamily}            %%% Thm head font
 {}                              %%% Punctuation after thm head
 {3ex}                           %%% Space after thm head
 {\thmname{#1}\thmnumber{ #2}\thmnote{ \bfseries(#3)}} %%% Thm head spec
\theoremstyle{SlantTheorem}
\newtheorem{problem}{Problem}[]

%\counterwithin*{problem}{section}



%%%%%%%%%%%%%%%%%%%%%%%%%%%%Jenny's code%%%%%%%%%%%%%%%%%%%%

%%% Solution environment
%\newenvironment{solution}{
%\ifhandout\setbox0\vbox\bgroup\else
%\begin{trivlist}\item[\hskip \labelsep\small\itshape\bfseries Solution\hspace{2ex}]
%\par\noindent\upshape\small
%\fi}
%{\ifhandout\egroup\else
%\end{trivlist}
%\fi}
%
%
%%% instructorIntro environment
%\ifhandout
%\newenvironment{instructorIntro}[1][false]%
%{%
%\def\givenatend{\boolean{#1}}\ifthenelse{\boolean{#1}}{\begin{trivlist}\item}{\setbox0\vbox\bgroup}{}
%}
%{%
%\ifthenelse{\givenatend}{\end{trivlist}}{\egroup}{}
%}
%\else
%\newenvironment{instructorIntro}[1][false]%
%{%
%  \ifthenelse{\boolean{#1}}{\begin{trivlist}\item[\hskip \labelsep\bfseries Instructor Notes:\hspace{2ex}]}
%{\begin{trivlist}\item[\hskip \labelsep\bfseries Instructor Notes:\hspace{2ex}]}
%{}
%}
%% %% line at the bottom} 
%{\end{trivlist}\par\addvspace{.5ex}\nobreak\noindent\hung} 
%\fi
%
%


\let\instructorNotes\relax
\let\endinstructorNotes\relax
%%% instructorNotes environment
\ifhandout
\newenvironment{instructorNotes}[1][false]%
{%
\def\givenatend{\boolean{#1}}\ifthenelse{\boolean{#1}}{\begin{trivlist}\item}{\setbox0\vbox\bgroup}{}
}
{%
\ifthenelse{\givenatend}{\end{trivlist}}{\egroup}{}
}
\else
\newenvironment{instructorNotes}[1][false]%
{%
  \ifthenelse{\boolean{#1}}{\begin{trivlist}\item[\hskip \labelsep\bfseries {\Large Instructor Notes: \\} \hspace{\textwidth} ]}
{\begin{trivlist}\item[\hskip \labelsep\bfseries {\Large Instructor Notes: \\} \hspace{\textwidth} ]}
{}
}
{\end{trivlist}}
\fi


%% Suggested Timing
\newcommand{\timing}[1]{{\bf Suggested Timing: \hspace{2ex}} #1}




\hypersetup{
    colorlinks=true,       % false: boxed links; true: colored links
    linkcolor=blue,          % color of internal links (change box color with linkbordercolor)
    citecolor=green,        % color of links to bibliography
    filecolor=magenta,      % color of file links
    urlcolor=cyan           % color of external links
}
\title{Get Lost!}

\author{Vic Ferdinand \& Betsy McNeal \& Jenny Sheldon}

\begin{document}
\begin{abstract}
\end{abstract}
\maketitle

\begin{instructorIntro}
The intent of this activity is to develop two definitions of angle:
(1) as two rays with a common vertex and the attribute measured by an angle is the relative direction between the rays (as opposed to a ``spacey'' measurement), and (2) as an amount of rotation.  The students are also asked (in Problem 7) to intuitively see that it makes sense for vertical angles to be congruent (they may need some prodding to show this algebraically:  That one angle is represented in both sums of two supplementary angles and thus the remaining angles must be the same measure.

You might start this activity with a whole-class discussion about the definitions of point, line, and ray if you haven't already.
\end{instructorIntro}

Two hikers - Wayward Wynona and Lost Larry - meet where the roads Clueless Corners and Loser's Lane intersect and start hitchhiking along the two straight roads.  Clueless Corners heads directly north and Loser’s Lane directly northeast.  Assume the hikers travel at the same speed.
\[
\includegraphics[height=1.5in]{graphics/angleRoads1}
\]

\begin{problem}
As they hike, what is happening to the distance between the two hikers?
\begin{solution}
The distance is increasing.
\end{solution}
\end{problem}

\begin{problem}
As they hike, what is happening to the space between the two hikers?
\begin{solution}
The space is increasing.
\end{solution}
\end{problem}

\begin{problem}
As they hike, what is happening to the relative directions between the two hikers?
\begin{solution}
The relative directions are staying the same.
\end{solution}
\end{problem}

\begin{problem}
Judging from the above questions, what is an angle and what attribute does it measure?

\begin{instructorNotes}
Here, the students should come up with the idea that an angle is the union of two rays meeting at a common vertex - NOT these misconceptions previoulsy discussed.
\end{instructorNotes}
\end{problem}


\begin{problem}
Some people say that an angle measures the amount of rotation that, say, a ballerina would turn. Relate this to the hiking situation.
\begin{instructorNotes}
     In this problem, students are asked to look at angle as the rotation from the first ray’s direction to the second.  This should only take a few minutes right after Get Lost!- perhaps as a class discussion with a person rotating through the directions – as they will do in the triangle activity!
    Both definitions will be used in the course.
\end{instructorNotes}
\end{problem}   

\newpage

\begin{problem}
If the hikers each hiked on Clueless Corners, but in opposite directions, what would be the angle between their two paths?  How is this related to the situation with the ballerina?

\begin{instructorNotes}
In this problem, the students should recognize that a half-turn (or half of a full circle) is between the hikers.  Since we usually use $360\degree$ for a full circle, this would be $180\degree$.  You might point out that if we chose a different measure for a full turn, then this measure would be different.
\end{instructorNotes}
\end{problem}


\begin{problem}
On a different day, the hikers meet again at the intersection of the two roads.  This time, Wayward Winona hikes south on Clueless Corners, and Lost Larry hikes southwest on Loser's Lane.  What can you say about the angle between their paths?  Make two arguments for your claim: one that could be understood by very young children, and one that uses arithmetic.  Identify which notion of ``angle'' you are using in each.

\[
\includegraphics[height=3in]{graphics/angleRoads2.pdf}
\]

\begin{instructorNotes}
Here you should introduce the term ``vertical'' or ``opposite'' angles to describe the angles between the hikers.  Make sure to discuss both pairs of vertical angles!

You can also begin here talking about different levels of arguments that students can make.  We have a folding argument for this problem, where the students would fold one angle on top of the other.  A related cutting or turning argument is also sometimes made.  These arguments are good for experimentation, but would require us to check all cases before making a conclusion.  They are also good experiences for even very young children as they develop geometric intuition.  The more algebraic argument $a+b = b+c$ so $a = c$ is more formal and allows us to say something about all sets of vertical angles without measuring them.
\end{instructorNotes}
\end{problem}




\end{document}