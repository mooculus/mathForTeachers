\documentclass{ximera}

\graphicspath{
  {./}
  {graphics/}
  {../graphics/}
}

\usepackage{chngcntr}

\let\question\relax
\let\endquestion\relax




\newtheoremstyle{SlantTheorem}{\topsep}{\fill}%%% space between body and thm
%\newtheoremstyle{SlantTheorem}{\topsep}{\topsep}%%% space between body and thm
 {\slshape}                      %%% Thm body font
 {}                              %%% Indent amount (empty = no indent)
 {\bfseries\sffamily}            %%% Thm head font
 {}                              %%% Punctuation after thm head
 {3ex}                           %%% Space after thm head
 {\thmname{#1}\thmnumber{ #2}\thmnote{ \bfseries(#3)}}%%% Thm head spec
\theoremstyle{SlantTheorem}
\newtheorem{question}{Question}
\counterwithin*{question}{section}



\let\instructorNotes\relax
\let\endinstructorNotes\relax
%%% instructorNotes environment
\ifhandout
\newenvironment{instructorNotes}[1][false]%
{%
\def\givenatend{\boolean{#1}}\ifthenelse{\boolean{#1}}{\begin{trivlist}\item}{\setbox0\vbox\bgroup}{}
}
{%
\ifthenelse{\givenatend}{\end{trivlist}}{\egroup}{}
}
\else
\newenvironment{instructorNotes}[1][false]%
{%
  \ifthenelse{\boolean{#1}}{\begin{trivlist}\item[\hskip \labelsep\bfseries {\Large Instructor Notes: \\} \hspace{\textwidth} ]}
{\begin{trivlist}\item[\hskip \labelsep\bfseries {\Large Instructor Notes: \\} \hspace{\textwidth} ]}
{}
}
{\end{trivlist}}
\fi


%% Suggested Timing
\newcommand{\timing}[1]{{\bf Suggested Timing: \hspace{2ex}} #1}
\title{Get Lost!}

\author{Vic Ferdinand, Betsy McNeal, Jenny Sheldon}

\begin{document}
\begin{abstract}
\end{abstract}
\maketitle



Two hikers - Wayward Wynona and Lost Larry - meet where the roads Clueless Corners and Loser's Lane intersect and start hitchhiking along the two straight roads.  Clueless Corners heads directly north and Loser’s Lane directly northeast.  Assume the hikers travel at the same speed.

\begin{center}
    \begin{tikzpicture}
    \draw[->] (0,0)--(0,5) node[rotate=90, above, pos=0.5] {Clueless Corners};
        \node at (0, 5.2) {N};
    \draw[->] (0,0)--(3,4) node[rotate=54, below, pos=0.5] {Loser's Lane};
    \end{tikzpicture}
\end{center}

\begin{problem} \label{GetLost1}
As they hike, what is happening to the distance between the two hikers?

\end{problem}

\begin{problem} \label{GetLost2}
As they hike, what is happening to the space between the two hikers?

\end{problem}

\begin{problem} \label{GetLost3}
As they hike, what is happening to the relative directions between the two hikers?

\end{problem}

\begin{problem} \label{GetLost4}
Judging from the above questions, what is an angle and what attribute does it measure?


\end{problem}


\begin{problem} \label{GetLost5}
Some people say that an angle measures the amount of rotation that, say, a ballerina would turn. Relate this to the hiking situation.

\end{problem}   

\newpage

\begin{problem} \label{GetLost6}
If the hikers each hiked on Clueless Corners, but in opposite directions, what would be the angle between their two paths?  How is this related to the situation with the ballerina?


\end{problem}


\begin{problem} \label{GetLost7}
On a different day, the hikers meet again at the intersection of the two roads.  This time, Wayward Winona hikes south on Clueless Corners, and Lost Larry hikes southwest on Loser's Lane.  What can you say about the angle between their paths?  Make two arguments for your claim: one that could be understood by very young children, and one that uses arithmetic.  Identify which notion of ``angle'' you are using in each.


\begin{center}
    \begin{tikzpicture}
    \draw[<->] (0,-5)--(0,5) node[rotate=90, above, pos=0.75] {Clueless Corners};
        \node at (0, 5.2) {N};
    \draw[<->] (-3,-4)--(3,4) node[rotate=54, below, pos=0.75] {Loser's Lane};
    \end{tikzpicture}
\end{center}
\end{problem}


\newpage

\begin{instructorNotes}
The intent of this activity is to develop two definitions of angle:
(1) as two rays with a common vertex and the attribute measured by an angle is the relative direction between the rays (as opposed to a ``spacey'' measurement), and (2) as an amount of rotation.  We like to have the flexibility of referring to both of these definitions throughout the course.  Also in this activity, students are asked to intuitively see that it makes sense for vertical angles to be congruent (but sometimes need some prodding to show this algebraically).

\begin{itemize}
    \item We usually start this activity with a whole-class discussion about the definitions of point, line, and ray, as this is the first activity in which these geometric objects come up for us.
    \item The first few problems give us an opportunity to talk not only about what an angle is, but also what it is not.  Problems \ref{GetLost1} through \ref{GetLost4} allow us to bring up some children's misconceptions as we develop the ``relative direction'' idea.
    \item Problem \ref{GetLost5} gives us the opportunity to introduce the second definition and relate it to the first.  Being able to look at an angle from two perspectives is something with which we have generally found our students to be unfamiliar.
    \item In Problem \ref{GetLost6}, the students should recognize that a half-turn (or half of a full circle) is between the hikers.  We take this opportunity to discuss that measuring angles using degrees is a choice we make, not a requirement for the definition of an angle.  We usually also point out other ways to measure angles.
    \item Problem \ref{GetLost7} gives us an opportunity to introduce the terminology for vertical (or opposite) angles, and to remind students that there are two pairs of vertical angles, not just one.
    \item Problem \ref{GetLost7} also gives us space to discuss different levels of arguments that students can make -- and the idea of different levels of argument is a theme for us during the semester.  We have a folding argument for this problem, where the students would fold one angle on top of the other.  We also see a related cutting or turning argument.  We remind students that these arguments are good for experimentation, but would require us to check all cases before making a conclusion.  We also point out that these arguments are also good experiences for even very young children as they develop geometric intuition.  Finally, we discuss how the more algebraic argument $a+b = b+c$ so $a = c$ is more formal and allows us to say something about all sets of vertical angles without measuring them.
\end{itemize}

\timing{This activity takes us about half a class period.  We give students 5-10 minutes to think about all of the problems, then spend 15-20 minutes in discussion.}
\end{instructorNotes}



\end{document}