%\documentclass [handout]{ximera}
\documentclass[nooutcomes] {ximera}



\usepackage{gensymb}
\usepackage{tabularx}
\usepackage{mdframed}
\usepackage{pdfpages}
%\usepackage{chngcntr}

\let\problem\relax
\let\endproblem\relax

\newcommand{\property}[2]{#1#2}




\newtheoremstyle{SlantTheorem}{\topsep}{\fill}%%% space between body and thm
 {\slshape}                      %%% Thm body font
 {}                              %%% Indent amount (empty = no indent)
 {\bfseries\sffamily}            %%% Thm head font
 {}                              %%% Punctuation after thm head
 {3ex}                           %%% Space after thm head
 {\thmname{#1}\thmnumber{ #2}\thmnote{ \bfseries(#3)}} %%% Thm head spec
\theoremstyle{SlantTheorem}
\newtheorem{problem}{Problem}[]

%\counterwithin*{problem}{section}



%%%%%%%%%%%%%%%%%%%%%%%%%%%%Jenny's code%%%%%%%%%%%%%%%%%%%%

%%% Solution environment
%\newenvironment{solution}{
%\ifhandout\setbox0\vbox\bgroup\else
%\begin{trivlist}\item[\hskip \labelsep\small\itshape\bfseries Solution\hspace{2ex}]
%\par\noindent\upshape\small
%\fi}
%{\ifhandout\egroup\else
%\end{trivlist}
%\fi}
%
%
%%% instructorIntro environment
%\ifhandout
%\newenvironment{instructorIntro}[1][false]%
%{%
%\def\givenatend{\boolean{#1}}\ifthenelse{\boolean{#1}}{\begin{trivlist}\item}{\setbox0\vbox\bgroup}{}
%}
%{%
%\ifthenelse{\givenatend}{\end{trivlist}}{\egroup}{}
%}
%\else
%\newenvironment{instructorIntro}[1][false]%
%{%
%  \ifthenelse{\boolean{#1}}{\begin{trivlist}\item[\hskip \labelsep\bfseries Instructor Notes:\hspace{2ex}]}
%{\begin{trivlist}\item[\hskip \labelsep\bfseries Instructor Notes:\hspace{2ex}]}
%{}
%}
%% %% line at the bottom} 
%{\end{trivlist}\par\addvspace{.5ex}\nobreak\noindent\hung} 
%\fi
%
%


\let\instructorNotes\relax
\let\endinstructorNotes\relax
%%% instructorNotes environment
\ifhandout
\newenvironment{instructorNotes}[1][false]%
{%
\def\givenatend{\boolean{#1}}\ifthenelse{\boolean{#1}}{\begin{trivlist}\item}{\setbox0\vbox\bgroup}{}
}
{%
\ifthenelse{\givenatend}{\end{trivlist}}{\egroup}{}
}
\else
\newenvironment{instructorNotes}[1][false]%
{%
  \ifthenelse{\boolean{#1}}{\begin{trivlist}\item[\hskip \labelsep\bfseries {\Large Instructor Notes: \\} \hspace{\textwidth} ]}
{\begin{trivlist}\item[\hskip \labelsep\bfseries {\Large Instructor Notes: \\} \hspace{\textwidth} ]}
{}
}
{\end{trivlist}}
\fi


%% Suggested Timing
\newcommand{\timing}[1]{{\bf Suggested Timing: \hspace{2ex}} #1}




\hypersetup{
    colorlinks=true,       % false: boxed links; true: colored links
    linkcolor=blue,          % color of internal links (change box color with linkbordercolor)
    citecolor=green,        % color of links to bibliography
    filecolor=magenta,      % color of file links
    urlcolor=cyan           % color of external links
}



\title{Where Are We?}
\begin{document}
\begin{abstract}
The purpose of this activity is to think about how children might locate something along a wall (line), on a floor (plane), or in a room (space).  What information is needed in order to communicate location to someone who is not in your room?  This will help us, as teachers, to figure out the essential elements of any coordinate system.  Let's begin from the child's point of view, rather than jumping into a coordinate system right away.
\end{abstract}
\maketitle



\begin{problem}

Assume your workspace (desk or bed) is along one wall of a room.  How could you describe to a person who can't see the room where along the wall you are in your room, using numbers?  What is the least information that would communicate your location? 

%\begin{solution}
%Answers will vary.  Essentially, you need a ``starting point'' or zero, and a unit length.
%\end{solution}
\end{problem}

\begin{problem}
Now assume you can stand anywhere on the floor in the room that contains your workspace.  How could you describe where you are using numbers?  What is the least information that would communicate your location? How is this question different from the previous one?

%\begin{solution}
%Answers will vary.  Essentially, you need a ``starting point'' or origin, a unit length, and a choice of two perpendicular directions or axes.
%\end{solution}
\end{problem}



\begin{problem}
Describe the location of at least 5 points on the floor of your workspace using your system from number 2.
\end{problem}

\begin{problem}
Now try to describe any point in the air in the room in as many ways as possible. How would your system need to change?
%\begin{solution}
%Answers will vary.  Essentially, you need a ``starting point'' or origin, a unit length, and a choice of three perpendicular directions or axes.
%\end{solution}
\end{problem}

\pagebreak

\begin{problem}
Assume you have a grid laid out in the Cartesian system.  Describe and draw ALL locations where:
\begin{enumerate}
\item The y-coordinate is 5.
\item The x-coordinate is -3.
\item The sum of the coordinates is 10.
%\item The product of the coordinates is 24.
\item The y-coordinate is the same as the x-coordinate.
\item The y-coordinate is double the x-coordinate.
\item The y-coordinate is 3 less than double the x-coordinate.
%\item The x-coordinate is the square of the y-coordinate.
\end{enumerate}

%\begin{solution}
%    \begin{enumerate}
%    \item This is the line $y = 5$, a horizontal line 5 units above the $x$-axis.
%    \item This is the line $x = -3$, a vertical line 3 units to the left of the $y$-axis.
%    \item This is $x + y = 10$ or $y = 10-x$, a line whose slope is $-1$ and whose $y$-intercept is 10.
%   % \item This is $xy = 24$ or $y = 24/x$, a hyperbola.
%    \item This is $y = x$, a line with slope 1 and $y$-intercept 0.
%    \item This is $y = 2x$, a line with slope 2 and $y$-intercept 0.
%    \item This is $y = 2x-3$, a line with slope 2 and $y$-intercept $-3$.
% %   \item This is $x = y^2$, which looks like a parabola ``turned on its side'' beginning at the origin and opening over the positive $x$-axis.
%    \end{enumerate}
%\end{solution}
\end{problem}


\begin{problem}
Let's return to the earlier activities, I Walk the Line and Gertrude Gumchewer.
    \begin{enumerate}
        \item Graph your solutions to Freelance Freddy, Slimy Sam, and Gertrude.  Does it make sense to ``connect the dots'' on each graph?
        \item Write an equation that describes each story in terms of the coordinates that you used.
        \item How are your equations and graphs alike? How are they different?
    \end{enumerate}    
%\begin{solution}

%\end{solution}
\end{problem}



\pagebreak

\begin{instructorNotes}

{\bf Main goal:} Students develop the idea of a coordinate system by trying to describe locations in a room using minimal information.


{\bf Overall picture:}


\begin{enumerate}
\item Students think about our usual coordinate system from a new perspective, i.e., beginning from the need to locate something in the plane or in space.
\item  Our intent in this activity is to get students to think about the essential features of any location system: reference point, direction, unit of distance. Be sure to highlight this in your discussion!
\item To shorten the time on problem 5, you can choose a subset of the parts for students to work on. See the timing section below. We want students to describe the locations in Problem 5 using explanations of the meaning of coordinates, rather than through memorized formulae. In other words, we want students to start thinking about a graph as the set of points which satisfies some condition.
\item  The last question is intended to connect this activity's ideas to the work done in I Walk the Line.  Now we do want students to think about the independent and dependent variables (without naming them specifically), plot these to provide a second representation of the charts done earlier, and finally find an equation for each situation as a third representation of the same relationship. If we have already discussed equations, students can use the same ideas as with problem 5 to plot their points.
\end{enumerate}





{\bf Good language:}


{\bf Suggested timing:} This activity should take approximately one class period. Give students about 10-15 minutes to think about the first four problems where they develop the coordinate system. Take about 15 minutes to discuss their ideas as a class, highlighting the important features of a coordinate system. Then give students 10 minutes to think about the last two problems and discuss as much as you have time for. Be sure to have students give presentations for at least a few of the graphs in problem 5 (a nice subset is (b), (c), and (e)).

\end{instructorNotes}

\end{document}