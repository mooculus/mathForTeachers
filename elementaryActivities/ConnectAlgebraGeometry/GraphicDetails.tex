\documentclass[nooutcomes,noauthor,handout]{ximera}
\usepackage{gensymb}
\usepackage{tabularx}
\usepackage{mdframed}
\usepackage{pdfpages}
%\usepackage{chngcntr}

\let\problem\relax
\let\endproblem\relax

\newcommand{\property}[2]{#1#2}




\newtheoremstyle{SlantTheorem}{\topsep}{\fill}%%% space between body and thm
 {\slshape}                      %%% Thm body font
 {}                              %%% Indent amount (empty = no indent)
 {\bfseries\sffamily}            %%% Thm head font
 {}                              %%% Punctuation after thm head
 {3ex}                           %%% Space after thm head
 {\thmname{#1}\thmnumber{ #2}\thmnote{ \bfseries(#3)}} %%% Thm head spec
\theoremstyle{SlantTheorem}
\newtheorem{problem}{Problem}[]

%\counterwithin*{problem}{section}



%%%%%%%%%%%%%%%%%%%%%%%%%%%%Jenny's code%%%%%%%%%%%%%%%%%%%%

%%% Solution environment
%\newenvironment{solution}{
%\ifhandout\setbox0\vbox\bgroup\else
%\begin{trivlist}\item[\hskip \labelsep\small\itshape\bfseries Solution\hspace{2ex}]
%\par\noindent\upshape\small
%\fi}
%{\ifhandout\egroup\else
%\end{trivlist}
%\fi}
%
%
%%% instructorIntro environment
%\ifhandout
%\newenvironment{instructorIntro}[1][false]%
%{%
%\def\givenatend{\boolean{#1}}\ifthenelse{\boolean{#1}}{\begin{trivlist}\item}{\setbox0\vbox\bgroup}{}
%}
%{%
%\ifthenelse{\givenatend}{\end{trivlist}}{\egroup}{}
%}
%\else
%\newenvironment{instructorIntro}[1][false]%
%{%
%  \ifthenelse{\boolean{#1}}{\begin{trivlist}\item[\hskip \labelsep\bfseries Instructor Notes:\hspace{2ex}]}
%{\begin{trivlist}\item[\hskip \labelsep\bfseries Instructor Notes:\hspace{2ex}]}
%{}
%}
%% %% line at the bottom} 
%{\end{trivlist}\par\addvspace{.5ex}\nobreak\noindent\hung} 
%\fi
%
%


\let\instructorNotes\relax
\let\endinstructorNotes\relax
%%% instructorNotes environment
\ifhandout
\newenvironment{instructorNotes}[1][false]%
{%
\def\givenatend{\boolean{#1}}\ifthenelse{\boolean{#1}}{\begin{trivlist}\item}{\setbox0\vbox\bgroup}{}
}
{%
\ifthenelse{\givenatend}{\end{trivlist}}{\egroup}{}
}
\else
\newenvironment{instructorNotes}[1][false]%
{%
  \ifthenelse{\boolean{#1}}{\begin{trivlist}\item[\hskip \labelsep\bfseries {\Large Instructor Notes: \\} \hspace{\textwidth} ]}
{\begin{trivlist}\item[\hskip \labelsep\bfseries {\Large Instructor Notes: \\} \hspace{\textwidth} ]}
{}
}
{\end{trivlist}}
\fi


%% Suggested Timing
\newcommand{\timing}[1]{{\bf Suggested Timing: \hspace{2ex}} #1}




\hypersetup{
    colorlinks=true,       % false: boxed links; true: colored links
    linkcolor=blue,          % color of internal links (change box color with linkbordercolor)
    citecolor=green,        % color of links to bibliography
    filecolor=magenta,      % color of file links
    urlcolor=cyan           % color of external links
}

\title{The Graphic Details}
\begin{document}
\begin{abstract}\end{abstract}
\maketitle


\begin{problem}
For each of the situations below, pick the graph that most reasonably
reflects the situation and the variables involved.

\begin{enumerate}
\item The daily high temperature recorded in Chicago from January to
  December. Graph $A$ is the top graph in red, graph $B$ is the middle graph 
  in blue, and graph $C$ is the lowest graph in green.

\begin{center}
\desmos{eoapji5nwg}{500}{500}
\end{center}

\item The number of gallons of milk you can buy with $\$5$ as the cost per gallon of
  milk increases. Graph $A$ is the curved graph in red, graph $B$ is the downward graph 
  in blue, and graph $C$ is the upward graph in green.

\begin{center}
\desmos{e4pxcxdeay}{500}{500}
\end{center}

\newpage


\item A hot air balloon rises, stays in the sky for an hour, then descends.

Graph A: \desmos{sywstxksds}{150}{150}
Graph B: \desmos{7gtzw9vkif}{150}{150}
Graph C: \desmos{mtuixrlt2i}{150}{150}
Graph D: \desmos{w5gniwguxc}{150}{150}

%\begin{figure}[h]
%\begin{center}
%\includegraphics[scale=0.3]{graphics/1c.pdf}
%\end{center}
%\end{figure}
%
%\item A child climbs up to the top of a slide and then slides
%  down it
%
%\begin{figure}[h]
%\begin{center}
%\includegraphics[scale=0.3]{graphics/1d.pdf}
%\end{center}
%\end{figure}
%
%\newpage
%
%\item A girl takes a ride on a Ferris wheel
%
%\begin{figure}[h]
%\begin{center}
%\includegraphics[scale=0.3]{graphics/1e.pdf}
%\end{center}
%\end{figure}
\end{enumerate}
%
%\begin{solution}
%\begin{enumerate}
%\item The lowest line is the best.
%\item The curved line is the best.
%\item Option (b).
%\item Option (d).
%\item Option (b).
%\end{enumerate}
%\end{solution}





\end{problem}



%\begin{problem}
%Water is poured at a constant rate into the three containers shown
%below. Which graph corresponds to which container?
%
%\begin{figure}[h]
%\begin{center}
%\includegraphics[scale=0.3]{graphics/2}
%\end{center}
%\end{figure}
%
%\begin{solution}
%    $(a) \leftrightarrow (ii)$ \\
%    $(b) \leftrightarrow (i)$\\
%    $(c) \leftrightarrow (iii)$
%\end{solution}
%
%
%\end{problem}
%
%\newpage
%
%\begin{problem}
%Water is poured at a constant rate into the six containers shown
%below. Which graph corresponds to which container?
%
%\begin{figure}[h]
%\begin{center}
%\includegraphics[scale=0.3]{graphics/3.pdf}
%\end{center}
%\end{figure}
%
%\begin{solution}
%    $(1) \leftrightarrow (F)$\\
%    $(2) \leftrightarrow (C)$\\
%    $(3) \leftrightarrow (E)$\\
%    $(4) \leftrightarrow (B)$\\
%    $(5) \leftrightarrow (D)$\\
%    $(6) \leftrightarrow (A)$
%\end{solution}
%
%\end{problem}
%
%\newpage
%
%\begin{problem}
%Match each graph with the situations described below:
%
%\begin{enumerate}
%
%\item The temperature of a frozen dinner from 30 minutes before it is
%  removed from the freezer until it is removed from the microwave and
%  placed on the table (consider time 0 to be the moment the dinner is
%  removed from the freezer).
%\item The value of a well--maintained 1970 Volkswagen Beetle from the
%  time it was purchased to the present.
%\item The level of water in the bathtub from the time you begin to
%  fill it to the time it is completely empty after your bath.
%\item Profit in terms of the number of items sold.
%\item The height of a baseball in terms of time from when it is thrown
%  straight up to the time it hits the ground.
%\item The speed of the baseball in terms of time from when it is
%  thrown straight up to the time it hits the ground.
%\end{enumerate}
%
%\begin{figure}[h]
%\begin{center}
%\includegraphics[scale=0.3]{graphics/4}
%\end{center}
%\end{figure}
%
%\begin{solution}
%\begin{enumerate}
%\item 2
%\item 4
%\item 5
%\item 6
%\item 5
%\item 3
%\end{enumerate}
%\end{solution}
%
%
%
%\end{problem}
\pagebreak
\begin{instructorNotes} 

{\bf Main goal:} Students connect the properties of a graph (shape, behavior, labeling on axes, etc) to physical situations without using a formula.


{\bf Overall picture:}

Here we begin to inspect graphical representations of functions or expressions rather than algebraic or tabular representations.   This is a connection between algebra and geometry! You will want to point out that a graph can be dealt with locally, as each point is a special case of the relationship between the variables, as well as globally.  The global point of view tells the story of what happens to one variable as the other variable(s) change.

For the first problem:
\begin{enumerate}
	\item Overall, make sure students are emphasizing their connection between the story situation and the graph. What is it about the story that they are seeing in the graph? What is it about the graph that matches with the story?
    \item \#1a might be controversial between graph (a) and (c). If students put several years of data together, you can point out the sinusoidal nature of graph (a), which helps students to distinguish which case is the correct one! You don't need to use the ``sinusoidal'' terminology to make this point.
    \item All throughout, one key is paying attention to what each axis represents.  Some of them might be true if the vertical axis changed its meaning.
    \item \#1b has been difficult until they try numbers (or extreme cases of cost big and small).  \# of apples = 5/cost. Also, be sure to help students recognize that this is an unusual situation where the milk is not pre-packaged; we can buy however much we want!
    \item \#1d's graphs (including graph d) have been described by some students as unrealistic, although d indicates the relative briefness of the time going down versus climbing.
    \item \#1e brings out the need for the concept of function:  can only be at one place at one time (although there are several times for the same height- not 1-1).  Might bring up $f(x)$ notation. Could also bring up parametric curves that aren't functions in asking where we would use non-function graphs (perhaps using TI-83).
\end{enumerate}

For the second problem:
\begin{enumerate}
    \item All of these, especially \#2-4, bring out the idea that it's important not only to study if a quantity increases or decreases, but how it does those things.  Not all the world is linear! (Idea of concavity and studying when rates change could be brought up here).
    \item For \#2, might point out that the horizontal axis represents volume while on \#3 it represents time (although both march at a steady rate).
\end{enumerate}

For the last problem:\\
This problem does not label axes.  The students should do that (Has the only graph of the activity with negative $y$-values).





{\bf Good language:} We want students to be as specific as possible when they are describing what they are looking at with these problems. After all, they will teach children who may have never seen a graph before! We emphasize the most basic descriptions we can, and encourage students when they write up solutions to these problems to think about how to translate their ``pointing'' in person at the graph to their writing.



{\bf Suggested timing:} We use two days for this activity: usually one day is spent on parts (a) and (b) of Problem 1, and then on the second day we finish the first problem and perhaps work on part of the second. On the first day, give students about 10 minutes to start with problem 1, and then use the rest of the time in discussion and debate. On the second day, give students about 15 minutes to continue where they left off on the first day, and then use the rest of the class in presentations, discussion, and debate.


\end{instructorNotes}

\end{document}

