\documentclass[nooutcomes,noauthor,handout]{ximera}

\graphicspath{
  {./}
  {graphics/}
  {../graphics/}
}

\usepackage{chngcntr}

\let\question\relax
\let\endquestion\relax




\newtheoremstyle{SlantTheorem}{\topsep}{\fill}%%% space between body and thm
%\newtheoremstyle{SlantTheorem}{\topsep}{\topsep}%%% space between body and thm
 {\slshape}                      %%% Thm body font
 {}                              %%% Indent amount (empty = no indent)
 {\bfseries\sffamily}            %%% Thm head font
 {}                              %%% Punctuation after thm head
 {3ex}                           %%% Space after thm head
 {\thmname{#1}\thmnumber{ #2}\thmnote{ \bfseries(#3)}}%%% Thm head spec
\theoremstyle{SlantTheorem}
\newtheorem{question}{Question}
\counterwithin*{question}{section}



\let\instructorNotes\relax
\let\endinstructorNotes\relax
%%% instructorNotes environment
\ifhandout
\newenvironment{instructorNotes}[1][false]%
{%
\def\givenatend{\boolean{#1}}\ifthenelse{\boolean{#1}}{\begin{trivlist}\item}{\setbox0\vbox\bgroup}{}
}
{%
\ifthenelse{\givenatend}{\end{trivlist}}{\egroup}{}
}
\else
\newenvironment{instructorNotes}[1][false]%
{%
  \ifthenelse{\boolean{#1}}{\begin{trivlist}\item[\hskip \labelsep\bfseries {\Large Instructor Notes: \\} \hspace{\textwidth} ]}
{\begin{trivlist}\item[\hskip \labelsep\bfseries {\Large Instructor Notes: \\} \hspace{\textwidth} ]}
{}
}
{\end{trivlist}}
\fi


%% Suggested Timing
\newcommand{\timing}[1]{{\bf Suggested Timing: \hspace{2ex}} #1}

\title{The Graphic Details}
\begin{document}
\begin{abstract}\end{abstract}
\maketitle

Teachers: \link[Amplify classroom version]{https://classroom.amplify.com/activity/68df36e0319cf52e6c07ccdb?utm_campaign=share&utm_content=activity}

Students: for the Amplify Classroom version, your teacher should give you a link.




\begin{problem}
For each of the situations below, pick the graph that most reasonably
reflects the situation and the variables involved.

\begin{enumerate}
\item The daily high temperature recorded in Chicago from January to
  December. Graph $A$ is the top graph in red, graph $B$ is the middle graph 
  in blue, and graph $C$ is the lowest graph in green.

\begin{center}
\desmos{eoapji5nwg}{500}{500}
\end{center}

\item The number of gallons of milk you can buy with $\$5$ as the cost per gallon of
  milk increases. Graph $A$ is the curved graph in red, graph $B$ is the downward graph 
  in blue, and graph $C$ is the upward graph in green.

\begin{center}
\desmos{e4pxcxdeay}{500}{500}
\end{center}


\item A hot air balloon rises, stays in the sky for an hour, then descends.

Graph A: \desmos{sywstxksds}{300}{300} \\
Graph B: \desmos{7gtzw9vkif}{300}{300} \\
Graph C: \desmos{mtuixrlt2i}{300}{300} \\
Graph D: \desmos{w5gniwguxc}{300}{300} \\


\item A child jumps on a trampoline.

Graph A: \desmos{k5eyu7j2gf}{300}{300} \\
Graph B: \desmos{eqkdvaheek}{300}{300} \\
Graph C: \desmos{iynfmpaj8a}{300}{300} \\
Graph D: \desmos{tf7otwtyt6}{300}{300} \\


\item A bird flies into a tree and lands.

Graph A: \desmos{nakvmxkfvy}{300}{300} \\
Graph B: \desmos{imp1hsvuah}{300}{300} \\
Graph C: \desmos{situ1qeavq}{300}{300} \\
Graph D: \desmos{3i9euc6kei}{300}{300} \\



\end{enumerate}

\end{problem}

\newpage

\begin{problem}
\begin{enumerate}
\item Water is poured at a constant rate into the container below which has a cross-section shaped like a V. Which graph below best represents the height of the water over time?

\begin{image}
\begin{tikzpicture}
\draw[thick] (0,0)--(1,-3)--(1.5, -3)--(2.5, 0);
\end{tikzpicture}
\end{image}

Graph A: \desmos{jbihjg68tq}{300}{300} \\
Graph B: \desmos{ev62l6soqq}{300}{300} \\
Graph C: \desmos{axg2ampezs}{300}{300} \\
Graph D: \desmos{oj8qp5jdrv}{300}{300} \\

\item Water is poured at a constant rate into the container below. The cross-section of this container is an S shape mirrored on each side of the vase. Draw a graph which represents the height of the water over time.

\begin{image}
\begin{tikzpicture}
\draw[thick] (0,0)--(1,0);
\draw[thick, smooth] plot coordinates {(1,0) (1.5, 1) (1.3, 2) (2,3)};
\draw[thick, smooth] plot coordinates {(0,0) (-0.5, 1) (-0.3, 2) (-1,3)};
\end{tikzpicture}
\end{image}
\end{enumerate}
\end{problem}


\begin{problem}
For each situation below, draw a graph that would represent the situation. Be sure to label your axes appropriately and explain why you drew your graph the way you did.

\begin{enumerate}
	\item The temperature of a frozen dinner from 30 minutes before it is removed from the freezer until it is removed from the microwave and placed on the table.
	\item The speed of a baseball in terms of time from when it is thrown straight up to the time it hits the ground.
	\item The value of a well-maintained classic car from the time it was originally purchased new until today.
\end{enumerate}
\end{problem}



%\begin{problem}
%Match each graph with the situations described below:
%
%\begin{enumerate}
%
%\item The temperature of a frozen dinner from 30 minutes before it is
%  removed from the freezer until it is removed from the microwave and
%  placed on the table (consider time 0 to be the moment the dinner is
%  removed from the freezer).
%\item The value of a well--maintained 1970 Volkswagen Beetle from the
%  time it was purchased to the present.
%\item The level of water in the bathtub from the time you begin to
%  fill it to the time it is completely empty after your bath.
%\item Profit in terms of the number of items sold.
%\item The height of a baseball in terms of time from when it is thrown
%  straight up to the time it hits the ground.
%\item The speed of the baseball in terms of time from when it is
%  thrown straight up to the time it hits the ground.
%\end{enumerate}

\pagebreak
\begin{instructorNotes} 

{\bf Main goal:} Students connect the properties of a graph (shape, behavior, labeling on axes, etc) to physical situations without using a formula.


{\bf Overall picture:}

Here we begin to inspect graphical representations of functions or expressions rather than algebraic or tabular representations.   This is a connection between algebra and geometry! You will want to point out that a graph can be dealt with locally, as each point is a special case of the relationship between the variables, as well as globally.  The global point of view tells the story of what happens to one variable as the other variable(s) change.


\begin{enumerate}
	\item Overall, make sure students are emphasizing their connection between the story situation and the graph. What is it about the story that they are seeing in the graph? What is it about the graph that matches with the story?
    \item \#1a might be controversial between the upper and lower graphs. If students put several years of data together, you can point out the sinusoidal nature of the lowest graph, which helps students to distinguish which case is the correct one! You don't need to use the ``sinusoidal'' terminology to make this point.
    \item All throughout, one key is paying attention to what each axis represents.  Some graphs might be more or less appropriate based on what the axes represent. Students should clarify in each case.
    \item The milk problem has been difficult until they try numbers (or extreme cases of cost big and small).  \# of gallons = 5/cost. Also, be sure to help students recognize that this is an unusual situation where the milk is not pre-packaged; we can buy however much we want!
    \item The hot air balloon brings out the need for the concept of function:  can only be at one place at one time (although there are several times for the same height- not 1-1).  Might bring up $f(x)$ notation. Could also bring up parametric curves that aren't functions in asking where we would use non-function graphs (perhaps using TI-83).
    \item In the vases problem, we bring out the idea that it's important not only to study if a quantity increases or decreases, but how it does those things.  Not all the world is linear! (Idea of concavity and studying when rates change could be brought up here).
\end{enumerate}



{\bf Good language:} We want students to be as specific as possible when they are describing what they are looking at with these problems. After all, they will teach children who may have never seen a graph before! We emphasize the most basic descriptions we can, and encourage students when they write up solutions to these problems to think about how to translate their ``pointing'' in person at the graph to their writing.



{\bf Suggested timing:} We use two days for this activity: usually one day is spent on parts (a) and (b) of Problem 1, and then on the second day we finish the first problem and perhaps work on part of the second. On the first day, give students about 10 minutes to start with problem 1, and then use the rest of the time in discussion and debate. On the second day, give students about 15 minutes to continue where they left off on the first day, and then use the rest of the class in presentations, discussion, and debate.


\end{instructorNotes}

\end{document}

