\documentclass[nooutcomes, noauthor, handout]{ximera}
\usepackage{gensymb}
\usepackage{tabularx}
\usepackage{mdframed}
\usepackage{pdfpages}
%\usepackage{chngcntr}

\let\problem\relax
\let\endproblem\relax

\newcommand{\property}[2]{#1#2}




\newtheoremstyle{SlantTheorem}{\topsep}{\fill}%%% space between body and thm
 {\slshape}                      %%% Thm body font
 {}                              %%% Indent amount (empty = no indent)
 {\bfseries\sffamily}            %%% Thm head font
 {}                              %%% Punctuation after thm head
 {3ex}                           %%% Space after thm head
 {\thmname{#1}\thmnumber{ #2}\thmnote{ \bfseries(#3)}} %%% Thm head spec
\theoremstyle{SlantTheorem}
\newtheorem{problem}{Problem}[]

%\counterwithin*{problem}{section}



%%%%%%%%%%%%%%%%%%%%%%%%%%%%Jenny's code%%%%%%%%%%%%%%%%%%%%

%%% Solution environment
%\newenvironment{solution}{
%\ifhandout\setbox0\vbox\bgroup\else
%\begin{trivlist}\item[\hskip \labelsep\small\itshape\bfseries Solution\hspace{2ex}]
%\par\noindent\upshape\small
%\fi}
%{\ifhandout\egroup\else
%\end{trivlist}
%\fi}
%
%
%%% instructorIntro environment
%\ifhandout
%\newenvironment{instructorIntro}[1][false]%
%{%
%\def\givenatend{\boolean{#1}}\ifthenelse{\boolean{#1}}{\begin{trivlist}\item}{\setbox0\vbox\bgroup}{}
%}
%{%
%\ifthenelse{\givenatend}{\end{trivlist}}{\egroup}{}
%}
%\else
%\newenvironment{instructorIntro}[1][false]%
%{%
%  \ifthenelse{\boolean{#1}}{\begin{trivlist}\item[\hskip \labelsep\bfseries Instructor Notes:\hspace{2ex}]}
%{\begin{trivlist}\item[\hskip \labelsep\bfseries Instructor Notes:\hspace{2ex}]}
%{}
%}
%% %% line at the bottom} 
%{\end{trivlist}\par\addvspace{.5ex}\nobreak\noindent\hung} 
%\fi
%
%


\let\instructorNotes\relax
\let\endinstructorNotes\relax
%%% instructorNotes environment
\ifhandout
\newenvironment{instructorNotes}[1][false]%
{%
\def\givenatend{\boolean{#1}}\ifthenelse{\boolean{#1}}{\begin{trivlist}\item}{\setbox0\vbox\bgroup}{}
}
{%
\ifthenelse{\givenatend}{\end{trivlist}}{\egroup}{}
}
\else
\newenvironment{instructorNotes}[1][false]%
{%
  \ifthenelse{\boolean{#1}}{\begin{trivlist}\item[\hskip \labelsep\bfseries {\Large Instructor Notes: \\} \hspace{\textwidth} ]}
{\begin{trivlist}\item[\hskip \labelsep\bfseries {\Large Instructor Notes: \\} \hspace{\textwidth} ]}
{}
}
{\end{trivlist}}
\fi


%% Suggested Timing
\newcommand{\timing}[1]{{\bf Suggested Timing: \hspace{2ex}} #1}




\hypersetup{
    colorlinks=true,       % false: boxed links; true: colored links
    linkcolor=blue,          % color of internal links (change box color with linkbordercolor)
    citecolor=green,        % color of links to bibliography
    filecolor=magenta,      % color of file links
    urlcolor=cyan           % color of external links
}
\title{I Walk The Line}

\begin{document}
\begin{abstract}
\end{abstract}
\maketitle


\begin{problem}
 \emph{Free-Lance Freddy works a different job each day.  At each job, he earns an hourly rate that depends on the
job.  He also carries some spare cash for lunch.  To make his
customers sweat, Freddy keeps a meter on his belt telling how much
money they currently owe (with his lunch money added in).  On Monday, 3 hours into his work as a gourmet burger flipper,
  Freddy's meter reads $\$42$. 7 hours into his work, his meter reads
  $\$86$.  If he works for 9 hours, how much money will he have?  When
  will he have $\$75$?}
  \vskip 3in
  
%%\item On Tuesday, Freddy is President of the United States.  After
%  2.53 hours of work, his meter reads $\$863.15$ and after 5.71 hours
%  of work, his meter reads $\$$1349.78.  If he works for 10.34 hours,
%  how much money will he have?  How much time will he be in office to
%  have $\$$1759.21?
%\pagebreak


\end{problem}


\pagebreak
\begin{problem}
\emph{Slimy Sam is on the lam from the law.  Being not-too-smart, he drives
the clunker of a car he stole east on I-70 across Ohio.  Because the
car can only go a maximum of 52 miles per hour, he floors it all the
way from where he stole the car (just now at the Rest Area 5 miles
west of the Indiana line) and goes as far as he can before running out
of gas 3.78 hours from now.}
\begin{enumerate}[label=(\roman*)]
 \item{Where will he be 3 hours after stealing the car?
 \item Where will he be when he runs out of gas and is arrested?
 \item  When will he be at mile marker 99 (east of Indiana)?
\item When will he be at mile marker 71.84?}
\end{enumerate}

\pagebreak

\begin{problem} Consider the problem about Freelance Freddy (problem 1).  Now that you have solved this problem by yourself, compare your work to each of the student methods described below.  

\begin{enumerate}
\item Autumn says that she cannot find the answer to this question because she does not know how much lunch money Freddy had.
\item Emilio computed the hourly rate to be $\$14$ and then concluded that the answer should be $\$126$.
\item Cassie used a chart to find an hourly rate of $\$11$ and then used the chart to find the lunch money amount to be $\$9$.

\end{enumerate}

*Consider each student's approach to the problem.  Do you agree or disagree with the student's method? What might their reasoning have been?  What is the correct answer?
\end{problem}

\pagebreak
\begin{problem}
Consider the problem about Slimy Sam.  Now that you have solved this problem by yourself, compare your work to each of the student methods described below.  

\begin{enumerate}

\item Emilio used a chart to solve this problem.  Construct a chart that Emilio might have used.  Explain what his reasoning might have been.

\item Another student, Autumn, used an equation. Write an equation that Autumn might have used.   Explain what her reasoning might have been.

\end{enumerate}

*What do you think of their methods?  Can both methods be used to obtain the correct answer?  Is there another method? Is there a best method? 
\end{problem}


\end{problem}

\pagebreak

\begin{problem}
How are the first two situations alike (Freddy and Sam)?  How are they different?  Did you solve them in the same way?  Why or why not?
\end{problem}

 \vskip 3in


\begin{problem} 
Now let's try to generalize these two situations.
\begin{enumerate}
    \item Suppose that Free Lance Freddy's meter shows that he has earned $\$d1 $ after working for $h1$ hours, and his meter shows $\$d2 $ after working for $h2 $ hours. How much money, $y$, will his meter show he has earned after working for a total of $x$ hours?
    \item Suppose that Slimy Sam's car can only go a maximum of S miles per hour all the way from where he stole the car (just now at the Rest Area $P$ miles west of the Indiana line) and goes as far as he can before running out of gas T hours from now.
\begin{enumerate}[label=(\roman*)]
 \item Where will he be $x$ hours after stealing the car?
 \item Where will he be when he runs out of gas and is arrested?
 \item  When will he be at mile marker $y$ (east of Indiana)?
%\item When will he be at mile marker 71.84?}
\end{enumerate}

\end{enumerate}


\end{problem}

\pagebreak
\begin{instructorNotes}


{\bf Main Goal:} We build on the idea of an arithmetic sequence, using an approach that could work with grade 4-6 students.  We intend to emphasize  recognition of a constant pattern of growth without reference to graphing or equations and connect to Gertrude the Gumchewer.


{\bf Overall picture:}

Day 1:

\begin{enumerate}
\item These problems are again presented to encourage problem-solving; presented in another way the students might directly attack with memorized formulas, which is not our goal. We are looking for students to develop these formulas through their problem-solving work.
\item If students finish early, encourage them to think of other solution methods rather than moving on to the student solutions. We would like to complete this discussion before we move to those ideas.
\item Encourage students to use the same methods from the Gertrude activity: reasoning, charts or tables, and equations.
\item With Freelance Freddy, the issue of the lunch money can be very confusing. The most common mistake is to include the lunch money in the hourly rate; encourage students to think about what their calculations actually mean! As usual, the students should be explaining their work in terms of the meanings of the underlying operations. The notion of ``start plus change'' can help here, especially if we think of the ``start'' as something other than zero.
\item With Slimy Sam, the most common issue is with the description of the problem and the idea that Sam starts west of the border (in Indiana). Encourage students to draw a picture of the situation. You may also need to help students understand that the mile-markers start at zero at the border and increase as you drive east.
\item Our intent in this activity is to get students looking at linear relationships with fresh eyes.  Toward that end, we discourage students from going on automatic pilot and slapping down the $y = mx + b$ form.  Without getting into equations themselves right now, we want to lay the foundation for the two most common forms of linear equations, point-slope (Freddy) and $y$-intercept (Slimy Sam).  Some students might recognize this on their own.
\end{enumerate}


Day 2:
\begin{enumerate}
\item The problems with sample student solutions could possibly go quickly if all of these methods have already been discussed! These are all solutions we have seen in the past from our own students, so it's good to help our students interact with these ideas if they were too shy to bring them up for the whole class.
\item  The comparison problem should help wrap-up and generate some comparison of methods and problem contexts.  Class discussion should emphasize  that both story situations involve a constant rate of change, that this is additive and depends on the amount of change in the independent variable, and that the two stories have different starting information.  
\item It's good to discuss with problem 3 the difference between the Gertrude situation (a fixed number of whole pieces of gum per day) and the Sam situation (a fixed number of miles, but partial miles are okay).
\item The generalization problem will be a struggle for some students with the variables, but this is also an extension for if we have time! If we make it through these problems, some students might recognize the equations that they've made by the end.
\end{enumerate}


Wrap-up: this should conclude most of our work with sequences, so you will want to highlight that we have worked mostly with arithmetic sequences, but there are more types of sequences out there. It's worth a few minutes potentially asking students if they know any sequences which are not arithmetic.



{\bf Good language:} We would like to introduce the terminology of an arithmetic sequence as we compare these problems with our work on Gertrude. As we define an arithmetic sequence, we'll also want to discuss sequences which are not arithmetic, as above!

Continue to help students connect their algebraic work to the story situation and use vocabulary appropriate for the children they will teach. Continue to help them identify groups and objects particularly when using multiplication.





{\bf Suggested timing:} On Day 1, we try to get through the first two problems: Freelance Freddy and  Slimy Sam. Give students about 20 minutes to start working on the problems, and then use the remaining time having students present their work at the board. On Day 2, start by giving students about 10 minutes to think through the student solutions, and then discuss as a class. Finally, either give students another 5-10 minutes to think about the connection question, or discuss as a whole class without time in groups. Finish with the generalization question if you have time!
\end{instructorNotes}



\end{document}