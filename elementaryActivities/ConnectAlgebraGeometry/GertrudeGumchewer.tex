
\documentclass{ximera}






\graphicspath{
  {./}
  {graphics/}
  {../graphics/}
}

\usepackage{chngcntr}

\let\question\relax
\let\endquestion\relax




\newtheoremstyle{SlantTheorem}{\topsep}{\fill}%%% space between body and thm
%\newtheoremstyle{SlantTheorem}{\topsep}{\topsep}%%% space between body and thm
 {\slshape}                      %%% Thm body font
 {}                              %%% Indent amount (empty = no indent)
 {\bfseries\sffamily}            %%% Thm head font
 {}                              %%% Punctuation after thm head
 {3ex}                           %%% Space after thm head
 {\thmname{#1}\thmnumber{ #2}\thmnote{ \bfseries(#3)}}%%% Thm head spec
\theoremstyle{SlantTheorem}
\newtheorem{question}{Question}
\counterwithin*{question}{section}



\let\instructorNotes\relax
\let\endinstructorNotes\relax
%%% instructorNotes environment
\ifhandout
\newenvironment{instructorNotes}[1][false]%
{%
\def\givenatend{\boolean{#1}}\ifthenelse{\boolean{#1}}{\begin{trivlist}\item}{\setbox0\vbox\bgroup}{}
}
{%
\ifthenelse{\givenatend}{\end{trivlist}}{\egroup}{}
}
\else
\newenvironment{instructorNotes}[1][false]%
{%
  \ifthenelse{\boolean{#1}}{\begin{trivlist}\item[\hskip \labelsep\bfseries {\Large Instructor Notes: \\} \hspace{\textwidth} ]}
{\begin{trivlist}\item[\hskip \labelsep\bfseries {\Large Instructor Notes: \\} \hspace{\textwidth} ]}
{}
}
{\end{trivlist}}
\fi


%% Suggested Timing
\newcommand{\timing}[1]{{\bf Suggested Timing: \hspace{2ex}} #1}

\title{Gertrude the Gumchewer }

\begin{document}

\begin{abstract}
    
\end{abstract}
\maketitle


\begin{problem}
Gertrude the Gumchewer has an addiction to \textit{Xtra Sugarloaded
  Gum}, and it's getting worse.  Each day, she goes to her always
stocked storage vault and grabs gum to chew.  At the beginning of her
habit, she chewed three pieces and then, each day, she chews 8 more
pieces than she chewed the day before to satisfy her ever-increasing
cravings.
\begin{enumerate}
\item How many pieces will she chew on the $10$th day of her habit?
\item How many pieces will she chew on the $793$rd day of her habit? How do you know you are right?
\item How many pieces will she chew on the $k$th day of her habit?
%\item How many pieces will she chew over the course of the first $793$
%  days of her habit?
\end{enumerate}
\end{problem}



\vskip 2in


\begin{problem}
Use the ideas of problem (1) to find how many days Gertie would have taken to get to 180 pieces in a day if the following shows her pattern on the first three days:
\[
19, 26, 33, \dots , 180
\]
Be ready to explain how you worked through this.  
%Give another story problem that is represented by this sum.  Make it as different as you can.
\end{problem}


\pagebreak

In the following problem, we are going to generalize our work from problems (1) and (2).  You should not just copy the work from the other problems with $m$'s, $n$'s, and $k$'s.
\begin{problem}

Assume now that Gertrude, at the beginning of her habit, chewed $m$
pieces of gum and then, each day, she chews $n$ more pieces than she
chewed the day before to satisfy her ever-increasing cravings.  How many pieces will she chew on the $kth$
  day of her habit? 
\begin{enumerate}
    \item Explain to each other in words what you would do before writing anything down.
    \item Make a chart showing how you would know how many pieces she has chewed on the 1st day, the 2nd, ..., the kth day.  Your chart entries should show any pattern that you see in the story.  (simplifying  your work too soon erases all trace of your thinking!)
    \item  Explain your formula and how you know it will work for any $m$, $n$ and $k$.  
\end{enumerate}  
 
\end{problem}


\pagebreak

\begin{instructorNotes}
{\bf Main goal:} We introduce the idea of a sequence. We use the meaning of operations and reasoning to describe sequences.


{\bf Overall picture:}

Day 1:
\begin{enumerate}
\item Students think about how to solve a new problem. This should feel like a puzzle at first, and we want to use our problem-solving skills to attack it!
\item We are looking for three main strategies: reasoning with operations, making a chart, and using an equation. Since these are different representations, we might discuss the advantages and disadvantages of each. Why might we prefer a chart? Why might we prefer an equation?
\item Using an equation should generate some good discussion: the problem is not clear as to whether Gertrude chewed 3 pieces on Day 0 or on Day 1. Either is a fine interpretation, but they result in different equations! This point should be discussed.
\item Look for students who use a recursive relationship for the first few days, and help the class contrast with an explicit form. You don't need to use this language, but the students should be able to tell the difference and talk about why each form is useful.

\item We are helping students to focus here on the underlying structure of the problem.

	\item Students should practice with their reasoning, charts, and tables in problem 2. We would again like to see explanations of all three. Getting more practice with these ideas is the main goal; if you don't get to the general case it's okay! Remember to emphasize again the advantages and disadvantages of each method -- which may be different in this case. Also be sure to connect back to the original situation with the gum-chewing as often as possible. We are really cementing the big ideas here.
	\item The final problem is another opportunity to bring these ideas together, and also to work on abstraction. Many students will be a bit uncomfortable with all the letters; just keep reminding everyone what the letters mean. You can even replace them with words if this makes students feel more comfortable.
\end{enumerate}


{\bf Good language:} Students should emphasize the meaning of any operations they use. In particular, we'd like to hear them talk about groups and identify any groupings they are using to justify multiplication. This is the thing students struggle most with when they are allowed to use an equation -- they forget to include the meaning of operations, and the groups and objects can be difficult to see, here. The combining meaning of addition should also come up!

Also be sure that students are connecting their algebra to the story situation. We don't want to get so lost in numbers and letters that we forget where these came from and why we are using them!

{\bf Suggested timing:} We aim to work through at least the first two problems, introducing lists, tables, and potentially equations. Give students about 15 minutes to work through these problems, encouraging groups to solve problem 1 in at least two different ways.
Use the next 30 minutes or so to have groups present their work, highlighting the different strategies used. Finish up with whole-class discussion on how the strategies are related.


\end{instructorNotes}




\end{document}