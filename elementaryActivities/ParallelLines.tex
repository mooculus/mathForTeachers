\documentclass{ximera}
\usepackage{gensymb}
\usepackage{tabularx}
\usepackage{mdframed}
\usepackage{pdfpages}
%\usepackage{chngcntr}

\let\problem\relax
\let\endproblem\relax

\newcommand{\property}[2]{#1#2}




\newtheoremstyle{SlantTheorem}{\topsep}{\fill}%%% space between body and thm
 {\slshape}                      %%% Thm body font
 {}                              %%% Indent amount (empty = no indent)
 {\bfseries\sffamily}            %%% Thm head font
 {}                              %%% Punctuation after thm head
 {3ex}                           %%% Space after thm head
 {\thmname{#1}\thmnumber{ #2}\thmnote{ \bfseries(#3)}} %%% Thm head spec
\theoremstyle{SlantTheorem}
\newtheorem{problem}{Problem}[]

%\counterwithin*{problem}{section}



%%%%%%%%%%%%%%%%%%%%%%%%%%%%Jenny's code%%%%%%%%%%%%%%%%%%%%

%%% Solution environment
%\newenvironment{solution}{
%\ifhandout\setbox0\vbox\bgroup\else
%\begin{trivlist}\item[\hskip \labelsep\small\itshape\bfseries Solution\hspace{2ex}]
%\par\noindent\upshape\small
%\fi}
%{\ifhandout\egroup\else
%\end{trivlist}
%\fi}
%
%
%%% instructorIntro environment
%\ifhandout
%\newenvironment{instructorIntro}[1][false]%
%{%
%\def\givenatend{\boolean{#1}}\ifthenelse{\boolean{#1}}{\begin{trivlist}\item}{\setbox0\vbox\bgroup}{}
%}
%{%
%\ifthenelse{\givenatend}{\end{trivlist}}{\egroup}{}
%}
%\else
%\newenvironment{instructorIntro}[1][false]%
%{%
%  \ifthenelse{\boolean{#1}}{\begin{trivlist}\item[\hskip \labelsep\bfseries Instructor Notes:\hspace{2ex}]}
%{\begin{trivlist}\item[\hskip \labelsep\bfseries Instructor Notes:\hspace{2ex}]}
%{}
%}
%% %% line at the bottom} 
%{\end{trivlist}\par\addvspace{.5ex}\nobreak\noindent\hung} 
%\fi
%
%


\let\instructorNotes\relax
\let\endinstructorNotes\relax
%%% instructorNotes environment
\ifhandout
\newenvironment{instructorNotes}[1][false]%
{%
\def\givenatend{\boolean{#1}}\ifthenelse{\boolean{#1}}{\begin{trivlist}\item}{\setbox0\vbox\bgroup}{}
}
{%
\ifthenelse{\givenatend}{\end{trivlist}}{\egroup}{}
}
\else
\newenvironment{instructorNotes}[1][false]%
{%
  \ifthenelse{\boolean{#1}}{\begin{trivlist}\item[\hskip \labelsep\bfseries {\Large Instructor Notes: \\} \hspace{\textwidth} ]}
{\begin{trivlist}\item[\hskip \labelsep\bfseries {\Large Instructor Notes: \\} \hspace{\textwidth} ]}
{}
}
{\end{trivlist}}
\fi


%% Suggested Timing
\newcommand{\timing}[1]{{\bf Suggested Timing: \hspace{2ex}} #1}




\hypersetup{
    colorlinks=true,       % false: boxed links; true: colored links
    linkcolor=blue,          % color of internal links (change box color with linkbordercolor)
    citecolor=green,        % color of links to bibliography
    filecolor=magenta,      % color of file links
    urlcolor=cyan           % color of external links
}

\title{Parallel Lines}
\author{Vic Ferdinand, Betsy McNeal, Jenny Sheldon}
\begin{document}
\begin{abstract}\end{abstract}
\maketitle


\begin{problem} \label{ParallelLines1}
Determine which of the following lines is parallel to line $m$.  You may use any method you choose - be ready to explain it to the class!  After you make your selections, jot down some notes about what it means for two lines to be parallel.

\begin{center}
    \begin{tikzpicture}[scale=1.5]
        \draw[domain=0:5] plot (\x, {0.3*\x }) node[below]{$m$};
        \draw[domain=0:5] plot (\x, {0.4*\x+3.6 }) node[below]{$a$};
        \draw[domain=0:5] plot (\x, {0.3*\x+3.4 }) node[below]{$b$};
        \draw[domain=0:5] plot (\x, {0.2*\x+3 }) node[below]{$c$};
        \draw[domain=0:5] plot (\x, {0.33*\x+2 }) node[below]{$d$};
    \end{tikzpicture}
\end{center}


\end{problem}


\newpage
\begin{problem} \label{ParallelLines2}

In each of the three examples below, you are given one pair of lines that appear to be parallel, and a transversal to those lines.  Which of the pairs of lines are actually parallel?  Justify your answer in at least two ways.


\begin{center}
    \begin{tikzpicture}
        \draw[domain=0:5] plot (\x, {0.2*\x});
        \draw[domain=0:5] plot (\x, {0.17*\x+2});
        \draw[domain=0.5:4.5] plot (\x, {\x - 1});
    \end{tikzpicture}
\end{center}

\begin{center}
    \begin{tikzpicture}
        \draw[domain=0:5] plot (\x, {0.2*\x});
        \draw[domain=0:5] plot (\x, {0.23*\x+2});
        \draw[domain=0.5:4.5] plot (\x, {-0.9*\x + 3});
    \end{tikzpicture}
\end{center}

\begin{center}
    \begin{tikzpicture}
        \draw[domain=0:4] plot (\x, {-1.2*\x});
        \draw[domain=1:5] plot (\x, {-1.2*\x+4});
        \draw[domain=-0.5:5] plot (\x, {0.2*\x - 1});
    \end{tikzpicture}
\end{center}





\end{problem}



\newpage

\begin{instructorNotes}
This activity is intended to help students discover what it means for two lines to be parallel.  As such, this is the first activity about parallelism that we do.  Since we would like students to discover the meaning of parallel for themselves, we do not make an actual definition until after the first exercise.  This activity is also intended to introduce the Parallel Postulate, though the instructor will need to direct students to this idea during the discussion.  

\begin{itemize}
    \item The goal of Problem \ref{ParallelLines1} is to give students an opportunity to try to define what it means for two lines to be parallel.  We usually let students experiment, then gather a large collection of their ideas for solving the problem on the board.  Once we have many ideas, we can discuss which are definitions and which are properties, as well as point out which would be easy or difficult to verify.  During this discussion, we usually introduce the Parallel Postulate, as some students are usually trying to use a verison to determine whether the lines are parallel.
    \item Some ideas we typically hear from students or introduce ourselves if they don't come up: 
        \begin{itemize}
            \item Measuring the distance between the ends of the lines.
            \item Drawing another line on the paper and trying to measure distance using this new line.
            \item Measuring from the bottom or top (or side) of the paper.
            \item Making and measuring some angles. 
            \item Folding the paper.
        \end{itemize}
    \item In Problem \ref{ParallelLines1}, line $b$ is parallel to line $m$.  Line $d$ is VERY close to parallel, but is not actually parallel!

\item Problem \ref{ParallelLines2} has several goals: first, students have an opportunity to practice using their definition(s) of ``parallel'' from the earlier problem.  Second, as students work with the Parallel Postulate (or its converse), they have an opportunity to practice using their protractors.  We have found that many students need a reintroduction to this tool.
\item In Problem \ref{ParallelLines2}, the third pair of lines is parallel.  Neither of the other two are, but are very close.
\end{itemize}



\timing{This activity takes us the whole class.  For the first page, we spend about 10 minutes in groups, followed by 10-15 minutes of discussion, then a 10-15 minute discussion about what we've discovered about the meaning of parallel.  The second problem then takes about 10 minutes in groups, followed by 15 minutes of presentations and discussion.}

\end{instructorNotes}


\end{document}

