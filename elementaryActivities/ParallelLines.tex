\documentclass{ximera}
\usepackage{gensymb}
\usepackage{tabularx}
\usepackage{mdframed}
\usepackage{pdfpages}
%\usepackage{chngcntr}

\let\problem\relax
\let\endproblem\relax

\newcommand{\property}[2]{#1#2}




\newtheoremstyle{SlantTheorem}{\topsep}{\fill}%%% space between body and thm
 {\slshape}                      %%% Thm body font
 {}                              %%% Indent amount (empty = no indent)
 {\bfseries\sffamily}            %%% Thm head font
 {}                              %%% Punctuation after thm head
 {3ex}                           %%% Space after thm head
 {\thmname{#1}\thmnumber{ #2}\thmnote{ \bfseries(#3)}} %%% Thm head spec
\theoremstyle{SlantTheorem}
\newtheorem{problem}{Problem}[]

%\counterwithin*{problem}{section}



%%%%%%%%%%%%%%%%%%%%%%%%%%%%Jenny's code%%%%%%%%%%%%%%%%%%%%

%%% Solution environment
%\newenvironment{solution}{
%\ifhandout\setbox0\vbox\bgroup\else
%\begin{trivlist}\item[\hskip \labelsep\small\itshape\bfseries Solution\hspace{2ex}]
%\par\noindent\upshape\small
%\fi}
%{\ifhandout\egroup\else
%\end{trivlist}
%\fi}
%
%
%%% instructorIntro environment
%\ifhandout
%\newenvironment{instructorIntro}[1][false]%
%{%
%\def\givenatend{\boolean{#1}}\ifthenelse{\boolean{#1}}{\begin{trivlist}\item}{\setbox0\vbox\bgroup}{}
%}
%{%
%\ifthenelse{\givenatend}{\end{trivlist}}{\egroup}{}
%}
%\else
%\newenvironment{instructorIntro}[1][false]%
%{%
%  \ifthenelse{\boolean{#1}}{\begin{trivlist}\item[\hskip \labelsep\bfseries Instructor Notes:\hspace{2ex}]}
%{\begin{trivlist}\item[\hskip \labelsep\bfseries Instructor Notes:\hspace{2ex}]}
%{}
%}
%% %% line at the bottom} 
%{\end{trivlist}\par\addvspace{.5ex}\nobreak\noindent\hung} 
%\fi
%
%


\let\instructorNotes\relax
\let\endinstructorNotes\relax
%%% instructorNotes environment
\ifhandout
\newenvironment{instructorNotes}[1][false]%
{%
\def\givenatend{\boolean{#1}}\ifthenelse{\boolean{#1}}{\begin{trivlist}\item}{\setbox0\vbox\bgroup}{}
}
{%
\ifthenelse{\givenatend}{\end{trivlist}}{\egroup}{}
}
\else
\newenvironment{instructorNotes}[1][false]%
{%
  \ifthenelse{\boolean{#1}}{\begin{trivlist}\item[\hskip \labelsep\bfseries {\Large Instructor Notes: \\} \hspace{\textwidth} ]}
{\begin{trivlist}\item[\hskip \labelsep\bfseries {\Large Instructor Notes: \\} \hspace{\textwidth} ]}
{}
}
{\end{trivlist}}
\fi


%% Suggested Timing
\newcommand{\timing}[1]{{\bf Suggested Timing: \hspace{2ex}} #1}




\hypersetup{
    colorlinks=true,       % false: boxed links; true: colored links
    linkcolor=blue,          % color of internal links (change box color with linkbordercolor)
    citecolor=green,        % color of links to bibliography
    filecolor=magenta,      % color of file links
    urlcolor=cyan           % color of external links
}

\title{Parallel Lines}
\author{Vic Ferdinand \& Betsy McNeal \& Jenny Sheldon}
\begin{document}
\begin{abstract}\end{abstract}
\maketitle

\begin{instructorIntro}
This activity is intended to help students discover what it means for two lines to be parallel.  This activity is intended to help students discover the meaning of parallel for themselves, so we do not make an actual definition until after the first exercise.  The second problem gives students a chance to apply the meanings of parallel that we have just discussed, and gives us an opportunity to use the converse of the parallel postulate.

\timing{This activity should take the whole class.  For the first page, about 10 minutes in groups, followed by 10-15 minutes of discussion, then a 10-15 minute discussion about what we've discovered about the meaning of parallel.  The second problem should take about 10 minutes in groups, followed by 15 minutes of presentations and discussion.}

\end{instructorIntro}


\begin{problem}
Determine which of the following lines is parallel to line $m$.  You may use any method you choose - be ready to explain it to the class!  After you make your selections, jot down some notes about what it means for two lines to be parallel.

\vskip 1.5in
\[
\includegraphics[width=4.5in]{graphics/AW-parallel-1.png}
\]

\begin{solution}
    Line $b$ is parallel to line $m$.  Line $d$ is VERY close to parallel, but is not actually parallel!
\end{solution}

\begin{instructorNotes}
\begin{enumerate}
\item Students should be encouraged to justify their answers by whatever means they can.  Since we haven't defined ``parallel'' yet, this can be a little imprecise.  One way to discuss the meaning of parallel is to make a list of the different ways that students justify which lines are parallel.
\item This activity is based on an activity in Aichele-Wolfe, in which the students are encouraged to use folding in order to determine which lines are parallel.  This idea may or may not come up, but it's good to mention if it doesn't.
\item A common method used by students is to measure the distance between the lines on both ends.  This can present a nice discussion - it is possible to determine which two lines are parallel using this method, but you have to measure very carefully!
\item Some students may suggest measuring some angles - remembering the parallel postulate from a previous course.  If this idea doesn't come up, you'll want to point it out during the discussion of parallel lines, and call it the Parallel Postulate.  Make sure to double-check the book's statement!
\end{enumerate}
\end{instructorNotes}
\end{problem}


\newpage
\begin{problem}

In each of the three examples below, you are given one pair of lines that appear to be parallel, and a transversal to those lines.  Which of the pairs of lines are actually parallel?  Justify your answer in at least two ways.

\[
\includegraphics[height=6in]{graphics/AW-parallel-2.png}
\]

\begin{solution}
    The third pair of lines is parallel.  Neither of the other two are.
\end{solution}

\begin{instructorNotes}
\begin{enumerate}
\item If this is the first time you've used protractors in class, you might have to remind many students in their groups how to use the protractor.  This is also a good thing to bring up in full-class discussion, to make sure that everyone remembers!
\item Students should use at least two ways to determine whether the lines are parallel - but should be able to discuss even more than two!  In discussion, it's nice to see if you can find an argument corresponding to each argument discussed in the first problem.  
\item There is exactly one set of parallel lines on the page!
\item Again, the issue of measurement is key.  The lines that are not parallel appear to actually be parallel, but a careful measurement will show they are not.
\item You might want to highlight how we are using the converse of the Parallel Postulate in this case - but we will still assume that this result holds.
\end{enumerate}
\end{instructorNotes}
\end{problem}


\end{document}

