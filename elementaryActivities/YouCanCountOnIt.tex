\documentclass[nooutcomes]{ximera}
\usepackage{gensymb}
\usepackage{tabularx}
\usepackage{mdframed}
\usepackage{pdfpages}
%\usepackage{chngcntr}

\let\problem\relax
\let\endproblem\relax

\newcommand{\property}[2]{#1#2}




\newtheoremstyle{SlantTheorem}{\topsep}{\fill}%%% space between body and thm
 {\slshape}                      %%% Thm body font
 {}                              %%% Indent amount (empty = no indent)
 {\bfseries\sffamily}            %%% Thm head font
 {}                              %%% Punctuation after thm head
 {3ex}                           %%% Space after thm head
 {\thmname{#1}\thmnumber{ #2}\thmnote{ \bfseries(#3)}} %%% Thm head spec
\theoremstyle{SlantTheorem}
\newtheorem{problem}{Problem}[]

%\counterwithin*{problem}{section}



%%%%%%%%%%%%%%%%%%%%%%%%%%%%Jenny's code%%%%%%%%%%%%%%%%%%%%

%%% Solution environment
%\newenvironment{solution}{
%\ifhandout\setbox0\vbox\bgroup\else
%\begin{trivlist}\item[\hskip \labelsep\small\itshape\bfseries Solution\hspace{2ex}]
%\par\noindent\upshape\small
%\fi}
%{\ifhandout\egroup\else
%\end{trivlist}
%\fi}
%
%
%%% instructorIntro environment
%\ifhandout
%\newenvironment{instructorIntro}[1][false]%
%{%
%\def\givenatend{\boolean{#1}}\ifthenelse{\boolean{#1}}{\begin{trivlist}\item}{\setbox0\vbox\bgroup}{}
%}
%{%
%\ifthenelse{\givenatend}{\end{trivlist}}{\egroup}{}
%}
%\else
%\newenvironment{instructorIntro}[1][false]%
%{%
%  \ifthenelse{\boolean{#1}}{\begin{trivlist}\item[\hskip \labelsep\bfseries Instructor Notes:\hspace{2ex}]}
%{\begin{trivlist}\item[\hskip \labelsep\bfseries Instructor Notes:\hspace{2ex}]}
%{}
%}
%% %% line at the bottom} 
%{\end{trivlist}\par\addvspace{.5ex}\nobreak\noindent\hung} 
%\fi
%
%


\let\instructorNotes\relax
\let\endinstructorNotes\relax
%%% instructorNotes environment
\ifhandout
\newenvironment{instructorNotes}[1][false]%
{%
\def\givenatend{\boolean{#1}}\ifthenelse{\boolean{#1}}{\begin{trivlist}\item}{\setbox0\vbox\bgroup}{}
}
{%
\ifthenelse{\givenatend}{\end{trivlist}}{\egroup}{}
}
\else
\newenvironment{instructorNotes}[1][false]%
{%
  \ifthenelse{\boolean{#1}}{\begin{trivlist}\item[\hskip \labelsep\bfseries {\Large Instructor Notes: \\} \hspace{\textwidth} ]}
{\begin{trivlist}\item[\hskip \labelsep\bfseries {\Large Instructor Notes: \\} \hspace{\textwidth} ]}
{}
}
{\end{trivlist}}
\fi


%% Suggested Timing
\newcommand{\timing}[1]{{\bf Suggested Timing: \hspace{2ex}} #1}




\hypersetup{
    colorlinks=true,       % false: boxed links; true: colored links
    linkcolor=blue,          % color of internal links (change box color with linkbordercolor)
    citecolor=green,        % color of links to bibliography
    filecolor=magenta,      % color of file links
    urlcolor=cyan           % color of external links
}
\title{You Can Count On It}
\author{Vic Ferdinand, Betsy McNeal, Robin Pemantle, Jenny Sheldon}

\begin{document}
\begin{abstract}
\end{abstract}
\maketitle



\begin{problem}
Nikki has three different stores to visit this afternoon: Target, WalMart, and Meijer.  How many ways are there for Nikki to complete her shopping trip?  Explain.
\end{problem}

%\begin{problem}
%Calvin is trying to complete the ``Columbus Coffee Experience", which includes buying coffee at five different locations.  

%\begin{enumerate}
%\item Assuming Calvin chooses five locations in advance, how many ways are there for him to reach his goal?
%\item Assuming Calvin does not choose the locations in advance, and there are 10 participating coffee shops, how many ways are there for him to reach his goal?
%\end{enumerate}
%\end{problem}



\begin{problem}
A certain passcode must be made in the following way: choose one of O, S ,or U, then choose a whole number between 6 and 9 (inclusive).  How many passcodes can be made?  Answer this question in the following ways:
\begin{enumerate}
\item By writing down all of the possibilities.
\item By organizing the possibilities in a table.
\item By drawing a tree diagram containing all of the possibilities.
\end{enumerate}
\end{problem}

\begin{problem}
You are a contestant on ``The Amazing Race''.  You must travel from your home to New York City, then from New York City to London, and finally from London to Paris.  You will travel from your home to New York City by car, train, bus, or plane.  You will travel from New York City to London by ship or by plane.  Finally, you will travel from London to Paris by ship, plane, or helicopter.
\begin{enumerate}
\item How many trips are possible from home to New York City to London?
\item How many trips are possible from home to New York City to London to Paris?
\end{enumerate}
\end{problem}

\begin{problem}
From 2001 - 2004, an Ohio license plate consisted of two letters followed by two digits followed by two letters.
\begin{enumerate}
\item How many different Ohio license plates could be made?
\item How many different Ohio license plates could be made if there are no repeats of numbers or letters allowed?
\end{enumerate}
\end{problem}


\newpage
\begin{instructorNotes}
This activity is intended to introduce our counting activities.  We use these activities about counting and probability to help cement the meaning of various operations for students.  Throughout these activities, we expect students to justify their operations of choice in their explanations.  There are a few main types of counting strategies we see from our students: arrays, ordered lists, tree diagrams, and algebraic expressions.  Students tend to want to use more complicated strategies before trying simpler strategies, so we are often reminding them to write down some examples, or to make sure their list is well organized. When discussing this activity as a whole class, we like to have students demonstrate multiple solution methods for each problem, even when not specifically requested by the problem.  

\timing{We use one class period for this activity.}
\end{instructorNotes}


\end{document}