\documentclass[nooutcomes]{ximera}
\usepackage{gensymb}
\usepackage{tabularx}
\usepackage{mdframed}
\usepackage{pdfpages}
%\usepackage{chngcntr}

\let\problem\relax
\let\endproblem\relax

\newcommand{\property}[2]{#1#2}




\newtheoremstyle{SlantTheorem}{\topsep}{\fill}%%% space between body and thm
 {\slshape}                      %%% Thm body font
 {}                              %%% Indent amount (empty = no indent)
 {\bfseries\sffamily}            %%% Thm head font
 {}                              %%% Punctuation after thm head
 {3ex}                           %%% Space after thm head
 {\thmname{#1}\thmnumber{ #2}\thmnote{ \bfseries(#3)}} %%% Thm head spec
\theoremstyle{SlantTheorem}
\newtheorem{problem}{Problem}[]

%\counterwithin*{problem}{section}



%%%%%%%%%%%%%%%%%%%%%%%%%%%%Jenny's code%%%%%%%%%%%%%%%%%%%%

%%% Solution environment
%\newenvironment{solution}{
%\ifhandout\setbox0\vbox\bgroup\else
%\begin{trivlist}\item[\hskip \labelsep\small\itshape\bfseries Solution\hspace{2ex}]
%\par\noindent\upshape\small
%\fi}
%{\ifhandout\egroup\else
%\end{trivlist}
%\fi}
%
%
%%% instructorIntro environment
%\ifhandout
%\newenvironment{instructorIntro}[1][false]%
%{%
%\def\givenatend{\boolean{#1}}\ifthenelse{\boolean{#1}}{\begin{trivlist}\item}{\setbox0\vbox\bgroup}{}
%}
%{%
%\ifthenelse{\givenatend}{\end{trivlist}}{\egroup}{}
%}
%\else
%\newenvironment{instructorIntro}[1][false]%
%{%
%  \ifthenelse{\boolean{#1}}{\begin{trivlist}\item[\hskip \labelsep\bfseries Instructor Notes:\hspace{2ex}]}
%{\begin{trivlist}\item[\hskip \labelsep\bfseries Instructor Notes:\hspace{2ex}]}
%{}
%}
%% %% line at the bottom} 
%{\end{trivlist}\par\addvspace{.5ex}\nobreak\noindent\hung} 
%\fi
%
%


\let\instructorNotes\relax
\let\endinstructorNotes\relax
%%% instructorNotes environment
\ifhandout
\newenvironment{instructorNotes}[1][false]%
{%
\def\givenatend{\boolean{#1}}\ifthenelse{\boolean{#1}}{\begin{trivlist}\item}{\setbox0\vbox\bgroup}{}
}
{%
\ifthenelse{\givenatend}{\end{trivlist}}{\egroup}{}
}
\else
\newenvironment{instructorNotes}[1][false]%
{%
  \ifthenelse{\boolean{#1}}{\begin{trivlist}\item[\hskip \labelsep\bfseries {\Large Instructor Notes: \\} \hspace{\textwidth} ]}
{\begin{trivlist}\item[\hskip \labelsep\bfseries {\Large Instructor Notes: \\} \hspace{\textwidth} ]}
{}
}
{\end{trivlist}}
\fi


%% Suggested Timing
\newcommand{\timing}[1]{{\bf Suggested Timing: \hspace{2ex}} #1}




\hypersetup{
    colorlinks=true,       % false: boxed links; true: colored links
    linkcolor=blue,          % color of internal links (change box color with linkbordercolor)
    citecolor=green,        % color of links to bibliography
    filecolor=magenta,      % color of file links
    urlcolor=cyan           % color of external links
}

\title{Back And Forth}
\author{Vic Ferdinand, Betsy McNeal, Jenny Sheldon}

\begin{document}
\begin{abstract}\end{abstract}
\maketitle




\begin{problem} \label{BackAndForth1}
    Jack and Jill are measuring the lengths of their desks.  The desks are identical, but Jack is using a piece of pencil lead to measure the length and Jill is using a necklace that she stretched out into a straight line.  Jack reports that his desk measures 20 units while Jill reports that her desk measures 5 units.
    \begin{enumerate}
        \item Jack claims that his desk must be bigger than Jill's, since his desk is 20 units long.  Is Jack correct?  Why or why not?
        \item Jill claims that her desk must be bigger than Jack's, since she used a bigger unit to measure the desk.  Is Jill correct?  Why or why not?
        \item What is the relationship between the two units used to measure the desks?  Justify your answer.
        \item Next, Jill measures the length of the chalkboard and finds it to be 13.5 units long.  When Jack measures the chalkboard, what will his answer be?  Justify your answer.  (Hint: some Math 1125 terminology should make its way into your explanation!)
    \end{enumerate}
    

\end{problem}

\begin{problem} \label{BackAndForth2}
    Jason knows that there are 3 feet in one yard.  He also knows that a football field is 100 yards long.  Since a yard is bigger than a foot, Jason thinks that a football field should be $100 \div 3 = 33 \frac{1}{3}$ feet long.
    \begin{enumerate}
    \item Explain Jason's reasoning.  What does he think we should do, and why did he come to that conclusion?
    \item What is the correct answer to this problem? Explain.
    \item How would you help Jason to understand his mistake?  Give at least two methods you could try.
    \end{enumerate}
   
    

\end{problem}


\begin{problem} \label{BackAndForth3}
These problems remind Mike of fraction comparison.  What might he be thinking?


\end{problem}

\newpage
\begin{instructorNotes}
This activity is an introduction to the concept of measurement conversion.  We will study the topic of measurement conversion throughout the semester, as we consider area and volume in more depth.  Most of our examples at this stage concern length.  This activity is our final activity about measurement before we move on to discussing area in more depth.  This activity follows ``Try This On For Size'', where we use standard units of measure, and is followed by ``Building Blocks'', where we discuss area.

We treat the concept of measurement with a strong emphasis on using non-standard units to measure.  This is an example of our theme of ``making the familiar strange'', or taking students out of their comfort zone so that they cannot just apply old knowledge or mathematical rules, but instead have to think critically about the matter at hand.  We have found students much more willing to focus on the cover-and-count meaning of measurement rather than just applying a formula when the students are working with units that they have never before used.  These non-standard units also help us to draw out general principles about measurement, again rather than simply formulae.

In particular, this activity is designed to bring up some misconceptions about measurement conversion, as well as give students a chance to do conversions without their preconceived notions.  We do not allow ``dimensional analysis'', or making a fence-post chart.  Instead, we expect students to use reasoning, and to be prepared to explain their answers in terms of the meaning of any operation they use.

\begin{itemize}
    \item In Problem \ref{BackAndForth1}, students confront two basic misconceptions about measurement conversion, and then convert using non-standard units of length.  The misconceptions are common amongst children - even though the desks are identical, getting different numbers for the measurement can be confusing to children.  The question about the relationship between the units is intentionally vague.  Students can relate this process to several topics we have covered previously in either semester, including multiplication and ratios.
    \item In Problem \ref{BackAndForth2}, we see a variety of answers for students' ideas about Jason's reasoning.  One misconception we make sure to discuss is that of ``multiplication makes things larger, and division makes things smaller''. 
    \item Problem \ref{BackAndForth3} is a challenge/bonus question!  Mike was reminded of the idea that more pieces in the denominator of a fraction results in you needing more of those pieces to make the same amount.
\end{itemize}



\timing{We spend about a half of a class period on this activity.  We give the students a few minutes to think about the first problem, discuss, and then move on to the second problem.}

\end{instructorNotes}


\end{document}