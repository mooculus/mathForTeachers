\documentclass[nooutcomes]{ximera}
\usepackage{gensymb}
\usepackage{tabularx}
\usepackage{mdframed}
\usepackage{pdfpages}
%\usepackage{chngcntr}

\let\problem\relax
\let\endproblem\relax

\newcommand{\property}[2]{#1#2}




\newtheoremstyle{SlantTheorem}{\topsep}{\fill}%%% space between body and thm
 {\slshape}                      %%% Thm body font
 {}                              %%% Indent amount (empty = no indent)
 {\bfseries\sffamily}            %%% Thm head font
 {}                              %%% Punctuation after thm head
 {3ex}                           %%% Space after thm head
 {\thmname{#1}\thmnumber{ #2}\thmnote{ \bfseries(#3)}} %%% Thm head spec
\theoremstyle{SlantTheorem}
\newtheorem{problem}{Problem}[]

%\counterwithin*{problem}{section}



%%%%%%%%%%%%%%%%%%%%%%%%%%%%Jenny's code%%%%%%%%%%%%%%%%%%%%

%%% Solution environment
%\newenvironment{solution}{
%\ifhandout\setbox0\vbox\bgroup\else
%\begin{trivlist}\item[\hskip \labelsep\small\itshape\bfseries Solution\hspace{2ex}]
%\par\noindent\upshape\small
%\fi}
%{\ifhandout\egroup\else
%\end{trivlist}
%\fi}
%
%
%%% instructorIntro environment
%\ifhandout
%\newenvironment{instructorIntro}[1][false]%
%{%
%\def\givenatend{\boolean{#1}}\ifthenelse{\boolean{#1}}{\begin{trivlist}\item}{\setbox0\vbox\bgroup}{}
%}
%{%
%\ifthenelse{\givenatend}{\end{trivlist}}{\egroup}{}
%}
%\else
%\newenvironment{instructorIntro}[1][false]%
%{%
%  \ifthenelse{\boolean{#1}}{\begin{trivlist}\item[\hskip \labelsep\bfseries Instructor Notes:\hspace{2ex}]}
%{\begin{trivlist}\item[\hskip \labelsep\bfseries Instructor Notes:\hspace{2ex}]}
%{}
%}
%% %% line at the bottom} 
%{\end{trivlist}\par\addvspace{.5ex}\nobreak\noindent\hung} 
%\fi
%
%


\let\instructorNotes\relax
\let\endinstructorNotes\relax
%%% instructorNotes environment
\ifhandout
\newenvironment{instructorNotes}[1][false]%
{%
\def\givenatend{\boolean{#1}}\ifthenelse{\boolean{#1}}{\begin{trivlist}\item}{\setbox0\vbox\bgroup}{}
}
{%
\ifthenelse{\givenatend}{\end{trivlist}}{\egroup}{}
}
\else
\newenvironment{instructorNotes}[1][false]%
{%
  \ifthenelse{\boolean{#1}}{\begin{trivlist}\item[\hskip \labelsep\bfseries {\Large Instructor Notes: \\} \hspace{\textwidth} ]}
{\begin{trivlist}\item[\hskip \labelsep\bfseries {\Large Instructor Notes: \\} \hspace{\textwidth} ]}
{}
}
{\end{trivlist}}
\fi


%% Suggested Timing
\newcommand{\timing}[1]{{\bf Suggested Timing: \hspace{2ex}} #1}




\hypersetup{
    colorlinks=true,       % false: boxed links; true: colored links
    linkcolor=blue,          % color of internal links (change box color with linkbordercolor)
    citecolor=green,        % color of links to bibliography
    filecolor=magenta,      % color of file links
    urlcolor=cyan           % color of external links
}

\title{Reflections, Rotations, Translations}
\author{Vic Ferdinand, Betsy McNeal, Robin Pemantle, Jenny Sheldon} 

\begin{document}
\begin{abstract}This activity begins our study of reflections, rotations, and translations.  We will accomplish these transformations using careful drawing, compass, ruler, protractor, and reasoning.
\end{abstract}
\maketitle



\begin{problem}
Point P is the reflection of point Q about some line.  Using your compass and ruler, draw this line of reflection.  Explain how you knew where and how to draw the line and why you are sure that reflecting Q across this line will put Q on P.\\
\vskip 2in

\vfill
\begin{center}
    \begin{tikzpicture}
        \draw[color=black,fill=black] (1, 4) circle (3pt) node[below left] {$Q$};  
        \draw[color=black, fill=black] (8,0) circle (3pt) node[below left] {$P$};
    \end{tikzpicture}
\end{center}


\vfill


\end{problem}

\newpage
\begin{problem}
Use your ruler to draw three different triangles according to the cases below, and then reflect each triangle across the given line. Explain how you drew the reflection of your triangles and why you are sure that they are correctly positioned relative to the line of reflection.\\
\begin{enumerate}
    \item The triangle lies entirely on one side of the line.
    \item One side of the triangle lies on the line.
    \item The triangle lies across the line.
\end{enumerate}
\vskip 1in

\begin{center}
    \begin{tikzpicture}
        \draw[thick, <->] (-3, -2) -- (4, 6);
    \end{tikzpicture}
\end{center}


\vfill


\end{problem}
\newpage
\begin{problem}
Using compass, ruler, and protractor rotate the point P 110 degrees clockwise about point Q.  Explain how you drew the new position of point P and why you are sure that this is the correct location of the rotated point P.\\
\vfill

\begin{center}
    \begin{tikzpicture}
        \draw[color=black,fill=black] (0, 0) circle (3pt) node[below left] {$Q$};  
        \draw[color=black, fill=black] (-4,-5) circle (3pt) node[below left] {$P$};
        \draw[color=white, fill=white] (4,5) circle (3pt);
    \end{tikzpicture}
\end{center}

\vfill
\end{problem}
\newpage

\begin{problem}
Using your compass, ruler, and protractor, rotate the given triangle 50 degrees clockwise about point Q.  Explain how you drew the new position of the triangle and why you are sure that this is the correct location of the rotated triangle.\\


\begin{center}
    \begin{tikzpicture}
        \draw[color=black,fill=black] (0, 0) circle (3pt) node[below left] {$Q$};  
        \draw[thick] (-6, -3)--(-2.5, -2.5) -- (-5, -4)--(-6,-3);
        \draw[color=white, fill=white] (4,5) circle (3pt);
    \end{tikzpicture}
\end{center}
\vfill
\end{problem}
\newpage

\begin{problem}
Use your ruler to translate the given point P the distance and direction given by the vector shown.  Explain how you decided where to put the new point.\\
\vskip 1.5in


\begin{center}
    \begin{tikzpicture}
        \draw[thick, ->] (-1, 1) -- (-4, -4);
        \draw[color=black, fill=black] (4, -1) circle (3pt) node[below left] {$P$};
        \draw[color=white, fill=white] (6, 0) circle (3pt);
    \end{tikzpicture}
\end{center}
\vfill
\end{problem}
\newpage

\begin{problem}
Use your ruler to translate the given triangle the distance and direction given by the vector shown.  Explain how you decided where to put the new triangle.\\
\vskip 1.5in
\begin{center}
    \begin{tikzpicture}
        \draw[thick, ->] (-2,4) -- (-5, 2);
        \draw (0,0)--(-1,-2)--(3,-2)--(0,0);
        \node[above left] at (0,0) {$C$};
        \node[below left] at (-1, -2) {$A$};
        \node[below right] at (3,-2) {$B$};
        \draw[color=white, fill=white] (5,0) circle (3pt);
    \end{tikzpicture}
\end{center}
\vfill
\end{problem}
\newpage

\begin{problem}
\begin{enumerate}
    \item Summarize the characteristics of a reflection.  What information must be specified in order to fully describe this kind of motion?  What relationship does the figure have to the line of reflection?
    \item Summarize the characteristics of a rotation.  What information must be specified in order to fully describe this kind of motion?  What relationship does the figure have to the point of rotation?
    \item Summarize the characteristics of a translation.  What information must be specified in order to fully describe this kind of motion?  What relationship does the figure have to its original location in the plane? To the vector describing the distance and direction of the slide?
\end{enumerate}


\end{problem}

\newpage
\begin{instructorNotes}
The purpose of this activity is to introduce the ideas of rigid motions (reflection, translation, rotation) without a coordinate system and have students work with these ideas using compass and protractor to develop their ability to visualize these transformations.  This is the first activity we do about rigid motions, so we have not previously used a coordinate system for these problems.  We prefer this approach because we have found that many students have previously learned rules for transforming coordinates for each of the rigid motions, but usually don't understand why these rules give the correct answers.  Our intent is to instead focus students on the meaning of each type of transformation.  Our students have found these transformation activities quite interesting and challenging at the right level.

Since we have not previously studied transformations, we begin with a discussion of what kind of moves in the plane (and then in space) would maintain the shape and size of an object. We then show pictorial definitions of each transformation, then have students work through the problems, and finally use the last problem to put together full definitions for each of the transformations based on the students' work in the problems.  One of the main things students need to do for each problem is work out how to use their tools (compass, ruler, protractor) to accurately produce the desired result.  We don't accept just a guess for the answer, but instead require students to justify why their answer is accurate.  
\begin{itemize}
    \item For reflections, we have found the students are frequently confused that they must make corresponding points the same distance from the line in a perpendicular manner.  Sometimes they do not realize that the distance from a point to a line changes according to the direction.  
    \item Sometimes students have trouble reflecting or rotating a whole object.  When we encounter this difficulty, especially during small group work, we encourage students to identify and move key points (e.g., vertices), connecting them up later.  In whole group discussion, we ask why this method works (reminding students that we are actually moving more than just three individual points).  Usually, someone in each group will recognize this and help those who are confused without intervention by the instructor.  It's nice to then ask the students to justify why this method works!
    \item Students often find rotations difficult to visualize - particularly once we work in the coordinate plane.  This can be a benefit of working with coordinate-free transformations first.
\end{itemize}




\timing{Our introduction takes about 10 minutes.  Afterwards, students will work on these problems for as long as they are given.  Our current calendar has about a day and a half for this activity.  We give the students about 10 minutes in small groups to work on the first two problems, and then spend about 20 minutes in discussion.  We repeat this timing with the next two problems, and then finally the last three problems with a longer discussion at the end to wrap everything up.  If we need to shorten the discussion time, we choose one problem from each grouping in addition to the summary, and leave the other problems for review.}

\end{instructorNotes}



\end{document}