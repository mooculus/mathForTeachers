\documentclass[handout]{ximera}
\usepackage{gensymb}
\usepackage{tabularx}
\usepackage{mdframed}
\usepackage{pdfpages}
%\usepackage{chngcntr}

\let\problem\relax
\let\endproblem\relax

\newcommand{\property}[2]{#1#2}




\newtheoremstyle{SlantTheorem}{\topsep}{\fill}%%% space between body and thm
 {\slshape}                      %%% Thm body font
 {}                              %%% Indent amount (empty = no indent)
 {\bfseries\sffamily}            %%% Thm head font
 {}                              %%% Punctuation after thm head
 {3ex}                           %%% Space after thm head
 {\thmname{#1}\thmnumber{ #2}\thmnote{ \bfseries(#3)}} %%% Thm head spec
\theoremstyle{SlantTheorem}
\newtheorem{problem}{Problem}[]

%\counterwithin*{problem}{section}



%%%%%%%%%%%%%%%%%%%%%%%%%%%%Jenny's code%%%%%%%%%%%%%%%%%%%%

%%% Solution environment
%\newenvironment{solution}{
%\ifhandout\setbox0\vbox\bgroup\else
%\begin{trivlist}\item[\hskip \labelsep\small\itshape\bfseries Solution\hspace{2ex}]
%\par\noindent\upshape\small
%\fi}
%{\ifhandout\egroup\else
%\end{trivlist}
%\fi}
%
%
%%% instructorIntro environment
%\ifhandout
%\newenvironment{instructorIntro}[1][false]%
%{%
%\def\givenatend{\boolean{#1}}\ifthenelse{\boolean{#1}}{\begin{trivlist}\item}{\setbox0\vbox\bgroup}{}
%}
%{%
%\ifthenelse{\givenatend}{\end{trivlist}}{\egroup}{}
%}
%\else
%\newenvironment{instructorIntro}[1][false]%
%{%
%  \ifthenelse{\boolean{#1}}{\begin{trivlist}\item[\hskip \labelsep\bfseries Instructor Notes:\hspace{2ex}]}
%{\begin{trivlist}\item[\hskip \labelsep\bfseries Instructor Notes:\hspace{2ex}]}
%{}
%}
%% %% line at the bottom} 
%{\end{trivlist}\par\addvspace{.5ex}\nobreak\noindent\hung} 
%\fi
%
%


\let\instructorNotes\relax
\let\endinstructorNotes\relax
%%% instructorNotes environment
\ifhandout
\newenvironment{instructorNotes}[1][false]%
{%
\def\givenatend{\boolean{#1}}\ifthenelse{\boolean{#1}}{\begin{trivlist}\item}{\setbox0\vbox\bgroup}{}
}
{%
\ifthenelse{\givenatend}{\end{trivlist}}{\egroup}{}
}
\else
\newenvironment{instructorNotes}[1][false]%
{%
  \ifthenelse{\boolean{#1}}{\begin{trivlist}\item[\hskip \labelsep\bfseries {\Large Instructor Notes: \\} \hspace{\textwidth} ]}
{\begin{trivlist}\item[\hskip \labelsep\bfseries {\Large Instructor Notes: \\} \hspace{\textwidth} ]}
{}
}
{\end{trivlist}}
\fi


%% Suggested Timing
\newcommand{\timing}[1]{{\bf Suggested Timing: \hspace{2ex}} #1}




\hypersetup{
    colorlinks=true,       % false: boxed links; true: colored links
    linkcolor=blue,          % color of internal links (change box color with linkbordercolor)
    citecolor=green,        % color of links to bibliography
    filecolor=magenta,      % color of file links
    urlcolor=cyan           % color of external links
}

\title{Reflections, Rotations, Translations}
\author{Vic Ferdinand, Betsy McNeal, Robin Pemantle, Jenny Sheldon} 

\begin{document}
\begin{abstract}This activity begins our study of reflections, rotations, and translations.  We will accomplish these transformations using careful drawing, compass, ruler, protractor, and reasoning.
\end{abstract}
\maketitle

\begin{instructorIntro}
This activity generally requires two class periods.  The purpose of the activity is to  introduce the ideas of rigid motions (reflection, translation, rotation) and have students work with these ideas using compass and protractor to develop their ability to visualize these transformations.  You can follow this activity by doing the same transformations on grid paper if you would like to also develop the relevant formulae.  Our intent here is to focus students on the meaning of each type of transformation without the distraction that grid paper often provides, since many students have learned to do transformations merely by transforming coordinates according to a rule.

In the past, students have found these transformation activities quite interesting and challenging at the right level.

You should begin with a discussion of what kind of moves in the plane (and then in space) would maintain the shape and size of an object. The students will need to at least be shown pictorial definitions of each transformation before they begin this exercise.  The last problem asks students to summarize the characteristics of reflections, rotations, and translations, so it's not necessary to give a full verbal statement of the definitions at this point.  

\timing{Your introduction may take about 10 minutes.  Afterwards, students will work on these problems for as long as they are given.  The current calendar has about a day and a half for this activity.  You could give the students about 10 minutes in small groups to work on the first two problems, and then about 20 minutes in discussion.  Repeat this timing with the next two problems, and then finally the last three problems with a longer discussion at the end to wrap everything up.  If you need to shorten the discussion time, you can choose one problem from each grouping in addition to the summary, and leave the other problems for review.}

\end{instructorIntro}

\begin{problem}
Point P is the reflection of point Q about some line.  Using your compass and ruler, draw this line of reflection.  Explain how you knew where and how to draw the line and why you are sure that reflecting Q across this line will put Q on P.\\
\vskip 2in
\[
\includegraphics[width=3in]{graphics/transformation1.pdf}
\]
\vfill
\begin{instructorNotes}
For reflections, the students will likely be confused that they must make corresponding points the same distance from the line in a perpendicular manner.  Sometimes they do not realize that the distance from a point to a line changes according to the direction.  

As the students work through this activity, be sure that they are actually constructing their solutions, not just guessing.  They should need compass, protractor, and ruler!

Check that students can justify that their technique will always work to get the desired result.   Student explanations should include that, under a reflection, all points maintain their perpendicular distance from the line of reflection.
\end{instructorNotes}

\end{problem}

\newpage
\begin{problem}
Use your ruler to draw three different triangles according to the cases below, and then reflect each triangle across the given line. Explain how you drew the reflection of your triangles and why you are sure that they are correctly positioned relative to the line of reflection.\\
\begin{enumerate}
    \item The triangle lies entirely on one side of the line.
    \item One side of the triangle lies on the line.
    \item The triangle lies across the line.
\end{enumerate}
\vskip 1in
\[
\includegraphics[scale=0.7]{graphics/transformation2.pdf}
\]
\vfill
\begin{instructorNotes}
Sometimes students have trouble reflecting or rotating a whole object.  You might encourage them to identify and move key points (e.g., vertices), connecting them up later.  Usually, someone in each group will recognize this and help those who are confused without intervention by the instructor.  It's nice to then ask the students to justify why this method works!
\end{instructorNotes}

\end{problem}
\newpage
\begin{problem}
Using compass, ruler, and protractor rotate the point P 110 degrees clockwise about point Q.  Explain how you drew the new position of point P and why you are sure that this is the correct location of the rotated point P.\\
\vskip 2in
\[
\includegraphics[width=5in]{graphics/transformation3.pdf}
\]

\begin{instructorNotes}
Students often find rotations difficult to visualize - particularly once we work in the coordinate plane.

Check that students can justify that their technique will always work to get the desired result.   Student explanations should include that a rotation follows a circular path - all points will stay the same distance from the given center.
\end{instructorNotes}
\vfill
\end{problem}
\newpage

\begin{problem}
Using your compass, ruler, and protractor, rotate the given triangle 50 degrees clockwise about point Q.  Explain how you drew the new position of the triangle and why you are sure that this is the correct location of the rotated triangle.\\


\[
\includegraphics[width=5in]{graphics/transformation4.pdf}
\]
\vfill
\end{problem}
\newpage

\begin{problem}
Use your ruler to translate the given point P the distance and direction given by the vector shown.  Explain how you decided where to put the new point.\\
\vskip 1.5in

\[
\includegraphics[width=5in]{graphics/transformation5.pdf}
\]
\vfill
\end{problem}
\newpage

\begin{problem}
Use your ruler to translate the given triangle the distance and direction given by the vector shown.  Explain how you decided where to put the new triangle.\\
\vskip 1.5in
\[
\includegraphics[width=5in]{graphics/transformation6.pdf}
\]
\vfill
\end{problem}
\newpage

\begin{problem}
\begin{enumerate}
    \item Summarize the characteristics of a reflection.  What information must be specified in order to fully describe this kind of motion?  What relationship does the figure have to the line of reflection?
    \item Summarize the characteristics of a rotation.  What information must be specified in order to fully describe this kind of motion?  What relationship does the figure have to the point of rotation?
    \item Summarize the characteristics of a translation.  What information must be specified in order to fully describe this kind of motion?  What relationship does the figure have to its original location in the plane? To the vector describing the distance and direction of the slide?
\end{enumerate}

\begin{instructorNotes}
{\bf Possible extensions:} If time (now or in a later activity), might expand these motions to what they would be like in 3-D (e.g., a plane of reflection, what rotating by an angle would be defined as, etc.)

You also might give the result of a motion and ask for the line of reflection or center of rotation, etc.
\end{instructorNotes}
\end{problem}





\end{document}