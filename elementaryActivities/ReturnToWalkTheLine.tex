\documentclass{ximera}
\usepackage{gensymb}
\usepackage{tabularx}
\usepackage{mdframed}
\usepackage{pdfpages}
%\usepackage{chngcntr}

\let\problem\relax
\let\endproblem\relax

\newcommand{\property}[2]{#1#2}




\newtheoremstyle{SlantTheorem}{\topsep}{\fill}%%% space between body and thm
 {\slshape}                      %%% Thm body font
 {}                              %%% Indent amount (empty = no indent)
 {\bfseries\sffamily}            %%% Thm head font
 {}                              %%% Punctuation after thm head
 {3ex}                           %%% Space after thm head
 {\thmname{#1}\thmnumber{ #2}\thmnote{ \bfseries(#3)}} %%% Thm head spec
\theoremstyle{SlantTheorem}
\newtheorem{problem}{Problem}[]

%\counterwithin*{problem}{section}



%%%%%%%%%%%%%%%%%%%%%%%%%%%%Jenny's code%%%%%%%%%%%%%%%%%%%%

%%% Solution environment
%\newenvironment{solution}{
%\ifhandout\setbox0\vbox\bgroup\else
%\begin{trivlist}\item[\hskip \labelsep\small\itshape\bfseries Solution\hspace{2ex}]
%\par\noindent\upshape\small
%\fi}
%{\ifhandout\egroup\else
%\end{trivlist}
%\fi}
%
%
%%% instructorIntro environment
%\ifhandout
%\newenvironment{instructorIntro}[1][false]%
%{%
%\def\givenatend{\boolean{#1}}\ifthenelse{\boolean{#1}}{\begin{trivlist}\item}{\setbox0\vbox\bgroup}{}
%}
%{%
%\ifthenelse{\givenatend}{\end{trivlist}}{\egroup}{}
%}
%\else
%\newenvironment{instructorIntro}[1][false]%
%{%
%  \ifthenelse{\boolean{#1}}{\begin{trivlist}\item[\hskip \labelsep\bfseries Instructor Notes:\hspace{2ex}]}
%{\begin{trivlist}\item[\hskip \labelsep\bfseries Instructor Notes:\hspace{2ex}]}
%{}
%}
%% %% line at the bottom} 
%{\end{trivlist}\par\addvspace{.5ex}\nobreak\noindent\hung} 
%\fi
%
%


\let\instructorNotes\relax
\let\endinstructorNotes\relax
%%% instructorNotes environment
\ifhandout
\newenvironment{instructorNotes}[1][false]%
{%
\def\givenatend{\boolean{#1}}\ifthenelse{\boolean{#1}}{\begin{trivlist}\item}{\setbox0\vbox\bgroup}{}
}
{%
\ifthenelse{\givenatend}{\end{trivlist}}{\egroup}{}
}
\else
\newenvironment{instructorNotes}[1][false]%
{%
  \ifthenelse{\boolean{#1}}{\begin{trivlist}\item[\hskip \labelsep\bfseries {\Large Instructor Notes: \\} \hspace{\textwidth} ]}
{\begin{trivlist}\item[\hskip \labelsep\bfseries {\Large Instructor Notes: \\} \hspace{\textwidth} ]}
{}
}
{\end{trivlist}}
\fi


%% Suggested Timing
\newcommand{\timing}[1]{{\bf Suggested Timing: \hspace{2ex}} #1}




\hypersetup{
    colorlinks=true,       % false: boxed links; true: colored links
    linkcolor=blue,          % color of internal links (change box color with linkbordercolor)
    citecolor=green,        % color of links to bibliography
    filecolor=magenta,      % color of file links
    urlcolor=cyan           % color of external links
}
\title{Return to Walk the Line}
\author{Vic Ferdinand, Betsy McNeal, Jenny Sheldon}

\begin{document}
\begin{abstract}
 (You have seen some of these problems earlier this semester.)
\end{abstract}
\maketitle

\begin{instructorIntro}
The goal of this activity is to understand the connection between linear relationships, slope, and scaling.  These ideas will be represented in both equations and graphs.  

The first thing that we are going to do is go back to the problems from the original Walk the Line and make sure that we answer those questions using graphs and equations. Once we have established the connection between constant rate of change and linear graphs and equations, then we discuss the relationship with scaling.

Our previous treatment was particularly related to arithmetic sequences, so the relationship to graphs and equations is new. 

\end{instructorIntro}

\begin{problem}
 Free-Lance Freddy works for varying hourly rates, depending on the
job.  He also carries some spare cash for lunch.  To make his
customers sweat, Freddy keeps a meter on his belt telling how much
money they currently owe (with his lunch money added in).  On Monday, 3 hours into his work as a gourmet burger flipper,
  Freddy's meter reads $\$42$. 7 hours into his work, his meter reads
  $\$86$.  
  \begin{enumerate}
  \item Draw a graph that could represent Freddy's situation. Be ready to explain how you drew your graph and how each part of your graph represents the story situation.
  \item Earlier this semester you answered the following question using a table:  {\bf If he works for 9 hours, how much money will he have?  When
  will he have $\$75$?}  Can you use your graph to answer the same question?
  \end{enumerate}
\end{problem}


\begin{problem}
\emph{Slimy Sam is on the lam from the law.  Being not-too-smart, he drives
the clunker of a car he stole east on I-70 across Ohio.  Because the
car can only go a maximum of 52 miles per hour, he floors it all the
way from where he stole the car (just now at the Rest Area 5 miles
west of the Indiana line) and goes as far as he can before running out
of gas 3.78 hours from now.}
 \begin{enumerate}
  \item Draw a graph that could represent Slimy's situation.  Be ready to explain how you drew your graph and how each part of your graph represents the story situation.

  \item Earlier this semester you answered the following questions using a table:  
  \begin{enumerate}[label=(\roman*)]
   {\bf \item Where will he be 3 hours after stealing the car?
 \item  Where will he be when he runs out of gas and is arrested?
 \item When will he be at mile marker 99 (east of Indiana)?
\item When will he be at mile marker 71.84?}
\end{enumerate}
Can you use your graph to answer the same questions?

\end{enumerate}

\end{problem}


\begin{problem} 
\emph{Counterfeit Cathy sells two kinds of fake cereal: Square Cheerios for
$\$$4 per pound and Sugarless Sugar Pops for $\$$5 per pound.  If Cathy's goal for today is to sell $\$$1000 of cereal, how
  much of each kind should she sell?}

\begin{enumerate}
  \item Can you answer this question using a table? If yes, be ready to describe how you used the table to answer this question. If no, explain why not.
    \item Draw a graph that could represent Cathy's situation.  Be ready to explain how you drew your graph and how each part of your graph represents the story situation.
  \item Can you use your graph to answer the same questions?
  \end{enumerate}
  
  \begin{instructorNotes}
  This problem is not on the original Walk the Line activity, so this context is new.  While Slimy Sam is the slope-intercept form of the line and Freelance-Freddy is the point-slope form of the line, Counterfeit Cathy is the general form ($Ax + By = C$).
  
  If you're running out of time, feel free to skip this problem.  Students often find it challenging.
  \end{instructorNotes}
  
\end{problem}

\begin{problem}
For each of the situations above, write an equation relating the two
variables (hours and current financial status, hours and position, 
pounds of Square Cheerios and pounds of Sugarless Sugar Pops) and
answer the following questions:
\begin{enumerate}
\item Could you have used your equations to help you solve the problems
  above? Are all three equations similar to each other? If yes, how are they alike?  If no, how are they different?
   \item How does each equation reflect the table you made earlier in the semester?
  \item  What are the different features of your graphs represented in each equation?
 \end{enumerate}
 
 \begin{instructorNotes}
 Students may have already used an equation of some sort to solve these problems (either earlier in the semester or earlier in the class).  The goal here is to focus in on the connection between these equations and their tables and graphs.  How is each part of the equation represented?  Where do we see the slope in each representation?  (etc).
 \end{instructorNotes}
\end{problem}

\begin{problem}
Compare and contrast the graphs you have drawn in this activity.  Make conjectures about why your observations are true.
\begin{instructorNotes}
In this question, we would like the students to connect the idea of constant rate of change with constant slope.  In particular, they should recognize that all these situations have lines as their graphs, and that lines represent a constant rate of change or constant slope.  Go ahead and press on this issue, since it will likely seem obvious to the students.  You should discuss it from ``both directions'': if a graph has constant slope, must it be a line?  If a graph is a line, must it have constant slope?  This idea should be made explicit.
\end{instructorNotes}
\end{problem}

\begin{problem}
These three problems all involve scaling.  Looking back over your work, what is the role of scaling in these three problems?

\begin{instructorNotes}
Our end goal here is to help students equate ``slope" with ``same ratio" with ``scaling".  This might take the form of observing similar triangles when they sketch the ``over and up", or perhaps in the verbal formulation that ``increasing the $x$ value by a factor 3 will result an increase (or decrease) of the $y$ value by a factor of 3". We will return to this concept in the ratio unit with ``Orange You Glad We're Back to Ratios''.
\end{instructorNotes}
\end{problem}






\end{document}