\documentclass{ximera}
\usepackage{gensymb}
\usepackage{tabularx}
\usepackage{mdframed}
\usepackage{pdfpages}
%\usepackage{chngcntr}

\let\problem\relax
\let\endproblem\relax

\newcommand{\property}[2]{#1#2}




\newtheoremstyle{SlantTheorem}{\topsep}{\fill}%%% space between body and thm
 {\slshape}                      %%% Thm body font
 {}                              %%% Indent amount (empty = no indent)
 {\bfseries\sffamily}            %%% Thm head font
 {}                              %%% Punctuation after thm head
 {3ex}                           %%% Space after thm head
 {\thmname{#1}\thmnumber{ #2}\thmnote{ \bfseries(#3)}} %%% Thm head spec
\theoremstyle{SlantTheorem}
\newtheorem{problem}{Problem}[]

%\counterwithin*{problem}{section}



%%%%%%%%%%%%%%%%%%%%%%%%%%%%Jenny's code%%%%%%%%%%%%%%%%%%%%

%%% Solution environment
%\newenvironment{solution}{
%\ifhandout\setbox0\vbox\bgroup\else
%\begin{trivlist}\item[\hskip \labelsep\small\itshape\bfseries Solution\hspace{2ex}]
%\par\noindent\upshape\small
%\fi}
%{\ifhandout\egroup\else
%\end{trivlist}
%\fi}
%
%
%%% instructorIntro environment
%\ifhandout
%\newenvironment{instructorIntro}[1][false]%
%{%
%\def\givenatend{\boolean{#1}}\ifthenelse{\boolean{#1}}{\begin{trivlist}\item}{\setbox0\vbox\bgroup}{}
%}
%{%
%\ifthenelse{\givenatend}{\end{trivlist}}{\egroup}{}
%}
%\else
%\newenvironment{instructorIntro}[1][false]%
%{%
%  \ifthenelse{\boolean{#1}}{\begin{trivlist}\item[\hskip \labelsep\bfseries Instructor Notes:\hspace{2ex}]}
%{\begin{trivlist}\item[\hskip \labelsep\bfseries Instructor Notes:\hspace{2ex}]}
%{}
%}
%% %% line at the bottom} 
%{\end{trivlist}\par\addvspace{.5ex}\nobreak\noindent\hung} 
%\fi
%
%


\let\instructorNotes\relax
\let\endinstructorNotes\relax
%%% instructorNotes environment
\ifhandout
\newenvironment{instructorNotes}[1][false]%
{%
\def\givenatend{\boolean{#1}}\ifthenelse{\boolean{#1}}{\begin{trivlist}\item}{\setbox0\vbox\bgroup}{}
}
{%
\ifthenelse{\givenatend}{\end{trivlist}}{\egroup}{}
}
\else
\newenvironment{instructorNotes}[1][false]%
{%
  \ifthenelse{\boolean{#1}}{\begin{trivlist}\item[\hskip \labelsep\bfseries {\Large Instructor Notes: \\} \hspace{\textwidth} ]}
{\begin{trivlist}\item[\hskip \labelsep\bfseries {\Large Instructor Notes: \\} \hspace{\textwidth} ]}
{}
}
{\end{trivlist}}
\fi


%% Suggested Timing
\newcommand{\timing}[1]{{\bf Suggested Timing: \hspace{2ex}} #1}




\hypersetup{
    colorlinks=true,       % false: boxed links; true: colored links
    linkcolor=blue,          % color of internal links (change box color with linkbordercolor)
    citecolor=green,        % color of links to bibliography
    filecolor=magenta,      % color of file links
    urlcolor=cyan           % color of external links
}

\title{Volume Problems}
\author{Vic Ferdinand, Betsy McNeal, Jenny Sheldon}

\begin{document}
\begin{abstract}
\end{abstract}

\maketitle

\begin{instructorIntro}
This activity is about volume and area.  This is the first time students work to calculate volumes of irregular figures and the first time we work with both volume and surface area.  We have already worked with moving and additivity principles of volume, so applying them should be straightforward.

The final question is difficult for students, and may require a lot of discussion and ideas from the class.  If you need to skip this problem for time, the topic of surface area is also included in one of the projects so you may encourage that group to come back to this problem.

Notice that both of the first two problems involve simple unit conversions, so you'll need to have covered that topic first.  Usually area and volume conversions are the previous activity.
\end{instructorIntro}

\begin{problem}
Find the volume of the following solid in two different ways.

\[
\includegraphics[height=1.5in]{graphics/prism1.pdf}
\]

\begin{instructorNotes}
There are more than two ways to solve this problem - feel free to have as many present as possible!  This should feel similar to our first exercise with moving and additivity with area.
\end{instructorNotes}
\end{problem}

\begin{problem}
The figure depicts a quonset hut that will house temporary workers.  (You can google ``quonset hut'' if you've never heard of one!)  The front and back faces are semi-circular with a height of 12 feet, and the floor of the hut is a square.

\[
\includegraphics[height=1.5in]{graphics/prism2.pdf}
\]

\begin{enumerate}
    \item Find the volume and surface area of the hut in two different ways (each).
    \item The ``roof'' of the hut will be constructed from steel which costs \$2.50 per square foot. The front and back will be constructed from steel which costs \$4 per square foot.  The ``floor'' of the hut will be constructed from wooden planks which cost \$1 per square yard. How much will it cost to build this hut?
\end{enumerate}

\begin{instructorNotes}
Make sure to have students demonstrate different methods.  You might ask why the volume can't be calculated by taking the area of the square base ($24 \times 24$) and multiplying by the height of the hut (all of the volume slices aren't the same in this case, so multiplication isn't appropriate).  

For the second part of this problem, 
\end{instructorNotes}
\end{problem}

\newpage
\begin{problem}
You are teaching a lesson about volume and surface area.
    \begin{itemize}
        \item Aly claims that the volume of a solid must always be less than the surface area of that solid, because the surface area holds in the volume.
        \item Simone claims that the volume of a solid must always be greater than the surface area of that solid, because the surface area is so thin.
    \end{itemize}
    Discuss the girls' ideas.  Is either of them completely correct?  How could you help each of the girls better understand the relationship between volume and surface area?
    
    \begin{instructorNotes}
    Students often appear happy to just accept ideas that sound plausible.  Make sure you encourage them to work with examples and make some computational comparisons.  Students may need to be encouraged to work with fractional side lengths or other non-whole numbers.
    
    There are some nice dimensionality issues that can come up here.  Why would it make sense to even compare area and volume?  Students will probably not notice how strange this question is without some prompting.  
    
    You might also talk about how moving and additivity of area affects the included volume (similarly to how we discussed the relationship between perimeter and area).  For instance, if you have a beach ball that's inflated, it encloses a certain amount of volume, but once you start to deflate the ball you are enclosing a different amount of volume with the same surface area.
    \end{instructorNotes}
\end{problem}




\end{document}