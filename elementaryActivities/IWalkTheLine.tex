\documentclass{ximera}
\usepackage{gensymb}
\usepackage{tabularx}
\usepackage{mdframed}
\usepackage{pdfpages}
%\usepackage{chngcntr}

\let\problem\relax
\let\endproblem\relax

\newcommand{\property}[2]{#1#2}




\newtheoremstyle{SlantTheorem}{\topsep}{\fill}%%% space between body and thm
 {\slshape}                      %%% Thm body font
 {}                              %%% Indent amount (empty = no indent)
 {\bfseries\sffamily}            %%% Thm head font
 {}                              %%% Punctuation after thm head
 {3ex}                           %%% Space after thm head
 {\thmname{#1}\thmnumber{ #2}\thmnote{ \bfseries(#3)}} %%% Thm head spec
\theoremstyle{SlantTheorem}
\newtheorem{problem}{Problem}[]

%\counterwithin*{problem}{section}



%%%%%%%%%%%%%%%%%%%%%%%%%%%%Jenny's code%%%%%%%%%%%%%%%%%%%%

%%% Solution environment
%\newenvironment{solution}{
%\ifhandout\setbox0\vbox\bgroup\else
%\begin{trivlist}\item[\hskip \labelsep\small\itshape\bfseries Solution\hspace{2ex}]
%\par\noindent\upshape\small
%\fi}
%{\ifhandout\egroup\else
%\end{trivlist}
%\fi}
%
%
%%% instructorIntro environment
%\ifhandout
%\newenvironment{instructorIntro}[1][false]%
%{%
%\def\givenatend{\boolean{#1}}\ifthenelse{\boolean{#1}}{\begin{trivlist}\item}{\setbox0\vbox\bgroup}{}
%}
%{%
%\ifthenelse{\givenatend}{\end{trivlist}}{\egroup}{}
%}
%\else
%\newenvironment{instructorIntro}[1][false]%
%{%
%  \ifthenelse{\boolean{#1}}{\begin{trivlist}\item[\hskip \labelsep\bfseries Instructor Notes:\hspace{2ex}]}
%{\begin{trivlist}\item[\hskip \labelsep\bfseries Instructor Notes:\hspace{2ex}]}
%{}
%}
%% %% line at the bottom} 
%{\end{trivlist}\par\addvspace{.5ex}\nobreak\noindent\hung} 
%\fi
%
%


\let\instructorNotes\relax
\let\endinstructorNotes\relax
%%% instructorNotes environment
\ifhandout
\newenvironment{instructorNotes}[1][false]%
{%
\def\givenatend{\boolean{#1}}\ifthenelse{\boolean{#1}}{\begin{trivlist}\item}{\setbox0\vbox\bgroup}{}
}
{%
\ifthenelse{\givenatend}{\end{trivlist}}{\egroup}{}
}
\else
\newenvironment{instructorNotes}[1][false]%
{%
  \ifthenelse{\boolean{#1}}{\begin{trivlist}\item[\hskip \labelsep\bfseries {\Large Instructor Notes: \\} \hspace{\textwidth} ]}
{\begin{trivlist}\item[\hskip \labelsep\bfseries {\Large Instructor Notes: \\} \hspace{\textwidth} ]}
{}
}
{\end{trivlist}}
\fi


%% Suggested Timing
\newcommand{\timing}[1]{{\bf Suggested Timing: \hspace{2ex}} #1}




\hypersetup{
    colorlinks=true,       % false: boxed links; true: colored links
    linkcolor=blue,          % color of internal links (change box color with linkbordercolor)
    citecolor=green,        % color of links to bibliography
    filecolor=magenta,      % color of file links
    urlcolor=cyan           % color of external links
}
\title{I Walk The Line}
\author{Vic Ferdinand, Betsy McNeal, Jenny Sheldon}

\begin{document}
\begin{abstract}
\end{abstract}
\maketitle

\begin{instructorIntro}
Goals for this activity:
\begin{enumerate}
\item Students think about how to solve a new problem.
\item  This first pass at ``I Walk the Line" is intended to build on the idea of an arithmetic sequence, an approach that could work with grade 4-6 students.  We intend to emphasize  recognition of a constant pattern of growth without reference to graphing or equations.
\item We want students to recognize that there is a constant rate of change in the following problems and that this is the same concept seen in Gertrude the Gumchewer.
\end{enumerate}

Teaching Notes:
\begin{enumerate}
\item Our intent in this activity is to get students looking at linear relationships with fresh eyes, that is, without automatically jumping to an algebraic expression. 
\item  After both problems and their discussion have been completed, there should be some discussion of how the two situations are alike and how they are different.  We intend to emphasize that both story situations involve a constant rate of change, that this is additive and depends on the amount of change in the independent variable, and that the two stories have different starting information.  
\item  We want to lay the foundation for the two most common forms of linear equations, point-slope and $y$-intercept, without getting into equations themselves right now.  
\end{enumerate}
\end{instructorIntro}

\begin{problem}
 \emph{Free-Lance Freddy works a different job each day.  At each job, he earns an hourly rate that depends on the
job.  He also carries some spare cash for lunch.  To make his
customers sweat, Freddy keeps a meter on his belt telling how much
money they currently owe (with his lunch money added in).  On Monday, 3 hours into his work as a gourmet burger flipper,
  Freddy's meter reads $\$42$. 7 hours into his work, his meter reads
  $\$86$.  If he works for 9 hours, how much money will he have?  When
  will he have $\$75$?}
%%\item On Tuesday, Freddy is President of the United States.  After
%  2.53 hours of work, his meter reads $\$863.15$ and after 5.71 hours
%  of work, his meter reads $\$$1349.78.  If he works for 10.34 hours,
%  how much money will he have?  How much time will he be in office to
%  have $\$$1759.21?

*Try to solve this problem yourself.  Then compare your work to each of the student methods described below.  

\begin{enumerate}
\item Autumn says that she cannot find the answer to this question because she does not know how much lunch money Freddy had.
\item Emilio computed the hourly rate to be $\$14$ and then concluded that the answer should be $\$126$.
\item Cassie used a chart to find an hourly rate of $\$11$ and then used the chart to find the lunch money amount to be $\$9$.

\end{enumerate}
*Consider each student's approach to the problem.  Do you agree or disagree with the student's method? What might their reasoning have been?  What is the correct answer?
\end{problem}

\newpage

\begin{problem}
\emph{Slimy Sam is on the lam from the law.  Being not-too-smart, he drives
the clunker of a car he stole east on I-70 across Ohio.  Because the
car can only go a maximum of 52 miles per hour, he floors it all the
way from where he stole the car (just now at the Rest Area 5 miles
west of the Indiana line) and goes as far as he can before running out
of gas 3.78 hours from now.}
\begin{enumerate}[label=(\roman*)]
 \item{Where will he be 3 hours after stealing the car?
 \item Where will he be when he runs out of gas and is arrested?
 \item  When will he be at mile marker 99 (east of Indiana)?
\item When will he be at mile marker 71.84?}
\end{enumerate}

Emilio used a chart to solve this problem and another student, Autumn, used an equation. 
\begin{enumerate}

\item Construct a chart that Emilio might have used.  Explain what his reasoning might have been.

\item Write an equation that Autumn might have used.   Explain what her reasoning might have been.

\end{enumerate}
*Consider Emilio's and Autumn's approaches to the problem.  What do you think of their methods?  Can both methods be used to obtain the correct answer?  Is there another method? Is there a best method? 
\end{problem}

\begin{problem}
Now let's try to generalize these two situations.
\begin{enumerate}
    \item Suppose that Free Lance Freddy's meter shows that he has earned $\$d1 $ after working for $h1$ hours, and his meter shows $\$d2 $ after working for $h2 $ hours. How much money, $y$, will his meter show he has earned after working for a total of $x$ hours?
    \item Suppose that Slimy Sam's car can only go a maximum of S miles per hour all the way from where he stole the car (just now at the Rest Area $P$ miles west of the Indiana line) and goes as far as he can before running out of gas T hours from now.
\begin{enumerate}[label=(\roman*)]
 \item Where will he be $x$ hours after stealing the car?
 \item Where will he be when he runs out of gas and is arrested?
 \item  When will he be at mile marker $y$ (east of Indiana)?
%\item When will he be at mile marker 71.84?}
\end{enumerate}

\end{enumerate}



\end{problem}

\end{document}