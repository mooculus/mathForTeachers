\documentclass[nooutcomes]{ximera}
\usepackage{gensymb}
\usepackage{tabularx}
\usepackage{mdframed}
\usepackage{pdfpages}
%\usepackage{chngcntr}

\let\problem\relax
\let\endproblem\relax

\newcommand{\property}[2]{#1#2}




\newtheoremstyle{SlantTheorem}{\topsep}{\fill}%%% space between body and thm
 {\slshape}                      %%% Thm body font
 {}                              %%% Indent amount (empty = no indent)
 {\bfseries\sffamily}            %%% Thm head font
 {}                              %%% Punctuation after thm head
 {3ex}                           %%% Space after thm head
 {\thmname{#1}\thmnumber{ #2}\thmnote{ \bfseries(#3)}} %%% Thm head spec
\theoremstyle{SlantTheorem}
\newtheorem{problem}{Problem}[]

%\counterwithin*{problem}{section}



%%%%%%%%%%%%%%%%%%%%%%%%%%%%Jenny's code%%%%%%%%%%%%%%%%%%%%

%%% Solution environment
%\newenvironment{solution}{
%\ifhandout\setbox0\vbox\bgroup\else
%\begin{trivlist}\item[\hskip \labelsep\small\itshape\bfseries Solution\hspace{2ex}]
%\par\noindent\upshape\small
%\fi}
%{\ifhandout\egroup\else
%\end{trivlist}
%\fi}
%
%
%%% instructorIntro environment
%\ifhandout
%\newenvironment{instructorIntro}[1][false]%
%{%
%\def\givenatend{\boolean{#1}}\ifthenelse{\boolean{#1}}{\begin{trivlist}\item}{\setbox0\vbox\bgroup}{}
%}
%{%
%\ifthenelse{\givenatend}{\end{trivlist}}{\egroup}{}
%}
%\else
%\newenvironment{instructorIntro}[1][false]%
%{%
%  \ifthenelse{\boolean{#1}}{\begin{trivlist}\item[\hskip \labelsep\bfseries Instructor Notes:\hspace{2ex}]}
%{\begin{trivlist}\item[\hskip \labelsep\bfseries Instructor Notes:\hspace{2ex}]}
%{}
%}
%% %% line at the bottom} 
%{\end{trivlist}\par\addvspace{.5ex}\nobreak\noindent\hung} 
%\fi
%
%


\let\instructorNotes\relax
\let\endinstructorNotes\relax
%%% instructorNotes environment
\ifhandout
\newenvironment{instructorNotes}[1][false]%
{%
\def\givenatend{\boolean{#1}}\ifthenelse{\boolean{#1}}{\begin{trivlist}\item}{\setbox0\vbox\bgroup}{}
}
{%
\ifthenelse{\givenatend}{\end{trivlist}}{\egroup}{}
}
\else
\newenvironment{instructorNotes}[1][false]%
{%
  \ifthenelse{\boolean{#1}}{\begin{trivlist}\item[\hskip \labelsep\bfseries {\Large Instructor Notes: \\} \hspace{\textwidth} ]}
{\begin{trivlist}\item[\hskip \labelsep\bfseries {\Large Instructor Notes: \\} \hspace{\textwidth} ]}
{}
}
{\end{trivlist}}
\fi


%% Suggested Timing
\newcommand{\timing}[1]{{\bf Suggested Timing: \hspace{2ex}} #1}




\hypersetup{
    colorlinks=true,       % false: boxed links; true: colored links
    linkcolor=blue,          % color of internal links (change box color with linkbordercolor)
    citecolor=green,        % color of links to bibliography
    filecolor=magenta,      % color of file links
    urlcolor=cyan           % color of external links
}

\title{Building Blocks}
\author{Vic Ferdinand, Betsy McNeal, Jenny Sheldon, Mike Steward}

\begin{document}
\begin{abstract}
\end{abstract}

\maketitle





\begin{problem} 
Measure the length of the line below; then using plastic tiles or a ruler, build a rectangle with area $4$ sq. in. with its base on the line.

\vspace{1 in}
\tikz{ \draw (-2 in,0) -- (2 in,0); \draw (-2 in, -2 pt) -- (-2 in, 2 pt); \draw (-1 in, -2 pt) -- (-1 in, 2 pt);\draw (0 in, -2 pt) -- (0 in, 2 pt);\draw (1 in, -2 pt) -- (1 in, 2 pt); \draw (2 in, -2 pt) -- (2 in, 2 pt);}


\end{problem}

\begin{problem}
Build a rectangle with area $12$ sq. in. with its base on the line.

\vspace{3.25 in}
\tikz{ \draw (-2 in,0) -- (2 in,0); \draw (-2 in, -2 pt) -- (-2 in, 2 pt); \draw (-1 in, -2 pt) -- (-1 in, 2 pt);\draw (0 in, -2 pt) -- (0 in, 2 pt);\draw (1 in, -2 pt) -- (1 in, 2 pt); \draw (2 in, -2 pt) -- (2 in, 2 pt);}


\end{problem}

\newpage
\begin{problem}
 We are familiar with the formula for the area of a rectangle $A = \ell \times w$.  Using our Math 1125 interpretation of multiplication, explain why this formula is accurate.


\end{problem}


Let's repeat the same questions with a new unit of area, pictured below.

\tikz{ \draw (0 cm,0) -- (2.3 cm,0); \draw (2.3 cm, 0) -- (3.5 cm, 2 cm); \draw (3.5 cm, 2 cm) -- (1.2 cm, 2 cm);\draw (1.2 cm, 2 cm) -- (0 cm, 0);}  = 1 unit of area


\begin{problem}
Build a parallelogram with an area of $4$ units on the line below.

\vspace{2.5 cm}
\tikz{ \draw (0 cm,0) -- (9.2 cm,0); \draw (0 cm, -2 pt) -- (0cm, 2 pt); \draw (2.3 cm, -2 pt) -- (2.3cm, 2 pt);\draw (4.6cm, -2 pt) -- (4.6 cm, 2 pt);\draw (6.9 cm, -2 pt) -- (6.9 cm, 2 pt); \draw (9.2 cm, -2 pt) -- (9.2 cm, 2 pt);}

\end{problem}

\begin{problem}
Build a parallelogram with an area of $12$ units on the line below.  Then build a parallelogram with an area of 20 units, and finally one with an area of 8 units.  What area formula is emerging?

\vspace{6.5cm}
\tikz{ \draw (0 cm,0) -- (9.2 cm,0); \draw (0 cm, -2 pt) -- (0cm, 2 pt); \draw (2.3 cm, -2 pt) -- (2.3cm, 2 pt);\draw (4.6cm, -2 pt) -- (4.6 cm, 2 pt);\draw (6.9 cm, -2 pt) -- (6.9 cm, 2 pt); \draw (9.2 cm, -2 pt) -- (9.2 cm, 2 pt);}

\end{problem}

\begin{problem}
Draw a new unit of area.  Now swap units of area with someone else in your group.  Can you draw a shape whose area in those units has the formula we noticed above?

\end{problem}

\newpage
\begin{instructorNotes}
This activity is meant to help students understand why and when the formula $A = L \times W$ is appropriate for calculating area.  This activity is designed in this way to be a precursor to ``Building Blocks Revisited'', where we consider why and when the volume formula $V = A \times H$ is appropriate.  

In this activity, students are instructed to create rectangles as arrays of squares.  The activity can educe the concepts of arrays, areas of rectangles, and in the later problems, nonstandard units of area.  The key point we make in this activity is the idea that the area formula for rectangles, while literally a product of two lengths, can really be thought of as $A = $ (number of rows) $\times$ (number of area units per row).  We are connecting this multiplication formula with which the students are very familiar to the ``covering'' idea of counting area that has been emphasized in our measurement unit.

This activity is our first activity about area, though we have already discussed the meaning of area in our measurement unit.  We follow this activity by beginning to consider area formulas for other shapes.

This activity calls for some 1 inch by 1 inch by 1 inch cubes (or 1 inch by 1 inch squares) and pattern blocks.



\timing{This activity takes about 1 class period.  Have the students work through the first three problems, then discuss and try to elicit the idea of the area formula described above.  Then have the students move on to the last three questions.}

\end{instructorNotes}


\end{document}