\documentclass{ximera}
\usepackage{gensymb}
\usepackage{tabularx}
\usepackage{mdframed}
\usepackage{pdfpages}
%\usepackage{chngcntr}

\let\problem\relax
\let\endproblem\relax

\newcommand{\property}[2]{#1#2}




\newtheoremstyle{SlantTheorem}{\topsep}{\fill}%%% space between body and thm
 {\slshape}                      %%% Thm body font
 {}                              %%% Indent amount (empty = no indent)
 {\bfseries\sffamily}            %%% Thm head font
 {}                              %%% Punctuation after thm head
 {3ex}                           %%% Space after thm head
 {\thmname{#1}\thmnumber{ #2}\thmnote{ \bfseries(#3)}} %%% Thm head spec
\theoremstyle{SlantTheorem}
\newtheorem{problem}{Problem}[]

%\counterwithin*{problem}{section}



%%%%%%%%%%%%%%%%%%%%%%%%%%%%Jenny's code%%%%%%%%%%%%%%%%%%%%

%%% Solution environment
%\newenvironment{solution}{
%\ifhandout\setbox0\vbox\bgroup\else
%\begin{trivlist}\item[\hskip \labelsep\small\itshape\bfseries Solution\hspace{2ex}]
%\par\noindent\upshape\small
%\fi}
%{\ifhandout\egroup\else
%\end{trivlist}
%\fi}
%
%
%%% instructorIntro environment
%\ifhandout
%\newenvironment{instructorIntro}[1][false]%
%{%
%\def\givenatend{\boolean{#1}}\ifthenelse{\boolean{#1}}{\begin{trivlist}\item}{\setbox0\vbox\bgroup}{}
%}
%{%
%\ifthenelse{\givenatend}{\end{trivlist}}{\egroup}{}
%}
%\else
%\newenvironment{instructorIntro}[1][false]%
%{%
%  \ifthenelse{\boolean{#1}}{\begin{trivlist}\item[\hskip \labelsep\bfseries Instructor Notes:\hspace{2ex}]}
%{\begin{trivlist}\item[\hskip \labelsep\bfseries Instructor Notes:\hspace{2ex}]}
%{}
%}
%% %% line at the bottom} 
%{\end{trivlist}\par\addvspace{.5ex}\nobreak\noindent\hung} 
%\fi
%
%


\let\instructorNotes\relax
\let\endinstructorNotes\relax
%%% instructorNotes environment
\ifhandout
\newenvironment{instructorNotes}[1][false]%
{%
\def\givenatend{\boolean{#1}}\ifthenelse{\boolean{#1}}{\begin{trivlist}\item}{\setbox0\vbox\bgroup}{}
}
{%
\ifthenelse{\givenatend}{\end{trivlist}}{\egroup}{}
}
\else
\newenvironment{instructorNotes}[1][false]%
{%
  \ifthenelse{\boolean{#1}}{\begin{trivlist}\item[\hskip \labelsep\bfseries {\Large Instructor Notes: \\} \hspace{\textwidth} ]}
{\begin{trivlist}\item[\hskip \labelsep\bfseries {\Large Instructor Notes: \\} \hspace{\textwidth} ]}
{}
}
{\end{trivlist}}
\fi


%% Suggested Timing
\newcommand{\timing}[1]{{\bf Suggested Timing: \hspace{2ex}} #1}




\hypersetup{
    colorlinks=true,       % false: boxed links; true: colored links
    linkcolor=blue,          % color of internal links (change box color with linkbordercolor)
    citecolor=green,        % color of links to bibliography
    filecolor=magenta,      % color of file links
    urlcolor=cyan           % color of external links
}
\title{Which Transformation Is This?}
\author{Vic Ferdinand, Betsy McNeal, Jenny Sheldon}

\begin{document}
\begin{abstract}
\end{abstract}
\maketitle

\begin{instructorIntro}

The goal of this activity is to have students explore how different transformations interact.  Your discussion should be driven more by student observation than to answer the question.  Up to this point, we have only considered the result of a single transformation.

This activity is a complicated version of this question.  You can skip this activity entirely if you don't have time for it, or replace with a simpler version.

The questions in this activity are a little more open-ended than many we ask.  Encourage the students to come up with their own explicit examples to justify their claims.

Tracing paper can be helpful for students to test their ideas!
\end{instructorIntro}

This figure shows a zig-zag shape that has been transformed using two different transformations.  The original shape is on the left, and the transformed shape is on the right.

\[
\includegraphics[height=2.25in]{graphics/transformNoHelp.png}
\]

\begin{problem}
Is exactly one of the transformations a reflection?  Why or why not?
\begin{instructorNotes}
The answer is no because the orientation of the figure hasn't changed.  The students will likely not know this terminology, so you can introduce it if you like.  But, they should be able to describe how one reflection would leave the shape ``flipped over''.
\end{instructorNotes}
\end{problem}

\begin{problem}
What is the result when we rotate twice around the same center of rotation? 
\begin{instructorNotes}
The result of doing two rotations (of angles measuring $A$ and $B$) is the same as doing one rotation (of angle $A+B$).
\end{instructorNotes}
\end{problem}

\begin{problem}
Find the two transformations that were used.
\begin{instructorNotes}
The two transformations are a translation and a rotation.  It's easiest to do the translation first (moving the lower-most point of the figures to match up) and then rotate using that point as the center.
\end{instructorNotes}
\end{problem}

\end{document}