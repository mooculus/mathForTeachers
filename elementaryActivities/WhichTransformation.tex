\documentclass[nooutcomes]{ximera}
\usepackage{gensymb}
\usepackage{tabularx}
\usepackage{mdframed}
\usepackage{pdfpages}
%\usepackage{chngcntr}

\let\problem\relax
\let\endproblem\relax

\newcommand{\property}[2]{#1#2}




\newtheoremstyle{SlantTheorem}{\topsep}{\fill}%%% space between body and thm
 {\slshape}                      %%% Thm body font
 {}                              %%% Indent amount (empty = no indent)
 {\bfseries\sffamily}            %%% Thm head font
 {}                              %%% Punctuation after thm head
 {3ex}                           %%% Space after thm head
 {\thmname{#1}\thmnumber{ #2}\thmnote{ \bfseries(#3)}} %%% Thm head spec
\theoremstyle{SlantTheorem}
\newtheorem{problem}{Problem}[]

%\counterwithin*{problem}{section}



%%%%%%%%%%%%%%%%%%%%%%%%%%%%Jenny's code%%%%%%%%%%%%%%%%%%%%

%%% Solution environment
%\newenvironment{solution}{
%\ifhandout\setbox0\vbox\bgroup\else
%\begin{trivlist}\item[\hskip \labelsep\small\itshape\bfseries Solution\hspace{2ex}]
%\par\noindent\upshape\small
%\fi}
%{\ifhandout\egroup\else
%\end{trivlist}
%\fi}
%
%
%%% instructorIntro environment
%\ifhandout
%\newenvironment{instructorIntro}[1][false]%
%{%
%\def\givenatend{\boolean{#1}}\ifthenelse{\boolean{#1}}{\begin{trivlist}\item}{\setbox0\vbox\bgroup}{}
%}
%{%
%\ifthenelse{\givenatend}{\end{trivlist}}{\egroup}{}
%}
%\else
%\newenvironment{instructorIntro}[1][false]%
%{%
%  \ifthenelse{\boolean{#1}}{\begin{trivlist}\item[\hskip \labelsep\bfseries Instructor Notes:\hspace{2ex}]}
%{\begin{trivlist}\item[\hskip \labelsep\bfseries Instructor Notes:\hspace{2ex}]}
%{}
%}
%% %% line at the bottom} 
%{\end{trivlist}\par\addvspace{.5ex}\nobreak\noindent\hung} 
%\fi
%
%


\let\instructorNotes\relax
\let\endinstructorNotes\relax
%%% instructorNotes environment
\ifhandout
\newenvironment{instructorNotes}[1][false]%
{%
\def\givenatend{\boolean{#1}}\ifthenelse{\boolean{#1}}{\begin{trivlist}\item}{\setbox0\vbox\bgroup}{}
}
{%
\ifthenelse{\givenatend}{\end{trivlist}}{\egroup}{}
}
\else
\newenvironment{instructorNotes}[1][false]%
{%
  \ifthenelse{\boolean{#1}}{\begin{trivlist}\item[\hskip \labelsep\bfseries {\Large Instructor Notes: \\} \hspace{\textwidth} ]}
{\begin{trivlist}\item[\hskip \labelsep\bfseries {\Large Instructor Notes: \\} \hspace{\textwidth} ]}
{}
}
{\end{trivlist}}
\fi


%% Suggested Timing
\newcommand{\timing}[1]{{\bf Suggested Timing: \hspace{2ex}} #1}




\hypersetup{
    colorlinks=true,       % false: boxed links; true: colored links
    linkcolor=blue,          % color of internal links (change box color with linkbordercolor)
    citecolor=green,        % color of links to bibliography
    filecolor=magenta,      % color of file links
    urlcolor=cyan           % color of external links
}
\title{Which Transformation Is This?}
\author{Vic Ferdinand, Betsy McNeal, Jenny Sheldon}

\begin{document}
\begin{abstract}
\end{abstract}
\maketitle



This figure shows a zig-zag shape that has been transformed using two different transformations.  The original shape is on the left, and the transformed shape is on the right.

\vskip 0.5in

\begin{center}
    \begin{tikzpicture}
        \draw (0,0)--(-1,2)--(2,3)--(-2,3);
        \draw[cm={cos(60) ,-sin(60) ,sin(60) ,cos(60) ,(3 cm,-1 cm)}] (0,0)--(-1,2)--(2,3)--(-2,3);
    \end{tikzpicture}
\end{center}
\vskip 0.5in


\begin{problem} \label{WhichTrans1}
Is exactly one of the transformations a reflection?  Why or why not?

\end{problem}

\begin{problem}
What is the result when we rotate twice around the same center of rotation? 

\end{problem}

\begin{problem}
Find the two transformations that were used.

\end{problem}

\newpage
\begin{instructorNotes}

The goal of this activity is to have students explore how different transformations interact.  Overall, we usually aim more to help students develop their ideas about transformations rather than just get to the correct answer.  This activity is a complicated version of this question with very open-ended problems, designed to draw out observations and questions.  There are certainly many simpler versions of this question, which is why we don't necessarily focus on getting the correct answer here.  Instead, we sometimes focus on examples that students generate to justify their claims.  Up to this point in our course, we have defined basic transformations (reflections, rotations, translations), but only considered the result of a single transformation.  We sometimes provide tracing paper for students to use with this activity.


\begin{itemize}
    \item Problem \ref{WhichTrans1} is designed to bring up the idea of orientation.  We would like the students to see that a single reflection has to leave the shape ``flipped over''.  
    \item The two transformations used to produce the second figure are a translation and a rotation.  Students tend to think it's easiest to do the translation first (moving the lower-most point of the figures to match up) and then to rotate using that point as the center.
\end{itemize}



\timing{We use half a class period for this activity.  We give students 5-10 minutes to think about the concepts in the activity with their small groups, then spend 15-20 minutes in whole-class discussion.  If necessary, we give students more time to generate examples of their ideas, and then discuss these examples, but this can stretch the activity to a full class period.}
\end{instructorNotes}

\end{document}