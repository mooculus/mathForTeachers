\documentclass{ximera}

\usepackage{gensymb}
\usepackage{tabularx}
\usepackage{mdframed}
\usepackage{pdfpages}
%\usepackage{chngcntr}

\let\problem\relax
\let\endproblem\relax

\newcommand{\property}[2]{#1#2}




\newtheoremstyle{SlantTheorem}{\topsep}{\fill}%%% space between body and thm
 {\slshape}                      %%% Thm body font
 {}                              %%% Indent amount (empty = no indent)
 {\bfseries\sffamily}            %%% Thm head font
 {}                              %%% Punctuation after thm head
 {3ex}                           %%% Space after thm head
 {\thmname{#1}\thmnumber{ #2}\thmnote{ \bfseries(#3)}} %%% Thm head spec
\theoremstyle{SlantTheorem}
\newtheorem{problem}{Problem}[]

%\counterwithin*{problem}{section}



%%%%%%%%%%%%%%%%%%%%%%%%%%%%Jenny's code%%%%%%%%%%%%%%%%%%%%

%%% Solution environment
%\newenvironment{solution}{
%\ifhandout\setbox0\vbox\bgroup\else
%\begin{trivlist}\item[\hskip \labelsep\small\itshape\bfseries Solution\hspace{2ex}]
%\par\noindent\upshape\small
%\fi}
%{\ifhandout\egroup\else
%\end{trivlist}
%\fi}
%
%
%%% instructorIntro environment
%\ifhandout
%\newenvironment{instructorIntro}[1][false]%
%{%
%\def\givenatend{\boolean{#1}}\ifthenelse{\boolean{#1}}{\begin{trivlist}\item}{\setbox0\vbox\bgroup}{}
%}
%{%
%\ifthenelse{\givenatend}{\end{trivlist}}{\egroup}{}
%}
%\else
%\newenvironment{instructorIntro}[1][false]%
%{%
%  \ifthenelse{\boolean{#1}}{\begin{trivlist}\item[\hskip \labelsep\bfseries Instructor Notes:\hspace{2ex}]}
%{\begin{trivlist}\item[\hskip \labelsep\bfseries Instructor Notes:\hspace{2ex}]}
%{}
%}
%% %% line at the bottom} 
%{\end{trivlist}\par\addvspace{.5ex}\nobreak\noindent\hung} 
%\fi
%
%


\let\instructorNotes\relax
\let\endinstructorNotes\relax
%%% instructorNotes environment
\ifhandout
\newenvironment{instructorNotes}[1][false]%
{%
\def\givenatend{\boolean{#1}}\ifthenelse{\boolean{#1}}{\begin{trivlist}\item}{\setbox0\vbox\bgroup}{}
}
{%
\ifthenelse{\givenatend}{\end{trivlist}}{\egroup}{}
}
\else
\newenvironment{instructorNotes}[1][false]%
{%
  \ifthenelse{\boolean{#1}}{\begin{trivlist}\item[\hskip \labelsep\bfseries {\Large Instructor Notes: \\} \hspace{\textwidth} ]}
{\begin{trivlist}\item[\hskip \labelsep\bfseries {\Large Instructor Notes: \\} \hspace{\textwidth} ]}
{}
}
{\end{trivlist}}
\fi


%% Suggested Timing
\newcommand{\timing}[1]{{\bf Suggested Timing: \hspace{2ex}} #1}




\hypersetup{
    colorlinks=true,       % false: boxed links; true: colored links
    linkcolor=blue,          % color of internal links (change box color with linkbordercolor)
    citecolor=green,        % color of links to bibliography
    filecolor=magenta,      % color of file links
    urlcolor=cyan           % color of external links
}

\title{History of Integers}
\begin{document}

\begin{abstract}
We look at integers from a historical point of view.
\end{abstract}

\maketitle

You might have noticed that there is one common class of numbers that we haven't much discussed, yet: \wordChoice{\choice{whole numbers} \choice[correct]{negative numbers} \choice{fractions} \choice{decimals}}.  One of the choices that the author of your textbook has made is to discuss the various kinds of operations (addition, subtraction, multiplication, division) as a whole, including many of the types of numbers in the discussion.  This helps to highlight the fact that multiplication, for instance is the same thing no matter what kinds of numbers you use.  Other textbooks take a different approach: looking at whole numbers and their operations, then fractions and their operations, and so on.  We happen to think that our textbook has made the better choice (and you can agree or disagree), but we have deviated a little in this instance. 

Now that you have noticed these facts, you might be wondering: why?

There are several reasons, but the ones we would like to discuss here are: the pattern of history and the difficulty of finding a good representation.

These two reasons are very much related.  If we look over how numbers as a concept were developed throughout history, we definitely find evidence of whole numbers by $3000$BC in Ancient Egypt.  Beginning around the same period of time in Ancient Babylon, people were starting to use a place-value system which included what we would call decimals, except the Ancient Babylonians used base $60$ instead of base $\answer[given]{10}$ as we do.  The Ancient Egyptians also had notation for unit fractions, and elaborate methods of combining these unit fractions to express any other fraction they would like.  If you're interested in this subject, you can find a nice timeline of very early mathematics at \link[MacTutor]{http://www-history.mcs.st-and.ac.uk/Chronology/30000BC_500BC.html}.

So, we have seen evidence of whole numbers, fractions, and even decimals from the beginning of mathematics.  Negative numbers, however, are a different story.  We see the first evidence of negative numbers appearing sometime between $200$BC and $200$AD in Chinese mathematics, and around the $7$th century AD in Indian mathematics, but Western European mathematicians didn't start really using negatives in their calculations until the $17$th century AD, but didn't allow them as answers to problems or give them equal status amongst the positive numbers, decimals, and fractions until the $19$th century! 

To emphasize this point a little bit, here are some things that happened in history {\em before} most mathematicians were working with negative numbers:
\begin{itemize}
	\item The bubonic plague hit Europe and Asia in the mid $1300$s.
	\item Gutenberg invented the printing press around $1440$.
	\item Christopher Columbus sails to the New World around $1492$.
	\item Martin Luther publishes his 95 theses, and the Reformation begins around $1517$.
	\item Galileo finds the Earth revolves around the sun around $1613$.
	\item Descartes and Pascal invent coordinate geometry around $1637$.
	\item Isaac Newton and Wilhelm Leibniz invent calculus around $1670$.
\end{itemize}

Much of mathematicians' hesitation to treat negative numbers as actual numbers came from the idea that mathematics was a subject that should make sense in the real world, and model real-world phenomena.  People didn't have a very good representation for negative numbers that made sense all the time.  For example, let's consider a few problems.

\begin{question}
For each of the problems below, write an expression that would solve the problem, i.e. $3+4$.
\begin{enumerate}
	\item Jaci has 8 apples, and Joseph has 12 apples.  How many apples do the two children have together? $\answer{8+12}$
	\item Jaci has $\frac12$ of an apple, and Joseph has $\frac57$ of an apple.  How many apples do the two children have together?  $\answer{\frac12 + \frac57}$
	\item Jaci has $-8$ apples, and Joseph has $-12$ apples.  How many apples do the two children have together? $\answer{-8 + -12}$
\end{enumerate}
\end{question}

Now, you might say that negative numbers weren't too bad in that situation: after all, you could likely answer the question because you've recognized the structure of addition is the same in all three cases.  But what does the problem actually {\em mean}?  What does it mean to have $-8$ apples?  Maybe you can reconcile this, as Indian mathematicians did in their earliest interpretations, to mean that Jaci owes someone $8$ apples, and Joseph owes someone $12$ apples, and perhaps this interpretation of negative numbers as debts makes sense in some cases.

Let's try another example.

\begin{question}
For each of the problems below, write an expression that would solve the problem, i.e. $3+4$.
\begin{enumerate}
	\item Sasha has 8 bags, and each bag contains 12 sheets of stickers.  How many sheets of stickers does Sasha have in total?  $\answer{(8)(12)}$
	\item Sasha has $\frac12$ of a box of stickers, and the box originally contained $\frac57$ of a sheet of stickers.  How many sheets of stickers does Sasha have? $\answer{\frac12\frac57}$
	\item Sasha has $-8$ bags, and each bag contains $-12$ sheets of stickers.  How many sheets of stickers does Sasha have in total? $\answer{(-8)(-12)}$
\end{enumerate}
\end{question}

Again, you could likely answer the third question, but the meaning here is likely even less clear.  If we understand the objects in the groups to be debts, what does it mean to have a negative group?  Why should the overall answer to this question be positive?

We will look at several kinds of models for these problems in order to try to make some sense of negative numbers.  We expect this to be pretty challenging: after all, mathematicians struggled to make sense of these ideas for centuries!





\end{document}