\documentclass{ximera}

\author{Vic Ferdinand}
\title{Instructor Overview}

\begin{document}
\begin{abstract}
    Introductory notes for instructors using the Calculus for Middle School Teachers materials.
\end{abstract}
\maketitle


The course for which these notes are intended is a calculus course specifically designed for prospective middle childhood mathematics teachers.  Its goal is to take their (supposed) knowledge of middle grades mathematics- knowledge that centers around the concepts of constant rate (slope) and area of basic shapes - and consider how we can extend and connect those concepts when the slope is not constant.  Because this is a three hour course that considers both differential (given our ``amount'' function, find something about rates) and integral (given our ``rate'' function, find something about the amounts or the change in amounts) calculus, we cannot go into all the topics covered in the typical Calculus I course.  Nor do we necessarily want to- this audience needs to see the relevance of this material from the point of view of what their students will later do with the mathematics they will teach them. 

The background of the audience is mixed.  Some have had a high school or college calculus course that they either succeeded in or struggled while others have never encountered calculus.  They do seem to have one thing in common:  The calculus (or precalculus) they experienced was one that emphasized skills over concepts.  The concepts courses for middle grades teachers at Ohio State works to intervene with conceptual thinking at the middle grades and precalculus level and this calculus course will make them think about calculus in the same way.  From the reviews I have received from students who initially thought they were ``calculus-savvy'' as well as those who had no calculus background, some``“conceptual reasoning'' inroads have apparently been made. 
Despite their (often) lack of conceptual knowledge, I have found those to have had some calculus be good helpers to those who have not, although they are by no means always mathematically correct!

In each of differential and integral calculus, the course will develop the meaning and computation of the derivative and integral through these steps:  Notation, approximation, and, where possible, exact computation (as well as ways to ease the exact computation through developing patterns and rules from concepts).  We also do this development both numerically and graphically.  Although all of these steps are developed using an applied point of view, we conclude by bringing in some ``famous'' applications of the quantity (rate or net change in amount).

Unfortunately, time does not permit topics that I wish they could experience, such as related rates, integration techniques that expand on substitution, and some multi-variable calculus (at least to see its relationships to single-variable calculus).  Also, virtually all functions in the course (except for logarithms and their derivatives) are continuous with no vertical asymptotes.

The course was originally piloted for three years using textbooks.  However, students responded better to the following group activities and so I have continued to develop and use them since 2009.  The activities should preferably be done in small groups of about 4 with reporting done by them at the board.  If time is an issue, they may be done as a whole class discussion as well.

Homework has usually been assigned and collected about every two weeks. Students are very proactive in doing their homework together if they are used to this from previous courses. I assign points to the problems according to how much work each involves.




\end{document}