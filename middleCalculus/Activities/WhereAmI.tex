\documentclass{ximera}
\usepackage{gensymb}
\usepackage{tabularx}
\usepackage{mdframed}
\usepackage{pdfpages}
%\usepackage{chngcntr}

\let\problem\relax
\let\endproblem\relax

\newcommand{\property}[2]{#1#2}




\newtheoremstyle{SlantTheorem}{\topsep}{\fill}%%% space between body and thm
 {\slshape}                      %%% Thm body font
 {}                              %%% Indent amount (empty = no indent)
 {\bfseries\sffamily}            %%% Thm head font
 {}                              %%% Punctuation after thm head
 {3ex}                           %%% Space after thm head
 {\thmname{#1}\thmnumber{ #2}\thmnote{ \bfseries(#3)}} %%% Thm head spec
\theoremstyle{SlantTheorem}
\newtheorem{problem}{Problem}[]

%\counterwithin*{problem}{section}



%%%%%%%%%%%%%%%%%%%%%%%%%%%%Jenny's code%%%%%%%%%%%%%%%%%%%%

%%% Solution environment
%\newenvironment{solution}{
%\ifhandout\setbox0\vbox\bgroup\else
%\begin{trivlist}\item[\hskip \labelsep\small\itshape\bfseries Solution\hspace{2ex}]
%\par\noindent\upshape\small
%\fi}
%{\ifhandout\egroup\else
%\end{trivlist}
%\fi}
%
%
%%% instructorIntro environment
%\ifhandout
%\newenvironment{instructorIntro}[1][false]%
%{%
%\def\givenatend{\boolean{#1}}\ifthenelse{\boolean{#1}}{\begin{trivlist}\item}{\setbox0\vbox\bgroup}{}
%}
%{%
%\ifthenelse{\givenatend}{\end{trivlist}}{\egroup}{}
%}
%\else
%\newenvironment{instructorIntro}[1][false]%
%{%
%  \ifthenelse{\boolean{#1}}{\begin{trivlist}\item[\hskip \labelsep\bfseries Instructor Notes:\hspace{2ex}]}
%{\begin{trivlist}\item[\hskip \labelsep\bfseries Instructor Notes:\hspace{2ex}]}
%{}
%}
%% %% line at the bottom} 
%{\end{trivlist}\par\addvspace{.5ex}\nobreak\noindent\hung} 
%\fi
%
%


\let\instructorNotes\relax
\let\endinstructorNotes\relax
%%% instructorNotes environment
\ifhandout
\newenvironment{instructorNotes}[1][false]%
{%
\def\givenatend{\boolean{#1}}\ifthenelse{\boolean{#1}}{\begin{trivlist}\item}{\setbox0\vbox\bgroup}{}
}
{%
\ifthenelse{\givenatend}{\end{trivlist}}{\egroup}{}
}
\else
\newenvironment{instructorNotes}[1][false]%
{%
  \ifthenelse{\boolean{#1}}{\begin{trivlist}\item[\hskip \labelsep\bfseries {\Large Instructor Notes: \\} \hspace{\textwidth} ]}
{\begin{trivlist}\item[\hskip \labelsep\bfseries {\Large Instructor Notes: \\} \hspace{\textwidth} ]}
{}
}
{\end{trivlist}}
\fi


%% Suggested Timing
\newcommand{\timing}[1]{{\bf Suggested Timing: \hspace{2ex}} #1}




\hypersetup{
    colorlinks=true,       % false: boxed links; true: colored links
    linkcolor=blue,          % color of internal links (change box color with linkbordercolor)
    citecolor=green,        % color of links to bibliography
    filecolor=magenta,      % color of file links
    urlcolor=cyan           % color of external links
}

\author{Vic Ferdinand}
\title{Where Am I?  How Fast Am I Going?}
\outcome{Use graphs of $f(t)$ and $f'(t)$ to describe position, velocity, and acceleration.}

\begin{document}
\begin{abstract}
\end{abstract}
\maketitle


\begin{instructorIntro}
Here, we reiterate their experience with $f$ and $f'$, but now we do it graphically (essentially letting them find the derivative before doing it numerically/algebraically).  It is here that they first recognize that we are finding the slope and treating $f'$ as the slope function (whose $y$-values are rates and not amounts- a difficult hurdle to overcome).  

They also, going deeper, will begin to recognize that when the function is increasing (decreasing), $f'$ is above (below) the $x$-axis and when $f$ is at a max or min value, $f'$ is $0$ ($x$-intercept of $f'$ graph.  We do mention cusps, where $f'$ is undefined, but don't dwell on them very much in the course).  They may even want to make a sign chart for $f'$ at this point.

A nice expansion of this activity would be for the students to draw their own functions and have their peers graph the derivatives.
\end{instructorIntro}

\begin{question} 
In the gondola of a hot air balloon, two instruments monitor the balloon's course over the course of a $6$-minute period.  An altimeter shows the balloon’s height and a rate-of-climb meter shows how fast the balloon rises or falls.  Here are the resultant graphs (Heights change a lot due to updrafts and downdrafts).

 
\begin{center}
\begin{tabular}{cc}
\begin{tikzpicture}
	\begin{axis}[
            width=2.2in,
            domain=0:8,
            ymax=6000, ymin=0,
            axis lines =middle,
            every axis y label/.style={at=(current axis.above origin),anchor=south},
            every axis x label/.style={at=(current axis.right of origin),anchor=west},
            xtick={1,...,8},
            ytick={1000,2000,...,6000},
            grid=both,
            grid style={dashed, gray},
          ]
	  \addplot [very thick, smooth] {3800+1350*x-410*x^2+27*x^3+818*sin(deg(pi*x/2))};
          \end{axis}
\end{tikzpicture}

&

\begin{tikzpicture}
	\begin{axis}[
            domain=0:8,
            width=2.2in,
            ymax=3000, ymin=-3000,
            axis lines =middle,
            every axis y label/.style={at=(current axis.above origin),anchor=south},
            every axis x label/.style={at=(current axis.right of origin),anchor=west},
            grid=both,
            grid style={dashed, gray},
            xtick={1,...,8},
            ytick={-3000,-2000,...,3000},
          ]
	  \addplot [very thick, smooth] {1350-820*x+81*x^2+818*cos(deg(pi*x/2))*pi/2};
          \end{axis}
\end{tikzpicture} \\
\footnotesize Altitude (ft) vs.\ time (min) & \footnotesize Vertical velocity (ft/min) vs.\ time (min)
\end{tabular}
\end{center}
 
 
 
 
 

\begin{enumerate}
\item Find when the balloon is going up (height increasing) and going down (height decreasing).  Look at each graph separately to answer the question.
\item 	At what time is the velocity of the balloon zero?  What is happening to the balloon at those times? Look at each graph separately to answer the question.
\item	Use the second graph to answer when the balloon is rising the fastest and when it is falling the fastest.  What does the first graph do at those times?
\end{enumerate}

\begin{instructorNotes}
Expect this problem to take a long time among all students!
\end{instructorNotes}
\end{question}

\begin{question} 
The graph below shows the position function of a car as it goes to work.  Answer the following questions:


\[
\begin{tikzpicture}
        \begin{axis}[
            axis lines =middle,
            clip=false,
            every axis y label/.style={at=(current axis.above origin),anchor=south},
            every axis x label/.style={at=(current axis.right of origin),anchor=west},
            ticks=none,
            axis lines =middle, xlabel=$t$, ylabel=$s$,
          ]
	  \addplot [very thick, smooth] coordinates {
          (0,0)
          (.5,.1)
          (1,.5)
          (2,2.4)
          (4,3)
          (5,3.5)
          (8,3.7)
          (9.5,4.8)
          (10,5)
          };
          \node at (axis cs:1,.5) [anchor=south east] {$A$};
          \node at (axis cs:2,2.4) [anchor=south east] {$B$};
          \node at (axis cs:4,3) [anchor=south east] {$C$};
          \node at (axis cs:5,3.5) [anchor=south east] {$D$};
          \node at (axis cs:8,3.7) [anchor=south east] {$E$};
          \addplot[color=black,fill=black,only marks,mark=*] coordinates{(1,.5)};  %% closed hole          
          \addplot[color=black,fill=black,only marks,mark=*] coordinates{(2,2.4)};  %% closed hole          
          \addplot[color=black,fill=black,only marks,mark=*] coordinates{(4,3)};  %% closed hole          
          \addplot[color=black,fill=black,only marks,mark=*] coordinates{(5,3.5)};  %% closed hole          
          \addplot[color=black,fill=black,only marks,mark=*] coordinates{(8,3.7)};  %% closed hole          
          \end{axis}
          \end{tikzpicture}
\]


\begin{enumerate}
\item	What was the initial velocity of the car?
\item	Was the car going faster at $B$ or at $C$?
\item Was the car slowing down or speeding up at $A$, $B$, and $C$?
\item	What happened between $D$ and $E$?
\item	At what times did the car turn back for home?
\end{enumerate}
\end{question}
\begin{question} 
For (a)-(c), suppose an object can move only along the positive $x$-axis.  Sketch the graph of the object's position vs. time graph and its velocity vs. time graph.
\begin{enumerate}
\item	The object is standing still.
\item	The object is moving away from the origin at a constant velocity.
\item	The object is moving toward the origin at a steadily increasing speed.
\end{enumerate}
\end{question}

\begin{question}	For each of the following, given the graph of the function, graph its derivative.


\begin{enumerate}
\item \leavevmode\vadjust{\vspace{-\baselineskip}}\newline 
      \begin{tikzpicture}
	\begin{axis}[
            domain=0:6,
            ymax=2, ymin=-2,
            xmax=6, xmin=0,
            axis lines =middle,
            every axis y label/.style={at=(current axis.above origin),anchor=south},
            every axis x label/.style={at=(current axis.right of origin),anchor=west},
            xtick={1,...,8},
            ytick={-2,...,2},
            grid=both,
            grid style={dashed, gray},
            width=4.5in,
            view={0}{90},unit vector ratio*=1.6 1 1
          ]
	  \addplot [very thick] {(-2/3)*x+2};
          \end{axis}
\end{tikzpicture}
\item \leavevmode\vadjust{\vspace{-\baselineskip}}\newline 
      \begin{tikzpicture}
	\begin{axis}[
            domain=0:8,
            ymax=2, ymin=-2,
            xmax=8, xmin=0,
            axis lines =middle,
            every axis y label/.style={at=(current axis.above origin),anchor=south},
            every axis x label/.style={at=(current axis.right of origin),anchor=west},
            xtick={1,...,8},
            ytick={-2,...,2},
            grid=both,
            grid style={dashed, gray},
            width=4.5in,
            view={0}{90},unit vector ratio*=1.6 1 1
          ]
	  \addplot [very thick] {.2*(x^2-8*x+7)};
          \end{axis}
          \end{tikzpicture}

\item  \leavevmode\vadjust{\vspace{-\baselineskip}}\newline
\pgfmathdeclarefunction{gauss}{2}{%
  \pgfmathparse{1/(#2*sqrt(2*pi))*exp(-((x-#1)^2)/(2*#2^2))}%
}
 \begin{tikzpicture}
	\begin{axis}[
            domain=0:8,
            ymax=2, ymin=-2,
            xmax=8, xmin=0,
            axis lines =middle,
            every axis y label/.style={at=(current axis.above origin),anchor=south},
            every axis x label/.style={at=(current axis.right of origin),anchor=west},
            xtick={1,...,8},
            ytick={-2,...,2},
            grid=both,
            grid style={dashed, gray},
            width=4.5in,
            view={0}{90},unit vector ratio*=1.6 1 1
          ]
	  \addplot [very thick,smooth] {2*gauss(4,.75)};
          \end{axis}
          \end{tikzpicture}

\item \leavevmode\vadjust{\vspace{-\baselineskip}}\newline 
      \begin{tikzpicture}
	\begin{axis}[
            domain=0:8,
            ymax=3, ymin=-2,
            xmax=8, xmin=0,
            axis lines =middle,
            every axis y label/.style={at=(current axis.above origin),anchor=south},
            every axis x label/.style={at=(current axis.right of origin),anchor=west},
            xtick={1,...,8},
            ytick={-2,...,3},
            grid=both,
            grid style={dashed, gray},
            width=4.5in,
            view={0}{90},unit vector ratio*=1.6 1 1
          ]
	  \addplot [very thick,smooth] {2*sin(1.5*deg(x))/(x+.5) +1};
          \end{axis}
          \end{tikzpicture}
\item \leavevmode\vadjust{\vspace{-\baselineskip}}\newline 
    \begin{tikzpicture}
	\begin{axis}[
            width=2.2in,
            domain=0:8,
            ymax=6000, ymin=0,
            axis lines =middle,
            every axis y label/.style={at=(current axis.above origin),anchor=south},
            every axis x label/.style={at=(current axis.right of origin),anchor=west},
            xtick={1,...,8},
            ytick={1000,2000,...,6000},
            grid=both,
            grid style={dashed, gray},
          ]
	  \addplot [very thick, smooth] {3800+1350*x-410*x^2+27*x^3+818*sin(deg(pi*x/2))};
          \end{axis}
    \end{tikzpicture}
\item \leavevmode\vadjust{\vspace{-\baselineskip}}\newline
    \begin{tikzpicture}
	\begin{axis}[
            domain=0:8,
            width=2.2in,
            ymax=3000, ymin=-3000,
            axis lines =middle,
            every axis y label/.style={at=(current axis.above origin),anchor=south},
            every axis x label/.style={at=(current axis.right of origin),anchor=west},
            grid=both,
            grid style={dashed, gray},
            xtick={1,...,8},
            ytick={-3000,-2000,...,3000},
          ]
	  \addplot [very thick, smooth] {1350-820*x+81*x^2+818*cos(deg(pi*x/2))*pi/2};
          \end{axis}
    \end{tikzpicture}
\end{enumerate}

\begin{instructorNotes}
Expect this problem, especially parts (c) and (d), to take a long time among all students!
\end{instructorNotes}

\end{question}


\end{document}