\documentclass[handout]{ximera}

\graphicspath{
  {./}
  {graphics/}
  {../graphics/}
}

\usepackage{chngcntr}

\let\question\relax
\let\endquestion\relax




\newtheoremstyle{SlantTheorem}{\topsep}{\fill}%%% space between body and thm
%\newtheoremstyle{SlantTheorem}{\topsep}{\topsep}%%% space between body and thm
 {\slshape}                      %%% Thm body font
 {}                              %%% Indent amount (empty = no indent)
 {\bfseries\sffamily}            %%% Thm head font
 {}                              %%% Punctuation after thm head
 {3ex}                           %%% Space after thm head
 {\thmname{#1}\thmnumber{ #2}\thmnote{ \bfseries(#3)}}%%% Thm head spec
\theoremstyle{SlantTheorem}
\newtheorem{question}{Question}
\counterwithin*{question}{section}



\let\instructorNotes\relax
\let\endinstructorNotes\relax
%%% instructorNotes environment
\ifhandout
\newenvironment{instructorNotes}[1][false]%
{%
\def\givenatend{\boolean{#1}}\ifthenelse{\boolean{#1}}{\begin{trivlist}\item}{\setbox0\vbox\bgroup}{}
}
{%
\ifthenelse{\givenatend}{\end{trivlist}}{\egroup}{}
}
\else
\newenvironment{instructorNotes}[1][false]%
{%
  \ifthenelse{\boolean{#1}}{\begin{trivlist}\item[\hskip \labelsep\bfseries {\Large Instructor Notes: \\} \hspace{\textwidth} ]}
{\begin{trivlist}\item[\hskip \labelsep\bfseries {\Large Instructor Notes: \\} \hspace{\textwidth} ]}
{}
}
{\end{trivlist}}
\fi


%% Suggested Timing
\newcommand{\timing}[1]{{\bf Suggested Timing: \hspace{2ex}} #1}

\author{Vic Ferdinand}
\title{A Moving Experience (Or Walking and Chewing Gum at the Same Time)}
\outcome{Understand related rates.}

\begin{document}
\begin{abstract}
\end{abstract}
\maketitle

\begin{instructorIntro}
(Note: For time purposes, this has so far served as the second of a $2$-part extra credit exercise assigned after the midterm is graded and due toward the end of the semester- Half of the credit for the activity work and the other half for a quiz – all done outside of class).

The emphasis here is on related rates.  The major idea the students should take out of this is that relationships learned in middle and high school can now be used to find information about rates when those several quantities are all changing at the same time (and that the relationship between their values can now give us the relationship between their rates, via implicit differentiation).  The major difficulty in the algebra is that we are no longer working specifically with finding slopes to implicitly-defined curves (where the rate is with respect to $x$), but rather quantities that are each implicitly ``functions'' of time (Thus, all things that are changing find their rates by deriving with respect to $t$- Thus, chain rule will always be used).  What stays the same is that we wait until the derivatives have been taken before plugging given (or fought for) values into the related rates equation.  Toward doing that, students need to identify which quantities are always constant and which change with time.  

I have found success in emphasizing that we end up with two equations:  a ``static'' equation (the precalculus relationship between the quantities) and a ``rate'' equation (derived from the static equation that involves both the quantities and their rates).  Try to dissuade students from finding formulas in terms of time for those quantities changing at constant rates).

A secondary result should be to recognize that, although some quantities might change at a constant rate, that does not imply that related quantities will as well.

I'm not sure whether it would be a good idea to connect with the geometry here (parametrics would take us down a lengthy road).

\end{instructorIntro}



\begin{exercise} 
We'll now extend our idea of finding slopes to dangerous curves to some ``practical'' problems (yeah, right).  Here, we're investigating situations where our variables are changing at the same time.  The ``classic'' problem is to consider a $12$-foot ladder against a wall that Horrible Harold pulls down from the bottom. The ladder is thus falling against the wall (and thus the top of the ladder's distance from the ground is changing).  But what else is changing at the same time?  Think of the distance from the base of the ladder to the wall.  If the top of the ladder is falling at a constant rate of $2$ feet per second, how fast is that second (base) distance changing? 

Let's investigate this exciting situation from what we know from middle school and tie that in with what we know about derivatives.  Draw a picture of the situation.  Is there a relationship between the length of the ladder and the distances from the base of the wall to the top and bottom of the ladder?  Write that down (Does it look familiar from ``Dangerous Curves''?).  

You might note that the two changing distances are now both considered functions of time as opposed to $y$ being considered a ``function'' of $x$ (because we're considering rates as ``feet per second'').  Differentiate both sides of the equation with respect to time.  You'll now have expressions involving $x$, $y$, $\frac{dx}{dt}$, and $\frac{dy}{dt}$.  Which quantities do we know already?  Which do we still need to find? 

Now work to find how fast the base distance is changing when the top of the ladder is $5$ feet above the ground.  Do the same for when the top of the ladder is $3$ feet above the ground.  If the top of the ladder is falling at a constant rate, does that mean the base distance of the ladder is changing at a constant rate?
\end{exercise} 
\begin{exercise} 
A $1$-foot ladder is leaning against a wall and Horrible Harold does the dirty deed again (wow, really an original problem!).  The angle between the bottom of the ladder and the floor is decreasing at a rate of $\frac{\pi}{10}$  radians per second.  How fast is the top of the ladder falling when that angle is  $\frac{\pi}{6}$ radians?  How is this different from our practice problem?   (Like above, draw a picture, write an equation relating the things that are and are not changing, differentiate both sides with respect to $t$, and find what you're looking for)
\end{exercise}
\begin{exercise}
Water is poured into a cheap conical cup at the rate of $8$ cubic centimeters per second.  The cone points directly down and has a height of $30$ cm and a base radius of $10$ cm.  How fast is the water level rising when the water is $4$ cm deep?  Note:  The formula for the volume of a cone is  $\frac{1}{3}\pi r^2 h$. (Like above, draw a picture, write an equation relating the things that are and are not changing, differentiate both sides with respect to $t$, and find what you’re looking for).
\end{exercise}

 
\begin{exercise} 
An airliner passes over an airport at noon traveling $500$ mph due west.  At $1$PM, another airliner passes over the same airport at the same elevation traveling due north at $550$ mph.  Assuming both airliners maintain their (equal) elevations, how fast is the distance between them changing at 2:30PM?
\end{exercise}
\begin{exercise} 
Runners stand on first and second base in a baseball game at Jacobs Field.  At the moment the ball is hit, the runner at 1st base runs to 2nd base at $18$ ft/sec.  Simultaneously, the runner at 2nd base runs to 3rd base at $20$ ft/sec. How fast is the distance between the runners changing $2$ seconds after the ball is hit?  Note:  The distance between consecutive bases is $90$ ft. and the bases lie at the corners of a square.
\end{exercise}
\begin{exercise} 
The hands of a clock in the tower of the Houses of Parliament in jolly old London are approximately $3$ meters and $2.5$ meters in length.  How fast is the distance between the tips of the hands changing at 9:00?  At 9:07?
\end{exercise}



\end{document}