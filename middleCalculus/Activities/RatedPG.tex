\documentclass{ximera}
\usepackage{gensymb}
\usepackage{tabularx}
\usepackage{mdframed}
\usepackage{pdfpages}
%\usepackage{chngcntr}

\let\problem\relax
\let\endproblem\relax

\newcommand{\property}[2]{#1#2}




\newtheoremstyle{SlantTheorem}{\topsep}{\fill}%%% space between body and thm
 {\slshape}                      %%% Thm body font
 {}                              %%% Indent amount (empty = no indent)
 {\bfseries\sffamily}            %%% Thm head font
 {}                              %%% Punctuation after thm head
 {3ex}                           %%% Space after thm head
 {\thmname{#1}\thmnumber{ #2}\thmnote{ \bfseries(#3)}} %%% Thm head spec
\theoremstyle{SlantTheorem}
\newtheorem{problem}{Problem}[]

%\counterwithin*{problem}{section}



%%%%%%%%%%%%%%%%%%%%%%%%%%%%Jenny's code%%%%%%%%%%%%%%%%%%%%

%%% Solution environment
%\newenvironment{solution}{
%\ifhandout\setbox0\vbox\bgroup\else
%\begin{trivlist}\item[\hskip \labelsep\small\itshape\bfseries Solution\hspace{2ex}]
%\par\noindent\upshape\small
%\fi}
%{\ifhandout\egroup\else
%\end{trivlist}
%\fi}
%
%
%%% instructorIntro environment
%\ifhandout
%\newenvironment{instructorIntro}[1][false]%
%{%
%\def\givenatend{\boolean{#1}}\ifthenelse{\boolean{#1}}{\begin{trivlist}\item}{\setbox0\vbox\bgroup}{}
%}
%{%
%\ifthenelse{\givenatend}{\end{trivlist}}{\egroup}{}
%}
%\else
%\newenvironment{instructorIntro}[1][false]%
%{%
%  \ifthenelse{\boolean{#1}}{\begin{trivlist}\item[\hskip \labelsep\bfseries Instructor Notes:\hspace{2ex}]}
%{\begin{trivlist}\item[\hskip \labelsep\bfseries Instructor Notes:\hspace{2ex}]}
%{}
%}
%% %% line at the bottom} 
%{\end{trivlist}\par\addvspace{.5ex}\nobreak\noindent\hung} 
%\fi
%
%


\let\instructorNotes\relax
\let\endinstructorNotes\relax
%%% instructorNotes environment
\ifhandout
\newenvironment{instructorNotes}[1][false]%
{%
\def\givenatend{\boolean{#1}}\ifthenelse{\boolean{#1}}{\begin{trivlist}\item}{\setbox0\vbox\bgroup}{}
}
{%
\ifthenelse{\givenatend}{\end{trivlist}}{\egroup}{}
}
\else
\newenvironment{instructorNotes}[1][false]%
{%
  \ifthenelse{\boolean{#1}}{\begin{trivlist}\item[\hskip \labelsep\bfseries {\Large Instructor Notes: \\} \hspace{\textwidth} ]}
{\begin{trivlist}\item[\hskip \labelsep\bfseries {\Large Instructor Notes: \\} \hspace{\textwidth} ]}
{}
}
{\end{trivlist}}
\fi


%% Suggested Timing
\newcommand{\timing}[1]{{\bf Suggested Timing: \hspace{2ex}} #1}




\hypersetup{
    colorlinks=true,       % false: boxed links; true: colored links
    linkcolor=blue,          % color of internal links (change box color with linkbordercolor)
    citecolor=green,        % color of links to bibliography
    filecolor=magenta,      % color of file links
    urlcolor=cyan           % color of external links
}

\author{Vic Ferdinand}
\title{Rated PG (for Language)}
\outcome{Use correct language for derivatives at a point.}
\outcome{Use correct notation for derivatives at a point.}


\begin{document}
\begin{abstract}
\end{abstract}
\maketitle


\begin{instructorIntro}
Here, we introduce the student to the notation and meaning of a new function called ``$f$ prime''.  It's critical to note that they pay attention to what quantity $f$ represents so they can carefully note what its derivative represents (the rate of change of the original quantity as $x$ changes ($x$ is usually time)).  Do this when introducing the activity and then let them go.  
\end{instructorIntro}

For each of the first three problems, write an equivalent statement using $f$  and $f'$ notation.

\begin{question} 
If $f$ is the population of Hicktown,
\begin{enumerate}
\item The population of Hicktown was $5.4$ million in $2018$.
\item The population of Hicktown was increasing at a rate of $110000$ people per year in $2018$.
\end{enumerate}
\end{question}
\begin{question} 
Joan decides to take a run outside.  Let $p(t)$ be the position Joan is from home (east or west) after $t$ minutes.  Assume Joan is always traveling in the same eastward direction.
\begin{enumerate}
\item 	She runs along a flat country road at $100$ meters per minute for $10$ minutes.
\item	She comes to a hill.  Over the next $3$ minutes, her velocity drops at a constant rate of $15$ meters per minute each minute.
\end{enumerate}


\begin{instructorNotes}
On this question and the next one, we see the rate of change of velocity with respect to time (with units (meters per minute) per minute).  Sometimes, I've suggested some numerical examples, such as ``At $t=11$, what will her velocity be?'').  They also should note the notational relationship between the derivatives of $p(t)$ and $v(t)$ and $v'(t)$.
\end{instructorNotes}

\end{question}

\begin{question}	Joan decides to take a run outside.  Let $v(t)$ be Joan's velocity at time $t$ in meters per minute. Assume Joan is always traveling in the same eastward direction.
\begin{enumerate}
\item She runs along a flat country road at $100$ meters per minute for $10$ minutes.
\item She comes to a hill.  Over the next $3$ minutes, her velocity drops at a constant rate of $15$ meters per minute each minute.
\end{enumerate}
\end{question} 

\begin{question} Let $H$ stand for the height of a balloon (in feet above the ground) as a function of time (in minutes).  Translate the following into English:
\begin{enumerate}
    \item $H(0) = 1000$
    \item $H'(0) = 0$
    \item $H'(5) = -52$
    \item $H(5) - H(0) - -201$
\end{enumerate}

\begin{instructorNotes}
Here, many students think that $H'(t) = 0$ means the balloon is standing still (once again, paying attention that $H$ represents the vertical and not the horizontal component of location/motion).  
\end{instructorNotes}


\end{question} 


\begin{question} 
Let $f$ be a function such that $f(0) = 5$ and $3 \leq f'(t) \leq 12$ for all $t$.  What can be said about the value of $f(4)$?  About $f(-4)$?

\begin{instructorNotes}
This problem is usually troublesome for all because they are wrestling with just the new notation (no context), its meaning, and, in the second part, working backwards.  Some try to assume $f$ is linear, which is OK, but they need to be reminded (as the course's existence depends on the fact that not all functions are linear!).

I often suggest to students that they start at $x = 0$ and argue what possible values $f(x)$ can take on at $x = 1$, $x = 2$, etc. rather than trying to do the problem all at once.
\end{instructorNotes}

\end{question}


\end{document}