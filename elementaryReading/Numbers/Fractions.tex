\documentclass{ximera}

\usepackage{gensymb}
\usepackage{tabularx}
\usepackage{mdframed}
\usepackage{pdfpages}
%\usepackage{chngcntr}

\let\problem\relax
\let\endproblem\relax

\newcommand{\property}[2]{#1#2}




\newtheoremstyle{SlantTheorem}{\topsep}{\fill}%%% space between body and thm
 {\slshape}                      %%% Thm body font
 {}                              %%% Indent amount (empty = no indent)
 {\bfseries\sffamily}            %%% Thm head font
 {}                              %%% Punctuation after thm head
 {3ex}                           %%% Space after thm head
 {\thmname{#1}\thmnumber{ #2}\thmnote{ \bfseries(#3)}} %%% Thm head spec
\theoremstyle{SlantTheorem}
\newtheorem{problem}{Problem}[]

%\counterwithin*{problem}{section}



%%%%%%%%%%%%%%%%%%%%%%%%%%%%Jenny's code%%%%%%%%%%%%%%%%%%%%

%%% Solution environment
%\newenvironment{solution}{
%\ifhandout\setbox0\vbox\bgroup\else
%\begin{trivlist}\item[\hskip \labelsep\small\itshape\bfseries Solution\hspace{2ex}]
%\par\noindent\upshape\small
%\fi}
%{\ifhandout\egroup\else
%\end{trivlist}
%\fi}
%
%
%%% instructorIntro environment
%\ifhandout
%\newenvironment{instructorIntro}[1][false]%
%{%
%\def\givenatend{\boolean{#1}}\ifthenelse{\boolean{#1}}{\begin{trivlist}\item}{\setbox0\vbox\bgroup}{}
%}
%{%
%\ifthenelse{\givenatend}{\end{trivlist}}{\egroup}{}
%}
%\else
%\newenvironment{instructorIntro}[1][false]%
%{%
%  \ifthenelse{\boolean{#1}}{\begin{trivlist}\item[\hskip \labelsep\bfseries Instructor Notes:\hspace{2ex}]}
%{\begin{trivlist}\item[\hskip \labelsep\bfseries Instructor Notes:\hspace{2ex}]}
%{}
%}
%% %% line at the bottom} 
%{\end{trivlist}\par\addvspace{.5ex}\nobreak\noindent\hung} 
%\fi
%
%


\let\instructorNotes\relax
\let\endinstructorNotes\relax
%%% instructorNotes environment
\ifhandout
\newenvironment{instructorNotes}[1][false]%
{%
\def\givenatend{\boolean{#1}}\ifthenelse{\boolean{#1}}{\begin{trivlist}\item}{\setbox0\vbox\bgroup}{}
}
{%
\ifthenelse{\givenatend}{\end{trivlist}}{\egroup}{}
}
\else
\newenvironment{instructorNotes}[1][false]%
{%
  \ifthenelse{\boolean{#1}}{\begin{trivlist}\item[\hskip \labelsep\bfseries {\Large Instructor Notes: \\} \hspace{\textwidth} ]}
{\begin{trivlist}\item[\hskip \labelsep\bfseries {\Large Instructor Notes: \\} \hspace{\textwidth} ]}
{}
}
{\end{trivlist}}
\fi


%% Suggested Timing
\newcommand{\timing}[1]{{\bf Suggested Timing: \hspace{2ex}} #1}




\hypersetup{
    colorlinks=true,       % false: boxed links; true: colored links
    linkcolor=blue,          % color of internal links (change box color with linkbordercolor)
    citecolor=green,        % color of links to bibliography
    filecolor=magenta,      % color of file links
    urlcolor=cyan           % color of external links
}


\title{Fractions}
\author{Jenny Sheldon}

\begin{document}

\begin{abstract}
We define and work with fractions.
\end{abstract}
\maketitle

\section{Activities for this section:} 
\link[Fraction intro]{https://ximera.osu.edu/m4t/elementaryActivities/SemesterOnePacket/elementaryActivities/Numbers/FractionIntro}, 
\link[The Bake Shop]{https://ximera.osu.edu/m4t/elementaryActivities/SemesterOnePacket/elementaryActivities/Numbers/BakeShop}, 
Pies, 
\link[Improper fractions]{https://ximera.osu.edu/m4t/elementaryActivities/SemesterOnePacket/elementaryActivities/Numbers/ImproperFractions}

\section{The definition of a fraction}

The numbers we discussed in the previous section, starting with $1$, then $2$, and so on, are called the \dfn{counting numbers}, because we typically use them to count things. We will also call these numbers \dfn{whole numbers}, but the whole numbers also include zero. Most of the examples that you imagined in the previous section were probably about counting whole things. What if we want to take a look at parts of things instead? This is where the idea of a fraction can naturally arise. 

\begin{definition}
Begin with some whole, and then cut this whole into $B$ equal pieces. Then a \dfn{unit fraction} $\frac{1}{B}$ of this whole is represented by one of these equal pieces.
\end{definition}
Any time we see a definition with some letters in it, it's a good idea to plug in some numbers to make sure this makes sense to us.

\begin{example}
Let's use a rectangle as our whole, and represent $\frac{1}{4}$ of that rectangle. Don't forget that we need to use our definition. The definition says that we have to start with some whole, and in this case that will be a rectangle. Let's draw it.

\begin{image}
\begin{tikzpicture}
	\draw[thick] (0,0)--(4,0)--(4,1)--(0,1)--(0,0);
\end{tikzpicture}
\end{image}

Next, our definition says that we need to cut this rectangle into $B$ equal parts, and since we are trying to represent $\frac{1}{4}$ in this case, our $B$ is $4$.  So, let's cut our rectangle into $4$ equal pieces.

\begin{image}
\begin{tikzpicture}
	\draw[thick] (0,0)--(4,0)--(4,1)--(0,1)--(0,0);
	\foreach \x in {1,2,3}
	\draw[thick] (\x, 0)--(\x,1);
\end{tikzpicture}
\end{image}

Finally, our definition says that $\frac{1}{4}$ of this rectangle is what we get by taking one of these four equal pieces. So let's shade one of them to really emphasize what we are looking for.

\begin{image}
\begin{tikzpicture}
	\draw[fill=yellow] (0,0)--(1,0)--(1,1)--(0,1)--(0,0);
	\draw[thick] (0,0)--(4,0)--(4,1)--(0,1)--(0,0);
	\foreach \x in {1,2,3}
	\draw[thick] (\x, 0)--(\x,1);
\end{tikzpicture}
\end{image}

This shaded piece represents $\frac{1}{4}$ of our rectangle.

\end{example}

\begin{question}
What fraction of the rectangle below is shaded? Notice that the whole rectangle is cut into $8$ equal pieces and one of them is shaded.
\begin{image}
\begin{tikzpicture}
	\draw[fill=yellow] (0,0)--(1,0)--(1,1)--(0,1)--(0,0);
	\draw[thick] (0,0)--(8,0)--(8,1)--(0,1)--(0,0);
	\foreach \x in {1,2,3, 4, 5, 6, 7}
	\draw[thick] (\x, 0)--(\x,1);
\end{tikzpicture}
\end{image}

\begin{prompt}
$\answer[given]{\frac{1}{8}}$
\end{prompt}
\end{question}

Unit fractions are not the only type of fractions we encounter, however, so we need a more general definition for fractions.

\begin{definition}
	Begin with some whole. The meaning of the \dfn{fraction} $\frac{A}{B}$ is $A$ pieces, each of size $\frac{1}{B}$ of that whole. The bottom number $B$ is called the \dfn{denominator}, and the top number $A$ is called the \dfn{numerator}.
\end{definition}
In other words, to get the fraction $\frac{A}{B}$, we need $A$ copies of the unit fraction $\frac{1}{B}$, where the unit fraction tells us the size of the pieces. This definition for fractions is the one used in most state standards, so we will practice with this quite a bit. These definitions of fractions are consistent with most \link[state mathematics standards]{https://education.ohio.gov/getattachment/Topics/Learning-in-Ohio/Mathematics/Ohio-s-Learning-Standards-in-Mathematics/MATH-Standards-Grade-3.pdf.aspx?lang=en-US} for grade 3.

\begin{example}
Let's use a rectangle as our whole, and represent $\frac{3}{5}$ of that rectangle. Don't forget that we need to use our definition, so we will start with our whole as a rectangle. Let's draw it.

\begin{image}
\begin{tikzpicture}
	\draw[thick] (0,0)--(5,0)--(5,1)--(0,1)--(0,0);
\end{tikzpicture}
\end{image}

Next, we need to know what the unit fraction $\frac{1}{B}$ looks like, so we need to cut our whole into $B$ equal parts. In this case our denominator is $5$, so we will cut our whole into $5$ equal parts.

\begin{image}
\begin{tikzpicture}
	\draw[thick] (0,0)--(5,0)--(5,1)--(0,1)--(0,0);
	\foreach \x in {1,2,3,4}
	\draw[thick] (\x, 0)--(\x,1);
\end{tikzpicture}
\end{image}

Finally, our numerator is $3$, so our definition says that $\frac{3}{5}$ of this rectangle is what we get by taking three copies of one of these equal pieces. So let's shade the three we want to represent our fraction.

\begin{image}
\begin{tikzpicture}
	\draw[fill=yellow] (0,0)--(3,0)--(3,1)--(0,1)--(0,0);
	\draw[thick] (0,0)--(4,0)--(4,1)--(0,1)--(0,0);
	\foreach \x in {1,2,3,4}
	\draw[thick] (\x, 0)--(\x,1);
\end{tikzpicture}
\end{image}

This shaded piece represents $\frac{3}{5}$ of our rectangle.

\end{example}

While we have so far looked at fractions whose numerator is smaller than the denominator, the definition of fractions works for any two numbers we can think to plug in for $A$ and $B$. That's part of the reason that the definition is written the way that it is, and we'll explore this a bit more in the next section.

\begin{question}
What fraction of the rectangle below is shaded? Notice that the whole rectangle is cut into $7$ equal pieces and two of them are shaded.
\begin{image}
\begin{tikzpicture}
	\draw[fill=yellow] (0,0)--(2,0)--(2,1)--(0,1)--(0,0);
	\draw[thick] (0,0)--(7,0)--(7,1)--(0,1)--(0,0);
	\foreach \x in {1,2,3, 4, 5, 6}
	\draw[thick] (\x, 0)--(\x,1);
\end{tikzpicture}
\end{image}

\begin{prompt}
$\answer[given]{\frac{2}{7}}$
\end{prompt}
\end{question}

It's very important to recognize the role of the whole when we are talking about fractions. Each of our definitions says that we need to begin with some whole. This whole should be a physical object, like a rectangle or a pan of brownies or all of the children in the class. Once we know what our whole is, then the denominator tells us how many equal pieces to partition our whole into. The numerator tells us how many pieces of the whole to take. In other words, there is a difference between the whole (a physical object) and the denominator (a number). When we talk about the numerator (a number), we can also use the word \dfn{part} to refer to the portion of the whole (a physical object) that the numerator counts. The whole and the part are physical things, while the denominator and the numerator are numbers that represent how those physical objects have been cut.

While we have discussed the numerator and the denominator separately in order to interpret a fraction, it's very important to remember that a fraction like $\frac{3}{5}$ of some whole represents a \emph{single number}, not two different numbers. This is especially important for understanding how to locate fractions on a number line. First, let's define what we mean by a number line.
\begin{definition}
To draw a \dfn{number line}, we start by drawing a line. Next, we need choose a distance that represents $1$, or $1$ unit. Then, we need to mark at least two locations on the number line using that unit. Once we do so, we can find other points on the number line by using the unit as a distance from the first number we marked.
\begin{image}
	\begin{tikzpicture}
	\draw[very thick, <->] (-0.3,0)--(6.3,0);
	\foreach \x in {0, 1, 2, 3, 4, 5, 6} \draw[thick] (\x, 0.3)--(\x, -0.3) node[below] {$\x$};
	\end{tikzpicture}
\end{image}
\end{definition}
It's very common to mark the location of zero and 1 on your number line as the first two points, since then you can use the distance between $0$ and $1$ as the unit length. Every other number is determined by this length: for instance $2$ is located $2$ units from zero.

Next, let's take a look at an example of locating a fraction on the number line.

\begin{example}
	Let's plot the fraction $\frac{3}{5}$ on the number line. First, notice that we just said ``the fraction $\frac{3}{5}$'' instead of ``$\frac{3}{5}$ of a rectangle''. What is the whole physical object for this fraction? In the case of plotting fractions on a number line, the whole is the length of one single unit, or the distance between $0$ and $1$ on the number line. 

\begin{image}
	\begin{tikzpicture}
	\draw[very thick, <->] (-0.3,0)-- (6.3,0);
	\foreach \x in {0, 2, 4, 6} \draw[thick] (\x, -0.3)--(\x, 0.3);
	\draw[<->] (0, 0.4)--(0,0.6)--(2, 0.6)--(2, 0.4);
	\node[above] at (1, 0.6) {1 unit};
	\node[below] at (0, -0.3) {$0$};
	\node[below] at (2, -0.3) {$1$};
	\node[below] at (4, -0.3) {$2$};
	\node[below] at (6, -0.3) {$3$};
	\end{tikzpicture}
\end{image}

Because we are drawing $\frac{3}{5}$ of this unit, we need to cut the unit into $\answer[given]{5}$ equal pieces.

\begin{image}
	\begin{tikzpicture}
	\draw[very thick, <->] (-0.3,0)-- (6.3,0);
	\foreach \x in {0, 2, 4, 6} \draw[thick] (\x, -0.3)--(\x, 0.3);
	\draw[<->] (0, 0.4)--(0,0.6)--(0.4, 0.6)--(0.4, 0.4);
	\node[above] at (0.2, 0.6) {$\frac{1}{5}$ of a unit};
	\node[below] at (0, -0.3) {$0$};
	\node[below] at (2, -0.3) {$1$};
	\node[below] at (4, -0.3) {$2$};
	\node[below] at (6, -0.3) {$3$};
	\foreach \x in {0.4, 0.8, 1.2, 1.6} \draw (\x, -0.2)--(\x, 0.2);
	\end{tikzpicture}
\end{image}

Each of these spaces represents $\frac{1}{5}$ of the unit. Since our numerator is $3$, we need $3$ copies of this $\frac{1}{5}$ of a unit. We could highlight $3$ pieces, but the convention on number lines is to mark our point with a dot.

\begin{image}
	\begin{tikzpicture}
	\draw[very thick, <->] (-0.3,0)-- (6.3,0);
	\foreach \x in {0, 2, 4, 6} \draw[thick] (\x, -0.3)--(\x, 0.3);
	\draw[<->] (0, 0.4)--(0,0.6)--(1.2, 0.6)--(1.2, 0.4);
	\node[above] at (0.6, 0.6) {$\frac{3}{5}$ of a unit};
	\node[below] at (0, -0.3) {$0$};
	\node[below] at (2, -0.3) {$1$};
	\node[below] at (4, -0.3) {$2$};
	\node[below] at (6, -0.3) {$3$};
	\foreach \x in {0.4, 0.8, 1.2, 1.6} \draw (\x, -0.2)--(\x, 0.2);
	\draw[fill=black] (1.2, 0) circle (2pt);
	\node[below] at (1.2, -0.2) {$\frac{3}{5}$};
	\end{tikzpicture}
\end{image}

\end{example}

Notice that in our previous example, when we locate numbers on the line, it's the distance from zero that tells us the number, not the number or tick marks. While we spend time drawing lots of tick marks, we are really paying attention to the spaces in between them. 

\begin{question}
Compare and contrast the example where we drew $\frac{3}{5}$ of a rectangle with the example where we plotted $\frac{3}{5}$ on a number line. What is the same? What is different?

\begin{freeResponse}
Write a few thoughts here.
\end{freeResponse}
\end{question}





\section{Improper fractions and mixed fractions}

Our definition of fractions is written in a way that helps us make sense of many kinds of fractions. Let's tackle another example.
	
\begin{example}
	Let's investigate how we could use our definition of fractions to draw a picture of $2 \frac{1}{4}$ pans of brownies.
	
	We will use a rectangle to represent one pan of brownies. When we write the whole number $2$ in front of a fraction, that whole number tells us that the part we are interested in consists of two wholes as well as a part of another whole. In other words, we will need $3$ copies of the whole in order to get started with drawing this picture.
	
	First, we need to identify our whole, which in this case is \wordChoice{\choice{brownie} \choice[correct]{a pan of brownies} \choice{4} \choice{2}}. Let's draw three copies of our whole.
	
	\begin{image}
		\begin{tikzpicture}
			\draw[thick] (0,0)--(4,0)--(4,1)--(0,1)--(0,0);
			\draw[thick] (5,0)--(9,0)--(9,1)--(5,1)--(5,0);
			\draw[thick] (10, 0)--(14,0)--(14,1)--(10,1)--(10,0);
			\node[below] at (2,0) {1 pan of brownies};
		\end{tikzpicture}
	\end{image}
	
	We know from the $2$ in front of our number that we need to shade $\answer[given]{2}$ full wholes, so let's start by doing that.
	
	\begin{image}
		\begin{tikzpicture}
			\draw[thick, fill=orange] (0,0)--(4,0)--(4,1)--(0,1)--(0,0);
			\draw[thick, fill=orange] (5,0)--(9,0)--(9,1)--(5,1)--(5,0);
			\draw[thick] (10, 0)--(14,0)--(14,1)--(10,1)--(10,0);
			\node[below] at (2,0) {1 pan of brownies};
		\end{tikzpicture}
	\end{image}	
	
	We also need $\frac{1}{4}$ of another whole, so we need to take the last pan of brownies and cut it into $\answer[given]{4}$ equal pieces because our denominator is $4$, and then we'll need to shade $\answer[given]{1}$ of those pieces because of numerator is $1$.

	\begin{image}
		\begin{tikzpicture}
			\draw[thick, fill=orange] (0,0)--(4,0)--(4,1)--(0,1)--(0,0);
			\draw[thick, fill=orange] (5,0)--(9,0)--(9,1)--(5,1)--(5,0);
			\draw[fill=orange] (10,0)--(11,0)--(11,1)--(10,1)--(10,0);
			\node[below] at (2,0) {1 pan of brownies};
			\draw[thick] (10, 0)--(14,0)--(14,1)--(10,1)--(10,0);
			\foreach \x in {11, 12, 13} \draw[thick] (\x,0)--(\x,1);
		\end{tikzpicture}
	\end{image}
	
	In this final picture, we've shaded $2 \frac{1}{4}$ pans of brownies.		
	
\end{example}

The number $2 \frac{1}{4}$ in the previous example is called a mixed number.
\begin{definition}
A \dfn{mixed number} is a number in the form $N \frac{A}{B}$, where the $N$ is a whole number and $\frac{A}{B}$ is a fraction of that same whole. We can also interpret this mixed number as $N + \frac{A}{B}$ of some whole (though we will wait until we talk about addition to discuss why it makes sense to put a plus sign there).
\end{definition}


We can also use our meaning of fractions to draw fractions whose numerator is larger than the denominator.

\begin{example}
Let's draw a picture of $\frac{8}{5}$ pounds of rice. Our whole in this case is \wordChoice{\choice{rice} \choice[correct]{one pound of rice} \choice{5}} so let's draw a rectangle and label it with our description of our whole.

\begin{image} \begin{tikzpicture}
\draw[thick] (0,0)--(5,0)--(5,1)--(0,1)--(0,0);
\node[below] at (2.5,0) {1 pound of rice};
\end{tikzpicture}\end{image}

Next, our denominator tells us how many equal-sized pieces to cut our whole into. In this case, the denominator is $\answer[given]{5}$, so let's cut our whole in the picture into $5$ equal pieces.

\begin{image} \begin{tikzpicture}
\draw[thick] (0,0)--(5,0)--(5,1)--(0,1)--(0,0);
\node[below] at (2.5,0) {1 pound of rice};
\foreach \x in {1, 2, 3, 4} \draw[thick] (\x, 0)--(\x,1);
\end{tikzpicture}\end{image}

Finally, our numerator tells us how many parts to shade. In this case the numerator is $\answer[given]{8}$, which is currently a problem since we don't have that many pieces. However, remember that another way to write our definition is to say that the numerator tells us how many copies of one piece we need. So, in this case, we can interpret the fraction to mean that we need $8$ copies of $\frac{1}{5}$ of our whole. So, we are going to use all five of the copies we already drew, and make 3 more copies of the piece we are using. We want to shade them all.

\begin{image} \begin{tikzpicture}
\draw[fill=pink] (0,0)--(8,0)--(8,1)--(0,1)--(0,0);
\draw[very thick] (0,0)--(5,0)--(5,1)--(0,1)--(0,0);
\draw (5,0)--(8,0)--(8,1)--(5,1);
\node[below] at (2.5,0) {1 pound of rice};
\foreach \x in {1, 2, 3, 4, 6, 5, 7} \draw (\x, 0)--(\x,1);
\end{tikzpicture}\end{image}

Notice that the original five pieces making up one full pound of rice are more darkly outlined than the other pieces. That dark outline shows us that our whole didn't change, and it is still made out of five equal pieces, even though we needed a few extra in order to make up the entire $\frac{8}{5}$ pound of rice. Since the numerator is bigger than the denominator, the part we are shading is bigger than the whole.

Some people prefer to draw the extra parts separated from the original whole so that we can see the whole a little more clearly in the picture. And some people prefer to draw more copies of the whole so that they have enough copies to shade. For instance, here is a different picture of $\frac{8}{5}$ of a pound of rice, where we have drawn two copies of our whole and then shaded $\frac{8}{5}$ of the whole.

\begin{image} \begin{tikzpicture}
\draw[fill=pink] (0,0)--(5,0)--(5,1)--(0,1)--(0,0);
\draw[fill=pink] (6,0)--(9,0)--(9,1)--(6,1)--(6,0);
\draw[thick] (0,0)--(5,0)--(5,1)--(0,1)--(0,0);
\draw[thick] (6,0)--(11,0)--(11,1)--(6,1)--(6,0);
\node[below] at (2.5,0) {1 pound of rice};
\foreach \x in {1, 2, 3, 4, 7, 8, 9, 10} \draw[thick] (\x, 0)--(\x,1);
\end{tikzpicture}\end{image}

However, notice in this second picture that it can be tempting to write that the shaded region is $\frac{8}{10}$ of the whole. However, since our whole is one pound of rice, and the one pound of rice is cut into $5$ equal pieces in the picture, we have to use $5$ as our denominator. If we changed the whole to be $2$ pounds of rice, then the denominator would be $10$ because the $2$ pounds are cut into $10$ equal pieces. What we choose and label as the whole can completely change our answer to a problem about fractions!

\end{example}

Fractions like $\frac{8}{5}$ are examples of improper fractions.

\begin{definition}
An \dfn{improper fraction} is a fraction whose numerator is larger than its denominator.
\end{definition}

As a note, some textbooks don't use this terminology and instead call them ``fractions greater than one''. Please always be sure to check your future textbooks for the terminology they use! 

We can also use our meaning of fractions to convert back and forth between improper fractions and mixed numbers.

\begin{example}
Let's use the picture we drew at the end of the previous example to write $\frac{8}{5}$ as a mixed number. Here is the picture again.

\begin{image} \begin{tikzpicture}
\draw[fill=pink] (0,0)--(5,0)--(5,1)--(0,1)--(0,0);
\draw[fill=pink] (6,0)--(9,0)--(9,1)--(6,1)--(6,0);
\draw[thick] (0,0)--(5,0)--(5,1)--(0,1)--(0,0);
\draw[thick] (6,0)--(11,0)--(11,1)--(6,1)--(6,0);
\node[below] at (2.5,0) {1 pound of rice};
\foreach \x in {1, 2, 3, 4, 7, 8, 9, 10} \draw[thick] (\x, 0)--(\x,1);
\end{tikzpicture}\end{image}

In our picture, we drew 2 pounds of rice. One of them was entirely filled in, and the second one was only partially filled in. The whole number part of a mixed number indicates how many full wholes we need, so in this case the whole number part would be $\answer[given]{1}$. The fraction part of a mixed number tells us the fraction of the last whole that we have. In this case, the second whole is $\answer[given]{\frac{3}{5}}$ shaded. Putting this together, we can see that 
\[
\frac{8}{5} \textrm{ of a pound of rice } = 1 \frac{3}{5} \textrm{ of a pound of rice.}
\]

\end{example}

\begin{question}
What fraction of the following rectangle is shaded? The picture has 3 rectangles, each cut into 2 equal pieces. The first two rectangles are completely shaded, and the third has one piece shaded.

\begin{image}\begin{tikzpicture}
\draw[thick, fill=green] (0,0)--(2,0)--(2,1)--(0,1)--(0,0);
\draw[thick, fill=green] (3,0)--(5,0)--(5,1)--(3,1)--(3,0);
\draw[thick, fill=green] (6,0)--(7,0)--(7,1)--(6,1)--(6,0);
\draw[thick] (7,0)--(8,0)--(8,1)--(7,1);
\foreach \x in {1, 4} \draw[thick] (\x,0)--(\x,1);
\end{tikzpicture}\end{image}

Write your answer as a mixed number: $\answer{2+\frac{1}{2}}$

Write your answer as an improper fraction: $\answer{\frac{5}{2}}$
\end{question}




\end{document}






