\documentclass{ximera}


\usepackage{gensymb}
\usepackage{tabularx}
\usepackage{mdframed}
\usepackage{pdfpages}
%\usepackage{chngcntr}

\let\problem\relax
\let\endproblem\relax

\newcommand{\property}[2]{#1#2}




\newtheoremstyle{SlantTheorem}{\topsep}{\fill}%%% space between body and thm
 {\slshape}                      %%% Thm body font
 {}                              %%% Indent amount (empty = no indent)
 {\bfseries\sffamily}            %%% Thm head font
 {}                              %%% Punctuation after thm head
 {3ex}                           %%% Space after thm head
 {\thmname{#1}\thmnumber{ #2}\thmnote{ \bfseries(#3)}} %%% Thm head spec
\theoremstyle{SlantTheorem}
\newtheorem{problem}{Problem}[]

%\counterwithin*{problem}{section}



%%%%%%%%%%%%%%%%%%%%%%%%%%%%Jenny's code%%%%%%%%%%%%%%%%%%%%

%%% Solution environment
%\newenvironment{solution}{
%\ifhandout\setbox0\vbox\bgroup\else
%\begin{trivlist}\item[\hskip \labelsep\small\itshape\bfseries Solution\hspace{2ex}]
%\par\noindent\upshape\small
%\fi}
%{\ifhandout\egroup\else
%\end{trivlist}
%\fi}
%
%
%%% instructorIntro environment
%\ifhandout
%\newenvironment{instructorIntro}[1][false]%
%{%
%\def\givenatend{\boolean{#1}}\ifthenelse{\boolean{#1}}{\begin{trivlist}\item}{\setbox0\vbox\bgroup}{}
%}
%{%
%\ifthenelse{\givenatend}{\end{trivlist}}{\egroup}{}
%}
%\else
%\newenvironment{instructorIntro}[1][false]%
%{%
%  \ifthenelse{\boolean{#1}}{\begin{trivlist}\item[\hskip \labelsep\bfseries Instructor Notes:\hspace{2ex}]}
%{\begin{trivlist}\item[\hskip \labelsep\bfseries Instructor Notes:\hspace{2ex}]}
%{}
%}
%% %% line at the bottom} 
%{\end{trivlist}\par\addvspace{.5ex}\nobreak\noindent\hung} 
%\fi
%
%


\let\instructorNotes\relax
\let\endinstructorNotes\relax
%%% instructorNotes environment
\ifhandout
\newenvironment{instructorNotes}[1][false]%
{%
\def\givenatend{\boolean{#1}}\ifthenelse{\boolean{#1}}{\begin{trivlist}\item}{\setbox0\vbox\bgroup}{}
}
{%
\ifthenelse{\givenatend}{\end{trivlist}}{\egroup}{}
}
\else
\newenvironment{instructorNotes}[1][false]%
{%
  \ifthenelse{\boolean{#1}}{\begin{trivlist}\item[\hskip \labelsep\bfseries {\Large Instructor Notes: \\} \hspace{\textwidth} ]}
{\begin{trivlist}\item[\hskip \labelsep\bfseries {\Large Instructor Notes: \\} \hspace{\textwidth} ]}
{}
}
{\end{trivlist}}
\fi


%% Suggested Timing
\newcommand{\timing}[1]{{\bf Suggested Timing: \hspace{2ex}} #1}




\hypersetup{
    colorlinks=true,       % false: boxed links; true: colored links
    linkcolor=blue,          % color of internal links (change box color with linkbordercolor)
    citecolor=green,        % color of links to bibliography
    filecolor=magenta,      % color of file links
    urlcolor=cyan           % color of external links
}

\title{Subtraction with Integers}
\author{Vic Ferdinand, Betsy McNeal, Jenny Sheldon}

\begin{document}

\begin{abstract}
We look at subtracting integers.
\end{abstract}
\maketitle

Let's look at subtraction with integers using the same three main ideas we used with addition: 
checks and bills, number lines, and investigating patterns.  Remember that we are going to 
build from what we know about negative numbers, so you may want to re-read these sections in 
our text.  Also, if you need to refresh your memory on checks and bills, chips, or number 
lines, you can go back to the section about models.

\subsection{Checks and Bills}

Let's begin with one of our most basic subtraction examples.
\begin{example}
Johnny has $9$ apples, and Suzy takes away $2$ of his apples.  How many apples does Johnny have 
now?  Write an expression using the subtraction sign which solves this problem. 
$\answer[given]{9 - 2}$
\end{example}

If we change this example to a story about checks and bills, we might use the following instead.
\begin{example}
Johnny opens his business for today with a net worth of \$$9$, and then he sends a check for \$$2$.  
What is Johnny's net worth now?
\begin{explanation}
Johnny begins with a net worth of \$$9$.  From our checks and bills model, we know that sending 
something means we should \wordChoice{\choice{add the value to} \choice[correct]{subtract the value 
from}} Johnny's total net worth.  Since the object he receives is a check, the value should be 
\wordChoice{\choice[correct]{positive} \choice{negative}}.  Therefore, the expression we would write 
which solves this problem is $\answer[given]{9 - 2}$.

Notice that Johnny's total net worth should decrease in this situation: if he sends a check to someone 
else, he should have less money overall!
\end{explanation}
\end{example}

Let's begin to include some negative numbers in our stories.
\begin{question}
Johnny opens his business for today with a net worth of \$$-8$, and then he sends a check for \$$4$.  
What is Johnny's net worth now?

\begin{prompt}
As an expression involving the subtraction sign, Johnny's net worth is now $\answer[given]{-8 - 4}$.
\end{prompt}
\end{question}
Again, after sending a check for \$$4$, Johnny's net worth should be less than it was at the beginning 
of the day.  

\begin{question}
Johnny opens his business for today with a net worth of \$$-22$, and then he sends a bill for \$$54$.  
What is Johnny's net worth now?

\begin{prompt}
As an expression involving the subtraction sign, Johnny's net worth is now $\answer[given]{-22 - (-54)}$.
\end{prompt}
\end{question}
In this question, notice that Johnny is now sending a bill.  Since we assume this bill will be paid 
by the person to whom it was sent, Johnny's total net worth should be greater than it was at the 
beginning of the day. 

You may have learned some rules about subtraction in the past.  In particular, you might be familiar 
with what happens when we subtract a negative number.  How can we make sense of what we already know, 
but in terms of checks and bills?
\begin{example}
When we subtract a negative number, our checks and bills method says we are 
\wordChoice{\choice{receiving} \choice[correct]{sending}} a 
\wordChoice{\choice{check} \choice[correct]{bill}}.  So, when writing such a story, we don't care what 
Johnny's net worth is at the beginning of the day.  Let's call it \$$A$.  Let's also say our bill is 
worth \$$B$.  We want to compute $A - (-B)$.  After the person to whom Johnny sent this bill pays, 
the result to Johnny will be that he has \$$B$ more than \$$A$.  So, we combine Johnny's original net 
worth \$$A$ with the additional amount \$$B$ to see that
\[
A - (-B) = \answer[given]{A + B}.
\]
\end{example}

We have a second way of thinking about subtraction problems which also lends itself nicely to stories 
about checks and bills.  Again, we begin with an example about apples, and then transition this 
example to one about checks and bills.
\begin{question}
Johnny had $12$ apples, and then Suzy gave Johnny some apples.  Johnny now has $19$ apples.  How many 
apples did Suzy give Johnny?

\begin{prompt}
As an expression using the subtraction sign, Johnny got $\answer[given]{19 - 12}$ apples from Suzy.
\end{prompt}
\end{question}

In this example, we might have been thinking of the related addition problem
\[
12 + ? = 19
\]
which is equivalent to the subtraction problem
\[
19 - 12.
\]

Now, let's use our checks and bills context!
\begin{question}
Johnny opens his business for today with a net worth of \$$8$ and then received something in the mail. 
After this, his net worth is now  \$$-3$.  What was the value of what Johnny received in the mail?

\begin{prompt}
As an expression using the subtraction sign, Johnny got $\answer[given]{-3 - 8}$ in the mail.
\end{prompt}
\end{question}

\begin{question}
Johnny opens his business for today with a net worth of \$$-2$ and then the mail arrives.  Afterwards, 
Johnny's net worth is now \$$-7$.  What was the value of what Johnny received in the mail?

\begin{prompt}
As an expression using the subtraction sign, Johnny got $\answer[given]{-7 - (-2)}$ in the mail.
\end{prompt}
\end{question}

Again, notice how the checks and bills story helps us understand why subtracting a negative number 
is equivalent to adding the value of the bill.  Don't hesitate to ask any questions you have about 
this concept!

With subtraction problems, notice that we are sometimes sending something in the mail, but 
sometimes receiving something as well!  Be careful as you phrase your questions that you have the 
appropriate operation!

Along these lines, we should again be very careful when asking questions in our story problems. 
Here are some common pitfalls.

\begin{example}
Johnny sends a check for \$$18$ and a bill for \$$14$.  What is Johnny's net worth now?
\begin{explanation}
We don't actually have enough information to answer this question, because we don't know what 
Johnny's net worth was when the day began.  If he began with a net worth of \$$10$, his new 
net worth is $\answer[given]{10 - 18 - (-14)}$.  If he began the day with \$$0$, his new net 
worth is $\answer[given]{0 - 18 - (-14)}$.  Neither of these expressions is the same as $18 - 14$.
\end{explanation}
\end{example}

\begin{example}
Johnny starts the day with a net worth of \$$7$, and then sends a bill for \$$5$.  How much 
did Johnny's net worth change?
\begin{explanation}
This story looks very similar to some of our other stories, but if we read the question 
carefully, we see an important difference.  We are asked how much Johnny's net worth 
{\em changes}, so we do not need the information about his beginning net worth.  Our bill was for 
\$$5$, so his net worth changes by $\answer[given]{-5}$.  This is not a story problem for $7 - (-5)$.
\end{explanation}
\end{example}

\begin{example}
Johnny starts the day with a net worth of \$$12$, and then receives a bill in the mail.  If 
Johnny now has \$$9$, how much is the bill Johnny received?
\begin{explanation}
Again, this story may seem familiar at first glance.  In this case, however, our story specified 
that what Johnny received in the mail was a bill.  The dollar amount of this bill must have been 
\$$\answer[given]{3}$, but notice that the answer to this question must be a positive number, 
even though $9 - 12$ is negative.
\end{explanation}
\end{example}

Any time you write a story problem, it's an excellent practice to go back and try to answer 
the question from an objective perspective.  Is the question really asking what you intend?  
Is there any other way the question could be interpreted?  Especially in class or on a homework 
assignment, ask someone else for their opinion!




\subsection{Number Lines}

Next, let's use a number line to solve some subtraction problems with integers.  You can write 
a story problem to go along with each of these expressions for extra practice.

\begin{example}
Imagine using a number line like the one below to solve the subtraction problem $5 - 3$.
\begin{center}
\begin{tikzpicture}[font=\Large]
\draw[<->] (-0.5,0) -- (12.5,0);
\foreach \x in {0,1.5, 3, 4.5, 6, 7.5, 9, 10.5,12}
\draw[shift={(\x,0)},color=black] (0pt,3pt) -- (0pt,-2pt);
\draw (0,0) node[below]{$-4$};
\draw (1.5,0) node[below]{$-3$};
\draw (3,0) node[below]{$-2$};
\draw (4.5,0) node[below]{$-1$};
\draw (6,0) node[below]{$0$};
\draw (7.5,0) node[below]{$1$};
\draw (9,0) node[below]{$2$};
\draw (10.5,0) node[below]{$3$};
\draw (12,0) node[below]{$4$};
\end{tikzpicture}  
\end{center}
We begin by standing on the number line at the tick marked with $\answer[given]{5}$.  Since we 
are subtracting, we face towards the \wordChoice{\choice{right} \choice[correct]{left}}.  We 
will move $\answer[given]{3}$ spaces \wordChoice{\choice[correct]{forward} \choice{backward}}, 
since $3$ is positive.  Where on the number line are we now? 

\begin{prompt}
We are located at the tick labeled $\answer[given]{2}$.
\end{prompt}
\end{example}

\begin{example}
Imagine using a number line like the one below to solve the subtraction problem $-8 - 4$.
\begin{center}
\begin{tikzpicture}[font=\Large]
\draw[<->] (-0.5,0) -- (12.5,0);
\foreach \x in {0,1.5, 3, 4.5, 6, 7.5, 9, 10.5,12}
\draw[shift={(\x,0)},color=black] (0pt,3pt) -- (0pt,-2pt);
\draw (0,0) node[below]{$-4$};
\draw (1.5,0) node[below]{$-3$};
\draw (3,0) node[below]{$-2$};
\draw (4.5,0) node[below]{$-1$};
\draw (6,0) node[below]{$0$};
\draw (7.5,0) node[below]{$1$};
\draw (9,0) node[below]{$2$};
\draw (10.5,0) node[below]{$3$};
\draw (12,0) node[below]{$4$};
\end{tikzpicture}  
\end{center}
We begin by standing on the number line at the tick marked with $\answer[given]{-8}$.  Since 
we are subtracting, we face towards the \wordChoice{\choice{right} \choice[correct]{left}}.  We 
will move $\answer[given]{4}$ spaces \wordChoice{\choice[correct]{forward} \choice{backward}}, 
since $4$ is positive.  Where on the number line are we now? 

\begin{prompt}
We are located at the tick labeled $\answer[given]{-12}$.
\end{prompt}
\end{example}

\begin{example}
Imagine using a number line like the one below to solve the subtraction problem $(-22) - (-54)$.
\begin{center}
\begin{tikzpicture}[font=\Large]
\draw[<->] (-0.5,0) -- (12.5,0);
\foreach \x in {0,1.5, 3, 4.5, 6, 7.5, 9, 10.5,12}
\draw[shift={(\x,0)},color=black] (0pt,3pt) -- (0pt,-2pt);
\draw (0,0) node[below]{$-4$};
\draw (1.5,0) node[below]{$-3$};
\draw (3,0) node[below]{$-2$};
\draw (4.5,0) node[below]{$-1$};
\draw (6,0) node[below]{$0$};
\draw (7.5,0) node[below]{$1$};
\draw (9,0) node[below]{$2$};
\draw (10.5,0) node[below]{$3$};
\draw (12,0) node[below]{$4$};
\end{tikzpicture}  
\end{center}
We begin by standing on the number line at the tick marked with $\answer[given]{-22}$.  Since 
we are subtracting, we face towards the \wordChoice{\choice{right} \choice[correct]{left}}.  We will 
move $\answer[given]{54}$ spaces \wordChoice{\choice{forward} \choice[correct]{backward}}, since 
$54$ is negative.  Where on the number line are we now? 

\begin{prompt}
We are located at the tick labeled $\answer[given]{32}$.
\end{prompt}
\end{example}

Why does subtracting a negative number give us the same result as adding that number?  Using 
number lines, we can see that if we are subtracting, we are facing {\em left} while moving
{\em backward}.  The net result is the same as if we were facing right while moving forward. 
Try this out with some friends if you are skeptical.

\subsection{Patterns}

Finally, we investigate subtraction of negative numbers via patterns.
\begin{example}
Consider the sequence of addition problems.
\begin{align*}
5 - 4 &= \answer[given]{1} \\
5 - 3 &= \answer[given]{2} \\
5 - 2 &= \answer[given]{3} \\
5 - 1 &= \answer[given]{4} \\
5 - 0 &= \answer[given]{5}
\end{align*}
As we move down the chart, moving one row down results in the final answer increasing by 
$\answer[given]{1}$.  So, if the pattern continues to hold, we expect the answer to 
$5 - (-1)$ to be $\answer[given]{6}$, since it is one less than $5$.  We can also notice that 
the answer to $5 - (-1)$ is the same as the answer to $5 + \answer[given]{1}$.
\end{example}
Try your hand at recognizing patterns with some other subtraction problems.

Finally, notice that no matter how we approach the problems in this section, we are getting 
consistent answers.  Also, no matter how we look at subtraction of a negative number, we can 
see that it should be the same as addition.  Subtraction as an operation remains the same, 
no matter how we model it.  


\end{document}
