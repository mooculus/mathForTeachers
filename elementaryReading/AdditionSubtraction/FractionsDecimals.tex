\documentclass{ximera}

\usepackage{gensymb}
\usepackage{tabularx}
\usepackage{mdframed}
\usepackage{pdfpages}
%\usepackage{chngcntr}

\let\problem\relax
\let\endproblem\relax

\newcommand{\property}[2]{#1#2}




\newtheoremstyle{SlantTheorem}{\topsep}{\fill}%%% space between body and thm
 {\slshape}                      %%% Thm body font
 {}                              %%% Indent amount (empty = no indent)
 {\bfseries\sffamily}            %%% Thm head font
 {}                              %%% Punctuation after thm head
 {3ex}                           %%% Space after thm head
 {\thmname{#1}\thmnumber{ #2}\thmnote{ \bfseries(#3)}} %%% Thm head spec
\theoremstyle{SlantTheorem}
\newtheorem{problem}{Problem}[]

%\counterwithin*{problem}{section}



%%%%%%%%%%%%%%%%%%%%%%%%%%%%Jenny's code%%%%%%%%%%%%%%%%%%%%

%%% Solution environment
%\newenvironment{solution}{
%\ifhandout\setbox0\vbox\bgroup\else
%\begin{trivlist}\item[\hskip \labelsep\small\itshape\bfseries Solution\hspace{2ex}]
%\par\noindent\upshape\small
%\fi}
%{\ifhandout\egroup\else
%\end{trivlist}
%\fi}
%
%
%%% instructorIntro environment
%\ifhandout
%\newenvironment{instructorIntro}[1][false]%
%{%
%\def\givenatend{\boolean{#1}}\ifthenelse{\boolean{#1}}{\begin{trivlist}\item}{\setbox0\vbox\bgroup}{}
%}
%{%
%\ifthenelse{\givenatend}{\end{trivlist}}{\egroup}{}
%}
%\else
%\newenvironment{instructorIntro}[1][false]%
%{%
%  \ifthenelse{\boolean{#1}}{\begin{trivlist}\item[\hskip \labelsep\bfseries Instructor Notes:\hspace{2ex}]}
%{\begin{trivlist}\item[\hskip \labelsep\bfseries Instructor Notes:\hspace{2ex}]}
%{}
%}
%% %% line at the bottom} 
%{\end{trivlist}\par\addvspace{.5ex}\nobreak\noindent\hung} 
%\fi
%
%


\let\instructorNotes\relax
\let\endinstructorNotes\relax
%%% instructorNotes environment
\ifhandout
\newenvironment{instructorNotes}[1][false]%
{%
\def\givenatend{\boolean{#1}}\ifthenelse{\boolean{#1}}{\begin{trivlist}\item}{\setbox0\vbox\bgroup}{}
}
{%
\ifthenelse{\givenatend}{\end{trivlist}}{\egroup}{}
}
\else
\newenvironment{instructorNotes}[1][false]%
{%
  \ifthenelse{\boolean{#1}}{\begin{trivlist}\item[\hskip \labelsep\bfseries {\Large Instructor Notes: \\} \hspace{\textwidth} ]}
{\begin{trivlist}\item[\hskip \labelsep\bfseries {\Large Instructor Notes: \\} \hspace{\textwidth} ]}
{}
}
{\end{trivlist}}
\fi


%% Suggested Timing
\newcommand{\timing}[1]{{\bf Suggested Timing: \hspace{2ex}} #1}




\hypersetup{
    colorlinks=true,       % false: boxed links; true: colored links
    linkcolor=blue,          % color of internal links (change box color with linkbordercolor)
    citecolor=green,        % color of links to bibliography
    filecolor=magenta,      % color of file links
    urlcolor=cyan           % color of external links
}


\title{Adding and subtracting with fractions and decimals}
\author{Jenny Sheldon}

\begin{document}

\begin{abstract}
We look at the meaning of addition and subtraction applied to fractions and decimals.
\end{abstract}
\maketitle

\section{Activities for this section:} 3Q, 3R

\section{Adding and subtracting fractions}

The main point that we would like for you to see in this section is that the meaning of addition and the meaning of subtraction fit with any kinds of numbers we would like to add or subtract, not just whole numbers. We don't need to change the way we think about these operations depending on the kind of numbers we are working with. We think this is an advantage, letting us build on things that we understand as we consider more and more difficult problems.

\begin{question}
Gino is baking some brownies. Starting with an empty bowl, he places $\frac{3}{4}$ of a cup of water and $1 \frac{1}{4}$ of a cup of oil into the bowl. How much liquid (how many cups of water and oil together) are now in the bowl?

\begin{explanation}
Let's draw a picture to solve this problem, and use our picture to think about what operation we could use to solve this problem.

First, we'll use the meaning of fractions to draw a picture of $\frac{3}{4}$ of a cup of water. In this case our whole is \wordChoice{\choice{3} \choice{4} \choice{water} \choice[correct]{one cup of water}}. The whole is cut into $\answer[given]{4}$ equal pieces, because the denominator tells us how many equal pieces to cut our whole into, and then we want to shade $\answer[given]{3}$ of those pieces, because the numerator tells us how many pieces to shade.

\begin{center}
\begin{tikzpicture}
\draw[fill=cyan] (0,0)--(3,0)--(3,2)--(0,2)--(0,0);
\draw[thick] (0,0)--(4,0)--(4,2)--(0,2)--(0,0);
\foreach \x in {1, 2, 3} \draw[thick] (\x, 0)--(\x, 2);
\node[below] at (2, 0) {one cup of water};
\end{tikzpicture}
\end{center}

Next, we want to represent $1 \frac{1}{4}$ cups of oil. This is one full cup of oil and $\frac{1}{4}$ of a second cup. In this case our whole is \wordChoice{\choice{4} \choice{water} \choice{one cup of water} \choice{oil} \choice[correct]{one cup of oil}}, and this whole will be cut into $\answer[given]{4}$ equal pieces. We need to shade one full cup, and then $\answer[given]{1}$ piece out of the second cup.

\begin{center}
\begin{tikzpicture}
\draw[fill=brown] (0,0)--(4,0)--(4,2)--(0,2)--(0,0);
\draw[fill=brown, thick] (5,0)--(6,0)--(6,2)--(5,2)--(5,0);
\draw[thick] (6,0)--(9,0)--(9,2)--(6,2);
\foreach \x in {1, 2, 3, 6, 7, 8} \draw[thick] (\x, 0)--(\x, 2);
\node[below] at (2, 0) {one cup of oil};
\end{tikzpicture}
\end{center}

Now, Gino takes all of the shaded pieces and puts them together in a bowl. We want to combine these pieces together, and that means we are going to use the operation \wordChoice{\choice[correct]{addition} \choice{subtraction} \choice{something else}} because the definition of addition is to combine things together. In this case, since the story problem mentions the water first, we will write the amount of water before the amount of oil in our expression. So the total amount of liquid in the bowl is
\[
\answer[given]{\frac{3}{4}} + \answer[given]{\frac{5}{4}} \textrm{ cups of liquid. }
\]

We could stop here, because this expression is the total amount of liquid in the bowl. However, most of the time we want to simplify this expression and write it without the plus sign, so let's continue working with our picture by putting together all of the shaded pieces that we had. We'll mark the water ones with a small $w$ and the oil ones with a small $o$ to make them easier to see in the picture.

\begin{center}
\begin{tikzpicture}
	\draw[thick, fill=cyan] (0,0)--(3,0)--(3,2)--(0,2)--(0,0);
	\draw[fill=brown, thick] (3,0)--(8,0)--(8,2)--(3,2)--(3,0);
	\foreach \x in {1, 2, 3, 4, 5, 6, 7} \draw[thick] (\x, 0)--(\x, 2);
	\foreach \a in {0.5, 1.5, 2.5} \node at (\a, 1) {$w$};
	\foreach \b in {3.5, 4.5, 5.5, 6.5, 7.5} \node at (\b, 1) {$o$};
\end{tikzpicture}
\end{center}
In order to see how much liquid this is, we need to think about what size each of these pieces are. Each of the original fractions had a denominator of $\answer[given]{4}$, so both the cup of water and the cup of oil were cut into $4$ equal pieces. This means that each of the pieces in our picture is $\frac{1}{4}$ of a cup of either water or oil. We need to change our whole to be \wordChoice{\choice{one cup of water} \choice{one cup of oil} \choice[correct]{one cup of liquid}} and since water and oil are both kinds of liquid we can think about all of our previous parts as $\frac{1}{\answer[given]{4}}$ of a cup of liquid. In other words, one cup of liquid is made up of $\answer[given]{4}$ of the pieces we already drew. Let's add that information to our picture.

\begin{center}
\begin{tikzpicture}
	\draw[thick, fill=cyan] (0,0)--(3,0)--(3,2)--(0,2)--(0,0);
	\draw[fill=brown, thick] (3,0)--(8,0)--(8,2)--(3,2)--(3,0);
	\foreach \x in {1, 2, 3, 4, 5, 6, 7} \draw[thick] (\x, 0)--(\x, 2);
	\foreach \a in {0.5, 1.5, 2.5} \node at (\a, 1) {$w$};
	\foreach \b in {3.5, 4.5, 5.5, 6.5, 7.5} \node at (\b, 1) {$o$};
	\draw[ultra thick] (0,0)--(4,0)--(4,2)--(0,2)--(0,0);
	\node[below] at (2,0) {one cup of liquid};
\end{tikzpicture}
\end{center}

There are a total of $\answer[given]{8}$ shaded pieces in the picture when we consider the water and oil together, and our whole cup of liquid is cut into $\answer[given]{4}$ equal pieces, so we could write our answer as 
\[
\frac{\answer[given]{8}}{\answer[given]{4}} \textrm{ cups of liquid, }
\]
or if we simplify this fraction we can see that it's the same as $2$ cups of liquid in total.

\end{explanation}
\end{question}


We hope that felt in some ways just like the problems we solved in the previous section, but there are two big ideas to watch out for when we work with fraction problems. The first is that we need to be particularly careful with how we ask the question when we are working with fractions. Let's consider a series of questions all referring to Gino and these brownies.

\begin{question}
Gino is baking some brownies. Starting with an empty bowl, he places $\frac{3}{4}$ of a cup of water and $1 \frac{1}{4}$ of a cup of oil into the bowl. How much water is in the bowl?

\begin{multipleChoice}
\choice[correct]{$\frac{3}{4}$ of a cup}
\choice{$1 \frac{1}{4}$ of a cup}
\choice{$2$ cups}
\choice{Some other number}
\choice{We cannot answer the question from this information.}
\end{multipleChoice}
\end{question}


\begin{question}
Gino is baking some brownies. Starting with an empty bowl, he places $\frac{3}{4}$ of a cup of water and $1 \frac{1}{4}$ of a cup of oil into the bowl. How full is the bowl?

\begin{multipleChoice}
\choice{$\frac{3}{4}$ of a cup}
\choice{$1 \frac{1}{4}$ of a cup}
\choice{$2$ cups}
\choice{Some other number}
\choice[correct]{We cannot answer the question from this information.}
\end{multipleChoice}
\end{question}


\begin{question}
Gino is baking some brownies. Starting with an empty bowl, he places $\frac{3}{4}$ of a cup of water and $1 \frac{1}{4}$ of a cup of oil into the bowl. How much more oil is in the bowl than water?

\begin{multipleChoice}
\choice{$\frac{3}{4}$ of a cup}
\choice{$1 \frac{1}{4}$ of a cup}
\choice{$2$ cups}
\choice[correct]{Some other number}
\choice{We cannot answer the question from this information.}
\end{multipleChoice}
\end{question}

Yikes. Even with the same setup, how we ask the question makes a big difference in what we get for our answer -- and there are some questions we can ask that we can't answer without more information! When we add and subtract fractions, it's important for each of the fractions to have the same whole, and for the question to also be asking for the same whole. That might seem a little surprising since we just worked with a problem where we added water and oil, but notice that each of the water and the oil can be considered liquid. So, we added $\frac{3}{4}$ of a cup of liquid to $1 \frac{1}{4}$ cup of liquid, and the answer was also in cups of liquid. On the other hand, if we asked something like ``how full is the bowl?'', we are taking $\frac{3}{4}$ of a cup of liquid and putting it together with $1 \frac{1}{4}$ of a cup of liquid, but asking for our answer in terms of a different whole, the bowl. Addition or subtraction just can't answer this question because the wholes aren't the same. On the other hand, the question ``how much more oil is in the bowl than water?" is a comparison question, and so it's not asking us to add $\frac{3}{4} + 1 \frac{1}{4}$. We could solve this problem (and we could maybe even use addition to do so), but we would draw a different picture and add different numbers than those in the original problem. Pay close attention to the wording any time you are writing or considering a word problem, and try to anticipate things that might be confusing. This takes practice!

The second idea to keep in mind when we are working with fractions is that if we would like to simplify a number like $\frac{5}{8} - \frac{1}{3}$, it's much easier to tell what fraction we are working with if all of our pieces are the same size. 

\begin{question}
Shonda ran $\frac{5}{8}$ of a mile on Monday, but only ran $\frac{1}{3}$ of a mile on Tuesday. How much farther did Shonda run on Monday than on Tuesday?

\begin{explanation}

We want to see how we could write the answer to this question as a subtraction expression, and then simplify our answer into a single fraction. We'll use a number line to represent our fractions this time. First, let's draw our whole, which in this problem is \wordChoice{\choice{8} \choice{3} \choice[correct]{one mile} \choice{distance}}. We'll draw two different number lines, one on top of the other, so that we can compare Monday's distance and Tuesday's distance.

\begin{center}
\begin{tikzpicture}
\draw[thick, <->] (-0.2,0)--(12.2,0);
\node[left] at (-0.5, 0) {Tuesday};
\node[left] at (-0.5, 2) {Monday};
\draw[thick, <->] (-0.2, 2)--(12.2,2);
\foreach \x/\y in {0/0, 0/2, 12/0, 12/2} {\draw[thick] (\x, \y-0.3)--(\x, \y+0.3);}
\node[below] at (0, -0.3) {$0$};
\node[below] at (12, -0.3) {$1$ mile};
\node[below] at (0, 1.7) {$0$};
\node[below] at (12, 1.7) {$1$ mile};
\end{tikzpicture}
\end{center}

Because Shonda ran $\frac{5}{8}$ of a mile on Monday, the denominator tells us that we need to cut the mile on Monday's line into $\answer[given]{8}$ equal pieces, and the numerator tells us that we need to shade $\answer[given]{5}$ of them. Because Shonda ran $\frac{1}{3}$ of a mile on Tuesday, the denominator tells us that we need to cut the mile on Tuesday's line into $\answer[given]{3}$ equal pieces, and the numerator tells us that we need to shad $\answer[given]{1}$ of them. Let's do that on our picture. %5/8 = 15/24 = 7.5/12 and 1/3=8/24=4/12

\begin{center}
\begin{tikzpicture}
\draw[fill=green] (0, 1.9)--(7.5,1.9)--(7.5,2.1)--(0,2.1)--(0,1.9);
\draw[fill=teal] (0, -0.1)--(4,-0.1)--(4,0.1)--(0,0.1)--(0,-0.1);
\draw[thick, <->] (-0.2,0)--(12.2,0);
\node[left] at (-0.5, 0) {Tuesday};
\node[left] at (-0.5, 2) {Monday};
\draw[thick, <->] (-0.2, 2)--(12.2,2);
\foreach \x/\y in {0/0, 0/2, 12/0, 12/2} {\draw[thick] (\x, \y-0.4)--(\x, \y+0.4);}
\foreach \a in {1.5, 3, 4.5, 6, 7.5, 9, 10.5} \draw[thick] (\a, 1.7)--(\a, 2.3);
\foreach \b in {4, 8} \draw[thick] (\b, -0.3)--(\b, 0.3);
\node[below] at (0, -0.3) {$0$};
\node[below] at (12, -0.3) {$1$ mile};
\node[below] at (0, 1.7) {$0$};
\node[below] at (12, 1.7) {$1$ mile};
\node[below] at (4, -0.3) {$\frac{1}{3}$ mi};
\node[above] at (7.5, 2.3) {$\frac{5}{8}$ mi};
\end{tikzpicture}
\end{center}
We want to find out how much farther Shonda ran on Monday, so we want to take away Tuesday's distance from Monday's distance. In a sense, we want to line up the end of Tuesday with the Monday line, and then cross off Tuesday's distance. Let's show that in our picture.

\begin{center}
\begin{tikzpicture}
\draw[fill=green] (0, 1.9)--(7.5,1.9)--(7.5,2.1)--(0,2.1)--(0,1.9);
\draw[fill=teal] (0, -0.1)--(4,-0.1)--(4,0.1)--(0,0.1)--(0,-0.1);
\draw[thick, <->] (-0.2,0)--(12.2,0);
\node[left] at (-0.5, 0) {Tuesday};
\node[left] at (-0.5, 2) {Monday};
\draw[thick, <->] (-0.2, 2)--(12.2,2);
\foreach \x/\y in {0/0, 0/2, 12/0, 12/2} {\draw[thick] (\x, \y-0.4)--(\x, \y+0.4);}
\foreach \a in {1.5, 3, 4.5, 6, 7.5, 9, 10.5} \draw[thick] (\a, 1.7)--(\a, 2.3);
\foreach \b in {4, 8} \draw[thick] (\b, -0.3)--(\b, 0.3);
\node[below] at (0, -0.3) {$0$};
\node[below] at (12, -0.3) {$1$ mile};
\node[below] at (0, 1.7) {$0$};
\node[below] at (12, 1.7) {$1$ mile};
\node[below] at (4, -0.3) {$\frac{1}{3}$ mi};
\node[above] at (7.5, 2.3) {$\frac{5}{8}$ mi};
\draw[thick, dashed] (4, 0.3)--(4, 2.5);
\draw[thick] (0, 2.5)--(4, 1.5);
\draw[thick] (0, 1.5)--(4, 2.5);
\end{tikzpicture}
\end{center}

This picture shows us taking away $\frac{1}{3}$ of a mile from $\frac{5}{8}$ of a mile, so the operation we are using here is \wordChoice{\choice{addition} \choice[correct]{subtraction} \choice{something else}}. As an expression, we could write our answer as
\[
\answer[given]{\frac{5}{8}} - \answer[given]{\frac{1}{3}}.
\]

We would also like to be able to simplify this answer, but right now that's difficult because we don't know exactly where the $\frac{1}{3}$ of a mile crosses Monday's number line. The main issue is that our pieces are not all the same size. So, let's cut all of these pieces some more. If we cut Monday's eight equal pieces into $\answer[given]{3}$ equal pieces each, and if we cut Tuesday's $3$ equal pieces into $\answer[given]{8}$ equal pieces each, we will have the one mile made out of $24$ equal pieces of both Monday's number line and Tuesday's number line. In other words, we are making equivalent fractions whose denominators are $24$. Let's go ahead and do that in our picture.

\begin{center}
\begin{tikzpicture}
\draw[fill=green] (0, 1.9)--(7.5,1.9)--(7.5,2.1)--(0,2.1)--(0,1.9);
\draw[fill=teal] (0, -0.1)--(4,-0.1)--(4,0.1)--(0,0.1)--(0,-0.1);
\draw[thick, <->] (-0.2,0)--(12.2,0);
\node[left] at (-0.5, 0) {Tuesday};
\node[left] at (-0.5, 2) {Monday};
\draw[thick, <->] (-0.2, 2)--(12.2,2);
\foreach \x/\y in {0/0, 0/2, 12/0, 12/2} {\draw[thick] (\x, \y-0.4)--(\x, \y+0.4);}
\foreach \a in {1.5, 3, 4.5, 6, 7.5, 9, 10.5} \draw[thick] (\a, 1.7)--(\a, 2.3);
\foreach \b in {4, 8} \draw[thick] (\b, -0.3)--(\b, 0.3);
\foreach \x/\y in {0.5/0, 0.5/2, 1/0, 1/2, 1.5/0, 1.5/2, 2/0, 2/2, 2.5/0, 2.5/2, 3/0, 3/2, 3.5/0, 3.5/2, 4/0, 4/2, 4.5/0, 4.5/2, 5/0, 5/2, 5.5/0, 5.5/2, 6/0, 6/2, 6.5/0, 6.5/2, 7/0, 7/2, 7.5/0, 7.5/2, 8/0, 8/2, 8.5/0, 8.5/2, 9/0, 9/2, 9.5/0, 9.5/2, 10/0, 10/2, 10.5/0, 10.5/2, 11/0, 11/2, 11.5/0, 11.5/2} {\draw[thick] (\x, \y-0.2)--(\x, \y+0.2);}
\node[below] at (0, -0.3) {$0$};
\node[below] at (12, -0.3) {$1$ mile};
\node[below] at (0, 1.7) {$0$};
\node[below] at (12, 1.7) {$1$ mile};
\node[below] at (4, -0.3) {$\frac{1}{3} = \frac{8}{24}$ mi};
\node[above] at (7.5, 2.3) {$\frac{5}{8} = \frac{15}{24}$ mi};
\draw[thick, dashed] (4, 0.3)--(4, 2.5);
\draw[thick] (0, 2.5)--(4, 1.5);
\draw[thick] (0, 1.5)--(4, 2.5);
\end{tikzpicture}
\end{center}

Now we can see that the $\frac{1}{3}$ mile mark is at the $8$th mark after zero, so it is equivalent to $\frac{8}{24}$ of a mile. We can also see that $\frac{5}{8}$ is at the $15$th mark after zero, so it is equivalent to $\frac{15}{24}$ of a mile. And now that both lines are marked in units of $\frac{1}{24}$ of a mile, we can take away the $8$ copies of $\frac{1}{24}$ of a mile from the $15$ copies of $\frac{1}{24}$ of a mile to see that we have $\answer[given]{7}$ copies of $\frac{1}{24}$ of a mile left. Using our meaning of fractions, we see that the answer to this problem is
\[
\frac{\answer[given]{7}}{24} \textrm{ of a mile.}
\]

\end{explanation}

\end{question}

You might remember that we have a rule for adding and subtracting fractions which says that we first make a common denominator and then add or subtract just the numerators. For instance, if we want to add
\[
\frac{5}{6} + \frac{4}{9}, 
\]
first we would exchange these fractions for ones with a common denominator. Frequently, we multiply the denominators together to get a common denominator, but any common denominator will do. In this case we'll use the denominator of $18$, which will keep our numbers a little bit smaller. So, we exchange these fractions for equivalent ones which have denominator $18$.
\[
\frac{15}{18} + \frac{8}{18}
\]
Notice that this step is the same thing we did in the previous example when we made all of the pieces of our mile the same size. Once the pieces are the same size, it's easy to combine the pieces or take away pieces and know how many pieces you have in total. In other words, now that we have a common denominator, we only need to add the numerators.
\[
\frac{5}{6} + \frac{4}{9} = \frac{15}{18} + \frac{8}{18} = \frac{15+8}{18} = \frac{23}{18}
\]
We hope that you see how this rule for adding and subtracting fractions makes sense with both our meaning of addition and subtraction as well as our meaning of fractions. We don't need to start by memorizing this one, we can just think through what we need in order to solve the problem. Once we've solved problems like this a bunch of times, we can think about the rule as a short cut, but we are less likely to forget it because we understand where it comes from.

Notice also that some students say things like ``in order to subtract fractions, we have to make a common denominator''. This is a little bit misleading. In our example about Shonda's running, we subtracted the fractions once we wrote down $\frac{5}{8} - \frac{1}{3}$. We didn't need to go any farther, because this expression is the answer to the problem and we've done our subtraction. The reason we made common denominators is that we wanted to \emph{simplify} that answer into something that's easier for us to work with. And for simplifying the answer, we definitely needed equally-sized pieces, or common denominators.


To wrap up our thinking about addition and fractions, let's return to something we said about mixed numbers. We said that a mixed number like $2 \frac{3}{7}$ could also be written as $2 + \frac{3}{7}$. Notice how this makes sense with our meaning of addition. The number $2 + \frac{3}{7}$ is the total amount that we get from combining together $2$ whole units and $\frac{3}{7}$ of another unit. And that's the same thing that we mean when we write $2 \frac{3}{7}$ next to each other. In other words, we can think about $2 + \frac{3}{7}$ using a picture like we did for addition. We will draw two wholes and $\frac{3}{7}$ of another whole, and then use arrows to show the result after combining all of that together. The result is $2 \frac{3}{7}$ of the same whole.

\begin{center}
\begin{tikzpicture}
\draw[thick, fill=lime] (0,0)--(3.5,0)--(3.5,1)--(0,1)--(0,0);
\node[below] at (1.75, 0) {one whole};
\draw[thick, fill=lime] (4,0)--(7.5,0)--(7.5,1)--(4,1)--(4,0);
\node[below] at (5.75, 0) {one whole};
\draw[thick, fill=lime] (10,0)--(11.5, 0)--(11.5, 1)--(10,1)--(10,0);
\draw[thick] (11.5,0)--(13.5, 0)--(13.5,1)--(11.5,1);
\foreach \x in {10.5, 11, 12, 12.5, 13} \draw[thick] (\x, 0)--(\x, 1);
\node[below] at (11.75, 0) {one whole};
\node[above] at (10.75, 1) {$\frac{3}{7}$ of a whole};
\draw[thick, ->] (4, -0.5)--(6, -1.5);
\draw[thick, ->] (11.75, -0.5)--(9.5,-1.5);
\node at (7.75, -1.3) {combine};
\draw[thick, fill=lime] (1.5,-3)--(10, -3)--(10,-2)--(1.5,-2)--(1.5,-3);
\draw[thick, dashed] (10, -3)--(12, -3)--(12, -2)--(10, -2);
\draw[ultra thick] (1.5, -3)--(5,-3)--(5,-2)--(1.5,-2)--(1.5,-3);
\node[below] at (3.25, -3) {one whole};
\draw[ultra thick] (5,-3)--(8.5, -3)--(8.5, -2)--(5,-2);
\node[below] at (6.75, -3) {one whole};
\foreach \x in {9, 9.5} \draw[thick] (\x, -3)--(\x, -2);
\foreach \x in {10.5, 11, 11.5} \draw[thick, dashed] (\x, -3)--(\x, -2);
\node[below] at (10.25, -3) {one whole};
\end{tikzpicture}
\end{center}


\section{Adding and subtracting decimal numbers}

You might also remember an algorithm for adding and subtracting decimal numbers; we'll tackle that in the next section when we tackle the addition and subtraction algorithms for whole numbers. Instead, let's take a look at how we could solve a decimal subtraction problem using a picture with base ten blocks.

\begin{question}
Raina knows that she needs to make $5.3$ liters of lemonade for a school fundraiser. If $2.84$ liters of lemonade need to be sugar free, how much regular (non-sugar free) lemonade does Raina need to make?

\begin{explanation}
Let's see how Raina's lemonade problem can be solved with a subtraction expression, and then let's simplify the answer into a single number.

First, the operation of subtraction means that we need to \wordChoice{\choice{combine} \choice[correct]{take away} \choice{ignore}} some things in our story problem. In this case, Raina starts with the full amount of lemonade, which is $5.3$ liters. Since we know how much lemonade needs to be sugar free, we can take that amount away from the total to find the amount of lemonade that is not sugar free. In other words, we are using subtraction to solve this problem because we are taking away one amount from another amount and counting what remains. The amount we start with is the $5.3$ liters, and the amount we take away is the $2.84$ liters, so our subtraction expression is
\[
\answer[given]{5.3} - \answer[given]{2.84}.
\]

Now, we'd like to simplify this answer into a single number. We will start by drawing the $5.3$ liters using base ten blocks. Since we know that one of our numbers in this problem has a value in the hundredths place, we will make the value of the smallest block equal to $\answer[given]{0.01}$, meaning that the bundle is one place to the left or has value $\answer[given]{0.1}$ and the superbundle is two places to the left or has value $1$. In other words, in order to represent the number $5.3$ we will use $\answer[given]{5}$ superbundles and $\answer[given]{3}$ bundles. Let's go ahead and draw this amount.

\begin{center}
\begin{tikzpicture}
\foreach \x in {0, 1.2, 2.4, 3.6, 4.8} \draw[thick, step=0.1] (\x,0) grid (\x+1, 1);
\foreach \x in {6, 6.2, 6.4} \draw[thick, step=0.1] (\x, 0) grid (\x+0.101, 1);
\end{tikzpicture}
\end{center}

Now, we want to take away $2.84$ liters from this amount. This means that we need to remove $\answer[given]{2}$ superbundles, $\answer[given]{8}$ bundles, and $\answer[given]{4}$ individual blocks. Let's start with the $2$ superbundles. We will mark them off with a red line.

\begin{center}
\begin{tikzpicture}
\foreach \x in {0, 1.2, 2.4, 3.6, 4.8} \draw[thick, step=0.1] (\x,0) grid (\x+1, 1);
\foreach \x in {6, 6.2, 6.4} \draw[thick, step=0.1] (\x, 0) grid (\x+0.101, 1);
\draw[ultra thick, red] (-0.2, -0.2)--(1.2, 1.2);
\draw[ultra thick, red] (1, -0.2)--(2.4, 1.2);
\end{tikzpicture}
\end{center}

Next, we need to remove $8$ bundles from this picture, but right now we only have $3$ bundles. Let's unbundle the last superbundle into $\answer[given]{10}$ bundles so that we have enough bundles to do this taking away. We will put a purple dotted box around the $10$ bundles that used to be this superbundle, just for this step so that you can see them.

\begin{center}
\begin{tikzpicture}
\foreach \x in {0, 1.2, 2.4, 3.6} \draw[thick, step=0.1] (\x,0) grid (\x+1, 1);
\foreach \x in {4.8, 5, 5.2, 5.4, 5.6, 5.8, 6, 6.2, 6.4, 6.6, 6.8, 7, 7.2} \draw[thick, step=0.1] (\x, 0) grid (\x+0.101, 1);
\draw[dotted, thick, violet] (4.7, -0.2)--(6.75, -0.2)--(6.75, 1.2)--(4.7, 1.2)--(4.7, -0.2);
\draw[ultra thick, red] (-0.2, -0.2)--(1.2, 1.2);
\draw[ultra thick, red] (1, -0.2)--(2.4, 1.2);
\end{tikzpicture}
\end{center}

Now we can take away $8$ bundles using our big red lines again. (We removed the purple dotted box from this one, since the bundles don't make a superbundle anymore.)

\begin{center}
\begin{tikzpicture}
\foreach \x in {0, 1.2, 2.4, 3.6} \draw[thick, step=0.1] (\x,0) grid (\x+1, 1);
\foreach \x in {4.8, 5, 5.2, 5.4, 5.6, 5.8, 6, 6.2, 6.4, 6.6, 6.8, 7, 7.2} \draw[thick, step=0.1] (\x, 0) grid (\x+0.101, 1);
\draw[ultra thick, red] (-0.2, -0.2)--(1.2, 1.2);
\draw[ultra thick, red] (1, -0.2)--(2.4, 1.2);
\foreach \a in {4.8, 5, 5.2, 5.4, 5.6, 5.8, 6, 6.2} \draw[ultra thick, red] (\a-0.1,-0.2)--(\a+0.1, 1.2);
\end{tikzpicture}
\end{center}

Our last removal will be the $4$ individual blocks, but again we don't have any individual blocks so we'll unbundle one of our bundles. This time we won't use the extra purple box since all of the individuals will be from that unbundled bundle.

\begin{center}
\begin{tikzpicture}
\foreach \x in {0, 1.2, 2.4, 3.6} \draw[thick, step=0.1] (\x,0) grid (\x+1, 1);
\foreach \x in {4.8, 5, 5.2, 5.4, 5.6, 5.8, 6, 6.2, 6.4, 6.6, 6.8, 7} \draw[thick, step=0.1] (\x, 0) grid (\x+0.101, 1);
\foreach \y in {0, 0.2, 0.4, 0.6, 0.8} \draw[thick, step=0.1] (7.2, \y)--(7.3, \y)--(7.3, \y+0.1)--(7.2, \y+0.1)--(7.2, \y);
\foreach \y in {0, 0.2, 0.4, 0.6, 0.8} \draw[thick, step=0.1] (7.4, \y)--(7.5, \y)--(7.5, \y+0.1)--(7.4, \y+0.1)--(7.4, \y);
\draw[ultra thick, red] (-0.2, -0.2)--(1.2, 1.2);
\draw[ultra thick, red] (1, -0.2)--(2.4, 1.2);
\foreach \a in {4.8, 5, 5.2, 5.4, 5.6, 5.8, 6, 6.2} \draw[ultra thick, red] (\a-0.1,-0.2)--(\a+0.1, 1.2);
\end{tikzpicture}
\end{center}

We are ready to remove the $4$ individual blocks with our big red lines.

\begin{center}
\begin{tikzpicture}
\foreach \x in {0, 1.2, 2.4, 3.6} \draw[thick, step=0.1] (\x,0) grid (\x+1, 1);
\foreach \x in {4.8, 5, 5.2, 5.4, 5.6, 5.8, 6, 6.2, 6.4, 6.6, 6.8, 7} \draw[thick, step=0.1] (\x, 0) grid (\x+0.101, 1);
\foreach \y in {0, 0.2, 0.4, 0.6, 0.8} \draw[thick, step=0.1] (7.2, \y)--(7.3, \y)--(7.3, \y+0.1)--(7.2, \y+0.1)--(7.2, \y);
\foreach \y in {0, 0.2, 0.4, 0.6, 0.8} \draw[thick, step=0.1] (7.4, \y)--(7.5, \y)--(7.5, \y+0.1)--(7.4, \y+0.1)--(7.4, \y);
\draw[ultra thick, red] (-0.2, -0.2)--(1.2, 1.2);
\draw[ultra thick, red] (1, -0.2)--(2.4, 1.2);
\foreach \a in {4.8, 5, 5.2, 5.4, 5.6, 5.8, 6, 6.2} \draw[ultra thick, red] (\a-0.1,-0.2)--(\a+0.1, 1.2);
\foreach \b in {0, 0.2}
	\foreach \a in {7.2, 7.4} 
		\draw[ultra thick, red] (\a-0.1, \b-0.1)--(\a+0.1, \b+0.1);
\end{tikzpicture}
\end{center}

Now that we are done removing things, we find the answer by counting the remaining amount. We see that we still have $\answer[given]{2}$ superbundles, $\answer[given]{4}$ bundles, and $\answer[given]{6}$ individual blocks, which means that the answer  $5.3 - 2.84$ can also be written as $\answer[given]{2.46}$. In other words, Raina will need to make $2.46$ liters of regular lemonade.


\end{explanation}
\end{question}

Notice how just like it was helpful to have the same whole for adding and subtracting fractions, it's useful to draw decimal numbers using the same value for one individual block. And pay attention to the way that pictures help us think about the problems we are solving in different ways than we might if we were just making calculations!


\begin{question}
In your own words, how do we see that addition and subtraction are the same operation no matter what kinds of numbers we are working with?
\begin{freeResponse}
Write your thoughts here!
\end{freeResponse}
\end{question}










\end{document}






