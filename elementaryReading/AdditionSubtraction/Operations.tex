\documentclass{ximera}


\graphicspath{
  {./}
  {graphics/}
  {../graphics/}
}

\usepackage{chngcntr}

\let\question\relax
\let\endquestion\relax




\newtheoremstyle{SlantTheorem}{\topsep}{\fill}%%% space between body and thm
%\newtheoremstyle{SlantTheorem}{\topsep}{\topsep}%%% space between body and thm
 {\slshape}                      %%% Thm body font
 {}                              %%% Indent amount (empty = no indent)
 {\bfseries\sffamily}            %%% Thm head font
 {}                              %%% Punctuation after thm head
 {3ex}                           %%% Space after thm head
 {\thmname{#1}\thmnumber{ #2}\thmnote{ \bfseries(#3)}}%%% Thm head spec
\theoremstyle{SlantTheorem}
\newtheorem{question}{Question}
\counterwithin*{question}{section}



\let\instructorNotes\relax
\let\endinstructorNotes\relax
%%% instructorNotes environment
\ifhandout
\newenvironment{instructorNotes}[1][false]%
{%
\def\givenatend{\boolean{#1}}\ifthenelse{\boolean{#1}}{\begin{trivlist}\item}{\setbox0\vbox\bgroup}{}
}
{%
\ifthenelse{\givenatend}{\end{trivlist}}{\egroup}{}
}
\else
\newenvironment{instructorNotes}[1][false]%
{%
  \ifthenelse{\boolean{#1}}{\begin{trivlist}\item[\hskip \labelsep\bfseries {\Large Instructor Notes: \\} \hspace{\textwidth} ]}
{\begin{trivlist}\item[\hskip \labelsep\bfseries {\Large Instructor Notes: \\} \hspace{\textwidth} ]}
{}
}
{\end{trivlist}}
\fi


%% Suggested Timing
\newcommand{\timing}[1]{{\bf Suggested Timing: \hspace{2ex}} #1}


\title{Operations}
\author{Jenny Sheldon}

\begin{document}

\begin{abstract}
We transition from talking about numbers to talking about operations.
\end{abstract}
\maketitle



\section{Operations and structure}

{\emph Note: many of the ideas in this section were influenced by a book called \link[Children's Mathematics]{https://www.heinemann.com/products/e05287.aspx} by Carpenter, et al. This book is sometimes used to teach Elementary Math Methods courses, so you may see it again later in your career. This book has many good examples of the way children think about solving problems like the ones in this section.}

Up until this point, we have been working with numbers one at a time, understanding what they mean and how to represent them. Now, we are going to turn our attention to operations, which require some different thinking. We want to recognize operations by what we will call their \dfn{structure}, or the underlying actions we are taking in order to solve problems. The reason that we want to focus on structure is that it helps us to know and describe why we choose a particular operation in a particular circumstance. For example, consider the following story problem.

\begin{question}
Alastair has $5$ pencils in his pencil box, and Blake has $8$ pencils in her pencil box. How many pencils do Alastair and Blake have all together?

\begin{multipleChoice}
\choice[correct]{We solve this problem by calculating $8+4$}.
\choice{We solve this problem by calculating $8-4$}.
\choice{We solve this problem by calculating $8 \times 4$}.
\choice{We solve this problem by calculating $8 \div 4$}.
\end{multipleChoice}
\end{question}

Okay, probably that wasn't the most difficult question you've ever solved, but let's take a step back. We would like you to be able to explain \emph{why} the correct answer is the correct one, as well as why the incorrect answers are incorrect. For instance, why is multiplication not the correct operation for the story above? By the end of this semester, we would like for you to be able to answer that question (though we aren't quite ready for it right now!) Especially when we have been working with stories like these for a long time, it can be difficult to express in words why we are making the choices that we are making. But these are skills we want you to have as teachers, so that you can help your future students understand how to make choices for themselves. We want solving math problems to be about more than guess-and-checking, but instead about confidently reasoning your way to the answer, and being sure that you've got the right thing.






\end{document}






