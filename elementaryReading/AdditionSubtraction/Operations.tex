\documentclass{ximera}

\usepackage{gensymb}
\usepackage{tabularx}
\usepackage{mdframed}
\usepackage{pdfpages}
%\usepackage{chngcntr}

\let\problem\relax
\let\endproblem\relax

\newcommand{\property}[2]{#1#2}




\newtheoremstyle{SlantTheorem}{\topsep}{\fill}%%% space between body and thm
 {\slshape}                      %%% Thm body font
 {}                              %%% Indent amount (empty = no indent)
 {\bfseries\sffamily}            %%% Thm head font
 {}                              %%% Punctuation after thm head
 {3ex}                           %%% Space after thm head
 {\thmname{#1}\thmnumber{ #2}\thmnote{ \bfseries(#3)}} %%% Thm head spec
\theoremstyle{SlantTheorem}
\newtheorem{problem}{Problem}[]

%\counterwithin*{problem}{section}



%%%%%%%%%%%%%%%%%%%%%%%%%%%%Jenny's code%%%%%%%%%%%%%%%%%%%%

%%% Solution environment
%\newenvironment{solution}{
%\ifhandout\setbox0\vbox\bgroup\else
%\begin{trivlist}\item[\hskip \labelsep\small\itshape\bfseries Solution\hspace{2ex}]
%\par\noindent\upshape\small
%\fi}
%{\ifhandout\egroup\else
%\end{trivlist}
%\fi}
%
%
%%% instructorIntro environment
%\ifhandout
%\newenvironment{instructorIntro}[1][false]%
%{%
%\def\givenatend{\boolean{#1}}\ifthenelse{\boolean{#1}}{\begin{trivlist}\item}{\setbox0\vbox\bgroup}{}
%}
%{%
%\ifthenelse{\givenatend}{\end{trivlist}}{\egroup}{}
%}
%\else
%\newenvironment{instructorIntro}[1][false]%
%{%
%  \ifthenelse{\boolean{#1}}{\begin{trivlist}\item[\hskip \labelsep\bfseries Instructor Notes:\hspace{2ex}]}
%{\begin{trivlist}\item[\hskip \labelsep\bfseries Instructor Notes:\hspace{2ex}]}
%{}
%}
%% %% line at the bottom} 
%{\end{trivlist}\par\addvspace{.5ex}\nobreak\noindent\hung} 
%\fi
%
%


\let\instructorNotes\relax
\let\endinstructorNotes\relax
%%% instructorNotes environment
\ifhandout
\newenvironment{instructorNotes}[1][false]%
{%
\def\givenatend{\boolean{#1}}\ifthenelse{\boolean{#1}}{\begin{trivlist}\item}{\setbox0\vbox\bgroup}{}
}
{%
\ifthenelse{\givenatend}{\end{trivlist}}{\egroup}{}
}
\else
\newenvironment{instructorNotes}[1][false]%
{%
  \ifthenelse{\boolean{#1}}{\begin{trivlist}\item[\hskip \labelsep\bfseries {\Large Instructor Notes: \\} \hspace{\textwidth} ]}
{\begin{trivlist}\item[\hskip \labelsep\bfseries {\Large Instructor Notes: \\} \hspace{\textwidth} ]}
{}
}
{\end{trivlist}}
\fi


%% Suggested Timing
\newcommand{\timing}[1]{{\bf Suggested Timing: \hspace{2ex}} #1}




\hypersetup{
    colorlinks=true,       % false: boxed links; true: colored links
    linkcolor=blue,          % color of internal links (change box color with linkbordercolor)
    citecolor=green,        % color of links to bibliography
    filecolor=magenta,      % color of file links
    urlcolor=cyan           % color of external links
}


\title{Operations}
\author{Jenny Sheldon}

\begin{document}

\begin{abstract}
We transition from talking about numbers to talking about operations.
\end{abstract}
\maketitle



\section{Operations and structure}


Up until this point, we have been working with numbers one at a time, understanding what they mean and how to represent them. Now, we are going to turn our attention to operations, which require a different way of thinking. We want to recognize operations by their \dfn{structure}, or the underlying physical actions we are taking in order to solve problems. The reason that we focus on structure is that it helps us to know and describe why we choose a particular operation in a particular circumstance. For example, consider the following story problem.

\begin{question}
Alastair has $5$ pencils in his pencil box, and Blake has $8$ pencils in her pencil box. How many pencils do Alastair and Blake have all together?

\begin{multipleChoice}
\choice[correct]{We solve this problem by calculating $8+5$}.
\choice{We solve this problem by calculating $8-5$}.
\choice{We solve this problem by calculating $8 \times 5$}.
\choice{We solve this problem by calculating $8 \div 5$}.
\end{multipleChoice}
\end{question}

Let's take a step back. By the end of the semester, you should be able to explain both why the correct answer is the correct one and why the incorrect answers are incorrect. For instance, why is multiplication not the correct operation for the story above? Especially when we have been working with stories like these for a long time, it can be difficult to express in words why we make certain choices. But these are skills that are necessary for teachers, so that you can help your future students understand how to make choices for themselves. We want solving math problems to be about more than guess-and-checking, but instead about confidently reasoning your way to the answer, and being sure that you've got the right thing.

Furthermore, one of the \link[standards for mathematical practice]{https://education.ohio.gov/getattachment/Topics/Learning-in-Ohio/Mathematics/Model-Curricula-in-Mathematics/Standards-for-Mathematical-Practice/Standards-for-Mathematical-Practice.pdf.aspx} is to look for and make use of structure. These standards for mathematical practice are overarching skills that children should learn alongside the specific content at every grade level, so we will also pay attention to them throughout the course. Here is the full list of standards for mathematical practice.
\begin{itemize}
\item Make sense of problems and persevere in solving them.
\item Reason abstractly and quantitatively.
\item Construct viable arguments and critique the reasoning of others.
\item Model with mathematics.
\item Use appropriate tools strategically.
\item Attend to precision.
\item Look for an make use of structure.
\item Look for and express regularity in repeated reasoning.
\end{itemize}
These practices are key for good explanations in this course!



\end{document}






