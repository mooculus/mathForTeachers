\documentclass{ximera}

\usepackage{gensymb}
\usepackage{tabularx}
\usepackage{mdframed}
\usepackage{pdfpages}
%\usepackage{chngcntr}

\let\problem\relax
\let\endproblem\relax

\newcommand{\property}[2]{#1#2}




\newtheoremstyle{SlantTheorem}{\topsep}{\fill}%%% space between body and thm
 {\slshape}                      %%% Thm body font
 {}                              %%% Indent amount (empty = no indent)
 {\bfseries\sffamily}            %%% Thm head font
 {}                              %%% Punctuation after thm head
 {3ex}                           %%% Space after thm head
 {\thmname{#1}\thmnumber{ #2}\thmnote{ \bfseries(#3)}} %%% Thm head spec
\theoremstyle{SlantTheorem}
\newtheorem{problem}{Problem}[]

%\counterwithin*{problem}{section}



%%%%%%%%%%%%%%%%%%%%%%%%%%%%Jenny's code%%%%%%%%%%%%%%%%%%%%

%%% Solution environment
%\newenvironment{solution}{
%\ifhandout\setbox0\vbox\bgroup\else
%\begin{trivlist}\item[\hskip \labelsep\small\itshape\bfseries Solution\hspace{2ex}]
%\par\noindent\upshape\small
%\fi}
%{\ifhandout\egroup\else
%\end{trivlist}
%\fi}
%
%
%%% instructorIntro environment
%\ifhandout
%\newenvironment{instructorIntro}[1][false]%
%{%
%\def\givenatend{\boolean{#1}}\ifthenelse{\boolean{#1}}{\begin{trivlist}\item}{\setbox0\vbox\bgroup}{}
%}
%{%
%\ifthenelse{\givenatend}{\end{trivlist}}{\egroup}{}
%}
%\else
%\newenvironment{instructorIntro}[1][false]%
%{%
%  \ifthenelse{\boolean{#1}}{\begin{trivlist}\item[\hskip \labelsep\bfseries Instructor Notes:\hspace{2ex}]}
%{\begin{trivlist}\item[\hskip \labelsep\bfseries Instructor Notes:\hspace{2ex}]}
%{}
%}
%% %% line at the bottom} 
%{\end{trivlist}\par\addvspace{.5ex}\nobreak\noindent\hung} 
%\fi
%
%


\let\instructorNotes\relax
\let\endinstructorNotes\relax
%%% instructorNotes environment
\ifhandout
\newenvironment{instructorNotes}[1][false]%
{%
\def\givenatend{\boolean{#1}}\ifthenelse{\boolean{#1}}{\begin{trivlist}\item}{\setbox0\vbox\bgroup}{}
}
{%
\ifthenelse{\givenatend}{\end{trivlist}}{\egroup}{}
}
\else
\newenvironment{instructorNotes}[1][false]%
{%
  \ifthenelse{\boolean{#1}}{\begin{trivlist}\item[\hskip \labelsep\bfseries {\Large Instructor Notes: \\} \hspace{\textwidth} ]}
{\begin{trivlist}\item[\hskip \labelsep\bfseries {\Large Instructor Notes: \\} \hspace{\textwidth} ]}
{}
}
{\end{trivlist}}
\fi


%% Suggested Timing
\newcommand{\timing}[1]{{\bf Suggested Timing: \hspace{2ex}} #1}




\hypersetup{
    colorlinks=true,       % false: boxed links; true: colored links
    linkcolor=blue,          % color of internal links (change box color with linkbordercolor)
    citecolor=green,        % color of links to bibliography
    filecolor=magenta,      % color of file links
    urlcolor=cyan           % color of external links
}

\title{Dilations}
\author{Jenny Sheldon}

\begin{document}

\begin{abstract}
We investigate a transformation that is not a rigid motion.
\end{abstract}
\maketitle

When we defined transformations a few sections ago, we actually left one of them out. (I'm sorry to have deceived you that way, but it seemed like we had enough to digest with transformations that were also rigid motions. Please decide for yourself whether we should have put them all together!) So, let's start by updating our definition for a transformation.
\begin{definition}
We use the term \dfn{transformation} to refer to any of the following: a reflection, a rotation, a translation, or a dilation.
\end{definition}
We've added \dfn{dilations} to the list of transformations, so let's start by talking about how we want to think about these transformations and then make a more precise definition.

The first thing to know about dilations is that they still preserve angles (hence why they still fall into our category of ``transformations''). So, if I start with an $80\degree$ angle, and then I dilate that angle, I'm going to end with an $\answer[given]{80}\degree$ angle. 

The second thing to know about dilations is that they are not rigid motions. In other words, they don't preserve distances. So if I have a shape with a side length of $4$cm, and I dilate that shape, the side length might not be $4$cm anymore. Remember that reflections, rotations, and translations do preserve distances, so this is a big difference between dilations and our other transformations. 

\begin{question}
If I start with a square whose side lengths are each $1$ inch and then I dilate this square, which of the following will have to be true after the dilation? Select all that apply.
\begin{selectAll}
\choice[correct]{The angles will all be $90$ degrees}
\choice{The sides will still be $1$ inch}
\choice[correct]{The shape will still have four sides}
\choice[correct]{The shape will still be a square}
\choice{The shape must be in the same location as it started}
\end{selectAll}
\end{question}

Now that we know how dilations compare and contrast to our other transformations, how should we think about them? A dilation is generally a stretching or a shrinking. Think about zooming in or out on your phone or tablet, and you've got a dilation.

\begin{definition}
A \dfn{dilation} is what happens to all the points in the plane when we choose a particular point, which we can call $O$, as well as a particular distance, which we call $r$, and we move each point $P$ in the plane such that the original distance between $O$ and $P$ is multiplied by $r$.
\end{definition}

As we've done previously, let's see how dilations work by considering a few examples. 

\begin{example}
Let's start by choosing a point $O$, marked on the figure below, and a distance of $r=2$. We'll also choose a point $P$ and see what happens to this point when we dilate using $O$ and $r$.
\begin{image}
\begin{tikzpicture}
\draw[fill=black] (0,0) circle (2pt) node[below]{$O$};
\draw[fill=black] (1,2) circle (2pt) node[above]{$P$};
\end{tikzpicture}
\end{image}
Our goal is to move $P$ twice as far from $O$ as it currently is. So let's start by drawing a dashed line from $O$ to $P$.
\begin{image}
\begin{tikzpicture}
\draw[fill=black] (0,0) circle (2pt) node[below]{$O$};
\draw[fill=black] (1,2) circle (2pt) node[above left]{$P$};
\draw[thick, dashed, ->] (0,0)--(3,6);
\end{tikzpicture}
\end{image}
Next, let's measure the distance between $O$ and $P$ with our ruler, or open our compass to that distance. Then we mark $\answer[given]{2}$ times that measurement along the line between $O$ and $P$ so that we move the point $2$ times as far from $O$.
\begin{image}
\begin{tikzpicture}
\draw[fill=black] (0,0) circle (2pt) node[below]{$O$};
\draw[fill=black] (1,2) circle (2pt) node[above left ]{$P$};
\draw[thick, dashed, ->] (0,0)--(3,6);
\draw[fill=black] (2, 4) circle (2pt) node[above left]{$\hat{P}$};
\end{tikzpicture}
\end{image}
This new point, $\hat{P}$, is the image of $P$ under this dilation.

\end{example}

\begin{question}
If the point $P$ was the same as the point $O$, where would $P$ move when we dilate?
\begin{multipleChoice}
	\choice{Above the point $O$}
	\choice[correct]{It would not move}
	\choice{Below the point $O$}
	\choice{None of the above}
\end{multipleChoice}
\end{question}

Next, let's look at what happens when we dilate an entire shape.

\begin{example}
This time, let's start with a point $O$ that we will mark in the image below and choose a value of $r=0.3$. The figure that we'll dilate will be a square. 
\begin{image}
\begin{tikzpicture}
\draw[fill=black] (0,0) circle (2pt) node[below]{$O$};
\draw[thick] (-4,4)--(-4,5)--(-5,5)--(-5,4)--(-4,4);
\draw[fill=black] (-4,4) circle (2pt) node[below] {$A$};
\draw[fill=black] (-4,5) circle (2pt) node[above] {$B$};
\draw[fill=black] (-5,5) circle (2pt) node[above] {$C$};
\draw[fill=black] (-5,4) circle (2pt) node[below] {$D$};
\end{tikzpicture}
\end{image}
Like we have done previously, let's dilate each vertex of the shape. You should imagine also dilating each point on each segment between the vertices! We start by drawing a dashed line between $O$ and each vertex.
\begin{image}
\begin{tikzpicture}
\draw[fill=black] (0,0) circle (2pt) node[below]{$O$};
\draw[thick] (-4,4)--(-4,5)--(-5,5)--(-5,4)--(-4,4);
\draw[fill=black] (-4,4) circle (2pt) node[below] {$A$};
\draw[fill=black] (-4,5) circle (2pt) node[above] {$B$};
\draw[fill=black] (-5,5) circle (2pt) node[above] {$C$};
\draw[fill=black] (-5,4) circle (2pt) node[below] {$D$};
\draw[thick, dashed, ->] (0,0)--(-5.5,5.5);
\draw[thick, dashed, ->] (0,0)--(-4.53, 5.663);
\draw[thick, dashed, ->] (0,0)--(-6, 4.8);
\end{tikzpicture}
\end{image}
Next, we measure the distance between $O$ and each point, multiply that distance by $r=0.3$, and mark the new distance on the line between $O$ and that point. We use a hat to distinguish the image from the original point as usual.
\begin{image}
\begin{tikzpicture}
\draw[fill=black] (0,0) circle (2pt) node[below]{$O$};
\draw[thick] (-4,4)--(-4,5)--(-5,5)--(-5,4)--(-4,4);
\draw[fill=black] (-4,4) circle (2pt) node[below] {$A$};
\draw[fill=black] (-4,5) circle (2pt) node[above] {$B$};
\draw[fill=black] (-5,5) circle (2pt) node[above] {$C$};
\draw[fill=black] (-5,4) circle (2pt) node[below] {$D$};
\draw[thick, dashed, ->] (0,0)--(-5.5,5.5);
\draw[thick, dashed, ->] (0,0)--(-4.53, 5.663);
\draw[thick, dashed, ->] (0,0)--(-6, 4.8);
\draw[fill=black] (-2.8, 2.8) circle (2pt) node[below]{$\hat{A}$};
\draw[fill=black] (-3.5, 3.5) circle (2pt) node[above]{$\hat{C}$};
\draw[fill=black] (-2.8, 3.5) circle (2pt) node[above right]{$\hat{B}$};
\draw[fill=black] (-3.5, 2.8) circle (2pt) node[below]{$\hat{D}$};
\draw[thick] (-2.8, 2.8)--(-2.8, 3.5)--(-3.5, 3.5)--(-3.5, 2.8)--(-2.8,2.8);
\end{tikzpicture}
\end{image}




\end{example}
Now that we've actually dilated a square, go back and look at your answers to the first question. Do they still make sense?

\begin{question}
True or false: using only one dilation, we can turn a figure sideways.
\begin{multipleChoice}
\choice{True}
\choice[correct]{False}
\end{multipleChoice}
\begin{feedback}
To turn a figure, we should generally think about using a rotation!
\end{feedback}
\end{question}

There are plenty of properties of dilations that we can explore, using things like rubber bands or dynamic geometry software. If you'd like to explore these things on your own, the Side-Splitter Theorem for triangles is an interesting consequence of these properties!

\end{document}
