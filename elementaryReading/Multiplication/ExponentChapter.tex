\documentclass{ximera}
 

\graphicspath{
  {./}
  {graphics/}
  {../graphics/}
}

\usepackage{chngcntr}

\let\question\relax
\let\endquestion\relax




\newtheoremstyle{SlantTheorem}{\topsep}{\fill}%%% space between body and thm
%\newtheoremstyle{SlantTheorem}{\topsep}{\topsep}%%% space between body and thm
 {\slshape}                      %%% Thm body font
 {}                              %%% Indent amount (empty = no indent)
 {\bfseries\sffamily}            %%% Thm head font
 {}                              %%% Punctuation after thm head
 {3ex}                           %%% Space after thm head
 {\thmname{#1}\thmnumber{ #2}\thmnote{ \bfseries(#3)}}%%% Thm head spec
\theoremstyle{SlantTheorem}
\newtheorem{question}{Question}
\counterwithin*{question}{section}



\let\instructorNotes\relax
\let\endinstructorNotes\relax
%%% instructorNotes environment
\ifhandout
\newenvironment{instructorNotes}[1][false]%
{%
\def\givenatend{\boolean{#1}}\ifthenelse{\boolean{#1}}{\begin{trivlist}\item}{\setbox0\vbox\bgroup}{}
}
{%
\ifthenelse{\givenatend}{\end{trivlist}}{\egroup}{}
}
\else
\newenvironment{instructorNotes}[1][false]%
{%
  \ifthenelse{\boolean{#1}}{\begin{trivlist}\item[\hskip \labelsep\bfseries {\Large Instructor Notes: \\} \hspace{\textwidth} ]}
{\begin{trivlist}\item[\hskip \labelsep\bfseries {\Large Instructor Notes: \\} \hspace{\textwidth} ]}
{}
}
{\end{trivlist}}
\fi


%% Suggested Timing
\newcommand{\timing}[1]{{\bf Suggested Timing: \hspace{2ex}} #1}
 
 
\title{Exponents}
\author{Scott Zimmerman}
 
\begin{document}
 
\begin{abstract}
We understand the meaning of raising a counting number to an integral power.
\end{abstract}
\maketitle
 
\section{Activity for this section:}
Properties of Powers, 
Integer Powers


 
\section{Exponents}

Recall that multiplication can be used to simplify addition. For example, if we have 3 groups with 5 objects in each group, when we have ``5 objects'' 3 times or
$$
5 + 5 + 5 = 15.
$$
We simplify this idea by saying that there are
$$
3 \times 5 = 15
$$
total objects.

We can do a similar technique to simplify multiplication.
First, consider the expression 
$$
5 \times 5 \times 5.
$$


\begin{question}
Which of the following questions could be answered by calculating $5 \times 5 \times 5$?
 
\begin{multipleChoice}
\choice{I have 5 trout, 5 carp, and 5 tuna. How many total fish do I have?}
\choice{I wrote 125 letters and put 5 letters in each envelope. How many envelopes did I use?}
\choice[correct]{I have 5 crates. Each crate contains 5 boxes of markers, and each box contains 5 markers. How many total markers do I have?}
\choice{None of these.}
\end{multipleChoice}
\end{question}

You may remember that a shorter way of writing $5 \times 5 \times 5$ is by writing $5^3$ read as ``5 to the power of 3'' or ``five to the third''.

\begin{definition}
For counting numbers $A$ and $B$, the expression $A^B$ is equal to 
$$
\underbrace{A \times A \times \cdots \times A}_{B \text{ copies}}.
$$
We call $A$ the \dfn{base} and $B$ the \dfn{power} or \dfn{exponent}.
\end{definition}

\begin{question}
Evaluate $5^3$.
 
\begin{prompt}
$\answer{125}$
\end{prompt}
\end{question}

\begin{question}
Rewrite $7^4$ as a product.
 
\begin{prompt}
$\answer{7} \times \answer{7} \times \answer{7} \times \answer{7}$
\end{prompt}
\end{question}

\section{Adding powers}

Sometimes, we see addition inside a power. For example, we might see $4^{2+3}$. Where does this come from? Let's explore!

\begin{example}
Let's understand the meaning of $4^2\times4^3$.
Well, $4^2$ is equal to $\answer[given]{16}$ and $4^3$ is equal to $\answer[given]{64}$.
This means that $(4^2)\times(4^3)$ represents the total number of objects in $\answer[given]{16}$ groups with $\answer[given]{64}$ objects in each group.

As we saw above, $4^2$ is also equal to
\begin{prompt}
    $\answer{4} \times \answer{4}$
\end{prompt},
and $4^3$ is equal to 
\begin{prompt}
    $\answer{4} \times \answer{4} \times \answer{4}$
\end{prompt}.
Using the \wordChoice{\choice[correct]{associative} \choice{commutative} \choice{distributive}} property of multiplication, we see that
$$
4^2 \times 4^3 = (4 \times 4) \times (4 \times 4 \times 4) = 4 \times 4 \times 4 \times 4 \times 4.
$$
This last product is equal to 
\begin{prompt}
    $4^{\answer{5}}$
\end{prompt}. We ended up with 5 ``4''s on the right hand side because we put together 2 ``4''s and 3 ``4''s. That is, we \wordChoice{\choice[correct]{added} \choice{subtracted} \choice{multiplied} \choice{divided}} 2 and 3 to find the total number of ``4''s we have.
In the end, we see that 
\begin{prompt}
$$
4^2\times4^3 = 4^{\text{\wordChoice{\choice[correct]{$2+3$} \choice{$2-3$} \choice{$2\times3$} \choice{$2 \div 3$}}}}.
$$
\end{prompt}
\end{example}


\begin{definition}
\label{addtomult}
For counting numbers $A$, $B$, and $C$, the expression $A^{B+C}$ is equal to 
$
A^B \times A^C.
$
\end{definition}

\begin{question}
Rewrite $7^{3+5}$ as a product.
 
\begin{prompt}
$7^{\answer{3}}\times 7^{\answer{5}}$
\end{prompt}
\end{question}

\begin{question}
Rewrite $6^2 \times 6^7$ as a single base to one power. (Do not include ``$+$'' in your answer!)
 
\begin{prompt}
$\answer{6}^{\answer{9}}$
\end{prompt}
\end{question}






\section{Multiplying Powers}

We've discussed addition inside a power. What about multiplication? Why might we end up with $4^{3 \times 2}$? Where does this come from? Let's explore!

\begin{example}
Let's understand the meaning of $(4^2)^3$.
Again, $4^2$ is equal to
\begin{prompt}
    $\answer{4} \times \answer{4}$.
\end{prompt}
Raising $4^2$ to the power of 3 means multiplying together $3$ copies of  $4^2$.
That is, $(4^2)^3$ is equal to 
\begin{prompt}
    $$
    (4^2)^3 = (\answer{4} \times \answer{4})^3
    =
    (\answer{4} \times \answer{4}) \times (\answer{4} \times \answer{4}) \times (\answer{4} \times \answer{4}).
    $$
\end{prompt}
According to the grouping pictured here, we have 
\begin{prompt}
    $\answer{3}$
\end{prompt}
groups with
\begin{prompt}
    $\answer{2}$
\end{prompt}
``4''s in each group. In other words, we have 
\wordChoice{\choice{$3+2$} \choice{$3-2$} \choice[correct]{$3\times2$} \choice{$3\div2$}}
total ``4''s multiplied together!
\end{example}



\begin{definition}
\label{powtoprod}
For counting numbers $A$, $B$, and $C$, the expression $A^{B\times C}$ is equal to 
$
(A^C)^B 
$
or 
$
(A^B)^C
$.
\end{definition}

\begin{question}
Which of the following is equal to $7^{3\times 5}$?
 
\begin{multipleChoice}
\choice{$7^{3+5}$}
\choice[correct]{$(7^5)^3$}
\choice{$\frac{7^3}{7^5}$}
\choice{$7^3 \times 7^5$}
\end{multipleChoice}
\end{question}

\begin{question}
Rewrite $(6^2)^7$ using a single power. (Do not include ``$\times$'' in your answer!)
 
\begin{prompt}
$6^{\answer{14}}$
\end{prompt}
\end{question}



\section{Power of 0}

What if the number in the exponent is 0? For example, what is the meaning of the expression $4^0$?
We'd like the ``multiplication to addition'' property of exponents (like the one in Definition~\ref{addtomult}) to always be true.
If that's true, then let's explore the meaning behind $4^0$.

\begin{example}
    If the ``multiplication to addition'' property of exponents is true, what should $4^3 \times 4^0$ be equal to?
    \begin{prompt}
        $4^{\answer{3}}$
    \end{prompt}

    In other words, when we multiply a number by $4^0$, we should just just end up with the number we started with. What special number can we multiply a starting number by so that it doesn't change?
    \begin{prompt}
        $\answer{1}$
    \end{prompt}
\end{example}

\begin{definition}
    For a counting number $A$,
    $$
    A^0 = 1.
    $$
\end{definition}





\section{Integer powers}


We've explored addition and multiplication within an exponent. What about subtraction?
In particular, let's give meaning to an expression like ``$5^{-2}$''.
Again, we'd like the ``multiplication to addition'' property of exponents (like the one in Definition~\ref{addtomult}) to always be true.

\begin{example}
    If the ``multiplication to addition'' property of exponents is true, what should $5^2 \times 5^{-2}$ be equal to?
    \begin{prompt}
        $5^{\answer{0}}$
    \end{prompt}
    And what single number is this expression equal to?
    \begin{prompt}
        $\answer{1}$
    \end{prompt}

    In other words, when we multiply $5^2$ by $5^{-2}$, we should end up with 1. Note that $5^2 = 25$. Combining these two facts, how should we define $5^{-2}$?
    
    \begin{prompt}
        $5^{-2} = \frac{1}{\answer{25}}$
    \end{prompt}
\end{example}

\begin{definition}
    For counting numbers $A$ and $B$,
    $$
    A^{-B} = \frac{1}{A^B}.
    $$
\end{definition}

\begin{question}
    Rewrite $4^{-3}$ as a fraction without exponents.

    \begin{prompt}
        $\frac{1}{\answer{64}}$
    \end{prompt}
\end{question}

\begin{question}
    Rewrite $6^{-1}$ as a fraction without exponents.

    \begin{prompt}
        $\frac{1}{\answer{6}}$
    \end{prompt}
\end{question}


\begin{question}
    Rewrite $\frac{1}{36}$ using a negative exponent.

    \begin{prompt}
        $6^{\answer{-2}}$
    \end{prompt}
\end{question}




\end{document}