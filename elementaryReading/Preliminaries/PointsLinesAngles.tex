\documentclass{ximera}

\usepackage{gensymb}
\usepackage{tabularx}
\usepackage{mdframed}
\usepackage{pdfpages}
%\usepackage{chngcntr}

\let\problem\relax
\let\endproblem\relax

\newcommand{\property}[2]{#1#2}




\newtheoremstyle{SlantTheorem}{\topsep}{\fill}%%% space between body and thm
 {\slshape}                      %%% Thm body font
 {}                              %%% Indent amount (empty = no indent)
 {\bfseries\sffamily}            %%% Thm head font
 {}                              %%% Punctuation after thm head
 {3ex}                           %%% Space after thm head
 {\thmname{#1}\thmnumber{ #2}\thmnote{ \bfseries(#3)}} %%% Thm head spec
\theoremstyle{SlantTheorem}
\newtheorem{problem}{Problem}[]

%\counterwithin*{problem}{section}



%%%%%%%%%%%%%%%%%%%%%%%%%%%%Jenny's code%%%%%%%%%%%%%%%%%%%%

%%% Solution environment
%\newenvironment{solution}{
%\ifhandout\setbox0\vbox\bgroup\else
%\begin{trivlist}\item[\hskip \labelsep\small\itshape\bfseries Solution\hspace{2ex}]
%\par\noindent\upshape\small
%\fi}
%{\ifhandout\egroup\else
%\end{trivlist}
%\fi}
%
%
%%% instructorIntro environment
%\ifhandout
%\newenvironment{instructorIntro}[1][false]%
%{%
%\def\givenatend{\boolean{#1}}\ifthenelse{\boolean{#1}}{\begin{trivlist}\item}{\setbox0\vbox\bgroup}{}
%}
%{%
%\ifthenelse{\givenatend}{\end{trivlist}}{\egroup}{}
%}
%\else
%\newenvironment{instructorIntro}[1][false]%
%{%
%  \ifthenelse{\boolean{#1}}{\begin{trivlist}\item[\hskip \labelsep\bfseries Instructor Notes:\hspace{2ex}]}
%{\begin{trivlist}\item[\hskip \labelsep\bfseries Instructor Notes:\hspace{2ex}]}
%{}
%}
%% %% line at the bottom} 
%{\end{trivlist}\par\addvspace{.5ex}\nobreak\noindent\hung} 
%\fi
%
%


\let\instructorNotes\relax
\let\endinstructorNotes\relax
%%% instructorNotes environment
\ifhandout
\newenvironment{instructorNotes}[1][false]%
{%
\def\givenatend{\boolean{#1}}\ifthenelse{\boolean{#1}}{\begin{trivlist}\item}{\setbox0\vbox\bgroup}{}
}
{%
\ifthenelse{\givenatend}{\end{trivlist}}{\egroup}{}
}
\else
\newenvironment{instructorNotes}[1][false]%
{%
  \ifthenelse{\boolean{#1}}{\begin{trivlist}\item[\hskip \labelsep\bfseries {\Large Instructor Notes: \\} \hspace{\textwidth} ]}
{\begin{trivlist}\item[\hskip \labelsep\bfseries {\Large Instructor Notes: \\} \hspace{\textwidth} ]}
{}
}
{\end{trivlist}}
\fi


%% Suggested Timing
\newcommand{\timing}[1]{{\bf Suggested Timing: \hspace{2ex}} #1}




\hypersetup{
    colorlinks=true,       % false: boxed links; true: colored links
    linkcolor=blue,          % color of internal links (change box color with linkbordercolor)
    citecolor=green,        % color of links to bibliography
    filecolor=magenta,      % color of file links
    urlcolor=cyan           % color of external links
}

\title{Points, Lines, and Angles}
\author{Jenny Sheldon}

\begin{document}

\begin{abstract}
We introduce some basic objects in geometry.
\end{abstract}
\maketitle

\subsection{Points and Lines}

We've already been talking about points and lines, and you probably already have a good idea about what we mean when we talk about these objects: in the figure below, $P$ is a point, and $L$ is a line. 
\begin{center}
	\begin{tikzpicture}
		\draw[thick, <->] (0,0)--(5,2) node[below] {$L$};
		\draw[fill=black] (2.5, 1) circle (2pt) node[above]{$P$};
	\end{tikzpicture}
\end{center}

If you are expecting a definition here, you might be surprised: points and lines actually don't have definitions. The mental images you have for these objects are just about the best we can do. In mathematics, we have some ideas that we essentially have to take for granted. We try to minimize the number of those ideas, but we will still have a few. If we tried to define points and lines, we would run into trouble pretty quickly! (Pause and try to do it yourself to see how tough it is!)

However, points and lines do have some important properties that we should consider.

While we usually draw a little circle to represent a point (like $P$ above), we have to remember that a point doesn't have any thickness at all. This is hard to imagine, but if you were our friend the little bug, and your entire world was the point $P$ above, you actually couldn't move at all. In fact, this means that points are $0$-dimensional, because there are $\answer[given]{0}$ different directions that the bug could move.

Also, in most kinds of geometry, we can identify points anywhere we like. So on the line $M$ below, we can identify points $Q$, $R$, and $S$, as well as many, many more points. 
\begin{center}
	\begin{tikzpicture}
		\draw[thick, <->] (0,0)--(4,-4) node[below] {$M$};
		\draw[fill=black] (0.2, -0.2) circle (2pt) node[above right]{$Q$};
		\draw[fill=black] (2.6, -2.6) circle (2pt) node[above]{$R$};
		\draw[fill=black] (3.15, -3.15) circle (2pt) node[above]{$S$};
	\end{tikzpicture}
\end{center}
(Wait: there are more kinds of geometry??!? Yes! But that's a story for another day.)

Lines extend forever in each direction, and we usually indicate that by placing arrows at each end. We can also draw a \dfn{ray}, which extends forever in one direction, or a \dfn{line segment} which has a definite length. In the figure below, $\overrightarrow{AB}$ is a ray, and $CD$ is a line segment.
\begin{center}
	\begin{tikzpicture}
		\draw[thick, <-] (0,0)--(2,4);
		\draw[fill=black] (2, 4) circle (2pt) node[above]{$A$};
		\draw[fill=black] (0.5, 1) circle (2pt) node[above left]{$B$};
		\draw[thick] (6,4)--(9,2);
		\draw[fill=black] (6,4) circle (2pt) node[above]{$C$};
		\draw[fill=black] (9,2) circle (2pt) node[above]{$D$};
	\end{tikzpicture}
\end{center}
Notice that we can name either a ray or a line segment using two points. We can do the same for lines.

Lines are also $\answer[given]{1}$-dimensional, meaning they also have no thickness. However, in this case, our little bug could move in the forward-back direction, just not right-left.

We have drawn all of our lines so far to be straight lines, and most of the time when we refer to a line we will want it to be straight. When we refer to curved lines, we will usually call them curves, but it's best to be specific most of the time!

Finally, we can tallk about a {\dfn plane}, which is like a piece of paper with no thickness at all that extends in every direction. It's easiest to visualize planes in three-dimensional space, and we typically draw a piece of the plane like in the image below.
\begin{center}
	\begin{tikzpicture}
		\draw[thick, dotted] (0,0)--(3,1)--(3,6)--(0,5)--(0,0);
	\end{tikzpicture}
\end{center}
Imagine that the edges drawn aren't actually edges, but keep extending forever. Even though this plane is drawn in three-dimensional space, the plane itself is $\answer[given]{2}$-dimensional, because if our friend the little bug lived on this plane, it could move forward-back and also right-left as well as all of the combinations of those two directions. But since the plane has no thickness, the little bug couldn't fly around in an up-down direction.

\begin{question}
Pause to think: Where could you draw a line on the plane above? Where you could locate a point on the plane? Could you locate a point on a line on the plane?
\begin{freeResponse} \end{freeResponse}
\end{question}


\subsection{Angles}

For angles, we have two different definitions. The first we can call the two rays definition.

\begin{definition}
An \dfn{angle}  is formed when two rays meet at a common endpoint, called the \dfn{vertex}.
\end{definition}

Our second definition we can call the turning definition.

\begin{definition}
An \dfn{angle}  is a fraction of a turn. The point around which the turning happens is called the \dfn{vertex}.
\end{definition}

\begin{center}
	\begin{tikzpicture}
		\draw[thick, <->] (4,0)--(0,0)--(4,2);
	\end{tikzpicture}
\end{center}

\begin{question}
Pause to think: how are the two definitions for an angle related to one another?
\begin{freeResponse} \end{freeResponse}
\end{question}

We can also indicate points on the rays that make up our angle, and name the angle according to those rays. The middle point should label the vertex as in $\angle ABC$ below.
\begin{center}
	\begin{tikzpicture}
		\draw[thick, <->] (4,0)--(0,0)--(4,2);
		\draw[fill=black] (3.5, 1.75) circle (2pt) node[above left] {$A$};
		\draw[fill=black] (0,0) circle (2pt) node[left]{$B$};
		\draw[fill=black] (3.5,0) circle (2pt) node[below]{$C$};
	\end{tikzpicture}
\end{center}
If the points $A$ and $C$ are clear or not labeled, we could also call the above angle $\angle B$ using just the vertex label.

We talk about the size or \dfn{measure} of an angle by first deciding what the size or measure of one full rotation should be. The most common measure for one full turn is to use $360$ degrees (usually written $360^{\circ}$. From the measure of a full turn, we can either use the fraction of a turn that we have to measure an angle, or we can fill up the angle with copies of another angle whose measure we know.
\begin{itemize}
	\item A half-turn would measure $\answer[given]{180}$ degrees, because it's half of 360 degrees.
	\item A quarter-turn would measure $\answer[given]{90}$ degrees, because it's a quarter of 360 degrees.
	\item A turn which is $\frac{1}{360}$ of a full turn would measure $\answer[given]{1}$ degree.
\end{itemize}

\begin{question}
In the image below, each angle measures $1^{\circ}$. What is the total measure of this angle?
\begin{center}
	\begin{tikzpicture}[scale=10]
  \coordinate (O) at (0,0);
  \coordinate (A) at (0:1cm);
  \foreach \x in {1,2,...,6} {
    \draw (O) -- (\x*1:1cm);
    %\draw (\x*1:0.3cm) arc (\x*1:\x*1+1:0.3cm);
  }
\end{tikzpicture}
\end{center}

The angle measures $\answer[given]{5}^{\circ}$.
\end{question}

However, there are other ways to measure a full turn. Another common way is to call a full turn $2\pi$ radians, so that a half-turn is $\pi$ radians and a quarter-turn is $\frac{\pi}{2}$ radians. Or, you could invent your own unit for angle measures: say that you wanted a full turn to measure $50$ angle-units. In that case, a half-turn would measure $25$ angle-units, and a quarter turn would measure $12.5$ angle-units. One fiftieth of a turn would measure $1$ angle-unit, and from there you could measure angles of $27.3$ angle-units as well as others.

We will measure angles using degrees (so that there are $360$ degrees in a full turn) unless we say otherwise.  A useful tool to measure angles is called a protractor.

VIDEO for using a protractor.

Here is a common misconception about angles.
\begin{question}
Which of the following angle has a larger measure?
\begin{center}
	\begin{tikzpicture}
	\draw[thick, <->] (4,0)--(0,0)--(4, 4.8);
	\node at (-0.2, -0.2) {$A$};
	\draw[thick, <->] (6,0)--(5,0)--(6,1.2);
	\node at (4.8, -0.2) {$B$};
	\end{tikzpicture}
\end{center}

\begin{multipleChoice}
\choice{Angle $A$}
\choice{Angle $B$}
\choice[correct] They have the same measure.
\end{multipleChoice}
\end{question}

Angles are used to describe many different phenomena in the world as well as in mathematics. We'll see them in many places throughout the course, and we hope you see them in many places in your every-day life as well!

\end{document}
