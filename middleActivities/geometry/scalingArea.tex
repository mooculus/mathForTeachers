%\documentclass[handout]{ximera}
\documentclass{ximera}


\graphicspath{
  {./}
  {graphics/}
  {../graphics/}
}

\usepackage{chngcntr}

\let\question\relax
\let\endquestion\relax




\newtheoremstyle{SlantTheorem}{\topsep}{\fill}%%% space between body and thm
%\newtheoremstyle{SlantTheorem}{\topsep}{\topsep}%%% space between body and thm
 {\slshape}                      %%% Thm body font
 {}                              %%% Indent amount (empty = no indent)
 {\bfseries\sffamily}            %%% Thm head font
 {}                              %%% Punctuation after thm head
 {3ex}                           %%% Space after thm head
 {\thmname{#1}\thmnumber{ #2}\thmnote{ \bfseries(#3)}}%%% Thm head spec
\theoremstyle{SlantTheorem}
\newtheorem{question}{Question}
\counterwithin*{question}{section}



\let\instructorNotes\relax
\let\endinstructorNotes\relax
%%% instructorNotes environment
\ifhandout
\newenvironment{instructorNotes}[1][false]%
{%
\def\givenatend{\boolean{#1}}\ifthenelse{\boolean{#1}}{\begin{trivlist}\item}{\setbox0\vbox\bgroup}{}
}
{%
\ifthenelse{\givenatend}{\end{trivlist}}{\egroup}{}
}
\else
\newenvironment{instructorNotes}[1][false]%
{%
  \ifthenelse{\boolean{#1}}{\begin{trivlist}\item[\hskip \labelsep\bfseries {\Large Instructor Notes: \\} \hspace{\textwidth} ]}
{\begin{trivlist}\item[\hskip \labelsep\bfseries {\Large Instructor Notes: \\} \hspace{\textwidth} ]}
{}
}
{\end{trivlist}}
\fi


%% Suggested Timing
\newcommand{\timing}[1]{{\bf Suggested Timing: \hspace{2ex}} #1}

\title{Scaling Area}
\author{Bart Snapp and Brad Findell}

\outcome{Learning outcome goes here.}

\begin{document}
\begin{abstract}
Abstract goes here.  
\end{abstract}
\maketitle

\fixnote{Include here or in the notes some content of the PowerPoint.  Need a scaling volume activity.  See comments for a start.}

\begin{problem}
Is a $3\times 5$ rectangle similar to a $4\times 6$ rectangle?  Explain your reasoning.  Now come up with another explanation. 
\end{problem}

\vspace{.5in} 

\begin{problem}
Use area formulas to explain what happens to the area of a rectangle under scaling by a factor of $k$?  What about a triangle?  What about a circle?  
\end{problem}

\begin{problem}
Below is a figure and a dilation of that figure about point $O$.  
\begin{image}
\includegraphics{dilation.pdf}
\end{image}
\begin{enumerate}
\item Find the scale factor of the dilation.  Explain your reasoning. 
\item What can you say about the areas of the two figures?  Explain your reasoning. 
\end{enumerate}
\end{problem}


%\begin{problem}
%Imagine a $2\times 3\times 4$ right rectangular prism.  
%\begin{enumerate}
%\item Find the volume of the prism.  From the meaning of volume, explain why your calculations make sense.  
%\item Explain generally why the volume formula makes sense for dimensions that are counting numbers.  
%\item If you scale the prism by a factor of 3, how many copies of the original prism can you fit in the new one?  What does that say about the volume of the new prism?   
%\end{enumerate}
%\end{problem}

\end{document}
