%\documentclass[handout]{ximera}
\documentclass[nooutcomes,noauthor]{ximera}


\graphicspath{
  {./}
  {graphics/}
  {../graphics/}
}

\usepackage{chngcntr}

\let\question\relax
\let\endquestion\relax




\newtheoremstyle{SlantTheorem}{\topsep}{\fill}%%% space between body and thm
%\newtheoremstyle{SlantTheorem}{\topsep}{\topsep}%%% space between body and thm
 {\slshape}                      %%% Thm body font
 {}                              %%% Indent amount (empty = no indent)
 {\bfseries\sffamily}            %%% Thm head font
 {}                              %%% Punctuation after thm head
 {3ex}                           %%% Space after thm head
 {\thmname{#1}\thmnumber{ #2}\thmnote{ \bfseries(#3)}}%%% Thm head spec
\theoremstyle{SlantTheorem}
\newtheorem{question}{Question}
\counterwithin*{question}{section}



\let\instructorNotes\relax
\let\endinstructorNotes\relax
%%% instructorNotes environment
\ifhandout
\newenvironment{instructorNotes}[1][false]%
{%
\def\givenatend{\boolean{#1}}\ifthenelse{\boolean{#1}}{\begin{trivlist}\item}{\setbox0\vbox\bgroup}{}
}
{%
\ifthenelse{\givenatend}{\end{trivlist}}{\egroup}{}
}
\else
\newenvironment{instructorNotes}[1][false]%
{%
  \ifthenelse{\boolean{#1}}{\begin{trivlist}\item[\hskip \labelsep\bfseries {\Large Instructor Notes: \\} \hspace{\textwidth} ]}
{\begin{trivlist}\item[\hskip \labelsep\bfseries {\Large Instructor Notes: \\} \hspace{\textwidth} ]}
{}
}
{\end{trivlist}}
\fi


%% Suggested Timing
\newcommand{\timing}[1]{{\bf Suggested Timing: \hspace{2ex}} #1}

\title{Walking and Turning}
\author{Bart Snapp and Brad Findell}

\outcome{Learning outcome goes here.}

\begin{document}
\begin{abstract}
  We reason about interior angles of a triangle.
\end{abstract}
\maketitle


\begin{problem}
Consider an arbitrary triangle shown below.  Let $a$, $b$, and $c$ be the measures of the interior angles of the triangle.  

Beginning at point $M$, the midpoint of a side of the triangle, imagine\footnote{You will find it helpful to actually walk around a triangle outlined on the floor with masking tape, for example.  Alternatively, on Carmen you may find videos of other people walking and turning.} walking around the triangle starting in the direction indicated by the arrow.  At each vertex, turn counterclockwise (viewed from above), as indicated by the directed arc.  Your journey ends when you return to point $M$.  
\begin{enumerate}
\item What do you want to prove about $a$, $b$, and $c$?  
\item Extend the sides of the triangle.  At each vertex mark the angle through which the walker turns at that vertex.  
\item How much does the walker turn during the whole journey?  
\item Based upon your ``walking and turning'' journey, write a proof of your claim from part (a).  
\begin{image}
\definecolor{uuuuuu}{rgb}{0.26666666666666666,0.26666666666666666,0.26666666666666666}
\definecolor{ududff}{rgb}{0.30196078431372547,0.30196078431372547,1}
\begin{tikzpicture}[line cap=round,line join=round,>=triangle 45,x=1cm,y=1cm]
\draw [line width=1.2pt] (0,4)-- (7,0);
\draw [line width=1.2pt] (6,3)-- (0,4);
\draw [line width=1.2pt] (6,3)-- (7,0);
\begin{scriptsize}
\draw [-latex,line width=1pt] (6,4) arc (0:270:4mm);
\draw [fill=ududff] (6,4) circle (1pt);
\draw [color=ududff] (6,3.8) node {Turning};
\draw [fill=uuuuuu] (6.5,1.5) circle (1.5pt);
\draw[color=ududff] (6.25,1.5) node {$M$};
\draw [-latex,line width=1pt] (6.8,1.6) -- (6.58,2.2);
\draw[color=ududff] (7.3,1.8) node {Walking};
\draw[color=ududff] (0.8,3.7) node {$b$};
\draw[color=ududff] (6.7,0.4) node {$c$};
\draw[color=ududff] (5.8,2.8) node {$a$};
\end{scriptsize}
\end{tikzpicture}
\end{image}
\vfill
\end{enumerate}
\end{problem}

\newpage

\begin{problem}
Now repeat the previous problem, this time turning clockwise at each vertex and walking backwards along a side, as needed.  Again, prove what you can about $a$, $b$, and $c$.  
\begin{enumerate}
\item What do you want to prove about $a$, $b$, and $c$?  
\item Extend the sides of the triangle.  At each vertex mark the angle through which the walker turns at that vertex.  
\item How much does the walker turn during the whole journey?  
\item Based upon your ``walking and turning'' journey, write a proof of your claim from part (a).  
\begin{image}
\definecolor{uuuuuu}{rgb}{0.26666666666666666,0.26666666666666666,0.26666666666666666}
\definecolor{ududff}{rgb}{0.30196078431372547,0.30196078431372547,1}
\begin{tikzpicture}[line cap=round,line join=round,>=triangle 45,x=1cm,y=1cm]
\draw [line width=1.2pt] (0,4)-- (7,0);
\draw [line width=1.2pt] (6,3)-- (0,4);
\draw [line width=1.2pt] (6,3)-- (7,0);
\begin{scriptsize}
\draw [-latex,line width=1pt] (6,4) arc (180:-90:4mm);
\draw [fill=ududff] (6,4) circle (1pt);
\draw [color=ududff] (6,3.8) node {Turning};
\draw [fill=uuuuuu] (6.5,1.5) circle (1.5pt);
\draw[color=ududff] (6.25,1.5) node {$M$};
\draw [-latex,line width=1pt] (6.8,1.6) -- (6.58,2.2);
\draw[color=ududff] (7.3,1.8) node {Walking};
\draw[color=ududff] (0.8,3.7) node {$b$};
\draw[color=ududff] (6.7,0.4) node {$c$};
\draw[color=ududff] (5.8,2.8) node {$a$};
\end{scriptsize}
\end{tikzpicture}
\end{image}
\vfill
\end{enumerate}
\end{problem}

\end{document}