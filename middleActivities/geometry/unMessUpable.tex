\newpage

\section{UnMessUpable Figures}

\begin{teachingnote}
If Euclid the Game is available, then use the Tutorial through level 12 in in place of problems 1, 2, 5 and 6.  

Note:  Most bugs in Euclid the Game are fixed by refreshing the page.  Maybe have them start Euclid the Game at home.  

The last two problems are optional.
\end{teachingnote}
 
Suppose we draw or a construct a geometric figure with pencil, paper, compass, and straightedge.  If we want to compare to another example of the geometric figure, we need to begin again from scratch.  With dynamic geometry software (e.g., \textsl{Geogebra}, \textsl{Geometer's Sketchpad}, or \textsl{Cabri}), we can alter the original figure by ``dragging'' vertices and segments to create many other examples.  For this to work properly, we want to \emph{construct} the figure rather than merely \emph{draw} it, so that a square, for example, remains a square even if we move its vertices.  Some folks call such figures ``UnMessUpable.'' 

\vspace{0.1in}
\begin{center}
\textbf{Rules of Engagement:}
\end{center}
\begin{itemize}
\itemsep0em
\item Before you begin, explore the menus and toolbars to see what the software provides.  
\item You may use tools that function as a compass or straight-edge would.  
\item You may use special tools (e.g., perpendicular bisector) that accomplish multistep     
compass-and-straightedge constructions in a single step.
\item Do not use tools for transformations (e.g., translations, reflections, or rotations).
\item Do not use tools that construct objects from measurements.  
\end{itemize}

\begin{center}
\textbf{Begin each problem in a new sketch.}
\end{center}

\begin{prob}
Construct a segment between two points.  Then construct an equilateral triangle with that segment as one of its sides.  Be sure that the triangle remains equilateral when you drag its vertices.   (Note:  Do not use a ``regular polygon'' tool.)
\end{prob}

\begin{prob}
Construct a segment between two points.  Then construct a square with that segment as one of its sides.  Be sure that it remains a square when you drag its vertices.  (Note:  Do not use a ``regular polygon'' tool.)
\end{prob}

\begin{prob}
Construct an UnMessUpable parallelogram.  (Hint:  Think about the definition.)  
\end{prob}

\begin{prob}
Construct a rectangle that, through dragging, can be long and thin, short and fat, or anything in between, but that is always a rectangle.
\end{prob}

\begin{prob}
\emph{Copy a segment.}  Construct a segment and a line.  Then copy the segment onto the line.  Hide the line so that the segment alone is clear.  Then drag the vertices that determine the initial segment to show that the copy is always congruent to it.  
\end{prob}

\begin{prob}
\emph{Copy an angle.}  Using the ray tool, construct an angle and a separate ray.  Then copy the angle onto the other ray.  Drag the vertices that determine the first angle to show that the copy is always congruent to it.  
\end{prob}

\begin{prob}
Construct a capital H so that the midline is always the perpendicular bisector of both sides.  
\end{prob}

\begin{prob}
Construct a quadrilateral so that one pair of opposite sides is always congruent.  
\end{prob}





