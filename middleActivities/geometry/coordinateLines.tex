%\documentclass[handout]{ximera}
\documentclass[nooutcomes,instructornotes]{ximera}


\graphicspath{
  {./}
  {graphics/}
  {../graphics/}
}

\usepackage{chngcntr}

\let\question\relax
\let\endquestion\relax




\newtheoremstyle{SlantTheorem}{\topsep}{\fill}%%% space between body and thm
%\newtheoremstyle{SlantTheorem}{\topsep}{\topsep}%%% space between body and thm
 {\slshape}                      %%% Thm body font
 {}                              %%% Indent amount (empty = no indent)
 {\bfseries\sffamily}            %%% Thm head font
 {}                              %%% Punctuation after thm head
 {3ex}                           %%% Space after thm head
 {\thmname{#1}\thmnumber{ #2}\thmnote{ \bfseries(#3)}}%%% Thm head spec
\theoremstyle{SlantTheorem}
\newtheorem{question}{Question}
\counterwithin*{question}{section}



\let\instructorNotes\relax
\let\endinstructorNotes\relax
%%% instructorNotes environment
\ifhandout
\newenvironment{instructorNotes}[1][false]%
{%
\def\givenatend{\boolean{#1}}\ifthenelse{\boolean{#1}}{\begin{trivlist}\item}{\setbox0\vbox\bgroup}{}
}
{%
\ifthenelse{\givenatend}{\end{trivlist}}{\egroup}{}
}
\else
\newenvironment{instructorNotes}[1][false]%
{%
  \ifthenelse{\boolean{#1}}{\begin{trivlist}\item[\hskip \labelsep\bfseries {\Large Instructor Notes: \\} \hspace{\textwidth} ]}
{\begin{trivlist}\item[\hskip \labelsep\bfseries {\Large Instructor Notes: \\} \hspace{\textwidth} ]}
{}
}
{\end{trivlist}}
\fi


%% Suggested Timing
\newcommand{\timing}[1]{{\bf Suggested Timing: \hspace{2ex}} #1}

\title{Coordinate Constructions}
\author{Bart Snapp and Brad Findell}

\outcome{Learning outcome goes here.}

\begin{document}
\begin{abstract}
  We use Cartesian coordinates to help understand lines and their equations.
\end{abstract}
\maketitle


%
%Given an equation and a point, check whether the point on the graph.  
%
%Given two points, write an equation of the line. 
%Given a point and a slope, write an equation of the line. 
%Given a line and a point, write an equation through the point parallel or perpendicular to the given line. 
%
%Given two lines, find their intersection. 
%Given a line and a circle, find their intersection. 
%
%Given a center and a point, write an equation of a circle 
%Given a center and radius, write and equation of a circle. 
%
%Given an equation of a line, find the slope and intercepts
%Given an equation of a circle, find the center and radius. 
%
%Given two points, write the equation of the perpendicular bisector
%Given three points, write an equation of an median, altitude, or angle bisector




\begin{problem}


Develop slope-slope form of a line as an equation that must be true for $(x,y)$ to be on the line through $(x_1,y_1)$ and $(x_2,y_2)$.  

Write and equation of the line through $(2,3)$ and $(6,8)$.  Then generalize

\[
\frac{y-8}{x-6} = \frac{8-3}{6-2}
\]

\[
\frac{y-y_2}{x-x_2} = \frac{y_2-y_1}{x_2-x_1}
\]


Discuss how slope-slope form can be easily converted to point-slope form and to a form $y =$ ``something equivalent $mx + b$'' as follow: 

\[
y-y_2 =\left(\frac{y_2-y_1}{x_2-x_1}\right) (x-x_2)
\]
\[
y =y_2 + \left(\frac{y_2-y_1}{x_2-x_1}\right) (x-x_2)
\]

Have some conversation about how to check whether a point is on the line.  

Write the equation of the line through

\end{problem}

\begin{problem}
Draw a graph of the line $3x+5y=12$.  
\end{problem}

\begin{teachingnote}
Most students will translate to slope-intercept form and use it to plot the $y$-intercept and a few other points.  We hope that a few students will observe that two points are sufficient to determine a line, and the easiest points to compute are the $x$- and $y$-intercepts.  
\end{teachingnote}


\begin{problem}
Let your birth month, $m$,  and your birth day, $d$, be coordinates of a point in the plane $(m,d)$.  Write an equation of line parallel to $3x+5y=12$ and containing $(m,d)$.  Write the equation in standard form. 

\begin{teachingnote}
Most students will translate to slope-intercept form, use the same slope, do quite a bit of 
algebra to find $b$ in $y=mx+b$, and finally translate back to standard form.  
\end{teachingnote}

\end{problem}


\end{document}
