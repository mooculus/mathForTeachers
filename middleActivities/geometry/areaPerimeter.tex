%\documentclass[handout]{ximera}
\documentclass{ximera}

\usepackage{gensymb}
\usepackage{tabularx}
\usepackage{mdframed}
\usepackage{pdfpages}
%\usepackage{chngcntr}

\let\problem\relax
\let\endproblem\relax

\newcommand{\property}[2]{#1#2}




\newtheoremstyle{SlantTheorem}{\topsep}{\fill}%%% space between body and thm
 {\slshape}                      %%% Thm body font
 {}                              %%% Indent amount (empty = no indent)
 {\bfseries\sffamily}            %%% Thm head font
 {}                              %%% Punctuation after thm head
 {3ex}                           %%% Space after thm head
 {\thmname{#1}\thmnumber{ #2}\thmnote{ \bfseries(#3)}} %%% Thm head spec
\theoremstyle{SlantTheorem}
\newtheorem{problem}{Problem}[]

%\counterwithin*{problem}{section}



%%%%%%%%%%%%%%%%%%%%%%%%%%%%Jenny's code%%%%%%%%%%%%%%%%%%%%

%%% Solution environment
%\newenvironment{solution}{
%\ifhandout\setbox0\vbox\bgroup\else
%\begin{trivlist}\item[\hskip \labelsep\small\itshape\bfseries Solution\hspace{2ex}]
%\par\noindent\upshape\small
%\fi}
%{\ifhandout\egroup\else
%\end{trivlist}
%\fi}
%
%
%%% instructorIntro environment
%\ifhandout
%\newenvironment{instructorIntro}[1][false]%
%{%
%\def\givenatend{\boolean{#1}}\ifthenelse{\boolean{#1}}{\begin{trivlist}\item}{\setbox0\vbox\bgroup}{}
%}
%{%
%\ifthenelse{\givenatend}{\end{trivlist}}{\egroup}{}
%}
%\else
%\newenvironment{instructorIntro}[1][false]%
%{%
%  \ifthenelse{\boolean{#1}}{\begin{trivlist}\item[\hskip \labelsep\bfseries Instructor Notes:\hspace{2ex}]}
%{\begin{trivlist}\item[\hskip \labelsep\bfseries Instructor Notes:\hspace{2ex}]}
%{}
%}
%% %% line at the bottom} 
%{\end{trivlist}\par\addvspace{.5ex}\nobreak\noindent\hung} 
%\fi
%
%


\let\instructorNotes\relax
\let\endinstructorNotes\relax
%%% instructorNotes environment
\ifhandout
\newenvironment{instructorNotes}[1][false]%
{%
\def\givenatend{\boolean{#1}}\ifthenelse{\boolean{#1}}{\begin{trivlist}\item}{\setbox0\vbox\bgroup}{}
}
{%
\ifthenelse{\givenatend}{\end{trivlist}}{\egroup}{}
}
\else
\newenvironment{instructorNotes}[1][false]%
{%
  \ifthenelse{\boolean{#1}}{\begin{trivlist}\item[\hskip \labelsep\bfseries {\Large Instructor Notes: \\} \hspace{\textwidth} ]}
{\begin{trivlist}\item[\hskip \labelsep\bfseries {\Large Instructor Notes: \\} \hspace{\textwidth} ]}
{}
}
{\end{trivlist}}
\fi


%% Suggested Timing
\newcommand{\timing}[1]{{\bf Suggested Timing: \hspace{2ex}} #1}




\hypersetup{
    colorlinks=true,       % false: boxed links; true: colored links
    linkcolor=blue,          % color of internal links (change box color with linkbordercolor)
    citecolor=green,        % color of links to bibliography
    filecolor=magenta,      % color of file links
    urlcolor=cyan           % color of external links
}

\title{Area and Perimeter}
\author{Bart Snapp and Brad Findell}

\outcome{Learning outcome goes here.}

\begin{document}
\begin{abstract}
Abstract goes here.  
\end{abstract}
\maketitle

\begin{teachingnote}
Supplies: graph paper for tables and graphs.  

Purposes for this class: (1) For a given area, square will have least perimeter; (2) for a given perimeter, square will have maximum area; (3) what is a function of what; (4) recognizing the type of function from the form of the expression, the shape of the graph, or the quantities involved; (5) connecting the dots, limiting cases, and other domain questions; 

Some students will have a ``sideways'' graph of perimeter versus length, which is a function, though not displayed conventionally.  Asking ``Is perimeter a function of length?'' is a better question than ``Is this a function?''

Other points:  These problems can be done at many different grade levels.  Edge pieces are merely to get them thinking about perimeter without telling them.  Need to mention that rational functions are quotients of polynomial functions.
\end{teachingnote}

\begin{problem} You have been asked to put together the dance floor for your sister's wedding.  The dance floor is made up of 24 square tiles that measure one meter on each side. 
\begin{enumerate}
\item Experiment with different rectangles that could be made using all of these tiles, and record your data in a table.  
\item Draw a graph of your data.  Describe patterns in the data, as seen in the table or graph.  
\item Can we connect the dots in the graphs?  Explain. 
\item How might we change the context so that the dimensions can be other than whole numbers?   In the new context, how would the previous answers change?
\end{enumerate}
\end{problem}

\begin{problem} Suppose the dance floor is held together by a border made of thin edge pieces one meter long.  
\begin{enumerate}
\item What determines how many edge pieces are needed?  Explain. 
\item Make a graph showing the perimeter vs. length for various rectangles with an area of 24 square meters.  
\item Describe the graph.  How do patterns that you observed in the table show up in the graph? 
\item For perimeter and length, is either one a function of the other?  Explain what that means.   
\item Which design would require the most edge pieces?  Explain.  
\item Which design would require the fewest edge pieces?  Explain.
\item If the context allows dimensions other than whole numbers, how would the previous answers change?
\end{enumerate}
\end{problem}

\begin{problem}
Suppose you had begun with a different number of floor tiles, such as 30, 21, or 19, or 36.  
\begin{enumerate}
\item In general, describe the rectangle with whole-number dimensions that has the greatest perimeter for a fixed area.  
\item If the context does not require whole-number dimensions, describe the rectangle with the least perimeter for a fixed area.  
\end{enumerate}
\end{problem}

\begin{problem}
The previous problems were about rectangles with constant area and changing perimeter.  
\begin{enumerate}
\item Make up a problem about rectangles with whole-number dimensions, constant perimeter, and changing area.   
\item Make a table of length, width, perimeter, and area for these rectangles.
\item Draw graphs of width versus length and area versus length for your rectangles.  
\item Now modify the context and your graphs to allow dimensions that are not whole numbers.   
\item Which rectangle will have a maximum area?  Explain.
\item Which rectangle will have a minimum area?  Explain.
\end{enumerate}
\end{problem}

\begin{problem}
So far we have considered rectangles with fixed area and those with fixed perimeter.  What about fixing the width or the length?  Since they behave in much the same way, let's fix the width.   
\begin{enumerate}
\item Make up a problem about rectangles with constant width and changing area and perimeter.   
\item Make a table of length, width, perimeter, and area for these rectangles.
\item Draw graphs of area versus length and perimeter versus length for your rectangles. 
\end{enumerate}
\end{problem}

\begin{problem}
What types of functions did you see in the previous problems?  Complete the following sentences with types of functions.  (Note:  If two functions are the same type, write answers that distinguish them from each other.)  
\begin{enumerate}
\item Fixed width: area vs. length is a $\rule[-1pt]{150pt}{.5pt}$.
\item Fixed width: perimeter vs. length is a $\rule[-1pt]{150pt}{.5pt}$.
\item Fixed perimeter: width vs. length is a $\rule[-1pt]{150pt}{.5pt}$.
\item Fixed perimeter: area vs. length is a $\rule[-1pt]{150pt}{.5pt}$.
\item Fixed area: width vs. length is a $\rule[-1pt]{150pt}{.5pt}$.
\item Fixed area: perimeter vs. length is a $\rule[-1pt]{150pt}{.5pt}$.
\end{enumerate}
\end{problem}

\begin{teachingnote}
\begin{enumerate}
\item Fixed width: area vs. length is a \textbf{direct proportion}.  
\item Fixed width: perimeter vs. length is a \textbf{linear function that is not a direct proportion}. 
\item Fixed perimeter: width vs. length is a \textbf{decreasing linear functions}. 
\item Fixed perimeter: area vs. length is a \textbf{quadratic function}. 
\item Fixed area: width vs. length is an \textbf{inverse proportion}.  
\item Fixed area: perimeter vs. length is a \textbf{rational function that is not an inverse proportion}. 
\end{enumerate}
\end{teachingnote}

\begin{problem}
Explain how and where you saw the following advanced algebra ideas in the above problems:  
\begin{enumerate}
\item Domain, range and ``limiting cases''
\item Rates of change, maxima, minima, and asymptotic behavior
\item Generalizing from a specific to a generic fixed quantity
\item Equation solving with several variables
\end{enumerate}
\end{problem}

\end{document}
