%\documentclass[handout]{ximera}
\documentclass[nooutcomes,instructornotes]{ximera}

\usepackage{gensymb}
\usepackage{tabularx}
\usepackage{mdframed}
\usepackage{pdfpages}
%\usepackage{chngcntr}

\let\problem\relax
\let\endproblem\relax

\newcommand{\property}[2]{#1#2}




\newtheoremstyle{SlantTheorem}{\topsep}{\fill}%%% space between body and thm
 {\slshape}                      %%% Thm body font
 {}                              %%% Indent amount (empty = no indent)
 {\bfseries\sffamily}            %%% Thm head font
 {}                              %%% Punctuation after thm head
 {3ex}                           %%% Space after thm head
 {\thmname{#1}\thmnumber{ #2}\thmnote{ \bfseries(#3)}} %%% Thm head spec
\theoremstyle{SlantTheorem}
\newtheorem{problem}{Problem}[]

%\counterwithin*{problem}{section}



%%%%%%%%%%%%%%%%%%%%%%%%%%%%Jenny's code%%%%%%%%%%%%%%%%%%%%

%%% Solution environment
%\newenvironment{solution}{
%\ifhandout\setbox0\vbox\bgroup\else
%\begin{trivlist}\item[\hskip \labelsep\small\itshape\bfseries Solution\hspace{2ex}]
%\par\noindent\upshape\small
%\fi}
%{\ifhandout\egroup\else
%\end{trivlist}
%\fi}
%
%
%%% instructorIntro environment
%\ifhandout
%\newenvironment{instructorIntro}[1][false]%
%{%
%\def\givenatend{\boolean{#1}}\ifthenelse{\boolean{#1}}{\begin{trivlist}\item}{\setbox0\vbox\bgroup}{}
%}
%{%
%\ifthenelse{\givenatend}{\end{trivlist}}{\egroup}{}
%}
%\else
%\newenvironment{instructorIntro}[1][false]%
%{%
%  \ifthenelse{\boolean{#1}}{\begin{trivlist}\item[\hskip \labelsep\bfseries Instructor Notes:\hspace{2ex}]}
%{\begin{trivlist}\item[\hskip \labelsep\bfseries Instructor Notes:\hspace{2ex}]}
%{}
%}
%% %% line at the bottom} 
%{\end{trivlist}\par\addvspace{.5ex}\nobreak\noindent\hung} 
%\fi
%
%


\let\instructorNotes\relax
\let\endinstructorNotes\relax
%%% instructorNotes environment
\ifhandout
\newenvironment{instructorNotes}[1][false]%
{%
\def\givenatend{\boolean{#1}}\ifthenelse{\boolean{#1}}{\begin{trivlist}\item}{\setbox0\vbox\bgroup}{}
}
{%
\ifthenelse{\givenatend}{\end{trivlist}}{\egroup}{}
}
\else
\newenvironment{instructorNotes}[1][false]%
{%
  \ifthenelse{\boolean{#1}}{\begin{trivlist}\item[\hskip \labelsep\bfseries {\Large Instructor Notes: \\} \hspace{\textwidth} ]}
{\begin{trivlist}\item[\hskip \labelsep\bfseries {\Large Instructor Notes: \\} \hspace{\textwidth} ]}
{}
}
{\end{trivlist}}
\fi


%% Suggested Timing
\newcommand{\timing}[1]{{\bf Suggested Timing: \hspace{2ex}} #1}




\hypersetup{
    colorlinks=true,       % false: boxed links; true: colored links
    linkcolor=blue,          % color of internal links (change box color with linkbordercolor)
    citecolor=green,        % color of links to bibliography
    filecolor=magenta,      % color of file links
    urlcolor=cyan           % color of external links
}

\title{Understanding and Using Absolute Value}
\author{Bart Snapp and Brad Findell}

\outcome{Learning outcome goes here.}

\begin{document}
\begin{abstract}
We think about absolute value.
\end{abstract}
\maketitle

\begin{teachingnote}
For the first problem, vote in class with eyes closed or with a voting system.  

Maybe switch problems 2 and 3. Maybe add the Pythagorean Theorem and distance formula across dimensions.  
\end{teachingnote}

\begin{problem}
True or False (and explain)
\begin{enumerate}
\item $-x$ is negative
\item $\sqrt{9} = \pm 3$
\item $\sqrt{x^2} = x$
\item $\sqrt{x^2} = |x|$
\item If $|x| = -x$ then $x$ is negative or 0. 
\end{enumerate}
\vfill
\end{problem}

\begin{problem}
Let's consider circles in city geometry.  \emph{Hint: First, remind yourself how to use the definition of circle and the distance formula in Euclidean coordinate geometry to derive the equation of a Euclidean circle with radius $r$ and center $(a, b)$.}  
\begin{enumerate}
\item Use the taxicab distance formula to derive the equation of a city-geometry circle with radius $r$ and center $(a, b)$.  
\item Write the equation of a city-geometry circle with radius 1, centered at the origin, and draw a graph of this city-geometry circle.  
\end{enumerate}
\vfill
\end{problem}

\newpage

To better understand the equation of this city-geometry circle, we need to firm up the idea of absolute value.  

\begin{problem}
Consider the following attempts to characterize the absolute value function.    
\begin{align}
|x| & \text{ is the ``magnitude'' of $x$---the size of $x$, ignoring its sign.} \\
|x| & \text{ is the distance from the origin to $x$.} \\
|x| & = \sqrt{x^2} \\
|x| & =
  \begin{cases}
   x    & \text{if } x \geq 0 \\
   -x   & \text{if } x < 0
  \end{cases}
\end{align}

\begin{enumerate}
\item Which characterization is the definition of the absolute value function? 
\item Are the other characterizations of the absolute value function equivalent to the definition?  Explain.  
\item Use one or more of these characterizations to develop meanings for $|x-a|$ and $|a-x|$ where $a$ is a constant.  
\item Use one or more of these characterizations to explain the solution(s) to $|x-5|=8$.  
\item What are the benefits of using more than one characterization of this idea?  
\end{enumerate}
\vfill
\end{problem}

\newpage 

\begin{problem}
Use the piecewise characterization of the absolute value function to explain why the equation $|x| + |y| = 1$ has the graph that it does. (Hint:  Consider various cases, depending upon the sign of $x$ and the sign of $y$.)
\vfill
\end{problem}

\end{document}
