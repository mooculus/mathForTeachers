%\documentclass[handout]{ximera}
\documentclass{ximera}

\usepackage{gensymb}
\usepackage{tabularx}
\usepackage{mdframed}
\usepackage{pdfpages}
%\usepackage{chngcntr}

\let\problem\relax
\let\endproblem\relax

\newcommand{\property}[2]{#1#2}




\newtheoremstyle{SlantTheorem}{\topsep}{\fill}%%% space between body and thm
 {\slshape}                      %%% Thm body font
 {}                              %%% Indent amount (empty = no indent)
 {\bfseries\sffamily}            %%% Thm head font
 {}                              %%% Punctuation after thm head
 {3ex}                           %%% Space after thm head
 {\thmname{#1}\thmnumber{ #2}\thmnote{ \bfseries(#3)}} %%% Thm head spec
\theoremstyle{SlantTheorem}
\newtheorem{problem}{Problem}[]

%\counterwithin*{problem}{section}



%%%%%%%%%%%%%%%%%%%%%%%%%%%%Jenny's code%%%%%%%%%%%%%%%%%%%%

%%% Solution environment
%\newenvironment{solution}{
%\ifhandout\setbox0\vbox\bgroup\else
%\begin{trivlist}\item[\hskip \labelsep\small\itshape\bfseries Solution\hspace{2ex}]
%\par\noindent\upshape\small
%\fi}
%{\ifhandout\egroup\else
%\end{trivlist}
%\fi}
%
%
%%% instructorIntro environment
%\ifhandout
%\newenvironment{instructorIntro}[1][false]%
%{%
%\def\givenatend{\boolean{#1}}\ifthenelse{\boolean{#1}}{\begin{trivlist}\item}{\setbox0\vbox\bgroup}{}
%}
%{%
%\ifthenelse{\givenatend}{\end{trivlist}}{\egroup}{}
%}
%\else
%\newenvironment{instructorIntro}[1][false]%
%{%
%  \ifthenelse{\boolean{#1}}{\begin{trivlist}\item[\hskip \labelsep\bfseries Instructor Notes:\hspace{2ex}]}
%{\begin{trivlist}\item[\hskip \labelsep\bfseries Instructor Notes:\hspace{2ex}]}
%{}
%}
%% %% line at the bottom} 
%{\end{trivlist}\par\addvspace{.5ex}\nobreak\noindent\hung} 
%\fi
%
%


\let\instructorNotes\relax
\let\endinstructorNotes\relax
%%% instructorNotes environment
\ifhandout
\newenvironment{instructorNotes}[1][false]%
{%
\def\givenatend{\boolean{#1}}\ifthenelse{\boolean{#1}}{\begin{trivlist}\item}{\setbox0\vbox\bgroup}{}
}
{%
\ifthenelse{\givenatend}{\end{trivlist}}{\egroup}{}
}
\else
\newenvironment{instructorNotes}[1][false]%
{%
  \ifthenelse{\boolean{#1}}{\begin{trivlist}\item[\hskip \labelsep\bfseries {\Large Instructor Notes: \\} \hspace{\textwidth} ]}
{\begin{trivlist}\item[\hskip \labelsep\bfseries {\Large Instructor Notes: \\} \hspace{\textwidth} ]}
{}
}
{\end{trivlist}}
\fi


%% Suggested Timing
\newcommand{\timing}[1]{{\bf Suggested Timing: \hspace{2ex}} #1}




\hypersetup{
    colorlinks=true,       % false: boxed links; true: colored links
    linkcolor=blue,          % color of internal links (change box color with linkbordercolor)
    citecolor=green,        % color of links to bibliography
    filecolor=magenta,      % color of file links
    urlcolor=cyan           % color of external links
}

\title{Parallels}
\author{Bart Snapp and Brad Findell}

\outcome{Learning outcome goes here.}

\begin{document}
\begin{abstract}
Abstract goes here.  
\end{abstract}
\maketitle

\begin{teachingnote}
\end{teachingnote}

In the following problems, you may assume the following: 

\begin{postulate}[Parallel Postulate]
Given a line and a point not on the line, there is exactly one line passing through the point which is parallel to the given line.
\end{postulate}

You may also use previously-established results, such as the following: 
\begin{itemize}
\itemsep -3pt
\item The measures of adjacent angles add as they should.
\item A straight angle measures $180^\circ$.  
\item A $180^\circ$ rotation about a point on a line takes the line to itself.  
\item A $180^\circ$ rotation about a point off a line takes the line to a parallel line.  
\end{itemize}

Now you may get started! 

\begin{problem}
Use adjacent angles to prove that vertical angles are equal.    
\end{problem}

\begin{problem}
Now use rotations to prove that vertical angles are equal.
\end{problem}

\fixnote{Include figures to set up the next two proofs, using the $180^\circ$ rotation ideas above.  The first of these uses the above parallel postulate, the second one does not.  Maybe switch the order.  Also, include teaching notes about non-Euclidean geometries.  Maybe add a problem connecting to Euclid's version of the parallel postulate.}

\begin{problem}
Prove:  If a pair of parallel lines is cut by a transversal, then alternate interior angles are equal and corresponding angles are equal.
\end{problem}

\begin{problem}
Prove: If a pair of alternate interior angles or a pair of corresponding angles of a transversal with respect to two lines are equal, then the lines are parallel.
\end{problem}

\begin{problem}
The previous two problems seem almost identical to one another.  How are they different?  
\end{problem}

\begin{problem}
Prove:  The angle sum of a triangle is $180^\circ$.
\end{problem}
\end{document}
