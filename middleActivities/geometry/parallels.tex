%\documentclass[handout]{ximera}
\documentclass[nooutcomes]{ximera}

\usepackage{gensymb}
\usepackage{tabularx}
\usepackage{mdframed}
\usepackage{pdfpages}
%\usepackage{chngcntr}

\let\problem\relax
\let\endproblem\relax

\newcommand{\property}[2]{#1#2}




\newtheoremstyle{SlantTheorem}{\topsep}{\fill}%%% space between body and thm
 {\slshape}                      %%% Thm body font
 {}                              %%% Indent amount (empty = no indent)
 {\bfseries\sffamily}            %%% Thm head font
 {}                              %%% Punctuation after thm head
 {3ex}                           %%% Space after thm head
 {\thmname{#1}\thmnumber{ #2}\thmnote{ \bfseries(#3)}} %%% Thm head spec
\theoremstyle{SlantTheorem}
\newtheorem{problem}{Problem}[]

%\counterwithin*{problem}{section}



%%%%%%%%%%%%%%%%%%%%%%%%%%%%Jenny's code%%%%%%%%%%%%%%%%%%%%

%%% Solution environment
%\newenvironment{solution}{
%\ifhandout\setbox0\vbox\bgroup\else
%\begin{trivlist}\item[\hskip \labelsep\small\itshape\bfseries Solution\hspace{2ex}]
%\par\noindent\upshape\small
%\fi}
%{\ifhandout\egroup\else
%\end{trivlist}
%\fi}
%
%
%%% instructorIntro environment
%\ifhandout
%\newenvironment{instructorIntro}[1][false]%
%{%
%\def\givenatend{\boolean{#1}}\ifthenelse{\boolean{#1}}{\begin{trivlist}\item}{\setbox0\vbox\bgroup}{}
%}
%{%
%\ifthenelse{\givenatend}{\end{trivlist}}{\egroup}{}
%}
%\else
%\newenvironment{instructorIntro}[1][false]%
%{%
%  \ifthenelse{\boolean{#1}}{\begin{trivlist}\item[\hskip \labelsep\bfseries Instructor Notes:\hspace{2ex}]}
%{\begin{trivlist}\item[\hskip \labelsep\bfseries Instructor Notes:\hspace{2ex}]}
%{}
%}
%% %% line at the bottom} 
%{\end{trivlist}\par\addvspace{.5ex}\nobreak\noindent\hung} 
%\fi
%
%


\let\instructorNotes\relax
\let\endinstructorNotes\relax
%%% instructorNotes environment
\ifhandout
\newenvironment{instructorNotes}[1][false]%
{%
\def\givenatend{\boolean{#1}}\ifthenelse{\boolean{#1}}{\begin{trivlist}\item}{\setbox0\vbox\bgroup}{}
}
{%
\ifthenelse{\givenatend}{\end{trivlist}}{\egroup}{}
}
\else
\newenvironment{instructorNotes}[1][false]%
{%
  \ifthenelse{\boolean{#1}}{\begin{trivlist}\item[\hskip \labelsep\bfseries {\Large Instructor Notes: \\} \hspace{\textwidth} ]}
{\begin{trivlist}\item[\hskip \labelsep\bfseries {\Large Instructor Notes: \\} \hspace{\textwidth} ]}
{}
}
{\end{trivlist}}
\fi


%% Suggested Timing
\newcommand{\timing}[1]{{\bf Suggested Timing: \hspace{2ex}} #1}




\hypersetup{
    colorlinks=true,       % false: boxed links; true: colored links
    linkcolor=blue,          % color of internal links (change box color with linkbordercolor)
    citecolor=green,        % color of links to bibliography
    filecolor=magenta,      % color of file links
    urlcolor=cyan           % color of external links
}

\title{Parallels}
\author{Bart Snapp and Brad Findell}

\outcome{Learning outcome goes here.}

\begin{document}
\begin{abstract}
  We seek to understand the Parallel Postulate and its consequences.
\end{abstract}
\maketitle

In the following problems, you may assume the following: 

\begin{postulate}[Parallel Postulate]
Given a line and a point not on the line, there is exactly one line passing through the point which is parallel to the given line.
\end{postulate}

You may also use previously-established results, such as the following: 
\begin{itemize}
\itemsep -3pt
\item The measures of adjacent angles add as they should.
\item A straight angle measures $180^\circ$.  
\item A $180^\circ$ rotation about a point on a line takes the line to itself.  
\item A $180^\circ$ rotation about a point off a line takes the line to a parallel line.  
\end{itemize}

Now you may get started! 

\begin{teachingnote}
To prove that vertical angles are equal, here are intuitions behind three different approaches: 
\begin{itemize}
\item Using linear pairs, vertical angles are supplements of the same angle.  
\item Rotate $180^\circ$ about the point of intersection. 
\item Reflect about an angle bisector. 
\end{itemize}
Student likely have seen the linear pairs approach before, but it is tempting to write down non-overlapping linear pairs and get stuck.  For the other approaches, which exploit symmetry, it might be necessary to suggest they think about transformations.  
\end{teachingnote}

\begin{problem}
Prove that vertical angles are equal.  Then try to prove it another way.  
\vfill
\end{problem}

\newpage

\begin{teachingnote}
The next two proofs use the $180^\circ$ rotation ideas above.  The first uses the above parallel postulate, the second one does not.  To remember which is which (as the instructor), it helps to compare to non-Euclidean geometries.  For example, in hyperbolic geometry the first theorem is false and the second is still true.  

It might also be worthwhile to compare to Euclid's version of the parallel postulate: That, if a straight line falling on two straight lines make the interior angles on the same side less than two right angles, the two straight lines, if produced indefinitely, meet on that side on which are the angles less than the two right angles.
\end{teachingnote}

\begin{problem}
Prove:  If a pair of parallel lines is cut by a transversal, then alternate interior angles are equal and corresponding angles are equal.
\vfill
\end{problem}

\begin{problem}
Prove: If a pair of alternate interior angles or a pair of corresponding angles of a transversal with respect to two lines are equal, then the lines are parallel.
\vfill
\end{problem}

\newpage
\begin{problem}
The previous two problems seem almost identical to one another.  How are they different?  
\vspace{1in}
\end{problem}

\begin{problem}
Prove:  The angle sum of a triangle is $180^\circ$.
\vfill
\end{problem}
\end{document}
