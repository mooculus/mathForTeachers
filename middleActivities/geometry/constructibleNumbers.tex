%\documentclass[handout]{ximera}
\documentclass[nooutcomes,instructornotes]{ximera}

\usepackage{gensymb}
\usepackage{tabularx}
\usepackage{mdframed}
\usepackage{pdfpages}
%\usepackage{chngcntr}

\let\problem\relax
\let\endproblem\relax

\newcommand{\property}[2]{#1#2}




\newtheoremstyle{SlantTheorem}{\topsep}{\fill}%%% space between body and thm
 {\slshape}                      %%% Thm body font
 {}                              %%% Indent amount (empty = no indent)
 {\bfseries\sffamily}            %%% Thm head font
 {}                              %%% Punctuation after thm head
 {3ex}                           %%% Space after thm head
 {\thmname{#1}\thmnumber{ #2}\thmnote{ \bfseries(#3)}} %%% Thm head spec
\theoremstyle{SlantTheorem}
\newtheorem{problem}{Problem}[]

%\counterwithin*{problem}{section}



%%%%%%%%%%%%%%%%%%%%%%%%%%%%Jenny's code%%%%%%%%%%%%%%%%%%%%

%%% Solution environment
%\newenvironment{solution}{
%\ifhandout\setbox0\vbox\bgroup\else
%\begin{trivlist}\item[\hskip \labelsep\small\itshape\bfseries Solution\hspace{2ex}]
%\par\noindent\upshape\small
%\fi}
%{\ifhandout\egroup\else
%\end{trivlist}
%\fi}
%
%
%%% instructorIntro environment
%\ifhandout
%\newenvironment{instructorIntro}[1][false]%
%{%
%\def\givenatend{\boolean{#1}}\ifthenelse{\boolean{#1}}{\begin{trivlist}\item}{\setbox0\vbox\bgroup}{}
%}
%{%
%\ifthenelse{\givenatend}{\end{trivlist}}{\egroup}{}
%}
%\else
%\newenvironment{instructorIntro}[1][false]%
%{%
%  \ifthenelse{\boolean{#1}}{\begin{trivlist}\item[\hskip \labelsep\bfseries Instructor Notes:\hspace{2ex}]}
%{\begin{trivlist}\item[\hskip \labelsep\bfseries Instructor Notes:\hspace{2ex}]}
%{}
%}
%% %% line at the bottom} 
%{\end{trivlist}\par\addvspace{.5ex}\nobreak\noindent\hung} 
%\fi
%
%


\let\instructorNotes\relax
\let\endinstructorNotes\relax
%%% instructorNotes environment
\ifhandout
\newenvironment{instructorNotes}[1][false]%
{%
\def\givenatend{\boolean{#1}}\ifthenelse{\boolean{#1}}{\begin{trivlist}\item}{\setbox0\vbox\bgroup}{}
}
{%
\ifthenelse{\givenatend}{\end{trivlist}}{\egroup}{}
}
\else
\newenvironment{instructorNotes}[1][false]%
{%
  \ifthenelse{\boolean{#1}}{\begin{trivlist}\item[\hskip \labelsep\bfseries {\Large Instructor Notes: \\} \hspace{\textwidth} ]}
{\begin{trivlist}\item[\hskip \labelsep\bfseries {\Large Instructor Notes: \\} \hspace{\textwidth} ]}
{}
}
{\end{trivlist}}
\fi


%% Suggested Timing
\newcommand{\timing}[1]{{\bf Suggested Timing: \hspace{2ex}} #1}




\hypersetup{
    colorlinks=true,       % false: boxed links; true: colored links
    linkcolor=blue,          % color of internal links (change box color with linkbordercolor)
    citecolor=green,        % color of links to bibliography
    filecolor=magenta,      % color of file links
    urlcolor=cyan           % color of external links
}

\title{Constructible Numbers}
\author{Bart Snapp and Brad Findell}

\outcome{Learning outcome goes here.}

\begin{document}
\begin{abstract}
  We use algebra to help us understand compass and straightedge
  constructions.
\end{abstract}
\maketitle

Compass and straightedge constructions involve drawing and finding intersections of two fundamental geometric objects:  lines and circles.  All more complicated constructions are combinations of pieces of these.  

In this activity, we explore what numbers are constructible (as lengths or distances) with compass and straightedge, assuming only that we begin with a segment of length 1.  We call such numbers \textit{constructible numbers}.  First we must establish how to do arithmetic with compass and straightedge.  

\subsection*{Arithmetic with Constructions}
\begin{teachingnote}
For multiplication, division, and square root, some students may need pictures.  
\end{teachingnote}
\begin{problem}
Suppose you are given a compass and a straightedge and segments of lengths $a$, $b$, and $1$.  
\begin{enumerate}
\item How would you construct a segment of length $a+b$? 
\vfill
\item How would you construct a segment of length $a-b$? 
\vfill
\item How would you construct a segment of length $ab$?  (Hint:  Use similar triangles.)  
\vfill
\item How would you construct a segment of length $a\div b$? 
\vfill
\item How would you construct a segment of length $\sqrt{a}$?  (Hint: Recall how to construct a geometric mean.)  
\vfill
\end{enumerate}
\end{problem}

\newpage
\begin{problem}
Beginning with a segment of length 1, how you might construct segments of the following lengths?  Describe briefly (to your partner) the arithmetic constructions you would use, in what order, and with which numbers.  
\begin{teachingnote}
This problem is about ``Seeing structure in expressions,'' one of the mathematical practices from the Common Core State Standards.
\end{teachingnote}
\begin{enumerate}
\item $\frac{7}{5}$
\item Any rational number, $p/q$
\item $3+2\sqrt{5}$
\item $\frac{3 + \sqrt{2-\sqrt{3}}}{1+\sqrt{5}}$
\end{enumerate}
\vspace{0.5in}
\end{problem}

\begin{problem}
Based on the previous problems, if you begin with a segment of length 1, describe the set of all numbers constructible with the methods used so far.   
\vfill
\end{problem}

\subsection*{Coordinate Constructions}
With the methods so far, we can construct neither $\sqrt[3]{2}$ nor $\pi$.  The question now is whether we have described the entire set of constructible numbers or whether there are additional constructions that will broaden our arithmetic and thereby enlarge the set.  

For this question, we turn to coordinate constructions, which allow us to use the methods of algebra to solve geometric problems.  A key habit here will be \textbf{imagining the algebra without actually doing it}---based on your extensive algebra experience with these kinds of problems.  

\newpage

\begin{problem}
Suppose you are given points $(p, q)$, and $(r, s)$ with integer coordinates.  
\begin{enumerate}
\item What arithmetic operations are involved in finding an equation $ax+by=c$ of the line containing these points?  
\item What can you conclude about the numbers $a$, $b$, and $c$? 
\item What if you begin with points that have coordinates that are rational numbers?  
\end{enumerate}
\vfill
\end{problem}

\begin{problem}
Suppose you are given equations of the form 
$$ax+by = c$$
$$dx+ey=f$$
where $a$, $b$, $c$, $d$, $e$, and $f$ are all integers.  
\begin{enumerate}
\item What kind of geometric objects do these equations describe in the $xy$-plane?  
\item What arithmetic operations would you use to solve the equations simultaneously? 
\item What can you conclude about the numbers $x$ and $y$ that are the (simultaneous) solutions of these equations?  
\item How will your answers change if $a$, $b$, $c$, $d$, $e$, and $f$ are all rational numbers?  
\end{enumerate}
\vfill
\end{problem}

\end{document}
