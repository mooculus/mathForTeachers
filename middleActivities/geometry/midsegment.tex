%\documentclass[handout]{ximera}
\documentclass[nooutcomes]{ximera}


\graphicspath{
  {./}
  {graphics/}
  {../graphics/}
}

\usepackage{chngcntr}

\let\question\relax
\let\endquestion\relax




\newtheoremstyle{SlantTheorem}{\topsep}{\fill}%%% space between body and thm
%\newtheoremstyle{SlantTheorem}{\topsep}{\topsep}%%% space between body and thm
 {\slshape}                      %%% Thm body font
 {}                              %%% Indent amount (empty = no indent)
 {\bfseries\sffamily}            %%% Thm head font
 {}                              %%% Punctuation after thm head
 {3ex}                           %%% Space after thm head
 {\thmname{#1}\thmnumber{ #2}\thmnote{ \bfseries(#3)}}%%% Thm head spec
\theoremstyle{SlantTheorem}
\newtheorem{question}{Question}
\counterwithin*{question}{section}



\let\instructorNotes\relax
\let\endinstructorNotes\relax
%%% instructorNotes environment
\ifhandout
\newenvironment{instructorNotes}[1][false]%
{%
\def\givenatend{\boolean{#1}}\ifthenelse{\boolean{#1}}{\begin{trivlist}\item}{\setbox0\vbox\bgroup}{}
}
{%
\ifthenelse{\givenatend}{\end{trivlist}}{\egroup}{}
}
\else
\newenvironment{instructorNotes}[1][false]%
{%
  \ifthenelse{\boolean{#1}}{\begin{trivlist}\item[\hskip \labelsep\bfseries {\Large Instructor Notes: \\} \hspace{\textwidth} ]}
{\begin{trivlist}\item[\hskip \labelsep\bfseries {\Large Instructor Notes: \\} \hspace{\textwidth} ]}
{}
}
{\end{trivlist}}
\fi


%% Suggested Timing
\newcommand{\timing}[1]{{\bf Suggested Timing: \hspace{2ex}} #1}

\title{Midsegments}
\author{Bart Snapp and Brad Findell}

\outcome{Learning outcome goes here.}

\begin{document}
\begin{abstract}
  We prove the medsegment theorem.
\end{abstract}
\maketitle

\begin{teachingnote}
Encourage both traditional and transformational proofs.  Don't use similarity here.
\end{teachingnote}

\begin{definition}
In a triangle, a \emph{midsegment} is a line joining the midpoints of two sides.  
\end{definition}

\begin{theorem}
Midsegment Theorem:  A midsegment in a triangle is parallel to and half the length of the corresponding side.
\end{theorem}

In this activity, we prove the midsegment theorem.  First, we need some results about parallelograms. 

\begin{problem}
Prove the following theorem:  If the diagonals of a quadrilateral bisect each other, then the quadrilateral is a parallelogram. 
\vfill
\end{problem}

\begin{problem}
Prove the following theorem:  If one pair of sides of a quadrilateral are congruent and parallel, then the quadrilateral is a parallelogram. 
\vfill
\end{problem}

\newpage
\begin{problem}
Prove the midsegment theorem.  (Hint:  Extend the midsegment $\overline{DE}$ to a point $X$ such that $EX=DE$, and then find quadrilaterals that must be parallelograms by the previous results.)  
\begin{image}
\includegraphics[scale=0.7]{midsegment1.pdf}
\end{image}
\vfill
\end{problem}

\end{document}
