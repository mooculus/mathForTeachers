\newpage
\section{Composing Transformations}

In this activity, we use the following notation:  
\begin{itemize}
\item $R_\theta$ denotes a counterclockwise rotation by $\theta$ about the origin.
\item $F_\ell$ denotes a reflection about the line $\ell$.  
\item $T_{(a,b)}$ denotes a translation by a vector $(a,b)$.  
\end{itemize}


\begin{prob}   Pick a specific point.  Use matrices to find the image of your point under the following sequences of transformations: 
\begin{enumerate}
\item $R_{90}$ followed by $F_{y=x}$
\item $F_{y=x}$  followed by $R_{90}$ 
\end{enumerate}
\end{prob}

\begin{prob} Repeat problem 1 for a general point.  
\end{prob}

\begin{prob}
Reason geometrically to identify a single transformation that accomplishes the sequences from problem 1.  (Hint:  Do the transformations physically with a square piece of paper marked with ``FRONT'' on the side that starts facing you.)
\end{prob}

\begin{prob}
Use your answers to the previous problems to write a single matrix for each of the sequences of transformations in problem 1.  
\end{prob}

\begin{prob}
Compute the following products of matrices:  
\begin{enumerate}
\item $R_{90} F_{y=x}$
\item $F_{y=x} R_{90}$ 
\end{enumerate}
\end{prob}

\begin{prob} What do you notice about your previous answers?  Explain why it works that way.  Hint: Without actually doing the computations, write a matrix expression that represents the result of each of the sequences of transformations.  
\end{prob}
