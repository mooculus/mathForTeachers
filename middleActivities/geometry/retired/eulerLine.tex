\newpage
\section{The Euler Line and the Nine-Point Circle} 

\begin{prob} 
Use \textsl{GeoGebra} to make the following constructions on an arbitrary triangle.  
\begin{itemize}
\itemsep -3pt
\item Construct the circumcenter of the 
triangle. Hide all extraneous lines and points. Label this point $C$.
\item Construct the centroid of the same triangle. Hide
all extraneous lines and points. Label this point $N$.
\item Construct the orthocenter of the same triangle. Hide
all extraneous lines and points. Label this point $O$.
\item Connect $C$ and $O$ with a segment. 
\end{itemize}
Did a miracle happen?  Describe what you notice about the 
segments $\overline{CN}$ and $\overline{ON}$.    
\end{prob}
\vfill
\begin{prob}
Keeping the same triangle as used in the previous problem, use \textsl{GeoGebra} to make the following construction:  
\begin{itemize}
\itemsep -3pt
\item Mark the midpoint of the segment that connects $C$ and $O$. Label this point $M$.
\item Mark the midpoints of each side. (Hint: Try to ``unhide'' those you have used already.)
\item Mark where the altitudes
meet the lines containing the sides of the triangle. Hide all
extraneous lines and points.
\item Mark the midpoints of
the segments joining the orthocenter and the vertices. Hide all
extraneous lines and points.
\item Draw a circle centered at $M$ that goes through one of the midpoints of the triangle.
\end{itemize}
 Did a miracle happen?  Describe what you notice.  
\end{prob}
\vfill
\begin{prob}
Complete the following sentences:  
\begin{enumerate}
\item The \emph{Euler line} contains the following points:  
%\vspace{.5in}
\vfill
\item The \emph{nine-point circle} contains the following points:  
\end{enumerate}
\end{prob}
\vfill
