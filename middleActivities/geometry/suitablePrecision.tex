%\documentclass[handout]{ximera}
\documentclass{ximera}

\usepackage{gensymb}
\usepackage{tabularx}
\usepackage{mdframed}
\usepackage{pdfpages}
%\usepackage{chngcntr}

\let\problem\relax
\let\endproblem\relax

\newcommand{\property}[2]{#1#2}




\newtheoremstyle{SlantTheorem}{\topsep}{\fill}%%% space between body and thm
 {\slshape}                      %%% Thm body font
 {}                              %%% Indent amount (empty = no indent)
 {\bfseries\sffamily}            %%% Thm head font
 {}                              %%% Punctuation after thm head
 {3ex}                           %%% Space after thm head
 {\thmname{#1}\thmnumber{ #2}\thmnote{ \bfseries(#3)}} %%% Thm head spec
\theoremstyle{SlantTheorem}
\newtheorem{problem}{Problem}[]

%\counterwithin*{problem}{section}



%%%%%%%%%%%%%%%%%%%%%%%%%%%%Jenny's code%%%%%%%%%%%%%%%%%%%%

%%% Solution environment
%\newenvironment{solution}{
%\ifhandout\setbox0\vbox\bgroup\else
%\begin{trivlist}\item[\hskip \labelsep\small\itshape\bfseries Solution\hspace{2ex}]
%\par\noindent\upshape\small
%\fi}
%{\ifhandout\egroup\else
%\end{trivlist}
%\fi}
%
%
%%% instructorIntro environment
%\ifhandout
%\newenvironment{instructorIntro}[1][false]%
%{%
%\def\givenatend{\boolean{#1}}\ifthenelse{\boolean{#1}}{\begin{trivlist}\item}{\setbox0\vbox\bgroup}{}
%}
%{%
%\ifthenelse{\givenatend}{\end{trivlist}}{\egroup}{}
%}
%\else
%\newenvironment{instructorIntro}[1][false]%
%{%
%  \ifthenelse{\boolean{#1}}{\begin{trivlist}\item[\hskip \labelsep\bfseries Instructor Notes:\hspace{2ex}]}
%{\begin{trivlist}\item[\hskip \labelsep\bfseries Instructor Notes:\hspace{2ex}]}
%{}
%}
%% %% line at the bottom} 
%{\end{trivlist}\par\addvspace{.5ex}\nobreak\noindent\hung} 
%\fi
%
%


\let\instructorNotes\relax
\let\endinstructorNotes\relax
%%% instructorNotes environment
\ifhandout
\newenvironment{instructorNotes}[1][false]%
{%
\def\givenatend{\boolean{#1}}\ifthenelse{\boolean{#1}}{\begin{trivlist}\item}{\setbox0\vbox\bgroup}{}
}
{%
\ifthenelse{\givenatend}{\end{trivlist}}{\egroup}{}
}
\else
\newenvironment{instructorNotes}[1][false]%
{%
  \ifthenelse{\boolean{#1}}{\begin{trivlist}\item[\hskip \labelsep\bfseries {\Large Instructor Notes: \\} \hspace{\textwidth} ]}
{\begin{trivlist}\item[\hskip \labelsep\bfseries {\Large Instructor Notes: \\} \hspace{\textwidth} ]}
{}
}
{\end{trivlist}}
\fi


%% Suggested Timing
\newcommand{\timing}[1]{{\bf Suggested Timing: \hspace{2ex}} #1}




\hypersetup{
    colorlinks=true,       % false: boxed links; true: colored links
    linkcolor=blue,          % color of internal links (change box color with linkbordercolor)
    citecolor=green,        % color of links to bibliography
    filecolor=magenta,      % color of file links
    urlcolor=cyan           % color of external links
}

\title{Suitable Precision in Language and Notation}
\author{Bart Snapp and Brad Findell}

\outcome{Learning outcome goes here.}

\begin{document}
\begin{abstract}
Abstract goes here.  
\end{abstract}
\maketitle

%\begin{teachingnote}
%\end{teachingnote}

Geometry is about points, lines, and other figures made up of points.  Points can have coordinates, which are numbers, but we save these approaches for later in the course.  

Even without coordinates, geometry involves numbers, especially as measures of lengths, angles, and areas.  

\begin{problem}
Let $M$ be the midpoint of $\overline{AB}$.   
\[
\definecolor{qqqqff}{rgb}{0.,0.,1.}
\begin{tikzpicture}[line cap=round,line join=round,>=triangle 45,x=1.0cm,y=1.0cm]
\clip(-0.5,-0.4) rectangle (5,0.15);
\draw [line width=0.8pt] (0.,0.)-- (4.,0.);
%\draw [line width=0.8pt] (2.,0.)-- (1.,0.);
\draw [fill=qqqqff] (0.,0.) circle (1.2pt);
\draw[color=qqqqff] (-0.16,-0.17) node {$A$};
\draw [fill=qqqqff] (4.,0.) circle (1.2pt);
\draw[color=qqqqff] (4.14,-0.17) node {$B$};
\draw [fill=qqqqff] (2.,0.) circle (1.2pt);
\draw[color=qqqqff] (2,-0.20) node {$M$};
\end{tikzpicture}
\]
Which of the following are true?  Explain.
\begin{enumerate}
\item $\overline{AB} = \overline{BA}$
\item $AB = BA$
\item $\overline{AM} = \overline{MB}$
\item $AM = MB$
\end{enumerate}
\end{problem}

\begin{problem}
Describe the geometric distinction between a segment and its length.  How are the two usually denoted differently?  
\end{problem}

\vspace{.8in}


\begin{problem}
Compare $\angle CAB$ and $\angle FDE$ in the figure below.  
\[
\definecolor{qqqqff}{rgb}{0.,0.,1.}
\begin{tikzpicture}[line cap=round,line join=round,>=triangle 45,x=1.0cm,y=1.0cm]
\clip(-0.5,-.5) rectangle (14,5.5);
\draw [line width=0.8pt] (1.,3.5)-- (0.,1.);
\draw [line width=0.8pt] (0.,1.)-- (3.,2.);
\draw [line width=0.8pt] (5.,0.)-- (11.,2.);
\draw [line width=0.8pt] (7.,5.)-- (5.,0.);
%\draw [fill=qqqqff] (0.,1.) circle (1.2pt);
\draw[color=qqqqff] (-0.22,0.89) node {$A$};
%\draw [fill=qqqqff] (3.,2.) circle (1.2pt);
\draw[color=qqqqff] (3.29,1.9) node {$B$};
%\draw [fill=qqqqff] (1.,3.5) circle (1.2pt);
\draw[color=qqqqff] (0.91,3.85) node {$C$};
%\draw [fill=qqqqff] (5.,0.) circle (1.2pt);
\draw[color=qqqqff] (4.80,-0.15) node {$D$};
%\draw [fill=qqqqff] (11.,2.) circle (1.2pt);
\draw[color=qqqqff] (11.25,1.95) node {$E$};
%\draw [fill=qqqqff] (7.,5.) circle (1.2pt);
\draw[color=qqqqff] (6.97,5.35) node {$F$};
\end{tikzpicture}
\]
Which of the following are true?  Explain. 
\begin{enumerate}
\item $\angle CAB = \angle BAC$
\item $\angle CAB = \angle FDE$
\item $m\angle CAB < m\angle FDE$
\item $m\angle CAB = m\angle FDE$
\end{enumerate}
\end{problem}


\begin{problem}
There are (at least) two ways of thinking about angles.  
\begin{enumerate}
\item Use precise language to describe an angle as a set of points.  
\vspace{.4in}
\item Use precise language to describe an angle as an amount of turning.  
\end{enumerate}
\vspace{.4in}
\end{problem}
\begin{teachingnote}
An angle is the union of two rays with a common endpoint, which is called the vertex of the angle.  The vertex of an angle can also be considered the center of a rotation that would map one ray that the other.  
\end{teachingnote}

\begin{problem}
Describe the geometric distinction between an angle and its measure.  How are the two usually denoted differently?  And how do your answers relate to the previous problem?  
\end{problem}
\vspace{.8in}

\begin{problem}
Use your meanings for angles to improve upon the following imprecise statements. 

\vspace{0.15in}

{\renewcommand{\arraystretch}{1.5}
\begin{tabular}{|>{\centering\arraybackslash}m{4cm}|>{\centering\arraybackslash}m{9.5cm}|}\hline
Statement & Improved Version  \\\hline

\rule{0pt}{1cm}A triangle has $180^\circ$. &  \\ \hline

\rule{0pt}{1cm}A line measures $180^\circ$. &  \\ \hline

\rule{0pt}{1cm}A circle is (or has) $360^\circ$. &  \\ \hline
 \hline
\end{tabular}}
\end{problem}

\begin{teachingnote}
\begin{itemize}
\itemsep0em
\item Let students struggle to figure out what is imprecise about the statements in the problem:  degrees measure angles, not lines, not triangles, not circles. 
\item The three vertices of the triangle are vertices of the three interior angles to be measured (and then summed).  
\item As a set of points, a straight angle is a line.  But a line is not a straight angle because an angle requires a vertex.  On a line, any point may be considered the vertex of a straight angle.
\item For a circle, we need its center, which is the vertex of central angles 
that can sum to $360^\circ$.  
\item Define right angle without using degrees: Two lines intersect to form four congruent angles.  Or two congruent angles that form a straight angle.
\end{itemize}  
\end{teachingnote}

\end{document}


