%\documentclass[handout]{ximera}
\documentclass{ximera}


\graphicspath{
  {./}
  {graphics/}
  {../graphics/}
}

\usepackage{chngcntr}

\let\question\relax
\let\endquestion\relax




\newtheoremstyle{SlantTheorem}{\topsep}{\fill}%%% space between body and thm
%\newtheoremstyle{SlantTheorem}{\topsep}{\topsep}%%% space between body and thm
 {\slshape}                      %%% Thm body font
 {}                              %%% Indent amount (empty = no indent)
 {\bfseries\sffamily}            %%% Thm head font
 {}                              %%% Punctuation after thm head
 {3ex}                           %%% Space after thm head
 {\thmname{#1}\thmnumber{ #2}\thmnote{ \bfseries(#3)}}%%% Thm head spec
\theoremstyle{SlantTheorem}
\newtheorem{question}{Question}
\counterwithin*{question}{section}



\let\instructorNotes\relax
\let\endinstructorNotes\relax
%%% instructorNotes environment
\ifhandout
\newenvironment{instructorNotes}[1][false]%
{%
\def\givenatend{\boolean{#1}}\ifthenelse{\boolean{#1}}{\begin{trivlist}\item}{\setbox0\vbox\bgroup}{}
}
{%
\ifthenelse{\givenatend}{\end{trivlist}}{\egroup}{}
}
\else
\newenvironment{instructorNotes}[1][false]%
{%
  \ifthenelse{\boolean{#1}}{\begin{trivlist}\item[\hskip \labelsep\bfseries {\Large Instructor Notes: \\} \hspace{\textwidth} ]}
{\begin{trivlist}\item[\hskip \labelsep\bfseries {\Large Instructor Notes: \\} \hspace{\textwidth} ]}
{}
}
{\end{trivlist}}
\fi


%% Suggested Timing
\newcommand{\timing}[1]{{\bf Suggested Timing: \hspace{2ex}} #1}

\title{Midsets Abound}
\author{Bart Snapp and Brad Findell}

\outcome{Learning outcome goes here.}

\begin{document}
\begin{abstract}
Abstract goes here.  
\end{abstract}
\maketitle
                                  

In this activity we are going to investigate \textit{midsets}.

\begin{definition}\index{midset}
Given two points $A$ and $B$, their \textbf{midset} is the set of points that are an equal distance away from both $A$ and $B$.
\end{definition}

\begin{problem} 
Draw two points in the plane $A$ and $B$. See if you can sketch the
Euclidean midset of these two points.
\end{problem}

\begin{problem}
See if you can use coordinate constructions to find the equation of
the midset of two points $A$ and $B$. If necessary, set $A = (2,3)$
and $B = (5,7)$.
\end{problem}

\newpage
\begin{teachingnote}
Use specific pairs of points and smaller grids.  Bring extra grids.
\end{teachingnote}  
\begin{problem}
Now working in city geometry, place two points and see if you can find
their midset.
\begin{image}
\includegraphics{complexPlane}
\end{image}
\end{problem}

\newpage
\begin{problem}
Let's try to classify the various midsets in city geometry:
\begin{image}
\includegraphics{complexPlane}
\end{image}
\end{problem}

\end{document}
