%\documentclass[handout]{ximera}
\documentclass{ximera}

\usepackage{gensymb}
\usepackage{tabularx}
\usepackage{mdframed}
\usepackage{pdfpages}
%\usepackage{chngcntr}

\let\problem\relax
\let\endproblem\relax

\newcommand{\property}[2]{#1#2}




\newtheoremstyle{SlantTheorem}{\topsep}{\fill}%%% space between body and thm
 {\slshape}                      %%% Thm body font
 {}                              %%% Indent amount (empty = no indent)
 {\bfseries\sffamily}            %%% Thm head font
 {}                              %%% Punctuation after thm head
 {3ex}                           %%% Space after thm head
 {\thmname{#1}\thmnumber{ #2}\thmnote{ \bfseries(#3)}} %%% Thm head spec
\theoremstyle{SlantTheorem}
\newtheorem{problem}{Problem}[]

%\counterwithin*{problem}{section}



%%%%%%%%%%%%%%%%%%%%%%%%%%%%Jenny's code%%%%%%%%%%%%%%%%%%%%

%%% Solution environment
%\newenvironment{solution}{
%\ifhandout\setbox0\vbox\bgroup\else
%\begin{trivlist}\item[\hskip \labelsep\small\itshape\bfseries Solution\hspace{2ex}]
%\par\noindent\upshape\small
%\fi}
%{\ifhandout\egroup\else
%\end{trivlist}
%\fi}
%
%
%%% instructorIntro environment
%\ifhandout
%\newenvironment{instructorIntro}[1][false]%
%{%
%\def\givenatend{\boolean{#1}}\ifthenelse{\boolean{#1}}{\begin{trivlist}\item}{\setbox0\vbox\bgroup}{}
%}
%{%
%\ifthenelse{\givenatend}{\end{trivlist}}{\egroup}{}
%}
%\else
%\newenvironment{instructorIntro}[1][false]%
%{%
%  \ifthenelse{\boolean{#1}}{\begin{trivlist}\item[\hskip \labelsep\bfseries Instructor Notes:\hspace{2ex}]}
%{\begin{trivlist}\item[\hskip \labelsep\bfseries Instructor Notes:\hspace{2ex}]}
%{}
%}
%% %% line at the bottom} 
%{\end{trivlist}\par\addvspace{.5ex}\nobreak\noindent\hung} 
%\fi
%
%


\let\instructorNotes\relax
\let\endinstructorNotes\relax
%%% instructorNotes environment
\ifhandout
\newenvironment{instructorNotes}[1][false]%
{%
\def\givenatend{\boolean{#1}}\ifthenelse{\boolean{#1}}{\begin{trivlist}\item}{\setbox0\vbox\bgroup}{}
}
{%
\ifthenelse{\givenatend}{\end{trivlist}}{\egroup}{}
}
\else
\newenvironment{instructorNotes}[1][false]%
{%
  \ifthenelse{\boolean{#1}}{\begin{trivlist}\item[\hskip \labelsep\bfseries {\Large Instructor Notes: \\} \hspace{\textwidth} ]}
{\begin{trivlist}\item[\hskip \labelsep\bfseries {\Large Instructor Notes: \\} \hspace{\textwidth} ]}
{}
}
{\end{trivlist}}
\fi


%% Suggested Timing
\newcommand{\timing}[1]{{\bf Suggested Timing: \hspace{2ex}} #1}




\hypersetup{
    colorlinks=true,       % false: boxed links; true: colored links
    linkcolor=blue,          % color of internal links (change box color with linkbordercolor)
    citecolor=green,        % color of links to bibliography
    filecolor=magenta,      % color of file links
    urlcolor=cyan           % color of external links
}

\title{Parametric Plots of Circles}
\author{Bart Snapp and Brad Findell}

\outcome{Learning outcome goes here.}

\begin{document}
\begin{abstract}
We explore parametric plots of circles. 
\end{abstract}
\maketitle

In this activity we'll investigate parametric plots of circles.


\begin{problem} 
One problem with the standard form for a circle, even the form for the unit circle
\[
x^2 + y^2 = 1,
\]
is that it is somewhat difficult to find points on the circle. We
claim that for any value of $t$,
\begin{align*}
x(t) &= \cos(t)\\
y(t) &= \sin(t) 
\end{align*}
will be a point on the unit circle. Can you give me some explanation
as to why this is true? Two hints, for two answers: The unit circle;
The Pythagorean identity.
\end{problem} 

\begin{problem}
Another way to think about parametric formulas for circles is to imagine 
\begin{align*}
x(\theta) &= \cos(\theta)\\
y(\theta) &= \sin(\theta) 
\end{align*}
where $\theta$ is an angle. What is the connection between value of
$\theta$ and the point $(x(\theta), y(\theta))$?
\end{problem}

\begin{problem}
One way to think about parametric formulas for circles is to imagine 
\begin{align*}
x(t) &= \cos(t)\\
y(t) &= \sin(t) 
\end{align*}
as ``drawing'' the circle as $t$ changes. Starting with $t=0$,
describe how the circle is ``drawn.''  Make a table of values of $t$, $x$, and $y$.  Use values of $t$ that are special angles.  Includes values of $t$ that are negative as well as some values of $t$ that are greater than $2\pi$.  
\end{problem}

\begin{problem}
One day you accidentally write down
\begin{align*}
x(t) &= \sin(t)\\
y(t) &= \cos(t) 
\end{align*}
Again, make a table of values of $t$, $x$, and $y$
What happens now? Do you still get a circle? How is this different
from what we did in the previous question?
\end{problem}

\begin{problem}
Do the formulas 
\begin{align*}
x(t) &= \cos(t)\\
y(t) &= \sin(t) 
\end{align*}
define a function? Discuss. Clearly identify the
domain and range as part of your discussion.  Remember, the domain is the set of input values and the range is the set of output values.  
\end{problem}

\begin{problem}
Reason with your previous tables of $x$- and $y$-values to determine the graph of the following parametric equations. 
\begin{align*}
x(t) &= 2\cos(t) + 3\\
y(t) &= 2\sin(t) - 4 
\end{align*}
Explain your reasoning.  
\end{problem}

\begin{problem} 
Now we will go backwards.  The standard form for a circle centered at a point $(a,b)$ with radius $c$ is given by
\[
(x-a)^2 + (y-b)^2 = r^2.
\]
Explain why this makes perfect sense from the definition of a circle. 
\end{problem} 


\begin{problem}
Here are three circles
\[
(x-1)^2 + (y+2)^2 = 4^2 \qquad (x+4)^2 + (y-2)^2 = 8 \qquad x^2+y^2 -4x+6y= 12.
\]
Convert each of these circles to parametric form.
\end{problem}

\end{document}
