%\documentclass[handout]{ximera}
\documentclass[nooutcomes]{ximera}

\usepackage{gensymb}
\usepackage{tabularx}
\usepackage{mdframed}
\usepackage{pdfpages}
%\usepackage{chngcntr}

\let\problem\relax
\let\endproblem\relax

\newcommand{\property}[2]{#1#2}




\newtheoremstyle{SlantTheorem}{\topsep}{\fill}%%% space between body and thm
 {\slshape}                      %%% Thm body font
 {}                              %%% Indent amount (empty = no indent)
 {\bfseries\sffamily}            %%% Thm head font
 {}                              %%% Punctuation after thm head
 {3ex}                           %%% Space after thm head
 {\thmname{#1}\thmnumber{ #2}\thmnote{ \bfseries(#3)}} %%% Thm head spec
\theoremstyle{SlantTheorem}
\newtheorem{problem}{Problem}[]

%\counterwithin*{problem}{section}



%%%%%%%%%%%%%%%%%%%%%%%%%%%%Jenny's code%%%%%%%%%%%%%%%%%%%%

%%% Solution environment
%\newenvironment{solution}{
%\ifhandout\setbox0\vbox\bgroup\else
%\begin{trivlist}\item[\hskip \labelsep\small\itshape\bfseries Solution\hspace{2ex}]
%\par\noindent\upshape\small
%\fi}
%{\ifhandout\egroup\else
%\end{trivlist}
%\fi}
%
%
%%% instructorIntro environment
%\ifhandout
%\newenvironment{instructorIntro}[1][false]%
%{%
%\def\givenatend{\boolean{#1}}\ifthenelse{\boolean{#1}}{\begin{trivlist}\item}{\setbox0\vbox\bgroup}{}
%}
%{%
%\ifthenelse{\givenatend}{\end{trivlist}}{\egroup}{}
%}
%\else
%\newenvironment{instructorIntro}[1][false]%
%{%
%  \ifthenelse{\boolean{#1}}{\begin{trivlist}\item[\hskip \labelsep\bfseries Instructor Notes:\hspace{2ex}]}
%{\begin{trivlist}\item[\hskip \labelsep\bfseries Instructor Notes:\hspace{2ex}]}
%{}
%}
%% %% line at the bottom} 
%{\end{trivlist}\par\addvspace{.5ex}\nobreak\noindent\hung} 
%\fi
%
%


\let\instructorNotes\relax
\let\endinstructorNotes\relax
%%% instructorNotes environment
\ifhandout
\newenvironment{instructorNotes}[1][false]%
{%
\def\givenatend{\boolean{#1}}\ifthenelse{\boolean{#1}}{\begin{trivlist}\item}{\setbox0\vbox\bgroup}{}
}
{%
\ifthenelse{\givenatend}{\end{trivlist}}{\egroup}{}
}
\else
\newenvironment{instructorNotes}[1][false]%
{%
  \ifthenelse{\boolean{#1}}{\begin{trivlist}\item[\hskip \labelsep\bfseries {\Large Instructor Notes: \\} \hspace{\textwidth} ]}
{\begin{trivlist}\item[\hskip \labelsep\bfseries {\Large Instructor Notes: \\} \hspace{\textwidth} ]}
{}
}
{\end{trivlist}}
\fi


%% Suggested Timing
\newcommand{\timing}[1]{{\bf Suggested Timing: \hspace{2ex}} #1}




\hypersetup{
    colorlinks=true,       % false: boxed links; true: colored links
    linkcolor=blue,          % color of internal links (change box color with linkbordercolor)
    citecolor=green,        % color of links to bibliography
    filecolor=magenta,      % color of file links
    urlcolor=cyan           % color of external links
}

\title{Isosceles Bisectors}
\author{Bart Snapp and Brad Findell}

\outcome{Learning outcome goes here.}

\begin{document}
\begin{abstract}
  We think about isosceles triangles.
\end{abstract}
\maketitle

\begin{teachingnote}
Students typically draw a new line. They must choose which line they are drawing and then they must assume no additional properties.  Then they look for triangle congruence.  Median: SSS.  Altitude: HL (Hypotenuse-Leg).  Angle bisector: SAS. Perpendicular bisector: doesn't work because it might not contain the opposite vertex.  

Work through analogous ideas for the converse. 
\end{teachingnote}

\begin{theorem}[Isosceles Triangle Theorem]
If two sides of a triangle are congruent, then the angles opposite those sides are congruent. 
\end{theorem}

\begin{problem}
Prove the Isosceles Triangle Theorem.  (Hint: In a previous activity, you noticed that in most triangles the median, perpendicular bisector, angle bisector, and altitude to a side lie on four different lines.  So if you draw a new line in your diagram, be sure to decide which of these lines you are drawing.)
\vfill
\end{problem}

\begin{problem}
Use your proof to show that, in an isosceles triangle, a median, perpendicular bisector, angle bisector, and altitude turn out to be the same line.
\vfill
\end{problem}

\newpage
\begin{problem}
Prove the Isosceles Triangle Theorem without drawing another line.  Hint:  Is there a way in which the triangle is congruent to itself? 
\vfill
\end{problem}

\begin{problem}
State the converse of the Isosceles Triangle Theorem and prove it.  
\vfill
\end{problem}

\newpage
\begin{problem}
Prove that the points on the perpendicular bisector of a segment are \emph{exactly those} that are equidistant from the endpoints of the segment.  Note that the phrase \emph{exactly those} requires that we prove a simpler statement as well as its converse:   
\begin{enumerate}
\item Prove that a point on the perpendicular bisector of a segment is equidistant from the endpoints of that segment.
\item Prove that a point that is equidistant from the endpoints of a segment lies on the perpendicular bisector of that segment.
\end{enumerate}
\vfill
\end{problem}

\begin{problem}
Prove that the perpendicular bisectors of a triangle are concurrent.  Hint:  Name the intersection of two of the perpendicular bisectors and then show that it must also lie on the third one.  (This is a standard approach for showing the concurrency of three lines.)  
\vfill
\end{problem}

\newpage
\begin{problem}
Draw a line (neither horizontal nor vertical) and a point not on the line.  Describe how to find the \emph{exact} distance from the point to the line. 
\vfill
\end{problem}

\begin{problem}
Prove that the points on an angle bisector are \emph{exactly those} that are equidistant from the sides of the angle. 
\vfill
\end{problem}

\begin{problem}
Prove that the angle bisectors of a triangle are concurrent. 
\vfill
\end{problem}
\end{document}
