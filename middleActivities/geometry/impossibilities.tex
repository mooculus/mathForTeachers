%\documentclass[handout]{ximera}
\documentclass[nooutcomes]{ximera}

\usepackage{gensymb}
\usepackage{tabularx}
\usepackage{mdframed}
\usepackage{pdfpages}
%\usepackage{chngcntr}

\let\problem\relax
\let\endproblem\relax

\newcommand{\property}[2]{#1#2}




\newtheoremstyle{SlantTheorem}{\topsep}{\fill}%%% space between body and thm
 {\slshape}                      %%% Thm body font
 {}                              %%% Indent amount (empty = no indent)
 {\bfseries\sffamily}            %%% Thm head font
 {}                              %%% Punctuation after thm head
 {3ex}                           %%% Space after thm head
 {\thmname{#1}\thmnumber{ #2}\thmnote{ \bfseries(#3)}} %%% Thm head spec
\theoremstyle{SlantTheorem}
\newtheorem{problem}{Problem}[]

%\counterwithin*{problem}{section}



%%%%%%%%%%%%%%%%%%%%%%%%%%%%Jenny's code%%%%%%%%%%%%%%%%%%%%

%%% Solution environment
%\newenvironment{solution}{
%\ifhandout\setbox0\vbox\bgroup\else
%\begin{trivlist}\item[\hskip \labelsep\small\itshape\bfseries Solution\hspace{2ex}]
%\par\noindent\upshape\small
%\fi}
%{\ifhandout\egroup\else
%\end{trivlist}
%\fi}
%
%
%%% instructorIntro environment
%\ifhandout
%\newenvironment{instructorIntro}[1][false]%
%{%
%\def\givenatend{\boolean{#1}}\ifthenelse{\boolean{#1}}{\begin{trivlist}\item}{\setbox0\vbox\bgroup}{}
%}
%{%
%\ifthenelse{\givenatend}{\end{trivlist}}{\egroup}{}
%}
%\else
%\newenvironment{instructorIntro}[1][false]%
%{%
%  \ifthenelse{\boolean{#1}}{\begin{trivlist}\item[\hskip \labelsep\bfseries Instructor Notes:\hspace{2ex}]}
%{\begin{trivlist}\item[\hskip \labelsep\bfseries Instructor Notes:\hspace{2ex}]}
%{}
%}
%% %% line at the bottom} 
%{\end{trivlist}\par\addvspace{.5ex}\nobreak\noindent\hung} 
%\fi
%
%


\let\instructorNotes\relax
\let\endinstructorNotes\relax
%%% instructorNotes environment
\ifhandout
\newenvironment{instructorNotes}[1][false]%
{%
\def\givenatend{\boolean{#1}}\ifthenelse{\boolean{#1}}{\begin{trivlist}\item}{\setbox0\vbox\bgroup}{}
}
{%
\ifthenelse{\givenatend}{\end{trivlist}}{\egroup}{}
}
\else
\newenvironment{instructorNotes}[1][false]%
{%
  \ifthenelse{\boolean{#1}}{\begin{trivlist}\item[\hskip \labelsep\bfseries {\Large Instructor Notes: \\} \hspace{\textwidth} ]}
{\begin{trivlist}\item[\hskip \labelsep\bfseries {\Large Instructor Notes: \\} \hspace{\textwidth} ]}
{}
}
{\end{trivlist}}
\fi


%% Suggested Timing
\newcommand{\timing}[1]{{\bf Suggested Timing: \hspace{2ex}} #1}




\hypersetup{
    colorlinks=true,       % false: boxed links; true: colored links
    linkcolor=blue,          % color of internal links (change box color with linkbordercolor)
    citecolor=green,        % color of links to bibliography
    filecolor=magenta,      % color of file links
    urlcolor=cyan           % color of external links
}

\title{Impossibilities}
\author{Bart Snapp and Brad Findell}

\outcome{Learning outcome goes here.}

\begin{document}
\begin{abstract}
  We investigate which numbers are constructible with compass and
  straightedge alone.
\end{abstract}
\maketitle

The idea that some numbers are not constructible is exactly what was
needed to address several problems first posed by the Greeks in
antiquity, such as doubling the cube and trisecting an angle.  In a
paper published in 1837, Pierre Wantzel used algebraic methods to
prove the impossibility of these geometric constructions.

\begin{problem}
Suppose you have a square of side length $s$ and you want to ``double the square.''  In other words, you want to construct a square with \textbf{twice the area}.  
\begin{enumerate}
\item What is the side length of the desired square?  Explain your reasoning. 
\item Is this side length constructible?  Explain.  
\end{enumerate}
\end{problem}

\begin{problem}
Suppose you have a cube of side length $s$ and you want to ``double the cube.''  In other words, you want to construct a cube with \textbf{twice the volume}.  
\begin{enumerate}
\item What is the side length of the desired cube?  Explain your reasoning. 
\item Is this side length constructible?  Explain.  
\end{enumerate}
\end{problem}

\begin{problem}
You may remember some double angle formulas from trigonometry.  There are also triple angle formulas.  For example, for any angle $\theta$,  
$\cos3\theta=4\cos^3\theta -3\cos\theta$.  
\begin{enumerate}
\item Write the above triple angle formula for $\theta = 20^\circ$.  %% solution of the equation $4x^3-3x=1/2$, 
\item Explain why $x = \cos20^\circ$ must be a root of the polynomial $8x^3-6x-1$.  
\item Explain how the rational root theorem implies that this polynomial has no linear factors.  
\item Explain why this polynomial must therefore be irreducible over the rational numbers.  
\item You may recall from Math 1165 that some methods of solving cubic equations involve extracting cube roots.  What does this imply about trisecting angles?  
\item You may recall, from earlier this semester, discussing a method for trisecting an angle with paper folding.  What does that method imply about the relationship between the numbers that are constructible by paper folding and those that are constructible by compass and straightedge?  Explain.  
\end{enumerate}
\end{problem}

\end{document}
