%\documentclass[handout]{ximera}
\documentclass[nooutcomes]{ximera}

\usepackage{gensymb}
\usepackage{tabularx}
\usepackage{mdframed}
\usepackage{pdfpages}
%\usepackage{chngcntr}

\let\problem\relax
\let\endproblem\relax

\newcommand{\property}[2]{#1#2}




\newtheoremstyle{SlantTheorem}{\topsep}{\fill}%%% space between body and thm
 {\slshape}                      %%% Thm body font
 {}                              %%% Indent amount (empty = no indent)
 {\bfseries\sffamily}            %%% Thm head font
 {}                              %%% Punctuation after thm head
 {3ex}                           %%% Space after thm head
 {\thmname{#1}\thmnumber{ #2}\thmnote{ \bfseries(#3)}} %%% Thm head spec
\theoremstyle{SlantTheorem}
\newtheorem{problem}{Problem}[]

%\counterwithin*{problem}{section}



%%%%%%%%%%%%%%%%%%%%%%%%%%%%Jenny's code%%%%%%%%%%%%%%%%%%%%

%%% Solution environment
%\newenvironment{solution}{
%\ifhandout\setbox0\vbox\bgroup\else
%\begin{trivlist}\item[\hskip \labelsep\small\itshape\bfseries Solution\hspace{2ex}]
%\par\noindent\upshape\small
%\fi}
%{\ifhandout\egroup\else
%\end{trivlist}
%\fi}
%
%
%%% instructorIntro environment
%\ifhandout
%\newenvironment{instructorIntro}[1][false]%
%{%
%\def\givenatend{\boolean{#1}}\ifthenelse{\boolean{#1}}{\begin{trivlist}\item}{\setbox0\vbox\bgroup}{}
%}
%{%
%\ifthenelse{\givenatend}{\end{trivlist}}{\egroup}{}
%}
%\else
%\newenvironment{instructorIntro}[1][false]%
%{%
%  \ifthenelse{\boolean{#1}}{\begin{trivlist}\item[\hskip \labelsep\bfseries Instructor Notes:\hspace{2ex}]}
%{\begin{trivlist}\item[\hskip \labelsep\bfseries Instructor Notes:\hspace{2ex}]}
%{}
%}
%% %% line at the bottom} 
%{\end{trivlist}\par\addvspace{.5ex}\nobreak\noindent\hung} 
%\fi
%
%


\let\instructorNotes\relax
\let\endinstructorNotes\relax
%%% instructorNotes environment
\ifhandout
\newenvironment{instructorNotes}[1][false]%
{%
\def\givenatend{\boolean{#1}}\ifthenelse{\boolean{#1}}{\begin{trivlist}\item}{\setbox0\vbox\bgroup}{}
}
{%
\ifthenelse{\givenatend}{\end{trivlist}}{\egroup}{}
}
\else
\newenvironment{instructorNotes}[1][false]%
{%
  \ifthenelse{\boolean{#1}}{\begin{trivlist}\item[\hskip \labelsep\bfseries {\Large Instructor Notes: \\} \hspace{\textwidth} ]}
{\begin{trivlist}\item[\hskip \labelsep\bfseries {\Large Instructor Notes: \\} \hspace{\textwidth} ]}
{}
}
{\end{trivlist}}
\fi


%% Suggested Timing
\newcommand{\timing}[1]{{\bf Suggested Timing: \hspace{2ex}} #1}




\hypersetup{
    colorlinks=true,       % false: boxed links; true: colored links
    linkcolor=blue,          % color of internal links (change box color with linkbordercolor)
    citecolor=green,        % color of links to bibliography
    filecolor=magenta,      % color of file links
    urlcolor=cyan           % color of external links
}

\title{Triangle Investigation}
\author{Bart Snapp and Brad Findell}

\outcome{Learning outcome goes here.}

\begin{document}
\begin{abstract}
  Find triangles given various conditions.
  

\end{abstract}
\maketitle

\begin{teachingnote}
Preactivity:  Do the first three of these at home.  The upshot is triangle congruence:  three measures are (often) enough.  

Some students will need to be reminded that, e.g., $B$ is vertex of $\angle ABC$.
\end{teachingnote}

\begin{problem}
Draw triangles satisfying the conditions given below.  You may use whatever tools you like (e.g., ruler, protractor, compass, sticks, tracing paper, or Geogebra).\margincomment{CCSS 7.G.2. Draw (freehand, with ruler and protractor, and with technology) geometric shapes with given conditions. Focus on constructing triangles from three measures of angles or sides, noticing when the conditions determine a unique triangle, more than one triangle, or no triangle.}  

In each part, use reasoning to determine whether the information provided determines a unique $\triangle ABC$, more than one triangle, or no triangle.%\standard{7.G.2}   

Note:  To check to see if two triangles are the same, attempt to lay one directly on top of the other.  

\begin{enumerate}

\item $AB = 4$ and $BC = 5$
\item $m\angle CAB = 25^\circ$, $m\angle ABC = 75^\circ$, $m\angle BCA = 80^\circ$
\item $m\angle CAB = 25^\circ$, $m\angle ABC = 65^\circ$, $m\angle BCA = 80^\circ$
\item $AB = 4$, $m\angle BAC = 30^\circ$, $m\angle ABC = 45^\circ$
\item $AB = 4$, $BC = 5$, $m\angle ABC = 60^\circ$
\item $BC = 7$, $CA = 8$, $AB = 9$
\item $BC = 4$, $CA = 8$, $AB = 3$
\item $m\angle ABC = 45^\circ$, $BC = 8$, $CA = 12$
\item $m\angle ABC = 30^\circ$, $BC = 10$, $CA = 7$
\item $m\angle ABC = 60^\circ$, $BC = 10$, $CA = 3$

\end{enumerate}

\end{problem}

\end{document}
