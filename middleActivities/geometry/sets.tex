%\documentclass[handout]{ximera}
\documentclass[nooutcomes]{ximera}

\usepackage{gensymb}
\usepackage{tabularx}
\usepackage{mdframed}
\usepackage{pdfpages}
%\usepackage{chngcntr}

\let\problem\relax
\let\endproblem\relax

\newcommand{\property}[2]{#1#2}




\newtheoremstyle{SlantTheorem}{\topsep}{\fill}%%% space between body and thm
 {\slshape}                      %%% Thm body font
 {}                              %%% Indent amount (empty = no indent)
 {\bfseries\sffamily}            %%% Thm head font
 {}                              %%% Punctuation after thm head
 {3ex}                           %%% Space after thm head
 {\thmname{#1}\thmnumber{ #2}\thmnote{ \bfseries(#3)}} %%% Thm head spec
\theoremstyle{SlantTheorem}
\newtheorem{problem}{Problem}[]

%\counterwithin*{problem}{section}



%%%%%%%%%%%%%%%%%%%%%%%%%%%%Jenny's code%%%%%%%%%%%%%%%%%%%%

%%% Solution environment
%\newenvironment{solution}{
%\ifhandout\setbox0\vbox\bgroup\else
%\begin{trivlist}\item[\hskip \labelsep\small\itshape\bfseries Solution\hspace{2ex}]
%\par\noindent\upshape\small
%\fi}
%{\ifhandout\egroup\else
%\end{trivlist}
%\fi}
%
%
%%% instructorIntro environment
%\ifhandout
%\newenvironment{instructorIntro}[1][false]%
%{%
%\def\givenatend{\boolean{#1}}\ifthenelse{\boolean{#1}}{\begin{trivlist}\item}{\setbox0\vbox\bgroup}{}
%}
%{%
%\ifthenelse{\givenatend}{\end{trivlist}}{\egroup}{}
%}
%\else
%\newenvironment{instructorIntro}[1][false]%
%{%
%  \ifthenelse{\boolean{#1}}{\begin{trivlist}\item[\hskip \labelsep\bfseries Instructor Notes:\hspace{2ex}]}
%{\begin{trivlist}\item[\hskip \labelsep\bfseries Instructor Notes:\hspace{2ex}]}
%{}
%}
%% %% line at the bottom} 
%{\end{trivlist}\par\addvspace{.5ex}\nobreak\noindent\hung} 
%\fi
%
%


\let\instructorNotes\relax
\let\endinstructorNotes\relax
%%% instructorNotes environment
\ifhandout
\newenvironment{instructorNotes}[1][false]%
{%
\def\givenatend{\boolean{#1}}\ifthenelse{\boolean{#1}}{\begin{trivlist}\item}{\setbox0\vbox\bgroup}{}
}
{%
\ifthenelse{\givenatend}{\end{trivlist}}{\egroup}{}
}
\else
\newenvironment{instructorNotes}[1][false]%
{%
  \ifthenelse{\boolean{#1}}{\begin{trivlist}\item[\hskip \labelsep\bfseries {\Large Instructor Notes: \\} \hspace{\textwidth} ]}
{\begin{trivlist}\item[\hskip \labelsep\bfseries {\Large Instructor Notes: \\} \hspace{\textwidth} ]}
{}
}
{\end{trivlist}}
\fi


%% Suggested Timing
\newcommand{\timing}[1]{{\bf Suggested Timing: \hspace{2ex}} #1}




\hypersetup{
    colorlinks=true,       % false: boxed links; true: colored links
    linkcolor=blue,          % color of internal links (change box color with linkbordercolor)
    citecolor=green,        % color of links to bibliography
    filecolor=magenta,      % color of file links
    urlcolor=cyan           % color of external links
}

\title{About Sets}
\author{Bart Snapp and Brad Findell}

\outcome{Learning outcome goes here.}

\begin{document}
\begin{abstract}
  We study sets, fundamental objects in mathematics.
\end{abstract}
\maketitle

In this activity, we remind ourselves of the language and notation of sets.  In school mathematics, we often talk about sets of numbers, sets of points, sets of geometric objects, sets of functions, and even sets of sets.  When listing elements of a set, we usually enclose them in curly brackets $\{\dots\}$, and separate them with commas. 


\begin{problem}
Let $A=\text{the set of divisors of 24}$, and let $B=\text{the set of divisors of 32}$.  
\begin{enumerate}
\item Use set notation to list the elements of $A$.  
\vfill
\item Use the the symbols $\in$ (is an element of) and $\subset$ (is a subset of) to make some true statements about set $A$. 
\vspace{0.3in}
\item Draw a Venn diagram showing sets $A$ and $B$ and the relationship between them.  
\vspace{1.5in}
\end{enumerate}
\end{problem}

\begin{problem}
The notation $A\cup B$ means the \emph{union} of sets $A$ and $B$, which is to say the set of elements that are in $A$ \textbf{or} in $B$.  (Note: In mathematics, the word ``or'' is used ``inclusively.'') 

$A\cup B = $ 
\vfill
\end{problem}

\begin{problem}
The notation $A\cap B$ means the \emph{intersection} of sets $A$ and $B$, which is to say the set of elements that are in $A$ \textbf{and} in $B$. 

$A\cap B = $
\vfill
\end{problem}

\begin{problem}
Suppose $C = \{5,7,13\}$ and $D = \{6,12\}$.  
\begin{enumerate}
\item What is $C\cap D$?  Does the term \emph{empty set} help?  How should it be notated?
\vspace{0.3in}
\item Two sets with an empty intersection are said to be \emph{disjoint}.  How might you notice disjoint sets on a Venn diagram?  
\vspace{0.5in}
\item Draw a Venn diagram showing sets $A$, $B$, $C$, and $D$. 
\vspace{1in}
\end{enumerate}
\end{problem}


\end{document}

