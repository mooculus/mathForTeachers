%\documentclass[handout]{ximera}
\documentclass[nooutcomes]{ximera}


\graphicspath{
  {./}
  {graphics/}
  {../graphics/}
}

\usepackage{chngcntr}

\let\question\relax
\let\endquestion\relax




\newtheoremstyle{SlantTheorem}{\topsep}{\fill}%%% space between body and thm
%\newtheoremstyle{SlantTheorem}{\topsep}{\topsep}%%% space between body and thm
 {\slshape}                      %%% Thm body font
 {}                              %%% Indent amount (empty = no indent)
 {\bfseries\sffamily}            %%% Thm head font
 {}                              %%% Punctuation after thm head
 {3ex}                           %%% Space after thm head
 {\thmname{#1}\thmnumber{ #2}\thmnote{ \bfseries(#3)}}%%% Thm head spec
\theoremstyle{SlantTheorem}
\newtheorem{question}{Question}
\counterwithin*{question}{section}



\let\instructorNotes\relax
\let\endinstructorNotes\relax
%%% instructorNotes environment
\ifhandout
\newenvironment{instructorNotes}[1][false]%
{%
\def\givenatend{\boolean{#1}}\ifthenelse{\boolean{#1}}{\begin{trivlist}\item}{\setbox0\vbox\bgroup}{}
}
{%
\ifthenelse{\givenatend}{\end{trivlist}}{\egroup}{}
}
\else
\newenvironment{instructorNotes}[1][false]%
{%
  \ifthenelse{\boolean{#1}}{\begin{trivlist}\item[\hskip \labelsep\bfseries {\Large Instructor Notes: \\} \hspace{\textwidth} ]}
{\begin{trivlist}\item[\hskip \labelsep\bfseries {\Large Instructor Notes: \\} \hspace{\textwidth} ]}
{}
}
{\end{trivlist}}
\fi


%% Suggested Timing
\newcommand{\timing}[1]{{\bf Suggested Timing: \hspace{2ex}} #1}

\title{Forget Something?}
\author{Bart Snapp and Brad Findell}

\outcome{Learning outcome goes here.}

\begin{document}
\begin{abstract}
  We study sets, fundamental objects in mathematics.
\end{abstract}
\maketitle


\begin{teachingnote}
It might help give students some concrete sets to try first.  On the Internet there are nice pictures of the four-set situation. Students are likely to see connections to the traffic lights problem from \emph{Numbers and Algebra}.
\end{teachingnote}

\begin{problem} 
Draw a Venn diagram with one set. List every possible relationship
between an element and this set. 
\vfill
\end{problem}

\begin{problem} 
Draw a Venn diagram with two intersecting sets. List every possible
relationship between an element and these sets.
\vfill
\end{problem}


\begin{problem} 
Draw a Venn diagram with three intersecting sets. List every possible
relationship between an element and these sets.
\vfill
\end{problem}

\newpage
\begin{problem}
Describe and explain any patterns you see occurring.
\vfill
\end{problem}

\begin{problem}\label{P:AVD}
Draw a Venn diagram with four intersecting sets. List every possible
relationship between an element and these sets.
\vfill
\end{problem}

\begin{problem}
Are you \textbf{sure} that your diagram for Problem~\ref{P:AVD} is
correct? If so explain why. If not, draw a correct Venn diagram.
\vfill
\end{problem}

\end{document}
