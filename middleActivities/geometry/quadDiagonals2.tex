\newpage

\section{Quadrilateral Diagonals}
\begin{teachingnote}
Supplies:  Fettuccini, scrap paper, 12-inch rulers.
\end{teachingnote}

Imagine you are working at a kite factory and you have been asked to design a new kite.  The kite will be a quadrilateral made of synthetic cloth, and it will be formed by two intersecting rods that serve as the diagonals of the quadrilateral and provide structure for the kite.  

\begin{prob}
To get started, review the definitions of all special quadrilaterals.  Be sure to include \emph{kite} on your list.  
\end{prob}

\begin{prob}
To consider the possible kite shapes, your task is to describe how conditions on the diagonals determine the quadrilateral.  Use fettuccine to model the intersecting rods, and use paper and pencil to draw the rod configurations and resulting kite shapes.  

Here are some hints:  

\begin{itemize}
\item Explore diagonals of various lengths, of the same length, and of different lengths.  
\item Explore various places at which to attach the diagonals to each other, including at one or both of their midpoints.  
\item Explore various angles that the diagonals might make with each other at their intersection, including the possibility of being perpendicular.  
\item Indicate what kinds of rotational or reflection symmetry you see in the resulting figure.
\end{itemize}
\end{prob}

\begin{prob}
Summarize your findings in a table organized like the one on the next page.  
\end{prob}

\newpage 

{
\renewcommand\arraystretch{2.8}
\renewcommand\tabcolsep{12pt}
\begin{table}[h]
\begin{tabular}{|l|p{6cm}|c|c|c|p{8cm}|}
\hline 
 &   & \multicolumn{3}{c|}{Diagonals} &  \\  %\hline
Quadrilateral & Definition (A quadrilateral with \dots) & \begin{sideways}Cong.\end{sideways} & 
\begin{sideways}Bisect\end{sideways} & \begin{sideways}Perp.\end{sideways} & Comments (e.g., symmetry, other properties) \\ \hline\hline
Square           &            &   &  &         &                  \\  \hline
Rectangle       &           &   &  &         &                  \\ \hline
Rhombus        &           &   &  &         &                  \\ \hline
Parallelogram &           &   &  &         &                  \\ \hline
Kite                &           &   &  &          &                  \\ \hline
Trapezoid       &           &   &  &         &                  \\ \hline
Isosceles Trapezoid       &           &   &  &         &                  \\ \hline
\end{tabular}
\end{table}
}




