\newpage 

\section{Circular Trigonometry}
\begin{teachingnote}
Need Trigonometry Checkup as preactivity, perhaps with a special office hour.  

Remind students of sine, cosine, and tangent in terms of $x$, $y$, and $r$ in the first quadrant.  

Given an angle, the approach is to pick an $x$ and $y$ that work and to see that the trig ratios should be independent of the choice.  

A key point here is ``extending the domain of an idea.''  Some things no longer work the same way.
\end{teachingnote}

As we have seen, right triangle trigonometry is restricted to acute angles.  But angles are often obtuse, so it is quite useful to extend trigonometry to angles greater than $90^\circ$.  Here is one approach:  Place the angle with the vertex at the origin in the coordinate plane and with one side of the angle (the initial side) along the positive $x$-axis.  
Measure to the other side of the angle (the terminal side) as a counter-clockwise rotation about the origin.   

\[
\includegraphics[scale=0.6]{../graphics/referenceTriangle}
\]

If we choose a point on the terminal side of this angle, we can draw what is called \emph{reference triangle} by dropping a perpendicular to the $x$-axis.  Then we can use the values of $x$, $y$, and $r$ from this triangle, just as before.  What is different in this picture is that $x$ is negative, as will be the case for any angle with a terminal side in the second quadrant.  


\begin{prob}
Draw a picture and use it to find the following values: 
\begin{enumerate}
\item $\sin 135^\circ = $
\item $\cos 135^\circ =$
\item $\tan 135^\circ =$
\end{enumerate}
\end{prob}

\begin{prob}
Draw a picture and use it to find the following values: 
\begin{enumerate}
\item $\sin 150^\circ =$
\item $\cos 150^\circ =$
\item $\tan 150^\circ =$
\end{enumerate}
\end{prob}

\begin{prob}
For some angles, the reference triangle is not actually a `triangle,' but that's okay.  Draw pictures to demonstrate the following: 
\begin{enumerate}
\item $\sin 90^\circ =$
\item $\cos 90^\circ =$
\item $\tan 90^\circ =$
\item $\sin 180^\circ =$
\item $\cos 180^\circ =$
\item $\tan 180^\circ =$
\end{enumerate}
\end{prob}

%\begin{prob}
%Now we can find the area of a triangle given two sides and an angle.\standardhs{G-SRT.9}  
%\end{prob}
%
%% Laws of Sines and Cosines \standardhs{G-SRT.10}, \standardhs{G-SRT.11}

Because angles are often about rotation, angles greater than $180^\circ$ can make sense, too.  And negative angles can describe rotation in the opposite direction.  If we consider the angle to change continuously, then rotation about the origin creates a situation that repeats every $360^\circ$.  This repetition provides the foundation for modeling lots of repetitive (periodic) contexts in the real world.  For this modeling, we need \emph{circular trigonometry}, which turns out to be much cleaner if (1) angles are measured not in degrees but in a more ``natural'' unit, called radians; and (2) we use \emph{the unit circle}, which is a circle of radius 1 centered at the origin.   

\begin{prob}
Below is the unit circle with special angles labeled in degrees, radians, and with coordinates.\standardhs{F-TF.2} 
$$\includegraphics[scale=0.25]{../graphics/unitCircle}$$
\begin{enumerate}
\item Explain what the various numbers mean in this unit circle.  
\item Use the unit circle to make a table showing (1) angle in degrees, (2) angle in radians, (3) sine of the angle, and (4) cosine of the angle.  
\item Use your table to draw a graph of $\sin\theta$ versus $\theta$.
\item Use your table to draw a graph of $\cos\theta$ versus $\theta$.
\item Explain why it makes sense to connect the dots. 
\item Extend your graphs to angles greater than $360^\circ$, and use the unit circle to explain why your extension makes sense. 
\item Extend your graphs to angles less than $0^\circ$, and use the unit circle to explain why your extension makes sense.
\end{enumerate}
\end{prob}

%\begin{enumerate}
%\item Understanding radian measure.\standardhs{G-C.5}, \standardhs{F-TF.1}
%
%\item Using the unit circle to extend trigonometry to angles of any measure.\standardhs{F-TF.2}
%
%\item Choosing and using trig functions to model periodic phenomena.\standardhs{F-TF.5}
%\item Fluency finding trig functions of special angles in radian measure.\standardhs{F-TF.3}
%
%\end{enumerate}

\fixnote{Possibly add an activity with questions about radian measure and modeling with trig functions.}

