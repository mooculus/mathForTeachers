%\documentclass[handout]{ximera}
\documentclass[nooutcomes]{ximera}


\graphicspath{
  {./}
  {graphics/}
  {../graphics/}
}

\usepackage{chngcntr}

\let\question\relax
\let\endquestion\relax




\newtheoremstyle{SlantTheorem}{\topsep}{\fill}%%% space between body and thm
%\newtheoremstyle{SlantTheorem}{\topsep}{\topsep}%%% space between body and thm
 {\slshape}                      %%% Thm body font
 {}                              %%% Indent amount (empty = no indent)
 {\bfseries\sffamily}            %%% Thm head font
 {}                              %%% Punctuation after thm head
 {3ex}                           %%% Space after thm head
 {\thmname{#1}\thmnumber{ #2}\thmnote{ \bfseries(#3)}}%%% Thm head spec
\theoremstyle{SlantTheorem}
\newtheorem{question}{Question}
\counterwithin*{question}{section}



\let\instructorNotes\relax
\let\endinstructorNotes\relax
%%% instructorNotes environment
\ifhandout
\newenvironment{instructorNotes}[1][false]%
{%
\def\givenatend{\boolean{#1}}\ifthenelse{\boolean{#1}}{\begin{trivlist}\item}{\setbox0\vbox\bgroup}{}
}
{%
\ifthenelse{\givenatend}{\end{trivlist}}{\egroup}{}
}
\else
\newenvironment{instructorNotes}[1][false]%
{%
  \ifthenelse{\boolean{#1}}{\begin{trivlist}\item[\hskip \labelsep\bfseries {\Large Instructor Notes: \\} \hspace{\textwidth} ]}
{\begin{trivlist}\item[\hskip \labelsep\bfseries {\Large Instructor Notes: \\} \hspace{\textwidth} ]}
{}
}
{\end{trivlist}}
\fi


%% Suggested Timing
\newcommand{\timing}[1]{{\bf Suggested Timing: \hspace{2ex}} #1}

\title{Angles in a Funky Shape}
\author{Jenny Sheldon, Bart Snapp, and Brad Findell}

\outcome{Learning outcome goes here.}

\begin{document}
\begin{abstract}
  Let's put your knowledge of interior angles to a test.
\end{abstract}
\maketitle

\begin{teachingnote}
Nonconvex is better than concave.  Think of the amount of turning to
identify which angle they are measuring.  Accuracy of protractor
measurement.  Triangulation is the point of the angle sum.  An error
worth discussing is triangulating incorrectly.
\end{teachingnote}

We are going to investigate the sum of the interior angles of a
funky shape.

\begin{problem}
Using a protractor, measure the interior angles of the crazy shape below:
\begin{image}
\includegraphics[scale=1.6]{funkyshape.pdf}
\end{image}
Use this table to record your findings:
\[
{\renewcommand{\arraystretch}{1.5}
\begin{array}{|c|c|c|c|c|c|c|c|c|}\hline
a & b & c & d & e & f & g & h & i \\\hline
\rule[7mm]{10mm}{0mm}  & \rule[7mm]{10mm}{0mm}    & \rule[7mm]{10mm}{0mm}   & \rule[7mm]{10mm}{0mm}   &  \rule[7mm]{10mm}{0mm}   & \rule[7mm]{10mm}{0mm}    & \rule[7mm]{10mm}{0mm}   & \rule[7mm]{10mm}{0mm}   & \rule[7mm]{10mm}{0mm}   \\ \hline
\end{array}}
\]
\end{problem}

\begin{problem}
Find the sum of the interior angles of the polygon above. 
\end{problem}


\begin{problem}
What should the sum be? Explain your reasoning.  
(You might find it useful to consider some of the angles to be ``reflex angles.''  Which ones?)  
\end{problem}


\end{document}
