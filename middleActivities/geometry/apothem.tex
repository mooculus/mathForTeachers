\newpage
\section{Apothem} %% remove * if added to main notes

Draw yourself a picture of a happy little equilateral triangle. Do
it---seriously.  Some people might call an equilateral triangle a
\textit{regular 3-gon}.

\begin{prob} 
What is an \textit{$n$-gon}? Give some relevant and revealing examples
and nonexamples.
\end{prob}

\begin{prob} 
When discussing $n$-gons, what are the allowable values for $n$?
\end{prob}

\begin{prob} 
What is a \textit{regular $n$-gon}? Give some relevant and revealing
examples and nonexamples.
\end{prob}


\begin{definition} 
A segment connecting the intersection of any two perpendicular
bisectors of sides of a polygon to either of those sides is called an
\textbf{apothem}\index{apothem}.
\end{definition}

\begin{prob}
Can you tell me in English what this defintion says? Provide some
examples of this definition in action.
\end{prob}



\begin{prob} 
Now consider some regular $n$-gon. Use \textsl{GeoGebra} to construct
apothems.
\end{prob}



\begin{prob}
Given a polygon, if you know the side length and the length of the
apothems, how do you find the area of your polygon? 
\end{prob}


\begin{prob}
Fix a value for $n$. Then use \textsl{GeoGebra} to help you fill in
the following table for various lengths of apothems.
\begin{center}
\begin{tabular}{|c||c|c|c|c|}\hline
$n$-gon & Apothem & Side & Perimeter & Area  \\\hline\hline
 \rule[0mm]{0mm}{7mm}\hspace{15mm}  &  & &  &  \\ \hline
 \rule[0mm]{0mm}{7mm}\hspace{15mm} &  & &  &  \\ \hline
 \rule[0mm]{0mm}{7mm}\hspace{15mm} &  & &  &  \\ \hline
 \rule[0mm]{0mm}{7mm}\hspace{15mm} &  & &  &  \\ \hline
 \rule[0mm]{0mm}{7mm}\hspace{15mm} &  & &  &  \\ \hline
 \rule[0mm]{0mm}{7mm}\hspace{15mm} &  & &  &  \\ \hline
 \rule[0mm]{0mm}{7mm}\hspace{15mm} &  & &  &  \\ \hline
\end{tabular}
\end{center}
\end{prob}

\begin{prob}
If $a$ is the length of the apothem, $P$ is the perimeter, and $A$ is
the area of the regular polygon, can you give a formula relating all
three of these quantities?
%\[
%A = \frac{pa}{2}
%\]
\end{prob}

