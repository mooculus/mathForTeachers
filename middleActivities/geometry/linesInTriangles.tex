%\documentclass[handout]{ximera}
\documentclass[nooutcomes]{ximera}


\graphicspath{
  {./}
  {graphics/}
  {../graphics/}
}

\usepackage{chngcntr}

\let\question\relax
\let\endquestion\relax




\newtheoremstyle{SlantTheorem}{\topsep}{\fill}%%% space between body and thm
%\newtheoremstyle{SlantTheorem}{\topsep}{\topsep}%%% space between body and thm
 {\slshape}                      %%% Thm body font
 {}                              %%% Indent amount (empty = no indent)
 {\bfseries\sffamily}            %%% Thm head font
 {}                              %%% Punctuation after thm head
 {3ex}                           %%% Space after thm head
 {\thmname{#1}\thmnumber{ #2}\thmnote{ \bfseries(#3)}}%%% Thm head spec
\theoremstyle{SlantTheorem}
\newtheorem{question}{Question}
\counterwithin*{question}{section}



\let\instructorNotes\relax
\let\endinstructorNotes\relax
%%% instructorNotes environment
\ifhandout
\newenvironment{instructorNotes}[1][false]%
{%
\def\givenatend{\boolean{#1}}\ifthenelse{\boolean{#1}}{\begin{trivlist}\item}{\setbox0\vbox\bgroup}{}
}
{%
\ifthenelse{\givenatend}{\end{trivlist}}{\egroup}{}
}
\else
\newenvironment{instructorNotes}[1][false]%
{%
  \ifthenelse{\boolean{#1}}{\begin{trivlist}\item[\hskip \labelsep\bfseries {\Large Instructor Notes: \\} \hspace{\textwidth} ]}
{\begin{trivlist}\item[\hskip \labelsep\bfseries {\Large Instructor Notes: \\} \hspace{\textwidth} ]}
{}
}
{\end{trivlist}}
\fi


%% Suggested Timing
\newcommand{\timing}[1]{{\bf Suggested Timing: \hspace{2ex}} #1}

\title{Lines in Triangles}
\author{Bart Snapp and Brad Findell}

\outcome{Learning outcome goes here.}

\begin{document}
\begin{abstract}
  We think about some special lines in triangles. 
\end{abstract}
\maketitle

\begin{teachingnote}
This preactivity for Isosceles Bisectors is about (1) the meanings of median, altitude, angle bisector, and perpendicular bisector; (2) drawing them carefully with protractor and ruler; and (3) noticing that they are four different lines in a general triangle. 
\end{teachingnote}

Two copies of a triangle are shown below.   In each triangle, \textbf{draw carefully} the designated lines.  \emph{Construction is not necessary:  careful measurements are allowed.}

\begin{problem}
In the triangle on the left, draw the median from $B$ to $\overline{AC}$, the altitude from $B$ to $\overline{AC}$, the angle bisector of $\angle B$, and the perpendicular bisector of $\overline{AC}$.  

%\begin{enumerate}
%\item Median from $B$ to $\overline{AC}$
%\item Altitude from $B$ to $\overline{AC}$
%\item Angle bisector of $\angle B$
%\item Perpendicular bisector of $\overline{AC}$
%\end{enumerate}
\end{problem}

\begin{problem}
In the triangle on the right, draw the median from $C$ to $\overline{AB}$, the altitude from $C$ to $\overline{AB}$, the angle bisector of $\angle C$, and the perpendicular bisector of $\overline{AB}$. 

%\begin{enumerate}
%\item Median from $C$ to $\overline{AB}$
%\item Altitude from $C$ to $\overline{AB}$
%\item Angle bisector of $\angle C$
%\item Perpendicular bisector of $\overline{AB}$
%\end{enumerate}
%
\end{problem}

\begin{problem}
In each triangle, you should have drawn four different lines.  What might you say about a triangle for which two or more of these lines turn out to be the same?  
\end{problem}

\includegraphics[scale=0.9]{obtuseTriangle.pdf}

\includegraphics[scale=0.9]{obtuseTriangle.pdf}

\end{document}
