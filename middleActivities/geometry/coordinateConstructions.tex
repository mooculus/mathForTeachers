%\documentclass[handout]{ximera}
\documentclass[nooutcomes]{ximera}

\usepackage{gensymb}
\usepackage{tabularx}
\usepackage{mdframed}
\usepackage{pdfpages}
%\usepackage{chngcntr}

\let\problem\relax
\let\endproblem\relax

\newcommand{\property}[2]{#1#2}




\newtheoremstyle{SlantTheorem}{\topsep}{\fill}%%% space between body and thm
 {\slshape}                      %%% Thm body font
 {}                              %%% Indent amount (empty = no indent)
 {\bfseries\sffamily}            %%% Thm head font
 {}                              %%% Punctuation after thm head
 {3ex}                           %%% Space after thm head
 {\thmname{#1}\thmnumber{ #2}\thmnote{ \bfseries(#3)}} %%% Thm head spec
\theoremstyle{SlantTheorem}
\newtheorem{problem}{Problem}[]

%\counterwithin*{problem}{section}



%%%%%%%%%%%%%%%%%%%%%%%%%%%%Jenny's code%%%%%%%%%%%%%%%%%%%%

%%% Solution environment
%\newenvironment{solution}{
%\ifhandout\setbox0\vbox\bgroup\else
%\begin{trivlist}\item[\hskip \labelsep\small\itshape\bfseries Solution\hspace{2ex}]
%\par\noindent\upshape\small
%\fi}
%{\ifhandout\egroup\else
%\end{trivlist}
%\fi}
%
%
%%% instructorIntro environment
%\ifhandout
%\newenvironment{instructorIntro}[1][false]%
%{%
%\def\givenatend{\boolean{#1}}\ifthenelse{\boolean{#1}}{\begin{trivlist}\item}{\setbox0\vbox\bgroup}{}
%}
%{%
%\ifthenelse{\givenatend}{\end{trivlist}}{\egroup}{}
%}
%\else
%\newenvironment{instructorIntro}[1][false]%
%{%
%  \ifthenelse{\boolean{#1}}{\begin{trivlist}\item[\hskip \labelsep\bfseries Instructor Notes:\hspace{2ex}]}
%{\begin{trivlist}\item[\hskip \labelsep\bfseries Instructor Notes:\hspace{2ex}]}
%{}
%}
%% %% line at the bottom} 
%{\end{trivlist}\par\addvspace{.5ex}\nobreak\noindent\hung} 
%\fi
%
%


\let\instructorNotes\relax
\let\endinstructorNotes\relax
%%% instructorNotes environment
\ifhandout
\newenvironment{instructorNotes}[1][false]%
{%
\def\givenatend{\boolean{#1}}\ifthenelse{\boolean{#1}}{\begin{trivlist}\item}{\setbox0\vbox\bgroup}{}
}
{%
\ifthenelse{\givenatend}{\end{trivlist}}{\egroup}{}
}
\else
\newenvironment{instructorNotes}[1][false]%
{%
  \ifthenelse{\boolean{#1}}{\begin{trivlist}\item[\hskip \labelsep\bfseries {\Large Instructor Notes: \\} \hspace{\textwidth} ]}
{\begin{trivlist}\item[\hskip \labelsep\bfseries {\Large Instructor Notes: \\} \hspace{\textwidth} ]}
{}
}
{\end{trivlist}}
\fi


%% Suggested Timing
\newcommand{\timing}[1]{{\bf Suggested Timing: \hspace{2ex}} #1}




\hypersetup{
    colorlinks=true,       % false: boxed links; true: colored links
    linkcolor=blue,          % color of internal links (change box color with linkbordercolor)
    citecolor=green,        % color of links to bibliography
    filecolor=magenta,      % color of file links
    urlcolor=cyan           % color of external links
}

\title{Coordinate Constructions}
\author{Bart Snapp and Brad Findell}

\outcome{Learning outcome goes here.}

\begin{document}
\begin{abstract}
  We use Cartesian coordinates to help use understand geometry.
\end{abstract}
\maketitle

In synthetic geometry, point, line and plane are taken to be undefined terms.  In analytic (coordinate) geometry, in contrast, we make the following definitions.  
\begin{definition}
A \emph{point} is an ordered pair $(x,y)$ of real numbers. A \emph{line} is the set of ordered pairs $(x,y)$ that satisfy an equation of the form $ax + by = c$, where $a$, $b$, and $c$ are real numbers and $a$ and $b$ are not both 0.   
\end{definition}

Many of the problems below are expressed generally.  You may find it useful to try some specific examples before the general case.  

\begin{teachingnote}
These problems begin with the general case.  Some students might need some specific examples to help ground their thinking.
\end{teachingnote}

\begin{problem}
In the above definition of a line in coordinate geometry, why is it important to require that $a$ and $b$ are not both 0?  
\vfill
\end{problem}

\begin{problem}
Given points $(x_1, y_1)$ and $(x_2, y_2)$, find the distance between them in the coordinate plane.
\vfill
\end{problem}

\begin{problem}
Find the midpoint of the segment from $(x_1, y_1)$ and $(x_2, y_2)$.  Explain why your formula makes sense. 
\vfill
\end{problem}

\begin{teachingnote}
Probably out of habit from the slope formula, some students will subtract coordinates to find midpoints.  This is a good place to connect algebraically various ways of finding the midpoint of two values, such as (1) taking half the difference and adding it to the lower number; (2) adding the two numbers and dividing by two. 
\end{teachingnote}   

\newpage

\begin{problem}
Recall that in synthetic geometry, a circle is defined as the set of points that are equidistant from a center.  Use this definition to determine the equation of circle with center $(h, k)$ and radius $r$.\margincomment{CCSS G-GPE.1. Derive the equation of a circle of given center and radius
  using the Pythagorean Theorem; complete the square to find the
  center and radius of a circle given by an equation.}
\vfill
\end{problem}

\begin{problem}
For each pair of points below, find an equation of the line containing the two points.  
\begin{enumerate}
\item Points $(2,3)$ and $(5,7)$.  
\item Points $(2,3)$ and $(2,7)$.  
\item Points $(2,3)$ and $(5,3)$. 
\item Points $(x_1, y_1)$ and $(x_2, y_2)$.  
\end{enumerate}
\vfill
\end{problem}

\newpage
\begin{problem}
Express each of your previous equations in the form $ax + by = c$ and also in the form $y = mx + b$.   What are the advantages and disadvantages of these forms?  
\vfill
\end{problem}

\begin{problem}
In school mathematics, lines are usually of the form $y = mx + b$.  Why is it unambiguous to talk about \emph{the slope} of such a line?  In other words, given a non-vertical line in the plane, explain why any two points on the line will yield the same slope.\margincomment{CCSS 8.EE.6. Use similar triangles to explain why the slope $m$ is the same
  between any two distinct points on a non-vertical line in the
  coordinate plane; derive the equation $y = mx$ for a line through the
  origin and the equation $y = mx + b$ for a line intercepting the
  vertical axis at $b$.}
\vfill
\end{problem}

\end{document}
