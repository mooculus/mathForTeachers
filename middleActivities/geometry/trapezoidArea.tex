%\documentclass[handout]{ximera}
\documentclass[nooutcomes,instructornotes]{ximera}


\graphicspath{
  {./}
  {graphics/}
  {../graphics/}
}

\usepackage{chngcntr}

\let\question\relax
\let\endquestion\relax




\newtheoremstyle{SlantTheorem}{\topsep}{\fill}%%% space between body and thm
%\newtheoremstyle{SlantTheorem}{\topsep}{\topsep}%%% space between body and thm
 {\slshape}                      %%% Thm body font
 {}                              %%% Indent amount (empty = no indent)
 {\bfseries\sffamily}            %%% Thm head font
 {}                              %%% Punctuation after thm head
 {3ex}                           %%% Space after thm head
 {\thmname{#1}\thmnumber{ #2}\thmnote{ \bfseries(#3)}}%%% Thm head spec
\theoremstyle{SlantTheorem}
\newtheorem{question}{Question}
\counterwithin*{question}{section}



\let\instructorNotes\relax
\let\endinstructorNotes\relax
%%% instructorNotes environment
\ifhandout
\newenvironment{instructorNotes}[1][false]%
{%
\def\givenatend{\boolean{#1}}\ifthenelse{\boolean{#1}}{\begin{trivlist}\item}{\setbox0\vbox\bgroup}{}
}
{%
\ifthenelse{\givenatend}{\end{trivlist}}{\egroup}{}
}
\else
\newenvironment{instructorNotes}[1][false]%
{%
  \ifthenelse{\boolean{#1}}{\begin{trivlist}\item[\hskip \labelsep\bfseries {\Large Instructor Notes: \\} \hspace{\textwidth} ]}
{\begin{trivlist}\item[\hskip \labelsep\bfseries {\Large Instructor Notes: \\} \hspace{\textwidth} ]}
{}
}
{\end{trivlist}}
\fi


%% Suggested Timing
\newcommand{\timing}[1]{{\bf Suggested Timing: \hspace{2ex}} #1}

\title{Trapezoid Area}
\author{Bart Snapp and Brad Findell}

\outcome{Learning outcome goes here.}

\begin{document}
\begin{abstract}
  We investigate trapezoids and how to compute their area.
\end{abstract}
\maketitle

\begin{teachingnote}
We set the stage with triangles.  A challenge for students is distinguishing the various `bases,' and realizing how helpful it is to label them with different letters.   
\end{teachingnote}

\begin{problem}Explain how the following picture ``proves'' that
  the area of a right triangle is half the base times the height.

\[
\definecolor{qqwuqq}{rgb}{0.,0.39215686274509803,0.}
\begin{tikzpicture}[line cap=round,line join=round,>=triangle 45,x=1.0cm,y=1.0cm]
\clip(-0.1,-0.1) rectangle (3.1,4.1);
\draw[line width=0.8pt,color=qqwuqq,fill=qqwuqq,fill opacity=0.10] (0.2828,0.) -- (0.2828,0.2828) -- (0.,0.2828) -- (0.,0.) -- cycle; 
\draw [line width=0.8pt] (0.,4.)-- (0.,0.);
\draw [line width=0.8pt] (0.,0.)-- (3.,0.);
\draw [line width=0.8pt] (3.,0.)-- (0.,4.);
\draw [line width=0.8pt,dash pattern=on 2pt off 2pt] (0.,4.)-- (3.,4.);
\draw [line width=0.8pt,dash pattern=on 2pt off 2pt] (3.,4.)-- (3.,0.);
\end{tikzpicture}
\]
%
%\[
%\includegraphics[scale=0.8]{pbpAreaRight.pdf}
%\]
\end{problem}

\begin{problem} Suppose you know that the area of a \textbf{right} triangle is
  half the base times the height. Explain how the following picture
  ``proves'' that the area of \textbf{every} triangle is half the base times the
  height.
\[
\definecolor{qqwuqq}{rgb}{0.,0.392,0.}
\begin{tikzpicture}[line cap=round,line join=round,>=triangle 45,x=1.0cm,y=1.0cm]
\clip(-0.5,-0.05) rectangle (5,2.55);
\draw[line width=0.8pt,color=qqwuqq,fill=qqwuqq,fill opacity=0.1] (3.,0.1954) -- (2.8046,0.1954) -- (2.8046,0.) -- (3.,0.) -- cycle; 
\draw[line width=0.8pt,color=qqwuqq,fill=qqwuqq,fill opacity=0.1] (3.1954,0.) -- (3.1954,0.1954) -- (3.,0.1954) -- (3.,0.) -- cycle; 
\draw [line width=0.8pt] (0.,0.)-- (3.,2.5);
\draw [line width=0.8pt] (3.,2.5)-- (4.,0.);
\draw [line width=0.8pt] (4.,0.)-- (0.,0.);
\draw [line width=0.8pt] (0.,0.)-- (3.,0.);
\draw [line width=0.8pt] (3.,0.)-- (4.,0.);
\draw [line width=0.8pt,dash pattern=on 2pt off 2pt] (3.,0.)-- (3.,2.5);
\end{tikzpicture}
\]
%\[
%\includegraphics[scale=0.8]{../graphics/pbpDisTri.pdf}
%\]
\vspace{1in}
\end{problem}

\begin{problem}
Now suppose that \textit{Geometry Giorgio} attempts to
solve a similar problem. Again knowing that the area of a right
triangle is half the base times the height, he draws the following
picture:
\[
\begin{tikzpicture}[line cap=round,line join=round,>=triangle 45,x=1.0cm,y=1.0cm]
\clip(-0.5,-0.05) rectangle (5.25,2.55);
\draw [line width=0.8pt,dash pattern=on 2pt off 2pt] (0.,0.)-- (0.,2.5);
\draw [line width=0.8pt,dash pattern=on 2pt off 2pt] (0.,2.5)-- (5.,2.5);
\draw [line width=0.8pt,dash pattern=on 2pt off 2pt] (5.,2.5)-- (5.,0.);
\draw [line width=0.8pt] (5.,2.5)-- (2.,0.);
\draw [line width=0.8pt] (5.,2.5)-- (0.,0.);
\draw [line width=0.8pt] (0.,0.)-- (2.,0.);
\draw [line width=0.8pt,dash pattern=on 2pt off 2pt] (2.,0.)-- (5.,0.);
\end{tikzpicture}
\]
%\[
%\includegraphics[scale=0.8]{../graphics/pbpDisTriGio.pdf}
%\]
\textit{Geometry Giorgio} states that the diagonal line cuts the
rectangle in half, and thus the area of the triangle is half the base
times the height. Is this correct reasoning? If so, give a complete
explanation. If not, give correct reasoning based on \textit{Geometry
  Giorgio}'s picture.
\vspace{2in}
\end{problem}

\begin{teachingnote}
The point of the next problem is connecting the geometric thinking with the algebraic thinking.  For example, how, algebraically and geometrically, does the first trapezoid formula look like an average?  In the last problem, students might not see the similar triangles.

Another approach, not included here: ``A rectangle plus two triangles.''
\end{teachingnote}


\begin{problem}
Now we explore several ways of justifying the formula for the area of a trapezoid, as labeled below. 
\begin{image}
\includegraphics[scale=0.6]{trapezoid1.pdf}
\end{image}
Complete the table on the following page so that, in each row, the explanation, the geometric figure, and the algebraic formula together describe a way of computing the area.  For comparison purposes, each illustration should include a trapezoid congruent to the trapezoid above.   

All of the area formulas will, of course, be equivalent to one another as expressions.  But each way of expressing the area will make the most sense with figure and the explanation from the same row.  

\newpage


\newlength{\formulawidth}
\settowidth{\formulawidth}{$\frac{1}{2}b_2(x+h)-\frac{1}{2}b_1x$, with $\frac{x}{b_1}=\frac{x+h}{b_2}$}  
\resizebox{6in}{!}{ % use \textwidth instead of 6 in?
{\renewcommand{\arraystretch}{1.5}
\begin{tabular}{|>{\centering\arraybackslash}m{2.5cm}|>{\centering\arraybackslash}m{9.5cm}|c|}\hline
Explanation & Figure & Area Formula \\\hline

Rectangle with width that is the average of the bases. & \includegraphics[scale=0.7]{trapezoid2.pdf} & $\left(\frac{b_1+b_2}{2}\right)h$ \\ \hline
                              & \includegraphics[scale=0.7]{trapezoid3.pdf} &                      \\ \hline
Two triangles with the same height and different bases. &                 & \\ \hline
 & & \\ 
\bigskip                              &  & $(b_1+b_2)\frac{h}{2}$ \\ 
 & & \\ \hline
          & \includegraphics[scale=0.7]{trapezoid6.pdf}&  \hspace{\formulawidth} \\ \hline
\end{tabular}}
}
\end{problem}
%
%   Answers  
%
\newpage
\begin{teachingnote}
Answers:

\resizebox{\textwidth}{!}{
{\renewcommand{\arraystretch}{1.5}
\begin{tabular}{|>{\centering\arraybackslash}m{2.5cm}|>{\centering\arraybackslash}m{9.5cm}|c|}\hline
Explanation & Figure & Area Formula \\\hline

Rectangle with width that is the average of the bases. & \includegraphics[scale=0.7]{trapezoid2.pdf} & $\left(\frac{b_1+b_2}{2}\right)h$ \\ \hline
Half of a large parallelogram. & \includegraphics[scale=0.7]{trapezoid3.pdf} & $\frac{1}{2}(b_1+b_2)h$ \\ \hline
Two triangles with the same height and different bases. & \includegraphics[scale=0.7]{trapezoid4.pdf} & $\frac{1}{2}b_1h + \frac{1}{2}b_2h$ \\ \hline
A parallelogram with half the height. & \includegraphics[scale=0.7]{trapezoid5.pdf} & $(b_1+b_2)\frac{h}{2}$ \\ \hline
Difference between two triangles, with $x$ as height of small triangle. 
          & \includegraphics[scale=0.7]{trapezoid6.pdf} &  $\frac{1}{2}b_2(x+h)-\frac{1}{2}b_1x$, with $\frac{x}{b_1}=\frac{x+h}{b_2}$ \\ \hline
\end{tabular}}
}
\end{teachingnote}

\end{document}
