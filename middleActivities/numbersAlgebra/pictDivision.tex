%\documentclass[handout]{ximera}
\documentclass[nooutcomes]{ximera}


\graphicspath{
  {./}
  {graphics/}
  {../graphics/}
}

\usepackage{chngcntr}

\let\question\relax
\let\endquestion\relax




\newtheoremstyle{SlantTheorem}{\topsep}{\fill}%%% space between body and thm
%\newtheoremstyle{SlantTheorem}{\topsep}{\topsep}%%% space between body and thm
 {\slshape}                      %%% Thm body font
 {}                              %%% Indent amount (empty = no indent)
 {\bfseries\sffamily}            %%% Thm head font
 {}                              %%% Punctuation after thm head
 {3ex}                           %%% Space after thm head
 {\thmname{#1}\thmnumber{ #2}\thmnote{ \bfseries(#3)}}%%% Thm head spec
\theoremstyle{SlantTheorem}
\newtheorem{question}{Question}
\counterwithin*{question}{section}



\let\instructorNotes\relax
\let\endinstructorNotes\relax
%%% instructorNotes environment
\ifhandout
\newenvironment{instructorNotes}[1][false]%
{%
\def\givenatend{\boolean{#1}}\ifthenelse{\boolean{#1}}{\begin{trivlist}\item}{\setbox0\vbox\bgroup}{}
}
{%
\ifthenelse{\givenatend}{\end{trivlist}}{\egroup}{}
}
\else
\newenvironment{instructorNotes}[1][false]%
{%
  \ifthenelse{\boolean{#1}}{\begin{trivlist}\item[\hskip \labelsep\bfseries {\Large Instructor Notes: \\} \hspace{\textwidth} ]}
{\begin{trivlist}\item[\hskip \labelsep\bfseries {\Large Instructor Notes: \\} \hspace{\textwidth} ]}
{}
}
{\end{trivlist}}
\fi


%% Suggested Timing
\newcommand{\timing}[1]{{\bf Suggested Timing: \hspace{2ex}} #1}

\title{Picture Yourself Dividing}
\author{Bart Snapp and Brad Findell}

\outcome{Learning outcome goes here.}

\begin{document}
\begin{abstract}
We think about division of fractions.
\end{abstract}
\maketitle


We want to understand how to visualize 
\[
\frac{a}{b} \div \frac{c}{d}
\]
Let's see if we can ease into this.

\begin{problem}
Draw a picture that shows how to compute:
\[
10\div 5
\]
Explain how your picture could be redrawn for other similar
numbers. Write two story problems solved by this expression, one
asking for ``how many groups'' and the other asking for ``how many in
one group.''
\end{problem}

\begin{teachingnote}
Some students will write story problems that make sense only for whole number divisors, dividends, quotients, and remainders.  These problems are (mentally) challenging to imagine with non-integer quantities.  Spend some class discussion on discrete vs. continuous quantities, and talk about how continuous quantities are crucial for thinking about fraction division.  
\end{teachingnote}

\begin{problem}
Try to use a similar process to the one you used in the first problem
to draw a picture that shows how to compute:
\[
\frac{1}{4} \div 3
\]
Explain how your picture could be redrawn for other similar numbers.
Write two story problems solved by this expression, one asking for
``how many groups'' and the other asking for ``how many in one
group.''
\end{problem}


\begin{problem}
Try to use a similar process to the one you used in the first two problems
to draw a picture that shows how to compute:
\[
3 \div \frac{1}{4}
\]
Explain how your picture could be redrawn for other similar numbers.
Write two story problems solved by this expression, one asking for
``how many groups'' and the other asking for ``how many in one
group.''
\end{problem}



\begin{problem}
Try to use a similar process to the one you used in the first three problems
to draw a picture that shows how to compute:
\[
\frac{7}{5} \div \frac{3}{4}
\]
Explain how your picture could be redrawn for other similar numbers.
Write two story problems solved by this expression, one asking for
``how many groups'' and the other asking for ``how many in each
group.''
\end{problem}

\begin{problem}
Explain how to draw pictures to visualize:
\[
\frac{a}{b} \div \frac{c}{d}
\]
\end{problem}

\begin{problem}
Use pictures to explain why:
\[
\frac{a}{b} \div \frac{c}{d} = \frac{a}{b} \cdot \frac{d}{c}
\]
\end{problem}


\begin{problem}
Alice and Rita wish to model the calculation $1\frac{3}{4} \div \frac{1}{2}$.  They propose the following story problem: 

\begin{quote}
Alice and Rita have $1\frac{3}{4}$ pizzas, as shown below.
\begin{image}
\begin{tikzpicture}[line cap=round,line join=round,>=triangle 45,x=1.0cm,y=1.0cm]
\clip(-1.2,-1.2) rectangle (6,1.2);
\draw [line width=0.8pt] (0.,0.) circle (1.cm);
\draw [shift={(3.,0.)},line width=0.8pt]  plot[domain=-4.7124:0.,variable=\t]({1.*1.*cos(\t r)+0.*1.*sin(\t r)},{0.*1.*cos(\t r)+1.*1.*sin(\t r)});
\draw [line width=0.8pt] (3.,1.)-- (3.,0.);
\draw [line width=0.8pt] (3.,0.)-- (4.,0.);
\end{tikzpicture}
\end{image}

They share the pizza fairly, as follows:   
\begin{image}
\begin{tikzpicture}[line cap=round,line join=round,>=triangle 45,x=1.0cm,y=1.0cm]
\clip(-1.2,-1.2) rectangle (6,1.2);
\draw [line width=0.8pt] (0.,0.) circle (1.cm);
\draw [line width=0.8pt] (0.,1.)-- (0.,-1.);
\draw [line width=0.8pt] (-1.,0.)-- (1.,0.);
\draw [line width=0.8pt] (2.,0.)-- (4.,0.);
\draw [line width=0.8pt] (3.,1.)-- (3.,-1.);
\draw [shift={(3.,0.)},line width=0.8pt]  plot[domain=-4.7124:0.,variable=\t]({1.*1.*cos(\t r)+0.*1.*sin(\t r)},{0.*1.*cos(\t r)+1.*1.*sin(\t r)});
\draw [line width=0.8pt] (3.,0.)-- (3.707,-0.707);
\draw (-0.4,0.4) node {A};
\draw (0.4,0.4) node {R};
\draw (0.4,-0.4) node {A};
\draw (-0.4,-0.4) node {R};
\draw (2.6,0.4) node {A};
\draw (2.6,-0.4) node {R};
\draw (3.2,-0.55) node {A};
\draw (3.6,-0.25) node {R};
\end{tikzpicture}
\end{image}
So they each get $3\frac{1}{2}$ pieces.  
\end{quote}

\begin{enumerate}
\item Did they get the right answer for $1\frac{3}{4} \div \frac{1}{2}$?  
\item Does this context model $1\frac{3}{4} \div \frac{1}{2}$ as they claim?  Explain?  
\end{enumerate}

\begin{teachingnote}
Their answer of $3\frac{1}{2}$ is numerically correct but their model is not.  The model shows division by $2$ rather than division by $\frac{1}{2}$.  They got the correct answer by switching to slices (i.e., quarter pizzas) rather than pizzas.  Their answer of $3\frac{1}{2}$ slices is actually $3\frac{1}{2}$ quarters or $\frac{7}{8}$ of a pizza.

They can begin with this context and correcly model $1\frac{3}{4} \div \frac{1}{2}$ in one of two ways. 
\begin{itemize}
\item Alice and Rita have $1\frac{3}{4}$ pizzas, which is $\frac{1}{2}$ of the amount of pizza they need for a meeting of the math club.  How much pizza do they need for the meeting? (How many in one group?)
\item Alice and Rita have $1\frac{3}{4}$ pizzas.  How many $\frac{1}{2}$ pizza portions can they make? (How many groups?)
\end{itemize}
\end{teachingnote}

\end{problem}


\end{document}
