%\documentclass[handout]{ximera}
\documentclass[nooutcomes]{ximera}

\usepackage{gensymb}
\usepackage{tabularx}
\usepackage{mdframed}
\usepackage{pdfpages}
%\usepackage{chngcntr}

\let\problem\relax
\let\endproblem\relax

\newcommand{\property}[2]{#1#2}




\newtheoremstyle{SlantTheorem}{\topsep}{\fill}%%% space between body and thm
 {\slshape}                      %%% Thm body font
 {}                              %%% Indent amount (empty = no indent)
 {\bfseries\sffamily}            %%% Thm head font
 {}                              %%% Punctuation after thm head
 {3ex}                           %%% Space after thm head
 {\thmname{#1}\thmnumber{ #2}\thmnote{ \bfseries(#3)}} %%% Thm head spec
\theoremstyle{SlantTheorem}
\newtheorem{problem}{Problem}[]

%\counterwithin*{problem}{section}



%%%%%%%%%%%%%%%%%%%%%%%%%%%%Jenny's code%%%%%%%%%%%%%%%%%%%%

%%% Solution environment
%\newenvironment{solution}{
%\ifhandout\setbox0\vbox\bgroup\else
%\begin{trivlist}\item[\hskip \labelsep\small\itshape\bfseries Solution\hspace{2ex}]
%\par\noindent\upshape\small
%\fi}
%{\ifhandout\egroup\else
%\end{trivlist}
%\fi}
%
%
%%% instructorIntro environment
%\ifhandout
%\newenvironment{instructorIntro}[1][false]%
%{%
%\def\givenatend{\boolean{#1}}\ifthenelse{\boolean{#1}}{\begin{trivlist}\item}{\setbox0\vbox\bgroup}{}
%}
%{%
%\ifthenelse{\givenatend}{\end{trivlist}}{\egroup}{}
%}
%\else
%\newenvironment{instructorIntro}[1][false]%
%{%
%  \ifthenelse{\boolean{#1}}{\begin{trivlist}\item[\hskip \labelsep\bfseries Instructor Notes:\hspace{2ex}]}
%{\begin{trivlist}\item[\hskip \labelsep\bfseries Instructor Notes:\hspace{2ex}]}
%{}
%}
%% %% line at the bottom} 
%{\end{trivlist}\par\addvspace{.5ex}\nobreak\noindent\hung} 
%\fi
%
%


\let\instructorNotes\relax
\let\endinstructorNotes\relax
%%% instructorNotes environment
\ifhandout
\newenvironment{instructorNotes}[1][false]%
{%
\def\givenatend{\boolean{#1}}\ifthenelse{\boolean{#1}}{\begin{trivlist}\item}{\setbox0\vbox\bgroup}{}
}
{%
\ifthenelse{\givenatend}{\end{trivlist}}{\egroup}{}
}
\else
\newenvironment{instructorNotes}[1][false]%
{%
  \ifthenelse{\boolean{#1}}{\begin{trivlist}\item[\hskip \labelsep\bfseries {\Large Instructor Notes: \\} \hspace{\textwidth} ]}
{\begin{trivlist}\item[\hskip \labelsep\bfseries {\Large Instructor Notes: \\} \hspace{\textwidth} ]}
{}
}
{\end{trivlist}}
\fi


%% Suggested Timing
\newcommand{\timing}[1]{{\bf Suggested Timing: \hspace{2ex}} #1}




\hypersetup{
    colorlinks=true,       % false: boxed links; true: colored links
    linkcolor=blue,          % color of internal links (change box color with linkbordercolor)
    citecolor=green,        % color of links to bibliography
    filecolor=magenta,      % color of file links
    urlcolor=cyan           % color of external links
}

\title{Picture Yourself Dividing}
\author{Bart Snapp and Brad Findell}

\outcome{Learning outcome goes here.}

\begin{document}
\begin{abstract}
We think about division.
\end{abstract}
\maketitle


We want to understand how to visualize 
\[
\frac{a}{b} \div \frac{c}{d}
\]
Let's see if we can ease into this.

\begin{problem}
Draw a picture that shows how to compute:
\[
10\div 5
\]
Explain how your picture could be redrawn for other similar
numbers. Write two story problems solved by this expression, one
asking for ``how many groups'' and the other asking for ``how many in
one group.''
\end{problem}

\begin{teachingnote}
Some students will write story problems that make sense only for whole number divisors, dividends, quotients, and remainders.  These problems are (mentally) challenging to imagine with non-integer quantities.  Spend some class discussion on discrete vs. continuous quantities, and talk about how continuous quantities are crucial for thinking about fraction division.  
\end{teachingnote}

\begin{problem}
Try to use a similar process to the one you used in the first problem
to draw a picture that shows how to compute:
\[
\frac{1}{4} \div 3
\]
Explain how your picture could be redrawn for other similar numbers.
Write two story problems solved by this expression, one asking for
``how many groups'' and the other asking for ``how many in one
group.''
\end{problem}


\begin{problem}
Try to use a similar process to the one you used in the first two problems
to draw a picture that shows how to compute:
\[
3 \div \frac{1}{4}
\]
Explain how your picture could be redrawn for other similar numbers.
Write two story problems solved by this expression, one asking for
``how many groups'' and the other asking for ``how many in one
group.''
\end{problem}

\fixnote{Also incorporate $1\frac{3}{4}\div \frac{1}{2}$, showing a common incorrect story problem.}

\begin{problem}
Try to use a similar process to the one you used in the first three problems
to draw a picture that shows how to compute:
\[
\frac{7}{5} \div \frac{3}{4}
\]
Explain how your picture could be redrawn for other similar numbers.
Write two story problems solved by this expression, one asking for
``how many groups'' and the other asking for ``how many in each
group.''
\end{problem}

\begin{problem}
Explain how to draw pictures to visualize:
\[
\frac{a}{b} \div \frac{c}{d}
\]
\end{problem}

\begin{problem}
Use pictures to explain why:
\[
\frac{a}{b} \div \frac{c}{d} = \frac{a}{b} \cdot \frac{d}{c}
\]
\end{problem}

\end{document}
