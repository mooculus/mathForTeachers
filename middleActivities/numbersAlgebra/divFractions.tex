%\documentclass[handout]{ximera}
\documentclass[nooutcomes]{ximera}

\usepackage{gensymb}
\usepackage{tabularx}
\usepackage{mdframed}
\usepackage{pdfpages}
%\usepackage{chngcntr}

\let\problem\relax
\let\endproblem\relax

\newcommand{\property}[2]{#1#2}




\newtheoremstyle{SlantTheorem}{\topsep}{\fill}%%% space between body and thm
 {\slshape}                      %%% Thm body font
 {}                              %%% Indent amount (empty = no indent)
 {\bfseries\sffamily}            %%% Thm head font
 {}                              %%% Punctuation after thm head
 {3ex}                           %%% Space after thm head
 {\thmname{#1}\thmnumber{ #2}\thmnote{ \bfseries(#3)}} %%% Thm head spec
\theoremstyle{SlantTheorem}
\newtheorem{problem}{Problem}[]

%\counterwithin*{problem}{section}



%%%%%%%%%%%%%%%%%%%%%%%%%%%%Jenny's code%%%%%%%%%%%%%%%%%%%%

%%% Solution environment
%\newenvironment{solution}{
%\ifhandout\setbox0\vbox\bgroup\else
%\begin{trivlist}\item[\hskip \labelsep\small\itshape\bfseries Solution\hspace{2ex}]
%\par\noindent\upshape\small
%\fi}
%{\ifhandout\egroup\else
%\end{trivlist}
%\fi}
%
%
%%% instructorIntro environment
%\ifhandout
%\newenvironment{instructorIntro}[1][false]%
%{%
%\def\givenatend{\boolean{#1}}\ifthenelse{\boolean{#1}}{\begin{trivlist}\item}{\setbox0\vbox\bgroup}{}
%}
%{%
%\ifthenelse{\givenatend}{\end{trivlist}}{\egroup}{}
%}
%\else
%\newenvironment{instructorIntro}[1][false]%
%{%
%  \ifthenelse{\boolean{#1}}{\begin{trivlist}\item[\hskip \labelsep\bfseries Instructor Notes:\hspace{2ex}]}
%{\begin{trivlist}\item[\hskip \labelsep\bfseries Instructor Notes:\hspace{2ex}]}
%{}
%}
%% %% line at the bottom} 
%{\end{trivlist}\par\addvspace{.5ex}\nobreak\noindent\hung} 
%\fi
%
%


\let\instructorNotes\relax
\let\endinstructorNotes\relax
%%% instructorNotes environment
\ifhandout
\newenvironment{instructorNotes}[1][false]%
{%
\def\givenatend{\boolean{#1}}\ifthenelse{\boolean{#1}}{\begin{trivlist}\item}{\setbox0\vbox\bgroup}{}
}
{%
\ifthenelse{\givenatend}{\end{trivlist}}{\egroup}{}
}
\else
\newenvironment{instructorNotes}[1][false]%
{%
  \ifthenelse{\boolean{#1}}{\begin{trivlist}\item[\hskip \labelsep\bfseries {\Large Instructor Notes: \\} \hspace{\textwidth} ]}
{\begin{trivlist}\item[\hskip \labelsep\bfseries {\Large Instructor Notes: \\} \hspace{\textwidth} ]}
{}
}
{\end{trivlist}}
\fi


%% Suggested Timing
\newcommand{\timing}[1]{{\bf Suggested Timing: \hspace{2ex}} #1}




\hypersetup{
    colorlinks=true,       % false: boxed links; true: colored links
    linkcolor=blue,          % color of internal links (change box color with linkbordercolor)
    citecolor=green,        % color of links to bibliography
    filecolor=magenta,      % color of file links
    urlcolor=cyan           % color of external links
}

\title{Flour Power}
\author{Bart Snapp and Brad Findell}

\outcome{Learning outcome goes here.}

\begin{document}
\begin{abstract}
  We use models to motivate division.
\end{abstract}
\maketitle

\label{A:FlourPower}

\begin{problem} 
Suppose a cookie recipe calls for $2$ cups of flour. If you have $6$
cups of flour total, how many batches of cookies can you make?
\begin{enumerate}
\item Draw a picture representing the situation, and use pictures to solve the problem.
\item Identify whether the problem is asking ``How many groups?'' or ``How many in one group?'' or something else entirely.
\item You find another recipe that calls for $1\frac{1}{2}$ cups per batch. If you have $6$ cups of flour, how many batches of these cookies can you make?  Again use pictures to solve the problem.
\item Somebody once told you that ``to divide fractions, you invert and
multiply.'' Discuss how this rule is manifested in this problem.
\end{enumerate}
\end{problem}

\begin{problem} 
You have $2$ snazzy stainless steel containers (both the same size), which hold a total of
$6$ cups of flour. How many cups of flour does $1$ container hold?
\begin{enumerate}
\item Draw a picture representing the situation, and use pictures to solve the problem.
\item Identify whether the problem is asking ``How many groups?'' or ``How many in one group?'' or something else entirely.
\item It turned out that the 6 cups of flour filled exactly $1\frac{1}{2}$ of your containers.  How many cups of flour does $1$ container hold?  Again use pictures to solve the problem.
\item Somebody once told you that ``to divide fractions, you invert and
multiply.'' Discuss how this rule is manifested in this problem.
\end{enumerate}
\end{problem}


%\begin{problem} 
%Now you have $3$ beautiful decorative bowls, which hold a total of
%$1/2$ cup of flour. How many cups of flour does $1$ decorative bowl
%hold?
%\begin{enumerate}
%\item Draw a picture representing the situation, and use your picture to solve the problem.
%\item Identify whether the problem is asking ``How many groups?'' or ``How many in one group?'' or something else entirely.
%\item Somebody once told you that ``to divide fractions, you invert and
%multiply.'' Discuss how this rule is manifested in this problem.
%\end{enumerate}
%\end{problem}
%

\end{document}
