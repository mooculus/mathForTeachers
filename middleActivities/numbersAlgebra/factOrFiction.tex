%\documentclass[handout]{ximera}
\documentclass[nooutcomes]{ximera}


\graphicspath{
  {./}
  {graphics/}
  {../graphics/}
}

\usepackage{chngcntr}

\let\question\relax
\let\endquestion\relax




\newtheoremstyle{SlantTheorem}{\topsep}{\fill}%%% space between body and thm
%\newtheoremstyle{SlantTheorem}{\topsep}{\topsep}%%% space between body and thm
 {\slshape}                      %%% Thm body font
 {}                              %%% Indent amount (empty = no indent)
 {\bfseries\sffamily}            %%% Thm head font
 {}                              %%% Punctuation after thm head
 {3ex}                           %%% Space after thm head
 {\thmname{#1}\thmnumber{ #2}\thmnote{ \bfseries(#3)}}%%% Thm head spec
\theoremstyle{SlantTheorem}
\newtheorem{question}{Question}
\counterwithin*{question}{section}



\let\instructorNotes\relax
\let\endinstructorNotes\relax
%%% instructorNotes environment
\ifhandout
\newenvironment{instructorNotes}[1][false]%
{%
\def\givenatend{\boolean{#1}}\ifthenelse{\boolean{#1}}{\begin{trivlist}\item}{\setbox0\vbox\bgroup}{}
}
{%
\ifthenelse{\givenatend}{\end{trivlist}}{\egroup}{}
}
\else
\newenvironment{instructorNotes}[1][false]%
{%
  \ifthenelse{\boolean{#1}}{\begin{trivlist}\item[\hskip \labelsep\bfseries {\Large Instructor Notes: \\} \hspace{\textwidth} ]}
{\begin{trivlist}\item[\hskip \labelsep\bfseries {\Large Instructor Notes: \\} \hspace{\textwidth} ]}
{}
}
{\end{trivlist}}
\fi


%% Suggested Timing
\newcommand{\timing}[1]{{\bf Suggested Timing: \hspace{2ex}} #1}

\title{Pascal's Triangle: Fact or Fiction?}
\author{Bart Snapp and Brad Findell}

\outcome{Learning outcome goes here.}

\begin{document}
\begin{abstract}
  We investigate the connection between binomial coefficients and
  Pascal's triangle.
\end{abstract}
\maketitle

\label{A:factOrFiction}

\begin{teachingnote}
Connect to previous activity as follows:  Let $\binom{n}{k}$ denote the entry in the $n^{th}$ row and $k^{th}$ column, think of it as choosing $k$ green lights from $n$ lights, and use the shorthand language ``$n$ choose $k$.''  An ultimate goal for this content is that students can explain these patterns in many ways: Pascal's triangle, traffic lights, binomial coefficients, $n$ choose $k$ as an idea, the $n$ choose $k$ formula, pizza toppings, flipping coins, etc.
\end{teachingnote}
\begin{problem}
Consider the numbers $\binom{n}{k}$.  These numbers can be arranged
into a ``triangle'' form that is popularly called ``Pascal's
Triangle''.  Assuming that the ``top'' entry is $\binom{0}{0}=1$, we
write the numbers row by row, with $n$ fixed for each row.  Write out
the first 7 rows of Pascal's Triangle.

\vspace{3in}


Note that there are many patterns to be found.  Your job is to justify
the following patterns in the context of relevant models. Here are three patterns.  Can you explain them?
\begin{enumerate}
\item $\binom{n}{k} = \binom{n}{n-k}$.
\item The sum of the entries in each row is $2^n$.
\item $\binom{n}{k-1} + \binom{n}{k} = \binom{n+1}{k}$.
\end{enumerate}
\end{problem}

\end{document}
