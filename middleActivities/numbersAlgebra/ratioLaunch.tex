\newpage
\section{Ratios and Proportional Relationships}\label{A:ratioLaunch}
Here begins our work with ratios and proportional reasoning, which are the cornerstone of middle school mathematics.  Try to avoid procedural approaches, such as, ``set up a proportion and cross multiply.''  Instead, try to reason from the context and \textbf{use pictures and tables to support your reasoning}.  

As you solve these problems, note how the problems simultaneously build on understandings of fractions and pave the way for functions.  


\subsection*{Stacking Paper}
\begin{prob}
Suppose you want to know how many sheets are in a particular stack of paper, but don't want to count the pages directly. You have the following information:
\begin{itemize}
\item The given stack has height 4.50 cm.
\item A ream of 500 sheets has height 6.25 cm.
\end{itemize}
How many sheets of paper do you think are in the given stack?
\end{prob}

% From Stanley, 2014.  See more at: 
% http://blogs.ams.org/matheducation/2014/11/20/proportionality-confusion/

\begin{teachingnote}
Points to make: 
\begin{itemize}
\item To solve this problem, many people write a proportion and cross multiply, which might be fine if the only goal is the answer.  
But writing a proportion and cross multiplying misses opportunities for proportional reasoning.  
\item Draw out unit rates (80 sheets/cm and 0.0125 cm/sheet) and scale factors.  
\item Draw out quantities that are in a proportional relationship and write equations relating them.  
\item What is proportional to what?  
\end{itemize}
\end{teachingnote}

\begin{prob}
In your solution to the previous problem, what did you assume was proportional to what other quantity?  Be precise.  
\end{prob}

\subsection*{Mixing Punch}

\begin{prob}
Jenny is mixing punch and is considering two recipes:  
\begin{itemize}
\item Recipe A:  3 parts orange juice for every 5 parts ginger ale
\item Recipe B:  2 parts orange juice for every 3 parts ginger ale
\end{itemize}
\begin{enumerate}
\item Which recipe will give juice that is the most ``orangey''?  Explain your reasoning. 
\item Use a table to show various ways to make recipe B.  
\item To make 12 gallons of recipe B, how much of each will you need? 
\end{enumerate}
\end{prob}


\begin{teachingnote}
Draw out part:part versus part:whole comparisons, using quotients, common numerators, and common denominators.  In all cases, interpret the fractions and quotients.  

Use ratio tables, graphs, etc.  Note that graphs could be made relating any two of the three quantities (orange juice, ginger ale, punch).  

Draw out various unit rates:  3/5 parts orange juice for every 1 part of ginger ale... 
\end{teachingnote}

\subsection*{Racing Snails}
\begin{prob}
Mike is racing snails that move at a constant speed: 
\begin{itemize}
\item Snail A travels 3 inches in 5 minutes.  
\item Snail B travels 2 inches in 3 minutes. 
\end{itemize}
\begin{enumerate}
\item Which snail moves faster?  Explain your reasoning.
\item Use a table to show other distances and times for snail B.  
\end{enumerate}
\end{prob}


\begin{teachingnote}
These problems are very much the same as the previous problems.  But this time, the units are of different types, and they don't combine to make a new whole.  

Draw out both 2/3 in/min and 1.5 min./in as meaningful unit rates.  

Ratios are sometimes represented by fractions, but there is an important distinction:  A fraction is a single number, whereas a ratio
is often conceived as a relationship between two quantities.  
\end{teachingnote}
