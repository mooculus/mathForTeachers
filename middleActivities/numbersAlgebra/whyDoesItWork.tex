%\documentclass[handout]{ximera}
\documentclass{ximera}

\usepackage{gensymb}
\usepackage{tabularx}
\usepackage{mdframed}
\usepackage{pdfpages}
%\usepackage{chngcntr}

\let\problem\relax
\let\endproblem\relax

\newcommand{\property}[2]{#1#2}




\newtheoremstyle{SlantTheorem}{\topsep}{\fill}%%% space between body and thm
 {\slshape}                      %%% Thm body font
 {}                              %%% Indent amount (empty = no indent)
 {\bfseries\sffamily}            %%% Thm head font
 {}                              %%% Punctuation after thm head
 {3ex}                           %%% Space after thm head
 {\thmname{#1}\thmnumber{ #2}\thmnote{ \bfseries(#3)}} %%% Thm head spec
\theoremstyle{SlantTheorem}
\newtheorem{problem}{Problem}[]

%\counterwithin*{problem}{section}



%%%%%%%%%%%%%%%%%%%%%%%%%%%%Jenny's code%%%%%%%%%%%%%%%%%%%%

%%% Solution environment
%\newenvironment{solution}{
%\ifhandout\setbox0\vbox\bgroup\else
%\begin{trivlist}\item[\hskip \labelsep\small\itshape\bfseries Solution\hspace{2ex}]
%\par\noindent\upshape\small
%\fi}
%{\ifhandout\egroup\else
%\end{trivlist}
%\fi}
%
%
%%% instructorIntro environment
%\ifhandout
%\newenvironment{instructorIntro}[1][false]%
%{%
%\def\givenatend{\boolean{#1}}\ifthenelse{\boolean{#1}}{\begin{trivlist}\item}{\setbox0\vbox\bgroup}{}
%}
%{%
%\ifthenelse{\givenatend}{\end{trivlist}}{\egroup}{}
%}
%\else
%\newenvironment{instructorIntro}[1][false]%
%{%
%  \ifthenelse{\boolean{#1}}{\begin{trivlist}\item[\hskip \labelsep\bfseries Instructor Notes:\hspace{2ex}]}
%{\begin{trivlist}\item[\hskip \labelsep\bfseries Instructor Notes:\hspace{2ex}]}
%{}
%}
%% %% line at the bottom} 
%{\end{trivlist}\par\addvspace{.5ex}\nobreak\noindent\hung} 
%\fi
%
%


\let\instructorNotes\relax
\let\endinstructorNotes\relax
%%% instructorNotes environment
\ifhandout
\newenvironment{instructorNotes}[1][false]%
{%
\def\givenatend{\boolean{#1}}\ifthenelse{\boolean{#1}}{\begin{trivlist}\item}{\setbox0\vbox\bgroup}{}
}
{%
\ifthenelse{\givenatend}{\end{trivlist}}{\egroup}{}
}
\else
\newenvironment{instructorNotes}[1][false]%
{%
  \ifthenelse{\boolean{#1}}{\begin{trivlist}\item[\hskip \labelsep\bfseries {\Large Instructor Notes: \\} \hspace{\textwidth} ]}
{\begin{trivlist}\item[\hskip \labelsep\bfseries {\Large Instructor Notes: \\} \hspace{\textwidth} ]}
{}
}
{\end{trivlist}}
\fi


%% Suggested Timing
\newcommand{\timing}[1]{{\bf Suggested Timing: \hspace{2ex}} #1}




\hypersetup{
    colorlinks=true,       % false: boxed links; true: colored links
    linkcolor=blue,          % color of internal links (change box color with linkbordercolor)
    citecolor=green,        % color of links to bibliography
    filecolor=magenta,      % color of file links
    urlcolor=cyan           % color of external links
}

\title{Why Does It Work?}
\author{Bart Snapp and Brad Findell}

\outcome{Learning outcome goes here.}

\begin{document}
\begin{abstract}
Abstract goes here.  
\end{abstract}
\maketitle

\label{A:GCDwork}

The Euclidean Algorithm\index{Euclidean Algorithm} is pretty
neat. Let's see if we can figure out \textbf{why} it works. As a gesture of friendship, I'll compute $\gcd(351,153)$:
\begin{align*}
351 &= \boldsymbol{153}\cdot 2 + \boldsymbol{45}\\ 
\boldsymbol{153} &= \boldsymbol{45}\cdot 3 + \boldsymbol{18}\\
\boldsymbol{45} &= \boldsymbol{18}\cdot 2 + \fbox{$\boldsymbol{9}$}\\
18 &= 9\cdot 2 + 0 \qquad \fbox{$\therefore \gcd(351,153) = 9$}
\end{align*}

Let's look at this line-by-line.

\paragraph{The First Line}
\begin{problem}
Since $351 = 153\cdot 2 + 45$, explain why $\gcd(153,45)$ divides $351$.
\end{problem}

\begin{problem}
Since $351 = 153\cdot 2 + 45$, explain why $\gcd(351,153)$ divides $45$.
\end{problem}

\begin{problem}
Since $351 = 153\cdot 2 + 45$, explain why $\gcd(351,153) = \gcd(153,45)$.
\end{problem}


\paragraph{The Second Line}
\begin{problem}
Since $153 = 45\cdot 3 + 18$, explain why $\gcd(45,18)$ divides $153$.
\end{problem}

\begin{problem}
Since $153 = 45\cdot 3 + 18$, explain why $\gcd(153,45)$ divides $18$.
\end{problem}

\begin{problem}
Since $153 = 45\cdot 3 + 18$, explain why $\gcd(153,45) = \gcd(45,18)$.
\end{problem}


\paragraph{The Third Line}
\begin{problem}
Since $45 = 18\cdot 2 + 9$, explain why $\gcd(18,9)$ divides $45$.
\end{problem}

\begin{problem}
Since $45 = 18\cdot 2 + 9$, explain why $\gcd(45,18)$ divides $9$.
\end{problem}

\begin{problem}
Since $45 = 18\cdot 2 + 9$, explain why $\gcd(45,18) = \gcd(18,9)$.
\end{problem}


\paragraph{The Final Line}

\begin{problem}
Why are we done? How do you know that the Euclidean Algorithm
will \textbf{always} terminate?
\end{problem}

\fixnote{New question:  What does the final line look like when the GCD is 1?}  

\end{document}
