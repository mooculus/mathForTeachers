%\documentclass[handout]{ximera}
\documentclass[nooutcomes]{ximera}


\graphicspath{
  {./}
  {graphics/}
  {../graphics/}
}

\usepackage{chngcntr}

\let\question\relax
\let\endquestion\relax




\newtheoremstyle{SlantTheorem}{\topsep}{\fill}%%% space between body and thm
%\newtheoremstyle{SlantTheorem}{\topsep}{\topsep}%%% space between body and thm
 {\slshape}                      %%% Thm body font
 {}                              %%% Indent amount (empty = no indent)
 {\bfseries\sffamily}            %%% Thm head font
 {}                              %%% Punctuation after thm head
 {3ex}                           %%% Space after thm head
 {\thmname{#1}\thmnumber{ #2}\thmnote{ \bfseries(#3)}}%%% Thm head spec
\theoremstyle{SlantTheorem}
\newtheorem{question}{Question}
\counterwithin*{question}{section}



\let\instructorNotes\relax
\let\endinstructorNotes\relax
%%% instructorNotes environment
\ifhandout
\newenvironment{instructorNotes}[1][false]%
{%
\def\givenatend{\boolean{#1}}\ifthenelse{\boolean{#1}}{\begin{trivlist}\item}{\setbox0\vbox\bgroup}{}
}
{%
\ifthenelse{\givenatend}{\end{trivlist}}{\egroup}{}
}
\else
\newenvironment{instructorNotes}[1][false]%
{%
  \ifthenelse{\boolean{#1}}{\begin{trivlist}\item[\hskip \labelsep\bfseries {\Large Instructor Notes: \\} \hspace{\textwidth} ]}
{\begin{trivlist}\item[\hskip \labelsep\bfseries {\Large Instructor Notes: \\} \hspace{\textwidth} ]}
{}
}
{\end{trivlist}}
\fi


%% Suggested Timing
\newcommand{\timing}[1]{{\bf Suggested Timing: \hspace{2ex}} #1}

\title{Decimals Aren't So Nice}
\author{Bart Snapp and Brad Findell}

\outcome{Learning outcome goes here.}

\begin{document}
\begin{abstract}
  We note that every number has an infinite decimal representation.
\end{abstract}
\maketitle

\label{A:DecNotNice}

\index{paradox!1=0.9@$1 = 0.999\dots$} 

We will investigate the following question: How is $0.999\dots$
related to $1$?



\begin{problem}
What symbol do you think you should use to fill in the box below?
\[
.999\dots \,\fbox{\rule[0mm]{0mm}{2mm}\hspace{2ex}}\, 1
\]
Should you use $<$, $>$, $=$ or something else entirely?
\end{problem}


\begin{problem}
What is $1 - .999\dots$?
\end{problem}

\begin{problem}
How do you write $1/3$ in decimal notation? Express
\[
\frac{1}{3} + \frac{1}{3} + \frac{1}{3}
\]
in both fraction and decimal notation.
\end{problem}

\begin{problem}
See what happens when you follow the directions below:
\begin{enumerate}
\item Set $x = .999\dots$.
\item Compute $10x$. 
\item Compute $10x-x$.
\item From the step immediately above, what does $9x$ equal?
\item From the step immediately above, what does $x$ equal?
\end{enumerate}
\end{problem}

\begin{problem}
Are there other numbers with this weird property?
\end{problem}

\end{document}
