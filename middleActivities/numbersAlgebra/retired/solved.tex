\newpage
\section{Problem Solved!}\label{A:ProblemSolved}

Here's an old puzzler:


\begin{prob}
A man is riding a camel across a desert when he encounters a novel
sight. Three young men are fiercely arguing surrounded by 17
camels. Dismounting, the stranger was told that their father had died,
leaving (as their only real inheritance) these 17 camels. Now, the
eldest son was to receive half of the camels, the second son one-third
of the camels, the youngest son one-ninth of the camels. How could
they divide the 17 camels amongst themselves?  Explain your
reasoning.
\end{prob}

Uncharacteristically, I will solve this problem for you:

\begin{proof}[Solution] 
The man should generously add his camel to the $17$ being argued
over. Now there are $18$ camels to divide amongst the three
brothers. With this being the case:
\begin{itemize}
\item The eldest son receives $9$ camels.
\item The middle son receives $6$ camels.
\item The youngest son receives $2$ camels.
\end{itemize}
Ah! Since $9+6+2 = 17$, there is one left over, the man's original
camel---he can now have it back.
\end{proof}

\begin{prob}
What do you think of this solution?
\end{prob}

\begin{prob}
Describe your thought process when addressing the above
problem.
\end{prob}
