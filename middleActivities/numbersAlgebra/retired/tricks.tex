\newpage
\section{Turning Tricks into Techniques}

We will show you three separate \textit{tricks}, which are all quite
similar. By considering these tricks, we will develop techniques for
solving problems.

\paragraph{First Trick}\index{paradox!1=0.9@$1 = 0.999\dots$} 
As we shall see, not only do fractions have problems with multiple
representations---decimals have their issues too. How is $0.999\dots$
related to $1$? I claim we have the following paradox: I intend to
show that
\[
0.999\ldots = 1.
\]  
To see this, set:
\[
x=0.999\dots
\]
Now we have:
\[
10x =9.999\dots 
\]
So we see that 
\begin{align*}
10x - x &= (9.999\ldots) -(0.999\ldots) \\
9x &= 9,
\end{align*}
and we are forced to conclude that $x=1$, but we started off with the
assumption that $x= 0.999\dots$, hence $1 = 0.999\dots$.

This paradox challenges a common implicit (and false) notion that
every number has exactly one decimal representation. We are forced to
conclude that $0.999\dots$ and $1$ are representations for the same
number! To be completely explicit, using a similar method as described
above we see that
\begin{align*}
4.999\dots &= 5,\\
7.3999\dots &= 7.4,\\
23.745999\dots &= 23.746,
\end{align*}
and so on. Hence numbers can have multiple decimal representations.


\paragraph{Second Trick}

Keeping the first trick used above in the back of your mind, consider
this next similar trick. What is:
\[
x = \sqrt{1+\sqrt{1+\sqrt{1+\sqrt{1+\cdots}}}}
\]
Now we have:
\[
x = \sqrt{1+x} 
\]
Squaring both sides we get:
\[
x^2 = 1+x.
\]
Now putting everything on the left-hand side:
\[
x^2-x-1 = 0.
\]
By the Quadratic Formula, 
\[
x = \frac{1\pm \sqrt{5}}{2},
\]
and since we can see that $x> 1$, we must conclude that 
\[
x = \frac{1+ \sqrt{5}}{2}.
\]
Well this seems like a strange number. Oh well, let's keep on going.

\paragraph{Third Trick}
Now what about:
\[
x =   1 + \frac{1}{\displaystyle 1 
          + \frac{1}{\displaystyle 1 
          + \frac{1}{\displaystyle 1 
       + \frac{1}{\displaystyle 1 + \cdots
}}}}
\]
Again using a similar trick as above,
\[
x = 1+ \frac{1}{x}.
\]
Multiplying both sides by $x$ we get:
\[
x^2 = x + 1.
\]
Now putting everything on the left-hand side:
\[
x^2-x-1 = 0.
\]
By the Quadratic Formula, 
\[
x = \frac{1\pm \sqrt{5}}{2},
\]
and since we can see that $x> 1$ we must conclude that 
\[
x = \frac{1+ \sqrt{5}}{2}.
\]
Wait a minute, this says that:
\[
\sqrt{1+\sqrt{1+\sqrt{1+\sqrt{1+\cdots}}}} = 
1 + \frac{1}{\displaystyle 1 
          + \frac{1}{\displaystyle 1 
          + \frac{1}{\displaystyle 1 
       + \frac{1}{\displaystyle 1 + \cdots
}}}}
\]
Wow! Who would have ever thought that?  These two crazy looking
formulas are equal, and despite the fact that the only number in them
is a one, they are both equal to the messy number:
\[
\frac{1+\sqrt{5}}{2} = 1.6180339887\dots
\]

So what about the continued fraction:
\[
x = 1 + \frac{1}{\displaystyle 2
          + \frac{1}{\displaystyle 2 
          + \frac{1}{\displaystyle 2 
       + \frac{1}{\displaystyle 2 + \cdots
}}}}
\]
Using the technique as above,
\[
x = 1+ \frac{1}{1+x}.
\]
Multiplying both sides by $1+x$ we get:
\[
x+x^2 = (1+x) +1.
\]
Now putting all the $x$'s on the left-hand side and all the numbers on the right:
\[
x^2 = 2.
\]
Ah! So  $x = \sqrt{2}$. Using your calculator, you can see that:
\[
\sqrt{2} = 1.4142135623\dots \index{sqrt2@$\sqrt{2}$}
\]
This means that:
\[
1.4142135623\ldots = 1 + \frac{1}{\displaystyle 2
          + \frac{1}{\displaystyle 2 
          + \frac{1}{\displaystyle 2 
       + \frac{1}{\displaystyle 2 + \cdots
}}}}
\]
Note that on the left-hand side you don't see much of a
pattern. However, on the right-hand side a clear pattern is
formed. This is part of the beauty of continued fractions. Now it
turns out that you can do this with other numbers and and get lots of
other cool patterns!


\item Find the exact value for $x$ when:
\[
x = \sqrt{2+\sqrt{2+\sqrt{2+\sqrt{2+\cdots}}}}
\]
Explain your work.
\item Find the exact value for $x$ when:
\[
x = \sqrt{6+\sqrt{6+\sqrt{6+\sqrt{6+\cdots}}}}
\]
Explain your work.
\item Find the exact value for $x$ when:
\[
x = \sqrt{12+\sqrt{12+\sqrt{12+\sqrt{12+\cdots}}}}
\]
Explain your work.
\item Find the exact value for $x$ when:
\[
x = \sqrt{20+\sqrt{20+\sqrt{20+\sqrt{20+\cdots}}}}
\]
Explain your work.
\item Find the exact value for $x$ when:
\[
x = \sqrt{30+\sqrt{30+\sqrt{30+\sqrt{30+\cdots}}}}
\]
Explain your work.
\item Find the exact value for $x$ when:
\[
  x = 2 + \frac{1}{\displaystyle 4 
          + \frac{1}{\displaystyle 4
          + \frac{1}{\displaystyle 4
          + \frac{1}{\displaystyle 4 +\cdots
}}}}
\]
Explain your work.
\item Find $x$ when:
\[
  x = 4 + \frac{1}{\displaystyle 6
          + \frac{1}{\displaystyle 6
          + \frac{1}{\displaystyle 6
          + \frac{1}{\displaystyle 6 +\cdots
}}}}
\]
Explain your work.
\item Find the exact value for $x$ when:
\[
  x = 4 + \frac{1}{\displaystyle 8 
          + \frac{1}{\displaystyle 8
          + \frac{1}{\displaystyle 8
          + \frac{1}{\displaystyle 8 +\cdots
}}}}
\]
Explain your work.
\item Find the exact value for $x$ when:
\[
  x = 3 + \frac{1}{\displaystyle 10 
          + \frac{1}{\displaystyle 10
          + \frac{1}{\displaystyle 10
          + \frac{1}{\displaystyle 10 +\cdots
}}}}
\]
Explain your work.
