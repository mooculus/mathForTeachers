\newpage
\section{To the Second Degree}\label{A:deg2Ext}

In this activity, we seek to understand why roots of polynomials with
real coefficients must always come in conjugate pairs.

\begin{prob}
Consider your favorite (non-real) complex number, I'll call it
$\xi$. Find a polynomial with real coefficients whose degree is as
small as possible having your number as a root. What is the degree
of your polynomial?
\end{prob}


\begin{prob}
I'll call the polynomial found in the first problem $s(x)$. Let $f(x)$
be some other polynomial with
\[
f(\xi) = 0.
\]
I claim $s(x) | f(x)$. Explain why if $s(x)\nmid f(x)$ then there exist $q(x)$ and $r(x)$ with
\[
f(x) = s(x) \cdot q(x) + r(x)\qquad\text{with }\deg(r) <\deg(s).
\]
\end{prob}

\begin{prob}
Plug in $\xi$ for $x$ in the equation above. What does this tell you
about $r(\xi)$? Is this possible?
\end{prob}

\begin{prob}
Explain why complex roots must always come in conjugate pairs. Also
plot some conjugate pairs in the complex plane and explain what
``conjugation'' means geometrically.
\end{prob}
