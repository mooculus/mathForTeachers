\newpage
\section{Broken Records}\label{A:orderMod}

Fill in the following table:
\[
{\renewcommand{\arraystretch}{1.8}
\renewcommand{\arraycolsep}{3mm}
\begin{array}{|r||c|c|c|c|c|c|c|c|c|c|}\hline
\text{modulus:} & 2 & 3 & 4 & 5 & 6 & 7 & 8 & 9 & 10 & 11 \\ \hline\hline
2\cdot 1 \equiv & & & & & & & & & & \\ \hline
2\cdot 2 \equiv & & & & & & & & & & \\ \hline
2\cdot 3 \equiv & & & & & & & & & & \\ \hline
2\cdot 4 \equiv & & & & & & & & & & \\ \hline
2\cdot 5 \equiv & & & & & & & & & & \\ \hline
2\cdot 6 \equiv & & & & & & & & & & \\ \hline
2\cdot 7 \equiv & & & & & & & & & & \\ \hline
2\cdot 8 \equiv & & & & & & & & & & \\ \hline
2\cdot 9 \equiv & & & & & & & & & & \\ \hline
2\cdot 10\equiv & & & & & & & & & & \\ \hline
2\cdot 11\equiv & & & & & & & & & & \\ \hline
\end{array}}
\]

\begin{prob} 
Find patterns in your table above, clearly describe the patterns you find.
\end{prob}

\begin{prob} 
Consider the patterns you found. Can you explain why they happen?
\end{prob}

\begin{prob}
When does a column have a $0$? When does a column have a $1$? 
\end{prob}

\begin{prob}
Describe what would happen if you extend the table for bigger
moduli and bigger multiplicands.
\end{prob}


\vfill

\newpage


\[
{\renewcommand{\arraystretch}{1.8}
\renewcommand{\arraycolsep}{3mm}
\begin{array}{|r||c|c|c|c|c|c|c|c|c|c|}\hline
\text{modulus:} & 2 & 3 & 4 & 5 & 6 & 7 & 8 & 9 & 10 & 11 \\ \hline\hline
3\cdot 1 \equiv & & & & & & & & & & \\ \hline
3\cdot 2 \equiv & & & & & & & & & & \\ \hline
3\cdot 3 \equiv & & & & & & & & & & \\ \hline
3\cdot 4 \equiv & & & & & & & & & & \\ \hline
3\cdot 5 \equiv & & & & & & & & & & \\ \hline
3\cdot 6 \equiv & & & & & & & & & & \\ \hline
3\cdot 7 \equiv & & & & & & & & & & \\ \hline
3\cdot 8 \equiv & & & & & & & & & & \\ \hline
3\cdot 9 \equiv & & & & & & & & & & \\ \hline
3\cdot 10\equiv & & & & & & & & & & \\ \hline
3\cdot 11\equiv & & & & & & & & & & \\ \hline
\end{array}}
\]

\begin{prob} 
Find patterns in your table above, clearly describe the patterns you find.
\end{prob}

\begin{prob} 
Consider the patterns you found. Can you explain why they happen?
\end{prob}


\begin{prob}
When does a column have a $0$? When does a column have a $1$? 
\end{prob}


\begin{prob}
Can you describe what would happen if you extend the table for bigger
moduli and bigger multiplicands?
\end{prob}

\begin{prob}
Describe precisely when a column of the table will contain
representatives for each integer modulo $n$. Explain why your
description is true.
\end{prob}

