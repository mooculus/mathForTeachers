\documentclass[handout,nooutcomes, noauthor]{ximera}

\title{Strategy For Success}

\begin{document}
\begin{abstract}
\end{abstract}
\maketitle

\begin{question}
    Before we get started, suppose you have two fractions that are equal, like maybe 
    \[
    \frac{3}{5} = \frac{x}{12}.
    \]
    
    Find the value of $x$ using a picture. Explain your thinking. \vfill
\end{question}

\begin{question}
    In the previous problem, explain how your picture shows you how you could have solved the problem using the equation $3\cdot 12 = 5 x$. (You might have to adjust your picture to see this!) Where do you see the multiplication in your picture? Where do you see the equals sign? \vfill
\end{question}

\begin{question}
    Summarize your work: why does cross-multiplication make sense to solve such an equation? \vfill
\end{question}

\newpage
\begin{problem}
Now, let's consider a paint problem.

{\em Jessie is planning to make a lovely shade of purple paint by mixing $5 $ cups of red paint with $3$ cups of blue paint. Suddenly, she realizes she actually has $12$ cups of red paint, not $5$. How many cups of blue paint should Jessie mix with her $12$ cups of red paint to make the same shade of purple paint?}

\begin{enumerate}
    \item Explain why the fraction $\frac{3}{5}$ is related to this problem. (Explain in terms of the meaning of fractions, not ratios!) \vfill
    \item If we use $x$ to represent the answer to this question, explain why the fraction $\frac{x}{12}$ is related to this problem. (Explain in terms of the meaning of fractions, not ratios!) \vfill
    \item Explain why the fractions in part (a) and (b) are equal to one another in this story.\vfill
\end{enumerate}
\end{problem}


\begin{question}
    
Summarize your work: why does it make sense to set up a proportion and cross-multiply to solve the paint problem? \vfill
\end{question}

\begin{question}
    What other proportions could you set up for the paint problem, and why? \vfill
\end{question}




\end{document}