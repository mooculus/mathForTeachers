\documentclass[handout,nooutcomes, noauthor]{ximera}

\title{Commonality}

\begin{document}
\begin{abstract}
\end{abstract}
\maketitle

\begin{problem}
You have received a large piece of paper which measures 40 inches by 60 inches.  You'd like to draw a large grid (of rectangles, not necessarily squares) on this paper.  What size rectangles could you use for your grid?  Did you get the same answer as each of your neighbors?
\vfill
\end{problem}


\begin{problem}
At a local carnival, you sneak in behind one of the slot machines and rig it to come up with three cherries (one in each window to hit the jackpot) at a time only known to you.  You know the first window will come up with a cherry every $15$ times, the second window every $24$ times, and the third window every $40$ times.  When should you step in to win the jackpot?  Did you get the same answer as each of your neighbors?
\vfill
\end{problem}

\newpage
\begin{problem}
Let $A = 2^{23} \times 3^{87} \times 5^{11} \times 7^{14}$ and $B = 2^{18} \times 3^{54} \times 5^{26} \times 7^{16}$.
\begin{enumerate}
    \item Find some common factors of $A$ and $B$.
    \item Find some common multiples of $A$ and $B$.
\end{enumerate}
\vfill
\end{problem}

\begin{problem}
The {\bf Division Theorem} states that given any (positive) integer $n$ and a nonzero (positive) integer $d$, there exist unique integers $q$ and $r$ such that $n=dq+r$ and \hspace{0.5in}.
\begin{enumerate}
    \item What is the missing condition in the theorem?  In other words, what could we specify to get a unique answer?  How could we get multiple answers?
    \item How might you draw a picture to illustrate this theorem?
    \item Would any of your answers above change if $n$ and $d$ were negative integers?  Would the theorem still be true?
\end{enumerate}
\vfill
\end{problem}



\end{document}