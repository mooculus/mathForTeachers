\newpage
\section{Meanings of Exponents}\label{A:MeaningsOfExponents}

Students in grades 3-7 can use their understanding of counting number arithmetic to build understandings of the arithmetic of negative integers and rational numbers.  Here are the key ideas: 
\begin{quote}
\begin{itemize}
\item The properties of operations (commutative, associative, and distributive properties) are established for counting numbers based on meanings of operations. 
\item As we extend arithmetic to negative integers and rational numbers, we want the properties of operations to continue to hold.  
\end{itemize}
\end{quote}

This activity follows an analogous process for exponents: Students use their understanding of counting number exponents to build an understanding of negative integer and rational exponents.  Here are the key ideas: 
\begin{quote}
\begin{itemize}
\item The rules of exponents are established for counting number exponents based on the meaning of an exponent.  
\item As we extend to negative and rational exponents, we want the rules of exponents to continue to hold.  
\end{itemize}
\end{quote}

\begin{prob}
Students sometimes say that $a^n$ means ``$a$ multiplied by itself $n$ times.''  But for counting number exponents, this is not correct.  For example, how many multiplications are there in $3^5$?  Write a better definition for $a^n$, where $n$ is a counting number.  
\end{prob}

\begin{prob}
Why is $x^3$ not the same function as $3^x$?  We often think of multiplication as ``repeated addition,'' and we find that adding $a$ copies of $b$ gives the same result as adding $b$ copies of $a$.  Does this idea work for thinking of exponentiation as ``repeated multiplication''?  Explain.  
\end{prob}

\begin{prob}
If you do not know (or do not remember) the rules for exponents, you can still use your definition of $a^n$ to figure out other ways of writing expressions with exponents.  Use \textbf{specific values} for letters in expressions of the form $a^na^m$, 
$a^n/a^m$, $(a^n)^m$, and $(ab)^n$ for counting-number exponents, to explain what the rules must be.  Choose specific values that help you explain generally.  
\end{prob}

\begin{prob}
\textbf{Patterns.}  One way to reason about the meanings of zero and negative exponents is to use patterns.  As you complete the following table, \textbf{imagine that you know nothing about zero and negative exponents.}  Instead, use the patterns in the values for positive exponents to reason about what the values should be for zero and negative exponents.  Then reason generally about the meaning of $a^0$ and $a^{-n}$, where $n$ is a counting number and $a$ is a real number.  Are there any values of $a$ for which your reasoning is not valid?  Explain.  
\end{prob}

\begin{minipage}{0.3\textwidth}
\begin{align*}
2^3 &=  \\
2^2 & = \\
2^1 &= \\
2^0 &= \\
2^{-1} &= \\
2^{-2} &= \\
2^{-3} &= \\
\end{align*}
\end{minipage}
\begin{minipage}{0.3\textwidth}
\begin{align*}
3^3 &=  \\
3^2 & = \\
3^1 &= \\
3^0 &= \\
3^{-1} &= \\
3^{-2} &= \\
3^{-3} &= \\
\end{align*}
\end{minipage}
\begin{minipage}{0.3\textwidth}
\begin{align*}
(-2)^3 &=  \\
(-2)^2 & = \\
(-2)^1 &= \\
(-2)^0 &= \\
(-2)^{-1} &= \\
(-2)^{-2} &= \\
(-2)^{-3} &= \\
\end{align*}
\end{minipage}
\begin{minipage}{0.3\textwidth}
\begin{align*}
\left(\frac{1}{2}\right)^3 &=  \\
\left(\frac{1}{2}\right)^2 & = \\
\left(\frac{1}{2}\right)^1 &= \\
\left(\frac{1}{2}\right)^0 &= \\
\left(\frac{1}{2}\right)^{-1} &= \\
\left(\frac{1}{2}\right)^{-2} &= \\
\left(\frac{1}{2}\right)^{-3} &= \\
\end{align*}
\end{minipage}


\begin{prob}
\textbf{Extending the rules.}  A careful way to approach zero and negative integer exponents is to use the rules of exponents (which you established above for counting-number exponents) to determine what 0 and negative integer exponents must mean if the exponent rules continue to hold in this extended domain.
\begin{enumerate}
\item Use the exponent rules to provide two explanations for a sensible definition of $a^0$, being clear about why your definition makes sense.  Note any restrictions on $a$.  
\item Use the exponent rules to provide two explanations for a sensible definition of $a^{-n}$, where $n$ is a counting number.  Again, note any restrictions on $a$.
\end{enumerate}
\end{prob}

\begin{prob}
While trying to decide what $3^{\frac{2}{5}}$ should mean, Katie wondered about the expression $\left(3^{\frac{2}{5}}\right)^5$.  What should Katie's expression be equal to?  Explain, using rules of exponents.  Then use Katie's idea to determine a value for $3^{\frac{2}{5}}$.  
\end{prob}



