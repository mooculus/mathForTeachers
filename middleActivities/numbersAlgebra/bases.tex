\documentclass{ximera}

\usepackage{gensymb}
\usepackage{tabularx}
\usepackage{mdframed}
\usepackage{pdfpages}
%\usepackage{chngcntr}

\let\problem\relax
\let\endproblem\relax

\newcommand{\property}[2]{#1#2}




\newtheoremstyle{SlantTheorem}{\topsep}{\fill}%%% space between body and thm
 {\slshape}                      %%% Thm body font
 {}                              %%% Indent amount (empty = no indent)
 {\bfseries\sffamily}            %%% Thm head font
 {}                              %%% Punctuation after thm head
 {3ex}                           %%% Space after thm head
 {\thmname{#1}\thmnumber{ #2}\thmnote{ \bfseries(#3)}} %%% Thm head spec
\theoremstyle{SlantTheorem}
\newtheorem{problem}{Problem}[]

%\counterwithin*{problem}{section}



%%%%%%%%%%%%%%%%%%%%%%%%%%%%Jenny's code%%%%%%%%%%%%%%%%%%%%

%%% Solution environment
%\newenvironment{solution}{
%\ifhandout\setbox0\vbox\bgroup\else
%\begin{trivlist}\item[\hskip \labelsep\small\itshape\bfseries Solution\hspace{2ex}]
%\par\noindent\upshape\small
%\fi}
%{\ifhandout\egroup\else
%\end{trivlist}
%\fi}
%
%
%%% instructorIntro environment
%\ifhandout
%\newenvironment{instructorIntro}[1][false]%
%{%
%\def\givenatend{\boolean{#1}}\ifthenelse{\boolean{#1}}{\begin{trivlist}\item}{\setbox0\vbox\bgroup}{}
%}
%{%
%\ifthenelse{\givenatend}{\end{trivlist}}{\egroup}{}
%}
%\else
%\newenvironment{instructorIntro}[1][false]%
%{%
%  \ifthenelse{\boolean{#1}}{\begin{trivlist}\item[\hskip \labelsep\bfseries Instructor Notes:\hspace{2ex}]}
%{\begin{trivlist}\item[\hskip \labelsep\bfseries Instructor Notes:\hspace{2ex}]}
%{}
%}
%% %% line at the bottom} 
%{\end{trivlist}\par\addvspace{.5ex}\nobreak\noindent\hung} 
%\fi
%
%


\let\instructorNotes\relax
\let\endinstructorNotes\relax
%%% instructorNotes environment
\ifhandout
\newenvironment{instructorNotes}[1][false]%
{%
\def\givenatend{\boolean{#1}}\ifthenelse{\boolean{#1}}{\begin{trivlist}\item}{\setbox0\vbox\bgroup}{}
}
{%
\ifthenelse{\givenatend}{\end{trivlist}}{\egroup}{}
}
\else
\newenvironment{instructorNotes}[1][false]%
{%
  \ifthenelse{\boolean{#1}}{\begin{trivlist}\item[\hskip \labelsep\bfseries {\Large Instructor Notes: \\} \hspace{\textwidth} ]}
{\begin{trivlist}\item[\hskip \labelsep\bfseries {\Large Instructor Notes: \\} \hspace{\textwidth} ]}
{}
}
{\end{trivlist}}
\fi


%% Suggested Timing
\newcommand{\timing}[1]{{\bf Suggested Timing: \hspace{2ex}} #1}




\hypersetup{
    colorlinks=true,       % false: boxed links; true: colored links
    linkcolor=blue,          % color of internal links (change box color with linkbordercolor)
    citecolor=green,        % color of links to bibliography
    filecolor=magenta,      % color of file links
    urlcolor=cyan           % color of external links
}

\title{Shelby and Scotty}

\author{Bart Snapp and Brad Findell}


\outcome{Understand base conversion.}

\begin{document}
\begin{abstract}
We find algorithms for changing bases.
\end{abstract}
\maketitle

\label{A:SS}

Note: In this activity, we use words (rather than numerals) to indicate bases.  
And we use use a subscript after a numeral to specify its base.  

Shelby and Scotty want to express the (base ten) number $27$ in base
four. However, they used very different methods to do this. Let's check
them out.

\begin{teachingnote}
Consider first asking students to do this problem themselves.  Then they are more likely ready to interpret the methods below, and chances are some students will do it like Shelby and others like Scotty. 
\end{teachingnote}

\begin{problem} Consider Shelby's work:
\[
4\,\begin{tabular}[b]{@{}r@{} r}
$6$ &\, R\,$\boldsymbol{3}$\\ \cline{1-1}
\Big)\begin{tabular}[t]{@{}l@{}}
$27$ 
\end{tabular}
\end{tabular}
\qquad
4\,\begin{tabular}[b]{@{}r@{} r}
$1$ &\, R\,$\boldsymbol{2}$\\ \cline{1-1}
\Big)\begin{tabular}[t]{@{}l@{}}
$6$ 
\end{tabular}
\end{tabular}
\qquad
4\,\begin{tabular}[b]{@{}r@{} r}
$0$ &\, R\,$\boldsymbol{1}$\\ \cline{1-1}
\Big)\begin{tabular}[t]{@{}l@{}}
$1$ 
\end{tabular}
\end{tabular} \qquad \Rightarrow \qquad \fbox{$123_\textrm{four}$}
\]
\begin{enumerate}
\item Describe how to perform this algorithm.
\item Provide an additional relevant and revealing example
  demonstrating that you understand the algorithm.
\end{enumerate}
\end{problem}

\begin{problem} 
Using the $27$ marks below, create an illustration (or series of illustrations) that models Shelby's method for changing bases.
\[
|\;\;|\;\;|\;\;|\;\;|\;\;|\;\;|\;\;|\;\;|\;\;|\;\;|\;\;|\;\;|\;\;|\;\;|\;\;|\;\;|\;\;|\;\;|\;\;|\;\;|\;\;|\;\;|\;\;|\;\;|\;\;|\;\;|
\]
Further, explain why Shelby's method works. 
\end{problem}


\begin{problem} Consider Scotty's work:
\[
4^3\,\begin{tabular}[b]{@{}r@{} r}
$0$ &\, R\,$27$\\ \cline{1-1}
\Big)\begin{tabular}[t]{@{}l@{}}
$27$ 
\end{tabular}
\end{tabular}
\qquad
4^2\,\begin{tabular}[b]{@{}r@{} r}
$\boldsymbol{1}$ &\, R\,$11$\\ \cline{1-1}
\Big)\begin{tabular}[t]{@{}l@{}}
$27$ 
\end{tabular}
\end{tabular}
\qquad
4\,\begin{tabular}[b]{@{}r@{} r}
$\boldsymbol{2}$ &\, R\,$\boldsymbol{3}$\\ \cline{1-1}
\Big)\begin{tabular}[t]{@{}l@{}}
$11$ 
\end{tabular}
\end{tabular} \qquad \Rightarrow \qquad \fbox{$123_\textrm{four}$}
\]
\begin{enumerate}
\item Describe how to perform this algorithm.
\item Provide an additional relevant and revealing example
  demonstrating that you understand the algorithm.
\end{enumerate}
\end{problem}


\begin{problem} 
Using the $27$ marks below, create an illustration (or series of illustrations) that models Scotty's method for changing bases.
\[
|\;\;|\;\;|\;\;|\;\;|\;\;|\;\;|\;\;|\;\;|\;\;|\;\;|\;\;|\;\;|\;\;|\;\;|\;\;|\;\;|\;\;|\;\;|\;\;|\;\;|\;\;|\;\;|\;\;|\;\;|\;\;|\;\;|
\]
Further, explain why Scotty's method works. 
\end{problem}

\begin{problem}
Use both methods to write $1644_\textrm{ten}$ in base seven.
\end{problem}

\begin{problem}
Now let's try to be more efficient.  
\begin{enumerate}
\item Convert $8630_\textrm{ten}$ to base thirteen.  Use $A$ for ten, $B$ for eleven, and $C$ for twelve.  
\item Quickly convert $2102_\textrm{three}$ to base nine.
\item Without using base ten, convert $341_\textrm{six}$ to base four.  
\item Without using base ten, convert $341_\textrm{six}$ to base eleven.
\end{enumerate}
\end{problem}
\end{document}

