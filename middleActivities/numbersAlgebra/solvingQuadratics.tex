%\documentclass[handout]{ximera}
\documentclass[nooutcomes]{ximera}

\usepackage{gensymb}
\usepackage{tabularx}
\usepackage{mdframed}
\usepackage{pdfpages}
%\usepackage{chngcntr}

\let\problem\relax
\let\endproblem\relax

\newcommand{\property}[2]{#1#2}




\newtheoremstyle{SlantTheorem}{\topsep}{\fill}%%% space between body and thm
 {\slshape}                      %%% Thm body font
 {}                              %%% Indent amount (empty = no indent)
 {\bfseries\sffamily}            %%% Thm head font
 {}                              %%% Punctuation after thm head
 {3ex}                           %%% Space after thm head
 {\thmname{#1}\thmnumber{ #2}\thmnote{ \bfseries(#3)}} %%% Thm head spec
\theoremstyle{SlantTheorem}
\newtheorem{problem}{Problem}[]

%\counterwithin*{problem}{section}



%%%%%%%%%%%%%%%%%%%%%%%%%%%%Jenny's code%%%%%%%%%%%%%%%%%%%%

%%% Solution environment
%\newenvironment{solution}{
%\ifhandout\setbox0\vbox\bgroup\else
%\begin{trivlist}\item[\hskip \labelsep\small\itshape\bfseries Solution\hspace{2ex}]
%\par\noindent\upshape\small
%\fi}
%{\ifhandout\egroup\else
%\end{trivlist}
%\fi}
%
%
%%% instructorIntro environment
%\ifhandout
%\newenvironment{instructorIntro}[1][false]%
%{%
%\def\givenatend{\boolean{#1}}\ifthenelse{\boolean{#1}}{\begin{trivlist}\item}{\setbox0\vbox\bgroup}{}
%}
%{%
%\ifthenelse{\givenatend}{\end{trivlist}}{\egroup}{}
%}
%\else
%\newenvironment{instructorIntro}[1][false]%
%{%
%  \ifthenelse{\boolean{#1}}{\begin{trivlist}\item[\hskip \labelsep\bfseries Instructor Notes:\hspace{2ex}]}
%{\begin{trivlist}\item[\hskip \labelsep\bfseries Instructor Notes:\hspace{2ex}]}
%{}
%}
%% %% line at the bottom} 
%{\end{trivlist}\par\addvspace{.5ex}\nobreak\noindent\hung} 
%\fi
%
%


\let\instructorNotes\relax
\let\endinstructorNotes\relax
%%% instructorNotes environment
\ifhandout
\newenvironment{instructorNotes}[1][false]%
{%
\def\givenatend{\boolean{#1}}\ifthenelse{\boolean{#1}}{\begin{trivlist}\item}{\setbox0\vbox\bgroup}{}
}
{%
\ifthenelse{\givenatend}{\end{trivlist}}{\egroup}{}
}
\else
\newenvironment{instructorNotes}[1][false]%
{%
  \ifthenelse{\boolean{#1}}{\begin{trivlist}\item[\hskip \labelsep\bfseries {\Large Instructor Notes: \\} \hspace{\textwidth} ]}
{\begin{trivlist}\item[\hskip \labelsep\bfseries {\Large Instructor Notes: \\} \hspace{\textwidth} ]}
{}
}
{\end{trivlist}}
\fi


%% Suggested Timing
\newcommand{\timing}[1]{{\bf Suggested Timing: \hspace{2ex}} #1}




\hypersetup{
    colorlinks=true,       % false: boxed links; true: colored links
    linkcolor=blue,          % color of internal links (change box color with linkbordercolor)
    citecolor=green,        % color of links to bibliography
    filecolor=magenta,      % color of file links
    urlcolor=cyan           % color of external links
}

\title{Solving Quadratics}
\author{Bart Snapp and Brad Findell}

\outcome{Learning outcome goes here.}

\begin{document}
\begin{abstract}
  We solve quadratic equations.
\end{abstract}
\maketitle

\label{A:solvingQuadratics}
Here we explore various methods for solving quadratic equations in one variable.  \textbf{Please read all instructions carefully.}

\begin{problem}
Is $\sqrt{4}=\pm 2$?  Explain. 
\end{problem}

\vfill

\begin{teachingnote}
Both $2$ and $-2$ are ``square roots'' of $4$ because $2^2=4$ and $(-2)^2=4$, and both of them are solutions to the equation $x^2=4$.  The question is whether the radical symbol refers to both of them, either of them (you choose?), or a specific one of them.  
\end{teachingnote}

\begin{problem}
Suppose that $\sqrt{4}=\pm 2$?  Then evaluate $\sqrt{4}+\sqrt{9}$.  
\end{problem}

\begin{teachingnote}
$$\sqrt{4}+\sqrt{9}=\pm2+\pm3=5, -1, 1, or -5$$
\end{teachingnote}

\vfill

\begin{problem}
What does your calculator say about $\sqrt{4}+\sqrt{9}$?  
\end{problem}

\vfill 

\begin{teachingnote}
Now emphasize the conventional meaning of the radical symbol:  For $a>0$ then $\sqrt{a}$ means the positive square root of $a$.  
\end{teachingnote}



\begin{problem}
In the following problems, you \textbf{may not use the quadratic formula}.  But just for the record, write down the quadratic formula.  
\end{problem}
\begin{teachingnote}
Many students will write only $\frac{-b\pm\sqrt{b^2-4ac}}{2a}$, but we want them to write the following:  
$$\text{If }ax^2+bx+c=0\text{ and }a\ne 0\text{, then }x=\frac{-b\pm\sqrt{b^2-4ac}}{2a}\text{.}$$
Note that if the radical symbol were to refer to both a positive and negative square root, then there would be no reason to write $\pm$ outside the radical symbol.  
\end{teachingnote}
\vspace{0.8in}

\begin{problem}
In the following list of equations, solve those that are \textbf{easy} to solve.  
\begin{enumerate}
\item $(x-3)(x+2)=0$
\item $(x-3)(x+2)=1$
\item $(2x-5)(3x+1)=0$
\item $(x-a)(x-b)=0$
\item $(x-1)(x-3)(x+2)(2x-3)=0$
\end{enumerate}
\end{problem}

\begin{problem}
Regarding the previous problem, state the property of numbers that made all but one of the equations easy to solve.  
\end{problem}

\begin{teachingnote}
Zero product property:  If $ab = 0$ then $a=0$ or $b=0$.  Note that this doesn't work when the right side is not 0.  
\end{teachingnote}

\vspace{0.3in}


%\begin{problem} For each part below, write a linear equation with the state solution.  
%\begin{enumerate}
%\item $x=\frac{2}{3}$
%\item $x=\frac{a}{b}$
%\end{enumerate}
%\end{problem}

\begin{problem}For each part below, write a quadratic equation with the stated solution(s) and no other solutions.  
\begin{enumerate}
\item $x=7$ or $x=-4$
\item $x=p$ or $x=q$
\item $x=3$
\item $x=\frac{1\pm\sqrt{5}}{2}$
 \end{enumerate}
\end{problem}

\begin{problem}
Find all solutions to $x^3-3x^2+x+1=0$.  Hint:  One solution is $x=1$.  
\end{problem}

\end{document}
