\documentclass[handout,nooutcomes, noauthor]{ximera}

\title{Differentiating Division}

\begin{document}
\begin{abstract}
\end{abstract}
\maketitle

Let's consider the division problem $26 \div 4$.

\begin{question}
What's the usual answer to this problem? Did your group members have a different idea for what the ``usual answer'' should be?  Why or why not?

\end{question}


\begin{problem}
Give an example of a story or situation in which the answer to $26 \div 4$ is ``$6$ remainder $2$''.  Explain how you know this is the appropriate answer in this situation.
\end{problem}

\begin{problem}
Give an example of a story or situation in which the answer to $26 \div 4$ is $7$.  Explain how you know this is the appropriate answer in this situation.
\end{problem}


\begin{problem}
Give an example of a story or situation in which the answer to $26 \div 4$ is $6$.  Explain how you know this is the appropriate answer in this situation.
\end{problem}


The Division Theorem states: Given any whole positive number $n$ and a nonzero positive whole number $d$, there exist unique integers $q$ and $r$ such that 
\[
n = dq + r
\]
with $0 \leq r < d$.

\begin{question}
What is this theorem saying?  What does it have to do with the questions on this page?
\end{question}

\begin{problem}
With your group, try to bring this all together.  What have you learned about division?  What is the main point of this activity?
\end{problem}

\end{document}