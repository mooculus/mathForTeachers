%\documentclass[handout]{ximera}
\documentclass[nooutcomes]{ximera}

\usepackage{gensymb}
\usepackage{tabularx}
\usepackage{mdframed}
\usepackage{pdfpages}
%\usepackage{chngcntr}

\let\problem\relax
\let\endproblem\relax

\newcommand{\property}[2]{#1#2}




\newtheoremstyle{SlantTheorem}{\topsep}{\fill}%%% space between body and thm
 {\slshape}                      %%% Thm body font
 {}                              %%% Indent amount (empty = no indent)
 {\bfseries\sffamily}            %%% Thm head font
 {}                              %%% Punctuation after thm head
 {3ex}                           %%% Space after thm head
 {\thmname{#1}\thmnumber{ #2}\thmnote{ \bfseries(#3)}} %%% Thm head spec
\theoremstyle{SlantTheorem}
\newtheorem{problem}{Problem}[]

%\counterwithin*{problem}{section}



%%%%%%%%%%%%%%%%%%%%%%%%%%%%Jenny's code%%%%%%%%%%%%%%%%%%%%

%%% Solution environment
%\newenvironment{solution}{
%\ifhandout\setbox0\vbox\bgroup\else
%\begin{trivlist}\item[\hskip \labelsep\small\itshape\bfseries Solution\hspace{2ex}]
%\par\noindent\upshape\small
%\fi}
%{\ifhandout\egroup\else
%\end{trivlist}
%\fi}
%
%
%%% instructorIntro environment
%\ifhandout
%\newenvironment{instructorIntro}[1][false]%
%{%
%\def\givenatend{\boolean{#1}}\ifthenelse{\boolean{#1}}{\begin{trivlist}\item}{\setbox0\vbox\bgroup}{}
%}
%{%
%\ifthenelse{\givenatend}{\end{trivlist}}{\egroup}{}
%}
%\else
%\newenvironment{instructorIntro}[1][false]%
%{%
%  \ifthenelse{\boolean{#1}}{\begin{trivlist}\item[\hskip \labelsep\bfseries Instructor Notes:\hspace{2ex}]}
%{\begin{trivlist}\item[\hskip \labelsep\bfseries Instructor Notes:\hspace{2ex}]}
%{}
%}
%% %% line at the bottom} 
%{\end{trivlist}\par\addvspace{.5ex}\nobreak\noindent\hung} 
%\fi
%
%


\let\instructorNotes\relax
\let\endinstructorNotes\relax
%%% instructorNotes environment
\ifhandout
\newenvironment{instructorNotes}[1][false]%
{%
\def\givenatend{\boolean{#1}}\ifthenelse{\boolean{#1}}{\begin{trivlist}\item}{\setbox0\vbox\bgroup}{}
}
{%
\ifthenelse{\givenatend}{\end{trivlist}}{\egroup}{}
}
\else
\newenvironment{instructorNotes}[1][false]%
{%
  \ifthenelse{\boolean{#1}}{\begin{trivlist}\item[\hskip \labelsep\bfseries {\Large Instructor Notes: \\} \hspace{\textwidth} ]}
{\begin{trivlist}\item[\hskip \labelsep\bfseries {\Large Instructor Notes: \\} \hspace{\textwidth} ]}
{}
}
{\end{trivlist}}
\fi


%% Suggested Timing
\newcommand{\timing}[1]{{\bf Suggested Timing: \hspace{2ex}} #1}




\hypersetup{
    colorlinks=true,       % false: boxed links; true: colored links
    linkcolor=blue,          % color of internal links (change box color with linkbordercolor)
    citecolor=green,        % color of links to bibliography
    filecolor=magenta,      % color of file links
    urlcolor=cyan           % color of external links
}

\title{Sieving It All Out}
\author{Bart Snapp and Brad Findell}

\outcome{Learning outcome goes here.}

\begin{document}
\begin{abstract}
We do the sieve of Eratosthenes.
\end{abstract}
\maketitle
\label{A:Sieve}

\begin{problem} 
Try to find all the primes from $1$ to $120$ \textit{without}
doing any division.  Try to circle numbers that are prime and 
cross out numbers that are not prime.  
As a gesture of friendship, here are the numbers from $1$ to $120$.
\[
\begin{array}{r r r r r r r r r r}
  1 &   2 &   3 &   4 &   5 &   6 &   7 &   8 &   9 &  10\\
 11 &  12 &  13 &  14 &  15 &  16 &  17 &  18 &  19 &  20\\
 21 &  22 &  23 &  24 &  25 &  26 &  27 &  28 &  29 &  30\\
 31 &  32 &  33 &  34 &  35 &  36 &  37 &  38 &  39 &  40\\
 41 &  42 &  43 &  44 &  45 &  46 &  47 &  48 &  49 &  50\\
 51 &  52 &  53 &  54 &  55 &  56 &  57 &  58 &  59 &  60\\
 61 &  62 &  63 &  64 &  65 &  66 &  67 &  68 &  69 &  70\\
 71 &  72 &  73 &  74 &  75 &  76 &  77 &  78 &  79 &  80\\
 81 &  82 &  83 &  84 &  85 &  86 &  87 &  88 &  89 &  90\\
 91 &  92 &  93 &  94 &  95 &  96 &  97 &  98 &  99 & 100\\
101 & 102 & 103 & 104 & 105 & 106 & 107 & 108 & 109 & 110\\
111 & 112 & 113 & 114 & 115 & 116 & 117 & 118 & 119 & 120\\
\end{array}
\]
Describe your method.  
\end{problem}

\newpage

\begin{problem}
Now let's be systematic.  Ignore 1 (we'll talk about why later).   
As you identify a prime, first circle it, then cross out its multiples that are not already crossed out.  
Keep track of your work so that you can answer the following questions:  
\begin{enumerate}
\item After circling a new prime, note the first number crossed out with that prime.  Record your results in a table.


{\renewcommand{\arraystretch}{1.4}
\begin{tabular}{c|c}
        prime    & first \# crossed out \\
%                     & crossed out   \\
\hline
         2       &                      \\
                 &                      \\
                 &                      \\
                 &                      \\
                 &                      \\
\end{tabular}
}

\item What was the biggest prime for which you crossed out at least one multiple?
\end{enumerate}

\begin{teachingnote}
When being systematic, the first number crossed out should be the square of the circled prime.  (All earlier multiples should have been crossed out because of smaller primes.)  

A quick check for the carefulness of the process is to look at $119 = 7 \times 17$. 

Side note:  When the sieve is done in six columns, we can observe that any prime greater than 3 must be one more or one less than a multiple of 6.
\end{teachingnote}

\[
\begin{array}{r r r r r r r r r r}
  1 &   2 &   3 &   4 &   5 &   6 &   7 &   8 &   9 &  10\\
 11 &  12 &  13 &  14 &  15 &  16 &  17 &  18 &  19 &  20\\
 21 &  22 &  23 &  24 &  25 &  26 &  27 &  28 &  29 &  30\\
 31 &  32 &  33 &  34 &  35 &  36 &  37 &  38 &  39 &  40\\
 41 &  42 &  43 &  44 &  45 &  46 &  47 &  48 &  49 &  50\\
 51 &  52 &  53 &  54 &  55 &  56 &  57 &  58 &  59 &  60\\
 61 &  62 &  63 &  64 &  65 &  66 &  67 &  68 &  69 &  70\\
 71 &  72 &  73 &  74 &  75 &  76 &  77 &  78 &  79 &  80\\
 81 &  82 &  83 &  84 &  85 &  86 &  87 &  88 &  89 &  90\\
 91 &  92 &  93 &  94 &  95 &  96 &  97 &  98 &  99 & 100\\
101 & 102 & 103 & 104 & 105 & 106 & 107 & 108 & 109 & 110\\
111 & 112 & 113 & 114 & 115 & 116 & 117 & 118 & 119 & 120\\
\end{array}
\]
\end{problem}


%\begin{problem}
%Find all of the prime factors of 1008. How can you be sure you've
%found them all?
%\end{problem}

\end{document}
