%\documentclass[handout]{ximera}
\documentclass{ximera}


\graphicspath{
  {./}
  {graphics/}
  {../graphics/}
}

\usepackage{chngcntr}

\let\question\relax
\let\endquestion\relax




\newtheoremstyle{SlantTheorem}{\topsep}{\fill}%%% space between body and thm
%\newtheoremstyle{SlantTheorem}{\topsep}{\topsep}%%% space between body and thm
 {\slshape}                      %%% Thm body font
 {}                              %%% Indent amount (empty = no indent)
 {\bfseries\sffamily}            %%% Thm head font
 {}                              %%% Punctuation after thm head
 {3ex}                           %%% Space after thm head
 {\thmname{#1}\thmnumber{ #2}\thmnote{ \bfseries(#3)}}%%% Thm head spec
\theoremstyle{SlantTheorem}
\newtheorem{question}{Question}
\counterwithin*{question}{section}



\let\instructorNotes\relax
\let\endinstructorNotes\relax
%%% instructorNotes environment
\ifhandout
\newenvironment{instructorNotes}[1][false]%
{%
\def\givenatend{\boolean{#1}}\ifthenelse{\boolean{#1}}{\begin{trivlist}\item}{\setbox0\vbox\bgroup}{}
}
{%
\ifthenelse{\givenatend}{\end{trivlist}}{\egroup}{}
}
\else
\newenvironment{instructorNotes}[1][false]%
{%
  \ifthenelse{\boolean{#1}}{\begin{trivlist}\item[\hskip \labelsep\bfseries {\Large Instructor Notes: \\} \hspace{\textwidth} ]}
{\begin{trivlist}\item[\hskip \labelsep\bfseries {\Large Instructor Notes: \\} \hspace{\textwidth} ]}
{}
}
{\end{trivlist}}
\fi


%% Suggested Timing
\newcommand{\timing}[1]{{\bf Suggested Timing: \hspace{2ex}} #1}

\title{Hundredths Grids for Rational Numbers}
\author{Bart Snapp and Brad Findell}

\outcome{Learning outcome goes here.}

\begin{document}
\begin{abstract}
Abstract goes here.  
\end{abstract}
\maketitle

\label{A:hundredthsGrids}


When a $10\times 10$ square is taken to be $1$ whole, it can be used as a ``hundredths grid'' 
to represent fractions and decimals between $0$ and $1$. For example, one of the grids below 
is shaded to represent $\frac{21}{100}$.

\begin{problem}
Shade the hundredths grids to show each of the following fractions.  Then use your shading to determine a decimal equivalent for each fraction.  
\fixnote{Turn these parts into enumerate?  Rearrange graphics?  Use TikZ for graphics?}
\begin{center}
\hfill (a) $\frac{3}{20}$ \hfill (b) $\frac{1}{8}$ \hfill (c) $\frac{1}{6}$ \hfill (d) $\frac{7}{12}$ \hfill
\end{center}

\begin{image}
\includegraphics{../graphics/hundredthsGridSample}\quad
\includegraphics{../graphics/hundredthsGrid0.pdf}
\end{image}
\begin{image}
\includegraphics{../graphics/hundredthsGrid0.pdf}\quad
\includegraphics{../graphics/hundredthsGrid0.pdf}
\end{image}
\begin{image}
\includegraphics{../graphics/hundredthsGrid0.pdf}\quad
\includegraphics{../graphics/hundredthsGrid0.pdf}
\end{image}

\end{problem}

\end{document}
