%\documentclass[handout]{ximera}
\documentclass[nooutcomes]{ximera}

\usepackage{gensymb}
\usepackage{tabularx}
\usepackage{mdframed}
\usepackage{pdfpages}
%\usepackage{chngcntr}

\let\problem\relax
\let\endproblem\relax

\newcommand{\property}[2]{#1#2}




\newtheoremstyle{SlantTheorem}{\topsep}{\fill}%%% space between body and thm
 {\slshape}                      %%% Thm body font
 {}                              %%% Indent amount (empty = no indent)
 {\bfseries\sffamily}            %%% Thm head font
 {}                              %%% Punctuation after thm head
 {3ex}                           %%% Space after thm head
 {\thmname{#1}\thmnumber{ #2}\thmnote{ \bfseries(#3)}} %%% Thm head spec
\theoremstyle{SlantTheorem}
\newtheorem{problem}{Problem}[]

%\counterwithin*{problem}{section}



%%%%%%%%%%%%%%%%%%%%%%%%%%%%Jenny's code%%%%%%%%%%%%%%%%%%%%

%%% Solution environment
%\newenvironment{solution}{
%\ifhandout\setbox0\vbox\bgroup\else
%\begin{trivlist}\item[\hskip \labelsep\small\itshape\bfseries Solution\hspace{2ex}]
%\par\noindent\upshape\small
%\fi}
%{\ifhandout\egroup\else
%\end{trivlist}
%\fi}
%
%
%%% instructorIntro environment
%\ifhandout
%\newenvironment{instructorIntro}[1][false]%
%{%
%\def\givenatend{\boolean{#1}}\ifthenelse{\boolean{#1}}{\begin{trivlist}\item}{\setbox0\vbox\bgroup}{}
%}
%{%
%\ifthenelse{\givenatend}{\end{trivlist}}{\egroup}{}
%}
%\else
%\newenvironment{instructorIntro}[1][false]%
%{%
%  \ifthenelse{\boolean{#1}}{\begin{trivlist}\item[\hskip \labelsep\bfseries Instructor Notes:\hspace{2ex}]}
%{\begin{trivlist}\item[\hskip \labelsep\bfseries Instructor Notes:\hspace{2ex}]}
%{}
%}
%% %% line at the bottom} 
%{\end{trivlist}\par\addvspace{.5ex}\nobreak\noindent\hung} 
%\fi
%
%


\let\instructorNotes\relax
\let\endinstructorNotes\relax
%%% instructorNotes environment
\ifhandout
\newenvironment{instructorNotes}[1][false]%
{%
\def\givenatend{\boolean{#1}}\ifthenelse{\boolean{#1}}{\begin{trivlist}\item}{\setbox0\vbox\bgroup}{}
}
{%
\ifthenelse{\givenatend}{\end{trivlist}}{\egroup}{}
}
\else
\newenvironment{instructorNotes}[1][false]%
{%
  \ifthenelse{\boolean{#1}}{\begin{trivlist}\item[\hskip \labelsep\bfseries {\Large Instructor Notes: \\} \hspace{\textwidth} ]}
{\begin{trivlist}\item[\hskip \labelsep\bfseries {\Large Instructor Notes: \\} \hspace{\textwidth} ]}
{}
}
{\end{trivlist}}
\fi


%% Suggested Timing
\newcommand{\timing}[1]{{\bf Suggested Timing: \hspace{2ex}} #1}




\hypersetup{
    colorlinks=true,       % false: boxed links; true: colored links
    linkcolor=blue,          % color of internal links (change box color with linkbordercolor)
    citecolor=green,        % color of links to bibliography
    filecolor=magenta,      % color of file links
    urlcolor=cyan           % color of external links
}

\title{Geometric Series}
\author{Bart Snapp and Brad Findell}

\outcome{Learning outcome goes here.}

\begin{document}
\begin{abstract}
We explore \emph{geometric series}, which are sums of consecutive
terms from an geometric sequence.
\end{abstract}
\maketitle

\label{A:geometicSeries}

Ms. Radigan's math class has been trying to compute the following sums:  
$$1+2+4+8+\dots+2^{19}$$
$$\frac{2}{3}+\frac{2}{9}+\frac{2}{27}+\dots+\frac{2}{3^{13}}$$

\begin{problem}
Kelsey used tables and looked for pattern in the \emph{sequence of partial sums}:  $1, 1+2, 1+2+4, \dots$.  Help her finish her idea for both sequences.    
\end{problem}

\begin{problem}
For the sum beginning with $\frac{2}{3}$, Erin started by drawing a large square (which she imagined as having area 1), and she shaded in $\frac{2}{3}$ of it.  Then she shaded in $\frac{2}{9}$ more, and so on.  Help her finish her idea.  
\end{problem}

\begin{problem}
Ryan wrote out all of the terms in the first sum, represented as powers of 2, beginning with $1+2+2^2+2^3$.  
Then he realized that because the terms formed a geometric sequence, he could multiply the series by the common ratio of 2, and the resulting series would be almost identical to the first, differing only at the beginning and the end.  By subtracting the first series from the second, all of the middle terms would cancel.  Help him finish his idea.  
\end{problem}

\begin{problem}
Ali said, ``Here is a thought experiment.  I take a sheet of paper, rip it perfectly into thirds, place one piece to start a pile that I will call A, another piece to start a pile I will call B, and I keep the third piece in my hands.  I then rip that piece into thirds, place one piece on pile A, one piece on pile B, and keep the third.  Notice that each of pile A and pile B have $\frac{1}{3}+\frac{1}{9}$ of a sheet of paper, and I still have $\frac{1}{9}$ of a sheet in my hands.  I continue this process until I place $\frac{1}{3^{13}}$ of a sheet on each pile and still have $\frac{1}{3^{13}}$ of a sheet in my hands.  

Help Ali finish her idea.  
\end{problem}


\begin{problem}
Sum the expression:  
\[
\frac{2}{3}+\frac{4}{9}+\frac{8}{27}+\dots+\frac{2^n}{3^n}
\]
What happens to this sum as $n$ gets really large?  
\end{problem}


\begin{problem}
Consider the expression: 
$$\frac{7}{10}+\frac{7}{100}+\frac{7}{1000}+\dots+\frac{7}{10^n}$$
\begin{enumerate}
\item Find the sum of the expression. 
\item What happens to this sum as $n$ gets really large?  
\item How does this help you explain why a particular repeating decimal is a particular rational number?  Be sure to indicate what repeating decimal and what rational number you are talking about.  
\end{enumerate}
\end{problem}


\begin{problem}
Suppose you have an geometric sequence beginning with $a$, with a constant ratio of $r$ and with $n$ terms.  
\begin{enumerate}
\item What is the $n^{th}$ term of the sequence?  
\item Use dots to write the series consisting of the first $n$ terms of this sequence.
\item Find the sum of this series.  
\end{enumerate}
\end{problem}

\end{document}
