\newpage
\section{Comparative Arithmetic}\label{A:CA}

\begin{teachingnote}
The point of the activity is that polynomial arithmetic is analogous to familiar base-ten algorithms, except there is no ``regrouping'' (i.e., carrying or borrowing).  Ultimately, we want students to see polynomials as numbers in base $x$ and to see base-ten numbers as polynomials in 10.
\end{teachingnote}

\begin{prob} Compute:
\[
\begin{array}{@{}r@{}}
131\\
+122\\ \hline
\end{array}
\qquad\text{and}
\qquad
\begin{array}{@{}r@{}}
x^2+3x+1\\
+x^2+2x+2\\ \hline
\end{array}
\]

\vspace{0.5in}
Compare, contrast, and describe your experiences.
\end{prob}

\begin{prob} Compute:
\[
\begin{array}{@{}r@{}}
139\\
+122\\ \hline
\end{array}
\qquad\text{and}
\qquad
\begin{array}{@{}r@{}}
x^2+3x+9\\
+x^2+2x+2\\ \hline
\end{array}
\]

\vspace{0.5in}
Compare, contrast, and describe your experiences. In particular,
discuss how this is different from the first problem.
\end{prob}


\begin{prob} Compute:
\[
\begin{array}{@{}r@{}}
121\\
\times 32\\ \hline
\end{array}
\qquad\text{and}
\qquad
\begin{array}{@{}r@{}}
x^2+2x+1\\
\times~~~3x+2\\ \hline
\end{array}
\]
\vspace{0.8in}

Compare, contrast, and describe your experiences.
\end{prob}

\begin{prob}
Expand:
\[
(x^2 + 2x + 1)(3x+2)
\]

\vspace{0.2in}
Compare, contrast, and describe your experiences. In particular, discuss how this problem relates to the one above.
\end{prob}

\begin{prob} Compute:
\[
\begin{array}{@{}r@{}}
214\\
\times 53\\ \hline
\end{array}
\qquad\text{and}
\qquad
\begin{array}{@{}r@{}}
2x^2+x+4\\
\times~~~5x+3\\ \hline
\end{array}
\]
Compare, contrast, and describe your experiences.
\end{prob}

\begin{prob}
Use long division to compute $2785\div 23$ and $(2x^3+7x^2+8x+5)\div (2x+3)$.  Compare, contrast, and describe your experiences.  
\end{prob}

\begin{prob}
Use long division to compute $6529\div 34$ and $(6x^3+5x^2+2x+9)\div (3x+4)$.  Compare, contrast, and describe your experiences.  
\end{prob}

