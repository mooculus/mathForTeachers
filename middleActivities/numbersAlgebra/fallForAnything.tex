%\documentclass[handout]{ximera}
\documentclass{ximera}

\usepackage{gensymb}
\usepackage{tabularx}
\usepackage{mdframed}
\usepackage{pdfpages}
%\usepackage{chngcntr}

\let\problem\relax
\let\endproblem\relax

\newcommand{\property}[2]{#1#2}




\newtheoremstyle{SlantTheorem}{\topsep}{\fill}%%% space between body and thm
 {\slshape}                      %%% Thm body font
 {}                              %%% Indent amount (empty = no indent)
 {\bfseries\sffamily}            %%% Thm head font
 {}                              %%% Punctuation after thm head
 {3ex}                           %%% Space after thm head
 {\thmname{#1}\thmnumber{ #2}\thmnote{ \bfseries(#3)}} %%% Thm head spec
\theoremstyle{SlantTheorem}
\newtheorem{problem}{Problem}[]

%\counterwithin*{problem}{section}



%%%%%%%%%%%%%%%%%%%%%%%%%%%%Jenny's code%%%%%%%%%%%%%%%%%%%%

%%% Solution environment
%\newenvironment{solution}{
%\ifhandout\setbox0\vbox\bgroup\else
%\begin{trivlist}\item[\hskip \labelsep\small\itshape\bfseries Solution\hspace{2ex}]
%\par\noindent\upshape\small
%\fi}
%{\ifhandout\egroup\else
%\end{trivlist}
%\fi}
%
%
%%% instructorIntro environment
%\ifhandout
%\newenvironment{instructorIntro}[1][false]%
%{%
%\def\givenatend{\boolean{#1}}\ifthenelse{\boolean{#1}}{\begin{trivlist}\item}{\setbox0\vbox\bgroup}{}
%}
%{%
%\ifthenelse{\givenatend}{\end{trivlist}}{\egroup}{}
%}
%\else
%\newenvironment{instructorIntro}[1][false]%
%{%
%  \ifthenelse{\boolean{#1}}{\begin{trivlist}\item[\hskip \labelsep\bfseries Instructor Notes:\hspace{2ex}]}
%{\begin{trivlist}\item[\hskip \labelsep\bfseries Instructor Notes:\hspace{2ex}]}
%{}
%}
%% %% line at the bottom} 
%{\end{trivlist}\par\addvspace{.5ex}\nobreak\noindent\hung} 
%\fi
%
%


\let\instructorNotes\relax
\let\endinstructorNotes\relax
%%% instructorNotes environment
\ifhandout
\newenvironment{instructorNotes}[1][false]%
{%
\def\givenatend{\boolean{#1}}\ifthenelse{\boolean{#1}}{\begin{trivlist}\item}{\setbox0\vbox\bgroup}{}
}
{%
\ifthenelse{\givenatend}{\end{trivlist}}{\egroup}{}
}
\else
\newenvironment{instructorNotes}[1][false]%
{%
  \ifthenelse{\boolean{#1}}{\begin{trivlist}\item[\hskip \labelsep\bfseries {\Large Instructor Notes: \\} \hspace{\textwidth} ]}
{\begin{trivlist}\item[\hskip \labelsep\bfseries {\Large Instructor Notes: \\} \hspace{\textwidth} ]}
{}
}
{\end{trivlist}}
\fi


%% Suggested Timing
\newcommand{\timing}[1]{{\bf Suggested Timing: \hspace{2ex}} #1}




\hypersetup{
    colorlinks=true,       % false: boxed links; true: colored links
    linkcolor=blue,          % color of internal links (change box color with linkbordercolor)
    citecolor=green,        % color of links to bibliography
    filecolor=magenta,      % color of file links
    urlcolor=cyan           % color of external links
}

\title{They'll Fall for Anything!}
\author{Bart Snapp and Brad Findell}

\outcome{Learning outcome goes here.}

\begin{document}
\begin{abstract}
Abstract goes here.  
\end{abstract}
\maketitle

\label{A:fallForAnything}

What is incorrect about the following reasoning? Be specific!

\begin{problem}
Herman says that if you pick a United States citizen at random, the
probability of selecting a citizen from Indiana is because Indiana is
one of 50 equally likely states to be selected.
\end{problem}


\begin{problem}
Jerry has set up a game in which one wins a prize if he/she selects an
orange chip from a bag.  There are two bags to choose from.  One has 2
orange and 4 green chips.  The other bag has 7 orange and 7 green
chips.  Jerry argues that you have a better chance of winning by
drawing from the second bag because there are more orange chips in it.
\end{problem}

\begin{problem}
Gil the Gambler says that it is just as likely to flip 5 coins and get
exactly 3 heads as it is to flip 10 coins and get exactly 6 heads
because
\[
\frac{3}{5} = \frac{6}{10}
\]
\end{problem}

\begin{problem}
We draw 4 cards without replacement from a deck of 52.  Know-it-all
Ned says the probability of obtaining all four 7's is
$\frac{4}{\binom{52}{4}}$ because there are ways to select the
$\binom{52}{4}$ 4 cards and there are four 7's in the deck.
\end{problem} 

\begin{problem}
At a festival, Stealin' Stan gives Crazy Chris the choice of one of
three prizes---each of which was hidden behind a door.  One of the
doors has a fabulous prize behind it while the other two doors each
have a ``zonk'' (a free used tube of toothpaste, etc.).  Crazy Chris
chooses Door \#1.  Before opening that door, Stealin' Stan shows Chris
that hidden behind Door \#3 is a zonk and gives Chris the option to
keep Door \#1 or switch to Door \#2.  Chris says, ``Big deal.  It doesn't
help my chances of winning to switch or not switch.''
\end{problem}

\end{document}
