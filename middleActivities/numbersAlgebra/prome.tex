%\documentclass[handout]{ximera}
\documentclass[nooutcomes]{ximera}

\usepackage{gensymb}
\usepackage{tabularx}
\usepackage{mdframed}
\usepackage{pdfpages}
%\usepackage{chngcntr}

\let\problem\relax
\let\endproblem\relax

\newcommand{\property}[2]{#1#2}




\newtheoremstyle{SlantTheorem}{\topsep}{\fill}%%% space between body and thm
 {\slshape}                      %%% Thm body font
 {}                              %%% Indent amount (empty = no indent)
 {\bfseries\sffamily}            %%% Thm head font
 {}                              %%% Punctuation after thm head
 {3ex}                           %%% Space after thm head
 {\thmname{#1}\thmnumber{ #2}\thmnote{ \bfseries(#3)}} %%% Thm head spec
\theoremstyle{SlantTheorem}
\newtheorem{problem}{Problem}[]

%\counterwithin*{problem}{section}



%%%%%%%%%%%%%%%%%%%%%%%%%%%%Jenny's code%%%%%%%%%%%%%%%%%%%%

%%% Solution environment
%\newenvironment{solution}{
%\ifhandout\setbox0\vbox\bgroup\else
%\begin{trivlist}\item[\hskip \labelsep\small\itshape\bfseries Solution\hspace{2ex}]
%\par\noindent\upshape\small
%\fi}
%{\ifhandout\egroup\else
%\end{trivlist}
%\fi}
%
%
%%% instructorIntro environment
%\ifhandout
%\newenvironment{instructorIntro}[1][false]%
%{%
%\def\givenatend{\boolean{#1}}\ifthenelse{\boolean{#1}}{\begin{trivlist}\item}{\setbox0\vbox\bgroup}{}
%}
%{%
%\ifthenelse{\givenatend}{\end{trivlist}}{\egroup}{}
%}
%\else
%\newenvironment{instructorIntro}[1][false]%
%{%
%  \ifthenelse{\boolean{#1}}{\begin{trivlist}\item[\hskip \labelsep\bfseries Instructor Notes:\hspace{2ex}]}
%{\begin{trivlist}\item[\hskip \labelsep\bfseries Instructor Notes:\hspace{2ex}]}
%{}
%}
%% %% line at the bottom} 
%{\end{trivlist}\par\addvspace{.5ex}\nobreak\noindent\hung} 
%\fi
%
%


\let\instructorNotes\relax
\let\endinstructorNotes\relax
%%% instructorNotes environment
\ifhandout
\newenvironment{instructorNotes}[1][false]%
{%
\def\givenatend{\boolean{#1}}\ifthenelse{\boolean{#1}}{\begin{trivlist}\item}{\setbox0\vbox\bgroup}{}
}
{%
\ifthenelse{\givenatend}{\end{trivlist}}{\egroup}{}
}
\else
\newenvironment{instructorNotes}[1][false]%
{%
  \ifthenelse{\boolean{#1}}{\begin{trivlist}\item[\hskip \labelsep\bfseries {\Large Instructor Notes: \\} \hspace{\textwidth} ]}
{\begin{trivlist}\item[\hskip \labelsep\bfseries {\Large Instructor Notes: \\} \hspace{\textwidth} ]}
{}
}
{\end{trivlist}}
\fi


%% Suggested Timing
\newcommand{\timing}[1]{{\bf Suggested Timing: \hspace{2ex}} #1}




\hypersetup{
    colorlinks=true,       % false: boxed links; true: colored links
    linkcolor=blue,          % color of internal links (change box color with linkbordercolor)
    citecolor=green,        % color of links to bibliography
    filecolor=magenta,      % color of file links
    urlcolor=cyan           % color of external links
}

\title{Prome Factorization}
\author{Bart Snapp and Brad Findell}

\outcome{Learning outcome goes here.}

\begin{document}
\begin{abstract}
  We imagine a system of numbers without unique factorization.
\end{abstract}
\maketitle

\label{A:Prome}

\begin{teachingnote}
In the course, we first assume Euclid's Lemma and use it to prove the Fundamental Theorem of Arithmetic (FTA).  In this activity, both Euclid's Lemma and the FTA fail.

It might help to begin the class with the following questions:  
\begin{itemize}
\item If $7|(ab)$ (where $a$ and $b$ integers), does it follow that $7$ must divide either $a$ or $b$? 
\item If $14|(ab)$ (where $a$ and $b$ integers), does it follow that $14$ must divide either $a$ or $b$? 
\end{itemize}

Through discussion, students should decide that the answers are ``yes'' and ``no,'' respectively, and the reason is that 7 is prime but 14 is not.  Some students should realize that, in the second case, the factors of 2 and 7 (of 14) might be ``split'' between $a$ and $b$.  This realization can be stated as Euclid's Lemma:  

\begin{center}
Suppose $a$ and $b$ are integers and $p$ is prime.  If $p|ab$, then $p|a$ or $p|b$.  
\end{center}

Students are not be responsible for its name.  At this point, we accept it without proof.  

The purpose of questions 1-4 is to see that this ``new'' number system is very much like the integers:  You can always add, subtract, and multiply, but you cannot necessarily divide.  Students can use their reasoning about integers to explain these facts about the system $3\Z$.  

The purpose of questions 5-7 is to notice that both Euclid's Lemma and Unique Factorization fail in this number system.  Some examples:  

$$36 = 3\times 12 = 6\times 6$$
$$72 = 3\times 24=6\times 12$$
\end{teachingnote}


Let's consider a crazy set of numbers---all multiples of $3$. Let's
use the symbol $3\Z$ to denote the set consisting of all multiples of
$3$. As a gesture of friendship, I have written down the first $100$
nonnegative integers in $3\Z$:

\[
\begin{array}{cccccccccc}
0   & 3   & 6   & 9   & 12  & 15  & 18  & 21  & 24  & 27  \\
\\
30  & 33  & 36  & 39  & 42  & 45  & 48  & 51  & 54  & 57  \\
\\
60  & 63  & 66  & 69  & 72  & 75  & 78  & 81  & 84  & 87  \\
\\
90  & 93  & 96  & 99  & 102 & 105 & 108 & 111 & 114 & 117 \\
\\
120 & 123 & 126 & 129 & 132 & 135 & 138 & 141 & 144 & 147 \\
\\
150 & 153 & 156 & 159 & 162 & 165 & 168 & 171 & 174 & 177 \\
\\
180 & 183 & 186 & 189 & 192 & 195 & 198 & 201 & 204 & 207 \\
\\
210 & 213 & 216 & 219 & 222 & 225 & 228 & 231 & 234 & 237 \\
\\
240 & 243 & 246 & 249 & 252 & 255 & 258 & 261 & 264 & 267 \\
\\
270 & 273 & 276 & 279 & 282 & 285 & 288 & 291 & 294 & 297
\end{array}
\]



\begin{problem}
Given any two integers in $3\Z$, will their sum be in $3\Z$? Explain
your reasoning.
\end{problem}

\begin{problem}
Given any two integers in $3\Z$, will their difference be in $3\Z$?
Explain your reasoning.
\end{problem}
\begin{teachingnote}
Yes, but students might need to be reminded that $3\Z$ includes negative integers.  
\end{teachingnote}

\begin{problem}
Given any two integers in $3\Z$, will their product be in $3\Z$?
Explain your reasoning.
\end{problem}

\begin{problem}
Given any two integers in $3\Z$, will their quotient be in $3\Z$?
Explain your reasoning.
\end{problem}

\begin{definition}
Call a positive integer \textbf{prome} in $3\Z$ if it cannot be
expressed as the product of two integers \textit{both} in $3\Z$.
\end{definition}

As an example, I tell you that $6$ is prome number in $3\Z$. You may
object because $6 = 2\cdot 3$, but remember---$2$ is not in $3\Z$!


\begin{problem}
List some of the prome numbers less than $297$.  Hint:  What numbers in $3\Z$ \emph{can} be expressed as a product of two integers \emph{both} in $3\Z$?  
\end{problem}

\begin{problem}
Can you give some sort of algebraic characterization of prome numbers
in $3\Z$? 
\end{problem}

\begin{problem}
Can you find numbers that factor completely into prome numbers in
\textit{two} different ways? How many can you find?
\end{problem}

\end{document}
