%\documentclass[handout]{ximera}
\documentclass[nooutcomes]{ximera}

\usepackage{gensymb}
\usepackage{tabularx}
\usepackage{mdframed}
\usepackage{pdfpages}
%\usepackage{chngcntr}

\let\problem\relax
\let\endproblem\relax

\newcommand{\property}[2]{#1#2}




\newtheoremstyle{SlantTheorem}{\topsep}{\fill}%%% space between body and thm
 {\slshape}                      %%% Thm body font
 {}                              %%% Indent amount (empty = no indent)
 {\bfseries\sffamily}            %%% Thm head font
 {}                              %%% Punctuation after thm head
 {3ex}                           %%% Space after thm head
 {\thmname{#1}\thmnumber{ #2}\thmnote{ \bfseries(#3)}} %%% Thm head spec
\theoremstyle{SlantTheorem}
\newtheorem{problem}{Problem}[]

%\counterwithin*{problem}{section}



%%%%%%%%%%%%%%%%%%%%%%%%%%%%Jenny's code%%%%%%%%%%%%%%%%%%%%

%%% Solution environment
%\newenvironment{solution}{
%\ifhandout\setbox0\vbox\bgroup\else
%\begin{trivlist}\item[\hskip \labelsep\small\itshape\bfseries Solution\hspace{2ex}]
%\par\noindent\upshape\small
%\fi}
%{\ifhandout\egroup\else
%\end{trivlist}
%\fi}
%
%
%%% instructorIntro environment
%\ifhandout
%\newenvironment{instructorIntro}[1][false]%
%{%
%\def\givenatend{\boolean{#1}}\ifthenelse{\boolean{#1}}{\begin{trivlist}\item}{\setbox0\vbox\bgroup}{}
%}
%{%
%\ifthenelse{\givenatend}{\end{trivlist}}{\egroup}{}
%}
%\else
%\newenvironment{instructorIntro}[1][false]%
%{%
%  \ifthenelse{\boolean{#1}}{\begin{trivlist}\item[\hskip \labelsep\bfseries Instructor Notes:\hspace{2ex}]}
%{\begin{trivlist}\item[\hskip \labelsep\bfseries Instructor Notes:\hspace{2ex}]}
%{}
%}
%% %% line at the bottom} 
%{\end{trivlist}\par\addvspace{.5ex}\nobreak\noindent\hung} 
%\fi
%
%


\let\instructorNotes\relax
\let\endinstructorNotes\relax
%%% instructorNotes environment
\ifhandout
\newenvironment{instructorNotes}[1][false]%
{%
\def\givenatend{\boolean{#1}}\ifthenelse{\boolean{#1}}{\begin{trivlist}\item}{\setbox0\vbox\bgroup}{}
}
{%
\ifthenelse{\givenatend}{\end{trivlist}}{\egroup}{}
}
\else
\newenvironment{instructorNotes}[1][false]%
{%
  \ifthenelse{\boolean{#1}}{\begin{trivlist}\item[\hskip \labelsep\bfseries {\Large Instructor Notes: \\} \hspace{\textwidth} ]}
{\begin{trivlist}\item[\hskip \labelsep\bfseries {\Large Instructor Notes: \\} \hspace{\textwidth} ]}
{}
}
{\end{trivlist}}
\fi


%% Suggested Timing
\newcommand{\timing}[1]{{\bf Suggested Timing: \hspace{2ex}} #1}




\hypersetup{
    colorlinks=true,       % false: boxed links; true: colored links
    linkcolor=blue,          % color of internal links (change box color with linkbordercolor)
    citecolor=green,        % color of links to bibliography
    filecolor=magenta,      % color of file links
    urlcolor=cyan           % color of external links
}

\title{I Walk the Line}
\author{Bart Snapp and Brad Findell}

\outcome{Learning outcome goes here.}

\begin{document}
\begin{abstract}
  We solve linear equations, with the context being our guide.
\end{abstract}
\maketitle

\label{A:walk}
Solve the problems below initially without using letters and without algebraic procedures.  Rely on numerical reasoning only, and then generalize your numerical approaches.  

\begin{problem}
Slimy Sam is on the lam from the law.  Being not-too-smart, he drives
the clunker of a car he stole east on I-70 across Ohio.  Because the
car can only go a maximum of 52 miles per hour, he floors it all the
way from where he stole the car (just now at the Rest Area 5 miles
west of the Indiana line) and goes as far as he can before running out
of gas 3.78 hours from now.

\begin{enumerate}
\item At what mile marker will he be 3 hours after stealing the car?
\item At what mile marker will he be when he runs out of gas and is
  arrested?
\item At what mile marker will he be $x$ hours after
  stealing the car?
\item At what time will he be at mile marker 99 (east of Indiana)?
\item At what time will he be at mile marker 71.84?
\item At what time will he be at mile marker $y$?
\item Do parts (c) and (f) supposing that the car
  goes $m$ miles per hour and Sam started $b$ miles east of the
  Ohio-Indiana border.
\item What ``form'' of an equation for a line does this problem
  motivate?
\end{enumerate}
\end{problem}

\begin{problem}
Free-Lance Freddy works for varying hourly rates, depending on the
job.  He also carries some spare cash for lunch.  To make his
customers sweat, Freddy keeps a meter on his belt telling how much
money they currently owe (with his lunch money added in).
\begin{enumerate}
\item On Monday, 3 hours into his work as a gourmet burger flipper,
  Freddy's meter reads $\$42$. 7 hours into his work, his meter reads
  $\$86$.  If he works for 12 hours, how much money will he have?  When
  will he have $\$196$? Solve this problem \textbf{without} finding his
  lunch money.
\item On Tuesday, Freddy is CEO of the of \textit{We Say So} Company.
  After 2.53 hours of work, his meter reads $\$863.15$ and after 5.71
  hours of work, his meter reads $\$$1349.78.  If he works for 10.34
  hours, how much money will he have?  How much time will he be in
  office to have $\$$1759.21?
\item On Wednesday, Freddy is starting goalie for the \textit{Columbus
  Blue Jackets}. After $x_1$ hours of work, his meter reads $y_1$
  dollars and after $x_2$ hours of work, his meter reads $y_2$
  dollars. Without finding his amount of lunch money, if he works for
  $x$ hours, how much money will he have?  How much time will he be in
  front of the net to have $y$ dollars?
\item What ``form'' of an equation for a line does this problem
  motivate?
\end{enumerate}
\end{problem}

\begin{problem} 
Counterfeit Cathy buys two kinds of fake cereal: Square Cheerios for
$\$$4 per pound and Sugarless Sugar Pops for $\$$5 per pound.

\begin{enumerate}
\item If Cathy's goal for today is to buy $\$$1000 of cereal, how
  much of each kind could she purchase? Give five possible answers.
\item Plot your answers.  What does the slope represent in this
  situation? What do the points where your curve intercepts the axes
  represent?
\item If she buys Square Cheerios for $a$ dollars per pound and
  Sugarless Sugar Pops for $b$ dollars per pound and she wants to buy
  $c$ dollars of cereal, write an equation that relates the amount of
  Sugar Pops Cathy buys to the amount of Cheerios she buys.  What
  ``form'' of the equation of a line does this problem motivate?
\item Write a function in the form 
\[
\text{pounds of Sugar Pops} = f(\text{pounds of Cheerios}).
\]
\end{enumerate}
\end{problem}


\begin{problem}
Given points $p= (3,7)$ and $q = (4,9)$, find the formula for the line
that connects these points.
\end{problem}


\begin{problem}
In each of the situations above, write an equation relating the two
variables (hours and position, hours and current financial status,
pounds of Square Cheerios and pounds of Sugarless Sugar Pops) and
answer the following questions:
\begin{enumerate}
\item How did (or could) the equations help you solve the problems
  above? What about a table or a graph?
 \item Organize the information in each problem into a table and then
 into a graph.  What patterns do you see, if any?
  \item  What do the different features of your graph represent for each situation?
 \end{enumerate}
\end{problem}

\end{document}
