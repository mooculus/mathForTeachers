%\documentclass[handout]{ximera}
\documentclass[nooutcomes]{ximera}

\usepackage{gensymb}
\usepackage{tabularx}
\usepackage{mdframed}
\usepackage{pdfpages}
%\usepackage{chngcntr}

\let\problem\relax
\let\endproblem\relax

\newcommand{\property}[2]{#1#2}




\newtheoremstyle{SlantTheorem}{\topsep}{\fill}%%% space between body and thm
 {\slshape}                      %%% Thm body font
 {}                              %%% Indent amount (empty = no indent)
 {\bfseries\sffamily}            %%% Thm head font
 {}                              %%% Punctuation after thm head
 {3ex}                           %%% Space after thm head
 {\thmname{#1}\thmnumber{ #2}\thmnote{ \bfseries(#3)}} %%% Thm head spec
\theoremstyle{SlantTheorem}
\newtheorem{problem}{Problem}[]

%\counterwithin*{problem}{section}



%%%%%%%%%%%%%%%%%%%%%%%%%%%%Jenny's code%%%%%%%%%%%%%%%%%%%%

%%% Solution environment
%\newenvironment{solution}{
%\ifhandout\setbox0\vbox\bgroup\else
%\begin{trivlist}\item[\hskip \labelsep\small\itshape\bfseries Solution\hspace{2ex}]
%\par\noindent\upshape\small
%\fi}
%{\ifhandout\egroup\else
%\end{trivlist}
%\fi}
%
%
%%% instructorIntro environment
%\ifhandout
%\newenvironment{instructorIntro}[1][false]%
%{%
%\def\givenatend{\boolean{#1}}\ifthenelse{\boolean{#1}}{\begin{trivlist}\item}{\setbox0\vbox\bgroup}{}
%}
%{%
%\ifthenelse{\givenatend}{\end{trivlist}}{\egroup}{}
%}
%\else
%\newenvironment{instructorIntro}[1][false]%
%{%
%  \ifthenelse{\boolean{#1}}{\begin{trivlist}\item[\hskip \labelsep\bfseries Instructor Notes:\hspace{2ex}]}
%{\begin{trivlist}\item[\hskip \labelsep\bfseries Instructor Notes:\hspace{2ex}]}
%{}
%}
%% %% line at the bottom} 
%{\end{trivlist}\par\addvspace{.5ex}\nobreak\noindent\hung} 
%\fi
%
%


\let\instructorNotes\relax
\let\endinstructorNotes\relax
%%% instructorNotes environment
\ifhandout
\newenvironment{instructorNotes}[1][false]%
{%
\def\givenatend{\boolean{#1}}\ifthenelse{\boolean{#1}}{\begin{trivlist}\item}{\setbox0\vbox\bgroup}{}
}
{%
\ifthenelse{\givenatend}{\end{trivlist}}{\egroup}{}
}
\else
\newenvironment{instructorNotes}[1][false]%
{%
  \ifthenelse{\boolean{#1}}{\begin{trivlist}\item[\hskip \labelsep\bfseries {\Large Instructor Notes: \\} \hspace{\textwidth} ]}
{\begin{trivlist}\item[\hskip \labelsep\bfseries {\Large Instructor Notes: \\} \hspace{\textwidth} ]}
{}
}
{\end{trivlist}}
\fi


%% Suggested Timing
\newcommand{\timing}[1]{{\bf Suggested Timing: \hspace{2ex}} #1}




\hypersetup{
    colorlinks=true,       % false: boxed links; true: colored links
    linkcolor=blue,          % color of internal links (change box color with linkbordercolor)
    citecolor=green,        % color of links to bibliography
    filecolor=magenta,      % color of file links
    urlcolor=cyan           % color of external links
}

\title{Picture Models for Fraction Operations}
\author{Bart Snapp and Brad Findell}

\outcome{Learning outcome goes here.}

\begin{document}
\begin{abstract}
  We develop picture models for operations between fractions.
\end{abstract}
\maketitle

\label{A:FO}
\begin{problem} 
Draw pictures that model:
\[
\frac{1}{5} + \frac{2}{5} = \frac{3}{5}
\]
Explain how your pictures show this. Write a story problem whose
solution is given by the expression above.
\end{problem}

\begin{problem} 
Draw pictures that model:
\[
\frac{2}{3} + \frac{1}{4} = \frac{11}{12}
\]
Explain how your pictures model this equation. Be sure to carefully
explain how common denominators are represented in your
pictures. Write a story problem whose solution is given by the
expression above.
\end{problem}

\begin{problem} 
Given $0<a\le b$ and $0<c\le d$, explain how to draw pictures
that model the sum:
\[
\frac{a}{b} + \frac{c}{d}
\]
Use pictures to find this sum and carefully explain how common
denominators are represented in your pictures.
\end{problem}

% The following problems have been replaced by a new activity about fraction multiplication
%
%\begin{problem} 
%Given positive integers $a$ and $b$, explain how to draw pictures that
%model the product $a\cdot b$---give an example of your process.
%\end{problem}
%
%\begin{problem} 
%Draw pictures that model:
%\[
%\frac{4}{5} \cdot \frac{2}{3} = \frac{8}{15}
%\]
%Explain how your pictures model this equation. Write a story problem
%whose solution is given by the expression above. Does your story work with 
%\[
%\frac{7}{5} \cdot \frac{2}{3} = \frac{14}{15}?
%\]
%\end{problem}
%
%\begin{problem} 
%Given $0<a\le b$ and $0<c\le d$, explain how to draw pictures
%that model the product:
%\[
%\frac{a}{b} \cdot \frac{c}{d}
%\]
%Use pictures to find this product and explain how this product is shown
%in your pictures---give an example of your process.
%\end{problem}

\end{document}
