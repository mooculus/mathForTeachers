%\documentclass[handout]{ximera}
\documentclass[nooutcomes]{ximera}


\graphicspath{
  {./}
  {graphics/}
  {../graphics/}
}

\usepackage{chngcntr}

\let\question\relax
\let\endquestion\relax




\newtheoremstyle{SlantTheorem}{\topsep}{\fill}%%% space between body and thm
%\newtheoremstyle{SlantTheorem}{\topsep}{\topsep}%%% space between body and thm
 {\slshape}                      %%% Thm body font
 {}                              %%% Indent amount (empty = no indent)
 {\bfseries\sffamily}            %%% Thm head font
 {}                              %%% Punctuation after thm head
 {3ex}                           %%% Space after thm head
 {\thmname{#1}\thmnumber{ #2}\thmnote{ \bfseries(#3)}}%%% Thm head spec
\theoremstyle{SlantTheorem}
\newtheorem{question}{Question}
\counterwithin*{question}{section}



\let\instructorNotes\relax
\let\endinstructorNotes\relax
%%% instructorNotes environment
\ifhandout
\newenvironment{instructorNotes}[1][false]%
{%
\def\givenatend{\boolean{#1}}\ifthenelse{\boolean{#1}}{\begin{trivlist}\item}{\setbox0\vbox\bgroup}{}
}
{%
\ifthenelse{\givenatend}{\end{trivlist}}{\egroup}{}
}
\else
\newenvironment{instructorNotes}[1][false]%
{%
  \ifthenelse{\boolean{#1}}{\begin{trivlist}\item[\hskip \labelsep\bfseries {\Large Instructor Notes: \\} \hspace{\textwidth} ]}
{\begin{trivlist}\item[\hskip \labelsep\bfseries {\Large Instructor Notes: \\} \hspace{\textwidth} ]}
{}
}
{\end{trivlist}}
\fi


%% Suggested Timing
\newcommand{\timing}[1]{{\bf Suggested Timing: \hspace{2ex}} #1}

\title{Maximums and Minimums}
\author{Bart Snapp and Brad Findell}

\outcome{Learning outcome goes here.}

\begin{document}
\begin{abstract}
  We think about different forms of quadratic equations.
\end{abstract}
\maketitle

\label{A:vertex}

%\begin{teachingnote}
%This activity will be necessary for computing least squares approximation.
%\end{teachingnote}

%
%While you might have encountered completing the
%square\index{completing the square} first when solving quadratic
%equations, its real power is in transforming the form of an
%expression to reveal properties of that expression. 
%
%In this activity, we'll see it in action.

  

In high school mathematics, you saw three different forms for quadratic functions.  In this activity, we explore the advantages and disadvantages of each.  

Note:  We use only real numbers for $x$.  And we begin by agreeing that the shape of the graph of a quadratic function is a parabola.  

\begin{problem}
Consider the function $f(x) = x^2 -3$. What are the maximum/minimum value(s) of $f(x)$, and for what $x$ values do they occur? 
Explain your reasoning.  Use this information to sketch a graph.  
\end{problem}

\begin{problem}
Consider the function $f(x) = 3(x-5)^2 +7$. What are the maximum/minimum value(s) of $f(x)$, and for what $x$ values do they occur? Explain your reasoning.  Use this information to sketch a graph.  
\end{problem}

\begin{problem}
Consider the function $f(x) = -2(x+3)^2 + 7$. What are the maximum/minimum value(s) of $f(x)$, and for what $x$ values do they occur? Explain your reasoning.  Use this information to sketch a graph.  
\end{problem}

\begin{problem}
What are the advantages of the form $f(x) = a(x-h)^2+k$ for a quadratic function?  Why is it called vertex form?  What do the values of $a$, $h$, and $k$ tell you about the graph?  
\end{problem}


\begin{problem}
Consider the function $f(x) = x^2 + 4x + 2$. Complete the square to
put this function into vertex form, and sketch a graph.
\end{problem}

\vfill

\begin{problem}
Consider the function $f(x) = 2x^2 - 8x + 6$. Complete the square to
put this function into vertex form, and sketch a graph.
\end{problem}

\vfill
\newpage

\begin{problem}
Consider the function $f(x) = (x+1)(x+5)$.  
\begin{enumerate}
\item What points on the graph are easy to locate?  
\item How can you use those points to find the vertex? 
\item Sketch the graph. 
\end{enumerate}
\end{problem}


\begin{problem}
Consider the function $f(x) = -2(x-2)(x+3)$.  
\begin{enumerate}
\item What points on the graph are easy to locate?  
\item How can you use those points to find the vertex? 
\item Sketch the graph. 
\end{enumerate}
\end{problem}

\begin{problem}
Consider the function $f(x) = a(x-r_1)(x-r_2)$.  
\begin{enumerate}
\item What do the values of $a$, $r_1$, and $r_2$ tell you about the graph?  
\item How can you use that information to find the vertex? 
\item What is this form called and why?  
\end{enumerate}
\end{problem}

\begin{problem}
Consider the function $f(x) = ax^2+bx+c$.  
\begin{enumerate}
\item What do the values of $a$, $b$, and $c$ tell you about the graph?  
\item What are the advantages and disadvantages of this form?  
\end{enumerate}
\end{problem}


%\begin{problem}
%Consider the parabola $f(x) = 3x^2 + 7x - 1$. Complete the square to
%put this expression in the form above and identify the maximum/minimum
%value(s) of this curve.
%\end{problem}
%
%
%\begin{problem}
%Given a parabola $f(x) = ax^2 + bx +c$. Complete the square to put
%this expression in the form above and identify the maximum/minimum
%value(s) of this curve.
%\end{problem}
%
%
%\begin{problem}
%Could you find the same formula found in the previous question by
%appealing to the symmetry of the roots?
%\end{problem}

\end{document}
