%\documentclass[handout]{ximera}
\documentclass[nooutcomes,instructornotes]{ximera}

\usepackage{gensymb}
\usepackage{tabularx}
\usepackage{mdframed}
\usepackage{pdfpages}
%\usepackage{chngcntr}

\let\problem\relax
\let\endproblem\relax

\newcommand{\property}[2]{#1#2}




\newtheoremstyle{SlantTheorem}{\topsep}{\fill}%%% space between body and thm
 {\slshape}                      %%% Thm body font
 {}                              %%% Indent amount (empty = no indent)
 {\bfseries\sffamily}            %%% Thm head font
 {}                              %%% Punctuation after thm head
 {3ex}                           %%% Space after thm head
 {\thmname{#1}\thmnumber{ #2}\thmnote{ \bfseries(#3)}} %%% Thm head spec
\theoremstyle{SlantTheorem}
\newtheorem{problem}{Problem}[]

%\counterwithin*{problem}{section}



%%%%%%%%%%%%%%%%%%%%%%%%%%%%Jenny's code%%%%%%%%%%%%%%%%%%%%

%%% Solution environment
%\newenvironment{solution}{
%\ifhandout\setbox0\vbox\bgroup\else
%\begin{trivlist}\item[\hskip \labelsep\small\itshape\bfseries Solution\hspace{2ex}]
%\par\noindent\upshape\small
%\fi}
%{\ifhandout\egroup\else
%\end{trivlist}
%\fi}
%
%
%%% instructorIntro environment
%\ifhandout
%\newenvironment{instructorIntro}[1][false]%
%{%
%\def\givenatend{\boolean{#1}}\ifthenelse{\boolean{#1}}{\begin{trivlist}\item}{\setbox0\vbox\bgroup}{}
%}
%{%
%\ifthenelse{\givenatend}{\end{trivlist}}{\egroup}{}
%}
%\else
%\newenvironment{instructorIntro}[1][false]%
%{%
%  \ifthenelse{\boolean{#1}}{\begin{trivlist}\item[\hskip \labelsep\bfseries Instructor Notes:\hspace{2ex}]}
%{\begin{trivlist}\item[\hskip \labelsep\bfseries Instructor Notes:\hspace{2ex}]}
%{}
%}
%% %% line at the bottom} 
%{\end{trivlist}\par\addvspace{.5ex}\nobreak\noindent\hung} 
%\fi
%
%


\let\instructorNotes\relax
\let\endinstructorNotes\relax
%%% instructorNotes environment
\ifhandout
\newenvironment{instructorNotes}[1][false]%
{%
\def\givenatend{\boolean{#1}}\ifthenelse{\boolean{#1}}{\begin{trivlist}\item}{\setbox0\vbox\bgroup}{}
}
{%
\ifthenelse{\givenatend}{\end{trivlist}}{\egroup}{}
}
\else
\newenvironment{instructorNotes}[1][false]%
{%
  \ifthenelse{\boolean{#1}}{\begin{trivlist}\item[\hskip \labelsep\bfseries {\Large Instructor Notes: \\} \hspace{\textwidth} ]}
{\begin{trivlist}\item[\hskip \labelsep\bfseries {\Large Instructor Notes: \\} \hspace{\textwidth} ]}
{}
}
{\end{trivlist}}
\fi


%% Suggested Timing
\newcommand{\timing}[1]{{\bf Suggested Timing: \hspace{2ex}} #1}




\hypersetup{
    colorlinks=true,       % false: boxed links; true: colored links
    linkcolor=blue,          % color of internal links (change box color with linkbordercolor)
    citecolor=green,        % color of links to bibliography
    filecolor=magenta,      % color of file links
    urlcolor=cyan           % color of external links
}

\title{Hieroglyphical Arithmetic}
\author{Bart Snapp and Brad Findell}

\outcome{Learning outcome goes here.}

\begin{document}
\begin{abstract}
    We use strange operation tables to solve arithmetic
    problems.
\end{abstract}
\maketitle

\label{A:HAr}
\emph{Note: This activity is based on an activity originally designed by Lee Wayand.}

\begin{problem}
Suppose the symbol $\,\text{\ding{72}}\,$ is used for some operation on mathematical objects $a$ and $b$.  
\begin{enumerate}
\item What does it mean to say that the operation $\,\text{\ding{72}}\,$ is \textbf{commutative}?
\item Name some familiar operations that are commutative. 
\item Name some familiar operations that are \textbf{not} commutative. 
\end{enumerate}
\end{problem}


\begin{problem}
Suppose the symbol $\,\text{\ding{72}}\,$ is used for some operation on mathematical objects $a$ and $b$.  
\begin{enumerate}
\item What does it mean to say that the operation $\,\text{\ding{72}}\,$ is \textbf{associative}?
\item Name some familiar operations that are associative. 
\item Name some familiar operations that are \textbf{not} associative. 
\end{enumerate}
\end{problem}

\begin{problem}
An operation $\,\text{\ding{72}}\,$ is called \textbf{closed} on a set of numbers if for all
numbers $a$ and $b$ in the set:
\[
a \,\text{\ding{72}}\, b \qquad \textrm{is another number in the set}.  
\]
\begin{enumerate}
\item Name an operation and a set that is closed under that operation.  
\item Name a set that is \textbf{not} closed under that same operation.  
\item Name another operation and a set that is closed under that operation. 
\item Name a set that is \textbf{not} closed under that same operation.  
\end{enumerate}
\end{problem}


\begin{problem}
Suppose some number system has two operations: $\,\text{\ding{72}}\,$ and $\,\text{\ding{66}}\,$.  

What does it mean for the operation $\,\text{\ding{72}}\,$ to be \textbf{distributive} over the 
operation $\,\text{\ding{66}}\,$?

\begin{teachingnote}
\[
a \,\text{\ding{72}}\,(b \,\text{\ding{66}}\, c)  = (a \,\text{\ding{72}}\, b) \,\text{\ding{66}}\, (a\,\text{\ding{72}}\, c)
\qquad\text{and}\qquad (b \,\text{\ding{66}}\, c)\,\text{\ding{72}}\, a  = (b \,\text{\ding{72}}\, a) \,\text{\ding{66}}\, (c\,\text{\ding{72}}\, a)
\]
\end{teachingnote}
\end{problem}

\subsection{A New Number System}
Consider the following addition and multiplication tables:

\[
\renewcommand{\arraystretch}{1.8}
\begin{array}{cl}
\begin{array}{|c||c|c|c|c|c|c|c|c|c|}\hline
 +  &\loo & \la & \lo & \lb & \lc & \ld & \lf & \lh & \li \\ \hline\hline
\loo& \ld & \lh &\loo & \li & \lb & \lo & \la & \lf & \lc \\ \hline
\la & \lh & \li & \la & \ld &\loo & \lf & \lb & \lc & \lo \\ \hline
\lo &\loo & \la & \lo & \lb & \lc & \ld & \lf & \lh & \li \\ \hline
\lb & \li & \ld & \lb & \lh & \la & \lc &\loo & \lo & \lf \\ \hline
\lc & \lb &\loo & \lc & \la & \lf & \li & \lo & \ld & \lh \\ \hline
\ld & \lo & \lf & \ld & \lc & \li &\loo & \lh & \la & \lb \\ \hline
\lf & \la & \lb & \lf &\loo & \lo & \lh & \lc & \li & \ld \\ \hline
\lh & \lf & \lc & \lh & \lo & \ld & \la & \li & \lb &\loo \\ \hline
\li & \lc & \lo & \li & \lf & \lh & \lb & \ld &\loo & \la \\ \hline
\end{array}
& 
\begin{array}{l}
\text{$\loo =$ fish} \\ 
\text{$\la =$ lolly-pop} \\ 
\text{$\lo =$ skull} \\ 
\text{$\lb =$ cinder-block} \\ 
\text{$\lc =$ DNA} \\ 
\text{$\ld =$ fork} \\ 
\text{$\lf =$ man} \\ 
\text{$\lh =$ balloon} \\ 
\text{$\li =$ eyeball} 
\end{array}
\end{array}
\]



\[
\renewcommand{\arraystretch}{1.8}
\begin{array}{cl}
\begin{array}{|c||c|c|c|c|c|c|c|c|c|}\hline
\cdot & \ld & \li & \lh & \lf & \lo & \la & \lb & \lc &\loo \\ \hline\hline
\ld   &\loo & \la & \lb & \lc & \lo & \li & \lh & \lf & \ld \\ \hline
\li   & \la & \lh & \lf & \ld & \lo & \lb & \lc &\loo & \li \\ \hline
\lh   & \lb & \lf & \ld & \la & \lo & \lc &\loo & \li & \lh \\ \hline
\lf   & \lc & \ld & \la & \lb & \lo &\loo & \li & \lh & \lf \\ \hline
\lo   & \lo & \lo & \lo & \lo & \lo & \lo & \lo & \lo & \lo \\ \hline
\la   & \li & \lb & \lc &\loo & \lo & \lh & \lf & \ld & \la \\ \hline
\lb   & \lh & \lc &\loo & \li & \lo & \lf & \ld & \la & \lb \\ \hline
\lc   & \lf &\loo & \li & \lh & \lo & \ld & \la & \lb & \lc \\ \hline
\loo  & \ld & \li & \lh & \lf & \lo & \la & \lb & \lc &\loo \\ \hline
\end{array}
& 
\begin{array}{l}
\text{$\loo =$ fish} \\ 
\text{$\la =$ lolly-pop} \\ 
\text{$\lo =$ skull} \\ 
\text{$\lb =$ cinder-block} \\ 
\text{$\lc =$ DNA} \\ 
\text{$\ld =$ fork} \\ 
\text{$\lf =$ man} \\ 
\text{$\lh =$ balloon} \\ 
\text{$\li =$ eyeball} 
\end{array}
\end{array}
\]


\begin{problem} 
Use the addition table to compute the following:
\[
\lb + \la \qquad\text{and}\qquad \lc + \lf
\]
\end{problem}

\begin{problem} 
Do you notice any patterns in the addition table? Tell us about them.
\begin{teachingnote}
Discuss \textbf{commutativity}, which can be seen by symmetry across the main diagonal of the table.  (This symmetry is much easier to see if these hieroglyphics are replaced by familiar symbols, such as Latin (or even Greek) letters.)  

Discuss \textbf{closure}, which is demonstrated by the fact that every entry in the interior of the table appears as both a row and a column heading.  For comparison, in the operation table for single-digit addition, 13 appears in the interior of the table but not as a row or column heading.  

For \textbf{associativity}, there is not a simple test in the table.  The naive approach requires checking that $(a + b) + c = a + (b + c)$ for all triples $a, b, c$.  
\end{teachingnote}
\end{problem}

\begin{problem} 
Can you tell me which glyph represents $0$? How did you arrive at this
conclusion?
\begin{teachingnote}
Call $\lo$ the \textbf{additive identity} in this number system.
\end{teachingnote}
\end{problem}

\begin{problem} 
Use the multiplication table to compute the following:
\[
\li \cdot \lb \qquad\text{and}\qquad \la \cdot \lf
\]
\end{problem}

\begin{problem} 
Do you notice any patterns in the multiplication table? Tell us about them.
\begin{teachingnote}
Again discuss \textbf{commutativity} and \textbf{closure}.  And again note that \textbf{associativity} is tedious to check fully.  
\end{teachingnote}

\end{problem}

\begin{problem} 
Can you tell me which glyph represents $1$? How did you arrive at this
conclusion?
\begin{teachingnote}
Call $\loo$ the \textbf{multiplicative identity}.  
\end{teachingnote}
\end{problem}


\begin{problem} Compute:
\[
\loo - \la \qquad\text{and}\qquad \ld - \lc
\]
\begin{teachingnote}
Most students will do the computation by solving (i.e., finding the unknowns in) the equations  
\[
\la + x = \loo  \qquad\text{and}\qquad \lc + x = \ld
\]
And the answers are $\lc$ and $\lh$, respectively.  

New questions and answers:  
\begin{enumerate}
\item Q: Find $-\la$ and $-\lc$.  
\item A: From the table, $-\la=\li$ and $-\lc=\lf$.  Call these \textbf{additive inverses}, and note that the sum of a value and its additive inverse is the additive identity.  
\item Q: Use $-\la$ and $-\lc$ to make the computations $\loo - \la$ and $\ld - \lc$ directly. 
\item A: First, $\loo - \la = \loo + \li =\lc$.  Then, $\ld - \lc = \ld + \lf = \lh$.  
\end {enumerate}

The upshot is that there are (at least) two ways of thinking about subtraction:  Solving an addition equation, and adding an additive inverse.  

Note: In the expression $\loo - \la$, the negative sign indicates a binary operation, subtraction.  In the expression $-\la$, the negative sign indicates a unary operation, negation.  This ``double use'' of the negative sign obscures some subtleties here.  
\end{teachingnote}
\end{problem}

\begin{problem} Compute:
\[
\lf \div \lc \qquad\text{and}\qquad \lh \div \li
\]
\begin{teachingnote}
Analogous to the previous problem, aim for two ways of thinking about division:  Solving a multiplication equation, and multiplying by multiplicative inverse. 
\[
\lf \div \lc = \ld \qquad\text{and}\qquad \lh \div \li = \li. \\
\]
It happens that $\lc$ are $\li$ multiplicative inverses of each other.  

Note:  The notations $a^{-1}$ and $1/a$ are both confusing and also far from $1\div a$, upon which the ideas are based.  
\end{teachingnote}
\end{problem}

\begin{problem} 
Keen Kelley was working with our tables above. All of a sudden, she
writes
\[
\loo + \loo + \loo = \lo
\]
and shouts ``Weird!'' Why is she so surprised? Try repeated addition
with other glyphs. What do you find? Can you explain this?
\begin{teachingnote}
Students might notice that for all elements in this set, the sum of three copies of an element is always the additive identity (i.e., $x + x + x = 0$).  This is a feature of this number system, which (for what its worth) is the field with 9 elements.  
\end{teachingnote}
\end{problem}


\begin{problem}
Can you find any other oddities of the arithmetic above? Hint: Try
repeated multiplication!
\begin{teachingnote}
One possibility: Students might use the shorthand that $\loo = 1$ and $\lo = 0$, and then conclude that $\ld = -1$ because $\loo+\ld=\lo$.  Then they might notice that $\lh\cdot \lh = \ld$, so $\lh$ is like $i$, a ``square root of negative 1.''
\end{teachingnote}
\end{problem}

\end{document}
