%\documentclass[handout]{ximera}
\documentclass[nooutcomes]{ximera}

\usepackage{gensymb}
\usepackage{tabularx}
\usepackage{mdframed}
\usepackage{pdfpages}
%\usepackage{chngcntr}

\let\problem\relax
\let\endproblem\relax

\newcommand{\property}[2]{#1#2}




\newtheoremstyle{SlantTheorem}{\topsep}{\fill}%%% space between body and thm
 {\slshape}                      %%% Thm body font
 {}                              %%% Indent amount (empty = no indent)
 {\bfseries\sffamily}            %%% Thm head font
 {}                              %%% Punctuation after thm head
 {3ex}                           %%% Space after thm head
 {\thmname{#1}\thmnumber{ #2}\thmnote{ \bfseries(#3)}} %%% Thm head spec
\theoremstyle{SlantTheorem}
\newtheorem{problem}{Problem}[]

%\counterwithin*{problem}{section}



%%%%%%%%%%%%%%%%%%%%%%%%%%%%Jenny's code%%%%%%%%%%%%%%%%%%%%

%%% Solution environment
%\newenvironment{solution}{
%\ifhandout\setbox0\vbox\bgroup\else
%\begin{trivlist}\item[\hskip \labelsep\small\itshape\bfseries Solution\hspace{2ex}]
%\par\noindent\upshape\small
%\fi}
%{\ifhandout\egroup\else
%\end{trivlist}
%\fi}
%
%
%%% instructorIntro environment
%\ifhandout
%\newenvironment{instructorIntro}[1][false]%
%{%
%\def\givenatend{\boolean{#1}}\ifthenelse{\boolean{#1}}{\begin{trivlist}\item}{\setbox0\vbox\bgroup}{}
%}
%{%
%\ifthenelse{\givenatend}{\end{trivlist}}{\egroup}{}
%}
%\else
%\newenvironment{instructorIntro}[1][false]%
%{%
%  \ifthenelse{\boolean{#1}}{\begin{trivlist}\item[\hskip \labelsep\bfseries Instructor Notes:\hspace{2ex}]}
%{\begin{trivlist}\item[\hskip \labelsep\bfseries Instructor Notes:\hspace{2ex}]}
%{}
%}
%% %% line at the bottom} 
%{\end{trivlist}\par\addvspace{.5ex}\nobreak\noindent\hung} 
%\fi
%
%


\let\instructorNotes\relax
\let\endinstructorNotes\relax
%%% instructorNotes environment
\ifhandout
\newenvironment{instructorNotes}[1][false]%
{%
\def\givenatend{\boolean{#1}}\ifthenelse{\boolean{#1}}{\begin{trivlist}\item}{\setbox0\vbox\bgroup}{}
}
{%
\ifthenelse{\givenatend}{\end{trivlist}}{\egroup}{}
}
\else
\newenvironment{instructorNotes}[1][false]%
{%
  \ifthenelse{\boolean{#1}}{\begin{trivlist}\item[\hskip \labelsep\bfseries {\Large Instructor Notes: \\} \hspace{\textwidth} ]}
{\begin{trivlist}\item[\hskip \labelsep\bfseries {\Large Instructor Notes: \\} \hspace{\textwidth} ]}
{}
}
{\end{trivlist}}
\fi


%% Suggested Timing
\newcommand{\timing}[1]{{\bf Suggested Timing: \hspace{2ex}} #1}




\hypersetup{
    colorlinks=true,       % false: boxed links; true: colored links
    linkcolor=blue,          % color of internal links (change box color with linkbordercolor)
    citecolor=green,        % color of links to bibliography
    filecolor=magenta,      % color of file links
    urlcolor=cyan           % color of external links
}

\title{Which Road Should We Take?}
\author{Bart Snapp and Brad Findell}

\outcome{Learning outcome goes here.}

\begin{document}
\begin{abstract}
  We explore distributions of dice rolls.
\end{abstract}
\maketitle

\begin{problem} Consider a six-sided die. Without actually rolling a die, guess the number of 1's, 2's, 3's, 4's, 5's, and 6's you would obtain in 50 rolls. Record your predictions in the chart below:
\begin{center}\textbf{Predictions}\end{center}
\[
\begin{tabular}{|c|c|c|c|c|c|c|}\hline
\# of 1's & \# of 2's & \# of 3's & \# of 4's & \# of 5's & \# of 6's & Total \\ \hline 
\rule[0mm]{0mm}{7mm} \hspace{15mm} &  \hspace{15mm} & \hspace{15mm} & \hspace{15mm} & \hspace{15mm} & \hspace{15mm} & \hspace{15mm}\\ \hline
\end{tabular}
\]


Now roll a die 50 times and record the number of 1's, 2's, 3's, 4's, 5's, and 6's you obtain.  
\begin{center}\textbf{Experimental Results}\end{center}
\[
\begin{tabular}{|c|c|c|c|c|c|c|}\hline
\# of 1's & \# of 2's & \# of 3's & \# of 4's & \# of 5's & \# of 6's & Total \\ \hline 
\rule[0mm]{0mm}{7mm} \hspace{15mm} &  \hspace{15mm} & \hspace{15mm} & \hspace{15mm} & \hspace{15mm} & \hspace{15mm} & \hspace{15mm}\\ \hline
\end{tabular}
\]
How did you come up with your predictions? How do your predictions
compare with your actual results? Now make a chart to combine your
data with that of the rest of the class.

\end{problem}

\paragraph{Experiment 1}  We investigated the results of throwing 
one die and recording what we saw (a $1$, a $2$, ..., or a $6$).  
We said that the probability of an event (for example, getting a ``3'' 
in this experiment) predicts the frequency with which we expect to see 
that event occur in a large number of trials.
You argued the $P(\text{seeing }3)=1/6$ (meaning we
expect to get a $3$ in about $1/6$ of our trials) because there were
six different outcomes, only one of them is a $3$, and you expected
each outcome to occur about the same number of times.

\paragraph{Experiment 2}
We are now investigating the results of throwing two dice and
recording the sum of the faces. We are trying to analyze the
probabilities associated with these sums.  Let's focus first on
$P(sum=2)=?$. We might have some different theories, such 
as the following:


\paragraph{Theory 1} 
$P(\text{sum}=2)=1/11$. 

It is proposed that a sum of $2$ was $1$ out
of the $11$ possible sums $\{2,3,4,5,6,7,8,9,10,11,12\}$.


\paragraph{Theory 2} 
$P(\text{sum}=2)=1/21$. 

It is proposed that a sum of $2$ was $1$ of
$21$ possible results, counting $1+3$ as the same as $3+1$:

\[
\begin{array}{cccccc}
1+1 & - & - & - & - & - \\
2+1 & 2+2 & - & - & - & - \\
3+1 & 3+2 & 3+3 & - & - & - \\
4+1 & 4+2 & 4+3 & 4+4 & - & - \\
5+1 & 5+2 & 5+3 & 5+4 & 5+5 & - \\
6+1 & 6+2 & 6+3 & 6+4 & 6+5 & 6+6 
\end{array}
\]

\begin{teachingnote}
Some might suggest that the sample space is better thought of as ordered pairs rather than sums.  In any case, the upshot is that for anything we do with the two dice, the sample space has 36 elements.  Then the following activity, Lumpy and Eddy, might be redundant.
\end{teachingnote}

\begin{problem}
Propose your own Theory 3.  
\end{problem}

%\subparagraph{Theory 3} $P(\text{sum}=2)=1/36$. 
%
%It was proposed that a
%sum of $2$ was $1$ of $36$ possible results, counting $1+3$ as
%different from $3+1$:
%\[
%\begin{array}{cccccc}
%1+1 & 1+2 & 1+3 & 1+4 & 1+5 & 1+6 \\
%2+1 & 2+2 & 2+3 & 2+4 & 2+5 & 2+6 \\
%3+1 & 3+2 & 3+3 & 3+4 & 3+5 & 3+6 \\
%4+1 & 4+2 & 4+3 & 4+4 & 4+5 & 4+6 \\
%5+1 & 5+2 & 5+3 & 5+4 & 5+5 & 5+6 \\
%6+1 & 6+2 & 6+3 & 6+4 & 6+5 & 6+6 
%\end{array}
%\]

\begin{problem}
Test all theories by computing $P(2)$, $P(3)$, \dots , $P(12)$ for each theory and comparing to the dice rolls recorded by the class.  What do you notice?  
\end{problem}

\begin{problem}
Which theory do you like best?  Why?
\end{problem}


\begin{problem}
 How could we test our theory further?
\end{problem}

\end{document}
