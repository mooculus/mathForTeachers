%\documentclass[handout]{ximera}
\documentclass[nooutcomes]{ximera}

\usepackage{gensymb}
\usepackage{tabularx}
\usepackage{mdframed}
\usepackage{pdfpages}
%\usepackage{chngcntr}

\let\problem\relax
\let\endproblem\relax

\newcommand{\property}[2]{#1#2}




\newtheoremstyle{SlantTheorem}{\topsep}{\fill}%%% space between body and thm
 {\slshape}                      %%% Thm body font
 {}                              %%% Indent amount (empty = no indent)
 {\bfseries\sffamily}            %%% Thm head font
 {}                              %%% Punctuation after thm head
 {3ex}                           %%% Space after thm head
 {\thmname{#1}\thmnumber{ #2}\thmnote{ \bfseries(#3)}} %%% Thm head spec
\theoremstyle{SlantTheorem}
\newtheorem{problem}{Problem}[]

%\counterwithin*{problem}{section}



%%%%%%%%%%%%%%%%%%%%%%%%%%%%Jenny's code%%%%%%%%%%%%%%%%%%%%

%%% Solution environment
%\newenvironment{solution}{
%\ifhandout\setbox0\vbox\bgroup\else
%\begin{trivlist}\item[\hskip \labelsep\small\itshape\bfseries Solution\hspace{2ex}]
%\par\noindent\upshape\small
%\fi}
%{\ifhandout\egroup\else
%\end{trivlist}
%\fi}
%
%
%%% instructorIntro environment
%\ifhandout
%\newenvironment{instructorIntro}[1][false]%
%{%
%\def\givenatend{\boolean{#1}}\ifthenelse{\boolean{#1}}{\begin{trivlist}\item}{\setbox0\vbox\bgroup}{}
%}
%{%
%\ifthenelse{\givenatend}{\end{trivlist}}{\egroup}{}
%}
%\else
%\newenvironment{instructorIntro}[1][false]%
%{%
%  \ifthenelse{\boolean{#1}}{\begin{trivlist}\item[\hskip \labelsep\bfseries Instructor Notes:\hspace{2ex}]}
%{\begin{trivlist}\item[\hskip \labelsep\bfseries Instructor Notes:\hspace{2ex}]}
%{}
%}
%% %% line at the bottom} 
%{\end{trivlist}\par\addvspace{.5ex}\nobreak\noindent\hung} 
%\fi
%
%


\let\instructorNotes\relax
\let\endinstructorNotes\relax
%%% instructorNotes environment
\ifhandout
\newenvironment{instructorNotes}[1][false]%
{%
\def\givenatend{\boolean{#1}}\ifthenelse{\boolean{#1}}{\begin{trivlist}\item}{\setbox0\vbox\bgroup}{}
}
{%
\ifthenelse{\givenatend}{\end{trivlist}}{\egroup}{}
}
\else
\newenvironment{instructorNotes}[1][false]%
{%
  \ifthenelse{\boolean{#1}}{\begin{trivlist}\item[\hskip \labelsep\bfseries {\Large Instructor Notes: \\} \hspace{\textwidth} ]}
{\begin{trivlist}\item[\hskip \labelsep\bfseries {\Large Instructor Notes: \\} \hspace{\textwidth} ]}
{}
}
{\end{trivlist}}
\fi


%% Suggested Timing
\newcommand{\timing}[1]{{\bf Suggested Timing: \hspace{2ex}} #1}




\hypersetup{
    colorlinks=true,       % false: boxed links; true: colored links
    linkcolor=blue,          % color of internal links (change box color with linkbordercolor)
    citecolor=green,        % color of links to bibliography
    filecolor=magenta,      % color of file links
    urlcolor=cyan           % color of external links
}

\title{Sketching Roots}
\author{Bart Snapp and Brad Findell}

\outcome{Learning outcome goes here.}

\begin{document}
\begin{abstract}
We seek to understand the connection between roots and the plots of
polynomials.
\end{abstract}
\maketitle

\label{A:sketchRoots}

In this activity we seek to better understand the connection between
roots and the plots of polynomials.  We will restrict our attention to polynomials with real coefficients.  

First, we need to be precise about the correct usage of some important language:  

\begin{itemize}
\item Expressions have \emph{values}.  
\item Equations have \emph{solutions}:  values of the variables that make the equation true. 
\item Functions have \emph{zeros}: input values that give output values of 0.
\item Polynomials (i.e., polynomial expressions) have \emph{roots}.  
\end{itemize}
These ideas are related, of course, as follows:  A zero of a polynomial function, $p(x)$, is a root of the polynomial $p(x)$ and a solution to the equation $p(x) = 0$.  

Please try to use this language correctly:  Equations do not have zeros, and functions do not have solutions.  

\begin{problem}
Give an example of a polynomial, and write a true sentence about related equations, functions, zeros, equations, and roots.  
\end{problem}

\begin{problem}
Sketch the plot of a quadratic polynomial with real coefficients that has:
\begin{enumerate}
\item Two real roots.
\item One repeated real root.
\item No real roots.
\end{enumerate}
In each case, give an example of such a polynomial.
\end{problem}

\begin{problem}
Can you have a quadratic polynomial with exactly one real root and
$1$ complex root?  Explain why or why not.
\end{problem}


\begin{problem}
Sketch the plot of a cubic polynomial with real coefficients that has:
\begin{enumerate}
\item Three distinct real roots.
\item One real root and two complex roots.
\end{enumerate}
In each case, give an example of such a polynomial.
\end{problem}

\begin{problem}
Can you have a cubic polynomial with no real roots?  Explain why or
why not. What about two distinct real roots and one complex root?
\end{problem}


\begin{problem}
For polynomials with real coefficients of degree $1$ to $5$, classify
exactly which types of roots can be found. For example, in our work
above, we classified polynomials of degree $2$ and $3$.
\end{problem}

\end{document}
