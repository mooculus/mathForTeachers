%\documentclass[handout]{ximera}
\documentclass{ximera}


\graphicspath{
  {./}
  {graphics/}
  {../graphics/}
}

\usepackage{chngcntr}

\let\question\relax
\let\endquestion\relax




\newtheoremstyle{SlantTheorem}{\topsep}{\fill}%%% space between body and thm
%\newtheoremstyle{SlantTheorem}{\topsep}{\topsep}%%% space between body and thm
 {\slshape}                      %%% Thm body font
 {}                              %%% Indent amount (empty = no indent)
 {\bfseries\sffamily}            %%% Thm head font
 {}                              %%% Punctuation after thm head
 {3ex}                           %%% Space after thm head
 {\thmname{#1}\thmnumber{ #2}\thmnote{ \bfseries(#3)}}%%% Thm head spec
\theoremstyle{SlantTheorem}
\newtheorem{question}{Question}
\counterwithin*{question}{section}



\let\instructorNotes\relax
\let\endinstructorNotes\relax
%%% instructorNotes environment
\ifhandout
\newenvironment{instructorNotes}[1][false]%
{%
\def\givenatend{\boolean{#1}}\ifthenelse{\boolean{#1}}{\begin{trivlist}\item}{\setbox0\vbox\bgroup}{}
}
{%
\ifthenelse{\givenatend}{\end{trivlist}}{\egroup}{}
}
\else
\newenvironment{instructorNotes}[1][false]%
{%
  \ifthenelse{\boolean{#1}}{\begin{trivlist}\item[\hskip \labelsep\bfseries {\Large Instructor Notes: \\} \hspace{\textwidth} ]}
{\begin{trivlist}\item[\hskip \labelsep\bfseries {\Large Instructor Notes: \\} \hspace{\textwidth} ]}
{}
}
{\end{trivlist}}
\fi


%% Suggested Timing
\newcommand{\timing}[1]{{\bf Suggested Timing: \hspace{2ex}} #1}

\title{Cross Something-ing}
\author{Bart Snapp and Brad Findell}

\outcome{Learning outcome goes here.}

\begin{document}
\begin{abstract}
Abstract goes here.  
\end{abstract}
\maketitle

\label{A:CrossSomething}


\begin{problem} 
What might someone call the following statements:
\begin{enumerate}
\item $\dfrac{a}{b} = \dfrac{c}{d} \Rightarrow ad = bc$
\item $\dfrac{a}{b}\cdot \dfrac{b}{c} = \dfrac{a}{c}$
\item $\dfrac{a}{b}\div \dfrac{c}{d} = \dfrac{ad}{bc}$
\item $\dfrac{a}{b} +\dfrac{c}{d} = \dfrac{ad+bc}{bd}$
\item $ad < bc \Rightarrow \dfrac{a}{b} < \dfrac{c}{d}$
\item $ad < bc \Rightarrow \dfrac{c}{d} < \dfrac{a}{b}$
\end{enumerate}
\end{problem}

\begin{problem}
Which of the above statements are true? What specific name might you
use to describe them?
\end{problem}

\begin{problem} 
Use pictures to help explain why the true statements above are true
and give counterexamples showing that the false statements are false.
\end{problem}


\begin{problem} 
Can you think of other statements that should be grouped with those
above?
\end{problem}

\begin{problem}
If mathematics is a subject where you should strive to ``say what you
mean and mean what you say,'' what issue might arise with
cross-multiplication?
\end{problem}

\end{document}
