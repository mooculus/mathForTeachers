%\documentclass[handout]{ximera}
\documentclass{ximera}

\usepackage{gensymb}
\usepackage{tabularx}
\usepackage{mdframed}
\usepackage{pdfpages}
%\usepackage{chngcntr}

\let\problem\relax
\let\endproblem\relax

\newcommand{\property}[2]{#1#2}




\newtheoremstyle{SlantTheorem}{\topsep}{\fill}%%% space between body and thm
 {\slshape}                      %%% Thm body font
 {}                              %%% Indent amount (empty = no indent)
 {\bfseries\sffamily}            %%% Thm head font
 {}                              %%% Punctuation after thm head
 {3ex}                           %%% Space after thm head
 {\thmname{#1}\thmnumber{ #2}\thmnote{ \bfseries(#3)}} %%% Thm head spec
\theoremstyle{SlantTheorem}
\newtheorem{problem}{Problem}[]

%\counterwithin*{problem}{section}



%%%%%%%%%%%%%%%%%%%%%%%%%%%%Jenny's code%%%%%%%%%%%%%%%%%%%%

%%% Solution environment
%\newenvironment{solution}{
%\ifhandout\setbox0\vbox\bgroup\else
%\begin{trivlist}\item[\hskip \labelsep\small\itshape\bfseries Solution\hspace{2ex}]
%\par\noindent\upshape\small
%\fi}
%{\ifhandout\egroup\else
%\end{trivlist}
%\fi}
%
%
%%% instructorIntro environment
%\ifhandout
%\newenvironment{instructorIntro}[1][false]%
%{%
%\def\givenatend{\boolean{#1}}\ifthenelse{\boolean{#1}}{\begin{trivlist}\item}{\setbox0\vbox\bgroup}{}
%}
%{%
%\ifthenelse{\givenatend}{\end{trivlist}}{\egroup}{}
%}
%\else
%\newenvironment{instructorIntro}[1][false]%
%{%
%  \ifthenelse{\boolean{#1}}{\begin{trivlist}\item[\hskip \labelsep\bfseries Instructor Notes:\hspace{2ex}]}
%{\begin{trivlist}\item[\hskip \labelsep\bfseries Instructor Notes:\hspace{2ex}]}
%{}
%}
%% %% line at the bottom} 
%{\end{trivlist}\par\addvspace{.5ex}\nobreak\noindent\hung} 
%\fi
%
%


\let\instructorNotes\relax
\let\endinstructorNotes\relax
%%% instructorNotes environment
\ifhandout
\newenvironment{instructorNotes}[1][false]%
{%
\def\givenatend{\boolean{#1}}\ifthenelse{\boolean{#1}}{\begin{trivlist}\item}{\setbox0\vbox\bgroup}{}
}
{%
\ifthenelse{\givenatend}{\end{trivlist}}{\egroup}{}
}
\else
\newenvironment{instructorNotes}[1][false]%
{%
  \ifthenelse{\boolean{#1}}{\begin{trivlist}\item[\hskip \labelsep\bfseries {\Large Instructor Notes: \\} \hspace{\textwidth} ]}
{\begin{trivlist}\item[\hskip \labelsep\bfseries {\Large Instructor Notes: \\} \hspace{\textwidth} ]}
{}
}
{\end{trivlist}}
\fi


%% Suggested Timing
\newcommand{\timing}[1]{{\bf Suggested Timing: \hspace{2ex}} #1}




\hypersetup{
    colorlinks=true,       % false: boxed links; true: colored links
    linkcolor=blue,          % color of internal links (change box color with linkbordercolor)
    citecolor=green,        % color of links to bibliography
    filecolor=magenta,      % color of file links
    urlcolor=cyan           % color of external links
}

\title{Shampoo, Rinse, \dots}
\author{Bart Snapp and Brad Findell}

\outcome{Learning outcome goes here.}

\begin{document}
\begin{abstract}
  We think about decimal notation.
\end{abstract}
\maketitle

\label{A:Shampoo}
\index{repeating decimal}
\index{terminating decimal}

\fixnote{Table doesn't look great.  Cross-references are not working.}
We're going to investigate the following question: If $a$ and $b$ are
integers with $b \ne 0$, what can you say about the decimal
representation of $a/b$? 

As a middle school teacher, you should know from memory the decimal equivalents of many fractions, and 
you should be able to compute others quickly in your head.  Use this activity to hone this skill, and use your calculator 
as backup support.  

\begin{problem}\label{AR:exp} 
Complete the following table.  For type, write ``T'' for ``Terminating,'' and use other letters for other types you observe.  

%\begin{multicols}{2}

\renewcommand{\arraystretch}{2}
\begin{tabular}{l}
$\begin{array}{|c|p{1.5in}|p{0.5in}|}
\hline
\text{Fraction} & Decimal & Type \\ \hline
\frac{1}{2} & &  \\  \hline
\frac{1}{3} & &  \\   \hline
\frac{1}{4} & &  \\   \hline
\frac{1}{5} & &  \\   \hline
\frac{1}{6} & &  \\   \hline
\frac{1}{7} & &  \\   \hline
\frac{1}{8} & &  \\   \hline
\frac{1}{9} & &  \\   \hline
\frac{1}{10} & &  \\   \hline
\frac{1}{11} & &  \\   \hline
\frac{1}{12} & &  \\   \hline
\frac{1}{13} & &  \\  \hline
\frac{1}{14} & &  \\  \hline
\frac{1}{15} & &  \\  \hline
\frac{1}{16} & &  \\   \hline
\frac{1}{20} & &  \\   \hline
\frac{1}{24} & &  \\   \hline
\frac{1}{25} & &  \\   \hline
\frac{1}{28} & &  \\   \hline
\frac{1}{32} & &  \\   \hline
\frac{1}{35} & &  \\   \hline
\frac{1}{40} & &  \\   \hline
\frac{1}{42} & &  \\   \hline
\frac{1}{48} & &  \\   \hline
\frac{1}{64} & &  \\   \hline
\frac{1}{80} & &  \\   \hline
\end{array}$
\end{tabular}

%\end{multicols}


\end{problem}


%\begin{problem}
%Write the following fractions in decimal notation. Which have a
%``terminating'' and which have a ``non-terminating'' decimal?
%\[
%\begin{array}{cccccccccccc}
% \dfrac{1}{2}, &  \dfrac{1}{3}, &  \dfrac{1}{4}, &  \dfrac{1}{5}, &  \dfrac{1}{6}, & \dfrac{1}{8}, & \dfrac{1}{9}, &  \dfrac{1}{10}, &  \dfrac{1}{11}, &  \dfrac{1}{12}, &  \dfrac{1}{13},  & \dfrac{1}{15}\\ \\
% \dfrac{1}{16}, & \dfrac{1}{20}, &  \dfrac{1}{24}, &  \dfrac{1}{25}, &  \dfrac{1}{28}, &  \dfrac{1}{32}, &  \dfrac{1}{35}, &  \dfrac{1}{40}, &  \dfrac{1}{42}, &  \dfrac{1}{48}, &  \dfrac{1}{64}, &  \dfrac{1}{80}.
%\end{array}
%\]
%\end{problem}

\begin{problem}\label{AR:conj}
Can you find a pattern from your results from Problem~\ref{AR:exp}?
Use your pattern to guess whether the following fractions
``terminate''?  
\[
\dfrac{1}{61}\qquad \dfrac{1}{625} \qquad \dfrac{1}{6251}
\]
\end{problem}


\begin{problem}
Can you explain why your conjecture from Problem~\ref{AR:conj} is true?
\end{problem}

\begin{problem} Now let's consider fractions with decimal representations that do not terminate.  
\begin{enumerate}
\item Use long division to compute $1/7$.
\item \index{Division Theorem!for integers}
State the Division Theorem for integers.
\item How does the Division Theorem for integers appears in your computation for $1/7$?
\item In each instance of the Division Theorem, what 
is the divisor? And what does this imply about the remainder?
\item Generalize:  When $a$ and $b$ are integers with $b\ne 0$, 
what can you say about the decimal representation of $a/b$, assuming
it does not terminate?  Explain your reasoning.  
\end{enumerate}
\end{problem}

\begin{problem}\label{AR:nines} 
Compute $\frac{1}{9}$, $\frac{1}{99}$, and $\frac{1}{999}$. Can you
find a pattern? Can you explain why your pattern holds?
\end{problem}

\begin{teachingnote}
From long division, $\frac{1}{9} = 0.\overline{1}$, $\frac{1}{99} = 0.\overline{01}$, and $\frac{1}{999} = 0.\overline{001}$.  
\end{teachingnote}

\begin{problem}
Use your work from Problem~\ref{AR:nines} to give the fraction form of
the following decimals:
\begin{enumerate}
\item $0.\overline{357}$
\item $23.\overline{459}$
\item $0.23\overline{4598}$
\item $76.3\overline{421}$
\end{enumerate}
\end{problem}

\begin{teachingnote}
The technique is to reason from decimal multiplication as follows:  

$$0.\overline{357} = 357\times  0.\overline{001} = 357 \times \frac{1}{999}$$

No need to simplify these anticipated solutions: 
\begin{enumerate}
\item $0.\overline{357} = \dfrac{357}{999}$
\item $23.\overline{459} = 23 + \dfrac{459}{999}$
\item $0.23\overline{4598} = \dfrac{23}{100}+ \dfrac{1}{100}\times\dfrac{4598}{9999}$
\item $76.3\overline{421} = 76 + \dfrac{3}{10}+ \dfrac{1}{10}\times\dfrac{421}{999}$
\end{enumerate}
\end{teachingnote}

\begin{problem} 
Assuming that the pattern holds, is the number
\[
.123456789101112131415161718192021\dots
\]
a rational number? Explain your reasoning.
\end{problem}

\begin{teachingnote}
Reasoning from the finite list of remainders, the decimal representation of any rational number either terminates or (eventually) repeats.  In this problem, the number shows an interesting and predictiable pattern, but it does not show a sequence of digits that appears in exactly the same way again and again.  
\end{teachingnote}

\end{document}
