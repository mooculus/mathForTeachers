%\documentclass[handout]{ximera}
\documentclass[nooutcomes]{ximera}

\usepackage{gensymb}
\usepackage{tabularx}
\usepackage{mdframed}
\usepackage{pdfpages}
%\usepackage{chngcntr}

\let\problem\relax
\let\endproblem\relax

\newcommand{\property}[2]{#1#2}




\newtheoremstyle{SlantTheorem}{\topsep}{\fill}%%% space between body and thm
 {\slshape}                      %%% Thm body font
 {}                              %%% Indent amount (empty = no indent)
 {\bfseries\sffamily}            %%% Thm head font
 {}                              %%% Punctuation after thm head
 {3ex}                           %%% Space after thm head
 {\thmname{#1}\thmnumber{ #2}\thmnote{ \bfseries(#3)}} %%% Thm head spec
\theoremstyle{SlantTheorem}
\newtheorem{problem}{Problem}[]

%\counterwithin*{problem}{section}



%%%%%%%%%%%%%%%%%%%%%%%%%%%%Jenny's code%%%%%%%%%%%%%%%%%%%%

%%% Solution environment
%\newenvironment{solution}{
%\ifhandout\setbox0\vbox\bgroup\else
%\begin{trivlist}\item[\hskip \labelsep\small\itshape\bfseries Solution\hspace{2ex}]
%\par\noindent\upshape\small
%\fi}
%{\ifhandout\egroup\else
%\end{trivlist}
%\fi}
%
%
%%% instructorIntro environment
%\ifhandout
%\newenvironment{instructorIntro}[1][false]%
%{%
%\def\givenatend{\boolean{#1}}\ifthenelse{\boolean{#1}}{\begin{trivlist}\item}{\setbox0\vbox\bgroup}{}
%}
%{%
%\ifthenelse{\givenatend}{\end{trivlist}}{\egroup}{}
%}
%\else
%\newenvironment{instructorIntro}[1][false]%
%{%
%  \ifthenelse{\boolean{#1}}{\begin{trivlist}\item[\hskip \labelsep\bfseries Instructor Notes:\hspace{2ex}]}
%{\begin{trivlist}\item[\hskip \labelsep\bfseries Instructor Notes:\hspace{2ex}]}
%{}
%}
%% %% line at the bottom} 
%{\end{trivlist}\par\addvspace{.5ex}\nobreak\noindent\hung} 
%\fi
%
%


\let\instructorNotes\relax
\let\endinstructorNotes\relax
%%% instructorNotes environment
\ifhandout
\newenvironment{instructorNotes}[1][false]%
{%
\def\givenatend{\boolean{#1}}\ifthenelse{\boolean{#1}}{\begin{trivlist}\item}{\setbox0\vbox\bgroup}{}
}
{%
\ifthenelse{\givenatend}{\end{trivlist}}{\egroup}{}
}
\else
\newenvironment{instructorNotes}[1][false]%
{%
  \ifthenelse{\boolean{#1}}{\begin{trivlist}\item[\hskip \labelsep\bfseries {\Large Instructor Notes: \\} \hspace{\textwidth} ]}
{\begin{trivlist}\item[\hskip \labelsep\bfseries {\Large Instructor Notes: \\} \hspace{\textwidth} ]}
{}
}
{\end{trivlist}}
\fi


%% Suggested Timing
\newcommand{\timing}[1]{{\bf Suggested Timing: \hspace{2ex}} #1}




\hypersetup{
    colorlinks=true,       % false: boxed links; true: colored links
    linkcolor=blue,          % color of internal links (change box color with linkbordercolor)
    citecolor=green,        % color of links to bibliography
    filecolor=magenta,      % color of file links
    urlcolor=cyan           % color of external links
}

\title{Complete Squares}
\author{Bart Snapp and Brad Findell}

\outcome{Learning outcome goes here.}

\begin{document}
\begin{abstract}
  We think about completing the square.
\end{abstract}
\maketitle

\label{A:completeSquares}

\begin{problem}
In the following list of equations, solve those that are \textbf{easy} to solve.  
\begin{enumerate}
\item $x^2=5$
\item $x^2 - 4 = 2$
\item $x^2 - 4x = 2$
\item $2x^2=1$
\item $(x-2)^2=5$
\end{enumerate}
\end{problem}
\vspace{0.3in}

\begin{problem}
Regarding the previous problem, state the property of numbers that made all but one of the equations easy to solve.  
\begin{teachingnote}
If $u^2=a$ then $u=\pm\sqrt{a}$.  Be sure to spend some time discussing why 
$$\pm\sqrt{\frac{1}{2}}=\pm\frac{1}{\sqrt{2}}=\pm\frac{\sqrt{2}}{2}.$$
\end{teachingnote}
\vspace{0.3in}
\end{problem}

\begin{problem}
Although $160$ is not a square in base ten, what could you add to $160$ so that the result would be a square number?  
\end{problem}

\begin{problem}
 Consider the polynomial expression $x^2+6x$ to be a number in base $x$.  We want to add to this polynomial so that the result is a square in base $x$.  
\begin{enumerate}
\item Use ``flats'' and ``longs'' to draw a picture of this polynomial as a number in base $x$, adding enough ``ones'' so that you can arrange the polynomial into a square.  
\vspace{0.5in}
\item What ``feature'' of the square does the new polynomial expression represent?  
\item Why does it make sense to call this technique ``completing the square''? 
\item Use your picture to help you solve the equation $x^2+6x=5$.  
\end{enumerate}
\end{problem}


\begin{problem}
Complete the square to solve the following equations: 
\begin{enumerate}
\item $x^2+3x=4$
\vspace{.8in}
\item $x^2+bx=q$
\vspace{.8in}
\item $2x^2+8x=12$
\vspace{.8in}
\item $ax^2+bx+c=0$
\vspace{.8in}
\end{enumerate}  
\end{problem}

\begin{problem}
Solve the following equation
\[
x^5 - 4x^4 - 18x^3 + 64x^2 + 17x -60 = 0
\]
assuming you know that $1$, $-1$, and $3$ are roots.
\end{problem}

\end{document}
