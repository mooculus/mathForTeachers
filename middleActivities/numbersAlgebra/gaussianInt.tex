\newpage
\section{Yet Another Division Theorem}\label{gaussianInt}


Take a minute to recall the \textit{Division Theorem}. Got it? OK we
can do something similar with complex numbers. Check this out:

\begin{definition}\index{Gaussian integers} 
A \textbf{Gaussian integer} is a number of the form
\[
a + bi
\]
where $a$ and $b$ are integers and $i$ is the square-root of negative
one.
\end{definition}

Just like with integers, we have a division theorem here too, check it
out (this time I'll play nice):

\begin{theorem}[Division Theorem]\index{Division Theorem!for Gaussian integers}
Given any Gaussian integer $\alpha$ and a nonzero Gaussian integer
$\beta$, there exist Gaussian integers $\theta$ and $\rho$ such that
\[
\alpha = \beta \cdot \theta + \rho \qquad \text{with}\qquad 
\rho\cdot\bar{\rho} <  \beta\cdot\bar{\beta}
\]
where $\bar{a + bi} = a - bi$.
\end{theorem}

Suppose you want to divide $7+7i$ by $1+2i$ and end up with quotient
and remainder that are both Gaussian integers. How do you do this?
We'll use the complex plane to help us out.
\[
\includegraphics{../graphics/complexPlane.pdf}
\]
\begin{prob} 
Mark $1+2i$ and $7 +7i$ on the complex plane. Use the grid above to
help you and be sure to label your work.
\end{prob}

\begin{prob}
Mark every Gaussian integer multiple of $1+2i$ on the plane
above. Explain what happens and explain why it happens.
\end{prob}


\begin{prob} 
Find the nearest multiple of $1+2i$ to $7+7i$.
\end{prob}

\begin{prob}
Use your work above to help find $\theta$ and $\rho$ such that
\[
7 + 7i  = (1+2i)\cdot \theta + \rho \qquad \text{with}\qquad 
\rho\cdot\bar{\rho} <  5.
\]
\end{prob}

\begin{prob}
Are the $\theta$ and $\rho$ you found above unique? Discuss.
\end{prob}

\begin{prob} 
Explain what is going on here in terms of geometry.
\end{prob}

\begin{prob}
Find $\theta$ and $\rho$ such that
\[
9 + 8i  = (5+2i)\cdot \theta + \rho \qquad \text{with}\qquad 
\rho\cdot\bar{\rho} <  29.
\]
As a gesture of friendship, I have provided a fresh grid for your
work.
\[
\includegraphics{../graphics/complexPlane.pdf}
\]
\end{prob}



\begin{prob}
Are the $\theta$ and $\rho$ you found above unique? Discuss.
\end{prob}
