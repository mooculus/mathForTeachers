%\documentclass[handout]{ximera}
\documentclass[nooutcomes]{ximera}

\usepackage{gensymb}
\usepackage{tabularx}
\usepackage{mdframed}
\usepackage{pdfpages}
%\usepackage{chngcntr}

\let\problem\relax
\let\endproblem\relax

\newcommand{\property}[2]{#1#2}




\newtheoremstyle{SlantTheorem}{\topsep}{\fill}%%% space between body and thm
 {\slshape}                      %%% Thm body font
 {}                              %%% Indent amount (empty = no indent)
 {\bfseries\sffamily}            %%% Thm head font
 {}                              %%% Punctuation after thm head
 {3ex}                           %%% Space after thm head
 {\thmname{#1}\thmnumber{ #2}\thmnote{ \bfseries(#3)}} %%% Thm head spec
\theoremstyle{SlantTheorem}
\newtheorem{problem}{Problem}[]

%\counterwithin*{problem}{section}



%%%%%%%%%%%%%%%%%%%%%%%%%%%%Jenny's code%%%%%%%%%%%%%%%%%%%%

%%% Solution environment
%\newenvironment{solution}{
%\ifhandout\setbox0\vbox\bgroup\else
%\begin{trivlist}\item[\hskip \labelsep\small\itshape\bfseries Solution\hspace{2ex}]
%\par\noindent\upshape\small
%\fi}
%{\ifhandout\egroup\else
%\end{trivlist}
%\fi}
%
%
%%% instructorIntro environment
%\ifhandout
%\newenvironment{instructorIntro}[1][false]%
%{%
%\def\givenatend{\boolean{#1}}\ifthenelse{\boolean{#1}}{\begin{trivlist}\item}{\setbox0\vbox\bgroup}{}
%}
%{%
%\ifthenelse{\givenatend}{\end{trivlist}}{\egroup}{}
%}
%\else
%\newenvironment{instructorIntro}[1][false]%
%{%
%  \ifthenelse{\boolean{#1}}{\begin{trivlist}\item[\hskip \labelsep\bfseries Instructor Notes:\hspace{2ex}]}
%{\begin{trivlist}\item[\hskip \labelsep\bfseries Instructor Notes:\hspace{2ex}]}
%{}
%}
%% %% line at the bottom} 
%{\end{trivlist}\par\addvspace{.5ex}\nobreak\noindent\hung} 
%\fi
%
%


\let\instructorNotes\relax
\let\endinstructorNotes\relax
%%% instructorNotes environment
\ifhandout
\newenvironment{instructorNotes}[1][false]%
{%
\def\givenatend{\boolean{#1}}\ifthenelse{\boolean{#1}}{\begin{trivlist}\item}{\setbox0\vbox\bgroup}{}
}
{%
\ifthenelse{\givenatend}{\end{trivlist}}{\egroup}{}
}
\else
\newenvironment{instructorNotes}[1][false]%
{%
  \ifthenelse{\boolean{#1}}{\begin{trivlist}\item[\hskip \labelsep\bfseries {\Large Instructor Notes: \\} \hspace{\textwidth} ]}
{\begin{trivlist}\item[\hskip \labelsep\bfseries {\Large Instructor Notes: \\} \hspace{\textwidth} ]}
{}
}
{\end{trivlist}}
\fi


%% Suggested Timing
\newcommand{\timing}[1]{{\bf Suggested Timing: \hspace{2ex}} #1}




\hypersetup{
    colorlinks=true,       % false: boxed links; true: colored links
    linkcolor=blue,          % color of internal links (change box color with linkbordercolor)
    citecolor=green,        % color of links to bibliography
    filecolor=magenta,      % color of file links
    urlcolor=cyan           % color of external links
}

\title{You Can Count on It!}
\author{Bart Snapp and Brad Findell}

\outcome{Learning outcome goes here.}

\begin{document}
\begin{abstract}
  We count the ways selections can be made.
\end{abstract}
\maketitle

\label{A:countOnIt}
\begin{teachingnote}
 The words permutation and combination tend to promote formulaic thinking rather than careful reasoning.  And students tend to ask ``Does order matter?'' in ways that don't help them with the reasoning. 

The major purposes of this activity: (1) the multiplication principle of counting, supported by a drawn or imagined tree diagram; and (2) strong explanatory thinking around ``$n$ choose $k$,'' connecting to both traffic lights and Pascal's triangle.  

Although students will find formulas for ``$n$ choose $k$'' to be useful, we need not emphasize formulas for permutations because the multiplication principle provides it for free.   (See also the final exam review document from 2014.)
\end{teachingnote}

\begin{problem}
The Diet-Lite restaurant offers 3 entr\'ees, 4 side dishes, 2 desserts
and 5 kinds of drinks.  
\begin{enumerate}
\item If you were going to select a dinner with one
entr\'ee and one side dish, how many different dinners could you order?  Explain your reasoning.  
\item If you were going to select a dinner with one
entr\'ee, one side dish, one dessert, and one drink, how many different
dinners could you order?
\end{enumerate}
\end{problem}

\begin{problem}
Suppose an Ohio license plate consists of two letters followed by two
digits followed by two letters.  How many different
license plates can be made if: 
\begin{enumerate}
\item There are no more restrictions on the
numbers or letters.
\item  There are no repeats of numbers or letters.
\end{enumerate}
\end{problem}

\begin{problem}
Naming officers and choosing a committee.
\begin{enumerate}
\item How many ways can a chairperson, secretary, and treasurer be named in a club of 10 people?  
\item How many ways can a committee of 3 people be chosen from this same club?
\item Explain using how the answer to (b) makes sense by beginning with the answer to (a) and then ``adjusting'' for overcounting.  
\item Generalize part (c) to explain a formula for the number of ways that a committee of $k$ people can be chosen from a club of $n$ members, where $k$ and $n$ are counting numbers with $k<n$.
\end{enumerate}
\end{problem}

\begin{teachingnote}
The following problems are optional.
\end{teachingnote}

\begin{problem}
Six coins are flipped separately (e.g., HTHHHT is different from THHHTH).  How many different results are
possible?
\end{problem}

\begin{problem}
A pizza shop always puts cheese on their pizzas.  If the shop offers
$n$ additional toppings, how many different pizzas can be ordered?
(Note: A plain cheese pizza is an option.)
\end{problem}


%\begin{problem}
%The Pig-Out restaurant offers 5 entrees, 8 side dishes, 12 desserts,
%and 6 kinds of drinks.  If you were going to select a dinner with 3
%entr\'ees, 4 side dishes, 7 desserts, and one drink, how many
%different dinners could you order?
%\end{problem}

\end{document}

