\newpage
\section{Least Squares Approximation}\label{A:leastSquares}

In this activity, we are going to investigate \textit{least
squares approximation}\index{least squares approximation}.

\begin{prob}
Consider the following data:
\{(2,3), (4,5), (6,11)\}
\[
\includegraphics{../graphics/complexPlane.pdf}
\]
Plot the data and use a ruler to sketch a ``best fit'' line.
\end{prob}

\begin{prob}
Now we are going to record some more data in the chart below:
\begin{enumerate}
\item  For each data point, use a ruler to measure the vertical distance between the point and the line. Record this in the first empty row of the table below.
\item For each data point, square the vertical distance. Record this in the second empty row of the table below.
\end{enumerate}
\[
{\renewcommand{\arraystretch}{1.8}
\renewcommand{\arraycolsep}{3mm}
\begin{array}{|l||c|c|c|c|}\hline
\text{Point} & (2,3) & (4,5) & (6,11)  \\ \hline\hline
\text{Vertical Distance} & & & \\ \hline
\text{Squares} & & & \\ \hline
\end{array}}
\]
\end{prob}

\begin{prob}
Add up the squares of the vertical distances. You want this to be as
small as possible. Compare your sum with that of a friend, or enemy.
Whoever got the smallest value has the best approximation of the given data.
\end{prob}

\begin{prob}
Now find the equation of the line you drew. Write it down and don't
forget it!
\end{prob}

So far we've just been ``eye-balling'' our data. Let's roll up our
sleaves and do some real math.

\begin{prob}
Suppose that your line is $\l(x) = ax + b$. Give an expression
representing the sum of squares you get with your data above.
\end{prob}

\begin{teachingnote}
Here we're looking for something like:
\[
(a\cdot 2+ b -3)^2 + (a\cdot 4+ b -5)^2 + (a\cdot 6+ b -11)^2
\]
\end{teachingnote}

\begin{prob}
Simplfy the expression above. You should now have a quadratic in two
variables $a$ and $b$. Find the minimum, thinking of this as quadratic equation in $a$ and then thinking of this as a quadratic equation in $b$. 
\end{prob}

\begin{prob}
You should now have two equations, and two unknowns---solve!
\end{prob}


\begin{prob}
Compare your computed formula with the line you guessed---how did you
do?
\end{prob}
