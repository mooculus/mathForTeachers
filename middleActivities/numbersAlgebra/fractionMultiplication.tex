%\documentclass[handout]{ximera}
\documentclass[nooutcomes]{ximera}


\graphicspath{
  {./}
  {graphics/}
  {../graphics/}
}

\usepackage{chngcntr}

\let\question\relax
\let\endquestion\relax




\newtheoremstyle{SlantTheorem}{\topsep}{\fill}%%% space between body and thm
%\newtheoremstyle{SlantTheorem}{\topsep}{\topsep}%%% space between body and thm
 {\slshape}                      %%% Thm body font
 {}                              %%% Indent amount (empty = no indent)
 {\bfseries\sffamily}            %%% Thm head font
 {}                              %%% Punctuation after thm head
 {3ex}                           %%% Space after thm head
 {\thmname{#1}\thmnumber{ #2}\thmnote{ \bfseries(#3)}}%%% Thm head spec
\theoremstyle{SlantTheorem}
\newtheorem{question}{Question}
\counterwithin*{question}{section}



\let\instructorNotes\relax
\let\endinstructorNotes\relax
%%% instructorNotes environment
\ifhandout
\newenvironment{instructorNotes}[1][false]%
{%
\def\givenatend{\boolean{#1}}\ifthenelse{\boolean{#1}}{\begin{trivlist}\item}{\setbox0\vbox\bgroup}{}
}
{%
\ifthenelse{\givenatend}{\end{trivlist}}{\egroup}{}
}
\else
\newenvironment{instructorNotes}[1][false]%
{%
  \ifthenelse{\boolean{#1}}{\begin{trivlist}\item[\hskip \labelsep\bfseries {\Large Instructor Notes: \\} \hspace{\textwidth} ]}
{\begin{trivlist}\item[\hskip \labelsep\bfseries {\Large Instructor Notes: \\} \hspace{\textwidth} ]}
{}
}
{\end{trivlist}}
\fi


%% Suggested Timing
\newcommand{\timing}[1]{{\bf Suggested Timing: \hspace{2ex}} #1}

\title{Fraction Multiplication}
\author{Bart Snapp and Brad Findell}

\outcome{Learning outcome goes here.}

\begin{document}
\begin{abstract}
  We think about what multiplication of fractions means.
\end{abstract}
\maketitle

\label{A:fractionMultiplication}

\begin{problem}
Suppose $x$ and $y$ are counting numbers.  
\begin{enumerate}
\item What is our convention for the meaning of $xy$ as repeated addition?  
\item In our convention for the meaning of the product $xy$, which letter describes 
\emph{how many groups} and which letter describes \emph{how many in one group}? 
\item In the product $xy$, the $x$ is called the \emph{multiplier} and $y$ is called the \emph{multiplicand}.  
Use these words to describe the meaning of $xy$ as repeated addition. 
\end{enumerate}
\end{problem}

\vspace{1in}

\begin{problem}
In the Common Core State Standards, fractions and fraction operations are built from \emph{unit fractions}, which are fractions with a $1$ in the numerator.  The meaning of a fraction $\frac{a}{b}$ involves three steps: (1) determining the whole; (2) describing the meaning of $\frac{1}{b}$; and (3) describe the meaning of the fraction $\frac{a}{b}$.  Use pictures to illustrate these three steps for the fraction
$\frac{3}{5}$.  
\end{problem}

\vspace{1in}

\begin{problem}
Now we combine the ideas from the previous two problems to describe meanings for simple multiplication of fractions.  
\begin{enumerate}
\item Without computing the result, describe the meaning of the product $5 \times \frac{1}{3}$.
\item Without computing the result, describe the meaning of the product $\frac{1}{3}\times 5$.
\item Without using the commutativity of multiplication (which we have not established for fractions), 
use these meanings and pictures to explain what the products should be. 
\end{enumerate}
\end{problem}

\vspace{1in}

\subsection*{Area Models}
\begin{problem}
Beginning with a unit square, use an area model to illustrate the following:  
\begin{enumerate}
\item $\frac{1}{3}\times \frac{1}{4}$ 
\item $\frac{7}{3}\times \frac{5}{4}$
\end{enumerate}
\end{problem}

\vspace{1.5in}
\
\begin{problem}
When computing $2\frac{1}{3}\times 3\frac{2}{5}$, Byron says that the answer is $6\frac{2}{15}$.  
\begin{enumerate}
\item Explain Byron's method. 
\item How do you know that he is incorrect?  
\item Use what is right about his method to show what he is missing. 
\end{enumerate}
\end{problem}

\end{document}
