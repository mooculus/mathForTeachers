%\documentclass[handout]{ximera}
\documentclass{ximera}

\usepackage{gensymb}
\usepackage{tabularx}
\usepackage{mdframed}
\usepackage{pdfpages}
%\usepackage{chngcntr}

\let\problem\relax
\let\endproblem\relax

\newcommand{\property}[2]{#1#2}




\newtheoremstyle{SlantTheorem}{\topsep}{\fill}%%% space between body and thm
 {\slshape}                      %%% Thm body font
 {}                              %%% Indent amount (empty = no indent)
 {\bfseries\sffamily}            %%% Thm head font
 {}                              %%% Punctuation after thm head
 {3ex}                           %%% Space after thm head
 {\thmname{#1}\thmnumber{ #2}\thmnote{ \bfseries(#3)}} %%% Thm head spec
\theoremstyle{SlantTheorem}
\newtheorem{problem}{Problem}[]

%\counterwithin*{problem}{section}



%%%%%%%%%%%%%%%%%%%%%%%%%%%%Jenny's code%%%%%%%%%%%%%%%%%%%%

%%% Solution environment
%\newenvironment{solution}{
%\ifhandout\setbox0\vbox\bgroup\else
%\begin{trivlist}\item[\hskip \labelsep\small\itshape\bfseries Solution\hspace{2ex}]
%\par\noindent\upshape\small
%\fi}
%{\ifhandout\egroup\else
%\end{trivlist}
%\fi}
%
%
%%% instructorIntro environment
%\ifhandout
%\newenvironment{instructorIntro}[1][false]%
%{%
%\def\givenatend{\boolean{#1}}\ifthenelse{\boolean{#1}}{\begin{trivlist}\item}{\setbox0\vbox\bgroup}{}
%}
%{%
%\ifthenelse{\givenatend}{\end{trivlist}}{\egroup}{}
%}
%\else
%\newenvironment{instructorIntro}[1][false]%
%{%
%  \ifthenelse{\boolean{#1}}{\begin{trivlist}\item[\hskip \labelsep\bfseries Instructor Notes:\hspace{2ex}]}
%{\begin{trivlist}\item[\hskip \labelsep\bfseries Instructor Notes:\hspace{2ex}]}
%{}
%}
%% %% line at the bottom} 
%{\end{trivlist}\par\addvspace{.5ex}\nobreak\noindent\hung} 
%\fi
%
%


\let\instructorNotes\relax
\let\endinstructorNotes\relax
%%% instructorNotes environment
\ifhandout
\newenvironment{instructorNotes}[1][false]%
{%
\def\givenatend{\boolean{#1}}\ifthenelse{\boolean{#1}}{\begin{trivlist}\item}{\setbox0\vbox\bgroup}{}
}
{%
\ifthenelse{\givenatend}{\end{trivlist}}{\egroup}{}
}
\else
\newenvironment{instructorNotes}[1][false]%
{%
  \ifthenelse{\boolean{#1}}{\begin{trivlist}\item[\hskip \labelsep\bfseries {\Large Instructor Notes: \\} \hspace{\textwidth} ]}
{\begin{trivlist}\item[\hskip \labelsep\bfseries {\Large Instructor Notes: \\} \hspace{\textwidth} ]}
{}
}
{\end{trivlist}}
\fi


%% Suggested Timing
\newcommand{\timing}[1]{{\bf Suggested Timing: \hspace{2ex}} #1}




\hypersetup{
    colorlinks=true,       % false: boxed links; true: colored links
    linkcolor=blue,          % color of internal links (change box color with linkbordercolor)
    citecolor=green,        % color of links to bibliography
    filecolor=magenta,      % color of file links
    urlcolor=cyan           % color of external links
}

\title{Hieroglyphical Algebra}
\author{Bart Snapp and Brad Findell}

\outcome{Learning outcome goes here.}

\begin{document}
\begin{abstract}
Abstract goes here.  
\end{abstract}
\maketitle

\label{A:HAlgebra} 
Note: This activity is based on an activity originally designed by Lee Wayand.

\fixnote{Fix the width on the tables below.}

Consider the following addition and multiplication tables:
\[
{\renewcommand{\arraystretch}{1.8}
\begin{array}{clc}
{\renewcommand{\arraystretch}{1.8}
\begin{array}{|c||c|c|c|c|c|c|c|c|c|}\hline
 +  &\loo & \la & \lo & \lb & \lc & \ld & \lf & \lh & \li \\ \hline\hline
\loo& \ld & \lh &\loo & \li & \lb & \lo & \la & \lf & \lc \\ \hline
\la & \lh & \li & \la & \ld &\loo & \lf & \lb & \lc & \lo \\ \hline
\lo &\loo & \la & \lo & \lb & \lc & \ld & \lf & \lh & \li \\ \hline
\lb & \li & \ld & \lb & \lh & \la & \lc &\loo & \lo & \lf \\ \hline
\lc & \lb &\loo & \lc & \la & \lf & \li & \lo & \ld & \lh \\ \hline
\ld & \lo & \lf & \ld & \lc & \li &\loo & \lh & \la & \lb \\ \hline
\lf & \la & \lb & \lf &\loo & \lo & \lh & \lc & \li & \ld \\ \hline
\lh & \lf & \lc & \lh & \lo & \ld & \la & \li & \lb &\loo \\ \hline
\li & \lc & \lo & \li & \lf & \lh & \lb & \ld &\loo & \la \\ \hline
\end{array}}
\vspace{.5cm}
& 
\begin{array}{l}
\text{$\loo =$ fish} \\ 
\text{$\la =$ lolly-pop} \\ 
\text{$\lo =$ skull} \\ 
\text{$\lb =$ cinder-block} \\ 
\text{$\lc =$ DNA} \\ 
\text{$\ld =$ fork} \\ 
\text{$\lf =$ man} \\ 
\text{$\lh =$ balloon} \\ 
\text{$\li =$ eyeball} 
\end{array}
& 
{\renewcommand{\arraystretch}{1.8}
\begin{array}{|c||c|c|c|c|c|c|c|c|c|}\hline
\cdot & \ld & \li & \lh & \lf & \lo & \la & \lb & \lc &\loo \\ \hline\hline
\ld   &\loo & \la & \lb & \lc & \lo & \li & \lh & \lf & \ld \\ \hline
\li   & \la & \lh & \lf & \ld & \lo & \lb & \lc &\loo & \li \\ \hline
\lh   & \lb & \lf & \ld & \la & \lo & \lc &\loo & \li & \lh \\ \hline
\lf   & \lc & \ld & \la & \lb & \lo &\loo & \li & \lh & \lf \\ \hline
\lo   & \lo & \lo & \lo & \lo & \lo & \lo & \lo & \lo & \lo \\ \hline
\la   & \li & \lb & \lc &\loo & \lo & \lh & \lf & \ld & \la \\ \hline
\lb   & \lh & \lc &\loo & \li & \lo & \lf & \ld & \la & \lb \\ \hline
\lc   & \lf &\loo & \li & \lh & \lo & \ld & \la & \lb & \lc \\ \hline
\loo  & \ld & \li & \lh & \lf & \lo & \la & \lb & \lc &\loo \\ \hline
\end{array}} 
\end{array}}
\]

\newpage



\begin{problem} 
Can you tell me which glyph represents $0$? How did you arrive at this
conclusion?
\end{problem}

\begin{problem} 
Can you tell me which glyph represents $1$? How did you arrive at this
conclusion?
\end{problem}

\begin{problem}
A number $x$ has an \textit{additive inverse} if you can find another number $y$ with 
\[
x + y = 0.
\]
and we say that ``$y$ is the additive inverse for $x$.'' If possible,
find the additive inverse of every number in the table above.
\end{problem}

\begin{problem}
A number $x$ has a \textit{multiplicative inverse} if you can find
another number $y$ with
\[
x\cdot y = 1.
\]
and we say that ``$y$ is the multiplicative inverse for $x$.'' If
possible, find the multiplicative inverse of every number in the
table above.
\end{problem}



\begin{problem} If possible, solve the following equations:
\begin{enumerate}
\item $\lh \cdot \lx - \lb = \ld$
\item $\dfrac{\ly}{\lb} = \dfrac{\ld}{\loo}$
\item $\bigg(\lz - \ld\bigg)\bigg(\lz + \lf\bigg)=\loo$
\item $\dfrac{\loo - \lw}{\lc} + \ld = \dfrac{\lh}{\lf}$
\end{enumerate}
In each case explain your reasoning.
\end{problem}

\begin{problem} If possible, solve the following equations:
\begin{enumerate}
\item $\lx \cdot \lx = \ld$
\item $\lz\cdot \lz = \lf$
\item $\ly\cdot \ly + \ly \cdot \lb = \loo$
\item $\lw \cdot \lw+ \lc = \lw$
\end{enumerate}
In each case explain your reasoning.
\end{problem}

\end{document}
