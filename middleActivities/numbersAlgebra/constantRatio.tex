%\documentclass[handout]{ximera}
\documentclass[nooutcomes]{ximera}


\graphicspath{
  {./}
  {graphics/}
  {../graphics/}
}

\usepackage{chngcntr}

\let\question\relax
\let\endquestion\relax




\newtheoremstyle{SlantTheorem}{\topsep}{\fill}%%% space between body and thm
%\newtheoremstyle{SlantTheorem}{\topsep}{\topsep}%%% space between body and thm
 {\slshape}                      %%% Thm body font
 {}                              %%% Indent amount (empty = no indent)
 {\bfseries\sffamily}            %%% Thm head font
 {}                              %%% Punctuation after thm head
 {3ex}                           %%% Space after thm head
 {\thmname{#1}\thmnumber{ #2}\thmnote{ \bfseries(#3)}}%%% Thm head spec
\theoremstyle{SlantTheorem}
\newtheorem{question}{Question}
\counterwithin*{question}{section}



\let\instructorNotes\relax
\let\endinstructorNotes\relax
%%% instructorNotes environment
\ifhandout
\newenvironment{instructorNotes}[1][false]%
{%
\def\givenatend{\boolean{#1}}\ifthenelse{\boolean{#1}}{\begin{trivlist}\item}{\setbox0\vbox\bgroup}{}
}
{%
\ifthenelse{\givenatend}{\end{trivlist}}{\egroup}{}
}
\else
\newenvironment{instructorNotes}[1][false]%
{%
  \ifthenelse{\boolean{#1}}{\begin{trivlist}\item[\hskip \labelsep\bfseries {\Large Instructor Notes: \\} \hspace{\textwidth} ]}
{\begin{trivlist}\item[\hskip \labelsep\bfseries {\Large Instructor Notes: \\} \hspace{\textwidth} ]}
{}
}
{\end{trivlist}}
\fi


%% Suggested Timing
\newcommand{\timing}[1]{{\bf Suggested Timing: \hspace{2ex}} #1}

\title{Constant Percentage (Ratio) Changes}
\author{Bart Snapp and Brad Findell}

\outcome{Learning outcome goes here.}

\begin{document}
\begin{abstract}
  We explore geometric series.
\end{abstract}
\maketitle

\label{A:ConstantRatio}
\begin{teachingnote}
Students will need graph paper.
\end{teachingnote}

\begin{problem}
Billy is a bouncing ball.  He is dropped from a height of 13 feet and each bounce goes up 92\% of the bounce before it.  Assume that the first time Billy hits the ground is bounce 1.  

\begin{enumerate}
\item Make a table of how high Billy bounced after each of the first 10 times he hit the ground.  Be sure to indicate the arithmetic process you go through for each bounce (i.e., not just the final height).  Find a pattern that will predict an answer.  

{\renewcommand{\arraystretch}{1.4}
\begin{tabular}{c|c}
        Bounce    & Height \\
\hline
         0       &          13        \\
                 &                      \\
                 &                      \\
                 &                      \\
                 &                      \\
                 &                      \\
                 &                      \\
                 &                      \\
                 &                      \\
                 &                      \\
\end{tabular}
}

\item Think of what a $7^\mathrm{th}$ grader would do to predict the next bounce's height given the previous bounce's height.  How would the $7^\mathrm{th}$ grader answer the previous question?  How does this differ from how you solved it?

\item Make a graph of your data about Billy.  Which variable do you plot on the horizontal axis?  Explain.  

\item Does it make sense to connect the dots on your graph?  Explain your reasoning.  

\item How high will Billy bounce after the 38th bounce?  How high will Billy bounce after the $n^\mathrm{th}$ bounce?  Explain your reasoning. 

\item  Use function notation, $f(n)$, and a recursive formula to specify the height of Billy's bounces, including the initial condition and general term.   

\item Use function notation, $f(n)$, and an explicit formula to specify the height of Billy's bounces.  Indicate the domain of the function.    

\item Using your table from above, compute the differences between the heights on successive bounces (e.g.,  $f(1) - f(0)$, $f(2) - f(1)$, etc.).  What do you notice?  Why does this happen?

\item Compare and contrast the explicit and recursive representations from Billy and from Gertrude.  How do the role(s) of the operations and initial values differ, remain the same, or relate?
\end{enumerate}
\end{problem}

\begin{problem}
Supppose 13 mg of a drug is administered to a patient once, and the amount of the drug in the patient's body decreases by 8\% each hour.  
\begin{enumerate}
\item Describe the amount of the drug in the patient's body $t$ hours after it was administered.  

\item Make a graph of your data about the amount of drug in the body over time.  Which variable do you plot on the horizontal axis?  Explain.  

\item Does it make sense to connect the dots on your graph?  Explain your reasoning.  

\item Use function notation, $g(t)$, and an explicit formula to specify the the amount of drug remaining in the body after $t$ hours.  Indicate the domain of the function. 

\item How is this problem fundamentally different from the Billy problem?  What is the same and different about the functions $f$ and $g$?  

\item Dumb Question:  At any one time, how many different amounts of the drug are possible in the patient's body?
\end{enumerate}
\end{problem}

\end{document}
