\documentclass[handout,nooutcomes, noauthor]{ximera}

\title{Meanings of Operations}

\begin{document}
\begin{abstract}
\end{abstract}
\maketitle

\begin{question}
What does addition mean?  Come up with a story or situation in which you would use addition to solve the problem.  Explain how you know that addition is the correct operation to use in this case.
\end{question}

\begin{question}
What does subtraction mean? (Are there different answers to this question?)  Come up with a story or situation in which you would use subtraction to solve the problem.  Explain how you know that subtraction is the correct operation to use in this case.
\end{question}

\begin{question}
What does multiplication mean?  Come up with a story or situation in which you would use multiplication to solve the problem.  Explain how you know that multiplication is the correct operation to use in this case.
\end{question}

\begin{question}
What does division mean? (Are there different answers to this question?)  Come up with a story or situation in which you would use division to solve the problem.  Explain how you know that division is the correct operation to use in this case.
\end{question}

\begin{problem}
In this activity, we are just trying to get the basic ideas down - we'll hopefully return to the meanings of operations as we work more closely with them.  Try to summarize with your group what you've done today, and think more broadly: how do we distinguish the various operations when we find them ``in the wild''?
\end{problem}


\end{document}