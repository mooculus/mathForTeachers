%\documentclass[handout]{xourse}
\documentclass[instructornotes]{xourse}

\usepackage{gensymb}
\usepackage{tabularx}
\usepackage{mdframed}
\usepackage{pdfpages}
%\usepackage{chngcntr}

\let\problem\relax
\let\endproblem\relax

\newcommand{\property}[2]{#1#2}




\newtheoremstyle{SlantTheorem}{\topsep}{\fill}%%% space between body and thm
 {\slshape}                      %%% Thm body font
 {}                              %%% Indent amount (empty = no indent)
 {\bfseries\sffamily}            %%% Thm head font
 {}                              %%% Punctuation after thm head
 {3ex}                           %%% Space after thm head
 {\thmname{#1}\thmnumber{ #2}\thmnote{ \bfseries(#3)}} %%% Thm head spec
\theoremstyle{SlantTheorem}
\newtheorem{problem}{Problem}[]

%\counterwithin*{problem}{section}



%%%%%%%%%%%%%%%%%%%%%%%%%%%%Jenny's code%%%%%%%%%%%%%%%%%%%%

%%% Solution environment
%\newenvironment{solution}{
%\ifhandout\setbox0\vbox\bgroup\else
%\begin{trivlist}\item[\hskip \labelsep\small\itshape\bfseries Solution\hspace{2ex}]
%\par\noindent\upshape\small
%\fi}
%{\ifhandout\egroup\else
%\end{trivlist}
%\fi}
%
%
%%% instructorIntro environment
%\ifhandout
%\newenvironment{instructorIntro}[1][false]%
%{%
%\def\givenatend{\boolean{#1}}\ifthenelse{\boolean{#1}}{\begin{trivlist}\item}{\setbox0\vbox\bgroup}{}
%}
%{%
%\ifthenelse{\givenatend}{\end{trivlist}}{\egroup}{}
%}
%\else
%\newenvironment{instructorIntro}[1][false]%
%{%
%  \ifthenelse{\boolean{#1}}{\begin{trivlist}\item[\hskip \labelsep\bfseries Instructor Notes:\hspace{2ex}]}
%{\begin{trivlist}\item[\hskip \labelsep\bfseries Instructor Notes:\hspace{2ex}]}
%{}
%}
%% %% line at the bottom} 
%{\end{trivlist}\par\addvspace{.5ex}\nobreak\noindent\hung} 
%\fi
%
%


\let\instructorNotes\relax
\let\endinstructorNotes\relax
%%% instructorNotes environment
\ifhandout
\newenvironment{instructorNotes}[1][false]%
{%
\def\givenatend{\boolean{#1}}\ifthenelse{\boolean{#1}}{\begin{trivlist}\item}{\setbox0\vbox\bgroup}{}
}
{%
\ifthenelse{\givenatend}{\end{trivlist}}{\egroup}{}
}
\else
\newenvironment{instructorNotes}[1][false]%
{%
  \ifthenelse{\boolean{#1}}{\begin{trivlist}\item[\hskip \labelsep\bfseries {\Large Instructor Notes: \\} \hspace{\textwidth} ]}
{\begin{trivlist}\item[\hskip \labelsep\bfseries {\Large Instructor Notes: \\} \hspace{\textwidth} ]}
{}
}
{\end{trivlist}}
\fi


%% Suggested Timing
\newcommand{\timing}[1]{{\bf Suggested Timing: \hspace{2ex}} #1}




\hypersetup{
    colorlinks=true,       % false: boxed links; true: colored links
    linkcolor=blue,          % color of internal links (change box color with linkbordercolor)
    citecolor=green,        % color of links to bibliography
    filecolor=magenta,      % color of file links
    urlcolor=cyan           % color of external links
}


\title{Number and Algebra Activities for Middle Grades Teachers}
\author{Bart Snapp and Brad Findell}

%\outcome{Learning outcome goes here.}

\begin{document}
\begin{abstract}
Abstract goes here.  
\end{abstract}
\maketitle

\part{Arithmetic and Algebra}
\newpage
%\activity{numbersAlgebra/shemp.tex} 
\activity{numbersAlgebra/bases.tex}  % Shelby and Scotty
\activity{numbersAlgebra/hieroArith.tex}
\activity{numbersAlgebra/blocksAlgo.tex}
\activity{numbersAlgebra/blocksMult.tex}
\activity{numbersAlgebra/compare.tex}

\part{Numbers}
\activity{numbersAlgebra/integerAddition}
\activity{numbersAlgebra/integerMultiplication}
\activity{numbersAlgebra/divisionMeaning.tex}
\activity{numbersAlgebra/divisibility}
\activity{numbersAlgebra/hallOfShoes.tex}
\activity{numbersAlgebra/sieve.tex}
\activity{numbersAlgebra/infPrimes.tex}
\activity{numbersAlgebra/countingFactors.tex}
\activity{numbersAlgebra/whyDoesItWork.tex}
\activity{numbersAlgebra/prome.tex}
\activity{numbersAlgebra/equivFrac.tex}
\activity{numbersAlgebra/fracPictMean.tex}
\activity{numbersAlgebra/fractionMultiplication}
\activity{numbersAlgebra/divFractions.tex}
\activity{numbersAlgebra/pictDivision.tex}
\activity{numbersAlgebra/crossMult.tex}
\activity{numbersAlgebra/hundredthsGrid}
\activity{numbersAlgebra/repeat.tex}
\activity{numbersAlgebra/decimals.tex}

\part{Ratios, Functions, and Beyond}
\activity{numbersAlgebra/ratioLaunch.tex}
\activity{numbersAlgebra/ratios.tex}
\activity{numbersAlgebra/ratioAddition.tex}
%\activity{numbersAlgebra/solved.tex}
\activity{numbersAlgebra/triathlete.tex}
\activity{numbersAlgebra/dreadedStoryProblem.tex}
\activity{numbersAlgebra/walkTheLine.tex}
\activity{numbersAlgebra/constantAmount.tex}
\activity{numbersAlgebra/constantRatio.tex}
%\activity{numbersAlgebra/exponentRules.tex}  Replaced by Meanings of Exponents
\activity{numbersAlgebra/meaningsOfExponents}
%\activity{numbersAlgebra/doesntAddUp.tex}  Replaced by arithmetic series
% \activity{numbersAlgebra/gertrude.tex}   % Replaced by ConstantAmount
% \activity{numbersAlgebra/billy.tex}    %  Replaced by ConstantRatio
\activity{numbersAlgebra/arithmeticSeries}
\activity{numbersAlgebra/geometicSeries}
\activity{numbersAlgebra/secondDifferences} % Needs \RequirePackage{xcolor,colortbl}, 
% but then some of the Euclidean Algorithm sections won't compile.  

\part{Solving Equations}
\activity{numbersAlgebra/hieroAlg.tex}
\activity{numbersAlgebra/otherSide.tex}
\activity{numbersAlgebra/solvingQuadratics}
\activity{numbersAlgebra/completeSquares}
\activity{numbersAlgebra/vertex.tex}
%\activity{numbersAlgebra/leastSquares.tex}
\activity{numbersAlgebra/solvingCubics}
%\activity{numbersAlgebra/otherCurves.tex}
\activity{numbersAlgebra/sketchRoots.tex}
\activity{numbersAlgebra/complexAddition.tex}
\activity{numbersAlgebra/complexMultiplication.tex}
%\activity{numbersAlgebra/deg2Ext.tex}
%\activity{numbersAlgebra/inverseFunctions.tex}

\part{Harmony of Numbers}
%\activity{numbersAlgebra/gaussianInt.tex}
%\activity{numbersAlgebra/orderMod.tex}
%\activity{numbersAlgebra/closeButNoCigar.tex}
\activity{numbersAlgebra/traffic.tex}
\activity{numbersAlgebra/factOrFiction.tex}
%\activity{numbersAlgebra/pyramid.tex}
\activity{numbersAlgebra/countOnIt.tex}
\activity{numbersAlgebra/whichRoad.tex}
\activity{numbersAlgebra/lumpyAndEddie.tex}
\activity{numbersAlgebra/climbTree.tex}
\activity{numbersAlgebra/fallForAnything.tex}
%\activity{numbersAlgebra/harmtri.tex}
%\activity{numbersAlgebra/whoops.tex}
%\activity{numbersAlgebra/largePrime.tex}
%% \part{Enrichment Topics}
%% \activity{../chapters/enrichment/continuedFractions.tex}
%% \activity{../chapters/enrichment/music.tex}



%
%\begin{teachingnote}
%Activities to be added or revised:  
%\begin{itemize}
%\item Beckmann 1G
%\item Product Puzzle from Connected Mathematics Project 
%\item A.26. Poor Old Horatio.  Need new ratio table problems.
%\end{itemize}
%\end{teachingnote}
%
% Fixnote summary and other needs
%
% Highlight Deep Insights, such as
%  * The point-slope form of a line turns a non-proportional situation into a proportional one. 
%  * The sequence of partial sums of an arithmetic series is a quadratic function. 
%  * With common denominators, division of fractions ? 
%
% Fix the footnotes
%
%More problems needed: 
% 1.1. Other bases
% 1.2. Doubling and halving algorithm for multiplication 
%         Mysterious base two game
% 2.1. Models of integer arithmetic
% 3.1. Easy problems on ratio and proportional reasoning
% 	Ratio tables, going through 1, tables that aren?t ratio tables
% 3.2. More problem on quadratics, exponentials?  (Some in LaTeX comments.)
%
% Discussion:  Meaning of the equals sign somewhere
%

\end{document}
