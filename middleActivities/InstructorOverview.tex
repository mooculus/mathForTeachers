\documentclass[nooutcomes]{ximera}

\title{Instructor Overview}
\author{Vic Ferdinand, Betsy McNeal, Jenny Sheldon, Brad Findell, and Bart Snapp}

\begin{document}

\begin{abstract}

\end{abstract}\maketitle

\includegraphics[height=1.6in]{OOECLogo.png}

In two documents, we have gathered a collection of the activities used at The Ohio State University in a 
two-semester sequence of mathematics content courses for preservice middle grades 
school teachers. Math 1165 focuses on numbers and algebra; Math 1166 focuses on geometry, with continuing attention to algebra.  At Ohio State, the courses are each five credit hours, and run five days per week.  Many of the activities are designed to take the whole class period, or 55 minutes, while others use approximately half the class.  

This collection of activities is not intended to be a full course or a textbook, or to cover all of the 
content required for preservice middle grades teachers, though the coverage is quite broad.  The activities 
may be used either individually or in small units, so that they can be reorganized or used in part, as an 
instructor deems most appropriate.  In the instructors' editions, we have included ``Teaching Notes'' to provide the mathematical motivation behind the activity, some answers, common conceptual challenges, and other insights from our experience.  

The source code for the activities is available on GitHub at \url{https://github.com/mooculus/mathForTeachers}.  Familiarity with \LaTeX as well as Ohio State's Ximera document class (\url{https://ximera.osu.edu/}) is recommended for editing the source code.

At Ohio State, we have several main goals for our course.  First, we aim to have students refresh 
and deepen their understanding of key mathematical procedures and concepts spanning middle grades 
mathematics.  Although most of our students have high interest and confidence in mathematics, they often 
cannot explain why procedures work or how concepts are related.  Throughout the 
course, we try to challenge their starting assumptions and provoke deeper reasoning by selecting 
problems and questions that cause them to approach these basic ideas from fresh 
points of view.  We constantly ask students why a procedure works as it does and what it 
means.  We do not accept a formula as an explanation.  We require explanations that are
based on the meaning of a number (e.g., a fraction) and the meaning of an operation (e.g., 
multiplication as a number of groups with the same number of objects in each group).

One theme we see in our activities is an intention to ``make the familiar strange''.  To us, this means 
we would like to take mathematical concepts that are familiar to students, but approach them from 
angles which make the problems or concepts unfamiliar.  When students recognize a problem, we have 
found that they are likely to try to apply a ``rule'' they remember from school, often 
without much thought as to whether that rule applies or is appropriate.  When we take away the 
familiar contexts, students seem more willing to use definitions and think through concepts instead
of just applying a rule.

Our second goal is to encourage a positive attitude towards mathematics and an enriched view 
of what mathematics is all about.  We hope to move students towards an understanding of mathematics
as a process of reasoning and away from a view of math as a body of 
rules and facts to be memorized. Such a view of mathematics is important for teachers to make a productive and 
creative classroom. A teacher who understands the intent of mathematics as a way of 
thinking will encourage her students to engage more deeply in that way of thinking.  

Both of these goals are addressed in our emphasis on explanation.  The more 
advanced students are challenged to view the focus on explanations and grade appropriate justifications as building toward the more rigorous proofs they experienced in high school.  
Students' explanations must follow directly and logically from the context in which we are working.  
All class meetings (lecture and recitation) at OSU are designed to get students thinking about, talking 
about, writing, and representing mathematical ideas.  This encourages deeper and more connected 
understanding of the mathematics on their part as well as relating the learning to their preparation as future teachers.  

Our class meetings are frequently composed of deep discussion about a handful of problems---problems intended to build problem-solving skills and encourage independent learning.  

Some instructors might read our activities and initially think they are easy or superficial.  The purpose of these notes is to illustrate how these activities can be used to probe the depths of each topic in the traditional content for elementary mathematics teachers.  Our descriptions are drawn from our actual experiences of teaching many future teachers over many years.

Because our courses include more than what is publicly posted here, we are happy to share much of 
that as well as answer any questions you may have about the activities, calendar, or other aspects of the course.  Please feel free to contact current instructors Brad Findell or Jenny Sheldon at Ohio State for more information.

We would like to give our deepest thanks to Sybilla Beckmann and acknowledge the significant role her textbook, \emph{Mathematics for Elementary Teachers with Activities} played in the development of our program for middle grades teachers.  We hope that we have properly separated our own problems and activities from those in her textbook. 

Finally, we would like to thank the Ohio Department of Higher Education, which in 2017 awarded a \$1.3 million grant to North Central State College and two partner institutions, The Ohio State University and Ohio Dominican University, with the goal of creating new and curating existing open materials for over twenty courses.  These materials were collected in this format as a result of this grant.


\end{document}
