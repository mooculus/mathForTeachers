
\graphicspath{
  {./}
  {graphics/}
  {geometry/graphics/}
  {numbersAlgebra/graphics/}
}

\newtheorem{postulate}{Postulate}


\let\image\relax
\let\endimage\relax
\NewEnviron{image}[1][3in]{% 
  \begin{center}\BODY\end{center}% resize and center
}




% Below are packages and macros previously used for the Math 1165 and Math 1166 notes.  Most are commented out and included here for convenience should they need to be used again.  

%
%\RequirePackage{marvosym}
%\RequirePackage[framemethod=TikZ]{mdframed}
%\RequirePackage{environ}
%\RequirePackage{tikz}
%\RequirePackage{amssymb}
%\RequirePackage{amsmath}
%\RequirePackage{mathtools} %% trying to get cases to work in caption
%\RequirePackage{ntheorem}
%\RequirePackage{thmtools}
%\RequirePackage{microtype}
%\RequirePackage{enumerate}
\RequirePackage{rotating}
\RequirePackage{xcolor,colortbl} % Required for Second Differences but causes problems for Euclidean Algorithm

%
%\RequirePackage{pifont}
%\RequirePackage{bigstrut}
%\RequirePackage[all,cmtip]{xy}
%
%
\usepackage{pgfplots}\pgfplotsset{compat=1.7}  % Needed for eclipseTheEllipse
%
%
\usetikzlibrary{shadows}
\usetikzlibrary{shapes}
\usetikzlibrary{decorations}
\usetikzlibrary{arrows}

\tikzset{>=stealth} %% makes all arrows the same
%
%
%%% Font
%\RequirePackage[T1]{fontenc}
%
%%% New Century Schoolbook
%%\RequirePackage{fouriernc}
%
%
%\usepackage{array}
%%\setlength{\extrarowheight}{-.2cm}   % Commented out by Findell to fix table headings.  Was this for typesetting division?  
\newdimen\digitwidth 
\settowidth\digitwidth{9}
%\def~{\hspace{\digitwidth}}  % Commented out by Findell because it messed up \title
\def\divrule#1#2{
\noalign{\moveright#1\digitwidth
\vbox{\hrule width#2\digitwidth}}}


\usepackage{framed}
\usepackage{comment}

%\newenvironment{teachingnote}{\begin{framed}\begin{itshape}\begin{bfseries}\noindent Teaching Note:}{\end{bfseries}\end{itshape}\end{framed}}

\newenvironment{teachingnote}{\begin{framed}\noindent \textbf{Teaching Note:}}{\end{framed}}
\newcommand{\teachingnotes}{With Teaching Notes}
%\excludecomment{teachingnote}
%\newcommand{\teachingnotes}{}

\def\fixnote#1{\begin{framed}{\color{red}Fixnote: #1}\end{framed}}  % Allows insertion of red notes about needed edits
%\def\fixnote#1{}

%%% This set of code is all of our user defined commands
\newcommand{\bysame}{\mbox{\rule{3em}{.4pt}}\,}
\newcommand{\N}{\mathbb N}
%\renewcommand{\C}{\mathbb C}
\newcommand{\C}{\mathbb C}
\newcommand{\W}{\mathbb W}
\newcommand{\Z}{\mathbb Z}
\newcommand{\Q}{\mathbb Q}
\newcommand{\R}{\mathbb R}
\newcommand{\A}{\mathbb A}
\newcommand{\D}{\mathcal D}
\newcommand{\F}{\mathcal F}
\newcommand{\ph}{\varphi}
\newcommand{\ep}{\varepsilon}
\newcommand{\aph}{\alpha}
\newcommand{\QM}{\begin{center}{\huge\textbf{?}}\end{center}}
\renewcommand{\le}{\leqslant}
\renewcommand{\ge}{\geqslant}
\renewcommand{\a}{\wedge}
\renewcommand{\v}{\vee}
\renewcommand{\l}{\ell}
\newcommand{\mat}{\mathsf}
\renewcommand{\vec}{\mathbf}
\renewcommand{\subset}{\subseteq}
\renewcommand{\supset}{\supseteq}
\renewcommand{\emptyset}{\varnothing}
\newcommand{\xto}{\xrightarrow}
\renewcommand{\qedsymbol}{$\blacksquare$}
%\renewcommand{\bibname}{References and Further Reading}
\renewcommand{\bar}{\protect\overline}
\renewcommand{\hat}{\protect\widehat}
\renewcommand{\tilde}{\widetilde}
\newcommand{\tri}{\triangle}
\newcommand{\minipad}{\vspace{1ex}}
\newcommand{\leftexp}[2]{{\vphantom{#2}}^{#1}{#2}}

%% More user defined commands
\renewcommand{\epsilon}{\varepsilon}
\renewcommand{\theta}{\vartheta} %% only for kmath
\renewcommand{\l}{\ell}
\renewcommand{\d}{\, d}
\newcommand{\ddx}{\frac{d}{dx}}
\newcommand{\dydx}{\frac{dy}{dx}}


%% Commands for special characters in Hieroglyphic activities
\usepackage{protosem} 
\newcommand{\lo}{\vcenter{\hbox{\textproto{a}}}}
\newcommand{\loo}{\vcenter{\hbox{\textproto{d}}}}
\newcommand{\la}{\vcenter{\hbox{\textproto{w}}}}
\newcommand{\lb}{\vcenter{\hbox{\textproto{H}}}}
\newcommand{\lc}{\vcenter{\hbox{\textproto{T}}}}
\newcommand{\ld}{\vcenter{\hbox{\textproto{K}}}}
\newcommand{\lf}{\vcenter{\hbox{\textproto{E}}}}
\newcommand{\lh}{\vcenter{\hbox{\textproto{l}}}}
\newcommand{\li}{\vcenter{\hbox{\textproto{o}}}}
\newcommand{\lx}{\vcenter{\hbox{\textproto{x}}}}
\newcommand{\ly}{\vcenter{\hbox{\textproto{R}}}}
\newcommand{\lz}{\vcenter{\hbox{\textproto{v}}}}
\newcommand{\lw}{\vcenter{\hbox{\textproto{q}}}}


%%
%% Colors
%%

%\colorlet{background}{white} % Color of the page
%\colorlet{textColor}{black} % Color of the text
%
\colorlet{penColor}{blue!50!black} % Color of a curve in a plot
%\colorlet{penColor2}{red!50!black} % Color of a curve in a plot
%\colorlet{penColor3}{red!50!blue} % Color of a curve in a plot
%\colorlet{penColor4}{green!50!black} % Color of a curve in a plot
%\colorlet{penColor5}{orange!80!black} % Color of a curve in a plot
%\colorlet{fill1}{blue!50!black!20} % Color of fill in a plot
%\colorlet{fill2}{blue!10} % Color of fill in a plot
%\colorlet{fillp}{fill1} % Color of positive area
%\colorlet{filln}{red!50!black!20} % Color of negative area
%\newcommand{\surfaceColor}{violet}
%\newcommand{\surfaceColorTwo}{redyellow}
%\newcommand{\sliceColor}{greenyellow}
%\colorlet{gridColor}{gray!50} % Color of grid in a plot



%\renewcommand{\footnote}[1]{~(#1)}

%\newenvironment{activitynote}{\begin{framed}\begin{itshape}\begin{bfseries}\noindent Note:}{\end{bfseries}\end{itshape}\end{framed}}
%\NewEnviron{activitynote}{\marginnote{\BODY}}




%\newcommand{\dd}[1]{\frac{d}{d #1}}
\def\dd{\@ifnextchar[{\@with}{\@without}}
\def\@with[#1]#2{\frac{d #1}{d #2}}
\def\@without#1{\frac{d}{d #1}}


%\everymath{\displaystyle}


%
%
\usepackage{multicol} % Use letters for lists
%\renewcommand{\theenumi}{$(\mathrm{\alph{enumi}})$}
%\renewcommand{\labelenumi}{\theenumi}
%\setlength{\columnsep}{3em}
%
%

\def\margincomment#1{$^{\mbox{\footnotesize\color{red}\textbullet}}$%
\marginpar{%
\setbox1=\hbox{\parbox{1.4in}{\footnotesize {\color{red}\textbullet} \sffamily #1}}\hbox{\box1}
}
}

